\documentclass[12pt]{article}

% Page size and tighter margins
\usepackage[a4paper,left=1.2cm,right=1.2cm,top=1.5cm,bottom=1.5cm]{geometry}

% Core packages
\usepackage{graphicx}
\usepackage{xcolor}
\usepackage{array}
\usepackage{tabularx}
\usepackage{multicol}
\usepackage[T1]{fontenc}
\usepackage[utf8]{inputenc}

\setlength{\parindent}{0pt}
\setlength{\tabcolsep}{6pt}
\renewcommand{\arraystretch}{1.15}

% Column types
\newcolumntype{Y}{>{\raggedright\arraybackslash}m{\dimexpr0.30\textwidth-2\tabcolsep-2\arrayrulewidth\relax}}
\newcolumntype{Z}{>{\raggedright\arraybackslash}m{\dimexpr0.70\textwidth-2\tabcolsep-2\arrayrulewidth\relax}}
\newcolumntype{C}[1]{>{\centering\arraybackslash}m{#1}}

% Gray subsection header box
\newcommand{\SubsectionBox}[1]{%
	\noindent\colorbox{gray!30}{%
		\parbox{\linewidth}{\textbf{#1}}%
	}\par\vspace{0.35cm}%
}

% Centered multi-line cell helper
\newcommand{\CellCenter}[1]{%
	\parbox{\linewidth}{\centering #1}%
}

\begin{document}

	% =========================
	% HEADER BOX (3 COLUMNS)
	% =========================
	\noindent
	\begin{tabularx}{\textwidth}{|C{2.8cm}|C{\dimexpr\textwidth-6cm-4\tabcolsep-4\arrayrulewidth\relax}|C{2.8cm}|}
		\hline
		\centering
		\vspace{3mm}
		\includegraphics[width=2.5cm]{../../preamble/logo.png}
		&
		\CellCenter{%
			\vspace{-5mm}
			\textbf{GLOBAL ECONOMICS}\par
			\textbf{GRADE: 10TH}\par
			\textbf{MIDTERM}\par
			\textbf{ANALYSIS OF DECISIONS}\par
			\textbf{TEACHER'S NAME: Nicolás López Cuéllar}
		}
		&
		\CellCenter{%
			\textbf{SECOND TERM}\par
			\textbf{2025--2026}%
		}
		\\
		\hline
	\end{tabularx}

	\vspace{0.5cm}

	% =========================
	% OBJECTIVE + CRITERIA
	% =========================
	\noindent
	\begin{tabular}{|Y|Z|}
		\hline
		{\small
			\textbf{Learning objective:} Analyze decision-making scenarios by building payoff tables, identifying states of nature, and applying maximax, maximin, minimax and expected value criteria.
		}
		&
		{\footnotesize
			\textbf{Assessment criteria:}\par
			C1: Describes the way a problem can be formulated for optimal decision making. Evaluates expected value. \par
			C2: Interprets decision alternatives, events, consequences, and states of nature.\par
			C3: Builds a payoff table from a description of the problem.\par
			C4: Explains the maximax criterion for decision making without probabilities.\par
			C5: Summarizes the maximin criterion for decision making without probabilities. \par 
			C6: Develops decision-making strategies using probabilities and the maximum opportunity criterion.\par
			C7: Uses probabilities and expected value to analyze a decision-making problem.\par
		}
		\\
		\hline
	\end{tabular}

	\vspace{0.4cm}

	% =========================
	% STUDENT LINE
	% =========================
	\noindent
	\textbf{Student’s name:} \rule{7cm}{0.4pt}\hfill
	\textbf{Group:} \rule{2cm}{0.4pt}\hfill
	\textbf{Date:} \rule{3cm}{0.4pt}


	% =========================
	% EXAM BODY
	% =========================
	\begin{multicols}{2}
		\SubsectionBox{Criteria assessment}\vspace{-0.25cm}
		Each assessment criterion is evaluated across three problems. A criterion is considered passed when it is correctly activated in at least two of the three problems.

		\vspace{0.25cm}
		\SubsectionBox{1. (C2,C3) Bakery Product Bundle Selection Under Multiple Demand Levels}\vspace{-0.25cm}
		A neighborhood bakery must select one of three seasonal bundles for the next month: Bundle A (artisan focus),
		Bundle B (balanced assortment), or Bundle C (value pack). Demand can be very high, high, medium, or low.
		Based on prior months, the probability of very high demand is $0.20$, high demand is $0.35$, medium demand is $0.30$,
		and low demand is $0.15$. Profits are measured in hundreds of dollars. If demand is very high, Bundle A yields 110,
		Bundle B yields 95, and Bundle C yields 80. If demand is high, Bundle A yields 88, Bundle B yields 76, and Bundle C yields 60.
		If demand is medium, Bundle A yields 52, Bundle B yields 58, and Bundle C yields 46. If demand is low, Bundle A yields $-6$,
		Bundle B yields 24, and Bundle C yields 32. Construct the payoff table.

		\vspace{0.25cm}
		\SubsectionBox{2. (C2,C3) Bike-Sharing Layout Decision Under Demand Uncertainty}\vspace{-0.25cm}
		A bike-sharing cooperative must choose a station layout for the next quarter: Layout A (dense stations)
		or Layout B (hub stations). Trip revenue depends on ridership demand, which can be high or low.
		The probability of high demand is $0.65$ and low demand is $0.35$. Maintenance costs depend on parts prices,
		which can be low, medium, or high with probabilities $0.40$, $0.35$, and $0.25$. Revenue is in thousands of dollars.
		If demand is high, revenue is 500 for Layout A and 540 for Layout B. If demand is low, revenue is 280 for Layout A
		and 300 for Layout B. Maintenance costs are 180 (low), 220 (medium), and 260 (high) for Layout A, and 210 (low),
		250 (medium), and 290 (high) for Layout B. Construct the payoff table.

		\vspace{0.25cm}
		\SubsectionBox{3. (C2,C3) Shipping Hub Decision Under Demand and Cost Uncertainty}\vspace{-0.25cm}
		A package shipping firm must choose either operating an in-house hub or outsourcing to regional partners for the next year.
		Demand uncertainty is high demand (0.60) or low demand (0.40), and delivery-cost uncertainty is low (0.55) or high (0.45).
		Under high demand the in-house hub completes 900 shipments while partners complete 850; under low demand the in-house hub completes
		520 shipments while partners complete 560. The in-house hub earns \$45 per shipment, while partners earn \$42 per shipment in high demand
		and \$40 per shipment in low demand. Cost exposure reflects annual operating costs: the in-house hub has \$18{,}000 in the low-cost state
		and \$24{,}000 in the high-cost state, while partners incur \$16{,}500 in the low-cost state and \$21{,}000 in the high-cost state.
		Construct the payoff table.

		\vspace{0.25cm}
		\SubsectionBox{4. (C1,C4,C5,C6) Community Theater Ticketing Strategy Under Demand Uncertainty}\vspace{-0.25cm}
		A community theater must choose one of three ticketing strategies for the upcoming season: Strategy A, Strategy B,
		or Strategy C. Demand can be strong or weak with probabilities $0.60$ and $0.40$. The final payoff table (in thousands of dollars)
		is given below. Use Maximax, Maximin, Minimax Regret, and Expected Value to select a strategy.

		\begin{center}
			\textit{Payoff table} \\
			\begin{tabular}{l c c c c}
				\hline
				State of nature & Prob. & Str. A & Str. B & Str. C \\
				\hline
				Strong demand & 0.60 & 60 & 52 & 45 \\
				Weak demand & 0.40 & 10 & 18 & 25 \\
				\hline
			\end{tabular}
		\end{center}

		\vspace{0.25cm}
		\SubsectionBox{5. (C1,C4,C5,C6) Holiday Inventory Plan Selection Under Demand Uncertainty}\vspace{-0.25cm}
		A retail chain must select one of three inventory plans for the holiday period: Plan A, Plan B, or Plan C.
		Demand can be strong, moderate, or weak with probabilities $0.30$, $0.50$, and $0.20$. The final payoff table
		(in thousands of dollars) is given below. Use Maximax, Maximin, Minimax Regret, and Expected Value to select a plan.

		\begin{center}
			\textit{Payoff table} \\
			\begin{tabular}{l c c c c}
				\hline
				State of nature & Prob. & A & B & C \\
				\hline
				Strong demand & 0.30 & 80 & 70 & 60 \\
				Moderate demand & 0.50 & 50 & 55 & 45 \\
				Weak demand & 0.20 & 10 & 30 & 40 \\
				\hline
			\end{tabular}
		\end{center}

		\vspace{0.25cm}
		\SubsectionBox{6. (C1,C4,C5,C6) Routing System Choice Under Fuel and Congestion Uncertainty}\vspace{-0.25cm}
		A shipping company must choose between two routing systems for the next season: System A or System B.
		Four states of nature summarize fuel and congestion conditions with probabilities $0.25$, $0.30$, $0.20$, and $0.25$.
		The final payoff table (in thousands of dollars) is given below. Use Maximax, Maximin, Minimax Regret, and Expected Value to select a system.

		\begin{center}
			\textit{Payoff table} \\
			\begin{tabular}{l c c c}
				\hline
				State of nature & Prob. & Sys. A & Sys. B \\
				\hline
				State $S_1$ & 0.25 & 40 & 35 \\
				State $S_2$ & 0.30 & 30 & 28 \\
				State $S_3$ & 0.20 & 15 & 20 \\
				State $S_4$ & 0.25 & -5 & 5 \\
				\hline
			\end{tabular}
		\end{center}

		\vspace{0.25cm}
		\SubsectionBox{7. (C7) Expected Value Comparison of Marketing Options}\vspace{-0.25cm}
		A publisher must choose between two marketing options, A and B, before knowing which market condition will occur.
		There are two possible states of nature: state $S_1$, which occurs with probability $p$, and state $S_2$, which occurs with probability $1-p$.
		The objective is to select the option that yields the higher expected profit, measured in consistent monetary units.
		The payoff table below summarizes the profit associated with each option under each state of nature.
		\begin{center}
			\textit{Payoff table} \\
			\begin{tabular}{l c c}
				\hline
				& $S_1$ $(p)$ & $S_2$ $(1-p)$ \\
				\hline
				A & 24 & 12 \\
				B & 18 & 16 \\
				\hline
			\end{tabular}
		\end{center}

		\vspace{0.25cm}
		\SubsectionBox{8. (C7) Expected Value Comparison of Production Plans}\vspace{-0.25cm}
		A manufacturer must select one production plan, labeled A, B, or C, before knowing which market condition will occur.
		There are two possible states of nature: state $S_1$, which occurs with probability $p$, and state $S_2$, which occurs with probability $1-p$.
		Each production plan generates a different profit depending on the realized state of nature, as summarized in the payoff table below.
		The objective is to determine which production plan maximizes expected profit as a function of $p$.

		\begin{center}
			\textit{Payoff table} \\
			\begin{tabular}{l c c}
				\hline
				& $S_1$ $(p)$ & $S_2$ $(1-p)$ \\
				\hline
				A & 32 & 4 \\
				B & 20 & 14 \\
				C & 12 & 10 \\
				\hline
			\end{tabular}
		\end{center}

		\vspace{0.25cm}
		\SubsectionBox{9. (C7) Expected Value Comparison of Zoning Options}\vspace{-0.25cm}
		A city council must choose between two zoning options, labeled A and B, before knowing which future condition will occur.
		There are two possible states of nature: state $S_1$, which occurs with probability $p$, and state $S_2$, which occurs with probability $1-p$.
		Each zoning option generates a different net return depending on the realized state of nature, as shown in the payoff table below.
		The objective is to determine which zoning option yields the higher expected return as a function of $p$.

		\begin{center}
			\textit{Payoff table} \\
			\begin{tabular}{l c c}
				\hline
				& $S_1$ $(p)$ & $S_2$ $(1-p)$ \\
				\hline
				A & 26 & 6 \\
				B & 20 & 12 \\
				\hline
			\end{tabular}
		\end{center}

	\end{multicols}

\end{document}
