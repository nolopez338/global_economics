\makeatletter
\def\input@path{{./}{../}{../../}{preamble/}{../preamble/}{../../preamble/}}
\makeatother
% ----------------------------------------------------------
% GENERAL 

% File
\documentclass[11pt]{book}

% Margins
\usepackage[margin=1in]{geometry}

\linespread{1.2}            % Line spacing
\usepackage[utf8]{inputenc}
\usepackage[T1]{fontenc}
\usepackage{lmodern}
\usepackage{microtype}
\setlength{\parindent}{0pt}
\setlength{\parskip}{6pt}
\usepackage{booktabs}

% ----------------------------------------------------------
% TABLES
\usepackage{multicol}
\usepackage{longtable} 
\usepackage{array}
\usepackage{booktabs}
\usepackage{tabularx}
\usepackage{multirow}

% ----------------------------------------------------------
% MATHEMATICS
\usepackage{amsmath}
\usepackage{amssymb}
\usepackage{amsfonts}
\usepackage{mathtools}

% ----------------------------------------------------------
% HIDDEN CONTENT
\usepackage{ifthen}
% Define a boolean switch
\newboolean{explicaciones}
% Set the boolean switch to true or false
% Change to true to show the content

% Explanations
\newcommand{\explicacion}[2]{
	\ifthenelse{\boolean{explicaciones}}{#1}{#2}
}
\newcommand{\mostrarExplicaciones}[1]{\setboolean{explicaciones}{#1}}

% ----------------------------------------------------------
% NUMBERING

\usepackage{fancyhdr}
\pagestyle{empty} % Ensures the entire document has no page numbers

\usepackage{tocloft}
\renewcommand{\cftdot}{} % Remove dots for sections, if any
\renewcommand{\cftsecleader}{\cftdotfill{\cftdotsep}} % Remove dots for sections, if any
\cftpagenumbersoff{section} % Removes page numbers from sections
\cftpagenumbersoff{subsection} % Removes page numbers from subsections

% ----------------------------------------------------------
% IMAGES 

% General settings
\usepackage{graphicx}       % Insert images
\usepackage{float}          % Position images
% \usepackage{subfigure}      % Subfigures
\graphicspath{{imgs}}       % Image location
\usepackage{subcaption}     % Subfigures II
\usepackage{verbatim}

% Figures
\usepackage{tikz}
\usetikzlibrary{arrows.meta,positioning,trees}

% Colors
\usepackage{xcolor}     
\definecolor{popUp}{HTML}{666666}
\definecolor{popUpIn}{HTML}{CED9E0}
\definecolor{backgroundC}{HTML}{D0E8F2}
\definecolor{backgroundCC}{HTML}{FFFFFF}
\definecolor{borders}{HTML}{8c120d}
\definecolor{padding}{HTML}{77D0D7}
\definecolor{links}{HTML}{CC6F5F}

% ----------------------------------------------------------
% FRAMES

% General settings
\usepackage{tcolorbox}
\usepackage{adjustbox}          % Adjusted frame  
\setlength{\fboxrule}{3pt}  % Line width
\setlength{\fboxsep}{3pt}   % Box padding

% General frames
\usepackage{mdframed}   

\mdfdefinestyle{estiloGeneral}{    % General style
	linecolor=black,
	linewidth=1.5pt,
	roundcorner=10pt,
	backgroundcolor=backgroundC,
	innerbottommargin=0pt
}
\mdfdefinestyle{code}{          % Code style
	linecolor=black,
	linewidth=1.5pt,
	roundcorner=10pt,
	backgroundcolor=darkgray!10,
	innerbottommargin=0pt
}

% Image frame
\newtcbox{\fboxC}{
	colback=backgroundC,
	colframe=popUp,
	arc=10pt,
	boxrule=3pt,
	boxsep=0pt, % Change the padding here
	nobeforeafter
}

% ----------------------------------------------------------
% PAGE SETTINGS

% Background 
\newcommand{\background}[0]{\begin{tikzpicture}[remember picture,overlay]
		\fill[backgroundC] (-2,2) rectangle (25cm, -550);
\end{tikzpicture}}

\newcommand{\backgroundC}[0]{\begin{tikzpicture}[remember picture,overlay]
		\fill[backgroundCC] (-2,2) rectangle (25cm, -550);
\end{tikzpicture}}

% Page width 
\newcommand{\anchoPag}[0]{20cm}

% ----------------------------------------------------------
% FONT

% General
\usepackage{tgbonum}        % Font
\usepackage{listings}       % Code typesetting
\usepackage[spanish]{babel} % Load Spanish
\selectlanguage{spanish}    % Select Spanish
\usepackage{enumitem}
\usepackage{bookmark}

\setlist[itemize]{leftmargin=1.2em, itemsep=0.35em, topsep=0.35em}

% --- Table helpers ---
\newcolumntype{L}[1]{>{\raggedright\arraybackslash}p{#1}}
\newcolumntype{Y}{>{\raggedright\arraybackslash}X}
\newcolumntype{C}{>{\centering\arraybackslash}X}
\renewcommand{\arraystretch}{1.1}

% Python style
\lstdefinestyle{python}{
	language=Python,
	basicstyle=\ttfamily\small,
	commentstyle=\color{green!50!black},
	keywordstyle=\color{blue},
	numberstyle=\tiny\color{gray},
	numbers=left,
	morekeywords={>, <},
	breakatwhitespace=false,
	showstringspaces=false,
	showtabs=false,
	showspaces=false
}

% ----------------------------------------------------------
% HYPERLINKS

% General
\usepackage{hyperref}       
\hypersetup{
	colorlinks=true,
	linkcolor=links,
	filecolor=magenta,      
	urlcolor=brown,
}

% Custom commands 

% Large link
\newcommand{\bigLink}[2]{\begin{center} \fboxC{\LARGE{\href{#1}{#2}}}\end{center}}

% Small link
\newcommand{\smallLink}[2]{\begin{center}\fboxC{\href{#1}{#2}}\end{center}}

% Bold link
\newcommand{\bfLink}[2]{\href{#1}{\textbf{#2}}}


% Small URL
\newcommand{\smallUrl}[1]{\begin{center}\fboxC{\url{#1}}\end{center}}


% ----------------------------------------------------------
% CUSTOM COMMANDS FOR FIGURES

\newcommand{\espacioImagenes}[0]{-1.2cm}

% Without frame
\newcommand{\fig}[3][\espacioImagenes]{
	\hspace*{#1}
	\centering
	\includegraphics[width=#2\textwidth]{#3}
}

% With frame
\newcommand{\ffig}[2]{\begin{figure}[h]
		\hspace*{\espacioImagenes}
		\centering
		\fbox{\includegraphics[width=#1\textwidth]{#2}}
\end{figure}}

% Hyperlink with frame
\newcommand{\hfig}[3]{\begin{figure}[h]
		\hspace*{-1.4cm}
		\centering
		\color{popUp}
		\fboxC{\href{#1}{\includegraphics[width=#2\textwidth]{#3}}}
	\end{figure}
}

% Hyperlink without frame
\newcommand{\hffig}[3]{\begin{figure}[h]
		\hspace*{-1.1cm}
		\centering
		\color{popUp}
		\href{#1}{\includegraphics[width=#2\textwidth]{#3}}
	\end{figure}
}

% ----------------------------------------------------------

% Start and Contents
\newcommand{\cuadro}[1]{
	\begin{mdframed}[style=estiloGeneral]
		#1 
	\end{mdframed}
}

% Explanation video image
\newcommand{\linkExplicacion}[1]{
	\hffig{#1}{0.5}{principal/videoExplicacion}
	\vspace{-0.5cm}
}

\newcommand{\subSecLink}[2]{
	\subsubsection*{\href{#1}{\textbf{#2}}}
}

% Spacing
\newcommand{\esp}[0]{\vspace{4mm}}

% Back to start
\newcommand{\secInicio}[0]{\begin{center}\hyperref[sec:inicio]{ 
			\includegraphics[width=0.1\textwidth]{principal/up}
	}\end{center}
}


\geometry{margin=0.85in}
\AtBeginDocument{\small}

\newcommand{\ExamNameField}{\noindent\textbf{Name:}\ \rule{0.7\linewidth}{0.4pt}\par\medskip}

\newcommand{\ExamTitleBlock}[3]{%
	\begin{center}
		\Large\textbf{#1}\\
		\textbf{#2}%
		\if\relax\detokenize{#3}\relax\else\\\textbf{#3}\fi
	\end{center}
	\vspace{0.5em}
}

\newcommand{\ExamSection}[1]{\par\medskip\textbf{#1}\par\smallskip}

\newenvironment{ExamCriteria}{%
	\begin{itemize}[leftmargin=1.6em, itemsep=0.3em, topsep=0.2em]
}{%
	\end{itemize}
}

\newenvironment{ExamProblems}{%
	\begin{enumerate}[label=\textbf{P\arabic*}, leftmargin=0pt, labelsep=0.6em, itemindent=2.2em, itemsep=0.8em]
}{%
	\end{enumerate}
}

\begin{document}
	\ExamTitleBlock{10th grade}{Term 2 C6 Catch-Up Activity (Solutions)}{}
	
	\ExamSection{Problems}
	\begin{ExamProblems}
		\item
		\subsection*{Problem description}
		A student entrepreneur is evaluating a short-term retail opportunity that depends heavily on how customer traffic develops during the weekend. The decision is straightforward in structure but uncertain in outcome, because actual demand can shift quickly based on consumer mood and local conditions. Management therefore needs a probability-based framework to connect each demand scenario with its corresponding financial consequence and to justify the final choice on expected performance rather than intuition.

		\begin{center}
			\textit{Payoff table} \par
			\begin{tabular}{l c c}
				\toprule
				State of nature & Probability & Rent kiosk \\
				\midrule
				High demand & 0.60 & 36 \\
				Low demand & 0.40 & 10 \\
				\bottomrule
			\end{tabular}
		\end{center}

		Which alternative should be selected according to the maximum opportunity (minimax regret) criterion?

		\subsection*{C6}
		Best payoff in each state:
		\[
		\begin{aligned}
			\text{High demand: } &\max\{36\} = 36, \\
			\text{Low demand: } &\max\{10\} = 10. \\
		\end{aligned}
		\]
		Regret table (best payoff $-$ payoff):
		\begin{center}
			\begin{tabular}{l c c c}
				\toprule
				Alternative & High demand (0.60) & Low demand (0.40) & Maximum regret \\
				\midrule
				Rent kiosk & $36-36=0$ & $10-10=0$ & 0 \\
				\bottomrule
			\end{tabular}
		\end{center}
		The minimax regret choice is Rent kiosk because it has the smallest maximum regret (0).
		\item
		\subsection*{Problem description}
		A coffee shop chain must select a bean purchasing approach for the next month while demand remains uncertain. One alternative emphasizes product quality and brand positioning, while the other emphasizes sourcing flexibility and cost control. Because the market can evolve in more than one direction, leadership must evaluate how each plan performs across possible demand states and use expected value reasoning to support a disciplined, forward-looking decision.

		\begin{center}
			\textit{Payoff table} \par
			\begin{tabular}{l c c c}
				\toprule
				State of nature & Probability & Plan A & Plan B \\
				\midrule
				High demand & 0.55 & 84 & 72 \\
				Low demand & 0.45 & 22 & 34 \\
				\bottomrule
			\end{tabular}
		\end{center}

		Which alternative should be selected according to the maximum opportunity (minimax regret) criterion?

		\subsection*{C6}
		Best payoff in each state:
		\[
		\begin{aligned}
			\text{High demand: } &\max\{84, 72\} = 84, \\
			\text{Low demand: } &\max\{22, 34\} = 34. \\
		\end{aligned}
		\]
		Regret table (best payoff $-$ payoff):
		\begin{center}
			\begin{tabular}{l c c c}
				\toprule
				Alternative & High demand (0.55) & Low demand (0.45) & Maximum regret \\
				\midrule
				Plan A & $84-84=0$ & $34-22=12$ & 12 \\
				Plan B & $84-72=12$ & $34-34=0$ & 12 \\
				\bottomrule
			\end{tabular}
		\end{center}
		The minimax regret choice is Plan A and Plan B because they have the smallest maximum regret (12).
		\item
		\subsection*{Problem description}
		A small factory is planning seasonal production and must choose between two operating modes with different cost structures and responsiveness profiles. Market conditions may strengthen, remain stable, or soften, and each state creates a different profit implication for each mode. Since management cannot know in advance which condition will occur, the decision should be anchored in expected value so that uncertainty is incorporated systematically into the production strategy.

		\begin{center}
			\textit{Payoff table} \par
			\begin{tabular}{l c c c}
				\toprule
				State of nature & Probability & Mode A & Mode B \\
				\midrule
				Strong market & 0.30 & 95 & 88 \\
				Stable market & 0.45 & 60 & 66 \\
				Weak market & 0.25 & 18 & 30 \\
				\bottomrule
			\end{tabular}
		\end{center}

		Which alternative should be selected according to the maximum opportunity (minimax regret) criterion?

		\subsection*{C6}
		Best payoff in each state:
		\[
		\begin{aligned}
			\text{Strong market: } &\max\{95, 88\} = 95, \\
			\text{Stable market: } &\max\{60, 66\} = 66, \\
			\text{Weak market: } &\max\{18, 30\} = 30. \\
		\end{aligned}
		\]
		Regret table (best payoff $-$ payoff):
		\begin{center}
			\begin{tabular}{l c c c c}
				\toprule
				Alternative & Strong market (0.30) & Stable market (0.45) & Weak market (0.25) & Maximum regret \\
				\midrule
				Mode A & $95-95=0$ & $66-60=6$ & $30-18=12$ & 12 \\
				Mode B & $95-88=7$ & $66-66=0$ & $30-30=0$ & 7 \\
				\bottomrule
			\end{tabular}
		\end{center}
		The minimax regret choice is Mode B because it has the smallest maximum regret (7).
		\item
		\subsection*{Problem description}
		A delivery startup is selecting a fleet strategy for the coming quarter in an environment where fuel conditions are uncertain and can materially affect operating margins. Each strategic option offers a different balance between control, flexibility, and exposure to cost volatility. To choose responsibly, the firm must evaluate outcomes across the plausible fuel-cost states and rely on expected value to identify the alternative with the strongest overall economic justification.

		\begin{center}
			\textit{Payoff table} \par
			\begin{tabular}{l c c c c}
				\toprule
				State of nature & Probability & Strategy A & Strategy B & Strategy C \\
				\midrule
				Low fuel cost & 0.25 & 120 & 108 & 96 \\
				Medium fuel cost & 0.50 & 86 & 92 & 88 \\
				High fuel cost & 0.25 & 44 & 58 & 70 \\
				\bottomrule
			\end{tabular}
		\end{center}

		Which alternative should be selected according to the maximum opportunity (minimax regret) criterion?

		\subsection*{C6}
		Best payoff in each state:
		\[
		\begin{aligned}
			\text{Low fuel cost: } &\max\{120, 108, 96\} = 120, \\
			\text{Medium fuel cost: } &\max\{86, 92, 88\} = 92, \\
			\text{High fuel cost: } &\max\{44, 58, 70\} = 70. \\
		\end{aligned}
		\]
		Regret table (best payoff $-$ payoff):
		\begin{center}
			\begin{tabular}{l c c c c}
				\toprule
				Alternative & Low fuel cost (0.25) & Medium fuel cost (0.50) & High fuel cost (0.25) & Maximum regret \\
				\midrule
				Strategy A & $120-120=0$ & $92-86=6$ & $70-44=26$ & 26 \\
				Strategy B & $120-108=12$ & $92-92=0$ & $70-58=12$ & 12 \\
				Strategy C & $120-96=24$ & $92-88=4$ & $70-70=0$ & 24 \\
				\bottomrule
			\end{tabular}
		\end{center}
		The minimax regret choice is Strategy B because it has the smallest maximum regret (12).
		\item
		\subsection*{Problem description}
		A supermarket is defining an inventory policy for imported fruit while facing uncertainty in supply-chain reliability. Management recognizes that logistics conditions can range from smooth to severely disrupted, and each situation affects availability, waste risk, and revenue potential differently. Because no single scenario is guaranteed, the firm must compare policies using probability-weighted outcomes and select the option that best supports resilient profit performance.

		\begin{center}
			\textit{Payoff table} \par
			\begin{tabular}{l c c c c}
				\toprule
				State of nature & Probability & Policy A & Policy B & Policy C \\
				\midrule
				Very smooth supply & 0.20 & 74 & 82 & 86 \\
				Smooth supply & 0.35 & 68 & 76 & 78 \\
				Disrupted supply & 0.30 & 36 & 50 & 58 \\
				Severely disrupted supply & 0.15 & 8 & 24 & 34 \\
				\bottomrule
			\end{tabular}
		\end{center}

		Which alternative should be selected according to the maximum opportunity (minimax regret) criterion?

		\subsection*{C6}
		Best payoff in each state:
		\[
		\begin{aligned}
			\text{Very smooth supply: } &\max\{74, 82, 86\} = 86, \\
			\text{Smooth supply: } &\max\{68, 76, 78\} = 78, \\
			\text{Disrupted supply: } &\max\{36, 50, 58\} = 58, \\
			\text{Severely disrupted supply: } &\max\{8, 24, 34\} = 34. \\
		\end{aligned}
		\]
		Regret table (best payoff $-$ payoff):
		\begin{center}
			\begin{tabular}{l c c c c c}
				\toprule
				Alt. & Very smooth (0.20) & Smooth (0.35) & Disrupted (0.30) & Severely disrupted (0.15) & Maximum regret \\
				\midrule
				Policy A & $86-74=12$ & $78-68=10$ & $58-36=22$ & $34-8=26$ & 26 \\
				Policy B & $86-82=4$ & $78-76=2$ & $58-50=8$ & $34-24=10$ & 10 \\
				Policy C & $86-86=0$ & $78-78=0$ & $58-58=0$ & $34-34=0$ & 0 \\
				\bottomrule
			\end{tabular}
		\end{center}
		The minimax regret choice is Policy C because it has the smallest maximum regret (0).
		\item
		\subsection*{Problem description}
		An electronics retailer is choosing a launch format for a new device in a season where demand intensity may vary significantly. Each format reflects a different channel strategy and operational commitment, leading to distinct financial outcomes under different market conditions. Decision-makers therefore need to assess each format across plausible seasonal states and use expected value analysis to justify the launch approach with the strongest overall return profile.

		\begin{center}
			\textit{Payoff table} \par
			\begin{tabular}{l c c c c c}
				\toprule
				State of nature & Probability & Format A & Format B & Format C & Format D \\
				\midrule
				Boom season & 0.20 & 150 & 132 & 144 & 120 \\
				Good season & 0.30 & 110 & 118 & 124 & 108 \\
				Moderate season & 0.30 & 72 & 82 & 94 & 88 \\
				Slow season & 0.20 & 30 & 48 & 62 & 66 \\
				\bottomrule
			\end{tabular}
		\end{center}

		Which alternative should be selected according to the maximum opportunity (minimax regret) criterion?

		\subsection*{C6}
		Best payoff in each state:
		\[
		\begin{aligned}
			\text{Boom season: } &\max\{150, 132, 144, 120\} = 150, \\
			\text{Good season: } &\max\{110, 118, 124, 108\} = 124, \\
			\text{Moderate season: } &\max\{72, 82, 94, 88\} = 94, \\
			\text{Slow season: } &\max\{30, 48, 62, 66\} = 66. \\
		\end{aligned}
		\]
		Regret table (best payoff $-$ payoff):
		\begin{center}
			\begin{tabular}{l c c c c c}
				\toprule
				Alternative & Boom (0.20) & Good (0.30) & Moderate (0.30) & Slow (0.20) & Maximum regret \\
				\midrule
				Format A & $150-150=0$ & $124-110=14$ & $94-72=22$ & $66-30=36$ & 36 \\
				Format B & $150-132=18$ & $124-118=6$ & $94-82=12$ & $66-48=18$ & 18 \\
				Format C & $150-144=6$ & $124-124=0$ & $94-94=0$ & $66-62=4$ & 6 \\
				Format D & $150-120=30$ & $124-108=16$ & $94-88=6$ & $66-66=0$ & 30 \\
				\bottomrule
			\end{tabular}
		\end{center}
		The minimax regret choice is Format C because it has the smallest maximum regret (6).
		\item
		\subsection*{Problem description}
		A grain exporter must choose among competing shipping contracts while freight market conditions remain uncertain. Contract design influences both upside potential and downside protection as freight tightness changes across the planning horizon. Since management cannot predict a single freight outcome with certainty, it should evaluate all contracts under the full set of plausible states and apply expected value logic to support a defensible commercial decision.

		\begin{center}
			\textit{Payoff table}
			\begin{tabular}{l c c c c c}
				\toprule
				State of nature & Probability & Contract A & Contract B & Contract C & Contract D \\
				\midrule
				Very favorable freight & 0.15 & 160 & 152 & 146 & 132 \\
				Favorable freight & 0.25 & 138 & 140 & 136 & 126 \\
				Neutral freight & 0.30 & 102 & 114 & 120 & 116 \\
				Tight freight & 0.20 & 66 & 80 & 92 & 98 \\
				Very tight freight & 0.10 & 24 & 40 & 54 & 70 \\
				\bottomrule
			\end{tabular}
		\end{center}

		Which alternative should be selected according to the maximum opportunity (minimax regret) criterion?

		\subsection*{C6}
		Best payoff in each state:
		\[
		\begin{aligned}
			\text{Very favorable freight: } &\max\{160, 152, 146, 132\} = 160, \\
			\text{Favorable freight: } &\max\{138, 140, 136, 126\} = 140, \\
			\text{Neutral freight: } &\max\{102, 114, 120, 116\} = 120, \\
			\text{Tight freight: } &\max\{66, 80, 92, 98\} = 98, \\
			\text{Very tight freight: } &\max\{24, 40, 54, 70\} = 70. \\
		\end{aligned}
		\]
		Regret table (best payoff $-$ payoff):
		\begin{center}
			\begin{tabular}{l c c c c c c}
				\toprule
				& \shortstack{Very favorable \\ (0.15)}
				& \shortstack{Favorable \\ (0.25)}
				& \shortstack{Neutral \\ (0.30)}
				& \shortstack{Tight \\ (0.20)}
				& \shortstack{Very tight \\ (0.10)}
				& Maximum regret \\
				\midrule
				Contract A & $160-160=0$ & $140-138=2$ & $120-102=18$ & $98-66=32$ & $70-24=46$ & 46 \\
				Contract B & $160-152=8$ & $140-140=0$ & $120-114=6$ & $98-80=18$ & $70-40=30$ & 30 \\
				Contract C & $160-146=14$ & $140-136=4$ & $120-120=0$ & $98-92=6$ & $70-54=16$ & 16 \\
				Contract D & $160-132=28$ & $140-126=14$ & $120-116=4$ & $98-98=0$ & $70-70=0$ & 28 \\
				\bottomrule
			\end{tabular}
		\end{center}
		The minimax regret choice is Contract C because it has the smallest maximum regret (16).
		\item
		\subsection*{Problem description}
		A fashion retailer is evaluating expansion plans for urban markets where demand can shift across several intensity levels. Each plan reflects a different growth posture, creating distinct exposure to both strong and weak market outcomes. To align strategy with financial discipline, management should compare alternatives across all plausible demand states and use expected value as the core criterion for selecting the expansion path.

		\begin{center}
			\textit{Payoff table} \par
			\begin{tabular}{l c c c c c c}
				\toprule
				State of nature & Probability & Plan A & Plan B & Plan C & Plan D & Plan E \\
				\midrule
				Very high demand & 0.12 & 190 & 182 & 176 & 168 & 156 \\
				High demand & 0.23 & 150 & 154 & 152 & 146 & 140 \\
				Medium demand & 0.30 & 108 & 120 & 128 & 126 & 124 \\
				Low demand & 0.22 & 58 & 76 & 90 & 98 & 104 \\
				Very low demand & 0.13 & 12 & 34 & 52 & 64 & 76 \\
				\bottomrule
			\end{tabular}
		\end{center}

		Which alternative should be selected according to the maximum opportunity (minimax regret) criterion?

		\subsection*{C6}
		Best payoff in each state:
		\[
		\begin{aligned}
			\text{Very high demand: } &\max\{190, 182, 176, 168, 156\} = 190, \\
			\text{High demand: } &\max\{150, 154, 152, 146, 140\} = 154, \\
			\text{Medium demand: } &\max\{108, 120, 128, 126, 124\} = 128, \\
			\text{Low demand: } &\max\{58, 76, 90, 98, 104\} = 104, \\
			\text{Very low demand: } &\max\{12, 34, 52, 64, 76\} = 76. \\
		\end{aligned}
		\]
		Regret table (best payoff $-$ payoff):
		\begin{center}
			\begin{tabular}{l c c c c c c}
				\toprule
				Alternative 
				& \shortstack{Very high \\ (0.12)}
				& \shortstack{High \\ (0.23)}
				& \shortstack{Medium \\ (0.30)}
				& \shortstack{Low \\ (0.22)}
				& \shortstack{Very low \\ (0.13)}
				& Maximum regret \\
				\midrule
				Plan A & $190-190=0$ & $154-150=4$ & $128-108=20$ & $104-58=46$ & $76-12=64$ & 64 \\
				Plan B & $190-182=8$ & $154-154=0$ & $128-120=8$ & $104-76=28$ & $76-34=42$ & 42 \\
				Plan C & $190-176=14$ & $154-152=2$ & $128-128=0$ & $104-90=14$ & $76-52=24$ & 24 \\
				Plan D & $190-168=22$ & $154-146=8$ & $128-126=2$ & $104-98=6$ & $76-64=12$ & 22 \\
				Plan E & $190-156=34$ & $154-140=14$ & $128-124=4$ & $104-104=0$ & $76-76=0$ & 34 \\
				\bottomrule
			\end{tabular}
		\end{center}
		The minimax regret choice is Plan D because it has the smallest maximum regret (22).
		\item
		\subsection*{Problem description}
		A regional energy distributor is selecting a pricing package in a context where weather-driven demand can change materially during the operating period. Each package offers a different trade-off between high-demand capture and low-demand protection. Because weather patterns are uncertain and financially significant, management needs a probability-weighted evaluation of outcomes to determine which package provides the most robust expected economic result.

		\begin{center}
			\textit{Payoff table}
			\begin{tabular}{l c c c c c c}
				\toprule
				State of nature & Probability & Package A & Package B & Package C & Package D & Package E \\
				\midrule
				Extreme cold & 0.10 & 220 & 210 & 204 & 196 & 186 \\
				Cold & 0.20 & 184 & 186 & 182 & 176 & 170 \\
				Mild & 0.25 & 142 & 152 & 158 & 156 & 152 \\
				Warm & 0.20 & 96 & 110 & 122 & 128 & 130 \\
				Hot & 0.15 & 58 & 74 & 88 & 98 & 106 \\
				Extreme hot & 0.10 & 22 & 40 & 54 & 68 & 82 \\
				\bottomrule
			\end{tabular}
		\end{center}

		Which alternative should be selected according to the maximum opportunity (minimax regret) criterion?

		\subsection*{C6}
		Best payoff in each state:
		\[
		\begin{aligned}
			\text{Extreme cold: } &\max\{220, 210, 204, 196, 186\} = 220, \\
			\text{Cold: } &\max\{184, 186, 182, 176, 170\} = 186, \\
			\text{Mild: } &\max\{142, 152, 158, 156, 152\} = 158, \\
			\text{Warm: } &\max\{96, 110, 122, 128, 130\} = 130, \\
			\text{Hot: } &\max\{58, 74, 88, 98, 106\} = 106, \\
			\text{Extreme hot: } &\max\{22, 40, 54, 68, 82\} = 82. \\
		\end{aligned}
		\]
		Regret table (best payoff $-$ payoff):
		
		\hspace*{-1.5cm}%
		\begin{tabular}{l c c c c c c c}
			\toprule
			Alternative 
			& \shortstack{Extreme cold\\(0.10)} 
			& \shortstack{Cold\\(0.20)} 
			& \shortstack{Mild\\(0.25)} 
			& \shortstack{Warm\\(0.20)} 
			& \shortstack{Hot\\(0.15)} 
			& \shortstack{Extreme hot\\(0.10)} 
			& \shortstack{Maximum\\regret} \\
			\midrule
			Package A & $220-220=0$ & $186-184=2$ & $158-142=16$ & $130-96=34$ & $106-58=48$ & $82-22=60$ & 60 \\
			Package B & $220-210=10$ & $186-186=0$ & $158-152=6$ & $130-110=20$ & $106-74=32$ & $82-40=42$ & 42 \\
			Package C & $220-204=16$ & $186-182=4$ & $158-158=0$ & $130-122=8$ & $106-88=18$ & $82-54=28$ & 28 \\
			Package D & $220-196=24$ & $186-176=10$ & $158-156=2$ & $130-128=2$ & $106-98=8$ & $82-68=14$ & 24 \\
			Package E & $220-186=34$ & $186-170=16$ & $158-152=6$ & $130-130=0$ & $106-106=0$ & $82-82=0$ & 34 \\
			\bottomrule
		\end{tabular}
	The minimax regret choice is Package D because it has the smallest maximum regret (24).
		\item
		\subsection*{Problem description}
		A national logistics group is deciding on a network design while macro-economic conditions may evolve through expansion, slowdown, contraction, and recovery patterns. Each design presents a different balance between growth capacity and resilience under weaker environments. Given this broad uncertainty set, executives should compare all designs across the plausible states of nature and rely on expected value analysis to choose the configuration with the strongest expected performance.

		\begin{center}
			\textit{Payoff table}
			\begin{tabular}{l c c c c c c c}
				\toprule
				State of nature & Probability & Design A & Design B & Design C & Design D & Design E & Design F \\
				\midrule
				Rapid expansion & 0.14 & 260 & 248 & 242 & 234 & 226 & 214 \\
				Steady growth & 0.20 & 220 & 222 & 218 & 212 & 206 & 198 \\
				Flat market & 0.24 & 170 & 182 & 188 & 190 & 188 & 184 \\
				Mild contraction & 0.16 & 118 & 132 & 146 & 156 & 162 & 164 \\
				Strong contraction & 0.12 & 70 & 86 & 102 & 118 & 130 & 142 \\
				Recovery transition & 0.14 & 136 & 148 & 158 & 164 & 168 & 170 \\
				\bottomrule
			\end{tabular}
		\end{center}

		Which alternative should be selected according to the maximum opportunity (minimax regret) criterion?

		\subsection*{C6}
		Best payoff in each state:
		\[
		\begin{aligned}
			\text{Rapid expansion: } &\max\{260, 248, 242, 234, 226, 214\} = 260, \\
			\text{Steady growth: } &\max\{220, 222, 218, 212, 206, 198\} = 222, \\
			\text{Flat market: } &\max\{170, 182, 188, 190, 188, 184\} = 190, \\
			\text{Mild contraction: } &\max\{118, 132, 146, 156, 162, 164\} = 164, \\
			\text{Strong contraction: } &\max\{70, 86, 102, 118, 130, 142\} = 142, \\
			\text{Recovery transition: } &\max\{136, 148, 158, 164, 168, 170\} = 170. \\
		\end{aligned}
		\]
		\begin{center} Regret table (best payoff $-$ payoff): \end{center}
		\hspace*{-2cm}%
		\begin{tabular}{l c c c c c c c}
			\toprule
			Alternative 
			& \shortstack{Rapid expansion\\(0.14)} 
			& \shortstack{Steady growth\\(0.20)} 
			& \shortstack{Flat market\\(0.24)} 
			& \shortstack{Mild contract.\\(0.16)} 
			& \shortstack{Contraction\\(0.12)} 
			& \shortstack{Recovery\\(0.14)} 
			& \shortstack{Maximum\\regret} \\
			\midrule
			Design A & $260-260=0$ & $222-220=2$ & $190-170=20$ & $164-118=46$ & $142-70=72$ & $170-136=34$ & 72 \\
			Design B & $260-248=12$ & $222-222=0$ & $190-182=8$ & $164-132=32$ & $142-86=56$ & $170-148=22$ & 56 \\
			Design C & $260-242=18$ & $222-218=4$ & $190-188=2$ & $164-146=18$ & $142-102=40$ & $170-158=12$ & 40 \\
			Design D & $260-234=26$ & $222-212=10$ & $190-190=0$ & $164-156=8$ & $142-118=24$ & $170-164=6$ & 26 \\
			Design E & $260-226=34$ & $222-206=16$ & $190-188=2$ & $164-162=2$ & $142-130=12$ & $170-168=2$ & 34 \\
			Design F & $260-214=46$ & $222-198=24$ & $190-184=6$ & $164-164=0$ & $142-142=0$ & $170-170=0$ & 46 \\
			\bottomrule
		\end{tabular}
		
		The minimax regret choice is Design D because it has the smallest maximum regret (26).
	\end{ExamProblems}
\end{document}
