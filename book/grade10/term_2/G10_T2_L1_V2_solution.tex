\makeatletter
\def\input@path{{./}{../}{../../}{preamble/}{../preamble/}{../../preamble/}}
\makeatother
% ----------------------------------------------------------
% GENERAL 

% File
\documentclass[11pt]{book}

% Margins
\usepackage[margin=1in]{geometry}

\linespread{1.2}            % Line spacing
\usepackage[utf8]{inputenc}
\usepackage[T1]{fontenc}
\usepackage{lmodern}
\usepackage{microtype}
\setlength{\parindent}{0pt}
\setlength{\parskip}{6pt}
\usepackage{booktabs}

% ----------------------------------------------------------
% TABLES
\usepackage{multicol}
\usepackage{longtable} 
\usepackage{array}
\usepackage{booktabs}
\usepackage{tabularx}
\usepackage{multirow}

% ----------------------------------------------------------
% MATHEMATICS
\usepackage{amsmath}
\usepackage{amssymb}
\usepackage{amsfonts}
\usepackage{mathtools}

% ----------------------------------------------------------
% HIDDEN CONTENT
\usepackage{ifthen}
% Define a boolean switch
\newboolean{explicaciones}
% Set the boolean switch to true or false
% Change to true to show the content

% Explanations
\newcommand{\explicacion}[2]{
	\ifthenelse{\boolean{explicaciones}}{#1}{#2}
}
\newcommand{\mostrarExplicaciones}[1]{\setboolean{explicaciones}{#1}}

% ----------------------------------------------------------
% NUMBERING

\usepackage{fancyhdr}
\pagestyle{empty} % Ensures the entire document has no page numbers

\usepackage{tocloft}
\renewcommand{\cftdot}{} % Remove dots for sections, if any
\renewcommand{\cftsecleader}{\cftdotfill{\cftdotsep}} % Remove dots for sections, if any
\cftpagenumbersoff{section} % Removes page numbers from sections
\cftpagenumbersoff{subsection} % Removes page numbers from subsections

% ----------------------------------------------------------
% IMAGES 

% General settings
\usepackage{graphicx}       % Insert images
\usepackage{float}          % Position images
% \usepackage{subfigure}      % Subfigures
\graphicspath{{imgs}}       % Image location
\usepackage{subcaption}     % Subfigures II
\usepackage{verbatim}

% Figures
\usepackage{tikz}
\usetikzlibrary{arrows.meta,positioning,trees}

% Colors
\usepackage{xcolor}     
\definecolor{popUp}{HTML}{666666}
\definecolor{popUpIn}{HTML}{CED9E0}
\definecolor{backgroundC}{HTML}{D0E8F2}
\definecolor{backgroundCC}{HTML}{FFFFFF}
\definecolor{borders}{HTML}{8c120d}
\definecolor{padding}{HTML}{77D0D7}
\definecolor{links}{HTML}{CC6F5F}

% ----------------------------------------------------------
% FRAMES

% General settings
\usepackage{tcolorbox}
\usepackage{adjustbox}          % Adjusted frame  
\setlength{\fboxrule}{3pt}  % Line width
\setlength{\fboxsep}{3pt}   % Box padding

% General frames
\usepackage{mdframed}   

\mdfdefinestyle{estiloGeneral}{    % General style
	linecolor=black,
	linewidth=1.5pt,
	roundcorner=10pt,
	backgroundcolor=backgroundC,
	innerbottommargin=0pt
}
\mdfdefinestyle{code}{          % Code style
	linecolor=black,
	linewidth=1.5pt,
	roundcorner=10pt,
	backgroundcolor=darkgray!10,
	innerbottommargin=0pt
}

% Image frame
\newtcbox{\fboxC}{
	colback=backgroundC,
	colframe=popUp,
	arc=10pt,
	boxrule=3pt,
	boxsep=0pt, % Change the padding here
	nobeforeafter
}

% ----------------------------------------------------------
% PAGE SETTINGS

% Background 
\newcommand{\background}[0]{\begin{tikzpicture}[remember picture,overlay]
		\fill[backgroundC] (-2,2) rectangle (25cm, -550);
\end{tikzpicture}}

\newcommand{\backgroundC}[0]{\begin{tikzpicture}[remember picture,overlay]
		\fill[backgroundCC] (-2,2) rectangle (25cm, -550);
\end{tikzpicture}}

% Page width 
\newcommand{\anchoPag}[0]{20cm}

% ----------------------------------------------------------
% FONT

% General
\usepackage{tgbonum}        % Font
\usepackage{listings}       % Code typesetting
\usepackage[spanish]{babel} % Load Spanish
\selectlanguage{spanish}    % Select Spanish
\usepackage{enumitem}
\usepackage{bookmark}

\setlist[itemize]{leftmargin=1.2em, itemsep=0.35em, topsep=0.35em}

% --- Table helpers ---
\newcolumntype{L}[1]{>{\raggedright\arraybackslash}p{#1}}
\newcolumntype{Y}{>{\raggedright\arraybackslash}X}
\newcolumntype{C}{>{\centering\arraybackslash}X}
\renewcommand{\arraystretch}{1.1}

% Python style
\lstdefinestyle{python}{
	language=Python,
	basicstyle=\ttfamily\small,
	commentstyle=\color{green!50!black},
	keywordstyle=\color{blue},
	numberstyle=\tiny\color{gray},
	numbers=left,
	morekeywords={>, <},
	breakatwhitespace=false,
	showstringspaces=false,
	showtabs=false,
	showspaces=false
}

% ----------------------------------------------------------
% HYPERLINKS

% General
\usepackage{hyperref}       
\hypersetup{
	colorlinks=true,
	linkcolor=links,
	filecolor=magenta,      
	urlcolor=brown,
}

% Custom commands 

% Large link
\newcommand{\bigLink}[2]{\begin{center} \fboxC{\LARGE{\href{#1}{#2}}}\end{center}}

% Small link
\newcommand{\smallLink}[2]{\begin{center}\fboxC{\href{#1}{#2}}\end{center}}

% Bold link
\newcommand{\bfLink}[2]{\href{#1}{\textbf{#2}}}


% Small URL
\newcommand{\smallUrl}[1]{\begin{center}\fboxC{\url{#1}}\end{center}}


% ----------------------------------------------------------
% CUSTOM COMMANDS FOR FIGURES

\newcommand{\espacioImagenes}[0]{-1.2cm}

% Without frame
\newcommand{\fig}[3][\espacioImagenes]{
	\hspace*{#1}
	\centering
	\includegraphics[width=#2\textwidth]{#3}
}

% With frame
\newcommand{\ffig}[2]{\begin{figure}[h]
		\hspace*{\espacioImagenes}
		\centering
		\fbox{\includegraphics[width=#1\textwidth]{#2}}
\end{figure}}

% Hyperlink with frame
\newcommand{\hfig}[3]{\begin{figure}[h]
		\hspace*{-1.4cm}
		\centering
		\color{popUp}
		\fboxC{\href{#1}{\includegraphics[width=#2\textwidth]{#3}}}
	\end{figure}
}

% Hyperlink without frame
\newcommand{\hffig}[3]{\begin{figure}[h]
		\hspace*{-1.1cm}
		\centering
		\color{popUp}
		\href{#1}{\includegraphics[width=#2\textwidth]{#3}}
	\end{figure}
}

% ----------------------------------------------------------

% Start and Contents
\newcommand{\cuadro}[1]{
	\begin{mdframed}[style=estiloGeneral]
		#1 
	\end{mdframed}
}

% Explanation video image
\newcommand{\linkExplicacion}[1]{
	\hffig{#1}{0.5}{principal/videoExplicacion}
	\vspace{-0.5cm}
}

\newcommand{\subSecLink}[2]{
	\subsubsection*{\href{#1}{\textbf{#2}}}
}

% Spacing
\newcommand{\esp}[0]{\vspace{4mm}}

% Back to start
\newcommand{\secInicio}[0]{\begin{center}\hyperref[sec:inicio]{ 
			\includegraphics[width=0.1\textwidth]{principal/up}
	}\end{center}
}


\geometry{margin=0.85in}
\AtBeginDocument{\small}

\newcommand{\ExamNameField}{\noindent\textbf{Name:}\ \rule{0.7\linewidth}{0.4pt}\par\medskip}

\newcommand{\ExamTitleBlock}[3]{%
	\begin{center}
		\Large\textbf{#1}\\
		\textbf{#2}%
		\if\relax\detokenize{#3}\relax\else\\\textbf{#3}\fi
	\end{center}
	\vspace{0.5em}
}

\newcommand{\ExamSection}[1]{\par\medskip\textbf{#1}\par\smallskip}

\newenvironment{ExamCriteria}{%
	\begin{itemize}[leftmargin=1.6em, itemsep=0.3em, topsep=0.2em]
}{%
	\end{itemize}
}

\newenvironment{ExamProblems}{%
	\begin{enumerate}[label=\textbf{P\arabic*}, leftmargin=0pt, labelsep=0.6em, itemindent=2.2em, itemsep=0.8em]
}{%
	\end{enumerate}
}

\begin{document}
	\ExamTitleBlock{10th grade}{Learning evidence 2.1: Analysis of Decisions (Solutions)}{}
	
	\ExamSection{Problems}
	\begin{ExamProblems}
		\item
		\subsection*{Problem description}
		A school cafeteria must choose one of three menu plans for a two-week festival: Plan A (local produce focus),
		Plan B (mixed menu), or Plan C (bulk staples). Demand for meals can be high, medium, or low. Based on past festivals, the probability of high demand is $0.35$, medium demand is $0.45$, and low demand is $0.20$. Profits are measured in hundreds of dollars. If demand is high, Plan A yields 92, Plan B yields 70, and Plan C yields 55. If demand is medium, Plan A yields 50, Plan B yields 62, and Plan C yields 48. If demand is low, Plan A yields $-8$, Plan B yields 30, and Plan C yields 38. The stated probabilities must be used for an expected value comparison as part of the decision analysis.

		\subsection*{C2}
		\begin{center}
			\begin{tabular}{l p{0.74\linewidth}}
				\toprule
				Decision Alternatives & Plan A (local produce), Plan B (mixed menu), Plan C (bulk staples) \\
				States of Nature & Demand high, medium, low with probabilities $0.35$, $0.45$, $0.20$ \\
				Events & Realized demand level in the serving period \\
				Consequences & Profit in hundreds of dollars after all costs \\
				\bottomrule
			\end{tabular}
		\end{center}

		\subsection*{C3}
		Payoff = revenue $-$ cost. In this problem, the profits for each plan are already provided and directly represent the corresponding payoffs.
		\begin{center}
			\textit{Payoff table} \\
			\begin{tabular}{l c c c c}
				\toprule
				State of nature & Probability & Plan A & Plan B & Plan C \\
				\midrule
				High demand & 0.35 & 92 & 70 & 55 \\
				Medium demand & 0.45 & 50 & 62 & 48 \\
				Low demand & 0.20 & -8 & 30 & 38 \\
				\bottomrule
			\end{tabular}
		\end{center}

		\subsection*{C4}
		\[
		\begin{aligned}
		\max(\text{Plan A}) &= \max\{92,50,-8\} = 92,\\
		\max(\text{Plan B}) &= \max\{70,62,30\} = 70,\\
		\max(\text{Plan C}) &= \max\{55,48,38\} = 55.
		\end{aligned}
		\]
		\[
		\max\{\max(\text{Plan A}), \max(\text{Plan B}), \max(\text{Plan C})\}
		= \max\{92, 70, 55\}
		= 92.
		\]
		The Maximax choice is Plan A. It is \emph{risky} because it only considers the highest possible profit.

		\subsection*{C5}
		\[
		\begin{aligned}
		\min(\text{Plan A}) &= \min\{92,50,-8\} = -8,\\
		\min(\text{Plan B}) &= \min\{70,62,30\} = 30,\\
		\min(\text{Plan C}) &= \min\{55,48,38\} = 38.
		\end{aligned}
		\]
		\[
		\max\{\min(\text{Plan A}), \min(\text{Plan B}), \min(\text{Plan C})\}
		= \max\{-8, 30, 38\}
		= 38.
		\]
		The Maximin choice is Plan C with 38 because it gives the best worst-case outcome. It is \emph{safe} because it prioritizes the least harmful result.

		\subsection*{C1}
		\[
		\begin{aligned}
		EV_A &= 0.35(92)+0.45(50)+0.20(-8)=32.2+22.5-1.6=53.1,\\
		EV_B &= 0.35(70)+0.45(62)+0.20(30)=24.5+27.9+6=58.4,\\
		EV_C &= 0.35(55)+0.45(48)+0.20(38)=19.25+21.6+7.6=48.45.
		\end{aligned}
		\]
		The expected value criterion chooses Plan B because $58.4$ is the largest expected profit. This differs from the Maximax choice Plan A and the Maximin choice Plan C, so the expected value gives a balanced decision using the probabilities.

		\item
		\subsection*{Problem description}
		A delivery cooperative is choosing a fleet strategy for the next season: Strategy A (electric vans),
		Strategy B (hybrid vans), or Strategy C (diesel vans). Demand for deliveries can be strong, steady, or weak, and fuel costs can be low or high. The cooperative estimates the demand probabilities as $0.30$ (strong), $0.50$ (steady), and $0.20$ (weak). Fuel costs are independent of demand, with probabilities $0.55$ (low) and $0.45$ (high). Revenue, in thousands of dollars, depends on the demand level and strategy. Under strong demand, revenue is 420 for Strategy A, 380 for Strategy B, and 340 for Strategy C. Under steady demand, revenue is 320 for Strategy A, 300 for Strategy B, and 270 for Strategy C. Under weak demand, revenue is 210 for Strategy A, 230 for Strategy B, and 220 for Strategy C. Operating costs, in thousands of dollars, depend on the fuel cost level and strategy. If fuel costs are low, operating costs are 160 for Strategy A, 190 for Strategy B, and 210 for Strategy C. If fuel costs are high, operating costs are 220 for Strategy A, 250 for Strategy B, and 290 for Strategy C. The cooperative should treat each combined demand and fuel-cost outcome as a state of nature and use the stated probabilities to compute expected profits.

		\subsection*{C2}
		\begin{center}
			\begin{tabular}{l p{0.74\linewidth}}
				\toprule
				Decision Alternatives & Strategy A (electric), Strategy B (hybrid), Strategy C (diesel) \\
				States of Nature & Demand strong, steady, weak with fuel costs low or high \\
				Events & Realized demand level paired with fuel cost conditions in the season \\
				Consequences & Profit in thousands of dollars from revenue minus operating costs \\
				\bottomrule
			\end{tabular}
		\end{center}

		\subsection*{C3}
		Payoff = revenue $-$ cost, where revenue is determined by demand and cost is determined by fuel prices.
		\begin{center}
			\begin{minipage}[t]{0.48\linewidth}
				\textit{Revenue parameters by strategy}
				\begin{center}
					\begin{tabular}{l c c c}
						\toprule
						Strategy & Strong & Steady & Weak \\
						\midrule
						Probabilities & 0.30 & 0.50 & 0.20 \\
						Strategy A & 420 & 320 & 210 \\
						Strategy B & 380 & 300 & 230 \\
						Strategy C & 340 & 270 & 220 \\
						\bottomrule
					\end{tabular}
				\end{center}
			\end{minipage}
			\hfill
			\begin{minipage}[t]{0.48\linewidth}
				\textit{Cost parameters by strategy}
				\begin{center}
					\begin{tabular}{l c c}
						\toprule
						Strategy & Low cost & High cost \\
						\midrule
						Probabilities& 0.55 & 0.45 \\
						Strategy A & 160 & 220 \\
						Strategy B & 190 & 250 \\
						Strategy C & 210 & 290 \\
						\bottomrule
					\end{tabular}
				\end{center}
			\end{minipage}
		\end{center}
		Profit is computed as Revenue minus Cost for each alternative and state.

		\begin{center}
			\textit{Profit parameters by strategy}
			\begin{tabular}{l p{0.17\linewidth} p{0.2\linewidth} p{0.2\linewidth} p{0.2\linewidth}}
				\toprule
				State of nature & Probability & Strategy A & Strategy B & Strategy C \\
				\midrule
				Strong demand, low cost &
				$\begin{array}{l}
					0.3 \cdot 0.55\\
					= 0.165
				\end{array}$ &
				$\begin{array}{l}
					420 - 160\\
					= 260
				\end{array}$ &
				$\begin{array}{l}
					380 - 190\\
					= 190
				\end{array}$ &
				$\begin{array}{l}
					340 - 210\\
					= 130
				\end{array}$ \\
				Strong demand, high cost &
				$\begin{array}{l}
					0.3 \cdot 0.45\\
					= 0.135
				\end{array}$ &
				$\begin{array}{l}
					420 - 220\\
					= 200
				\end{array}$ &
				$\begin{array}{l}
					380 - 250\\
					= 130
				\end{array}$ &
				$\begin{array}{l}
					340 - 290\\
					= 50
				\end{array}$ \\
				Steady demand, low cost &
				$\begin{array}{l}
					0.5 \cdot 0.55\\
					= 0.275
				\end{array}$ &
				$\begin{array}{l}
					320 - 160\\
					= 160
				\end{array}$ &
				$\begin{array}{l}
					300 - 190\\
					= 110
				\end{array}$ &
				$\begin{array}{l}
					270 - 210\\
					= 60
				\end{array}$ \\
				Steady demand, high cost &
				$\begin{array}{l}
					0.5 \cdot 0.55 \\
					= 0.225
				\end{array}$ &
				$\begin{array}{l}
					320 - 220\\
					= 100
				\end{array}$ &
				$\begin{array}{l}
					300 - 250\\
					= 50
				\end{array}$ &
				$\begin{array}{l}
					270 - 290\\
					= -20
				\end{array}$ \\
				Weak demand, low cost &
				$\begin{array}{l}
					0.2 \cdot 0.55\\
					= 0.11
				\end{array}$ &
				$\begin{array}{l}
					210 - 160\\
					= 50
				\end{array}$ &
				$\begin{array}{l}
					230 - 190\\
					= 40
				\end{array}$ &
				$\begin{array}{l}
					220 - 210\\
					= 10
				\end{array}$ \\
				Weak demand, high cost &
				$\begin{array}{l}
					0.2 \cdot 0.55\\
					= 0.09
				\end{array}$ &
				$\begin{array}{l}
					210 - 220\\
					= -10
				\end{array}$ &
				$\begin{array}{l}
					230 - 250\\
					= -20
				\end{array}$ &
				$\begin{array}{l}
					220 - 290\\
					= -70
				\end{array}$ \\
				\bottomrule
			\end{tabular}
		\end{center}
		\begin{center}
			\textit{Payoff table} \\
			\begin{tabular}{l c c c c}
				\toprule
				State of nature & Probability & Strategy A & Strategy B & Strategy C \\
				\midrule
				Strong demand, low cost & 0.165 & 260 & 190 & 130 \\
				Strong demand, high cost & 0.135 & 200 & 130 & 50 \\
				Steady demand, low cost & 0.275 & 160 & 110 & 60 \\
				Steady demand, high cost & 0.225 & 100 & 50 & -20 \\
				Weak demand, low cost & 0.11 & 50 & 40 & 10 \\
				Weak demand, high cost & 0.09 & -10 & -20 & -70 \\
				\bottomrule
			\end{tabular}
		\end{center}

		\subsection*{C4}
		\[
		\begin{aligned}
		\max(\text{Strategy A}) &= \max\{260,200,160,100,50,-10\} = 260,\\
		\max(\text{Strategy B}) &= \max\{190,130,110,50,40,-20\} = 190,\\
		\max(\text{Strategy C}) &= \max\{130,50,60,-20,10,-70\} = 130.
		\end{aligned}
		\]
		\[
		\max\{\max(\text{Strategy A}), \max(\text{Strategy B}), \max(\text{Strategy C})\}
		= \max\{260, 190, 130\}
		= 260.
		\]
		The Maximax choice is Strategy A. The criterion is \emph{risky} because it focuses only on the best possible payoff and ignores how likely it is.

		\subsection*{C5}
		\[
		\begin{aligned}
		\min(\text{Strategy A}) &= \min\{260,200,160,100,50,-10\} = -10,\\
		\min(\text{Strategy B}) &= \min\{190,130,110,50,40,-20\} = -20,\\
		\min(\text{Strategy C}) &= \min\{130,50,60,-20,10,-70\} = -70.
		\end{aligned}
		\]
		\[
		\max\{\min(\text{Strategy A}), \min(\text{Strategy B}), \min(\text{Strategy C})\}
		= \max\{-10, -20, -70\}
		= -10.
		\]
		The Maximin choice is Strategy A with $-10$ because it has the least severe worst-case loss. This is \emph{safe} because it protects against the worst outcome.

		\subsection*{C1}
		Expected values use the combined-state probabilities.
		\[
		\begin{aligned}
		EV_A &= 0.165(260)+0.135(200)+0.275(160)+0.225(100)+0.11(50)+0.09(-10)\\
		&= 42.9+27+44+22.5+5.5-0.9=141.0,\\
		EV_B &= 0.165(190)+0.135(130)+0.275(110)+0.225(50)+0.11(40)+0.09(-20)\\
		&= 31.35+17.55+30.25+11.25+4.4-1.8=93.0,\\
		EV_C &= 0.165(130)+0.135(50)+0.275(60)+0.225(-20)+0.11(10)+0.09(-70)\\
		&= 21.45+6.75+16.5-4.5+1.1-6.3=35.0.
		\end{aligned}
		\]
		The expected value criterion chooses Strategy A because $141.0$ is the largest expected profit. This aligns with both the Maximax and Maximin outcomes for Strategy A, so it is the best overall decision under uncertainty.

		\item
		\subsection*{Problem description}
		A community sports center must choose a membership model for the next year: Model A (standard monthly pass),
		Model B (premium pass with classes), or Model C (flex pass). Demand for memberships can be high, medium, or low, and staffing costs can be favorable or unfavorable. Demand probabilities are $0.25$ (high), $0.50$ (medium), and $0.25$ (low). Staffing costs are independent of demand, with probabilities $0.60$ (favorable) and $0.40$ (unfavorable). Expected membership counts depend on demand and model: under high demand the center expects 620 members with Model A, 480 with Model B, and 700 with Model C; under medium demand it expects 460 with Model A, 360 with Model B, and 520 with Model C; under low demand it expects 300 with Model A, 240 with Model B, and 360 with Model C. Membership fees are $35$ for Model A, $48$ for Model B, and $30$ for Model C per member per year. Variable staffing cost per member is $18$ when costs are favorable and $26$ when costs are unfavorable. All monetary amounts in this point can be treated in dollars. Use the combined demand and staffing outcomes as states of nature and the stated probabilities to compute expected profits.

		\subsection*{C2}
		\begin{center}
			\begin{tabular}{l p{0.74\linewidth}}
				\toprule
				Decision Alternatives & Model A (standard), Model B (premium), Model C (flex) \\
				States of Nature & Demand high, medium, low with staffing costs favorable or unfavorable \\
				Events & Realized demand level paired with staffing cost conditions in the year \\
				Consequences & Annual profit in dollars from membership fees minus staffing costs \\
				\bottomrule
			\end{tabular}
		\end{center}

		\subsection*{C3}
		Payoff = revenue $-$ cost, where revenue = (members $\times$ fee) and cost = (members $\times$ variable cost).

		\begin{center}
			\textit{Members per model by state of nature}\\
			\begin{tabular}{l c c c}
				\toprule
				Model & High & Medium & Low \\
				\midrule
				Model A & 620 & 460 & 300 \\
				Model B & 480 & 360 & 240 \\
				Model C & 700 & 520 & 360 \\
				\bottomrule
		\end{tabular}
		\end{center}
		\begin{center}
			\begin{minipage}[t]{0.48\linewidth}
				\textit{Cost parameters by member}
				\begin{center}
					\begin{tabular}{l c c}
						\toprule
						Model & Favorable & Unfavorable \\
						\midrule
						Model A & 18 & 26 \\
						Model B & 18 & 26 \\
						Model C & 18 & 26 \\
						\bottomrule
					\end{tabular}
				\end{center}
			\end{minipage}
			\hfill
			\begin{minipage}[t]{0.48\linewidth}
				\textit{Revenue parameters by member}
				\begin{center}
					\begin{tabular}{l c c c}
						\toprule
						Model & High & Medium & Low \\
						\midrule
						Model A & 35 & 35 & 35 \\
						Model B & 48 & 48 & 48 \\
						Model C & 30 & 30 & 30 \\
						\bottomrule
					\end{tabular}
				\end{center}
			\end{minipage}
		\end{center}
		\begin{center}
			\textit{Revenue parameters by model} \\
			\begin{tabular}{l c c c}
				\toprule
				State of nature & Model A & Model B & Model C \\
				\midrule
				High demand &
				$\begin{array}{l}
					620\cdot 35\\
					= 21700
				\end{array}$ &
				$\begin{array}{l}
					480\cdot 48\\
					= 23040
				\end{array}$ &
				$\begin{array}{l}
					700\cdot 30\\
					= 21000
				\end{array}$ \\
				Medium demand &
				$\begin{array}{l}
					460\cdot 35\\
					= 16100
				\end{array}$ &
				$\begin{array}{l}
					360\cdot 48\\
					= 17280
				\end{array}$ &
				$\begin{array}{l}
					520\cdot 30\\
					= 15600
				\end{array}$ \\
				Low demand &
				$\begin{array}{l}
					300\cdot 35\\
					= 10500
				\end{array}$ &
				$\begin{array}{l}
					240\cdot 48\\
					= 11520
				\end{array}$ &
				$\begin{array}{l}
					360\cdot 30\\
					= 10800
				\end{array}$ \\
				\bottomrule
			\end{tabular}
		\end{center}
		\begin{center}
			\textit{Costs parameters by model} \\
			\begin{tabular}{l c c c}
				\toprule
				State of nature & Model A & Model B & Model C \\
				\midrule
				High demand, favorable cost &
				$\begin{array}{l}
					620\cdot 18\\
					= 11160
				\end{array}$ &
				$\begin{array}{l}
					480\cdot 18\\
					= 8640
				\end{array}$ &
				$\begin{array}{l}
					700\cdot 18\\
					= 12600
				\end{array}$ \\
				High demand, unfavorable cost &
				$\begin{array}{l}
					620\cdot 26\\
					= 16120
				\end{array}$ &
				$\begin{array}{l}
					480\cdot 26\\
					= 12480
				\end{array}$ &
				$\begin{array}{l}
					700\cdot 26\\
					= 18200
				\end{array}$ \\
				Medium demand, favorable cost &
				$\begin{array}{l}
					460\cdot 18\\
					= 8280
				\end{array}$ &
				$\begin{array}{l}
					360\cdot 18\\
					= 6480
				\end{array}$ &
				$\begin{array}{l}
					520\cdot 18\\
					= 9360
				\end{array}$ \\
				Medium demand, unfavorable cost &
				$\begin{array}{l}
					460\cdot 26\\
					= 11960
				\end{array}$ &
				$\begin{array}{l}
					360\cdot 26\\
					= 9360
				\end{array}$ &
				$\begin{array}{l}
					520\cdot 26\\
					= 13520
				\end{array}$ \\
				Low demand, favorable cost &
				$\begin{array}{l}
					300\cdot 18\\
					= 5400
				\end{array}$ &
				$\begin{array}{l}
					240\cdot 18\\
					= 4320
				\end{array}$ &
				$\begin{array}{l}
					360\cdot 18\\
					= 6480
				\end{array}$ \\
				Low demand, unfavorable cost &
				$\begin{array}{l}
					300\cdot 26\\
					= 7800
				\end{array}$ &
				$\begin{array}{l}
					240\cdot 26\\
					= 6240
				\end{array}$ &
				$\begin{array}{l}
					360\cdot 26\\
					= 9360
				\end{array}$ \\
				\bottomrule
			\end{tabular}
		\end{center}
		Profit is computed as Revenue minus Cost for each alternative and state.
		\begin{center}
			\textit{Profit parameters by model} \\
			\begin{tabular}{l p{0.19\linewidth} p{0.19\linewidth} p{0.19\linewidth} p{0.19\linewidth}}
				\toprule
				State of nature & Model A & Model B & Model C & Probabilities \\
				\midrule
				High demand, favorable cost &
				$\begin{array}{l}
					21700 - 11160\\
					= 10540
				\end{array}$ &
				$\begin{array}{l}
					23040 - 8640\\
					= 14400
				\end{array}$ &
				$\begin{array}{l}
					21000 - 12600\\
					= 8400
				\end{array}$ &
				$\begin{array}{l}
					0.25 \cdot 0.60\\
					= 0.15
				\end{array}$ \\
				High demand, unfavorable cost &
				$\begin{array}{l}
					21700 - 16120\\
					= 5580
				\end{array}$ &
				$\begin{array}{l}
					23040 - 12480\\
					= 10560
				\end{array}$ &
				$\begin{array}{l}
					21000 - 18200\\
					= 2800
				\end{array}$ &
				$\begin{array}{l}
					0.25 \cdot 0.40\\
					= 0.10
				\end{array}$ \\
				Medium demand, favorable cost &
				$\begin{array}{l}
					16100 - 8280\\
					= 7820
				\end{array}$ &
				$\begin{array}{l}
					17280 - 6480\\
					= 10800
				\end{array}$ &
				$\begin{array}{l}
					15600 - 9360\\
					= 6240
				\end{array}$ &
				$\begin{array}{l}
					0.50 \cdot 0.60\\
					= 0.30
				\end{array}$ \\
				Medium demand, unfavorable cost &
				$\begin{array}{l}
					16100 - 11960\\
					= 4140
				\end{array}$ &
				$\begin{array}{l}
					17280 - 9360\\
					= 7920
				\end{array}$ &
				$\begin{array}{l}
					15600 - 13520\\
					= 2080
				\end{array}$ &
				$\begin{array}{l}
					0.50 \cdot 0.40\\
					= 0.20
				\end{array}$ \\
				Low demand, favorable cost &
				$\begin{array}{l}
					10500 - 5400\\
					= 5100
				\end{array}$ &
				$\begin{array}{l}
					11520 - 4320\\
					= 7200
				\end{array}$ &
				$\begin{array}{l}
					10800 - 6480\\
					= 4320
				\end{array}$ &
				$\begin{array}{l}
					0.25 \cdot 0.60\\
					= 0.15
				\end{array}$ \\
				Low demand, unfavorable cost &
				$\begin{array}{l}
					10500 - 7800\\
					= 2700
				\end{array}$ &
				$\begin{array}{l}
					11520 - 6240\\
					= 5280
				\end{array}$ &
				$\begin{array}{l}
					10800 - 9360\\
					= 1440
				\end{array}$ &
				$\begin{array}{l}
					0.25 \cdot 0.40\\
					= 0.10
				\end{array}$ \\
				\bottomrule
			\end{tabular}
		\end{center}
		Combined probabilities are $0.25\cdot 0.60=0.15$, $0.25\cdot 0.40=0.10$, $0.50\cdot 0.60=0.30$, $0.50\cdot 0.40=0.20$, $0.25\cdot 0.60=0.15$, $0.25\cdot 0.40=0.10$.
		\begin{center}
			\begin{tabular}{l c c c c}
				\toprule
				State of nature & Probability & Model A & Model B & Model C \\
				\midrule
				High demand, favorable cost & 0.15 & 10540 & 14400 & 8400 \\
				High demand, unfavorable cost & 0.10 & 5580 & 10560 & 2800 \\
				Medium demand, favorable cost & 0.30 & 7820 & 10800 & 6240 \\
				Medium demand, unfavorable cost & 0.20 & 4140 & 7920 & 2080 \\
				Low demand, favorable cost & 0.15 & 5100 & 7200 & 4320 \\
				Low demand, unfavorable cost & 0.10 & 2700 & 5280 & 1440 \\
				\bottomrule
			\end{tabular}
		\end{center}

		\subsection*{C4}
		\[
		\begin{aligned}
		\max(\text{Model A}) &= \max\{10540,5580,7820,4140,5100,2700\} = 10540,\\
		\max(\text{Model B}) &= \max\{14400,10560,10800,7920,7200,5280\} = 14400,\\
		\max(\text{Model C}) &= \max\{8400,2800,6240,2080,4320,1440\} = 8400.
		\end{aligned}
		\]
		\[
		\max\{\max(\text{Model A}), \max(\text{Model B}), \max(\text{Model C})\}
		= \max\{10540, 14400, 8400\}
		= 14400.
		\]
		The Maximax choice is Model B with 14400. The criterion is \emph{risky} because it focuses only on the best possible payoff and ignores how likely it is.

		\subsection*{C5}
		\[
		\begin{aligned}
		\min(\text{Model A}) &= \min\{10540,5580,7820,4140,5100,2700\} = 2700,\\
		\min(\text{Model B}) &= \min\{14400,10560,10800,7920,7200,5280\} = 5280,\\
		\min(\text{Model C}) &= \min\{8400,2800,6240,2080,4320,1440\} = 1440.
		\end{aligned}
		\]
		\[
		\max\{\min(\text{Model A}), \min(\text{Model B}), \min(\text{Model C})\}
		= \max\{2700, 5280, 1440\}
		= 5280.
		\]
		The Maximin choice is Model B with 5280 because it has the least severe worst-case loss. This is \emph{safe} because it protects against the worst outcome.

		\subsection*{C1}
		Expected values use the combined-state probabilities.
		\[
		\begin{aligned}
		EV_A &= 0.15(10540)+0.10(5580)+0.30(7820)+0.20(4140)+0.15(5100)+0.10(2700)\\
		&= 1581+558+2346+828+765+270=6348,\\
		EV_B &= 0.15(14400)+0.10(10560)+0.30(10800)+0.20(7920)+0.15(7200)+0.10(5280)\\
		&= 2160+1056+3240+1584+1080+528=9648,\\
		EV_C &= 0.15(8400)+0.10(2800)+0.30(6240)+0.20(2080)+0.15(4320)+0.10(1440)\\
		&= 1260+280+1872+416+648+144=4620.
		\end{aligned}
		\]
		The expected value criterion chooses Model B because 9648 is the largest expected profit. This aligns with both the Maximax and Maximin outcomes for Model B, so it is the best overall decision under uncertainty.
	\end{ExamProblems}
\end{document}
