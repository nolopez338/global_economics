\makeatletter
\def\input@path{{./}{../}{../../}{preamble/}{../preamble/}{../../preamble/}}
\makeatother
% ----------------------------------------------------------
% GENERAL 

% File
\documentclass[11pt]{book}

% Margins
\usepackage[margin=1in]{geometry}

\linespread{1.2}            % Line spacing
\usepackage[utf8]{inputenc}
\usepackage[T1]{fontenc}
\usepackage{lmodern}
\usepackage{microtype}
\setlength{\parindent}{0pt}
\setlength{\parskip}{6pt}
\usepackage{booktabs}

% ----------------------------------------------------------
% TABLES
\usepackage{multicol}
\usepackage{longtable} 
\usepackage{array}
\usepackage{booktabs}
\usepackage{tabularx}
\usepackage{multirow}

% ----------------------------------------------------------
% MATHEMATICS
\usepackage{amsmath}
\usepackage{amssymb}
\usepackage{amsfonts}
\usepackage{mathtools}

% ----------------------------------------------------------
% HIDDEN CONTENT
\usepackage{ifthen}
% Define a boolean switch
\newboolean{explicaciones}
% Set the boolean switch to true or false
% Change to true to show the content

% Explanations
\newcommand{\explicacion}[2]{
	\ifthenelse{\boolean{explicaciones}}{#1}{#2}
}
\newcommand{\mostrarExplicaciones}[1]{\setboolean{explicaciones}{#1}}

% ----------------------------------------------------------
% NUMBERING

\usepackage{fancyhdr}
\pagestyle{empty} % Ensures the entire document has no page numbers

\usepackage{tocloft}
\renewcommand{\cftdot}{} % Remove dots for sections, if any
\renewcommand{\cftsecleader}{\cftdotfill{\cftdotsep}} % Remove dots for sections, if any
\cftpagenumbersoff{section} % Removes page numbers from sections
\cftpagenumbersoff{subsection} % Removes page numbers from subsections

% ----------------------------------------------------------
% IMAGES 

% General settings
\usepackage{graphicx}       % Insert images
\usepackage{float}          % Position images
% \usepackage{subfigure}      % Subfigures
\graphicspath{{imgs}}       % Image location
\usepackage{subcaption}     % Subfigures II
\usepackage{verbatim}

% Figures
\usepackage{tikz}
\usetikzlibrary{arrows.meta,positioning,trees}

% Colors
\usepackage{xcolor}     
\definecolor{popUp}{HTML}{666666}
\definecolor{popUpIn}{HTML}{CED9E0}
\definecolor{backgroundC}{HTML}{D0E8F2}
\definecolor{backgroundCC}{HTML}{FFFFFF}
\definecolor{borders}{HTML}{8c120d}
\definecolor{padding}{HTML}{77D0D7}
\definecolor{links}{HTML}{CC6F5F}

% ----------------------------------------------------------
% FRAMES

% General settings
\usepackage{tcolorbox}
\usepackage{adjustbox}          % Adjusted frame  
\setlength{\fboxrule}{3pt}  % Line width
\setlength{\fboxsep}{3pt}   % Box padding

% General frames
\usepackage{mdframed}   

\mdfdefinestyle{estiloGeneral}{    % General style
	linecolor=black,
	linewidth=1.5pt,
	roundcorner=10pt,
	backgroundcolor=backgroundC,
	innerbottommargin=0pt
}
\mdfdefinestyle{code}{          % Code style
	linecolor=black,
	linewidth=1.5pt,
	roundcorner=10pt,
	backgroundcolor=darkgray!10,
	innerbottommargin=0pt
}

% Image frame
\newtcbox{\fboxC}{
	colback=backgroundC,
	colframe=popUp,
	arc=10pt,
	boxrule=3pt,
	boxsep=0pt, % Change the padding here
	nobeforeafter
}

% ----------------------------------------------------------
% PAGE SETTINGS

% Background 
\newcommand{\background}[0]{\begin{tikzpicture}[remember picture,overlay]
		\fill[backgroundC] (-2,2) rectangle (25cm, -550);
\end{tikzpicture}}

\newcommand{\backgroundC}[0]{\begin{tikzpicture}[remember picture,overlay]
		\fill[backgroundCC] (-2,2) rectangle (25cm, -550);
\end{tikzpicture}}

% Page width 
\newcommand{\anchoPag}[0]{20cm}

% ----------------------------------------------------------
% FONT

% General
\usepackage{tgbonum}        % Font
\usepackage{listings}       % Code typesetting
\usepackage[spanish]{babel} % Load Spanish
\selectlanguage{spanish}    % Select Spanish
\usepackage{enumitem}
\usepackage{bookmark}

\setlist[itemize]{leftmargin=1.2em, itemsep=0.35em, topsep=0.35em}

% --- Table helpers ---
\newcolumntype{L}[1]{>{\raggedright\arraybackslash}p{#1}}
\newcolumntype{Y}{>{\raggedright\arraybackslash}X}
\newcolumntype{C}{>{\centering\arraybackslash}X}
\renewcommand{\arraystretch}{1.1}

% Python style
\lstdefinestyle{python}{
	language=Python,
	basicstyle=\ttfamily\small,
	commentstyle=\color{green!50!black},
	keywordstyle=\color{blue},
	numberstyle=\tiny\color{gray},
	numbers=left,
	morekeywords={>, <},
	breakatwhitespace=false,
	showstringspaces=false,
	showtabs=false,
	showspaces=false
}

% ----------------------------------------------------------
% HYPERLINKS

% General
\usepackage{hyperref}       
\hypersetup{
	colorlinks=true,
	linkcolor=links,
	filecolor=magenta,      
	urlcolor=brown,
}

% Custom commands 

% Large link
\newcommand{\bigLink}[2]{\begin{center} \fboxC{\LARGE{\href{#1}{#2}}}\end{center}}

% Small link
\newcommand{\smallLink}[2]{\begin{center}\fboxC{\href{#1}{#2}}\end{center}}

% Bold link
\newcommand{\bfLink}[2]{\href{#1}{\textbf{#2}}}


% Small URL
\newcommand{\smallUrl}[1]{\begin{center}\fboxC{\url{#1}}\end{center}}


% ----------------------------------------------------------
% CUSTOM COMMANDS FOR FIGURES

\newcommand{\espacioImagenes}[0]{-1.2cm}

% Without frame
\newcommand{\fig}[3][\espacioImagenes]{
	\hspace*{#1}
	\centering
	\includegraphics[width=#2\textwidth]{#3}
}

% With frame
\newcommand{\ffig}[2]{\begin{figure}[h]
		\hspace*{\espacioImagenes}
		\centering
		\fbox{\includegraphics[width=#1\textwidth]{#2}}
\end{figure}}

% Hyperlink with frame
\newcommand{\hfig}[3]{\begin{figure}[h]
		\hspace*{-1.4cm}
		\centering
		\color{popUp}
		\fboxC{\href{#1}{\includegraphics[width=#2\textwidth]{#3}}}
	\end{figure}
}

% Hyperlink without frame
\newcommand{\hffig}[3]{\begin{figure}[h]
		\hspace*{-1.1cm}
		\centering
		\color{popUp}
		\href{#1}{\includegraphics[width=#2\textwidth]{#3}}
	\end{figure}
}

% ----------------------------------------------------------

% Start and Contents
\newcommand{\cuadro}[1]{
	\begin{mdframed}[style=estiloGeneral]
		#1 
	\end{mdframed}
}

% Explanation video image
\newcommand{\linkExplicacion}[1]{
	\hffig{#1}{0.5}{principal/videoExplicacion}
	\vspace{-0.5cm}
}

\newcommand{\subSecLink}[2]{
	\subsubsection*{\href{#1}{\textbf{#2}}}
}

% Spacing
\newcommand{\esp}[0]{\vspace{4mm}}

% Back to start
\newcommand{\secInicio}[0]{\begin{center}\hyperref[sec:inicio]{ 
			\includegraphics[width=0.1\textwidth]{principal/up}
	}\end{center}
}


\geometry{margin=0.85in}
\AtBeginDocument{\small}

\newcommand{\ExamNameField}{\noindent\textbf{Name:}\ \rule{0.7\linewidth}{0.4pt}\par\medskip}

\newcommand{\ExamTitleBlock}[3]{%
	\begin{center}
		\Large\textbf{#1}\\
		\textbf{#2}%
		\if\relax\detokenize{#3}\relax\else\\\textbf{#3}\fi
	\end{center}
	\vspace{0.5em}
}

\newcommand{\ExamSection}[1]{\par\medskip\textbf{#1}\par\smallskip}

\newenvironment{ExamCriteria}{%
	\begin{itemize}[leftmargin=1.6em, itemsep=0.3em, topsep=0.2em]
}{%
	\end{itemize}
}

\newenvironment{ExamProblems}{%
	\begin{enumerate}[label=\textbf{P\arabic*}, leftmargin=0pt, labelsep=0.6em, itemindent=2.2em, itemsep=0.8em]
}{%
	\end{enumerate}
}

\begin{document}
	\ExamTitleBlock{10th grade}{Midterm Analysis of Decisions (Mock Solutions)}{}
	
	\ExamSection{Problems}
	\begin{ExamProblems}
		\item
		\subsection*{Problem description}
		A bike-sharing cooperative must choose a station layout for the next quarter: Layout A (dense stations),
		Layout B (hub stations), or Layout C (hybrid stations). Trip revenue depends on ridership demand,
		which can be high or low. The probability of high demand is $0.60$ and low demand is $0.40$.
		Maintenance costs depend on parts prices, which can be low, medium, or high with probabilities
		$0.50$, $0.30$, and $0.20$. Revenue is in thousands of dollars. If demand is high, revenue is 520 for Layout A,
		560 for Layout B, and 545 for Layout C. If demand is low, revenue is 300 for Layout A, 320 for Layout B,
		and 315 for Layout C. Maintenance costs are 190 (low), 230 (medium), and 275 (high) for Layout A;
		220 (low), 255 (medium), and 300 (high) for Layout B; and 205 (low), 245 (medium), and 285 (high) for Layout C.
		Construct the payoff table.
		
		\subsection*{C2}
		\begin{center}
			\begin{tabular}{l p{0.74\linewidth}}
				\toprule
				Decision Alternatives & Layout A (dense), Layout B (hub), Layout C (hybrid) \\
				States of Nature & Demand high, low with parts prices low, medium, or high \\
				Events & Realized demand level paired with parts price conditions in the quarter \\
				Consequences & Profit in thousands of dollars from revenue minus maintenance costs \\
				\bottomrule
			\end{tabular}
		\end{center}
		
		\subsection*{C3}
		Payoff = revenue $-$ cost, where revenue is determined by demand and cost is determined by parts prices.
		\begin{center}
			\begin{minipage}[t]{0.48\linewidth}
				\textit{Revenue parameters by layout}
				\begin{center}
					\begin{tabular}{l c c}
						\toprule
						Layout & High & Low \\
						\midrule
						Probabilities & 0.60 & 0.40 \\
						Layout A & 520 & 300 \\
						Layout B & 560 & 320 \\
						Layout C & 545 & 315 \\
						\bottomrule
					\end{tabular}
				\end{center}
			\end{minipage}
			\hfill
			\begin{minipage}[t]{0.48\linewidth}
				\textit{Cost parameters by layout}
				\begin{center}
					\begin{tabular}{l c c c}
						\toprule
						Layout & Low & Medium & High \\
						\midrule
						Probabilities & 0.50 & 0.30 & 0.20 \\
						Layout A & 190 & 230 & 275 \\
						Layout B & 220 & 255 & 300 \\
						Layout C & 205 & 245 & 285 \\
						\bottomrule
					\end{tabular}
				\end{center}
			\end{minipage}
		\end{center}
		Profit is computed as Revenue minus Cost for each alternative and state.
		
		\begin{center}
			\textit{Profit parameters by layout}
			\begin{tabular}{l p{0.12\linewidth} p{0.16\linewidth} p{0.16\linewidth} p{0.16\linewidth}}
				\toprule
				State of nature & Probability & Layout A & Layout B & Layout C \\
				\midrule
				High demand, low cost &
				$\begin{array}{l}
					0.60 \cdot 0.50\\
					= 0.30
				\end{array}$ &
				$\begin{array}{l}
					520 - 190\\
					= 330
				\end{array}$ &
				$\begin{array}{l}
					560 - 220\\
					= 340
				\end{array}$ &
				$\begin{array}{l}
					545 - 205\\
					= 340
				\end{array}$ \\
				High demand, medium cost &
				$\begin{array}{l}
					0.60 \cdot 0.30\\
					= 0.18
				\end{array}$ &
				$\begin{array}{l}
					520 - 230\\
					= 290
				\end{array}$ &
				$\begin{array}{l}
					560 - 255\\
					= 305
				\end{array}$ &
				$\begin{array}{l}
					545 - 245\\
					= 300
				\end{array}$ \\
				High demand, high cost &
				$\begin{array}{l}
					0.60 \cdot 0.20\\
					= 0.12
				\end{array}$ &
				$\begin{array}{l}
					520 - 275\\
					= 245
				\end{array}$ &
				$\begin{array}{l}
					560 - 300\\
					= 260
				\end{array}$ &
				$\begin{array}{l}
					545 - 285\\
					= 260
				\end{array}$ \\
				Low demand, low cost &
				$\begin{array}{l}
					0.40 \cdot 0.50\\
					= 0.20
				\end{array}$ &
				$\begin{array}{l}
					300 - 190\\
					= 110
				\end{array}$ &
				$\begin{array}{l}
					320 - 220\\
					= 100
				\end{array}$ &
				$\begin{array}{l}
					315 - 205\\
					= 110
				\end{array}$ \\
				Low demand, medium cost &
				$\begin{array}{l}
					0.40 \cdot 0.30\\
					= 0.12
				\end{array}$ &
				$\begin{array}{l}
					300 - 230\\
					= 70
				\end{array}$ &
				$\begin{array}{l}
					320 - 255\\
					= 65
				\end{array}$ &
				$\begin{array}{l}
					315 - 245\\
					= 70
				\end{array}$ \\
				Low demand, high cost &
				$\begin{array}{l}
					0.40 \cdot 0.20\\
					= 0.08
				\end{array}$ &
				$\begin{array}{l}
					300 - 275\\
					= 25
				\end{array}$ &
				$\begin{array}{l}
					320 - 300\\
					= 20
				\end{array}$ &
				$\begin{array}{l}
					315 - 285\\
					= 30
				\end{array}$ \\
				\bottomrule
			\end{tabular}
		\end{center}
		\begin{center}
			\textit{Payoff table} \\
			\begin{tabular}{l c c c c}
				\toprule
				State of nature & Probability & Layout A & Layout B & Layout C \\
				\midrule
				High demand, low cost & 0.30 & 330 & 340 & 340 \\
				High demand, medium cost & 0.18 & 290 & 305 & 300 \\
				High demand, high cost & 0.12 & 245 & 260 & 260 \\
				Low demand, low cost & 0.20 & 110 & 100 & 110 \\
				Low demand, medium cost & 0.12 & 70 & 65 & 70 \\
				Low demand, high cost & 0.08 & 25 & 20 & 30 \\
				\bottomrule
			\end{tabular}
		\end{center}
		
		\item
		\subsection*{Logistics Network Decision Under Demand and Cost Uncertainty}
		A logistics firm must choose one network model for the next year: Model A (in-house hub),
		Model B (partner network), or Model C (hybrid cross-dock). Demand uncertainty is high demand (0.55)
		or low demand (0.45), and delivery-cost uncertainty is low (0.50) or high (0.50).
		Under high demand, Model A completes 960 shipments, Model B completes 900 shipments, and Model C completes 940 shipments.
		Under low demand, Model A completes 580 shipments, Model B completes 620 shipments, and Model C completes 600 shipments.
		Model A earns \$46 per shipment, Model B earns \$43 in high demand and \$41 in low demand,
		and Model C earns \$44 per shipment. Cost exposure reflects annual operating costs:
		Model A has \$20{,}000 (low cost) and \$26{,}000 (high cost),
		Model B has \$17{,}200 (low cost) and \$22{,}400 (high cost), and
		Model C has \$18{,}600 (low cost) and \$24{,}000 (high cost).
		Construct the payoff table.
		
		\subsection*{C2}
		\begin{center}
			\begin{tabular}{l p{0.74\linewidth}}
				\toprule
				Decision Alternatives & Model A (in-house); Model B (partner); Model C (hybrid) \\
				States of Nature & Demand high, low with delivery costs low or high \\
				Events & Order volume paired with delivery cost conditions after the plan is chosen \\
				Consequences & Profit (revenue $-$ costs) for each model and state \\
				Probabilities & Demand: 0.55, 0.45; Cost: 0.50, 0.50 \\
				\bottomrule
			\end{tabular}
		\end{center}
		
		\subsection*{C3}
		Revenue by demand state (quantity $\times$ unit price = total revenue):
		\begin{center}
			\begin{tabular}{l c l l l}
				\toprule
				State & Probability & Alternative & Quantity & Unit price (USD) \\
				\midrule
				High demand & 0.55 & Model A & 960 shipments & \$46 \\
				High demand & 0.55 & Model B & 900 shipments & \$43 \\
				High demand & 0.55 & Model C & 940 shipments & \$44 \\
				Low demand & 0.45 & Model A & 580 shipments & \$46 \\
				Low demand & 0.45 & Model B & 620 shipments & \$41 \\
				Low demand & 0.45 & Model C & 600 shipments & \$44 \\
				\bottomrule
			\end{tabular}
		\end{center}
		\begin{center}
			\begin{tabular}{l c l c}
				\toprule
				State & Probability & Alternative & Total revenue (USD) \\
				\midrule
				High demand & 0.55 & Model A & $960 \times 46 = 44{,}160$ \\
				High demand & 0.55 & Model B & $900 \times 43 = 38{,}700$ \\
				High demand & 0.55 & Model C & $940 \times 44 = 41{,}360$ \\
				Low demand & 0.45 & Model A & $580 \times 46 = 26{,}680$ \\
				Low demand & 0.45 & Model B & $620 \times 41 = 25{,}420$ \\
				Low demand & 0.45 & Model C & $600 \times 44 = 26{,}400$ \\
				\bottomrule
			\end{tabular}
		\end{center}
		
		Cost by delivery-cost state (USD):
		\begin{center}
			\begin{tabular}{l c c c c}
				\toprule
				State & Probability & Model A & Model B & Model C \\
				\midrule
				Low cost & 0.50 & 20{,}000 & 17{,}200 & 18{,}600 \\
				High cost & 0.50 & 26{,}000 & 22{,}400 & 24{,}000 \\
				\bottomrule
			\end{tabular}
		\end{center}
		
		Profit table (profit = revenue $-$ costs):
		\begin{center}
			\begin{tabular}{l p{0.12\linewidth} p{0.16\linewidth} p{0.16\linewidth} p{0.16\linewidth}}
				\toprule
				State of nature & Probability & Model A & Model B & Model C \\
				\midrule
				High demand, low cost &
				$\begin{array}{l}
					0.55 \times 0.50\\
					= 0.275
				\end{array}$ &
				$\begin{array}{l}
					44{,}160 - 20{,}000\\
					= 24{,}160
				\end{array}$ &
				$\begin{array}{l}
					38{,}700 - 17{,}200\\
					= 21{,}500
				\end{array}$ &
				$\begin{array}{l}
					41{,}360 - 18{,}600\\
					= 22{,}760
				\end{array}$ \\
				High demand, high cost &
				$\begin{array}{l}
					0.55 \times 0.50\\
					= 0.275
				\end{array}$ &
				$\begin{array}{l}
					44{,}160 - 26{,}000\\
					= 18{,}160
				\end{array}$ &
				$\begin{array}{l}
					38{,}700 - 22{,}400\\
					= 16{,}300
				\end{array}$ &
				$\begin{array}{l}
					41{,}360 - 24{,}000\\
					= 17{,}360
				\end{array}$ \\
				Low demand, low cost &
				$\begin{array}{l}
					0.45 \times 0.50\\
					= 0.225
				\end{array}$ &
				$\begin{array}{l}
					26{,}680 - 20{,}000\\
					= 6{,}680
				\end{array}$ &
				$\begin{array}{l}
					25{,}420 - 17{,}200\\
					= 8{,}220
				\end{array}$ &
				$\begin{array}{l}
					26{,}400 - 18{,}600\\
					= 7{,}800
				\end{array}$ \\
				Low demand, high cost &
				$\begin{array}{l}
					0.45 \times 0.50\\
					= 0.225
				\end{array}$ &
				$\begin{array}{l}
					26{,}680 - 26{,}000\\
					= 680
				\end{array}$ &
				$\begin{array}{l}
					25{,}420 - 22{,}400\\
					= 3{,}020
				\end{array}$ &
				$\begin{array}{l}
					26{,}400 - 24{,}000\\
					= 2{,}400
				\end{array}$ \\
				\bottomrule
			\end{tabular}
		\end{center}
		
		Final payoff table (USD):
		\begin{center}
			\begin{tabular}{l c c c c}
				\toprule
				State of nature & Probability & Model A & Model B & Model C \\
				\midrule
				High demand, low cost & 0.275 & 24{,}160 & 21{,}500 & 22{,}760 \\
				High demand, high cost & 0.275 & 18{,}160 & 16{,}300 & 17{,}360 \\
				Low demand, low cost & 0.225 & 6{,}680 & 8{,}220 & 7{,}800 \\
				Low demand, high cost & 0.225 & 680 & 3{,}020 & 2{,}400 \\
				\bottomrule
			\end{tabular}
		\end{center}
		
		\item
		\subsection*{Holiday Inventory Plan Selection Under Expanded Demand Uncertainty}
		A retail chain must select one of five inventory plans for the holiday period: Plan A (aggressive stock),
		Plan B (balanced stock), Plan C (data-driven mix), Plan D (conservative stock), or Plan E (minimal stock).
		Demand can be strong, moderate, or weak with probabilities $0.30$, $0.45$, and $0.25$.
		The final payoff table (in thousands of dollars) is given below. Use Maximax, Maximin, Minimax Regret,
		and Expected Value to select a plan.
		
		\begin{center}
			\textit{Payoff table} \\
			\begin{tabular}{l c c c c c c}
				\toprule
				State of nature & Probability & Plan A & Plan B & Plan C & Plan D & Plan E \\
				\midrule
				Strong demand & 0.30 & 120 & 110 & 98 & 88 & 76 \\
				Moderate demand & 0.45 & 72 & 78 & 82 & 74 & 68 \\
				Weak demand & 0.25 & 20 & 34 & 46 & 52 & 58 \\
				\bottomrule
			\end{tabular}
		\end{center}
		
		\subsection*{C4}
		\[
		\begin{aligned}
		\max(\text{Plan A}) &= \max\{120,72,20\} = 120,\\
		\max(\text{Plan B}) &= \max\{110,78,34\} = 110,\\
		\max(\text{Plan C}) &= \max\{98,82,46\} = 98,\\
		\max(\text{Plan D}) &= \max\{88,74,52\} = 88,\\
		\max(\text{Plan E}) &= \max\{76,68,58\} = 76.
		\end{aligned}
		\]
		\[
		\max\{\max(\text{Plan A}), \max(\text{Plan B}), \max(\text{Plan C}), \max(\text{Plan D}), \max(\text{Plan E})\}
		= \max\{120, 110, 98, 88, 76\}
		= 120.
		\]
		The Maximax choice is Plan A because it has the highest best payoff.
		
		\subsection*{C5}
		\[
		\begin{aligned}
		\min(\text{Plan A}) &= \min\{120,72,20\} = 20,\\
		\min(\text{Plan B}) &= \min\{110,78,34\} = 34,\\
		\min(\text{Plan C}) &= \min\{98,82,46\} = 46,\\
		\min(\text{Plan D}) &= \min\{88,74,52\} = 52,\\
		\min(\text{Plan E}) &= \min\{76,68,58\} = 58.
		\end{aligned}
		\]
		\[
		\max\{\min(\text{Plan A}), \min(\text{Plan B}), \min(\text{Plan C}), \min(\text{Plan D}), \min(\text{Plan E})\}
		= \max\{20, 34, 46, 52, 58\}
		= 58.
		\]
		The Maximin choice is Plan E because it has the best worst-case payoff.
		
		\subsection*{C6}
		Best payoff in each state:
		\[
		\begin{aligned}
			\text{Strong demand: } &\max\{120, 110, 98, 88, 76\} = 120, \\
			\text{Moderate demand: } &\max\{72, 78, 82, 74, 68\} = 82, \\
			\text{Weak demand: } &\max\{20, 34, 46, 52, 58\} = 58.
		\end{aligned}
		\]
		Regret table (best payoff $-$ payoff):
		\begin{center}
			\begin{tabular}{l c c c c}
				\toprule
				Alternative & Strong $(0.30)$ & Moderate $(0.45)$ & Weak $(0.25)$ & Maximum regret \\
				\midrule
				Plan A & $120-120=0$ & $82-72=10$ & $58-20=38$ & 38 \\
				Plan B & $120-110=10$ & $82-78=4$ & $58-34=24$ & 24 \\
				Plan C & $120-98=22$ & $82-82=0$ & $58-46=12$ & 22 \\
				Plan D & $120-88=32$ & $82-74=8$ & $58-52=6$ & 32 \\
				Plan E & $120-76=44$ & $82-68=14$ & $58-58=0$ & 44 \\
				\bottomrule
			\end{tabular}
		\end{center}
		The minimax regret choice is Plan C because it has the smallest maximum regret $(22)$.
		
		\subsection*{C1}
		\[
		\begin{aligned}
		EV_A &= 0.30(120)+0.45(72)+0.25(20)=36+32.4+5=73.4,\\
		EV_B &= 0.30(110)+0.45(78)+0.25(34)=33+35.1+8.5=76.6,\\
		EV_C &= 0.30(98)+0.45(82)+0.25(46)=29.4+36.9+11.5=77.8,\\
		EV_D &= 0.30(88)+0.45(74)+0.25(52)=26.4+33.3+13=72.7,\\
		EV_E &= 0.30(76)+0.45(68)+0.25(58)=22.8+30.6+14.5=67.9.
		\end{aligned}
		\]
		The expected value criterion chooses Plan C because $77.8$ is the largest expected payoff.
		
		\item
		\subsection*{Routing System Choice Under Fuel and Congestion Uncertainty}
		A shipping company must choose between four routing systems for the next season: System A, System B,
		System C, or System D. Four states of nature summarize fuel and congestion conditions
		with probabilities $0.15$, $0.35$, $0.25$, and $0.25$. The final payoff table (in thousands of dollars)
		is given below. Use Maximax, Maximin, Minimax Regret, and Expected Value to select a system.
		
		\begin{center}
			\textit{Payoff table} \\
			\begin{tabular}{l c c c c c}
				\toprule
				State of nature & Probability & System A & System B & System C & System D \\
				\midrule
				State $S_1$ & 0.15 & 55 & 50 & 48 & 44 \\
				State $S_2$ & 0.35 & 34 & 36 & 32 & 30 \\
				State $S_3$ & 0.25 & 18 & 22 & 24 & 20 \\
				State $S_4$ & 0.25 & -10 & 4 & 8 & 12 \\
				\bottomrule
			\end{tabular}
		\end{center}
		
		\subsection*{C4}
		\[
		\begin{aligned}
		\max(\text{System A}) &= \max\{55,34,18,-10\} = 55,\\
		\max(\text{System B}) &= \max\{50,36,22,4\} = 50,\\
		\max(\text{System C}) &= \max\{48,32,24,8\} = 48,\\
		\max(\text{System D}) &= \max\{44,30,20,12\} = 44.
		\end{aligned}
		\]
		\[
		\max\{\max(\text{System A}), \max(\text{System B}), \max(\text{System C}), \max(\text{System D})\}
		= \max\{55, 50, 48, 44\}
		= 55.
		\]
		The Maximax choice is System A because it has the highest possible payoff.
		
		\subsection*{C5}
		\[
		\begin{aligned}
		\min(\text{System A}) &= \min\{55,34,18,-10\} = -10,\\
		\min(\text{System B}) &= \min\{50,36,22,4\} = 4,\\
		\min(\text{System C}) &= \min\{48,32,24,8\} = 8,\\
		\min(\text{System D}) &= \min\{44,30,20,12\} = 12.
		\end{aligned}
		\]
		\[
		\max\{\min(\text{System A}), \min(\text{System B}), \min(\text{System C}), \min(\text{System D})\}
		= \max\{-10, 4, 8, 12\}
		= 12.
		\]
		The Maximin choice is System D because it has the best worst-case payoff.
		
		\subsection*{C6}
		Best payoff in each state:
		\[
		\begin{aligned}
			\text{State } S_1: &\max\{55, 50, 48, 44\} = 55, \\
			\text{State } S_2: &\max\{34, 36, 32, 30\} = 36, \\
			\text{State } S_3: &\max\{18, 22, 24, 20\} = 24, \\
			\text{State } S_4: &\max\{-10, 4, 8, 12\} = 12.
		\end{aligned}
		\]
		Regret table (best payoff $-$ payoff):
		\begin{center}
			\begin{tabular}{l c c c c c}
				\toprule
				Alternative & $S_1$ $(0.15)$ & $S_2$ $(0.35)$ & $S_3$ $(0.25)$ & $S_4$ $(0.25)$ & Maximum regret \\
				\midrule
				System A & $55-55=0$ & $36-34=2$ & $24-18=6$ & $12-(-10)=22$ & 22 \\
				System B & $55-50=5$ & $36-36=0$ & $24-22=2$ & $12-4=8$ & 8 \\
				System C & $55-48=7$ & $36-32=4$ & $24-24=0$ & $12-8=4$ & 7 \\
				System D & $55-44=11$ & $36-30=6$ & $24-20=4$ & $12-12=0$ & 11 \\
				\bottomrule
			\end{tabular}
		\end{center}
		The minimax regret choice is System C because it has the smallest maximum regret $(7)$.
		
		\subsection*{C1}
		\[
		\begin{aligned}
		EV_A &= 0.15(55)+0.35(34)+0.25(18)+0.25(-10)=8.25+11.9+4.5-2.5=22.15,\\
		EV_B &= 0.15(50)+0.35(36)+0.25(22)+0.25(4)=7.5+12.6+5.5+1=26.6,\\
		EV_C &= 0.15(48)+0.35(32)+0.25(24)+0.25(8)=7.2+11.2+6+2=26.4,\\
		EV_D &= 0.15(44)+0.35(30)+0.25(20)+0.25(12)=6.6+10.5+5+3=25.1.
		\end{aligned}
		\]
		The expected value criterion chooses System B because $26.6$ is the largest expected payoff.
		
		\item
		\subsection*{Expected Value Comparison of Production Plans with Three States}
		A manufacturer must select one production plan, labeled A, B, or C, before knowing which market condition will occur.
		There are three possible states of nature: state $S_1$, which occurs with probability $\frac{p}{2}$,
		state $S_2$, which occurs with probability $\frac{p}{2}$, and state $S_3$, which occurs with probability $1-p$.
		Each production plan generates a different profit depending on the realized state of nature, as summarized in the payoff table below.
		The objective is to determine which production plan maximizes expected profit as a function of $p$.
		
		\begin{center}
			\textit{Payoff table} \\
			\begin{tabular}{l c c c}
				\toprule
				& $S_1$ $\left(\frac{p}{2}\right)$ & $S_2$ $\left(\frac{p}{2}\right)$ & $S_3$ $(1-p)$ \\
				\midrule
				A & 44 & 24 & 6 \\
				B & 30 & 30 & 14 \\
				C & 18 & 20 & 22 \\
				\bottomrule
			\end{tabular}
		\end{center}
		
		\subsection*{C7}
		\[
		EV(A)=44\frac{p}{2}+24\frac{p}{2}+6(1-p)=22p+12p+6-6p=28p+6
		\]
		\[
		EV(B)=30\frac{p}{2}+30\frac{p}{2}+14(1-p)=15p+15p+14-14p=16p+14
		\]
		\[
		EV(C)=18\frac{p}{2}+20\frac{p}{2}+22(1-p)=9p+10p+22-22p=22-3p
		\]
		
		Pairwise comparison conditions in $>0$ form:
		\[
		EV(A)-EV(B)>0 \Rightarrow (28p+6)-(16p+14)>0 \Rightarrow 12p-8>0
		\]
		\[
		EV(B)-EV(C)>0 \Rightarrow (16p+14)-(22-3p)>0 \Rightarrow 19p-8>0
		\]
		\[
		EV(A)-EV(C)>0 \Rightarrow (28p+6)-(22-3p)>0 \Rightarrow 31p-16>0
		\]
		Plan C is optimal when $p<\frac{8}{19}$, Plan B is optimal when $\frac{8}{19} \le p \le \frac{2}{3}$,
		and Plan A is optimal when $p>\frac{2}{3}$.
		
		\item
		\subsection*{Expected Value Comparison of Zoning Options with Three Alternatives}
		A city council must choose among three zoning options, labeled A, B, and C, before knowing which future condition will occur.
		There are two possible states of nature: state $S_1$, which occurs with probability $p$, and state $S_2$, which occurs with probability $1-p$.
		Each zoning option generates a different net return depending on the realized state of nature, as shown in the payoff table below.
		The objective is to determine which zoning option yields the higher expected return as a function of $p$.
		
		\begin{center}
			\textit{Payoff table} \\
			\begin{tabular}{l c c}
				\toprule
				& $S_1$ $(p)$ & $S_2$ $(1-p)$ \\
				\midrule
				A & 34 & 8 \\
				B & 24 & 14 \\
				C & 18 & 18 \\
				\bottomrule
			\end{tabular}
		\end{center}
		
		\subsection*{C7}
		\[
		EV(A)=34p+8(1-p)=26p+8
		\]
		\[
		EV(B)=24p+14(1-p)=10p+14
		\]
		\[
		EV(C)=18p+18(1-p)=18
		\]
		Pairwise comparison conditions in $>0$ form:
		\[
		EV(A)-EV(B)>0 \Rightarrow (26p+8)-(10p+14)>0 \Rightarrow 16p-6>0
		\]
		\[
		EV(A)-EV(C)>0 \Rightarrow (26p+8)-18>0 \Rightarrow 26p-10>0
		\]
		\[
		EV(B)-EV(C)>0 \Rightarrow (10p+14)-18>0 \Rightarrow 10p-4>0
		\]
		Option C is optimal when $p < \frac{5}{13}$, and option A is optimal when $p>\frac{5}{13}$.
		At $p=\frac{5}{13}$, options A and C are tied.
		Option B is never optimal because it is below option C for $p\le 0.4$ and below option A for $p>0.4$.
	\end{ExamProblems}
\end{document}
