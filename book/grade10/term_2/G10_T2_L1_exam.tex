\documentclass[12pt]{article}

% Page size and tighter margins
\usepackage[a4paper,left=1.2cm,right=1.2cm,top=1.5cm,bottom=1.5cm]{geometry}

% Core packages
\usepackage{graphicx}
\usepackage{xcolor}
\usepackage{array}
\usepackage{tabularx}
\usepackage{multicol}
\usepackage[T1]{fontenc}
\usepackage[utf8]{inputenc}

\setlength{\parindent}{0pt}
\setlength{\tabcolsep}{6pt}
\renewcommand{\arraystretch}{1.15}

% Column types
\newcolumntype{Y}{>{\raggedright\arraybackslash}m{\dimexpr0.30\textwidth-2\tabcolsep-2\arrayrulewidth\relax}}
\newcolumntype{Z}{>{\raggedright\arraybackslash}m{\dimexpr0.70\textwidth-2\tabcolsep-2\arrayrulewidth\relax}}
\newcolumntype{C}[1]{>{\centering\arraybackslash}m{#1}}

% Gray subsection header box
\newcommand{\SubsectionBox}[1]{%
	\noindent\colorbox{gray!30}{%
		\parbox{\linewidth}{\textbf{#1}}%
	}\par\vspace{0.35cm}%
}

% Centered multi-line cell helper
\newcommand{\CellCenter}[1]{%
	\parbox{\linewidth}{\centering #1}%
}

\begin{document}

	% =========================
	% HEADER BOX (3 COLUMNS)
	% =========================
	\noindent
	\begin{tabularx}{\textwidth}{|C{2.8cm}|C{\dimexpr\textwidth-6cm-4\tabcolsep-4\arrayrulewidth\relax}|C{2.8cm}|}
		\hline
		\centering
		\vspace{3mm}
		\includegraphics[width=2.5cm]{../../preamble/logo.png}
		&
		\CellCenter{%
			\vspace{-5mm}
			\textbf{GLOBAL ECONOMICS}\par
			\textbf{GRADE: 10TH}\par
			\textbf{LEARNING EVIDENCE 1}\par
			\textbf{ANALYSIS OF DECISIONS}\par
			\textbf{TEACHER'S NAME: Nicolás López Cuéllar}
		}
		&
		\CellCenter{%
			\textbf{SECOND TERM}\par
			\textbf{2025--2026}%
		}
		\\
		\hline
	\end{tabularx}

	\vspace{0.5cm}

	% =========================
	% OBJECTIVE + CRITERIA
	% =========================
	\noindent
	\begin{tabular}{|Y|Z|}
		\hline
		{\small
			\textbf{Learning objective:} Analyze decision-making scenarios by building payoff tables, identifying states of nature, and applying maximax, maximin, and expected value criteria.
		}
		&
		{\footnotesize
			\textbf{Assessment criteria:}\par
			C2: Interprets decision alternatives, events, consequences, and states of nature.\par
			C3: Builds a payoff table from a description of the problem.\par
			C4: Explains the maximax criterion for decision making without probabilities.\par
			C5: Summarizes the maximin criterion for decision making without probabilities. \par 
			C1: Describes the way a problem can be formulated for optimal decision making. Evaluates expected value. \par
		}
		\\
		\hline
	\end{tabular}

	\vspace{0.4cm}

	% =========================
	% STUDENT LINE
	% =========================
	\noindent
	\textbf{Student’s name:} \rule{7cm}{0.4pt}\hfill
	\textbf{Group:} \rule{2cm}{0.4pt}\hfill
	\textbf{Date:} \rule{3cm}{0.4pt}


	% =========================
	% EXAM BODY
	% =========================
	\begin{multicols}{2}
		\SubsectionBox{Criteria assessment}\vspace{-0.25cm}
		Each assessment criterion is evaluated across the three problems. A criterion is considered passed when it is correctly activated in at least two of the three problems.
		
		\vspace{0.25cm}
		\SubsectionBox{1. Cafeteria menu plans}\vspace{-0.25cm}
		A school cafeteria must choose one of three menu plans for a two-week festival, Plan A (local produce focus),
		Plan B (mixed menu), or Plan C (bulk staples). Demand for meals can be very high, high, medium, or low. Based on past festivals, the probability of very high demand is $0.15$, high demand is $0.30$, medium demand is $0.35$, and low demand is $0.20$. Profits are measured in hundreds of dollars. If demand is very high, Plan A yields 105, Plan B yields 85, and Plan C yields 70. If demand is high, Plan A yields 92, Plan B yields 70, and Plan C yields 55. If demand is medium, Plan A yields 50, Plan B yields 62, and Plan C yields 48. If demand is low, Plan A yields $-8$, Plan B yields 30, and Plan C yields 38. Which decision is the risky choice under the Maximax criterion, the safe choice under the Maximin criterion, and the balanced choice under the Expected Value criterion?

		\vspace{0.25cm}
		\SubsectionBox{2. Delivery cooperative fleet strategy}\vspace{-0.25cm}
		A delivery cooperative is choosing a fleet strategy for the next season, Strategy A (electric vans)
		or Strategy B (hybrid vans). Demand for deliveries can be strong or weak, and fuel costs can be low or high. The cooperative estimates the demand probabilities as $0.60$ (strong) and $0.40$ (weak). Fuel costs are independent of demand, with probabilities $0.55$ (low) and $0.45$ (high). Revenue, in thousands of dollars, depends on the demand level and strategy. Under strong demand, revenue is 420 for Strategy A and 460 for Strategy B. Under weak demand, revenue is 210 for Strategy A and 230 for Strategy B. Operating costs, in thousands of dollars, depend on the fuel cost level and strategy. If fuel costs are low, operating costs are 160 for Strategy A and 190 for Strategy B. If fuel costs are high, operating costs are 220 for Strategy A and 250 for Strategy B. Which decision is the risky choice under the Maximax criterion, the safe choice under the Maximin criterion, and the balanced choice under the Expected Value criterion?
		
		\vspace{0.25cm}
		\SubsectionBox{3. Community sports center model}\vspace{-0.25cm}
		A community sports center must choose a membership model for the next year, Model A (standard monthly pass)
		or Model B (flex pass). Demand for memberships can be high or low, and staffing costs can be favorable or unfavorable. Demand probabilities are $0.50$ (high) and $0.50$ (low). Staffing costs are independent of demand, with probabilities $0.60$ (favorable) and $0.40$ (unfavorable). Expected membership counts depend on demand and model. Under high demand the center expects 620 members with Model A and 700 with Model B. Under low demand it expects 300 with Model A and 360 with Model B. Membership fees are $35$ for Model A and $40$ for Model B per member per year. Variable staffing cost per member is $18$ when costs are favorable and $26$ when costs are unfavorable. All monetary amounts in this point can be treated in dollars. Which decision is the risky choice under the Maximax criterion, the safe choice under the Maximin criterion, and the balanced choice under the Expected Value criterion?

	\end{multicols}

\end{document}
