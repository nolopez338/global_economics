\makeatletter
\def\input@path{{./}{../}{../../}{preamble/}{../preamble/}{../../preamble/}}
\makeatother
% ----------------------------------------------------------
% GENERAL 

% File
\documentclass[11pt]{book}

% Margins
\usepackage[margin=1in]{geometry}

\linespread{1.2}            % Line spacing
\usepackage[utf8]{inputenc}
\usepackage[T1]{fontenc}
\usepackage{lmodern}
\usepackage{microtype}
\setlength{\parindent}{0pt}
\setlength{\parskip}{6pt}
\usepackage{booktabs}

% ----------------------------------------------------------
% TABLES
\usepackage{multicol}
\usepackage{longtable} 
\usepackage{array}
\usepackage{booktabs}
\usepackage{tabularx}
\usepackage{multirow}

% ----------------------------------------------------------
% MATHEMATICS
\usepackage{amsmath}
\usepackage{amssymb}
\usepackage{amsfonts}
\usepackage{mathtools}

% ----------------------------------------------------------
% HIDDEN CONTENT
\usepackage{ifthen}
% Define a boolean switch
\newboolean{explicaciones}
% Set the boolean switch to true or false
% Change to true to show the content

% Explanations
\newcommand{\explicacion}[2]{
	\ifthenelse{\boolean{explicaciones}}{#1}{#2}
}
\newcommand{\mostrarExplicaciones}[1]{\setboolean{explicaciones}{#1}}

% ----------------------------------------------------------
% NUMBERING

\usepackage{fancyhdr}
\pagestyle{empty} % Ensures the entire document has no page numbers

\usepackage{tocloft}
\renewcommand{\cftdot}{} % Remove dots for sections, if any
\renewcommand{\cftsecleader}{\cftdotfill{\cftdotsep}} % Remove dots for sections, if any
\cftpagenumbersoff{section} % Removes page numbers from sections
\cftpagenumbersoff{subsection} % Removes page numbers from subsections

% ----------------------------------------------------------
% IMAGES 

% General settings
\usepackage{graphicx}       % Insert images
\usepackage{float}          % Position images
% \usepackage{subfigure}      % Subfigures
\graphicspath{{imgs}}       % Image location
\usepackage{subcaption}     % Subfigures II
\usepackage{verbatim}

% Figures
\usepackage{tikz}
\usetikzlibrary{arrows.meta,positioning,trees}

% Colors
\usepackage{xcolor}     
\definecolor{popUp}{HTML}{666666}
\definecolor{popUpIn}{HTML}{CED9E0}
\definecolor{backgroundC}{HTML}{D0E8F2}
\definecolor{backgroundCC}{HTML}{FFFFFF}
\definecolor{borders}{HTML}{8c120d}
\definecolor{padding}{HTML}{77D0D7}
\definecolor{links}{HTML}{CC6F5F}

% ----------------------------------------------------------
% FRAMES

% General settings
\usepackage{tcolorbox}
\usepackage{adjustbox}          % Adjusted frame  
\setlength{\fboxrule}{3pt}  % Line width
\setlength{\fboxsep}{3pt}   % Box padding

% General frames
\usepackage{mdframed}   

\mdfdefinestyle{estiloGeneral}{    % General style
	linecolor=black,
	linewidth=1.5pt,
	roundcorner=10pt,
	backgroundcolor=backgroundC,
	innerbottommargin=0pt
}
\mdfdefinestyle{code}{          % Code style
	linecolor=black,
	linewidth=1.5pt,
	roundcorner=10pt,
	backgroundcolor=darkgray!10,
	innerbottommargin=0pt
}

% Image frame
\newtcbox{\fboxC}{
	colback=backgroundC,
	colframe=popUp,
	arc=10pt,
	boxrule=3pt,
	boxsep=0pt, % Change the padding here
	nobeforeafter
}

% ----------------------------------------------------------
% PAGE SETTINGS

% Background 
\newcommand{\background}[0]{\begin{tikzpicture}[remember picture,overlay]
		\fill[backgroundC] (-2,2) rectangle (25cm, -550);
\end{tikzpicture}}

\newcommand{\backgroundC}[0]{\begin{tikzpicture}[remember picture,overlay]
		\fill[backgroundCC] (-2,2) rectangle (25cm, -550);
\end{tikzpicture}}

% Page width 
\newcommand{\anchoPag}[0]{20cm}

% ----------------------------------------------------------
% FONT

% General
\usepackage{tgbonum}        % Font
\usepackage{listings}       % Code typesetting
\usepackage[spanish]{babel} % Load Spanish
\selectlanguage{spanish}    % Select Spanish
\usepackage{enumitem}
\usepackage{bookmark}

\setlist[itemize]{leftmargin=1.2em, itemsep=0.35em, topsep=0.35em}

% --- Table helpers ---
\newcolumntype{L}[1]{>{\raggedright\arraybackslash}p{#1}}
\newcolumntype{Y}{>{\raggedright\arraybackslash}X}
\newcolumntype{C}{>{\centering\arraybackslash}X}
\renewcommand{\arraystretch}{1.1}

% Python style
\lstdefinestyle{python}{
	language=Python,
	basicstyle=\ttfamily\small,
	commentstyle=\color{green!50!black},
	keywordstyle=\color{blue},
	numberstyle=\tiny\color{gray},
	numbers=left,
	morekeywords={>, <},
	breakatwhitespace=false,
	showstringspaces=false,
	showtabs=false,
	showspaces=false
}

% ----------------------------------------------------------
% HYPERLINKS

% General
\usepackage{hyperref}       
\hypersetup{
	colorlinks=true,
	linkcolor=links,
	filecolor=magenta,      
	urlcolor=brown,
}

% Custom commands 

% Large link
\newcommand{\bigLink}[2]{\begin{center} \fboxC{\LARGE{\href{#1}{#2}}}\end{center}}

% Small link
\newcommand{\smallLink}[2]{\begin{center}\fboxC{\href{#1}{#2}}\end{center}}

% Bold link
\newcommand{\bfLink}[2]{\href{#1}{\textbf{#2}}}


% Small URL
\newcommand{\smallUrl}[1]{\begin{center}\fboxC{\url{#1}}\end{center}}


% ----------------------------------------------------------
% CUSTOM COMMANDS FOR FIGURES

\newcommand{\espacioImagenes}[0]{-1.2cm}

% Without frame
\newcommand{\fig}[3][\espacioImagenes]{
	\hspace*{#1}
	\centering
	\includegraphics[width=#2\textwidth]{#3}
}

% With frame
\newcommand{\ffig}[2]{\begin{figure}[h]
		\hspace*{\espacioImagenes}
		\centering
		\fbox{\includegraphics[width=#1\textwidth]{#2}}
\end{figure}}

% Hyperlink with frame
\newcommand{\hfig}[3]{\begin{figure}[h]
		\hspace*{-1.4cm}
		\centering
		\color{popUp}
		\fboxC{\href{#1}{\includegraphics[width=#2\textwidth]{#3}}}
	\end{figure}
}

% Hyperlink without frame
\newcommand{\hffig}[3]{\begin{figure}[h]
		\hspace*{-1.1cm}
		\centering
		\color{popUp}
		\href{#1}{\includegraphics[width=#2\textwidth]{#3}}
	\end{figure}
}

% ----------------------------------------------------------

% Start and Contents
\newcommand{\cuadro}[1]{
	\begin{mdframed}[style=estiloGeneral]
		#1 
	\end{mdframed}
}

% Explanation video image
\newcommand{\linkExplicacion}[1]{
	\hffig{#1}{0.5}{principal/videoExplicacion}
	\vspace{-0.5cm}
}

\newcommand{\subSecLink}[2]{
	\subsubsection*{\href{#1}{\textbf{#2}}}
}

% Spacing
\newcommand{\esp}[0]{\vspace{4mm}}

% Back to start
\newcommand{\secInicio}[0]{\begin{center}\hyperref[sec:inicio]{ 
			\includegraphics[width=0.1\textwidth]{principal/up}
	}\end{center}
}


\geometry{margin=0.85in}
\AtBeginDocument{\small}

\newcommand{\ExamNameField}{\noindent\textbf{Name:}\ \rule{0.7\linewidth}{0.4pt}\par\medskip}

\newcommand{\ExamTitleBlock}[3]{%
	\begin{center}
		\Large\textbf{#1}\\
		\textbf{#2}%
		\if\relax\detokenize{#3}\relax\else\\\textbf{#3}\fi
	\end{center}
	\vspace{0.5em}
}

\newcommand{\ExamSection}[1]{\par\medskip\textbf{#1}\par\smallskip}

\newenvironment{ExamCriteria}{%
	\begin{itemize}[leftmargin=1.6em, itemsep=0.3em, topsep=0.2em]
}{%
	\end{itemize}
}

\newenvironment{ExamProblems}{%
	\begin{enumerate}[label=\textbf{P\arabic*}, leftmargin=0pt, labelsep=0.6em, itemindent=2.2em, itemsep=0.8em]
}{%
	\end{enumerate}
}

\begin{document}
	\ExamTitleBlock{10th grade}{Learning evidence 2.1: Analysis of Decisions (Practice Solutions)}{}
	\ExamSection{Evaluated criteria}
	This exam evaluates the following criteria. Each criterion is evaluated across the items below. A student passes a criterion if it is correctly applied in the solution.
	\begin{ExamCriteria}
		\item C2 Interprets decision alternatives, events, consequences, and states of nature.
		\item C3 Builds a payoff table from a description of the problem.
		\item C4 Explains the maximax criterion for decision making without probabilities.
		\item C5 Summarizes the maximin criterion for decision making without probabilities.
		\item C1 Describes how a problem can be formulated for optimal decision making, computes the expected value, and interprets the result.
	\end{ExamCriteria}
	\ExamSection{Problems}
	\begin{ExamProblems}
		\item
		\subsection*{Problem description}
		A community recreation center must choose a membership model for the next year: Model A (standard monthly pass),
		Model B (premium pass with classes), or Model C (flex pass). Demand for memberships can be high, medium, or low, and staffing costs can be favorable or unfavorable. The demand probabilities are $0.30$ (high), $0.45$ (medium), and $0.25$ (low). Staffing costs are independent of demand, with probabilities $0.55$ (favorable) and $0.45$ (unfavorable). Expected membership counts depend on demand and model: under high demand the center expects 640 members with Model A, 500 with Model B, and 720 with Model C; under medium demand it expects 480 with Model A, 380 with Model B, and 540 with Model C; under low demand it expects 320 with Model A, 250 with Model B, and 380 with Model C. Membership fees are $36$ for Model A, $50$ for Model B, and $32$ for Model C per member per year. Variable staffing cost per member is $19$ when costs are favorable and $27$ when costs are unfavorable. Fixed annual costs are $6{,}800$ for Model A, $9{,}200$ for Model B, and $5{,}400$ for Model C. All monetary amounts in this problem can be treated as dollars. Use the combined demand and staffing outcomes as states of nature and the stated probabilities to compute expected profits.

		\subsection*{C2}
		\begin{center}
			\begin{tabular}{l p{0.74\linewidth}}
				\toprule
				Decision Alternatives & Model A (standard), Model B (premium), Model C (flex) \\
				States of Nature & Demand high, medium, low with staffing costs favorable or unfavorable \\
				Events & Realized demand level paired with staffing cost conditions in the year \\
				Consequences & Annual profit in dollars from membership fees minus staffing and fixed costs \\
				\bottomrule
			\end{tabular}
		\end{center}

		\subsection*{C3}
		Payoff = revenue $-$ cost, where revenue = (members $\times$ fee) and cost = (members $\times$ variable cost) $+$ fixed cost.
		\textit{Members per model by state of nature}
		\begin{center}
			\begin{tabular}{l c c c}
				\toprule
				Model & High demand & Medium demand & Low demand \\
				\midrule
				Model A & 640 & 480 & 320 \\
				Model B & 500 & 380 & 250 \\
				Model C & 720 & 540 & 380 \\
				\bottomrule
		\end{tabular}
		\end{center}
		\begin{center}
			\begin{minipage}[t]{0.48\linewidth}
				\textit{Cost parameters by model}
				\begin{center}
					\begin{tabular}{l c c}
						\toprule
						Model & Favorable cost & Unfavorable cost \\
						\midrule
						Model A & 19 & 27 \\
						Model B & 19 & 27 \\
						Model C & 19 & 27 \\
						\bottomrule
					\end{tabular}
				\end{center}
			\end{minipage}
			\hfill
			\begin{minipage}[t]{0.48\linewidth}
				\textit{Revenue parameters by model}
				\begin{center}
					\begin{tabular}{l c c c}
						\toprule
						Model & High & Medium & Low \\
						\midrule
						Model A & 36 & 36 & 36 \\
						Model B & 50 & 50 & 50 \\
						Model C & 32 & 32 & 32 \\
						\bottomrule
					\end{tabular}
				\end{center}
			\end{minipage}
		\end{center}
		\begin{center}
			\begin{tabular}{l c c c}
				\toprule
				State of nature & Model A revenue & Model B revenue & Model C revenue \\
				\midrule
				High demand, favorable cost &
				$\begin{array}{l}
					640\cdot 36\\
					= 23040
				\end{array}$ &
				$\begin{array}{l}
					500\cdot 50\\
					= 25000
				\end{array}$ &
				$\begin{array}{l}
					720\cdot 32\\
					= 23040
				\end{array}$ \\
				High demand, unfavorable cost &
				$\begin{array}{l}
					640\cdot 36\\
					= 23040
				\end{array}$ &
				$\begin{array}{l}
					500\cdot 50\\
					= 25000
				\end{array}$ &
				$\begin{array}{l}
					720\cdot 32\\
					= 23040
				\end{array}$ \\
				Medium demand, favorable cost &
				$\begin{array}{l}
					480\cdot 36\\
					= 17280
				\end{array}$ &
				$\begin{array}{l}
					380\cdot 50\\
					= 19000
				\end{array}$ &
				$\begin{array}{l}
					540\cdot 32\\
					= 17280
				\end{array}$ \\
				Medium demand, unfavorable cost &
				$\begin{array}{l}
					480\cdot 36\\
					= 17280
				\end{array}$ &
				$\begin{array}{l}
					380\cdot 50\\
					= 19000
				\end{array}$ &
				$\begin{array}{l}
					540\cdot 32\\
					= 17280
				\end{array}$ \\
				Low demand, favorable cost &
				$\begin{array}{l}
					320\cdot 36\\
					= 11520
				\end{array}$ &
				$\begin{array}{l}
					250\cdot 50\\
					= 12500
				\end{array}$ &
				$\begin{array}{l}
					380\cdot 32\\
					= 12160
				\end{array}$ \\
				Low demand, unfavorable cost &
				$\begin{array}{l}
					320\cdot 36\\
					= 11520
				\end{array}$ &
				$\begin{array}{l}
					250\cdot 50\\
					= 12500
				\end{array}$ &
				$\begin{array}{l}
					380\cdot 32\\
					= 12160
				\end{array}$ \\
				\bottomrule
			\end{tabular}
		\end{center}
		\begin{center}
			\begin{tabular}{l c c c}
				\toprule
				State of nature & Model A cost & Model B cost & Model C cost \\
				\midrule
				High demand, favorable cost &
				$\begin{array}{l}
					640\cdot 19 + 6800\\
					= 18960
				\end{array}$ &
				$\begin{array}{l}
					500\cdot 19 + 9200\\
					= 18700
				\end{array}$ &
				$\begin{array}{l}
					720\cdot 19 + 5400\\
					= 19080
				\end{array}$ \\
				High demand, unfavorable cost &
				$\begin{array}{l}
					640\cdot 27 + 6800\\
					= 24080
				\end{array}$ &
				$\begin{array}{l}
					500\cdot 27 + 9200\\
					= 22700
				\end{array}$ &
				$\begin{array}{l}
					720\cdot 27 + 5400\\
					= 24840
				\end{array}$ \\
				Medium demand, favorable cost &
				$\begin{array}{l}
					480\cdot 19 + 6800\\
					= 15920
				\end{array}$ &
				$\begin{array}{l}
					380\cdot 19 + 9200\\
					= 16420
				\end{array}$ &
				$\begin{array}{l}
					540\cdot 19 + 5400\\
					= 15660
				\end{array}$ \\
				Medium demand, unfavorable cost &
				$\begin{array}{l}
					480\cdot 27 + 6800\\
					= 19760
				\end{array}$ &
				$\begin{array}{l}
					380\cdot 27 + 9200\\
					= 19460
				\end{array}$ &
				$\begin{array}{l}
					540\cdot 27 + 5400\\
					= 19980
				\end{array}$ \\
				Low demand, favorable cost &
				$\begin{array}{l}
					320\cdot 19 + 6800\\
					= 12880
				\end{array}$ &
				$\begin{array}{l}
					250\cdot 19 + 9200\\
					= 13950
				\end{array}$ &
				$\begin{array}{l}
					380\cdot 19 + 5400\\
					= 12620
				\end{array}$ \\
				Low demand, unfavorable cost &
				$\begin{array}{l}
					320\cdot 27 + 6800\\
					= 15440
				\end{array}$ &
				$\begin{array}{l}
					250\cdot 27 + 9200\\
					= 15950
				\end{array}$ &
				$\begin{array}{l}
					380\cdot 27 + 5400\\
					= 15660
				\end{array}$ \\
				\bottomrule
			\end{tabular}
		\end{center}
		Profit is computed as Revenue minus Cost for each alternative and state.
		\begin{center}
			\begin{tabular}{l p{0.23\linewidth} p{0.23\linewidth} p{0.23\linewidth}}
				\toprule
				State of nature & Model A & Model B & Model C \\
				\midrule
				High demand, favorable cost &
				$\begin{array}{l}
					23040 - 18960\\
					= 4080
				\end{array}$ &
				$\begin{array}{l}
					25000 - 18700\\
					= 6300
				\end{array}$ &
				$\begin{array}{l}
					23040 - 19080\\
					= 3960
				\end{array}$ \\
				High demand, unfavorable cost &
				$\begin{array}{l}
					23040 - 24080\\
					= -1040
				\end{array}$ &
				$\begin{array}{l}
					25000 - 22700\\
					= 2300
				\end{array}$ &
				$\begin{array}{l}
					23040 - 24840\\
					= -1800
				\end{array}$ \\
				Medium demand, favorable cost &
				$\begin{array}{l}
					17280 - 15920\\
					= 1360
				\end{array}$ &
				$\begin{array}{l}
					19000 - 16420\\
					= 2580
				\end{array}$ &
				$\begin{array}{l}
					17280 - 15660\\
					= 1620
				\end{array}$ \\
				Medium demand, unfavorable cost &
				$\begin{array}{l}
					17280 - 19760\\
					= -2480
				\end{array}$ &
				$\begin{array}{l}
					19000 - 19460\\
					= -460
				\end{array}$ &
				$\begin{array}{l}
					17280 - 19980\\
					= -2700
				\end{array}$ \\
				Low demand, favorable cost &
				$\begin{array}{l}
					11520 - 12880\\
					= -1360
				\end{array}$ &
				$\begin{array}{l}
					12500 - 13950\\
					= -1450
				\end{array}$ &
				$\begin{array}{l}
					12160 - 12620\\
					= -460
				\end{array}$ \\
				Low demand, unfavorable cost &
				$\begin{array}{l}
					11520 - 15440\\
					= -3920
				\end{array}$ &
				$\begin{array}{l}
					12500 - 15950\\
					= -3450
				\end{array}$ &
				$\begin{array}{l}
					12160 - 15660\\
					= -3500
				\end{array}$ \\
				\bottomrule
			\end{tabular}
		\end{center}
		Combined probabilities are $0.30\cdot 0.55=0.165$, $0.30\cdot 0.45=0.135$, $0.45\cdot 0.55=0.2475$, $0.45\cdot 0.45=0.2025$, $0.25\cdot 0.55=0.1375$, $0.25\cdot 0.45=0.1125$.
		\begin{center}
			\begin{tabular}{l c c c c}
				\toprule
				State of nature & Probability & Model A & Model B & Model C \\
				\midrule
				High demand, favorable cost & 0.165 & 4080 & 6300 & 3960 \\
				High demand, unfavorable cost & 0.135 & -1040 & 2300 & -1800 \\
				Medium demand, favorable cost & 0.2475 & 1360 & 2580 & 1620 \\
				Medium demand, unfavorable cost & 0.2025 & -2480 & -460 & -2700 \\
				Low demand, favorable cost & 0.1375 & -1360 & -1450 & -460 \\
				Low demand, unfavorable cost & 0.1125 & -3920 & -3450 & -3500 \\
				\bottomrule
			\end{tabular}
		\end{center}

		\subsection*{C4}
		\[
		\begin{aligned}
		\max(\text{Model A}) &= \max\{4080,-1040,1360,-2480,-1360,-3920\} = 4080,\\
		\max(\text{Model B}) &= \max\{6300,2300,2580,-460,-1450,-3450\} = 6300,\\
		\max(\text{Model C}) &= \max\{3960,-1800,1620,-2700,-460,-3500\} = 3960.
		\end{aligned}
		\]
		\[
		\max\{\max(\text{Model A}), \max(\text{Model B}), \max(\text{Model C})\}
		= \max\{4080, 6300, 3960\}
		= 6300.
		\]
		The Maximax choice is Model B with 6300. The criterion is \emph{risky} because it focuses only on the best possible payoff and ignores how likely it is.

		\subsection*{C5}
		\[
		\begin{aligned}
		\min(\text{Model A}) &= \min\{4080,-1040,1360,-2480,-1360,-3920\} = -3920,\\
		\min(\text{Model B}) &= \min\{6300,2300,2580,-460,-1450,-3450\} = -3450,\\
		\min(\text{Model C}) &= \min\{3960,-1800,1620,-2700,-460,-3500\} = -3500.
		\end{aligned}
		\]
		\[
		\max\{\min(\text{Model A}), \min(\text{Model B}), \min(\text{Model C})\}
		= \max\{-3920, -3450, -3500\}
		= -3450.
		\]
		The Maximin choice is Model B with $-3450$ because it has the least severe worst-case loss. This is \emph{safe} because it protects against the worst outcome.

		\subsection*{C1}
		Expected values use the combined-state probabilities.
		\[
		\begin{aligned}
		EV_A &= 0.165(4080)+0.135(-1040)+0.2475(1360)+0.2025(-2480)+0.1375(-1360)+0.1125(-3920)\\
		&= 673.2-140.4+336.6-502.2-187-441=-260.8,\\
		EV_B &= 0.165(6300)+0.135(2300)+0.2475(2580)+0.2025(-460)+0.1375(-1450)+0.1125(-3450)\\
		&= 1039.5+310.5+638.55-93.15-199.375-388.125=1307.9,\\
		EV_C &= 0.165(3960)+0.135(-1800)+0.2475(1620)+0.2025(-2700)+0.1375(-460)+0.1125(-3500)\\
		&= 653.4-243+400.95-546.75-63.25-393.75=-192.4.
		\end{aligned}
		\]
		The expected value criterion chooses Model B because $1307.9$ is the largest expected profit. This aligns with both the Maximax and Maximin outcomes for Model B, so it is the best overall decision under uncertainty.

		\item
		\subsection*{Problem description}
		A school club must choose one of three snack plans for a two-week fundraiser: Plan A (fresh-made items),
		Plan B (mixed menu), or Plan C (budget staples). Demand for snacks can be high, medium, or low. Based on past fundraisers, the probability of high demand is $0.30$, medium demand is $0.50$, and low demand is $0.20$. Profits are measured in hundreds of dollars. If demand is high, Plan A yields 88, Plan B yields 74, and Plan C yields 60. If demand is medium, Plan A yields 54, Plan B yields 66, and Plan C yields 50. If demand is low, Plan A yields $-6$, Plan B yields 28, and Plan C yields 36. The stated probabilities must be used for an expected value comparison as part of the decision analysis.

		\subsection*{C2}
		\begin{center}
			\begin{tabular}{l p{0.74\linewidth}}
				\toprule
				Decision Alternatives & Plan A (fresh-made items), Plan B (mixed menu), Plan C (budget staples) \\
				States of Nature & Demand high, medium, low with probabilities $0.30$, $0.50$, $0.20$ \\
				Events & Realized demand level in the serving period \\
				Consequences & Profit in hundreds of dollars after all costs \\
				\bottomrule
			\end{tabular}
		\end{center}

		\subsection*{C3}
		Payoff = revenue $-$ cost, and the problem already provides the resulting profits for each plan and demand level.
		\[
		\begin{aligned}
		\text{High demand: } &\Pi_A=88,\ \Pi_B=74,\ \Pi_C=60,\\
		\text{Medium demand: } &\Pi_A=54,\ \Pi_B=66,\ \Pi_C=50,\\
		\text{Low demand: } &\Pi_A=-6,\ \Pi_B=28,\ \Pi_C=36.
		\end{aligned}
		\]
		\begin{center}
			\begin{tabular}{l c c c c}
				\toprule
				State of nature & Probability & Plan A & Plan B & Plan C \\
				\midrule
				High demand & 0.30 & 88 & 74 & 60 \\
				Medium demand & 0.50 & 54 & 66 & 50 \\
				Low demand & 0.20 & -6 & 28 & 36 \\
				\bottomrule
			\end{tabular}
		\end{center}

		\subsection*{C4}
		\[
		\begin{aligned}
		\max(\text{Plan A}) &= \max\{88,54,-6\} = 88,\\
		\max(\text{Plan B}) &= \max\{74,66,28\} = 74,\\
		\max(\text{Plan C}) &= \max\{60,50,36\} = 60.
		\end{aligned}
		\]
		\[
		\max\{\max(\text{Plan A}), \max(\text{Plan B}), \max(\text{Plan C})\}
		= \max\{88, 74, 60\}
		= 88.
		\]
		The Maximax choice is Plan A. It is \emph{risky} because it only considers the highest possible profit.

		\subsection*{C5}
		\[
		\begin{aligned}
		\min(\text{Plan A}) &= \min\{88,54,-6\} = -6,\\
		\min(\text{Plan B}) &= \min\{74,66,28\} = 28,\\
		\min(\text{Plan C}) &= \min\{60,50,36\} = 36.
		\end{aligned}
		\]
		\[
		\max\{\min(\text{Plan A}), \min(\text{Plan B}), \min(\text{Plan C})\}
		= \max\{-6, 28, 36\}
		= 36.
		\]
		The Maximin choice is Plan C with 36 because it gives the best worst-case outcome. It is \emph{safe} because it prioritizes the least harmful result.

		\subsection*{C1}
		\[
		\begin{aligned}
		EV_A &= 0.30(88)+0.50(54)+0.20(-6)=26.4+27-1.2=52.2,\\
		EV_B &= 0.30(74)+0.50(66)+0.20(28)=22.2+33+5.6=60.8,\\
		EV_C &= 0.30(60)+0.50(50)+0.20(36)=18+25+7.2=50.2.
		\end{aligned}
		\]
		The expected value criterion chooses Plan B because $60.8$ is the largest expected profit. This differs from the Maximax choice (Plan A) and the Maximin choice (Plan C), so the expected value gives a balanced decision using the probabilities.
	\end{ExamProblems}
\end{document}
