\documentclass[12pt]{article}

% Page size and tighter margins
\usepackage[a4paper,left=1.2cm,right=1.2cm,top=1.5cm,bottom=1.5cm]{geometry}

% Core packages
\usepackage{graphicx}
\usepackage{xcolor}
\usepackage{array}
\usepackage{tabularx}
\usepackage{multicol}
\usepackage{amsmath}
\usepackage[T1]{fontenc}
\usepackage[utf8]{inputenc}

\setlength{\parindent}{0pt}
\setlength{\tabcolsep}{6pt}
\renewcommand{\arraystretch}{1.15}
\setlength{\emergencystretch}{3em}

% Column types
\newcolumntype{Y}{>{\raggedright\arraybackslash}m{\dimexpr0.30\textwidth-2\tabcolsep-2\arrayrulewidth\relax}}
\newcolumntype{Z}{>{\raggedright\arraybackslash}m{\dimexpr0.70\textwidth-2\tabcolsep-2\arrayrulewidth\relax}}
\newcolumntype{C}[1]{>{\centering\arraybackslash}m{#1}}

% Gray subsection header box
\newcommand{\SubsectionBox}[1]{%
	\noindent\colorbox{gray!30}{%
		\parbox{\linewidth}{\textbf{#1}}%
	}\par\vspace{0.35cm}%
}

% Centered multi-line cell helper
\newcommand{\CellCenter}[1]{%
	\parbox{\linewidth}{\centering #1}%
}

\begin{document}
	
	% =========================
	% HEADER BOX (3 COLUMNS)
	% =========================
	\noindent
	\begin{tabularx}{\textwidth}{|C{2.8cm}|C{\dimexpr\textwidth-6cm-4\tabcolsep-4\arrayrulewidth\relax}|C{2.8cm}|}
		\hline
		\centering
		\vspace{3mm}
		\includegraphics[width=2.5cm]{../../preamble/logo.png}
		&
		\CellCenter{%
			\vspace{-5mm}
			\textbf{GLOBAL ECONOMICS}\par
			\textbf{COMPUTATIONAL THINKING}\par
			\textbf{GRADE: 10TH}\par
			\textbf{TERM 2 -- PROJECT}\par
			\textbf{JAVASCRIPT DECISION ANALYSIS PROJECT}\par
			\textbf{Nicolás López Cuéllar, Jaime David Navarrete}
		}
		&
		\CellCenter{%
			\textbf{SECOND TERM}\par
			\textbf{2025--2026}%
		}
		\\
		\hline
	\end{tabularx}
	
	\vspace{0.5cm}
	
	% =========================
	% OBJECTIVE + CRITERIA
	% =========================
	\noindent
	\begin{tabular}{|Y|Z|}
		\hline
		{\small
			\textbf{Learning objective:} Present and defend a JavaScript tool that evaluates all possibilities in a decision-making problem using the strategies presented.
		}
		&
		{\footnotesize
			\textbf{Assessment criteria:}\par
			\textbf{Global Economics}\par
			C10: Evaluates all possibilities in a decision-making problem using the strategies presented.\par
			\medskip
			\textbf{Computational Thinking}\par
			C8: Designs interactive elements in web page for a purpose.\par
			C9: Creates functions in a web page for a purpose.\par
			C10: Evaluates interactivity in web pages based on requirements.\par
		}
		\\
		\hline
	\end{tabular}
	
	\vspace{0.4cm}
	
	% =========================
	% STUDENT LINE
	% =========================
	
	\begin{multicols}{2}
		
		\vspace{0.25cm}
		\SubsectionBox{1. Project brief}\vspace{-0.25cm}
		
		\textbf{1.1 Purpose of the project}\par
		This programming project requires students to design and implement a JavaScript tool that automates C8 decision-analysis calculations (Bayes/Expected Value, Maximin, and Minimax Regret).  
		The results produced by this tool must then be used to support a C9 decision-tree conclusion.  
		The presentation and defense of the implemented program constitute the evidence for Criterion C10.
		
		\vspace{0.25cm}
		\textbf{1.2 Input specification}\par
		The program must accept a single payoff table containing:
		\begin{itemize}
			\item States of nature with their corresponding probabilities.
			\item A set of decision alternatives.
			\item The payoff for each alternative–state pair.
		\end{itemize}
		
		Example input format (values in thousand USD):
		
		\begin{center}
			\begin{tabular}{l c c c c c c}
				\hline
				State & Probability & $A_1$ & $A_2$ & $A_3$ & $A_4$ & $A_5$ \\
				\hline
				$S_1$ & 0.40 & 140 & 130 & 155 & 148 & 138 \\
				$S_2$ & 0.35 & 95 & 105 & 90 & 110 & 100 \\
				$S_3$ & 0.25 & 60 & 80 & 45 & 85 & 72 \\
				\hline
			\end{tabular}
		\end{center}
		
		\vspace{0.25cm}
		\textbf{1.3 Required Outputs}\par
		For each alternative, the program must compute:
		
		\begin{itemize}
			\item Maximum payoff (for Maximax).
			\item Minimum payoff (for Maximin).
			\item Expected value using the given probabilities (Bayes criterion).
			\item Maximum regret (Minimax Regret criterion).
		\end{itemize}
		
		The results must be displayed in a structured summary table such as:
		\begin{center}
			\begin{tabular}{l c c c c}
				\hline
				Alt. & Max P. & Min P. & EV & Max Regret \\
				\hline
				$A_1$ & $\max P(A_1)$ & $\min P(A_1)$ & $EV(A_1)$ & $\max R(A_1)$ \\
				$A_2$ & $\max P(A_2)$ & $\min P(A_2)$ & $EV(A_2)$ & $\max R(A_2)$ \\
				$A_3$ & $\max P(A_3)$ & $\min P(A_3)$ & $EV(A_3)$ & $\max R(A_3)$ \\
				$A_4$ & $\max P(A_4)$ & $\min P(A_4)$ & $EV(A_4)$ & $\max R(A_4)$ \\
				$A_5$ & $\max P(A_5)$ & $\min P(A_5)$ & $EV(A_5)$ & $\max R(A_5)$ \\
				\hline
			\end{tabular}
		\end{center}
		
		\vspace{0.25cm}
		\textbf{1.4 Decision workflow (C10 → C8 → C9)}\par
		
		\begin{enumerate}
			\item \textbf{(C10)} Execute the JavaScript program to generate the structured summary table.
			\item \textbf{(C8)} Identify and justify recommendations using the computed results.
			\item \textbf{(C9)} Use the selected C8 recommendations as the basis for constructing the C9 decision tree.
		\end{enumerate}
		
		After presenting the program and its outputs, students must apply this workflow to the assigned project problems in order to activate Criteria C8 and C9.
		
		\newpage
		\SubsectionBox{2. Solved Example 1}
		
		\textbf{1. Program-Based Ranking and Structured Decision Refinement}\par
		A firm must choose one strategy among $A_1,A_2,A_3,A_4,A_5$ under uncertain market conditions.
		
		\begin{center}
			\begin{tabular}{l  c c c c c c}
				\hline
				State & Probability & $A_1$ & $A_2$ & $A_3$ & $A_4$ & $A_5$ \\
				\hline
				$H$ & 0.45 & 150 & 142 & 165 & 155 & 148 \\
				$M$ & 0.35 & 98 & 112 & 90 & 118 & 104 \\
				$L$ & 0.20 & 52 & 78 & 35 & 70 & 88 \\
				\hline
			\end{tabular}
		\end{center}
		
		After obtaining the C10 summary and computing the C8 results, apply the following structured C9 decision procedure:
		\begin{itemize}
			\item Step 1: Retain the top three alternatives according to Expected Value (EV).
			\item Step 2: Within that subset, eliminate the alternative with the highest Max regret.
			\item Step 3: From the remaining alternatives, select the one with the highest minimum payoff (Maximin).
		\end{itemize}
		
		\textbf{C10 summary table (program output)}\par
		\begin{center}
			\begin{tabular}{l c c c c}
				\hline
				Alt. & Max P. & Min P. & EV & Max regret \\
				\hline
				$A_1$ & 150 & 52 & 112.20 & 36 \\
				$A_2$ & 142 & 78 & 118.70 & 23 \\
				$A_3$ & 165 & 35 & 112.75 & 53 \\
				$A_4$ & 155 & 70 & 125.05 & 18 \\
				$A_5$ & 148 & 88 & 120.60 & 17 \\
				\hline
			\end{tabular}
		\end{center}
		
		\textbf{C8 solution (ranking and recommendations)}\par
		\begin{itemize}
			\item EV ranking: \par $A_4>A_5>A_2>A_3>A_1$.
			\item Maximax ranking:\par $A_3>A_4>A_1>A_5>A_2$.
			\item Maximin ranking: \par $A_5>A_2>A_4>A_1>A_3$.
			\item Minimax Regret ranking:\par $A_5<A_4<A_2<A_1<A_3$.
		\end{itemize}
		
		\textbf{C9 solution (apply filters)}\par
		\begin{itemize}
			\item \textbf{(Step 1)} EV top 3: $\{A_4,A_5,A_2\}$.
			\item \textbf{(Step 2)} Lowest two regrets: $\{A_4,A_5\}$.
			\item \textbf{(Step 3)} Maximin selects $A_5$.
			\item Final decision: $A_5$.
		\end{itemize}
		
		
		
		\SubsectionBox{3. Solved Example 2}
		
		\textbf{Multi-Criteria Filtering and Sequential Decision Selection}\par
		An insurance broker considers five portfolio mixes $Y_1$ to $Y_5$ under three climate-risk states. 
		Payoffs are in thousand USD.

		\begin{center}
			\begin{tabular}{l c c c c c c}
				\hline
				State & Prob. & $Y_1$ & $Y_2$ & $Y_3$ & $Y_4$ & $Y_5$ \\
				\hline
				$W_1$ & 0.40 & 136 & 126 & 146 & 116 & 130 \\
				$W_2$ & 0.35 & 94 & 98 & 84 & 103 & 90 \\
				$W_3$ & 0.25 & 66 & 74 & 58 & 88 & 69 \\
				\hline
			\end{tabular}
		\end{center}
		
		After obtaining the C10 summary and computing the C8 results, apply the following structured C9 decision procedure:
		\begin{itemize}
			\item Step 1: Retain the top three alternatives according to Expected Value (EV).
			\item Step 2: Within that subset, apply the Minimax Regret criterion and keep the top two performing options.
			\item Step 3: Then, apply the Maximin criterion to determine the final choice.
		\end{itemize}
		
		\textbf{C10 summary table (program output)}\par
		\begin{center}
			\begin{tabular}{l c c c c}
				\hline
				Alt. & Max P. & Min P. & EV & Max regret \\
				\hline
				$Y_1$ & 136 & 66 & 103.80 & 22 \\
				$Y_2$ & 126 & 74 & 103.20 & 20 \\
				$Y_3$ & 146 & 58 & 102.30 & 30 \\
				$Y_4$ & 116 & 88 & 104.45 & 30 \\
				$Y_5$ & 130 & 69 & 100.75 & 19 \\
				\hline
			\end{tabular}
		\end{center}
		
		\textbf{C8 solution (ranking and recommendations)}\par
		\begin{itemize}
			\item EV ranking: \par $Y_4>Y_1>Y_2>Y_3>Y_5$.
			\item Maximax ranking: \par $Y_3>Y_1>Y_5>Y_2>Y_4$.
			\item Maximin ranking: \par $Y_4>Y_2>Y_5>Y_1>Y_3$.
			\item Minimax Regret ranking: \par $Y_5<Y_2<Y_1<Y_3=Y_4$.
		\end{itemize}
		
		\textbf{C9 solution (apply filters)}\par
		\begin{itemize}
			\item \textbf{(Step 1)} EV top 3: $\{Y_4,Y_1,Y_2\}$.
			\item \textbf{(Step 2)} Minimax Regret keeps $\{Y_2,Y_1\}$.
			\item \textbf{(Step 3)} Maximin selects $Y_2$.
			\item Final decision: $Y_2$.
		\end{itemize}
		
		
		
	\end{multicols}
	
\end{document}
