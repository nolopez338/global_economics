\makeatletter
\def\input@path{{./}{../}{../../}{preamble/}{../preamble/}{../../preamble/}}
\makeatother
% ----------------------------------------------------------
% GENERAL 

% File
\documentclass[11pt]{book}

% Margins
\usepackage[margin=1in]{geometry}

\linespread{1.2}            % Line spacing
\usepackage[utf8]{inputenc}
\usepackage[T1]{fontenc}
\usepackage{lmodern}
\usepackage{microtype}
\setlength{\parindent}{0pt}
\setlength{\parskip}{6pt}
\usepackage{booktabs}

% ----------------------------------------------------------
% TABLES
\usepackage{multicol}
\usepackage{longtable} 
\usepackage{array}
\usepackage{booktabs}
\usepackage{tabularx}
\usepackage{multirow}

% ----------------------------------------------------------
% MATHEMATICS
\usepackage{amsmath}
\usepackage{amssymb}
\usepackage{amsfonts}
\usepackage{mathtools}

% ----------------------------------------------------------
% HIDDEN CONTENT
\usepackage{ifthen}
% Define a boolean switch
\newboolean{explicaciones}
% Set the boolean switch to true or false
% Change to true to show the content

% Explanations
\newcommand{\explicacion}[2]{
	\ifthenelse{\boolean{explicaciones}}{#1}{#2}
}
\newcommand{\mostrarExplicaciones}[1]{\setboolean{explicaciones}{#1}}

% ----------------------------------------------------------
% NUMBERING

\usepackage{fancyhdr}
\pagestyle{empty} % Ensures the entire document has no page numbers

\usepackage{tocloft}
\renewcommand{\cftdot}{} % Remove dots for sections, if any
\renewcommand{\cftsecleader}{\cftdotfill{\cftdotsep}} % Remove dots for sections, if any
\cftpagenumbersoff{section} % Removes page numbers from sections
\cftpagenumbersoff{subsection} % Removes page numbers from subsections

% ----------------------------------------------------------
% IMAGES 

% General settings
\usepackage{graphicx}       % Insert images
\usepackage{float}          % Position images
% \usepackage{subfigure}      % Subfigures
\graphicspath{{imgs}}       % Image location
\usepackage{subcaption}     % Subfigures II
\usepackage{verbatim}

% Figures
\usepackage{tikz}
\usetikzlibrary{arrows.meta,positioning,trees}

% Colors
\usepackage{xcolor}     
\definecolor{popUp}{HTML}{666666}
\definecolor{popUpIn}{HTML}{CED9E0}
\definecolor{backgroundC}{HTML}{D0E8F2}
\definecolor{backgroundCC}{HTML}{FFFFFF}
\definecolor{borders}{HTML}{8c120d}
\definecolor{padding}{HTML}{77D0D7}
\definecolor{links}{HTML}{CC6F5F}

% ----------------------------------------------------------
% FRAMES

% General settings
\usepackage{tcolorbox}
\usepackage{adjustbox}          % Adjusted frame  
\setlength{\fboxrule}{3pt}  % Line width
\setlength{\fboxsep}{3pt}   % Box padding

% General frames
\usepackage{mdframed}   

\mdfdefinestyle{estiloGeneral}{    % General style
	linecolor=black,
	linewidth=1.5pt,
	roundcorner=10pt,
	backgroundcolor=backgroundC,
	innerbottommargin=0pt
}
\mdfdefinestyle{code}{          % Code style
	linecolor=black,
	linewidth=1.5pt,
	roundcorner=10pt,
	backgroundcolor=darkgray!10,
	innerbottommargin=0pt
}

% Image frame
\newtcbox{\fboxC}{
	colback=backgroundC,
	colframe=popUp,
	arc=10pt,
	boxrule=3pt,
	boxsep=0pt, % Change the padding here
	nobeforeafter
}

% ----------------------------------------------------------
% PAGE SETTINGS

% Background 
\newcommand{\background}[0]{\begin{tikzpicture}[remember picture,overlay]
		\fill[backgroundC] (-2,2) rectangle (25cm, -550);
\end{tikzpicture}}

\newcommand{\backgroundC}[0]{\begin{tikzpicture}[remember picture,overlay]
		\fill[backgroundCC] (-2,2) rectangle (25cm, -550);
\end{tikzpicture}}

% Page width 
\newcommand{\anchoPag}[0]{20cm}

% ----------------------------------------------------------
% FONT

% General
\usepackage{tgbonum}        % Font
\usepackage{listings}       % Code typesetting
\usepackage[spanish]{babel} % Load Spanish
\selectlanguage{spanish}    % Select Spanish
\usepackage{enumitem}
\usepackage{bookmark}

\setlist[itemize]{leftmargin=1.2em, itemsep=0.35em, topsep=0.35em}

% --- Table helpers ---
\newcolumntype{L}[1]{>{\raggedright\arraybackslash}p{#1}}
\newcolumntype{Y}{>{\raggedright\arraybackslash}X}
\newcolumntype{C}{>{\centering\arraybackslash}X}
\renewcommand{\arraystretch}{1.1}

% Python style
\lstdefinestyle{python}{
	language=Python,
	basicstyle=\ttfamily\small,
	commentstyle=\color{green!50!black},
	keywordstyle=\color{blue},
	numberstyle=\tiny\color{gray},
	numbers=left,
	morekeywords={>, <},
	breakatwhitespace=false,
	showstringspaces=false,
	showtabs=false,
	showspaces=false
}

% ----------------------------------------------------------
% HYPERLINKS

% General
\usepackage{hyperref}       
\hypersetup{
	colorlinks=true,
	linkcolor=links,
	filecolor=magenta,      
	urlcolor=brown,
}

% Custom commands 

% Large link
\newcommand{\bigLink}[2]{\begin{center} \fboxC{\LARGE{\href{#1}{#2}}}\end{center}}

% Small link
\newcommand{\smallLink}[2]{\begin{center}\fboxC{\href{#1}{#2}}\end{center}}

% Bold link
\newcommand{\bfLink}[2]{\href{#1}{\textbf{#2}}}


% Small URL
\newcommand{\smallUrl}[1]{\begin{center}\fboxC{\url{#1}}\end{center}}


% ----------------------------------------------------------
% CUSTOM COMMANDS FOR FIGURES

\newcommand{\espacioImagenes}[0]{-1.2cm}

% Without frame
\newcommand{\fig}[3][\espacioImagenes]{
	\hspace*{#1}
	\centering
	\includegraphics[width=#2\textwidth]{#3}
}

% With frame
\newcommand{\ffig}[2]{\begin{figure}[h]
		\hspace*{\espacioImagenes}
		\centering
		\fbox{\includegraphics[width=#1\textwidth]{#2}}
\end{figure}}

% Hyperlink with frame
\newcommand{\hfig}[3]{\begin{figure}[h]
		\hspace*{-1.4cm}
		\centering
		\color{popUp}
		\fboxC{\href{#1}{\includegraphics[width=#2\textwidth]{#3}}}
	\end{figure}
}

% Hyperlink without frame
\newcommand{\hffig}[3]{\begin{figure}[h]
		\hspace*{-1.1cm}
		\centering
		\color{popUp}
		\href{#1}{\includegraphics[width=#2\textwidth]{#3}}
	\end{figure}
}

% ----------------------------------------------------------

% Start and Contents
\newcommand{\cuadro}[1]{
	\begin{mdframed}[style=estiloGeneral]
		#1 
	\end{mdframed}
}

% Explanation video image
\newcommand{\linkExplicacion}[1]{
	\hffig{#1}{0.5}{principal/videoExplicacion}
	\vspace{-0.5cm}
}

\newcommand{\subSecLink}[2]{
	\subsubsection*{\href{#1}{\textbf{#2}}}
}

% Spacing
\newcommand{\esp}[0]{\vspace{4mm}}

% Back to start
\newcommand{\secInicio}[0]{\begin{center}\hyperref[sec:inicio]{ 
			\includegraphics[width=0.1\textwidth]{principal/up}
	}\end{center}
}


\geometry{margin=0.85in}
\AtBeginDocument{\small}

\newcommand{\ExamNameField}{\noindent\textbf{Name:}\ \rule{0.7\linewidth}{0.4pt}\par\medskip}

\newcommand{\ExamTitleBlock}[3]{%
	\begin{center}
		\Large\textbf{#1}\\
		\textbf{#2}%
		\if\relax\detokenize{#3}\relax\else\\\textbf{#3}\fi
	\end{center}
	\vspace{0.5em}
}

\newcommand{\ExamSection}[1]{\par\medskip\textbf{#1}\par\smallskip}

\newenvironment{ExamCriteria}{%
	\begin{itemize}[leftmargin=1.6em, itemsep=0.3em, topsep=0.2em]
}{%
	\end{itemize}
}

\newenvironment{ExamProblems}{%
	\begin{enumerate}[label=\textbf{P\arabic*}, leftmargin=0pt, labelsep=0.6em, itemindent=2.2em, itemsep=0.8em]
}{%
	\end{enumerate}
}

\begin{document}
	\ExamTitleBlock{10th grade}{Term 2 Criterion 2 Catch-Up (Solutions)}{}
	
	\ExamSection{Problems}
	\begin{ExamProblems}
		\item
		\subsection*{Problem description}
		A bottled tea company must choose one of two monthly launch options for a new flavor line:
		Option A (premium glass packaging) or Option B (standard recyclable packaging).
		Actual demand for the month is uncertain and can be high or low.
		Estimated net profits (thousand USD) are already available:
		A gives 98 under high demand and 20 under low demand;
		B gives 80 under high demand and 37 under low demand.
		
		\subsection*{Solution}
		\textbf{Step 1 --- Identify the decision maker.} The decision maker is the bottled tea company's product manager.
		
		\textbf{Step 2 --- Identify the decision alternatives.} The controllable alternatives are Option A (premium glass) and Option B (standard recyclable).
		
		\textbf{Step 3 --- Identify the states of nature (uncertain events).} The uncertain state is market demand, which can be high or low. This demand level is not controlled by management.
		
		\textbf{Step 4 --- Interpret what each payoff represents.} Each payoff is the monthly profit consequence of choosing one packaging option and then observing one demand state. For example, 20 means: if Option A is chosen and low demand occurs, profit is 20 thousand USD.
		
		\textbf{Step 5 --- Present the payoff table.}
		\begin{center}
			\begin{tabular}{l c c}
				\toprule
				Alternative & High demand & Low demand \\
				\midrule
				Option A & 98 & 20 \\
				Option B & 80 & 37 \\
				\bottomrule
			\end{tabular}
		\end{center}
		
		In conclusion, the decision is packaging format, the uncertainty is demand, and each table entry is the economic consequence (profit) of one decision--state combination.
		
		\item
		\subsection*{Problem description}
		A regional home-goods wholesaler must choose one of three stocking policies for a holiday campaign:
		Policy A (premium assortment), Policy B (balanced assortment), or Policy C (value assortment).
		Demand can be strong, moderate, or weak.
		Net profits (thousand USD) are estimated as follows:
		A: 142, 90, 14; B: 124, 96, 40; C: 107, 78, 50.
		
		\subsection*{Solution}
		\textbf{Step 1 --- Identify the decision maker.} The decision maker is the wholesaler's operations director.
		
		\textbf{Step 2 --- Identify the decision alternatives.} The alternatives are Policy A, Policy B, and Policy C.
		
		\textbf{Step 3 --- Identify the states of nature (uncertain events).} The uncertain event is campaign demand, with three states: strong, moderate, and weak.
		
		\textbf{Step 4 --- Interpret what each payoff represents.} Every payoff is the profit outcome from one stocking policy under one realized demand state. For instance, 40 means Policy B under weak demand yields 40 thousand USD.
		
		\textbf{Step 5 --- Present the payoff table.}
		\begin{center}
			\begin{tabular}{l c c c}
				\toprule
				Alternative & Strong demand & Moderate demand & Weak demand \\
				\midrule
				Policy A & 142 & 90 & 14 \\
				Policy B & 124 & 96 & 40 \\
				Policy C & 107 & 78 & 50 \\
				\bottomrule
			\end{tabular}
		\end{center}
		
		Therefore, the table is interpreted as a map from controllable stocking decisions and uncontrollable demand events to profit consequences.
		
		\item
		\subsection*{Problem description}
		A sportswear producer must choose one of two monthly production mixes:
		Mix A (performance line) or Mix B (basic line).
		Demand may be very high, high, medium, or low.
		Profit equals units sold times contribution per unit.
		Contribution per unit is \$19 for Mix A and \$15 for Mix B.
		Expected units sold are:
		A: 8{,}800 (very high), 7{,}100 (high), 4{,}900 (medium), 2{,}500 (low);
		B: 10{,}000, 8{,}100, 5{,}900, 3{,}300 respectively.
		
		\subsection*{Solution}
		\textbf{Step 1 --- Identify the decision maker.} The decision maker is the sportswear producer's planning manager.
		
		\textbf{Step 2 --- Identify the decision alternatives.} The alternatives are Mix A and Mix B.
		
		\textbf{Step 3 --- Identify the states of nature (uncertain events).} The uncertain event is demand intensity with states: very high, high, medium, and low.
		
		\textbf{Step 4 --- Interpret what each payoff represents.} Each payoff is monthly profit after selecting one mix and observing a demand state. Briefly computing profits (thousand USD):
		\[
		\begin{aligned}
		A:&\;8{,}800\times19=167.2,\;7{,}100\times19=134.9,\;4{,}900\times19=93.1,\;2{,}500\times19=47.5,\\
		B:&\;10{,}000\times15=150,\;8{,}100\times15=121.5,\;5{,}900\times15=88.5,\;3{,}300\times15=49.5.
		\end{aligned}
		\]
		Thus, for example, 47.5 means Mix A chosen and low demand realized.
		
		\textbf{Step 5 --- Present the payoff table (thousand USD).}
		\begin{center}
			\begin{tabular}{l c c c c}
				\toprule
				Alternative & Very high & High & Medium & Low \\
				\midrule
				Mix A & 167.2 & 134.9 & 93.1 & 47.5 \\
				Mix B & 150 & 121.5 & 88.5 & 49.5 \\
				\bottomrule
			\end{tabular}
		\end{center}
		
		In conclusion, the economic meaning is clear: management controls the mix, demand is uncertain, and profits are consequences contingent on that uncertainty.
		
		\item
		\subsection*{Problem description}
		A digital news publisher must select one of three distribution plans for next month:
		Plan A (direct subscriptions), Plan B (platform partnership), or Plan C (hybrid channel).
		Demand can be high or low.
		Profit model: \(\text{Profit}=\text{Fixed component}+(\text{Subscribers}\times\text{Contribution per subscriber})\).
		Parameters:
		A: fixed \$25{,}000, contribution \$6.4;
		B: fixed \$16{,}000, contribution \$5.9;
		C: fixed \$20{,}000, contribution \$6.0.
		Subscribers by state:
		High --- A 17{,}500, B 20{,}800, C 19{,}200;
		Low --- A 9{,}200, B 11{,}700, C 10{,}500.
		
		\subsection*{Solution}
		\textbf{Step 1 --- Identify the decision maker.} The decision maker is the publisher's commercial director.
		
		\textbf{Step 2 --- Identify the decision alternatives.} The alternatives are Plan A, Plan B, and Plan C.
		
		\textbf{Step 3 --- Identify the states of nature (uncertain events).} The uncertain event is subscriber demand, high or low.
		
		\textbf{Step 4 --- Interpret what each payoff represents.} Each payoff is monthly profit generated when one plan is selected and one demand state occurs. Brief computations (thousand USD):
		\[
		\begin{aligned}
		A_H&=25+17{,}500\times6.4/1000=137, & A_L&=25+9{,}200\times6.4/1000=83.88,\\
		B_H&=16+20{,}800\times5.9/1000=138.72, & B_L&=16+11{,}700\times5.9/1000=85.03,\\
		C_H&=20+19{,}200\times6.0/1000=135.2, & C_L&=20+10{,}500\times6.0/1000=83.
		\end{aligned}
		\]
		
		\textbf{Step 5 --- Present the payoff table (thousand USD).}
		\begin{center}
			\begin{tabular}{l c c}
				\toprule
				Alternative & High demand & Low demand \\
				\midrule
				Plan A & 137 & 83.88 \\
				Plan B & 138.72 & 85.03 \\
				Plan C & 135.2 & 83 \\
				\bottomrule
			\end{tabular}
		\end{center}
		
		Hence, each cell should be interpreted as a consequence of a chosen distribution plan under an uncontrollable demand outcome.
		
		\item
		\subsection*{Problem description}
		A packaging supplier must choose between two contract structures for a seasonal client:
		Contract A or Contract B. Demand may be strong, normal, or weak.
		Contract A profits are given directly (thousand USD): 114, 76, 28.
		Contract B uses
		\(\text{Profit}_B=22{,}000+(\text{Units sold}\times4.6)\),
		with units sold 17{,}500 (strong), 11{,}800 (normal), 7{,}200 (weak).
		
		\subsection*{Solution}
		\textbf{Step 1 --- Identify the decision maker.} The decision maker is the supplier's contracts manager.
		
		\textbf{Step 2 --- Identify the decision alternatives.} Alternatives are Contract A and Contract B.
		
		\textbf{Step 3 --- Identify the states of nature (uncertain events).} The uncertain event is client demand intensity: strong, normal, or weak.
		
		\textbf{Step 4 --- Interpret what each payoff represents.} A payoff is the profit consequence from selecting a contract before demand is realized. For B (thousand USD):
		\[
		\begin{aligned}
		B_{\text{strong}}&=22+17{,}500\times4.6/1000=102.5,\\
		B_{\text{normal}}&=22+11{,}800\times4.6/1000=76.28,\\
		B_{\text{weak}}&=22+7{,}200\times4.6/1000=55.12.
		\end{aligned}
		\]
		So 55.12 means Contract B chosen and weak demand occurs.
		
		\textbf{Step 5 --- Present the payoff table (thousand USD).}
		\begin{center}
			\begin{tabular}{l c c c}
				\toprule
				Alternative & Strong & Normal & Weak \\
				\midrule
				Contract A & 114 & 76 & 28 \\
				Contract B & 102.5 & 76.28 & 55.12 \\
				\bottomrule
			\end{tabular}
		\end{center}
		
		In conclusion, the decision is contract selection, while demand is the external event determining which consequence is realized.
		
		\item
		\subsection*{Problem description}
		An urban courier company must select one of two fleet plans for a quarterly delivery contract:
		Fleet A or Fleet B.
		Revenue can be high or low; operating cost can be low, medium, or high.
		All values are given directly in thousand USD.
		Revenue: A = 430 (high), 295 (low); B = 458 (high), 305 (low).
		Costs: A = 175, 210, 246 for low/medium/high cost;
		B = 188, 224, 264 for low/medium/high cost.
		Profit equals revenue minus cost.
		
		\subsection*{Solution}
		\textbf{Step 1 --- Identify the decision maker.} The decision maker is the courier company's operations head.
		
		\textbf{Step 2 --- Identify the decision alternatives.} Alternatives are Fleet A and Fleet B.
		
		\textbf{Step 3 --- Identify the states of nature (uncertain events).} States are combined events: revenue state (high/low) with cost state (low/medium/high), giving six states of nature.
		
		\textbf{Step 4 --- Interpret what each payoff represents.} Each payoff is the quarterly profit after one fleet decision and one combined market-cost event. Example: for Fleet B in low revenue and high cost, payoff is \(305-264=41\).
		
		\textbf{Step 5 --- Present the payoff table (thousand USD).}
		\begin{center}
			\begin{tabular}{l c c}
				\toprule
				State of nature & Fleet A & Fleet B \\
				\midrule
				High revenue, low cost & 255 & 270 \\
				High revenue, medium cost & 220 & 234 \\
				High revenue, high cost & 184 & 194 \\
				Low revenue, low cost & 120 & 117 \\
				Low revenue, medium cost & 85 & 81 \\
				Low revenue, high cost & 49 & 41 \\
				\bottomrule
			\end{tabular}
		\end{center}
		
		Therefore, the table should be read as a consequence matrix linking each fleet choice to each uncontrollable revenue--cost scenario.
		
		\item
		\subsection*{Problem description}
		A corporate meal-service provider must choose one of three service plans: A, B, or C.
		Meal demand can be high or low, while input costs can be low or high.
		Revenue is computed from quantity and price; costs are given directly.
		Revenue inputs:
		A: 13{,}800 meals (high), 8{,}900 (low), price \$12.2;
		B: 15{,}200, 10{,}300, price \$11.6;
		C: 13{,}000, 8{,}600, price \$12.5.
		Cost data (thousand USD):
		A: 103 (low cost), 129 (high cost);
		B: 109, 136;
		C: 99, 125.
		Profit = revenue $-$ cost.
		
		\subsection*{Solution}
		\textbf{Step 1 --- Identify the decision maker.} The decision maker is the provider's account director.
		
		\textbf{Step 2 --- Identify the decision alternatives.} Alternatives are Plans A, B, and C.
		
		\textbf{Step 3 --- Identify the states of nature (uncertain events).} States combine demand (high/low) and cost (low/high), producing four uncertain events.
		
		\textbf{Step 4 --- Interpret what each payoff represents.} A payoff is profit consequence under one plan and one combined event. Revenues (thousand USD) are:
		\(R_A^H=168.36,\;R_A^L=108.58,\;R_B^H=176.32,\;R_B^L=119.48,\;R_C^H=162.5,\;R_C^L=107.5\).
		Then payoffs are revenue minus cost.
		
		\textbf{Step 5 --- Present the payoff table (thousand USD).}
		\begin{center}
			\begin{tabular}{l c c c}
				\toprule
				State of nature & Plan A & Plan B & Plan C \\
				\midrule
				High demand, low cost & 65.36 & 67.32 & 63.5 \\
				High demand, high cost & 39.36 & 40.32 & 37.5 \\
				Low demand, low cost & 5.58 & 10.48 & 8.5 \\
				Low demand, high cost & -20.42 & -16.52 & -17.5 \\
				\bottomrule
			\end{tabular}
		\end{center}
		
		In summary, management controls the plan, not demand or input prices, and each cell reports the corresponding financial consequence.
		
		\item
		\subsection*{Problem description}
		A vocational training operator must choose between Program A and Program B.
		Tuition revenue is given directly for enrollment-demand states high, medium, and low.
		Costs depend on instructor rates (low-cost or high-cost state) and must be computed from participant counts.
		Revenue (thousand USD):
		A: 365, 285, 212;
		B: 395, 305, 233.
		Cost model: \(\text{Cost}=\text{Participants}\times\text{Unit cost}\).
		A: 980 participants (low-cost state) and 1{,}280 (high-cost state), unit cost \$122.
		B: 1{,}040 and 1{,}340, unit cost \$126.
		Profit = revenue $-$ cost.
		
		\subsection*{Solution}
		\textbf{Step 1 --- Identify the decision maker.} The decision maker is the operator's academic director.
		
		\textbf{Step 2 --- Identify the decision alternatives.} Alternatives are Program A and Program B.
		
		\textbf{Step 3 --- Identify the states of nature (uncertain events).} States combine enrollment demand (high/medium/low) with instructor-cost condition (low/high), resulting in six uncertain states.
		
		\textbf{Step 4 --- Interpret what each payoff represents.} A payoff is net program profit after selecting a program and observing both market enrollment and instructor-rate conditions. Costs (thousand USD):
		\(C_A^L=119.56,\;C_A^H=156.16,\;C_B^L=131.04,\;C_B^H=168.84\).
		
		\textbf{Step 5 --- Present the payoff table (thousand USD).}
		\begin{center}
			\begin{tabular}{l c c}
				\toprule
				State of nature & Program A & Program B \\
				\midrule
				High revenue, low cost & 245.44 & 263.96 \\
				High revenue, high cost & 208.84 & 226.16 \\
				Medium revenue, low cost & 165.44 & 173.96 \\
				Medium revenue, high cost & 128.84 & 136.16 \\
				Low revenue, low cost & 92.44 & 101.96 \\
				Low revenue, high cost & 55.84 & 64.16 \\
				\bottomrule
			\end{tabular}
		\end{center}
		
		Hence, the table communicates consequences of each program choice under enrollment and instructor-cost uncertainty.
		
		\item
		\subsection*{Problem description}
		A smart-appliance firm must choose one of three release configurations:
		Configuration P, Q, or R.
		Both revenue and cost include fixed and variable components.
		Demand can be high, medium, or low; operating condition can be normal or disrupted.
		Revenue models (USD):
		\(R_P=148{,}000+52q\), \(R_Q=132{,}000+55q\), \(R_R=144{,}000+53q\).
		Demand quantities:
		P: 8{,}700, 6{,}900, 5{,}000;
		Q: 8{,}500, 6{,}800, 5{,}200;
		R: 8{,}800, 6{,}950, 5{,}100.
		Cost models (USD):
		\(C_P=121{,}000+27n\), \(C_Q=106{,}000+29n\), \(C_R=113{,}000+28n\).
		Cost activity quantities:
		P: 8{,}100 (normal), 9{,}000 (disrupted);
		Q: 7{,}900, 8{,}900;
		R: 8{,}000, 9{,}000.
		
		\subsection*{Solution}
		\textbf{Step 1 --- Identify the decision maker.} The decision maker is the firm's launch strategy team.
		
		\textbf{Step 2 --- Identify the decision alternatives.} Alternatives are Configurations P, Q, and R.
		
		\textbf{Step 3 --- Identify the states of nature (uncertain events).} Uncertain events are demand level (high/medium/low) and operating condition (normal/disrupted), giving six states.
		
		\textbf{Step 4 --- Interpret what each payoff represents.} Each payoff is the profit consequence of one release configuration after both demand and operating conditions are realized. Computed values (thousand USD):
		Revenues ---
		\(R_P^H=600.4,\;R_P^M=506.8,\;R_P^L=408\),
		\(R_Q^H=599.5,\;R_Q^M=506,\;R_Q^L=418\),
		\(R_R^H=610.4,\;R_R^M=512.35,\;R_R^L=414.3\).
		Costs ---
		\(C_P^N=339.7,\;C_P^D=364\),
		\(C_Q^N=335.1,\;C_Q^D=364.1\),
		\(C_R^N=337,\;C_R^D=365\).
		
		\textbf{Step 5 --- Present the payoff table (thousand USD).}
		\begin{center}
			\begin{tabular}{l c c c}
				\toprule
				State of nature & P & Q & R \\
				\midrule
				High demand, normal cost & 260.7 & 264.4 & 273.4 \\
				High demand, disrupted cost & 236.4 & 235.4 & 245.4 \\
				Medium demand, normal cost & 167.1 & 170.9 & 175.35 \\
				Medium demand, disrupted cost & 142.8 & 141.9 & 147.35 \\
				Low demand, normal cost & 68.3 & 82.9 & 77.3 \\
				Low demand, disrupted cost & 44 & 53.9 & 49.3 \\
				\bottomrule
			\end{tabular}
		\end{center}
		
		In conclusion, interpretation focuses on how each strategic configuration leads to different consequences under external demand and cost events.
		
		\item
		\subsection*{Problem description}
		A specialty coffee chain must decide between two expansion formats for a district rollout:
		Format A (large stores) or Format B (compact stores).
		Demand can be high or low, and ingredient costs can be stable or volatile.
		Projected net profits are already estimated in thousand USD:
		Format A: 210 (high, stable), 184 (high, volatile), 74 (low, stable), 48 (low, volatile).
		Format B: 198 (high, stable), 170 (high, volatile), 92 (low, stable), 64 (low, volatile).
		
		\subsection*{Solution}
		\textbf{Step 1 --- Identify the decision maker.} The decision maker is the chain's expansion committee.
		
		\textbf{Step 2 --- Identify the decision alternatives.} Alternatives are Format A and Format B.
		
		\textbf{Step 3 --- Identify the states of nature (uncertain events).} States are combined demand--cost events: high/stable, high/volatile, low/stable, and low/volatile.
		
		\textbf{Step 4 --- Interpret what each payoff represents.} Each payoff is the resulting district profit after committing to one format and then observing a particular market and input-cost environment. For example, 64 means Format B with low demand and volatile ingredient costs.
		
		\textbf{Step 5 --- Present the payoff table (thousand USD).}
		\begin{center}
			\begin{tabular}{l c c}
				\toprule
				State of nature & Format A & Format B \\
				\midrule
				High demand, stable costs & 210 & 198 \\
				High demand, volatile costs & 184 & 170 \\
				Low demand, stable costs & 74 & 92 \\
				Low demand, volatile costs & 48 & 64 \\
				\bottomrule
			\end{tabular}
		\end{center}
		
		Therefore, the table is interpreted as a consequence summary: the chosen expansion format interacts with external demand and cost conditions to determine profit.
	\end{ExamProblems}
\end{document}
