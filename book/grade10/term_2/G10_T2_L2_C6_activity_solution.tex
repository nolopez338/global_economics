\makeatletter
\def\input@path{{./}{../}{../../}{preamble/}{../preamble/}{../../preamble/}}
\makeatother
% ----------------------------------------------------------
% GENERAL 

% File
\documentclass[11pt]{book}

% Margins
\usepackage[margin=1in]{geometry}

\linespread{1.2}            % Line spacing
\usepackage[utf8]{inputenc}
\usepackage[T1]{fontenc}
\usepackage{lmodern}
\usepackage{microtype}
\setlength{\parindent}{0pt}
\setlength{\parskip}{6pt}
\usepackage{booktabs}

% ----------------------------------------------------------
% TABLES
\usepackage{multicol}
\usepackage{longtable} 
\usepackage{array}
\usepackage{booktabs}
\usepackage{tabularx}
\usepackage{multirow}

% ----------------------------------------------------------
% MATHEMATICS
\usepackage{amsmath}
\usepackage{amssymb}
\usepackage{amsfonts}
\usepackage{mathtools}

% ----------------------------------------------------------
% HIDDEN CONTENT
\usepackage{ifthen}
% Define a boolean switch
\newboolean{explicaciones}
% Set the boolean switch to true or false
% Change to true to show the content

% Explanations
\newcommand{\explicacion}[2]{
	\ifthenelse{\boolean{explicaciones}}{#1}{#2}
}
\newcommand{\mostrarExplicaciones}[1]{\setboolean{explicaciones}{#1}}

% ----------------------------------------------------------
% NUMBERING

\usepackage{fancyhdr}
\pagestyle{empty} % Ensures the entire document has no page numbers

\usepackage{tocloft}
\renewcommand{\cftdot}{} % Remove dots for sections, if any
\renewcommand{\cftsecleader}{\cftdotfill{\cftdotsep}} % Remove dots for sections, if any
\cftpagenumbersoff{section} % Removes page numbers from sections
\cftpagenumbersoff{subsection} % Removes page numbers from subsections

% ----------------------------------------------------------
% IMAGES 

% General settings
\usepackage{graphicx}       % Insert images
\usepackage{float}          % Position images
% \usepackage{subfigure}      % Subfigures
\graphicspath{{imgs}}       % Image location
\usepackage{subcaption}     % Subfigures II
\usepackage{verbatim}

% Figures
\usepackage{tikz}
\usetikzlibrary{arrows.meta,positioning,trees}

% Colors
\usepackage{xcolor}     
\definecolor{popUp}{HTML}{666666}
\definecolor{popUpIn}{HTML}{CED9E0}
\definecolor{backgroundC}{HTML}{D0E8F2}
\definecolor{backgroundCC}{HTML}{FFFFFF}
\definecolor{borders}{HTML}{8c120d}
\definecolor{padding}{HTML}{77D0D7}
\definecolor{links}{HTML}{CC6F5F}

% ----------------------------------------------------------
% FRAMES

% General settings
\usepackage{tcolorbox}
\usepackage{adjustbox}          % Adjusted frame  
\setlength{\fboxrule}{3pt}  % Line width
\setlength{\fboxsep}{3pt}   % Box padding

% General frames
\usepackage{mdframed}   

\mdfdefinestyle{estiloGeneral}{    % General style
	linecolor=black,
	linewidth=1.5pt,
	roundcorner=10pt,
	backgroundcolor=backgroundC,
	innerbottommargin=0pt
}
\mdfdefinestyle{code}{          % Code style
	linecolor=black,
	linewidth=1.5pt,
	roundcorner=10pt,
	backgroundcolor=darkgray!10,
	innerbottommargin=0pt
}

% Image frame
\newtcbox{\fboxC}{
	colback=backgroundC,
	colframe=popUp,
	arc=10pt,
	boxrule=3pt,
	boxsep=0pt, % Change the padding here
	nobeforeafter
}

% ----------------------------------------------------------
% PAGE SETTINGS

% Background 
\newcommand{\background}[0]{\begin{tikzpicture}[remember picture,overlay]
		\fill[backgroundC] (-2,2) rectangle (25cm, -550);
\end{tikzpicture}}

\newcommand{\backgroundC}[0]{\begin{tikzpicture}[remember picture,overlay]
		\fill[backgroundCC] (-2,2) rectangle (25cm, -550);
\end{tikzpicture}}

% Page width 
\newcommand{\anchoPag}[0]{20cm}

% ----------------------------------------------------------
% FONT

% General
\usepackage{tgbonum}        % Font
\usepackage{listings}       % Code typesetting
\usepackage[spanish]{babel} % Load Spanish
\selectlanguage{spanish}    % Select Spanish
\usepackage{enumitem}
\usepackage{bookmark}

\setlist[itemize]{leftmargin=1.2em, itemsep=0.35em, topsep=0.35em}

% --- Table helpers ---
\newcolumntype{L}[1]{>{\raggedright\arraybackslash}p{#1}}
\newcolumntype{Y}{>{\raggedright\arraybackslash}X}
\newcolumntype{C}{>{\centering\arraybackslash}X}
\renewcommand{\arraystretch}{1.1}

% Python style
\lstdefinestyle{python}{
	language=Python,
	basicstyle=\ttfamily\small,
	commentstyle=\color{green!50!black},
	keywordstyle=\color{blue},
	numberstyle=\tiny\color{gray},
	numbers=left,
	morekeywords={>, <},
	breakatwhitespace=false,
	showstringspaces=false,
	showtabs=false,
	showspaces=false
}

% ----------------------------------------------------------
% HYPERLINKS

% General
\usepackage{hyperref}       
\hypersetup{
	colorlinks=true,
	linkcolor=links,
	filecolor=magenta,      
	urlcolor=brown,
}

% Custom commands 

% Large link
\newcommand{\bigLink}[2]{\begin{center} \fboxC{\LARGE{\href{#1}{#2}}}\end{center}}

% Small link
\newcommand{\smallLink}[2]{\begin{center}\fboxC{\href{#1}{#2}}\end{center}}

% Bold link
\newcommand{\bfLink}[2]{\href{#1}{\textbf{#2}}}


% Small URL
\newcommand{\smallUrl}[1]{\begin{center}\fboxC{\url{#1}}\end{center}}


% ----------------------------------------------------------
% CUSTOM COMMANDS FOR FIGURES

\newcommand{\espacioImagenes}[0]{-1.2cm}

% Without frame
\newcommand{\fig}[3][\espacioImagenes]{
	\hspace*{#1}
	\centering
	\includegraphics[width=#2\textwidth]{#3}
}

% With frame
\newcommand{\ffig}[2]{\begin{figure}[h]
		\hspace*{\espacioImagenes}
		\centering
		\fbox{\includegraphics[width=#1\textwidth]{#2}}
\end{figure}}

% Hyperlink with frame
\newcommand{\hfig}[3]{\begin{figure}[h]
		\hspace*{-1.4cm}
		\centering
		\color{popUp}
		\fboxC{\href{#1}{\includegraphics[width=#2\textwidth]{#3}}}
	\end{figure}
}

% Hyperlink without frame
\newcommand{\hffig}[3]{\begin{figure}[h]
		\hspace*{-1.1cm}
		\centering
		\color{popUp}
		\href{#1}{\includegraphics[width=#2\textwidth]{#3}}
	\end{figure}
}

% ----------------------------------------------------------

% Start and Contents
\newcommand{\cuadro}[1]{
	\begin{mdframed}[style=estiloGeneral]
		#1 
	\end{mdframed}
}

% Explanation video image
\newcommand{\linkExplicacion}[1]{
	\hffig{#1}{0.5}{principal/videoExplicacion}
	\vspace{-0.5cm}
}

\newcommand{\subSecLink}[2]{
	\subsubsection*{\href{#1}{\textbf{#2}}}
}

% Spacing
\newcommand{\esp}[0]{\vspace{4mm}}

% Back to start
\newcommand{\secInicio}[0]{\begin{center}\hyperref[sec:inicio]{ 
			\includegraphics[width=0.1\textwidth]{principal/up}
	}\end{center}
}


\geometry{margin=0.85in}
\AtBeginDocument{\small}

\newcommand{\ExamNameField}{\noindent\textbf{Name:}\ \rule{0.7\linewidth}{0.4pt}\par\medskip}

\newcommand{\ExamTitleBlock}[3]{%
	\begin{center}
		\Large\textbf{#1}\\
		\textbf{#2}%
		\if\relax\detokenize{#3}\relax\else\\\textbf{#3}\fi
	\end{center}
	\vspace{0.5em}
}

\newcommand{\ExamSection}[1]{\par\medskip\textbf{#1}\par\smallskip}

\newenvironment{ExamCriteria}{%
	\begin{itemize}[leftmargin=1.6em, itemsep=0.3em, topsep=0.2em]
}{%
	\end{itemize}
}

\newenvironment{ExamProblems}{%
	\begin{enumerate}[label=\textbf{P\arabic*}, leftmargin=0pt, labelsep=0.6em, itemindent=2.2em, itemsep=0.8em]
}{%
	\end{enumerate}
}

\begin{document}
	\ExamTitleBlock{10th grade}{Learning evidence T2 L2 C6 activity solutions}{}
	
	\ExamSection{Problems}
	\begin{ExamProblems}
		\item
		\subsection*{Problem description}
		A neighborhood bakery is deciding on a weekend market setup. The bakery can expand its stall or keep a standard setup. Expected profit depends on turnout, with high turnout probability $0.60$ and low turnout probability $0.40$. Profit is measured in hundreds of dollars.

		\begin{center}
			\textit{Payoff table}\\
			\begin{tabular}{l c c}
				\toprule
				Alternative & High turnout $(0.60)$ & Low turnout $(0.40)$ \\
				\midrule
				Expand stall & 24 & 6 \\
				Standard stall & 18 & 10 \\
				\bottomrule
			\end{tabular}
		\end{center}

		\subsection*{C6}
		\textbf{Maximum opportunity criterion (minimax regret).} Best payoff in each state.
		\[
		\begin{aligned}
		\text{High turnout gives } &\max\{24,18\}=24,\\
		\text{Low turnout gives } &\max\{6,10\}=10.
		\end{aligned}
		\]

		Regret table.
		\[
		\begin{array}{lccc}
			\toprule
			\text{Alternative} & \text{High turnout} & \text{Low turnout} & \text{Maximum regret}\\
			\midrule
			\text{Expand stall} & 24-24=0 & 10-6=4 & 4\\
			\text{Standard stall} & 24-18=6 & 10-10=0 & 6\\
			\bottomrule
		\end{array}
		\]

		Minimax regret choice is \emph{Expand stall} because its maximum regret is $4$.

		\item
		\subsection*{Problem description}
		A school fundraiser chooses between a booth sale and a delivery sale. Attendance can be high, medium, or low with probabilities $0.35$, $0.40$, and $0.25$. Profits are in hundreds of dollars.

		\begin{center}
			\textit{Payoff table}\\
			\begin{tabular}{l c c c}
				\toprule
				Alternative & High $(0.35)$ & Medium $(0.40)$ & Low $(0.25)$ \\
				\midrule
				Booth sale & 30 & 18 & 4 \\
				Delivery sale & 24 & 16 & 12 \\
				\bottomrule
			\end{tabular}
		\end{center}

		\subsection*{C6}
		\textbf{Maximum opportunity criterion (minimax regret).} Best payoff in each state.
		\[
		\begin{aligned}
		\text{High attendance gives } &\max\{30,24\}=30,\\
		\text{Medium attendance gives } &\max\{18,16\}=18,\\
		\text{Low attendance gives } &\max\{4,12\}=12.
		\end{aligned}
		\]

		Regret table.
		\[
		\begin{array}{lcccc}
			\toprule
			\text{Alternative} & \text{High} & \text{Medium} & \text{Low} & \text{Maximum regret}\\
			\midrule
			\text{Booth sale} & 30-30=0 & 18-18=0 & 12-4=8 & 8\\
			\text{Delivery sale} & 30-24=6 & 18-16=2 & 12-12=0 & 6\\
			\bottomrule
		\end{array}
		\]

		Minimax regret choice is \emph{Delivery sale} because its maximum regret is $6$.

		\item
		\subsection*{Problem description}
		A farm cooperative is choosing a storage plan for a harvest season. It can use Plan A, Plan B, or Plan C. Demand can be strong, moderate, or weak with probabilities $0.30$, $0.45$, and $0.25$. Profits are in hundreds of dollars.

		\begin{center}
			\textit{Payoff table}\\
			\begin{tabular}{l c c c}
				\toprule
				Alternative & Strong $(0.30)$ & Moderate $(0.45)$ & Weak $(0.25)$ \\
				\midrule
				Plan A & 40 & 22 & 5 \\
				Plan B & 32 & 26 & 12 \\
				Plan C & 25 & 20 & 18 \\
				\bottomrule
			\end{tabular}
		\end{center}

		\subsection*{C6}
		\textbf{Maximum opportunity criterion (minimax regret).} Best payoff in each state.
		\[
		\begin{aligned}
		\text{Strong demand gives } &\max\{40,32,25\}=40,\\
		\text{Moderate demand gives } &\max\{22,26,20\}=26,\\
		\text{Weak demand gives } &\max\{5,12,18\}=18.
		\end{aligned}
		\]

		Regret table.
		\[
		\begin{array}{lcccc}
			\toprule
			\text{Alternative} & \text{Strong} & \text{Moderate} & \text{Weak} & \text{Maximum regret}\\
			\midrule
			\text{Plan A} & 40-40=0 & 26-22=4 & 18-5=13 & 13\\
			\text{Plan B} & 40-32=8 & 26-26=0 & 18-12=6 & 8\\
			\text{Plan C} & 40-25=15 & 26-20=6 & 18-18=0 & 15\\
			\bottomrule
		\end{array}
		\]

		Minimax regret choice is \emph{Plan B} because its maximum regret is $8$.

		\item
		\subsection*{Problem description}
		A community theater must choose between a matinee focus and an evening focus. Ticket demand can be strong with probability $0.55$ or weak with probability $0.45$. Profits are in hundreds of dollars.

		\begin{center}
			\textit{Payoff table}\\
			\begin{tabular}{l c c}
				\toprule
				Alternative & Strong demand $(0.55)$ & Weak demand $(0.45)$ \\
				\midrule
				Matinee focus & 20 & 8 \\
				Evening focus & 16 & 12 \\
				\bottomrule
			\end{tabular}
		\end{center}

		\subsection*{C6}
		\textbf{Maximum opportunity criterion (minimax regret).} Best payoff in each state.
		\[
		\begin{aligned}
		\text{Strong demand gives } &\max\{20,16\}=20,\\
		\text{Weak demand gives } &\max\{8,12\}=12.
		\end{aligned}
		\]

		Regret table.
		\[
		\begin{array}{lccc}
			\toprule
			\text{Alternative} & \text{Strong demand} & \text{Weak demand} & \text{Maximum regret}\\
			\midrule
			\text{Matinee focus} & 20-20=0 & 12-8=4 & 4\\
			\text{Evening focus} & 20-16=4 & 12-12=0 & 4\\
			\bottomrule
		\end{array}
		\]

		Both alternatives have maximum regret $4$, so either choice satisfies the criterion.

		\item
		\subsection*{Problem description}
		A transit agency is choosing between a digital campaign and a print campaign. Public interest can be high, medium, or low with probabilities $0.30$, $0.40$, and $0.30$. Profits are in hundreds of dollars.

		\begin{center}
			\textit{Payoff table}\\
			\begin{tabular}{l c c c}
				\toprule
				Alternative & High $(0.30)$ & Medium $(0.40)$ & Low $(0.30)$ \\
				\midrule
				Digital campaign & 28 & 14 & 6 \\
				Print campaign & 22 & 16 & 10 \\
				\bottomrule
			\end{tabular}
		\end{center}

		\subsection*{C6}
		\textbf{Maximum opportunity criterion (minimax regret).} Best payoff in each state.
		\[
		\begin{aligned}
		\text{High interest gives } &\max\{28,22\}=28,\\
		\text{Medium interest gives } &\max\{14,16\}=16,\\
		\text{Low interest gives } &\max\{6,10\}=10.
		\end{aligned}
		\]

		Regret table.
		\[
		\begin{array}{lcccc}
			\toprule
			\text{Alternative} & \text{High} & \text{Medium} & \text{Low} & \text{Maximum regret}\\
			\midrule
			\text{Digital campaign} & 28-28=0 & 16-14=2 & 10-6=4 & 4\\
			\text{Print campaign} & 28-22=6 & 16-16=0 & 10-10=0 & 6\\
			\bottomrule
		\end{array}
		\]

		Minimax regret choice is \emph{Digital campaign} because its maximum regret is $4$.
	\end{ExamProblems}
\end{document}
