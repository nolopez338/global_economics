\makeatletter
\def\input@path{{./}{../}{../../}{preamble/}{../preamble/}{../../preamble/}}
\makeatother
% ----------------------------------------------------------
% GENERAL 

% File
\documentclass[11pt]{book}

% Margins
\usepackage[margin=1in]{geometry}

\linespread{1.2}            % Line spacing
\usepackage[utf8]{inputenc}
\usepackage[T1]{fontenc}
\usepackage{lmodern}
\usepackage{microtype}
\setlength{\parindent}{0pt}
\setlength{\parskip}{6pt}
\usepackage{booktabs}

% ----------------------------------------------------------
% TABLES
\usepackage{multicol}
\usepackage{longtable} 
\usepackage{array}
\usepackage{booktabs}
\usepackage{tabularx}
\usepackage{multirow}

% ----------------------------------------------------------
% MATHEMATICS
\usepackage{amsmath}
\usepackage{amssymb}
\usepackage{amsfonts}
\usepackage{mathtools}

% ----------------------------------------------------------
% HIDDEN CONTENT
\usepackage{ifthen}
% Define a boolean switch
\newboolean{explicaciones}
% Set the boolean switch to true or false
% Change to true to show the content

% Explanations
\newcommand{\explicacion}[2]{
	\ifthenelse{\boolean{explicaciones}}{#1}{#2}
}
\newcommand{\mostrarExplicaciones}[1]{\setboolean{explicaciones}{#1}}

% ----------------------------------------------------------
% NUMBERING

\usepackage{fancyhdr}
\pagestyle{empty} % Ensures the entire document has no page numbers

\usepackage{tocloft}
\renewcommand{\cftdot}{} % Remove dots for sections, if any
\renewcommand{\cftsecleader}{\cftdotfill{\cftdotsep}} % Remove dots for sections, if any
\cftpagenumbersoff{section} % Removes page numbers from sections
\cftpagenumbersoff{subsection} % Removes page numbers from subsections

% ----------------------------------------------------------
% IMAGES 

% General settings
\usepackage{graphicx}       % Insert images
\usepackage{float}          % Position images
% \usepackage{subfigure}      % Subfigures
\graphicspath{{imgs}}       % Image location
\usepackage{subcaption}     % Subfigures II
\usepackage{verbatim}

% Figures
\usepackage{tikz}
\usetikzlibrary{arrows.meta,positioning,trees}

% Colors
\usepackage{xcolor}     
\definecolor{popUp}{HTML}{666666}
\definecolor{popUpIn}{HTML}{CED9E0}
\definecolor{backgroundC}{HTML}{D0E8F2}
\definecolor{backgroundCC}{HTML}{FFFFFF}
\definecolor{borders}{HTML}{8c120d}
\definecolor{padding}{HTML}{77D0D7}
\definecolor{links}{HTML}{CC6F5F}

% ----------------------------------------------------------
% FRAMES

% General settings
\usepackage{tcolorbox}
\usepackage{adjustbox}          % Adjusted frame  
\setlength{\fboxrule}{3pt}  % Line width
\setlength{\fboxsep}{3pt}   % Box padding

% General frames
\usepackage{mdframed}   

\mdfdefinestyle{estiloGeneral}{    % General style
	linecolor=black,
	linewidth=1.5pt,
	roundcorner=10pt,
	backgroundcolor=backgroundC,
	innerbottommargin=0pt
}
\mdfdefinestyle{code}{          % Code style
	linecolor=black,
	linewidth=1.5pt,
	roundcorner=10pt,
	backgroundcolor=darkgray!10,
	innerbottommargin=0pt
}

% Image frame
\newtcbox{\fboxC}{
	colback=backgroundC,
	colframe=popUp,
	arc=10pt,
	boxrule=3pt,
	boxsep=0pt, % Change the padding here
	nobeforeafter
}

% ----------------------------------------------------------
% PAGE SETTINGS

% Background 
\newcommand{\background}[0]{\begin{tikzpicture}[remember picture,overlay]
		\fill[backgroundC] (-2,2) rectangle (25cm, -550);
\end{tikzpicture}}

\newcommand{\backgroundC}[0]{\begin{tikzpicture}[remember picture,overlay]
		\fill[backgroundCC] (-2,2) rectangle (25cm, -550);
\end{tikzpicture}}

% Page width 
\newcommand{\anchoPag}[0]{20cm}

% ----------------------------------------------------------
% FONT

% General
\usepackage{tgbonum}        % Font
\usepackage{listings}       % Code typesetting
\usepackage[spanish]{babel} % Load Spanish
\selectlanguage{spanish}    % Select Spanish
\usepackage{enumitem}
\usepackage{bookmark}

\setlist[itemize]{leftmargin=1.2em, itemsep=0.35em, topsep=0.35em}

% --- Table helpers ---
\newcolumntype{L}[1]{>{\raggedright\arraybackslash}p{#1}}
\newcolumntype{Y}{>{\raggedright\arraybackslash}X}
\newcolumntype{C}{>{\centering\arraybackslash}X}
\renewcommand{\arraystretch}{1.1}

% Python style
\lstdefinestyle{python}{
	language=Python,
	basicstyle=\ttfamily\small,
	commentstyle=\color{green!50!black},
	keywordstyle=\color{blue},
	numberstyle=\tiny\color{gray},
	numbers=left,
	morekeywords={>, <},
	breakatwhitespace=false,
	showstringspaces=false,
	showtabs=false,
	showspaces=false
}

% ----------------------------------------------------------
% HYPERLINKS

% General
\usepackage{hyperref}       
\hypersetup{
	colorlinks=true,
	linkcolor=links,
	filecolor=magenta,      
	urlcolor=brown,
}

% Custom commands 

% Large link
\newcommand{\bigLink}[2]{\begin{center} \fboxC{\LARGE{\href{#1}{#2}}}\end{center}}

% Small link
\newcommand{\smallLink}[2]{\begin{center}\fboxC{\href{#1}{#2}}\end{center}}

% Bold link
\newcommand{\bfLink}[2]{\href{#1}{\textbf{#2}}}


% Small URL
\newcommand{\smallUrl}[1]{\begin{center}\fboxC{\url{#1}}\end{center}}


% ----------------------------------------------------------
% CUSTOM COMMANDS FOR FIGURES

\newcommand{\espacioImagenes}[0]{-1.2cm}

% Without frame
\newcommand{\fig}[3][\espacioImagenes]{
	\hspace*{#1}
	\centering
	\includegraphics[width=#2\textwidth]{#3}
}

% With frame
\newcommand{\ffig}[2]{\begin{figure}[h]
		\hspace*{\espacioImagenes}
		\centering
		\fbox{\includegraphics[width=#1\textwidth]{#2}}
\end{figure}}

% Hyperlink with frame
\newcommand{\hfig}[3]{\begin{figure}[h]
		\hspace*{-1.4cm}
		\centering
		\color{popUp}
		\fboxC{\href{#1}{\includegraphics[width=#2\textwidth]{#3}}}
	\end{figure}
}

% Hyperlink without frame
\newcommand{\hffig}[3]{\begin{figure}[h]
		\hspace*{-1.1cm}
		\centering
		\color{popUp}
		\href{#1}{\includegraphics[width=#2\textwidth]{#3}}
	\end{figure}
}

% ----------------------------------------------------------

% Start and Contents
\newcommand{\cuadro}[1]{
	\begin{mdframed}[style=estiloGeneral]
		#1 
	\end{mdframed}
}

% Explanation video image
\newcommand{\linkExplicacion}[1]{
	\hffig{#1}{0.5}{principal/videoExplicacion}
	\vspace{-0.5cm}
}

\newcommand{\subSecLink}[2]{
	\subsubsection*{\href{#1}{\textbf{#2}}}
}

% Spacing
\newcommand{\esp}[0]{\vspace{4mm}}

% Back to start
\newcommand{\secInicio}[0]{\begin{center}\hyperref[sec:inicio]{ 
			\includegraphics[width=0.1\textwidth]{principal/up}
	}\end{center}
}


\geometry{margin=0.85in}
\AtBeginDocument{\small}

\newcommand{\ExamNameField}{\noindent\textbf{Name:}\ \rule{0.7\linewidth}{0.4pt}\par\medskip}

\newcommand{\ExamTitleBlock}[3]{%
	\begin{center}
		\Large\textbf{#1}\\
		\textbf{#2}%
		\if\relax\detokenize{#3}\relax\else\\\textbf{#3}\fi
	\end{center}
	\vspace{0.5em}
}

\newcommand{\ExamSection}[1]{\par\medskip\textbf{#1}\par\smallskip}

\newenvironment{ExamCriteria}{%
	\begin{itemize}[leftmargin=1.6em, itemsep=0.3em, topsep=0.2em]
}{%
	\end{itemize}
}

\newenvironment{ExamProblems}{%
	\begin{enumerate}[label=\textbf{P\arabic*}, leftmargin=0pt, labelsep=0.6em, itemindent=2.2em, itemsep=0.8em]
}{%
	\end{enumerate}
}

\begin{document}
	\ExamTitleBlock{10th grade}{Learning evidence 2.1 Decision analysis supplementary solutions 1}{}
	
	\ExamSection{Problems}
	\begin{ExamProblems}
		\item
		\subsection*{Problem description}
		A weekend coffee cart must choose a service plan for a street market. Plan A is a specialty menu and Plan B is a classic menu. Customer turnout can be high or low with probabilities $0.55$ and $0.45$. Profit outcomes are given directly: under high turnout Plan A earns $700$ and Plan B earns $480$, and under low turnout Plan A earns $250$ and Plan B earns $200$. Which decision is the risky choice under Maximax, the safe choice under Maximin, and the balanced choice under Expected Value?

		\subsection*{C2}
		\begin{center}
			\begin{tabular}{l p{0.74\linewidth}}
				\toprule
				Decision Alternatives & Plan A (specialty menu), Plan B (classic menu) \\
				States of Nature & Turnout high or low with probabilities $0.55$ and $0.45$ \\
				Events & Realized customer turnout in the market day \\
				Consequences & Profit in dollars \\
				\bottomrule
			\end{tabular}
		\end{center}

		\subsection*{C3}
		Payoff is the profit in dollars given for each alternative and state.
		\begin{center}
			\begin{tabular}{l c c c}
				\toprule
				State of nature & Probability & Plan A & Plan B \\
				\midrule
				High turnout & 0.55 & 700 & 480 \\
				Low turnout & 0.45 & 250 & 200 \\
				\bottomrule
		\end{tabular}
		\end{center}

		\subsection*{C4}
		\[
		\begin{aligned}
		\max(\text{Plan A}) &= \max\{700,250\} = 700,\\
		\max(\text{Plan B}) &= \max\{480,200\} = 480.
		\end{aligned}
		\]
		\[
		\max\{\max(\text{Plan A}), \max(\text{Plan B})\}
		= \max\{700, 480\}
		= 700.
		\]
		The Maximax choice is Plan A with 700. The criterion is \emph{risky} because it focuses only on the best possible payoff and ignores how likely it is.

		\subsection*{C5}
		\[
		\begin{aligned}
		\min(\text{Plan A}) &= \min\{700,250\} = 250,\\
		\min(\text{Plan B}) &= \min\{480,200\} = 200.
		\end{aligned}
		\]
		\[
		\max\{\min(\text{Plan A}), \min(\text{Plan B})\}
		= \max\{250, 200\}
		= 250.
		\]
		The Maximin choice is Plan A with 250 because it has the least severe worst case. This is \emph{safe} because it protects against the worst outcome.

		\subsection*{C1}
		\[
		\begin{aligned}
		EV_A &= 0.55(700)+0.45(250)=385+112.5=497.5,\\
		EV_B &= 0.55(480)+0.45(200)=264+90=354.
		\end{aligned}
		\]
		The expected value criterion chooses Plan A because $497.5$ is the largest expected profit. This aligns with the Maximax and Maximin outcomes for Plan A, so it is the balanced decision using the probabilities.

		\item
		\subsection*{Problem description}
		A campus bookstore must choose a stocking plan for a semester. Plan A is premium notebooks, Plan B is standard notebooks, and Plan C is discount notebooks. Demand for notebooks can be strong, moderate, or weak with probabilities $0.35$, $0.45$, and $0.20$. Expected unit sales depend on demand and plan. Under strong demand Plan A sells 420 units, Plan B sells 520 units, and Plan C sells 600 units. Under moderate demand Plan A sells 300 units, Plan B sells 380 units, and Plan C sells 450 units. Under weak demand Plan A sells 180 units, Plan B sells 240 units, and Plan C sells 300 units. Unit prices are $22$, $18$, and $15$ for Plans A, B, and C. Use these parameters to calculate the payoff table. Which decision is the risky choice under Maximax, the safe choice under Maximin, and the balanced choice under Expected Value?

		\subsection*{C2}
		\begin{center}
			\begin{tabular}{l p{0.74\linewidth}}
				\toprule
				Decision Alternatives & Plan A (premium), Plan B (standard), Plan C (discount) \\
				States of Nature & Demand strong, moderate, weak with probabilities $0.35$, $0.45$, $0.20$ \\
				Events & Realized demand level during the semester \\
				Consequences & Payoff in dollars from revenue \\
				\bottomrule
			\end{tabular}
		\end{center}

		\subsection*{C3}
		Payoff equals revenue, where revenue = units $\times$ price.
		\begin{center}
			\textit{Units by plan and demand} \\
			\begin{tabular}{l c c c}
				\toprule
				Plan & Strong & Moderate & Weak \\
				\midrule
				Plan A & 420 & 300 & 180 \\
				Plan B & 520 & 380 & 240 \\
				Plan C & 600 & 450 & 300 \\
				\bottomrule
		\end{tabular}
		\end{center}
		\begin{center}
			\textit{Price parameters by plan}
			\begin{center}
				\begin{tabular}{l c}
					\toprule
					Plan & Price per unit \\
					\midrule
					Plan A & 22 \\
					Plan B & 18 \\
					Plan C & 15 \\
					\bottomrule
				\end{tabular}
			\end{center}
		\end{center}
		\begin{center}
			\textit{Revenue by demand} \\
			\begin{tabular}{l c c c}
				\toprule
				State of nature & Plan A revenue & Plan B revenue & Plan C revenue \\
				\midrule
				Strong demand &
				$\begin{array}{l}
					420\cdot 22\\
					= 9240
				\end{array}$ &
				$\begin{array}{l}
					520\cdot 18\\
					= 9360
				\end{array}$ &
				$\begin{array}{l}
					600\cdot 15\\
					= 9000
				\end{array}$ \\
				Moderate demand &
				$\begin{array}{l}
					300\cdot 22\\
					= 6600
				\end{array}$ &
				$\begin{array}{l}
					380\cdot 18\\
					= 6840
				\end{array}$ &
				$\begin{array}{l}
					450\cdot 15\\
					= 6750
				\end{array}$ \\
				Weak demand &
				$\begin{array}{l}
					180\cdot 22\\
					= 3960
				\end{array}$ &
				$\begin{array}{l}
					240\cdot 18\\
					= 4320
				\end{array}$ &
				$\begin{array}{l}
					300\cdot 15\\
					= 4500
				\end{array}$ \\
				\bottomrule
		\end{tabular}
		\end{center}
		Payoff equals revenue, so the payoff table matches the revenue values.
		\begin{center}
			\textit{Payoff table} \\
			\begin{tabular}{l c c c c}
				\toprule
				State of nature & Probability & Plan A & Plan B & Plan C \\
				\midrule
				Strong demand & 0.35 & 9240 & 9360 & 9000 \\
				Moderate demand & 0.45 & 6600 & 6840 & 6750 \\
				Weak demand & 0.20 & 3960 & 4320 & 4500 \\
				\bottomrule
		\end{tabular}
		\end{center}

		\subsection*{C4}
		\[
		\begin{aligned}
		\max(\text{Plan A}) &= \max\{9240,6600,3960\} = 9240,\\
		\max(\text{Plan B}) &= \max\{9360,6840,4320\} = 9360,\\
		\max(\text{Plan C}) &= \max\{9000,6750,4500\} = 9000.
		\end{aligned}
		\]
		\[
		\max\{\max(\text{Plan A}), \max(\text{Plan B}), \max(\text{Plan C})\}
		= \max\{9240, 9360, 9000\}
		= 9360.
		\]
		The Maximax choice is Plan B with 9360. The criterion is \emph{risky} because it focuses only on the best possible payoff and ignores how likely it is.

		\subsection*{C5}
		\[
		\begin{aligned}
		\min(\text{Plan A}) &= \min\{9240,6600,3960\} = 3960,\\
		\min(\text{Plan B}) &= \min\{9360,6840,4320\} = 4320,\\
		\min(\text{Plan C}) &= \min\{9000,6750,4500\} = 4500.
		\end{aligned}
		\]
		\[
		\max\{\min(\text{Plan A}), \min(\text{Plan B}), \min(\text{Plan C})\}
		= \max\{3960, 4320, 4500\}
		= 4500.
		\]
		The Maximin choice is Plan C with 4500 because it gives the best worst case outcome. This is \emph{safe} because it prioritizes the least harmful result.

		\subsection*{C1}
		\[
		\begin{aligned}
		EV_A &= 0.35(9240)+0.45(6600)+0.20(3960)=3234+2970+792=6996,\\
		EV_B &= 0.35(9360)+0.45(6840)+0.20(4320)=3276+3078+864=7218,\\
		EV_C &= 0.35(9000)+0.45(6750)+0.20(4500)=3150+3037.5+900=7087.5.
		\end{aligned}
		\]
			The expected value criterion chooses Plan B because $7218$ is the largest expected payoff. This matches the Maximax outcome and differs from the Maximin outcome, so it is the balanced decision using the probabilities.
	
			\item
			\subsection*{Problem description}
			A meal subscription service must choose a weekly menu plan for the next quarter. Plan A is a chef series, Plan B is a family bundle, and Plan C is a value bundle. Demand can be surge, high, moderate, or low with probabilities $0.20$, $0.35$, $0.30$, and $0.15$. Expected subscriptions depend on demand and plan. Under surge demand Plan A sells 500 subscriptions, Plan B sells 620 subscriptions, and Plan C sells 700 subscriptions. Under high demand Plan A sells 420 subscriptions, Plan B sells 520 subscriptions, and Plan C sells 600 subscriptions. Under moderate demand Plan A sells 300 subscriptions, Plan B sells 380 subscriptions, and Plan C sells 450 subscriptions. Under low demand Plan A sells 200 subscriptions, Plan B sells 260 subscriptions, and Plan C sells 320 subscriptions. Unit prices are $24$, $19$, and $16$ for Plans A, B, and C. Variable costs per subscription are $11$, $9$, and $7$. Use these parameters to calculate the profit payoff table. Which decision is the risky choice under Maximax, the safe choice under Maximin, and the balanced choice under Expected Value?
	
			\subsection*{C2}
			\begin{center}
				\begin{tabular}{l p{0.74\linewidth}}
					\toprule
					Decision Alternatives & Plan A (chef series), Plan B (family bundle), Plan C (value bundle) \\
					States of Nature & Demand surge, high, moderate, low with probabilities $0.20$, $0.35$, $0.30$, $0.15$ \\
					Events & Realized demand level for the quarter \\
					Consequences & Profit in dollars after revenue and costs \\
					\bottomrule
				\end{tabular}
			\end{center}
	
			\subsection*{C3}
			Payoff = revenue $-$ cost, where revenue = (subscriptions $\times$ price) and cost = (subscriptions $\times$ variable cost).
			\begin{center}
				\textit{Subscriptions by plan and demand} \\
				\begin{tabular}{l c c c c}
					\toprule
					Plan & Surge & High & Moderate & Low \\
					\midrule
					Plan A & 500 & 420 & 300 & 200 \\
					Plan B & 620 & 520 & 380 & 260 \\
					Plan C & 700 & 600 & 450 & 320 \\
					\bottomrule
			\end{tabular}
			\end{center}
			\begin{center}
				\begin{minipage}[t]{0.48\linewidth}
					\textit{Price parameters by plan}
					\begin{center}
						\begin{tabular}{l c}
							\toprule
							Plan & Price per subscription \\
							\midrule
							Plan A & 24 \\
							Plan B & 19 \\
							Plan C & 16 \\
							\bottomrule
						\end{tabular}
					\end{center}
				\end{minipage}
				\hfill
				\begin{minipage}[t]{0.48\linewidth}
					\textit{Cost parameters by plan}
					\begin{center}
						\begin{tabular}{l c}
							\toprule
							Plan & Variable cost \\
							\midrule
							Plan A & 11 \\
							Plan B & 9 \\
							Plan C & 7 \\
							\bottomrule
						\end{tabular}
					\end{center}
				\end{minipage}
			\end{center}
			\begin{center}
				\textit{Revenue by demand} \\
				\begin{tabular}{l c c c}
					\toprule
					State of nature & Plan A revenue & Plan B revenue & Plan C revenue \\
					\midrule
					Surge demand &
					$\begin{array}{l}
						500\cdot 24\\
						= 12000
					\end{array}$ &
					$\begin{array}{l}
						620\cdot 19\\
						= 11780
					\end{array}$ &
					$\begin{array}{l}
						700\cdot 16\\
						= 11200
					\end{array}$ \\
					High demand &
					$\begin{array}{l}
						420\cdot 24\\
						= 10080
					\end{array}$ &
					$\begin{array}{l}
						520\cdot 19\\
						= 9880
					\end{array}$ &
					$\begin{array}{l}
						600\cdot 16\\
						= 9600
					\end{array}$ \\
					Moderate demand &
					$\begin{array}{l}
						300\cdot 24\\
						= 7200
					\end{array}$ &
					$\begin{array}{l}
						380\cdot 19\\
						= 7220
					\end{array}$ &
					$\begin{array}{l}
						450\cdot 16\\
						= 7200
					\end{array}$ \\
					Low demand &
					$\begin{array}{l}
						200\cdot 24\\
						= 4800
					\end{array}$ &
					$\begin{array}{l}
						260\cdot 19\\
						= 4940
					\end{array}$ &
					$\begin{array}{l}
						320\cdot 16\\
						= 5120
					\end{array}$ \\
					\bottomrule
			\end{tabular}
			\end{center}
			\begin{center}
				\textit{Cost by demand} \\
				\begin{tabular}{l c c c}
					\toprule
					State of nature & Plan A cost & Plan B cost & Plan C cost \\
					\midrule
					Surge demand &
					$\begin{array}{l}
						500\cdot 11\\
						= 5500
					\end{array}$ &
					$\begin{array}{l}
						620\cdot 9\\
						= 5580
					\end{array}$ &
					$\begin{array}{l}
						700\cdot 7\\
						= 4900
					\end{array}$ \\
					High demand &
					$\begin{array}{l}
						420\cdot 11\\
						= 4620
					\end{array}$ &
					$\begin{array}{l}
						520\cdot 9\\
						= 4680
					\end{array}$ &
					$\begin{array}{l}
						600\cdot 7\\
						= 4200
					\end{array}$ \\
					Moderate demand &
					$\begin{array}{l}
						300\cdot 11\\
						= 3300
					\end{array}$ &
					$\begin{array}{l}
						380\cdot 9\\
						= 3420
					\end{array}$ &
					$\begin{array}{l}
						450\cdot 7\\
						= 3150
					\end{array}$ \\
					Low demand &
					$\begin{array}{l}
						200\cdot 11\\
						= 2200
					\end{array}$ &
					$\begin{array}{l}
						260\cdot 9\\
						= 2340
					\end{array}$ &
					$\begin{array}{l}
						320\cdot 7\\
						= 2240
					\end{array}$ \\
					\bottomrule
			\end{tabular}
			\end{center}
			\begin{center}
				\textit{Profit parameters by plan}
				\begin{tabular}{l p{0.17\linewidth} p{0.2\linewidth} p{0.2\linewidth} p{0.2\linewidth}}
					\toprule
					State of nature & Probability & Plan A & Plan B & Plan C \\
					\midrule
					Surge demand &
					$\begin{array}{l}
						0.20
					\end{array}$ &
					$\begin{array}{l}
						12000 - 5500\\
						= 6500
					\end{array}$ &
					$\begin{array}{l}
						11780 - 5580\\
						= 6200
					\end{array}$ &
					$\begin{array}{l}
						11200 - 4900\\
						= 6300
					\end{array}$ \\
					High demand &
					$\begin{array}{l}
						0.35
					\end{array}$ &
					$\begin{array}{l}
						10080 - 4620\\
						= 5460
					\end{array}$ &
					$\begin{array}{l}
						9880 - 4680\\
						= 5200
					\end{array}$ &
					$\begin{array}{l}
						9600 - 4200\\
						= 5400
					\end{array}$ \\
					Moderate demand &
					$\begin{array}{l}
						0.30
					\end{array}$ &
					$\begin{array}{l}
						7200 - 3300\\
						= 3900
					\end{array}$ &
					$\begin{array}{l}
						7220 - 3420\\
						= 3800
					\end{array}$ &
					$\begin{array}{l}
						7200 - 3150\\
						= 4050
					\end{array}$ \\
					Low demand &
					$\begin{array}{l}
						0.15
					\end{array}$ &
					$\begin{array}{l}
						4800 - 2200\\
						= 2600
					\end{array}$ &
					$\begin{array}{l}
						4940 - 2340\\
						= 2600
					\end{array}$ &
					$\begin{array}{l}
						5120 - 2240\\
						= 2880
					\end{array}$ \\
					\bottomrule
			\end{tabular}
			\end{center}
			\begin{center}
				\textit{Payoff table} \\
				\begin{tabular}{l c c c c}
					\toprule
					State of nature & Probability & Plan A & Plan B & Plan C \\
					\midrule
					Surge demand & 0.20 & 6500 & 6200 & 6300 \\
					High demand & 0.35 & 5460 & 5200 & 5400 \\
					Moderate demand & 0.30 & 3900 & 3800 & 4050 \\
					Low demand & 0.15 & 2600 & 2600 & 2880 \\
					\bottomrule
			\end{tabular}
			\end{center}
	
			\subsection*{C4}
			\[
			\begin{aligned}
			\max(\text{Plan A}) &= \max\{6500,5460,3900,2600\} = 6500,\\
			\max(\text{Plan B}) &= \max\{6200,5200,3800,2600\} = 6200,\\
			\max(\text{Plan C}) &= \max\{6300,5400,4050,2880\} = 6300.
			\end{aligned}
			\]
			\[
			\max\{\max(\text{Plan A}), \max(\text{Plan B}), \max(\text{Plan C})\}
			= \max\{6500, 6200, 6300\}
			= 6500.
			\]
			The Maximax choice is Plan A with 6500. The criterion is \emph{risky} because it focuses only on the best possible payoff and ignores how likely it is.
	
			\subsection*{C5}
			\[
			\begin{aligned}
			\min(\text{Plan A}) &= \min\{6500,5460,3900,2600\} = 2600,\\
			\min(\text{Plan B}) &= \min\{6200,5200,3800,2600\} = 2600,\\
			\min(\text{Plan C}) &= \min\{6300,5400,4050,2880\} = 2880.
			\end{aligned}
			\]
			\[
			\max\{\min(\text{Plan A}), \min(\text{Plan B}), \min(\text{Plan C})\}
			= \max\{2600, 2600, 2880\}
			= 2880.
			\]
			The Maximin choice is Plan C with 2880 because it has the least severe worst case. This is \emph{safe} because it protects against the worst outcome.
	
			\subsection*{C1}
			Expected values use the demand probabilities.
			\[
			\begin{aligned}
			EV_A &= 0.20(6500)+0.35(5460)+0.30(3900)+0.15(2600)\\
			&= 1300+1911+1170+390=4771,\\
			EV_B &= 0.20(6200)+0.35(5200)+0.30(3800)+0.15(2600)\\
			&= 1240+1820+1140+390=4590,\\
			EV_C &= 0.20(6300)+0.35(5400)+0.30(4050)+0.15(2880)\\
			&= 1260+1890+1215+432=4797.
			\end{aligned}
			\]
			The expected value criterion chooses Plan C because $4797$ is the largest expected profit. This differs from the Maximax choice and matches the Maximin choice, so the expected value emphasizes weighted outcomes.
	
			\item
			\subsection*{Problem description}
			A drone delivery service must choose a routing plan for the next season. Strategy A uses an in house routing system and Strategy B uses a contracted routing partner. Demand can be strong or weak with probabilities $0.55$ and $0.45$. Battery cost conditions can be favorable or unfavorable with probabilities $0.70$ and $0.30$, and these conditions are independent of demand. Revenue in thousands of dollars depends on demand and strategy. Under strong demand revenue is 620 for Strategy A and 660 for Strategy B. Under weak demand revenue is 340 for Strategy A and 320 for Strategy B. Operating costs in thousands of dollars depend on battery cost conditions and strategy. Under favorable costs operating costs are 260 for Strategy A and 300 for Strategy B. Under unfavorable costs operating costs are 340 for Strategy A and 380 for Strategy B. Which decision is the risky choice under Maximax, the safe choice under Maximin, and the balanced choice under Expected Value?

		\subsection*{C2}
		\begin{center}
			\begin{tabular}{l p{0.74\linewidth}}
				\toprule
				Decision Alternatives & Strategy A (in house), Strategy B (contracted) \\
				States of Nature & Demand strong or weak with battery costs favorable or unfavorable \\
				Events & Realized demand level paired with battery cost conditions \\
				Consequences & Profit in thousands of dollars from revenue minus operating costs \\
				\bottomrule
			\end{tabular}
		\end{center}

		\subsection*{C3}
		Payoff equals revenue minus cost, where revenue is determined by demand and cost is determined by battery conditions.
		\begin{center}
			\begin{minipage}[t]{0.48\linewidth}
				\textit{Revenue parameters by strategy}
				\begin{center}
					\begin{tabular}{l c c}
						\toprule
						Strategy & Strong & Weak \\
						\midrule
						Probabilities & 0.55 & 0.45 \\
						Strategy A & 620 & 340 \\
						Strategy B & 660 & 320 \\
						\bottomrule
					\end{tabular}
				\end{center}
			\end{minipage}
			\hfill
			\begin{minipage}[t]{0.48\linewidth}
				\textit{Cost parameters by strategy}
				\begin{center}
					\begin{tabular}{l c c}
						\toprule
						Strategy & Favorable & Unfavorable \\
						\midrule
						Probabilities & 0.70 & 0.30 \\
						Strategy A & 260 & 340 \\
						Strategy B & 300 & 380 \\
						\bottomrule
					\end{tabular}
				\end{center}
			\end{minipage}
		\end{center}
		Profit is computed as revenue minus cost for each alternative and state.
		\begin{center}
			\textit{Profit parameters by strategy}
			\begin{tabular}{l p{0.17\linewidth} p{0.2\linewidth} p{0.2\linewidth}}
				\toprule
				State of nature & Probability & Strategy A & Strategy B \\
				\midrule
				Strong demand, favorable cost &
				$\begin{array}{l}
					0.55\cdot 0.70\\
					= 0.385
				\end{array}$ &
				$\begin{array}{l}
					620 - 260\\
					= 360
				\end{array}$ &
				$\begin{array}{l}
					660 - 300\\
					= 360
				\end{array}$ \\
				Strong demand, unfavorable cost &
				$\begin{array}{l}
					0.55\cdot 0.30\\
					= 0.165
				\end{array}$ &
				$\begin{array}{l}
					620 - 340\\
					= 280
				\end{array}$ &
				$\begin{array}{l}
					660 - 380\\
					= 280
				\end{array}$ \\
				Weak demand, favorable cost &
				$\begin{array}{l}
					0.45\cdot 0.70\\
					= 0.315
				\end{array}$ &
				$\begin{array}{l}
					340 - 260\\
					= 80
				\end{array}$ &
				$\begin{array}{l}
					320 - 300\\
					= 20
				\end{array}$ \\
				Weak demand, unfavorable cost &
				$\begin{array}{l}
					0.45\cdot 0.30\\
					= 0.135
				\end{array}$ &
				$\begin{array}{l}
					340 - 340\\
					= 0
				\end{array}$ &
				$\begin{array}{l}
					320 - 380\\
					= -60
				\end{array}$ \\
				\bottomrule
		\end{tabular}
		\end{center}
		Combined probabilities are $0.55\cdot 0.70=0.385$, $0.55\cdot 0.30=0.165$, $0.45\cdot 0.70=0.315$, $0.45\cdot 0.30=0.135$.
		\begin{center}
			\textit{Payoff table} \\
			\begin{tabular}{l c c c}
				\toprule
				State of nature & Probability & Strategy A & Strategy B \\
				\midrule
				Strong demand, favorable cost & 0.385 & 360 & 360 \\
				Strong demand, unfavorable cost & 0.165 & 280 & 280 \\
				Weak demand, favorable cost & 0.315 & 80 & 20 \\
				Weak demand, unfavorable cost & 0.135 & 0 & -60 \\
				\bottomrule
		\end{tabular}
		\end{center}

		\subsection*{C4}
		\[
		\begin{aligned}
		\max(\text{Strategy A}) &= \max\{360,280,80,0\} = 360,\\
		\max(\text{Strategy B}) &= \max\{360,280,20,-60\} = 360.
		\end{aligned}
		\]
		\[
		\max\{\max(\text{Strategy A}), \max(\text{Strategy B})\}
		= \max\{360, 360\}
		= 360.
		\]
		The Maximax choice is Strategy A or Strategy B with 360. The criterion is \emph{risky} because it focuses only on the best possible payoff and ignores how likely it is.

		\subsection*{C5}
		\[
		\begin{aligned}
		\min(\text{Strategy A}) &= \min\{360,280,80,0\} = 0,\\
		\min(\text{Strategy B}) &= \min\{360,280,20,-60\} = -60.
		\end{aligned}
		\]
		\[
		\max\{\min(\text{Strategy A}), \min(\text{Strategy B})\}
		= \max\{0, -60\}
		= 0.
		\]
		The Maximin choice is Strategy A with 0 because it has the least severe worst case. This is \emph{safe} because it protects against the worst outcome.

		\subsection*{C1}
		Expected values use the combined state probabilities.
		\[
		\begin{aligned}
		EV_A &= 0.385(360)+0.165(280)+0.315(80)+0.135(0)\\
		&= 138.6+46.2+25.2+0=210,\\
		EV_B &= 0.385(360)+0.165(280)+0.315(20)+0.135(-60)\\
		&= 138.6+46.2+6.3-8.1=183.
		\end{aligned}
		\]
		The expected value criterion chooses Strategy A because $210$ is the largest expected profit. This differs from the Maximin choice and matches the Maximax outcome, so the expected value gives a balanced decision using the probabilities.

		\item
		\subsection*{Problem description}
		A regional utility must choose a generation mix for the next year. Strategy A is solar heavy, Strategy B is wind heavy, and Strategy C is balanced. Demand can be strong or weak with probabilities $0.55$ and $0.45$. Operating cost conditions can be favorable or unfavorable with probabilities $0.60$ and $0.40$, and these conditions are independent of demand. Revenue in thousands of dollars depends on demand and strategy. Under strong demand revenue is 680 for Strategy A, 620 for Strategy B, and 650 for Strategy C. Under weak demand revenue is 360 for Strategy A, 380 for Strategy B, and 370 for Strategy C. Operating costs in thousands of dollars depend on cost conditions and strategy. Under favorable costs operating costs are 300 for Strategy A, 280 for Strategy B, and 290 for Strategy C. Under unfavorable costs operating costs are 380 for Strategy A, 360 for Strategy B, and 370 for Strategy C. Which decision is the risky choice under Maximax, the safe choice under Maximin, and the balanced choice under Expected Value?

		\subsection*{C2}
		\begin{center}
			\begin{tabular}{l p{0.74\linewidth}}
				\toprule
				Decision Alternatives & Strategy A (solar), Strategy B (wind), Strategy C (balanced) \\
				States of Nature & Demand strong or weak with cost conditions favorable or unfavorable \\
				Events & Realized demand level paired with operating cost conditions \\
				Consequences & Profit in thousands of dollars from revenue minus operating costs \\
				\bottomrule
			\end{tabular}
		\end{center}

		\subsection*{C3}
		Payoff = revenue $-$ cost, where revenue is determined by demand and cost is determined by operating conditions.
		\begin{center}
			\begin{minipage}[t]{0.48\linewidth}
				\textit{Revenue parameters by strategy}
				\begin{center}
					\begin{tabular}{l c c}
						\toprule
						Strategy & Strong & Weak \\
						\midrule
						Probabilities & 0.55 & 0.45 \\
						Strategy A & 680 & 360 \\
						Strategy B & 620 & 380 \\
						Strategy C & 650 & 370 \\
						\bottomrule
					\end{tabular}
				\end{center}
			\end{minipage}
			\hfill
			\begin{minipage}[t]{0.48\linewidth}
				\textit{Cost parameters by strategy}
				\begin{center}
					\begin{tabular}{l c c}
						\toprule
						Strategy & Favorable & Unfavorable \\
						\midrule
						Probabilities & 0.60 & 0.40 \\
						Strategy A & 300 & 380 \\
						Strategy B & 280 & 360 \\
						Strategy C & 290 & 370 \\
						\bottomrule
					\end{tabular}
				\end{center}
			\end{minipage}
		\end{center}
		Profit is computed as Revenue minus Cost for each alternative and state.
		\begin{center}
			\textit{Profit parameters by strategy}
			\begin{tabular}{l p{0.17\linewidth} p{0.2\linewidth} p{0.2\linewidth} p{0.2\linewidth}}
				\toprule
				State of nature & Probability & Strategy A & Strategy B & Strategy C \\
				\midrule
				Strong demand, favorable cost &
				$\begin{array}{l}
					0.55\cdot 0.60\\
					= 0.33
				\end{array}$ &
				$\begin{array}{l}
					680 - 300\\
					= 380
				\end{array}$ &
				$\begin{array}{l}
					620 - 280\\
					= 340
				\end{array}$ &
				$\begin{array}{l}
					650 - 290\\
					= 360
				\end{array}$ \\
				Strong demand, unfavorable cost &
				$\begin{array}{l}
					0.55\cdot 0.40\\
					= 0.22
				\end{array}$ &
				$\begin{array}{l}
					680 - 380\\
					= 300
				\end{array}$ &
				$\begin{array}{l}
					620 - 360\\
					= 260
				\end{array}$ &
				$\begin{array}{l}
					650 - 370\\
					= 280
				\end{array}$ \\
				Weak demand, favorable cost &
				$\begin{array}{l}
					0.45\cdot 0.60\\
					= 0.27
				\end{array}$ &
				$\begin{array}{l}
					360 - 300\\
					= 60
				\end{array}$ &
				$\begin{array}{l}
					380 - 280\\
					= 100
				\end{array}$ &
				$\begin{array}{l}
					370 - 290\\
					= 80
				\end{array}$ \\
				Weak demand, unfavorable cost &
				$\begin{array}{l}
					0.45\cdot 0.40\\
					= 0.18
				\end{array}$ &
				$\begin{array}{l}
					360 - 380\\
					= -20
				\end{array}$ &
				$\begin{array}{l}
					380 - 360\\
					= 20
				\end{array}$ &
				$\begin{array}{l}
					370 - 370\\
					= 0
				\end{array}$ \\
				\bottomrule
		\end{tabular}
		\end{center}
		\begin{center}
			\textit{Payoff table} \\
			\begin{tabular}{l c c c c}
				\toprule
				State of nature & Probability & Strategy A & Strategy B & Strategy C \\
				\midrule
				Strong demand, favorable cost & 0.33 & 380 & 340 & 360 \\
				Strong demand, unfavorable cost & 0.22 & 300 & 260 & 280 \\
				Weak demand, favorable cost & 0.27 & 60 & 100 & 80 \\
				Weak demand, unfavorable cost & 0.18 & -20 & 20 & 0 \\
				\bottomrule
		\end{tabular}
		\end{center}

		\subsection*{C4}
		\[
		\begin{aligned}
		\max(\text{Strategy A}) &= \max\{380,300,60,-20\} = 380,\\
		\max(\text{Strategy B}) &= \max\{340,260,100,20\} = 340,\\
		\max(\text{Strategy C}) &= \max\{360,280,80,0\} = 360.
		\end{aligned}
		\]
		\[
		\max\{\max(\text{Strategy A}), \max(\text{Strategy B}), \max(\text{Strategy C})\}
		= \max\{380, 340, 360\}
		= 380.
		\]
		The Maximax choice is Strategy A with 380. The criterion is \emph{risky} because it focuses only on the best possible payoff and ignores how likely it is.

		\subsection*{C5}
		\[
		\begin{aligned}
		\min(\text{Strategy A}) &= \min\{380,300,60,-20\} = -20,\\
		\min(\text{Strategy B}) &= \min\{340,260,100,20\} = 20,\\
		\min(\text{Strategy C}) &= \min\{360,280,80,0\} = 0.
		\end{aligned}
		\]
		\[
		\max\{\min(\text{Strategy A}), \min(\text{Strategy B}), \min(\text{Strategy C})\}
		= \max\{-20, 20, 0\}
		= 20.
		\]
		The Maximin choice is Strategy B with 20 because it has the least severe worst case. This is \emph{safe} because it protects against the worst outcome.

		\subsection*{C1}
		Expected values use the combined state probabilities.
		\[
		\begin{aligned}
		EV_A &= 0.33(380)+0.22(300)+0.27(60)+0.18(-20)\\
		&= 125.4+66+16.2-3.6=204,\\
		EV_B &= 0.33(340)+0.22(260)+0.27(100)+0.18(20)\\
		&= 112.2+57.2+27+3.6=200,\\
		EV_C &= 0.33(360)+0.22(280)+0.27(80)+0.18(0)\\
		&= 118.8+61.6+21.6+0=202.
		\end{aligned}
		\]
		The expected value criterion chooses Strategy A because $204$ is the largest expected profit. This differs from the Maximin choice and matches the Maximax choice, so the expected value gives a balanced decision using the probabilities.

		\item
		\subsection*{Problem description}
		A community fitness hub must choose a membership model for the next year. Model A is a weekday focus and Model B is a family pass. Demand can be high, medium, or low with probabilities $0.40$, $0.35$, and $0.25$. Staffing costs can be favorable or unfavorable with probabilities $0.55$ and $0.45$, and these costs are independent of demand. Expected membership counts depend on demand and model. Under high demand Model A has 700 members and Model B has 650 members. Under medium demand Model A has 520 members and Model B has 480 members. Under low demand Model A has 360 members and Model B has 320 members. Membership fees are $38$ for Model A and $42$ for Model B per member per year. Variable staffing cost per member is $20$ when costs are favorable and $28$ when costs are unfavorable. Which decision is the risky choice under Maximax, the safe choice under Maximin, and the balanced choice under Expected Value?

		\subsection*{C2}
		\begin{center}
			\begin{tabular}{l p{0.74\linewidth}}
				\toprule
				Decision Alternatives & Model A (weekday), Model B (family) \\
				States of Nature & Demand high, medium, low with staffing costs favorable or unfavorable \\
				Events & Realized demand level paired with staffing cost conditions \\
				Consequences & Annual profit in dollars from membership fees minus staffing costs \\
				\bottomrule
			\end{tabular}
		\end{center}

		\subsection*{C3}
		Payoff = revenue $-$ cost, where revenue = (members $\times$ fee) and cost = (members $\times$ variable cost).
		\begin{center}
			\textit{Members per model by demand} \\
			\begin{tabular}{l c c c}
				\toprule
				Model & High & Medium & Low \\
				\midrule
				Model A & 700 & 520 & 360 \\
				Model B & 650 & 480 & 320 \\
				\bottomrule
		\end{tabular}
		\end{center}
		\begin{center}
			\begin{tabular}{@{}l@{\hspace{0.06\linewidth}}l@{}}
				\textit{Cost parameters by staffing condition} & \textit{Revenue parameters by member} \\
				\begin{tabular}{l c c}
					\toprule
					Condition & Favorable & Unfavorable \\
					\midrule
					Variable cost & 20 & 28 \\
					\bottomrule
				\end{tabular}
				&
				\begin{tabular}{l c}
					\toprule
					Model & Fee per member \\
					\midrule
					Model A & 38 \\
					Model B & 42 \\
					\bottomrule
				\end{tabular}
			\end{tabular}
		\end{center}

		\begin{center}
			\textit{Revenue by demand} \\
			\begin{tabular}{l c c}
				\toprule
				State of nature & Model A revenue & Model B revenue \\
				\midrule
				High demand &
				$\begin{array}{l}
					700\cdot 38\\
					= 26600
				\end{array}$ &
				$\begin{array}{l}
					650\cdot 42\\
					= 27300
				\end{array}$ \\
				Medium demand &
				$\begin{array}{l}
					520\cdot 38\\
					= 19760
				\end{array}$ &
				$\begin{array}{l}
					480\cdot 42\\
					= 20160
				\end{array}$ \\
				Low demand &
				$\begin{array}{l}
					360\cdot 38\\
					= 13680
				\end{array}$ &
				$\begin{array}{l}
					320\cdot 42\\
					= 13440
				\end{array}$ \\
				\bottomrule
		\end{tabular}
		\end{center}
		\begin{center}
			\textit{Cost by demand and staffing condition} \\
			\begin{tabular}{l c c}
				\toprule
				State of nature & Model A cost & Model B cost \\
				\midrule
				High demand, favorable cost &
				$\begin{array}{l}
					700\cdot 20\\
					= 14000
				\end{array}$ &
				$\begin{array}{l}
					650\cdot 20\\
					= 13000
				\end{array}$ \\
				High demand, unfavorable cost &
				$\begin{array}{l}
					700\cdot 28\\
					= 19600
				\end{array}$ &
				$\begin{array}{l}
					650\cdot 28\\
					= 18200
				\end{array}$ \\
				Medium demand, favorable cost &
				$\begin{array}{l}
					520\cdot 20\\
					= 10400
				\end{array}$ &
				$\begin{array}{l}
					480\cdot 20\\
					= 9600
				\end{array}$ \\
				Medium demand, unfavorable cost &
				$\begin{array}{l}
					520\cdot 28\\
					= 14560
				\end{array}$ &
				$\begin{array}{l}
					480\cdot 28\\
					= 13440
				\end{array}$ \\
				Low demand, favorable cost &
				$\begin{array}{l}
					360\cdot 20\\
					= 7200
				\end{array}$ &
				$\begin{array}{l}
					320\cdot 20\\
					= 6400
				\end{array}$ \\
				Low demand, unfavorable cost &
				$\begin{array}{l}
					360\cdot 28\\
					= 10080
				\end{array}$ &
				$\begin{array}{l}
					320\cdot 28\\
					= 8960
				\end{array}$ \\
				\bottomrule
		\end{tabular}
		\end{center}
		Profit is computed as Revenue minus Cost for each alternative and state.
		\begin{center}
		\textit{Profit parameters by model} \\
		\begin{tabular}{p{0.24\linewidth} p{0.24\linewidth} p{0.24\linewidth} p{0.24\linewidth}}
			\toprule
			State of nature & Model A & Model B & Probabilities \\
			\midrule
			\parbox[t]{\linewidth}{High demand,\\ favorable cost} &
			$\begin{array}{l}
				26600 - 14000\\
				= 12600
			\end{array}$ &
			$\begin{array}{l}
				27300 - 13000\\
				= 14300
			\end{array}$ &
			$\begin{array}{l}
				0.40\cdot 0.55\\
				= 0.22
			\end{array}$ \\
			
			\parbox[t]{\linewidth}{High demand,\\ unfavorable cost} &
			$\begin{array}{l}
				26600 - 19600\\
				= 7000
			\end{array}$ &
			$\begin{array}{l}
				27300 - 18200\\
				= 9100
			\end{array}$ &
			$\begin{array}{l}
				0.40\cdot 0.45\\
				= 0.18
			\end{array}$ \\
			
			\parbox[t]{\linewidth}{Medium demand,\\ favorable cost} &
			$\begin{array}{l}
				19760 - 10400\\
				= 9360
			\end{array}$ &
			$\begin{array}{l}
				20160 - 9600\\
				= 10560
			\end{array}$ &
			$\begin{array}{l}
				0.35\cdot 0.55\\
				= 0.1925
			\end{array}$ \\
			
			\parbox[t]{\linewidth}{Medium demand,\\ unfavorable cost} &
			$\begin{array}{l}
				19760 - 14560\\
				= 5200
			\end{array}$ &
			$\begin{array}{l}
				20160 - 13440\\
				= 6720
			\end{array}$ &
			$\begin{array}{l}
				0.35\cdot 0.45\\
				= 0.1575
			\end{array}$ \\
			
			\parbox[t]{\linewidth}{Low demand,\\ favorable cost} &
			$\begin{array}{l}
				13680 - 7200\\
				= 6480
			\end{array}$ &
			$\begin{array}{l}
				13440 - 6400\\
				= 7040
			\end{array}$ &
			$\begin{array}{l}
				0.25\cdot 0.55\\
				= 0.1375
			\end{array}$ \\
			
			\parbox[t]{\linewidth}{Low demand,\\ unfavorable cost} &
			$\begin{array}{l}
				13680 - 10080\\
				= 3600
			\end{array}$ &
			$\begin{array}{l}
				13440 - 8960\\
				= 4480
			\end{array}$ &
			$\begin{array}{l}
				0.25\cdot 0.45\\
				= 0.1125
			\end{array}$ \\
			\bottomrule
		\end{tabular}
	\end{center}
	Combined probabilities are $0.40\cdot 0.55=0.22$, $0.40\cdot 0.45=0.18$, $0.35\cdot 0.55=0.1925$, $0.35\cdot 0.45=0.1575$, $0.25\cdot 0.55=0.1375$, $0.25\cdot 0.45=0.1125$.
	\begin{center}
		\begin{tabular}{l c c c}
			\toprule
			State of nature & Probability & Model A & Model B \\
			\midrule
			High demand, favorable cost & 0.22 & 12600 & 14300 \\
			High demand, unfavorable cost & 0.18 & 7000 & 9100 \\
			Medium demand, favorable cost & 0.1925 & 9360 & 10560 \\
			Medium demand, unfavorable cost & 0.1575 & 5200 & 6720 \\
			Low demand, favorable cost & 0.1375 & 6480 & 7040 \\
			Low demand, unfavorable cost & 0.1125 & 3600 & 4480 \\
			\bottomrule
	\end{tabular}
	\end{center}

		\subsection*{C4}
		\[
		\begin{aligned}
		\max(\text{Model A}) &= \max\{12600,7000,9360,5200,6480,3600\} = 12600,\\
		\max(\text{Model B}) &= \max\{14300,9100,10560,6720,7040,4480\} = 14300.
		\end{aligned}
		\]
		\[
		\max\{\max(\text{Model A}), \max(\text{Model B})\}
		= \max\{12600, 14300\}
		= 14300.
		\]
		The Maximax choice is Model B with 14300. The criterion is \emph{risky} because it focuses only on the best possible payoff and ignores how likely it is.

		\subsection*{C5}
		\[
		\begin{aligned}
		\min(\text{Model A}) &= \min\{12600,7000,9360,5200,6480,3600\} = 3600,\\
		\min(\text{Model B}) &= \min\{14300,9100,10560,6720,7040,4480\} = 4480.
		\end{aligned}
		\]
		\[
		\max\{\min(\text{Model A}), \min(\text{Model B})\}
		= \max\{3600, 4480\}
		= 4480.
		\]
		The Maximin choice is Model B with 4480 because it has the least severe worst case. This is \emph{safe} because it protects against the worst outcome.

		\subsection*{C1}
		Expected values use the combined state probabilities.
		\[
		\begin{aligned}
		EV_A &= 0.22(12600)+0.18(7000)+0.1925(9360)+0.1575(5200)+0.1375(6480)+0.1125(3600)\\
		&= 2772+1260+1801.2+819+891+405=7948.8,\\
		EV_B &= 0.22(14300)+0.18(9100)+0.1925(10560)+0.1575(6720)+0.1375(7040)+0.1125(4480)\\
		&= 3146+1638+2032.8+1058.4+968+504=9347.2.
		\end{aligned}
		\]
		The expected value criterion chooses Model B because $9347.2$ is the largest expected profit. This aligns with the Maximax and Maximin outcomes for Model B, so the expected value gives the balanced decision using the probabilities.

		\item
		\subsection*{Problem description}
		A community fitness hub must choose a membership model for the next year. Model A is a weekday focus, Model B is a family pass, and Model C is an evening pass. Demand can be high or low with probabilities $0.60$ and $0.40$. Staffing costs can be favorable or unfavorable with probabilities $0.55$ and $0.45$, and these costs are independent of demand. Expected membership counts depend on demand and model. Under high demand Model A has 700 members, Model B has 650 members, and Model C has 760 members. Under low demand Model A has 360 members, Model B has 320 members, and Model C has 420 members. Membership fees are $38$ for Model A, $42$ for Model B, and $34$ for Model C per member per year. Variable staffing cost per member is $20$ when costs are favorable and $28$ when costs are unfavorable. Fixed annual costs are $8200$ for Model A, $9600$ for Model B, and $7000$ for Model C. Which decision is the risky choice under Maximax, the safe choice under Maximin, and the balanced choice under Expected Value?

		\subsection*{C2}
		\begin{center}
			\begin{tabular}{l p{0.74\linewidth}}
				\toprule
				Decision Alternatives & Model A (weekday), Model B (family), Model C (evening) \\
				States of Nature & Demand high or low with staffing costs favorable or unfavorable \\
				Events & Realized demand level paired with staffing cost conditions \\
				Consequences & Annual profit in dollars from membership fees minus staffing and fixed costs \\
				\bottomrule
			\end{tabular}
		\end{center}

		\subsection*{C3}
		Payoff = revenue $-$ cost, where revenue = (members $\times$ fee) and cost = (members $\times$ variable cost) $+$ fixed cost.
		\begin{center}
			\textit{Members per model by demand} \\
			\begin{tabular}{l c c}
				\toprule
				Model & High & Low \\
				\midrule
				Model A & 700 & 360 \\
				Model B & 650 & 320 \\
				Model C & 760 & 420 \\
				\bottomrule
		\end{tabular}
		\end{center}
		\begin{center}
			\begin{tabular}{@{}l@{\hspace{0.06\linewidth}}l@{}}
				\textit{Cost parameters by model} & \textit{Revenue parameters by member} \\
				\begin{tabular}{l c c c}
					\toprule
					Model & Favorable & Unfavorable & Fixed cost \\
					\midrule
					Model A & 20 & 28 & 8200 \\
					Model B & 20 & 28 & 9600 \\
					Model C & 20 & 28 & 7000 \\
					\bottomrule
				\end{tabular}
				&
				\begin{tabular}{l c}
					\toprule
					Model & Fee per member \\
					\midrule
					Model A & 38 \\
					Model B & 42 \\
					Model C & 34 \\
					\bottomrule
				\end{tabular}
			\end{tabular}
		\end{center}

		\begin{center}
			\textit{Revenue by demand} \\
			\begin{tabular}{l c c c}
				\toprule
				State of nature & Model A revenue & Model B revenue & Model C revenue \\
				\midrule
				High demand &
				$\begin{array}{l}
					700\cdot 38\\
					= 26600
				\end{array}$ &
				$\begin{array}{l}
					650\cdot 42\\
					= 27300
				\end{array}$ &
				$\begin{array}{l}
					760\cdot 34\\
					= 25840
				\end{array}$ \\
				Low demand &
				$\begin{array}{l}
					360\cdot 38\\
					= 13680
				\end{array}$ &
				$\begin{array}{l}
					320\cdot 42\\
					= 13440
				\end{array}$ &
				$\begin{array}{l}
					420\cdot 34\\
					= 14280
				\end{array}$ \\
				\bottomrule
		\end{tabular}
		\end{center}
		\begin{center}
			\textit{Cost by demand and staffing condition} \\
			\begin{tabular}{l c c c}
				\toprule
				State of nature & Model A cost & Model B cost & Model C cost \\
				\midrule
				High demand, favorable cost &
				$\begin{array}{l}
					700\cdot 20 + 8200\\
					= 22200
				\end{array}$ &
				$\begin{array}{l}
					650\cdot 20 + 9600\\
					= 22600
				\end{array}$ &
				$\begin{array}{l}
					760\cdot 20 + 7000\\
					= 22200
				\end{array}$ \\
				High demand, unfavorable cost &
				$\begin{array}{l}
					700\cdot 28 + 8200\\
					= 27800
				\end{array}$ &
				$\begin{array}{l}
					650\cdot 28 + 9600\\
					= 27800
				\end{array}$ &
				$\begin{array}{l}
					760\cdot 28 + 7000\\
					= 28280
				\end{array}$ \\
				Low demand, favorable cost &
				$\begin{array}{l}
					360\cdot 20 + 8200\\
					= 15400
				\end{array}$ &
				$\begin{array}{l}
					320\cdot 20 + 9600\\
					= 16000
				\end{array}$ &
				$\begin{array}{l}
					420\cdot 20 + 7000\\
					= 15400
				\end{array}$ \\
				Low demand, unfavorable cost &
				$\begin{array}{l}
					360\cdot 28 + 8200\\
					= 18280
				\end{array}$ &
				$\begin{array}{l}
					320\cdot 28 + 9600\\
					= 18560
				\end{array}$ &
				$\begin{array}{l}
					420\cdot 28 + 7000\\
					= 18760
				\end{array}$ \\
				\bottomrule
		\end{tabular}
		\end{center}
		Profit is computed as Revenue minus Cost for each alternative and state.
		\begin{center}
		\textit{Profit parameters by model} \\
		\begin{tabular}{p{0.19\linewidth} p{0.19\linewidth} p{0.19\linewidth} p{0.19\linewidth} p{0.19\linewidth}}
			\toprule
			State of nature & Model A & Model B & Model C & Probabilities \\
			\midrule
			\parbox[t]{\linewidth}{High demand,\\ favorable cost} &
			$\begin{array}{l}
				26600 - 22200\\
				= 4400
			\end{array}$ &
			$\begin{array}{l}
				27300 - 22600\\
				= 4700
			\end{array}$ &
			$\begin{array}{l}
				25840 - 22200\\
				= 3640
			\end{array}$ &
			$\begin{array}{l}
				0.60\cdot 0.55\\
				= 0.33
			\end{array}$ \\
			
			\parbox[t]{\linewidth}{High demand,\\ unfavorable cost} &
			$\begin{array}{l}
				26600 - 27800\\
				= -1200
			\end{array}$ &
			$\begin{array}{l}
				27300 - 27800\\
				= -500
			\end{array}$ &
			$\begin{array}{l}
				25840 - 28280\\
				= -2440
			\end{array}$ &
			$\begin{array}{l}
				0.60\cdot 0.45\\
				= 0.27
			\end{array}$ \\
			
			\parbox[t]{\linewidth}{Low demand,\\ favorable cost} &
			$\begin{array}{l}
				13680 - 15400\\
				= -1720
			\end{array}$ &
			$\begin{array}{l}
				13440 - 16000\\
				= -2560
			\end{array}$ &
			$\begin{array}{l}
				14280 - 15400\\
				= -1120
			\end{array}$ &
			$\begin{array}{l}
				0.40\cdot 0.55\\
				= 0.22
			\end{array}$ \\
			
			\parbox[t]{\linewidth}{Low demand,\\ unfavorable cost} &
			$\begin{array}{l}
				13680 - 18280\\
				= -4600
			\end{array}$ &
			$\begin{array}{l}
				13440 - 18560\\
				= -5120
			\end{array}$ &
			$\begin{array}{l}
				14280 - 18760\\
				= -4480
			\end{array}$ &
			$\begin{array}{l}
				0.40\cdot 0.45\\
				= 0.18
			\end{array}$ \\
			\bottomrule
		\end{tabular}
	\end{center}
	Combined probabilities are $0.60\cdot 0.55=0.33$, $0.60\cdot 0.45=0.27$, $0.40\cdot 0.55=0.22$, $0.40\cdot 0.45=0.18$.
	\begin{center}
		\begin{tabular}{l c c c c}
			\toprule
			State of nature & Probability & Model A & Model B & Model C \\
			\midrule
			High demand, favorable cost & 0.33 & 4400 & 4700 & 3640 \\
			High demand, unfavorable cost & 0.27 & -1200 & -500 & -2440 \\
			Low demand, favorable cost & 0.22 & -1720 & -2560 & -1120 \\
			Low demand, unfavorable cost & 0.18 & -4600 & -5120 & -4480 \\
			\bottomrule
	\end{tabular}
	\end{center}

		\subsection*{C4}
		\[
		\begin{aligned}
		\max(\text{Model A}) &= \max\{4400,-1200,-1720,-4600\} = 4400,\\
		\max(\text{Model B}) &= \max\{4700,-500,-2560,-5120\} = 4700,\\
		\max(\text{Model C}) &= \max\{3640,-2440,-1120,-4480\} = 3640.
		\end{aligned}
		\]
		\[
		\max\{\max(\text{Model A}), \max(\text{Model B}), \max(\text{Model C})\}
		= \max\{4400, 4700, 3640\}
		= 4700.
		\]
		The Maximax choice is Model B with 4700. The criterion is \emph{risky} because it focuses only on the best possible payoff and ignores how likely it is.

		\subsection*{C5}
		\[
		\begin{aligned}
		\min(\text{Model A}) &= \min\{4400,-1200,-1720,-4600\} = -4600,\\
		\min(\text{Model B}) &= \min\{4700,-500,-2560,-5120\} = -5120,\\
		\min(\text{Model C}) &= \min\{3640,-2440,-1120,-4480\} = -4480.
		\end{aligned}
		\]
		\[
		\max\{\min(\text{Model A}), \min(\text{Model B}), \min(\text{Model C})\}
		= \max\{-4600, -5120, -4480\}
		= -4480.
		\]
		The Maximin choice is Model C with $-4480$ because it has the least severe worst case loss. This is \emph{safe} because it protects against the worst outcome.

		\subsection*{C1}
		Expected values use the combined state probabilities.
		\[
		\begin{aligned}
		EV_A &= 0.33(4400)+0.27(-1200)+0.22(-1720)+0.18(-4600)\\
		&= 1452-324-378.4-828=-78.4,\\
		EV_B &= 0.33(4700)+0.27(-500)+0.22(-2560)+0.18(-5120)\\
		&= 1551-135-563.2-921.6=-68.8,\\
		EV_C &= 0.33(3640)+0.27(-2440)+0.22(-1120)+0.18(-4480)\\
		&= 1201.2-658.8-246.4-806.4=-510.4.
		\end{aligned}
		\]
		The expected value criterion chooses Model B because $-68.8$ is the largest expected profit. This matches the Maximax choice and differs from the Maximin choice, so the expected value gives a balanced decision using the probabilities.
	\end{ExamProblems}
\end{document}
