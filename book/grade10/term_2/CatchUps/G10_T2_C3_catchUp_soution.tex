\makeatletter
\def\input@path{{./}{../}{../../}{preamble/}{../preamble/}{../../preamble/}}
\makeatother
% ----------------------------------------------------------
% GENERAL 

% File
\documentclass[11pt]{book}

% Margins
\usepackage[margin=1in]{geometry}

\linespread{1.2}            % Line spacing
\usepackage[utf8]{inputenc}
\usepackage[T1]{fontenc}
\usepackage{lmodern}
\usepackage{microtype}
\setlength{\parindent}{0pt}
\setlength{\parskip}{6pt}
\usepackage{booktabs}

% ----------------------------------------------------------
% TABLES
\usepackage{multicol}
\usepackage{longtable} 
\usepackage{array}
\usepackage{booktabs}
\usepackage{tabularx}
\usepackage{multirow}

% ----------------------------------------------------------
% MATHEMATICS
\usepackage{amsmath}
\usepackage{amssymb}
\usepackage{amsfonts}
\usepackage{mathtools}

% ----------------------------------------------------------
% HIDDEN CONTENT
\usepackage{ifthen}
% Define a boolean switch
\newboolean{explicaciones}
% Set the boolean switch to true or false
% Change to true to show the content

% Explanations
\newcommand{\explicacion}[2]{
	\ifthenelse{\boolean{explicaciones}}{#1}{#2}
}
\newcommand{\mostrarExplicaciones}[1]{\setboolean{explicaciones}{#1}}

% ----------------------------------------------------------
% NUMBERING

\usepackage{fancyhdr}
\pagestyle{empty} % Ensures the entire document has no page numbers

\usepackage{tocloft}
\renewcommand{\cftdot}{} % Remove dots for sections, if any
\renewcommand{\cftsecleader}{\cftdotfill{\cftdotsep}} % Remove dots for sections, if any
\cftpagenumbersoff{section} % Removes page numbers from sections
\cftpagenumbersoff{subsection} % Removes page numbers from subsections

% ----------------------------------------------------------
% IMAGES 

% General settings
\usepackage{graphicx}       % Insert images
\usepackage{float}          % Position images
% \usepackage{subfigure}      % Subfigures
\graphicspath{{imgs}}       % Image location
\usepackage{subcaption}     % Subfigures II
\usepackage{verbatim}

% Figures
\usepackage{tikz}
\usetikzlibrary{arrows.meta,positioning,trees}

% Colors
\usepackage{xcolor}     
\definecolor{popUp}{HTML}{666666}
\definecolor{popUpIn}{HTML}{CED9E0}
\definecolor{backgroundC}{HTML}{D0E8F2}
\definecolor{backgroundCC}{HTML}{FFFFFF}
\definecolor{borders}{HTML}{8c120d}
\definecolor{padding}{HTML}{77D0D7}
\definecolor{links}{HTML}{CC6F5F}

% ----------------------------------------------------------
% FRAMES

% General settings
\usepackage{tcolorbox}
\usepackage{adjustbox}          % Adjusted frame  
\setlength{\fboxrule}{3pt}  % Line width
\setlength{\fboxsep}{3pt}   % Box padding

% General frames
\usepackage{mdframed}   

\mdfdefinestyle{estiloGeneral}{    % General style
	linecolor=black,
	linewidth=1.5pt,
	roundcorner=10pt,
	backgroundcolor=backgroundC,
	innerbottommargin=0pt
}
\mdfdefinestyle{code}{          % Code style
	linecolor=black,
	linewidth=1.5pt,
	roundcorner=10pt,
	backgroundcolor=darkgray!10,
	innerbottommargin=0pt
}

% Image frame
\newtcbox{\fboxC}{
	colback=backgroundC,
	colframe=popUp,
	arc=10pt,
	boxrule=3pt,
	boxsep=0pt, % Change the padding here
	nobeforeafter
}

% ----------------------------------------------------------
% PAGE SETTINGS

% Background 
\newcommand{\background}[0]{\begin{tikzpicture}[remember picture,overlay]
		\fill[backgroundC] (-2,2) rectangle (25cm, -550);
\end{tikzpicture}}

\newcommand{\backgroundC}[0]{\begin{tikzpicture}[remember picture,overlay]
		\fill[backgroundCC] (-2,2) rectangle (25cm, -550);
\end{tikzpicture}}

% Page width 
\newcommand{\anchoPag}[0]{20cm}

% ----------------------------------------------------------
% FONT

% General
\usepackage{tgbonum}        % Font
\usepackage{listings}       % Code typesetting
\usepackage[spanish]{babel} % Load Spanish
\selectlanguage{spanish}    % Select Spanish
\usepackage{enumitem}
\usepackage{bookmark}

\setlist[itemize]{leftmargin=1.2em, itemsep=0.35em, topsep=0.35em}

% --- Table helpers ---
\newcolumntype{L}[1]{>{\raggedright\arraybackslash}p{#1}}
\newcolumntype{Y}{>{\raggedright\arraybackslash}X}
\newcolumntype{C}{>{\centering\arraybackslash}X}
\renewcommand{\arraystretch}{1.1}

% Python style
\lstdefinestyle{python}{
	language=Python,
	basicstyle=\ttfamily\small,
	commentstyle=\color{green!50!black},
	keywordstyle=\color{blue},
	numberstyle=\tiny\color{gray},
	numbers=left,
	morekeywords={>, <},
	breakatwhitespace=false,
	showstringspaces=false,
	showtabs=false,
	showspaces=false
}

% ----------------------------------------------------------
% HYPERLINKS

% General
\usepackage{hyperref}       
\hypersetup{
	colorlinks=true,
	linkcolor=links,
	filecolor=magenta,      
	urlcolor=brown,
}

% Custom commands 

% Large link
\newcommand{\bigLink}[2]{\begin{center} \fboxC{\LARGE{\href{#1}{#2}}}\end{center}}

% Small link
\newcommand{\smallLink}[2]{\begin{center}\fboxC{\href{#1}{#2}}\end{center}}

% Bold link
\newcommand{\bfLink}[2]{\href{#1}{\textbf{#2}}}


% Small URL
\newcommand{\smallUrl}[1]{\begin{center}\fboxC{\url{#1}}\end{center}}


% ----------------------------------------------------------
% CUSTOM COMMANDS FOR FIGURES

\newcommand{\espacioImagenes}[0]{-1.2cm}

% Without frame
\newcommand{\fig}[3][\espacioImagenes]{
	\hspace*{#1}
	\centering
	\includegraphics[width=#2\textwidth]{#3}
}

% With frame
\newcommand{\ffig}[2]{\begin{figure}[h]
		\hspace*{\espacioImagenes}
		\centering
		\fbox{\includegraphics[width=#1\textwidth]{#2}}
\end{figure}}

% Hyperlink with frame
\newcommand{\hfig}[3]{\begin{figure}[h]
		\hspace*{-1.4cm}
		\centering
		\color{popUp}
		\fboxC{\href{#1}{\includegraphics[width=#2\textwidth]{#3}}}
	\end{figure}
}

% Hyperlink without frame
\newcommand{\hffig}[3]{\begin{figure}[h]
		\hspace*{-1.1cm}
		\centering
		\color{popUp}
		\href{#1}{\includegraphics[width=#2\textwidth]{#3}}
	\end{figure}
}

% ----------------------------------------------------------

% Start and Contents
\newcommand{\cuadro}[1]{
	\begin{mdframed}[style=estiloGeneral]
		#1 
	\end{mdframed}
}

% Explanation video image
\newcommand{\linkExplicacion}[1]{
	\hffig{#1}{0.5}{principal/videoExplicacion}
	\vspace{-0.5cm}
}

\newcommand{\subSecLink}[2]{
	\subsubsection*{\href{#1}{\textbf{#2}}}
}

% Spacing
\newcommand{\esp}[0]{\vspace{4mm}}

% Back to start
\newcommand{\secInicio}[0]{\begin{center}\hyperref[sec:inicio]{ 
			\includegraphics[width=0.1\textwidth]{principal/up}
	}\end{center}
}


\geometry{margin=0.85in}
\AtBeginDocument{\small}

\newcommand{\ExamNameField}{\noindent\textbf{Name:}\ \rule{0.7\linewidth}{0.4pt}\par\medskip}

\newcommand{\ExamTitleBlock}[3]{%
	\begin{center}
		\Large\textbf{#1}\\
		\textbf{#2}%
		\if\relax\detokenize{#3}\relax\else\\\textbf{#3}\fi
	\end{center}
	\vspace{0.5em}
}

\newcommand{\ExamSection}[1]{\par\medskip\textbf{#1}\par\smallskip}

\newenvironment{ExamCriteria}{%
	\begin{itemize}[leftmargin=1.6em, itemsep=0.3em, topsep=0.2em]
}{%
	\end{itemize}
}

\newenvironment{ExamProblems}{%
	\begin{enumerate}[label=\textbf{P\arabic*}, leftmargin=0pt, labelsep=0.6em, itemindent=2.2em, itemsep=0.8em]
}{%
	\end{enumerate}
}

\begin{document}
	\ExamTitleBlock{10th grade}{Term 2 Criterion 3 Catch-Up (Solutions)}{}
	
	\ExamSection{Problems}
	\begin{ExamProblems}
		\item
		\subsection*{Problem description}
		A neighborhood juice bar must choose one of two monthly operating formats: Format A (premium menu)
		or Format B (standard menu). Market demand for the month can be high or low.
		
		The projected payoff values are already net profits in thousands of dollars, so no separate revenue or cost calculation is required.
		Under high demand, the estimated profit is 96 for Format A and 78 for Format B. Under low demand, the estimated profit is 18 for Format A and 36 for Format B.
		
		Management must construct the payoff table using these direct profit entries for each format--demand combination.
		\subsection*{Given data}
		\begin{center}
			\begin{tabular}{l c c}
				\toprule
				Alternative & High demand & Low demand \\
				\midrule
				Format A & 96 & 18 \\
				Format B & 78 & 36 \\
				\bottomrule
			\end{tabular}
		\end{center}
		
		\subsection*{Solution}
		\textbf{Step 1: Identify alternatives and states of nature.} The alternatives are Format A and Format B. The states of nature are high demand and low demand.
		
		\textbf{Step 2: Compute revenue and/or costs if needed.} No additional computation is needed because profits are given directly.
		
		\textbf{Step 3: Compute payoff.} Since profit values are already provided, each value is directly the payoff.
		
		\textbf{Step 4: Final payoff table.}
		\begin{center}
			\begin{tabular}{l c c}
				\toprule
				Alternative & High demand & Low demand \\
				\midrule
				Format A & 96 & 18 \\
				Format B & 78 & 36 \\
				\bottomrule
			\end{tabular}
		\end{center}
		
		In conclusion, the C3 work is complete because the decision alternatives and demand states are identified, and the payoff table is correctly formed from the provided profit data.
		
		\item
		\subsection*{Problem description}
		A retail snack distributor is selecting one of three inventory strategies for a one-week city event: Strategy A (premium assortment),
		Strategy B (balanced assortment), or Strategy C (economy assortment). Demand can be strong, moderate, or weak.
		
		All numerical values are already estimated profits in thousands of dollars.
		For strong demand, profits are 140 (A), 122 (B), and 105 (C). For moderate demand, profits are 88 (A), 94 (B), and 76 (C).
		For weak demand, profits are 12 (A), 38 (B), and 49 (C).
		
		The firm must construct the payoff table directly from these profit outcomes for each strategy and demand state.
		\subsection*{Given data}
		\begin{center}
			\begin{tabular}{l c c c}
				\toprule
				Alternative & Strong demand & Moderate demand & Weak demand \\
				\midrule
				Strategy A & 140 & 88 & 12 \\
				Strategy B & 122 & 94 & 38 \\
				Strategy C & 105 & 76 & 49 \\
				\bottomrule
			\end{tabular}
		\end{center}
		
		\subsection*{Solution}
		\textbf{Step 1: Identify alternatives and states of nature.} Alternatives: A, B, and C. States: strong, moderate, and weak demand.
		
		\textbf{Step 2: Compute revenue and/or costs if needed.} No extra calculations are required because the data are already profits.
		
		\textbf{Step 3: Compute payoff.} Every profit entry is the payoff for its alternative--state pair.
		
		\textbf{Step 4: Final payoff table.}
		\begin{center}
			\begin{tabular}{l c c c}
				\toprule
				Alternative & Strong demand & Moderate demand & Weak demand \\
				\midrule
				Strategy A & 140 & 88 & 12 \\
				Strategy B & 122 & 94 & 38 \\
				Strategy C & 105 & 76 & 49 \\
				\bottomrule
			\end{tabular}
		\end{center}
		
		Therefore, the payoff table is correctly constructed from direct profit information across three alternatives and three states of nature.
		
		\item
		\subsection*{Problem description}
		A local garment manufacturer must choose between two production plans for the next season: Plan A (premium shirts)
		or Plan B (basic shirts). Market demand can be very high, high, medium, or low.
		
		For each plan, the per-unit amount is a profit contribution per shirt: Plan A earns \$18 per shirt and Plan B earns \$14 per shirt.
		The state-dependent quantities are expected shirt sales volumes:
		\begin{itemize}
			\item Very high demand: Plan A sells 9{,}000 shirts, Plan B sells 10{,}200 shirts.
			\item High demand: Plan A sells 7{,}200 shirts, Plan B sells 8{,}300 shirts.
			\item Medium demand: Plan A sells 5{,}000 shirts, Plan B sells 6{,}000 shirts.
			\item Low demand: Plan A sells 2{,}600 shirts, Plan B sells 3{,}400 shirts.
		\end{itemize}
		
		The manufacturer must first compute profit for each state using
		\(\text{Profit}=\text{Quantity sold}\times\text{Profit per unit}\), then construct the payoff table in thousands of dollars.
		\subsection*{Given data}
		\begin{center}
			\begin{tabular}{l c c c c}
				\toprule
				Alternative & Very high & High & Medium & Low \\
				\midrule
				Plan A quantity & 9{,}000 & 7{,}200 & 5{,}000 & 2{,}600 \\
				Plan B quantity & 10{,}200 & 8{,}300 & 6{,}000 & 3{,}400 \\
				\bottomrule
			\end{tabular}
		\end{center}
		Profit per unit: Plan A = \$18, Plan B = \$14.
		
		\subsection*{Solution}
		\textbf{Step 1: Identify alternatives and states of nature.} Alternatives are Plan A and Plan B. States are very high, high, medium, and low demand.
		
		\textbf{Step 2: Compute profits from quantity and unit margin.}
		\[
		\text{Profit} = \text{Quantity} \times \text{Profit per unit}
		\]
		\[
		\begin{aligned}
		\text{A, very high} &= 9{,}000 \times 18 = 162{,}000 = 162 \\
		\text{A, high} &= 7{,}200 \times 18 = 129{,}600 = 129.6 \\
		\text{A, medium} &= 5{,}000 \times 18 = 90{,}000 = 90 \\
		\text{A, low} &= 2{,}600 \times 18 = 46{,}800 = 46.8 \\
		\text{B, very high} &= 10{,}200 \times 14 = 142{,}800 = 142.8 \\
		\text{B, high} &= 8{,}300 \times 14 = 116{,}200 = 116.2 \\
		\text{B, medium} &= 6{,}000 \times 14 = 84{,}000 = 84 \\
		\text{B, low} &= 3{,}400 \times 14 = 47{,}600 = 47.6
		\end{aligned}
		\]
		
		\textbf{Step 3: Compute payoff.} These computed profits are the payoffs for each alternative--state combination.
		
		\textbf{Step 4: Final payoff table (thousand USD).}
		\begin{center}
			\begin{tabular}{l c c c c}
				\toprule
				Alternative & Very high & High & Medium & Low \\
				\midrule
				Plan A & 162 & 129.6 & 90 & 46.8 \\
				Plan B & 142.8 & 116.2 & 84 & 47.6 \\
				\bottomrule
			\end{tabular}
		\end{center}
		
		In summary, the payoff table is built by converting quantity data into profits through quantity times unit margin under four demand states.
		
		\item
		\subsection*{Problem description}
		A regional publisher must choose among three magazine distribution models for the coming month: Model A (city kiosks),
		Model B (subscription contracts), or Model C (mixed channel). Demand can be high or low.
		
		For each model, the fixed component is a monthly lump-sum profit contribution, and the per-copy amount is a variable contribution per copy sold.
		Model A: fixed component \$24{,}000 and contribution \$6.5 per copy.
		Model B: fixed component \$15{,}000 and contribution \$5.8 per copy.
		Model C: fixed component \$19{,}000 and contribution \$6.1 per copy.
		
		The quantity values are expected copies sold under each demand state:
		High demand: A = 18{,}000, B = 21{,}000, C = 19{,}500.
		Low demand: A = 9{,}500, B = 12{,}000, C = 10{,}700.
		
		Management must compute profit for each alternative using
		\(\text{Profit}=\text{Fixed component}+(\text{Quantity}\times\text{Contribution per copy})\), and then construct the payoff table in thousands of dollars.
		\subsection*{Given data}
		\begin{center}
			\begin{tabular}{l c c}
				\toprule
				Alternative & High demand copies & Low demand copies \\
				\midrule
				Model A & 18{,}000 & 9{,}500 \\
				Model B & 21{,}000 & 12{,}000 \\
				Model C & 19{,}500 & 10{,}700 \\
				\bottomrule
			\end{tabular}
		\end{center}
		
		\subsection*{Solution}
		\textbf{Step 1: Identify alternatives and states of nature.} Alternatives are Models A, B, and C. States are high demand and low demand.
		
		\textbf{Step 2: Compute profits with fixed plus variable components.}
		\[
		\text{Profit} = \text{Fixed component} + (\text{Quantity} \times \text{Contribution per copy})
		\]
		\[
		\begin{aligned}
		\text{A, high} &= 24{,}000 + (18{,}000 \times 6.5) = 141{,}000 = 141 \\
		\text{A, low} &= 24{,}000 + (9{,}500 \times 6.5) = 85{,}750 = 85.75 \\
		\text{B, high} &= 15{,}000 + (21{,}000 \times 5.8) = 136{,}800 = 136.8 \\
		\text{B, low} &= 15{,}000 + (12{,}000 \times 5.8) = 84{,}600 = 84.6 \\
		\text{C, high} &= 19{,}000 + (19{,}500 \times 6.1) = 137{,}950 = 137.95 \\
		\text{C, low} &= 19{,}000 + (10{,}700 \times 6.1) = 84{,}270 = 84.27
		\end{aligned}
		\]
		
		\textbf{Step 3: Compute payoff.} Each computed profit is the payoff.
		
		\textbf{Step 4: Final payoff table (thousand USD).}
		\begin{center}
			\begin{tabular}{l c c}
				\toprule
				Alternative & High demand & Low demand \\
				\midrule
				Model A & 141 & 85.75 \\
				Model B & 136.8 & 84.6 \\
				Model C & 137.95 & 84.27 \\
				\bottomrule
			\end{tabular}
		\end{center}
		
		Hence, the C3 objective is satisfied by explicitly applying the fixed-plus-variable formula and then organizing the resulting payoffs in table form.
		
		\item
		\subsection*{Problem description}
		A commercial printing firm must choose between two production contracts for a peak exam-season cycle: Contract A or Contract B.
		Demand may be strong, normal, or weak.
		
		For Contract A, the listed numbers are already profits in thousands of dollars: 112 under strong demand, 74 under normal demand, and 26 under weak demand.
		For Contract B, profit is not given directly and must be computed from a lump-sum fixed component plus a variable margin per booklet sold:
		\[
		\text{Profit}_B = 20{,}000 + (\text{Booklets sold} \times 4.8)
		\]
		For Contract B, the quantity values are expected booklet sales: 18{,}000 (strong), 12{,}000 (normal), and 7{,}500 (weak).
		
		The firm must compute Contract B profits first, then construct the payoff table in thousands of dollars with all profits arranged by demand state.
		\subsection*{Given data}
		\begin{center}
			\begin{tabular}{l c c c}
				\toprule
				Alternative & Strong & Normal & Weak \\
				\midrule
				Contract A (direct profit) & 112 & 74 & 26 \\
				Contract B booklets sold & 18{,}000 & 12{,}000 & 7{,}500 \\
				\bottomrule
			\end{tabular}
		\end{center}
		
		\subsection*{Solution}
		\textbf{Step 1: Identify alternatives and states of nature.} Alternatives are Contract A and Contract B. States are strong, normal, and weak demand.
		
		\textbf{Step 2: Compute revenue and/or costs if needed.} Contract A needs no computation (profit given directly). Contract B uses fixed plus variable profit:
		\[
		\begin{aligned}
		\text{B, strong} &= 20{,}000 + (18{,}000 \times 4.8) = 106{,}400 = 106.4 \\
		\text{B, normal} &= 20{,}000 + (12{,}000 \times 4.8) = 77{,}600 = 77.6 \\
		\text{B, weak} &= 20{,}000 + (7{,}500 \times 4.8) = 56{,}000 = 56
		\end{aligned}
		\]
		
		\textbf{Step 3: Compute payoff.} Payoffs are A: (112, 74, 26) and B: (106.4, 77.6, 56).
		
		\textbf{Step 4: Final payoff table (thousand USD).}
		\begin{center}
			\begin{tabular}{l c c c}
				\toprule
				Alternative & Strong & Normal & Weak \\
				\midrule
				Contract A & 112 & 74 & 26 \\
				Contract B & 106.4 & 77.6 & 56 \\
				\bottomrule
			\end{tabular}
		\end{center}
		
		This payoff table combines a direct-profit alternative and a computed fixed-plus-variable alternative, which demonstrates accurate C3 table construction.
		
		\item
		\subsection*{Problem description}
		A local bus operator must choose between Fleet A and Fleet B for a passenger transport contract.
		Revenue can be high or low, while fuel and operating costs can be low, medium, or high.
		
		All figures are given directly in thousands of dollars and are separated into revenue values and cost values.
		Revenue states:
		Fleet A revenue is 420 in the high-revenue state and 290 in the low-revenue state.
		Fleet B revenue is 450 in the high-revenue state and 300 in the low-revenue state.
		
		Cost states:
		Fleet A cost is 170 in the low-cost state, 205 in the medium-cost state, and 240 in the high-cost state.
		Fleet B cost is 185 in the low-cost state, 220 in the medium-cost state, and 260 in the high-cost state.
		
		The operator must construct the payoff table by combining each revenue state with each cost state and applying
		\(\text{Profit}=\text{Revenue}-\text{Cost}\).
		\subsection*{Given data tables}
		\begin{center}
			\begin{minipage}[t]{0.48\linewidth}
				\textit{Revenue data (given directly)}
				\begin{center}
					\begin{tabular}{l c c}
						\toprule
						Fleet & High revenue & Low revenue \\
						\midrule
						Fleet A & 420 & 290 \\
						Fleet B & 450 & 300 \\
						\bottomrule
					\end{tabular}
				\end{center}
			\end{minipage}
			\hfill
			\begin{minipage}[t]{0.48\linewidth}
				\textit{Cost data (given directly)}
				\begin{center}
					\begin{tabular}{l c c c}
						\toprule
						Fleet & Low cost & Medium cost & High cost \\
						\midrule
						Fleet A & 170 & 205 & 240 \\
						Fleet B & 185 & 220 & 260 \\
						\bottomrule
					\end{tabular}
				\end{center}
			\end{minipage}
		\end{center}
		
		\subsection*{Solution}
		\textbf{Step 1: Identify alternatives and states of nature.} Alternatives: Fleet A and Fleet B. Revenue states: high, low. Cost states: low, medium, high.
		
		\textbf{Step 2: Compute revenue and/or costs if needed.} Not required because both are given directly.
		
		\textbf{Step 3: Compute payoff using Profit = Revenue $-$ Cost.}
		\begin{center}
			\begin{tabular}{l c c}
				\toprule
				State combination & Fleet A payoff & Fleet B payoff \\
				\midrule
				High revenue, low cost & $420-170=250$ & $450-185=265$ \\
				High revenue, medium cost & $420-205=215$ & $450-220=230$ \\
				High revenue, high cost & $420-240=180$ & $450-260=190$ \\
				Low revenue, low cost & $290-170=120$ & $300-185=115$ \\
				Low revenue, medium cost & $290-205=85$ & $300-220=80$ \\
				Low revenue, high cost & $290-240=50$ & $300-260=40$ \\
				\bottomrule
			\end{tabular}
		\end{center}
		
		\textbf{Step 4: Final payoff table (thousand USD).}
		\begin{center}
			\begin{tabular}{l c c}
				\toprule
				State of nature & Fleet A & Fleet B \\
				\midrule
				High revenue, low cost & 250 & 265 \\
				High revenue, medium cost & 215 & 230 \\
				High revenue, high cost & 180 & 190 \\
				Low revenue, low cost & 120 & 115 \\
				Low revenue, medium cost & 85 & 80 \\
				Low revenue, high cost & 50 & 40 \\
				\bottomrule
			\end{tabular}
		\end{center}
		
		Therefore, this PA2-style table is correctly built from direct revenue and direct cost data across all combined states.
		
		\item
		\subsection*{Problem description}
		A contract catering company must choose among three lunch-service plans: Plan A, Plan B, or Plan C.
		Revenue depends on quantity sold and selling price per meal, while cost values are given directly.
		Revenue has two demand states (high and low), and cost has two states (low cost and high cost).
		
		Revenue-input quantities and prices are:
		Plan A sells 14{,}000 meals in high demand or 9{,}000 meals in low demand, at a price of \$12 per meal.
		Plan B sells 15{,}500 meals in high demand or 10{,}500 meals in low demand, at a price of \$11.5 per meal.
		Plan C sells 13{,}200 meals in high demand or 8{,}800 meals in low demand, at a price of \$12.4 per meal.
		
		Cost values are already provided in thousand USD:
		Plan A cost is 102 (low-cost state) and 128 (high-cost state);
		Plan B cost is 108 (low-cost state) and 134 (high-cost state);
		Plan C cost is 98 (low-cost state) and 124 (high-cost state).
		
		The firm must first compute revenue from
		\(\text{Revenue}=\text{Quantity}\times\text{Price per meal}\), then construct the payoff table using
		\(\text{Profit}=\text{Revenue}-\text{Cost}\) for every revenue--cost state pair.
		\subsection*{Given data tables}
		\begin{center}
			\begin{minipage}[t]{0.48\linewidth}
				\textit{Revenue-side quantities and prices}
				\begin{center}
					\begin{tabular}{l c c c}
						\toprule
						Plan & High qty & Low qty & Price per meal \\
						\midrule
						A & 14{,}000 & 9{,}000 & 12.0 \\
						B & 15{,}500 & 10{,}500 & 11.5 \\
						C & 13{,}200 & 8{,}800 & 12.4 \\
						\bottomrule
					\end{tabular}
				\end{center}
			\end{minipage}
			\hfill
			\begin{minipage}[t]{0.48\linewidth}
				\textit{Cost-side data (given directly)}
				\begin{center}
					\begin{tabular}{l c c}
						\toprule
						Plan & Low cost & High cost \\
						\midrule
						A & 102 & 128 \\
						B & 108 & 134 \\
						C & 98 & 124 \\
						\bottomrule
					\end{tabular}
				\end{center}
			\end{minipage}
		\end{center}
		
		\subsection*{Solution}
		\textbf{Step 1: Identify alternatives and states of nature.} Alternatives are A, B, C. Revenue states are high and low meal demand. Cost states are low and high cost.
		
		\textbf{Step 2: Compute revenue where required.}
		\[
		\text{Revenue} = \text{Quantity} \times \text{Price per meal}
		\]
		\[
		\begin{aligned}
		R_{A,H} &= 14{,}000\times12 = 168{,}000 = 168, & R_{A,L} &= 9{,}000\times12 = 108{,}000 = 108,\\
		R_{B,H} &= 15{,}500\times11.5 = 178{,}250 = 178.25, & R_{B,L} &= 10{,}500\times11.5 = 120{,}750 = 120.75,\\
		R_{C,H} &= 13{,}200\times12.4 = 163{,}680 = 163.68, & R_{C,L} &= 8{,}800\times12.4 = 109{,}120 = 109.12.
		\end{aligned}
		\]
		
		\textbf{Step 3: Compute payoff (Profit = Revenue $-$ Cost).}
		For example, for Plan B under high revenue and high cost,
		\(178.25-134=44.25\) thousand USD.
		
		\textbf{Step 4: Final payoff table (thousand USD).}
		\begin{center}
			\begin{tabular}{l c c c}
				\toprule
				State of nature & Plan A & Plan B & Plan C \\
				\midrule
				High revenue, low cost & 66 & 70.25 & 65.68 \\
				High revenue, high cost & 40 & 44.25 & 39.68 \\
				Low revenue, low cost & 6 & 12.75 & 11.12 \\
				Low revenue, high cost & -20 & -13.25 & -14.88 \\
				\bottomrule
			\end{tabular}
		\end{center}
		
		In conclusion, this payoff table correctly combines computed revenues with directly given costs for every revenue--cost state pair.
		
		\item
		\subsection*{Problem description}
		A private training provider must choose between Program A and Program B.
		Revenue is given directly for three enrollment-demand states (high, medium, and low), while costs must be computed from participant counts and unit cost.
		
		Revenue values (thousand USD) are:
		Program A revenue is 360 (high), 280 (medium), and 210 (low).
		Program B revenue is 390 (high), 300 (medium), and 230 (low).
		
		Cost inputs are:
		For Program A, participant quantity is 1{,}000 in the low-cost state and 1{,}300 in the high-cost state, with unit cost \$120 per participant.
		For Program B, participant quantity is 1{,}050 in the low-cost state and 1{,}350 in the high-cost state, with unit cost \$125 per participant.
		
		Management must first compute costs using
		\(\text{Cost}=\text{Quantity}\times\text{Unit cost}\), then construct the payoff table in thousands of dollars with
		\(\text{Profit}=\text{Revenue}-\text{Cost}\).
		\subsection*{Given data tables}
		\begin{center}
			\begin{minipage}[t]{0.48\linewidth}
				\textit{Revenue-side data (given directly)}
				\begin{center}
					\begin{tabular}{l c c c}
						\toprule
						Program & High & Medium & Low \\
						\midrule
						A & 360 & 280 & 210 \\
						B & 390 & 300 & 230 \\
						\bottomrule
					\end{tabular}
				\end{center}
			\end{minipage}
			\hfill
			\begin{minipage}[t]{0.48\linewidth}
				\textit{Cost-side quantity and unit cost}
				\begin{center}
					\begin{tabular}{l c c c}
						\toprule
						Program & Qty (low cost) & Qty (high cost) & Unit cost \\
						\midrule
						A & 1{,}000 & 1{,}300 & 120 \\
						B & 1{,}050 & 1{,}350 & 125 \\
						\bottomrule
					\end{tabular}
				\end{center}
			\end{minipage}
		\end{center}
		
		\subsection*{Solution}
		\textbf{Step 1: Identify alternatives and states of nature.} Alternatives are Program A and Program B. Revenue states are high, medium, low. Cost states are low-cost and high-cost.
		
		\textbf{Step 2: Compute costs where required.}
		\[
		\text{Cost} = \text{Quantity} \times \text{Unit cost}
		\]
		\[
		\begin{aligned}
		C_{A,L} &= 1{,}000\times120 = 120{,}000 = 120, & C_{A,H} &= 1{,}300\times120 = 156{,}000 = 156,\\
		C_{B,L} &= 1{,}050\times125 = 131{,}250 = 131.25, & C_{B,H} &= 1{,}350\times125 = 168{,}750 = 168.75.
		\end{aligned}
		\]
		
		\textbf{Step 3: Compute payoff (Profit = Revenue $-$ Cost).}
		For Program A with medium revenue and high cost: \(280-156=124\).
		For Program B with low revenue and high cost: \(230-168.75=61.25\).
		
		\textbf{Step 4: Final payoff table (thousand USD).}
		\begin{center}
			\begin{tabular}{l c c}
				\toprule
				State of nature & Program A & Program B \\
				\midrule
				High revenue, low cost & 240 & 258.75 \\
				High revenue, high cost & 204 & 221.25 \\
				Medium revenue, low cost & 160 & 168.75 \\
				Medium revenue, high cost & 124 & 131.25 \\
				Low revenue, low cost & 90 & 98.75 \\
				Low revenue, high cost & 54 & 61.25 \\
				\bottomrule
			\end{tabular}
		\end{center}
		
		Thus, the payoff table is correctly obtained by pairing each revenue state with each computed cost state.
		
		\item
		\subsection*{Problem description}
		A small electronics assembler must choose one of three sourcing contracts: Contract A, Contract B, or Contract C.
		For each contract, both revenue and cost must be calculated from quantities and unit rates.
		Revenue has two demand states (high and medium), and cost has three component-price states (low, medium, high).
		
		Revenue-side inputs are:
		\begin{itemize}
			\item Contract A: quantity sold is 3{,}200 units in high demand and 2{,}500 units in medium demand; selling price is \$210 per unit.
			\item Contract B: quantity sold is 3{,}450 units in high demand and 2{,}650 units in medium demand; selling price is \$205 per unit.
			\item Contract C: quantity sold is 3{,}300 units in high demand and 2{,}600 units in medium demand; selling price is \$208 per unit.
		\end{itemize}
		Cost-side inputs are:
		\begin{itemize}
			\item Contract A: cost quantity is 3{,}100 units in each cost state, with unit cost \$128 (low), \$136 (medium), and \$145 (high).
			\item Contract B: cost quantity is 3{,}300 units in each cost state, with unit cost \$124 (low), \$133 (medium), and \$142 (high).
			\item Contract C: cost quantity is 3{,}180 units in each cost state, with unit cost \$126 (low), \$135 (medium), and \$144 (high).
		\end{itemize}
		
		The assembler must compute revenue and cost first using
		\(\text{Revenue}=q\times p\) and \(\text{Cost}=q\times c\), then construct the payoff table in thousands of dollars from
		\(\text{Profit}=\text{Revenue}-\text{Cost}\).
		\subsection*{Given data tables}
		\begin{center}
			\begin{minipage}[t]{0.48\linewidth}
				\textit{Revenue computation inputs}
				\begin{center}
					\begin{tabular}{l c c c}
						\toprule
						Contract & Qty high & Qty medium & Price \\
						\midrule
						A & 3{,}200 & 2{,}500 & 210 \\
						B & 3{,}450 & 2{,}650 & 205 \\
						C & 3{,}300 & 2{,}600 & 208 \\
						\bottomrule
					\end{tabular}
				\end{center}
			\end{minipage}
			\hfill
			\begin{minipage}[t]{0.48\linewidth}
				\textit{Cost computation inputs}
				\begin{center}
					\begin{tabular}{l c c c}
						\toprule
						Contract & Unit cost low & Unit cost med & Unit cost high \\
						\midrule
						A ($q=3{,}100$) & 128 & 136 & 145 \\
						B ($q=3{,}300$) & 124 & 133 & 142 \\
						C ($q=3{,}180$) & 126 & 135 & 144 \\
						\bottomrule
					\end{tabular}
				\end{center}
			\end{minipage}
		\end{center}
		
		\subsection*{Solution}
		\textbf{Step 1: Identify alternatives and states of nature.} Alternatives are Contracts A, B, C. Revenue states are high and medium demand. Cost states are low, medium, and high cost.
		
		\textbf{Step 2: Compute revenue and costs.}
		\[
		\text{Revenue}=q\times p, \qquad \text{Cost}=q\times c
		\]
		Revenues (thousand USD):
		\[
		R_A(H)=672,\; R_A(M)=525,\; R_B(H)=707.25,\; R_B(M)=543.25,\; R_C(H)=686.4,\; R_C(M)=540.8.
		\]
		Costs (thousand USD):
		\[
		\begin{aligned}
		C_A(L)&=396.8, & C_A(M)&=421.6, & C_A(H)&=449.5,\\
		C_B(L)&=409.2, & C_B(M)&=438.9, & C_B(H)&=468.6,\\
		C_C(L)&=400.68, & C_C(M)&=429.3, & C_C(H)&=457.92.
		\end{aligned}
		\]
		
		\textbf{Step 3: Compute payoff (Profit = Revenue $-$ Cost).}
		For example, Contract C under medium demand and high cost gives
		\(540.8-457.92=82.88\) thousand USD.
		
		\textbf{Step 4: Final payoff table (thousand USD).}
		\begin{center}
			\begin{tabular}{l c c c}
				\toprule
				State of nature & Contract A & Contract B & Contract C \\
				\midrule
				High revenue, low cost & 275.2 & 298.05 & 285.72 \\
				High revenue, medium cost & 250.4 & 268.35 & 257.10 \\
				High revenue, high cost & 222.5 & 238.65 & 228.48 \\
				Medium revenue, low cost & 128.2 & 134.05 & 140.12 \\
				Medium revenue, medium cost & 103.4 & 104.35 & 111.50 \\
				Medium revenue, high cost & 75.5 & 74.65 & 82.88 \\
				\bottomrule
			\end{tabular}
		\end{center}
		
		In conclusion, this problem demonstrates full C3 construction when both revenue and costs must first be computed from unit-rate models.
		
		\item
		\subsection*{Problem description}
		An agricultural exporter must choose among three logistics plans: Plan X, Plan Y, or Plan Z.
		Both revenue and cost include a fixed component plus a variable component linked to activity volume.
		Revenue has three demand states (high, medium, low), and cost has two operating states (normal and disrupted).
		
		Revenue models are:
		\[
		\begin{aligned}
		R_X &= 150{,}000 + (q\times52),\\
		R_Y &= 130{,}000 + (q\times55),\\
		R_Z &= 145{,}000 + (q\times53).
		\end{aligned}
		\]
		In these formulas, the fixed terms (150{,}000, 130{,}000, 145{,}000) are fixed revenue components, and 52, 55, and 53 are variable revenue per unit.
		Demand-state quantities are:
		X: high 9{,}000, medium 7{,}000, low 5{,}000;
		Y: high 8{,}700, medium 6{,}900, low 5{,}300;
		Z: high 8{,}900, medium 6{,}950, low 5{,}150.
		
		Cost models are:
		\[
		\begin{aligned}
		C_X &= 120{,}000 + (n\times27),\\
		C_Y &= 105{,}000 + (n\times29),\\
		C_Z &= 112{,}000 + (n\times28).
		\end{aligned}
		\]
		Here, the fixed terms (120{,}000, 105{,}000, 112{,}000) are fixed costs, and 27, 29, and 28 are variable cost per unit of activity.
		Cost-state activity quantities are:
		X: normal 8{,}200, disrupted 9{,}100;
		Y: normal 8{,}000, disrupted 9{,}000;
		Z: normal 8{,}100, disrupted 9{,}050.
		
		The exporter must compute revenue and cost for each state first, then construct the payoff table in thousands of dollars using
		\(\text{Profit}=\text{Revenue}-\text{Cost}\).
		\subsection*{Given data tables}
		\begin{center}
			\begin{minipage}[t]{0.48\linewidth}
				\textit{Revenue side (fixed + variable)}
				\begin{center}
					\begin{tabular}{l c c c}
						\toprule
						Plan & High $q$ & Medium $q$ & Low $q$ \\
						\midrule
						X & 9{,}000 & 7{,}000 & 5{,}000 \\
						Y & 8{,}700 & 6{,}900 & 5{,}300 \\
						Z & 8{,}900 & 6{,}950 & 5{,}150 \\
						\bottomrule
					\end{tabular}
				\end{center}
			\end{minipage}
			\hfill
			\begin{minipage}[t]{0.48\linewidth}
				\textit{Cost side (fixed + variable)}
				\begin{center}
					\begin{tabular}{l c c}
						\toprule
						Plan & Normal $n$ & Disrupted $n$ \\
						\midrule
						X & 8{,}200 & 9{,}100 \\
						Y & 8{,}000 & 9{,}000 \\
						Z & 8{,}100 & 9{,}050 \\
						\bottomrule
					\end{tabular}
				\end{center}
			\end{minipage}
		\end{center}
		
		\subsection*{Solution}
		\textbf{Step 1: Identify alternatives and states of nature.} Alternatives: X, Y, Z. Revenue states: high, medium, low. Cost states: normal and disrupted.
		
		\textbf{Step 2: Compute revenue and costs with fixed + variable formulas.}
		Revenues (thousand USD):
		\[
		\begin{aligned}
		R_X(H)&=618, & R_X(M)&=514, & R_X(L)&=410,\\
		R_Y(H)&=608.5, & R_Y(M)&=509.5, & R_Y(L)&=421.5,\\
		R_Z(H)&=616.7, & R_Z(M)&=513.35, & R_Z(L)&=417.95.
		\end{aligned}
		\]
		Costs (thousand USD):
		\[
		\begin{aligned}
		C_X(N)&=341.4, & C_X(D)&=365.7,\\
		C_Y(N)&=337, & C_Y(D)&=366,\\
		C_Z(N)&=338.8, & C_Z(D)&=365.4.
		\end{aligned}
		\]
		
		\textbf{Step 3: Compute payoff (Profit = Revenue $-$ Cost).}
		Example: for Plan Y with medium demand and disrupted costs,
		\(509.5-366=143.5\).
		
		\textbf{Step 4: Final payoff table (thousand USD).}
		\begin{center}
			\begin{tabular}{l c c c}
				\toprule
				State of nature & Plan X & Plan Y & Plan Z \\
				\midrule
				High revenue, normal cost & 276.6 & 271.5 & 277.9 \\
				High revenue, disrupted cost & 252.3 & 242.5 & 251.3 \\
				Medium revenue, normal cost & 172.6 & 172.5 & 174.55 \\
				Medium revenue, disrupted cost & 148.3 & 143.5 & 147.95 \\
				Low revenue, normal cost & 68.6 & 84.5 & 79.15 \\
				Low revenue, disrupted cost & 44.3 & 55.5 & 52.55 \\
				\bottomrule
			\end{tabular}
		\end{center}
		
		In summary, the final payoff matrix is coherent because both sides of the profit equation include fixed and variable components before subtraction.
	\end{ExamProblems}
\end{document}
