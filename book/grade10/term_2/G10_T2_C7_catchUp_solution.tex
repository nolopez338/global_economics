\makeatletter
\def\input@path{{./}{../}{../../}{preamble/}{../preamble/}{../../preamble/}}
\makeatother
% ----------------------------------------------------------
% GENERAL 

% File
\documentclass[11pt]{book}

% Margins
\usepackage[margin=1in]{geometry}

\linespread{1.2}            % Line spacing
\usepackage[utf8]{inputenc}
\usepackage[T1]{fontenc}
\usepackage{lmodern}
\usepackage{microtype}
\setlength{\parindent}{0pt}
\setlength{\parskip}{6pt}
\usepackage{booktabs}

% ----------------------------------------------------------
% TABLES
\usepackage{multicol}
\usepackage{longtable} 
\usepackage{array}
\usepackage{booktabs}
\usepackage{tabularx}
\usepackage{multirow}

% ----------------------------------------------------------
% MATHEMATICS
\usepackage{amsmath}
\usepackage{amssymb}
\usepackage{amsfonts}
\usepackage{mathtools}

% ----------------------------------------------------------
% HIDDEN CONTENT
\usepackage{ifthen}
% Define a boolean switch
\newboolean{explicaciones}
% Set the boolean switch to true or false
% Change to true to show the content

% Explanations
\newcommand{\explicacion}[2]{
	\ifthenelse{\boolean{explicaciones}}{#1}{#2}
}
\newcommand{\mostrarExplicaciones}[1]{\setboolean{explicaciones}{#1}}

% ----------------------------------------------------------
% NUMBERING

\usepackage{fancyhdr}
\pagestyle{empty} % Ensures the entire document has no page numbers

\usepackage{tocloft}
\renewcommand{\cftdot}{} % Remove dots for sections, if any
\renewcommand{\cftsecleader}{\cftdotfill{\cftdotsep}} % Remove dots for sections, if any
\cftpagenumbersoff{section} % Removes page numbers from sections
\cftpagenumbersoff{subsection} % Removes page numbers from subsections

% ----------------------------------------------------------
% IMAGES 

% General settings
\usepackage{graphicx}       % Insert images
\usepackage{float}          % Position images
% \usepackage{subfigure}      % Subfigures
\graphicspath{{imgs}}       % Image location
\usepackage{subcaption}     % Subfigures II
\usepackage{verbatim}

% Figures
\usepackage{tikz}
\usetikzlibrary{arrows.meta,positioning,trees}

% Colors
\usepackage{xcolor}     
\definecolor{popUp}{HTML}{666666}
\definecolor{popUpIn}{HTML}{CED9E0}
\definecolor{backgroundC}{HTML}{D0E8F2}
\definecolor{backgroundCC}{HTML}{FFFFFF}
\definecolor{borders}{HTML}{8c120d}
\definecolor{padding}{HTML}{77D0D7}
\definecolor{links}{HTML}{CC6F5F}

% ----------------------------------------------------------
% FRAMES

% General settings
\usepackage{tcolorbox}
\usepackage{adjustbox}          % Adjusted frame  
\setlength{\fboxrule}{3pt}  % Line width
\setlength{\fboxsep}{3pt}   % Box padding

% General frames
\usepackage{mdframed}   

\mdfdefinestyle{estiloGeneral}{    % General style
	linecolor=black,
	linewidth=1.5pt,
	roundcorner=10pt,
	backgroundcolor=backgroundC,
	innerbottommargin=0pt
}
\mdfdefinestyle{code}{          % Code style
	linecolor=black,
	linewidth=1.5pt,
	roundcorner=10pt,
	backgroundcolor=darkgray!10,
	innerbottommargin=0pt
}

% Image frame
\newtcbox{\fboxC}{
	colback=backgroundC,
	colframe=popUp,
	arc=10pt,
	boxrule=3pt,
	boxsep=0pt, % Change the padding here
	nobeforeafter
}

% ----------------------------------------------------------
% PAGE SETTINGS

% Background 
\newcommand{\background}[0]{\begin{tikzpicture}[remember picture,overlay]
		\fill[backgroundC] (-2,2) rectangle (25cm, -550);
\end{tikzpicture}}

\newcommand{\backgroundC}[0]{\begin{tikzpicture}[remember picture,overlay]
		\fill[backgroundCC] (-2,2) rectangle (25cm, -550);
\end{tikzpicture}}

% Page width 
\newcommand{\anchoPag}[0]{20cm}

% ----------------------------------------------------------
% FONT

% General
\usepackage{tgbonum}        % Font
\usepackage{listings}       % Code typesetting
\usepackage[spanish]{babel} % Load Spanish
\selectlanguage{spanish}    % Select Spanish
\usepackage{enumitem}
\usepackage{bookmark}

\setlist[itemize]{leftmargin=1.2em, itemsep=0.35em, topsep=0.35em}

% --- Table helpers ---
\newcolumntype{L}[1]{>{\raggedright\arraybackslash}p{#1}}
\newcolumntype{Y}{>{\raggedright\arraybackslash}X}
\newcolumntype{C}{>{\centering\arraybackslash}X}
\renewcommand{\arraystretch}{1.1}

% Python style
\lstdefinestyle{python}{
	language=Python,
	basicstyle=\ttfamily\small,
	commentstyle=\color{green!50!black},
	keywordstyle=\color{blue},
	numberstyle=\tiny\color{gray},
	numbers=left,
	morekeywords={>, <},
	breakatwhitespace=false,
	showstringspaces=false,
	showtabs=false,
	showspaces=false
}

% ----------------------------------------------------------
% HYPERLINKS

% General
\usepackage{hyperref}       
\hypersetup{
	colorlinks=true,
	linkcolor=links,
	filecolor=magenta,      
	urlcolor=brown,
}

% Custom commands 

% Large link
\newcommand{\bigLink}[2]{\begin{center} \fboxC{\LARGE{\href{#1}{#2}}}\end{center}}

% Small link
\newcommand{\smallLink}[2]{\begin{center}\fboxC{\href{#1}{#2}}\end{center}}

% Bold link
\newcommand{\bfLink}[2]{\href{#1}{\textbf{#2}}}


% Small URL
\newcommand{\smallUrl}[1]{\begin{center}\fboxC{\url{#1}}\end{center}}


% ----------------------------------------------------------
% CUSTOM COMMANDS FOR FIGURES

\newcommand{\espacioImagenes}[0]{-1.2cm}

% Without frame
\newcommand{\fig}[3][\espacioImagenes]{
	\hspace*{#1}
	\centering
	\includegraphics[width=#2\textwidth]{#3}
}

% With frame
\newcommand{\ffig}[2]{\begin{figure}[h]
		\hspace*{\espacioImagenes}
		\centering
		\fbox{\includegraphics[width=#1\textwidth]{#2}}
\end{figure}}

% Hyperlink with frame
\newcommand{\hfig}[3]{\begin{figure}[h]
		\hspace*{-1.4cm}
		\centering
		\color{popUp}
		\fboxC{\href{#1}{\includegraphics[width=#2\textwidth]{#3}}}
	\end{figure}
}

% Hyperlink without frame
\newcommand{\hffig}[3]{\begin{figure}[h]
		\hspace*{-1.1cm}
		\centering
		\color{popUp}
		\href{#1}{\includegraphics[width=#2\textwidth]{#3}}
	\end{figure}
}

% ----------------------------------------------------------

% Start and Contents
\newcommand{\cuadro}[1]{
	\begin{mdframed}[style=estiloGeneral]
		#1 
	\end{mdframed}
}

% Explanation video image
\newcommand{\linkExplicacion}[1]{
	\hffig{#1}{0.5}{principal/videoExplicacion}
	\vspace{-0.5cm}
}

\newcommand{\subSecLink}[2]{
	\subsubsection*{\href{#1}{\textbf{#2}}}
}

% Spacing
\newcommand{\esp}[0]{\vspace{4mm}}

% Back to start
\newcommand{\secInicio}[0]{\begin{center}\hyperref[sec:inicio]{ 
			\includegraphics[width=0.1\textwidth]{principal/up}
	}\end{center}
}


\geometry{margin=0.85in}
\AtBeginDocument{\small}

\newcommand{\ExamNameField}{\noindent\textbf{Name:}\ \rule{0.7\linewidth}{0.4pt}\par\medskip}

\newcommand{\ExamTitleBlock}[3]{%
	\begin{center}
		\Large\textbf{#1}\\
		\textbf{#2}%
		\if\relax\detokenize{#3}\relax\else\\\textbf{#3}\fi
	\end{center}
	\vspace{0.5em}
}

\newcommand{\ExamSection}[1]{\par\medskip\textbf{#1}\par\smallskip}

\newenvironment{ExamCriteria}{%
	\begin{itemize}[leftmargin=1.6em, itemsep=0.3em, topsep=0.2em]
}{%
	\end{itemize}
}

\newenvironment{ExamProblems}{%
	\begin{enumerate}[label=\textbf{P\arabic*}, leftmargin=0pt, labelsep=0.6em, itemindent=2.2em, itemsep=0.8em]
}{%
	\end{enumerate}
}


\begin{document}
	\ExamTitleBlock{10th grade}{Catch-up activity T2 C7 Decision analysis solutions}{}
	
	\section*{Contents}
	\noindent\textbf{C7 Uses probabilities and expected value to analyze a decision-making problem.}
	\begin{itemize}
		\item \hyperlink{c7-cu-ex1}{Problem 1 --- Single Project under Uncertainty}
		\item \hyperlink{c7-cu-ex2}{Problem 2 --- Technology Upgrade}
		\item \hyperlink{c7-cu-ex3}{Problem 3 --- Two Investment Alternatives}
		\item \hyperlink{c7-cu-ex4}{Problem 4 --- Three Production Plans}
		\item \hyperlink{c7-cu-ex5}{Problem 5 --- Two Alternatives with Three States}
		\item \hyperlink{c7-cu-ex6}{Problem 6 --- Policy Choice under Three States}
		\item \hyperlink{c7-cu-ex7}{Problem 7 --- New Service Platform Launch}
		\item \hyperlink{c7-cu-ex8}{Problem 8 --- Regional Store Opening}
		\item \hyperlink{c7-cu-ex9}{Problem 9 --- Marketing Strategy Choice}
		\item \hyperlink{c7-cu-ex10}{Problem 10 --- Investment Portfolio Selection}
	\end{itemize}
	
	\ExamSection{C7 Uses probabilities and expected value to analyze a decision-making problem.}
	
	\begin{ExamProblems}
		
		\hypertarget{c7-cu-ex1}{}
		\item
		\subsection*{Problem 1 --- Single Project under Uncertainty}
		
		\textbf{Problem.}
		A renewable energy firm is deciding whether to build a small solar farm. The decision is whether to build (one alternative) or not build, and management uses expected value because similar projects are repeated over time. The uncertain states are:
		\begin{itemize}
			\item \(S_1\): high electricity prices (probability \(p\)),
			\item \(S_2\): medium electricity prices (probability \(0.25\)),
			\item \(S_3\): low electricity prices (probability \(0.75-p\)).
		\end{itemize}
		Payoffs are net profits in millions of dollars from building the farm.
		
		\[
		\begin{array}{lccc}
			\hline
			& S_1\,(p) & S_2\,(0.25) & S_3\,(0.75-p) \\
			\hline
			\text{Build the farm} & 30 & 6 & -12 \\
			\hline
		\end{array}
		\]
		
		Use expected value and determine for which values of \(p\) building the farm is profitable.
		
		\textbf{Solution.}
		\[
		EV=30p+6(0.25)+(-12)(0.75-p)
		\]
		\[
		EV=30p+1.5-9+12p=42p-7.5
		\]
		Profitability condition:
		\[
		EV>0\Rightarrow 42p-7.5>0
		\]
		\[
		42p>7.5\Rightarrow p>\frac{7.5}{42}=\frac{5}{28}
		\]
		So building the farm is profitable when \(p>\frac{5}{28}\), with the feasibility condition \(0\le p\le0.75\).
		
		% --------------------------------------------------
		
		\hypertarget{c7-cu-ex2}{}
		\item
		\subsection*{Problem 2 --- Technology Upgrade}
		
		\textbf{Problem.}
		A delivery company is considering upgrading its routing software. The choice is to upgrade or not, and expected value is appropriate because the system is used repeatedly. The uncertain states are:
		\begin{itemize}
			\item \(S_1\): fuel prices stay very high (probability \(p\)),
			\item \(S_2\): fuel prices are moderate (probability \(0.30\)),
			\item \(S_3\): fuel prices become low (probability \(0.70-p\)).
		\end{itemize}
		Payoffs are net savings in millions of dollars from upgrading.
		
		\[
		\begin{array}{lccc}
			\hline
			& S_1\,(p) & S_2\,(0.30) & S_3\,(0.70-p) \\
			\hline
			\text{Upgrade} & 36 & 8 & -18 \\
			\hline
		\end{array}
		\]
		
		Using expected value, determine for which values of \(p\) the upgrade is profitable.
		
		\textbf{Solution.}
		\[
		EV=36p+8(0.30)+(-18)(0.70-p)
		\]
		\[
		EV=36p+2.4-12.6+18p=54p-10.2
		\]
		\[
		EV>0\Rightarrow 54p-10.2>0\Rightarrow p>\frac{10.2}{54}=\frac{17}{90}
		\]
		So the upgrade is profitable when \(p>\frac{17}{90}\), with \(0\le p\le0.70\).
		
		% --------------------------------------------------
		
		\hypertarget{c7-cu-ex3}{}
		\item
		\subsection*{Problem 3 --- Two Investment Alternatives}
		
		\textbf{Problem.}
		An entrepreneur must choose between two investment alternatives, A and B. Expected value is used to select the higher long-run average return. The uncertain states are:
		\begin{itemize}
			\item \(S_1\): strong market growth (probability \(p\)),
			\item \(S_2\): stable market (probability \(0.25\)),
			\item \(S_3\): weak market (probability \(0.75-p\)).
		\end{itemize}
		Payoffs are net profits in millions of dollars.
		
		\[
		\begin{array}{lccc}
			\hline
			& S_1\,(p) & S_2\,(0.25) & S_3\,(0.75-p) \\
			\hline
			A & 34 & 12 & 0 \\
			B & 26 & 14 & 6 \\
			\hline
		\end{array}
		\]
		
		Using expected value, determine for which values of \(p\) option A yields a higher expected value than option B.
		
		\textbf{Solution.}
		\[
		EV(A)=34p+12(0.25)+0(0.75-p)=34p+3
		\]
		\[
		EV(B)=26p+14(0.25)+6(0.75-p)=26p+3.5+4.5-6p=20p+8
		\]
		A is better when \(EV(A)>EV(B)\):
		\[
		34p+3>20p+8\Rightarrow 14p>5\Rightarrow p>\frac{5}{14}
		\]
		Therefore, option A yields a higher expected value than B when \(p>\frac{5}{14}\). For \(p=\frac{5}{14}\), they tie.
		
		% --------------------------------------------------
		
		\hypertarget{c7-cu-ex4}{}
		\item
		\subsection*{Problem 4 --- Three Production Plans}
		
		\textbf{Problem.}
		A manufacturing firm must choose among three production plans (A, B, C). The firm uses expected value because it seeks the plan with the highest average profit over many similar quarters. The uncertain states are:
		\begin{itemize}
			\item \(S_1\): strong demand (probability \(p\)),
			\item \(S_2\): moderate demand (probability \(0.30\)),
			\item \(S_3\): weak demand (probability \(0.70-p\)).
		\end{itemize}
		Payoffs are net profits in millions of dollars.
		
		\[
		\begin{array}{lccc}
			\hline
			& S_1\,(p) & S_2\,(0.30) & S_3\,(0.70-p) \\
			\hline
			A & 44 & 14 & -10 \\
			B & 32 & 18 & 2 \\
			C & 22 & 16 & 8 \\
			\hline
		\end{array}
		\]
		
		Using expected value, determine for which values of \(p\) each plan is optimal.
		
		\textbf{Solution.}
		\[
		EV(A)=44p+14(0.30)+(-10)(0.70-p)=44p+4.2-7+10p=54p-2.8
		\]
		\[
		EV(B)=32p+18(0.30)+2(0.70-p)=32p+5.4+1.4-2p=30p+6.8
		\]
		\[
		EV(C)=22p+16(0.30)+8(0.70-p)=22p+4.8+5.6-8p=14p+10.4
		\]
		Compare pairwise:
		\[
		EV(A)=EV(B)\Rightarrow 54p-2.8=30p+6.8\Rightarrow 24p=9.6\Rightarrow p=0.40
		\]
		\[
		EV(B)=EV(C)\Rightarrow 30p+6.8=14p+10.4\Rightarrow 16p=3.6\Rightarrow p=0.225
		\]
		\[
		EV(A)=EV(C)\Rightarrow 54p-2.8=14p+10.4\Rightarrow 40p=13.2\Rightarrow p=0.33
		\]
		Using these cut points:
		\begin{itemize}
			\item If \(0\le p<0.225\), plan C is optimal.
			\item At \(p=0.225\), plans B and C tie and both exceed A.
			\item If \(0.225<p<0.40\), plan B is optimal.
			\item At \(p=0.40\), plans A and B tie and both exceed C.
			\item If \(0.40<p\le0.70\), plan A is optimal.
		\end{itemize}
		
		% --------------------------------------------------
		
		\hypertarget{c7-cu-ex5}{}
		\item
		\subsection*{Problem 5 --- Two Alternatives with Three States}
		
		\textbf{Problem.}
		A shipping company must choose between two routing strategies, A and B. Expected value is used because the company wants the higher average profit over many shipments. The uncertain states are:
		\begin{itemize}
			\item \(S_1\): low congestion (probability \(p_1\)),
			\item \(S_2\): medium congestion (probability \(p_2\)),
			\item \(S_3\): high congestion (probability \(p_3=1-p_1-p_2\)).
		\end{itemize}
		Payoffs are net profits in millions of dollars.
		
		\[
		\begin{array}{lccc}
			\hline
			& S_1\,(p_1) & S_2\,(p_2) & S_3\,(1-p_1-p_2) \\
			\hline
			A & 28 & 10 & -12 \\
			B & 20 & 16 & 0 \\
			\hline
		\end{array}
		\]
		
		Using expected value, determine when strategy A yields a higher expected value than strategy B.
		
		\textbf{Solution.}
		\[
		EV(A)=28p_1+10p_2+(-12)(1-p_1-p_2)
		\]
		\[
		EV(A)=28p_1+10p_2-12+12p_1+12p_2=40p_1+22p_2-12
		\]
		\[
		EV(B)=20p_1+16p_2+0(1-p_1-p_2)=20p_1+16p_2
		\]
		A is better when:
		\[
		EV(A)-EV(B)>0
		\]
		\[
		(40p_1+22p_2-12)-(20p_1+16p_2)>0
		\]
		\[
		20p_1+6p_2-12>0\Rightarrow 10p_1+3p_2>6
		\]
		So strategy A yields a higher expected value when \(10p_1+3p_2>6\), with \(p_1\ge0\), \(p_2\ge0\), and \(p_1+p_2\le1\).
		
		% --------------------------------------------------
		
		\hypertarget{c7-cu-ex6}{}
		\item
		\subsection*{Problem 6 --- Policy Choice under Three States}
		
		\textbf{Problem.}
		A local government compares two flood-prevention policies, A and B. Expected value is used because the city wants the highest average net benefit over many years. The uncertain states are:
		\begin{itemize}
			\item \(S_1\): heavy rainfall season (probability \(p_1\)),
			\item \(S_2\): moderate rainfall season (probability \(p_2\)),
			\item \(S_3\): dry season (probability \(1-p_1-p_2\)).
		\end{itemize}
		Payoffs are net benefits in millions of dollars.
		
		\[
		\begin{array}{lccc}
			\hline
			& S_1\,(p_1) & S_2\,(p_2) & S_3\,(1-p_1-p_2) \\
			\hline
			A & 24 & 8 & -10 \\
			B & 16 & 14 & 4 \\
			\hline
		\end{array}
		\]
		
		Using expected value, determine when policy A yields a higher expected value than policy B.
		
		\textbf{Solution.}
		\[
		EV(A)=24p_1+8p_2+(-10)(1-p_1-p_2)=34p_1+18p_2-10
		\]
		\[
		EV(B)=16p_1+14p_2+4(1-p_1-p_2)=12p_1+10p_2+4
		\]
		\[
		EV(A)-EV(B)>0
		\]
		\[
		(34p_1+18p_2-10)-(12p_1+10p_2+4)>0
		\]
		\[
		22p_1+8p_2-14>0\Rightarrow 11p_1+4p_2>7
		\]
		Therefore, policy A is preferred when \(11p_1+4p_2>7\).
		
		% --------------------------------------------------
		
		\hypertarget{c7-cu-ex7}{}
		\item
		\subsection*{Problem 7 --- New Service Platform Launch}
		
		\textbf{Problem.}
		A software company must choose among three launch plans (A, B, C) for a new service platform. Expected value is used because the company repeats similar launches and wants the highest average profit. The uncertain states are:
		\begin{itemize}
			\item \(S_1\): strong adoption (probability \(0.50\)),
			\item \(S_2\): moderate adoption (probability \(0.30\)),
			\item \(S_3\): weak adoption (probability \(0.20\)).
		\end{itemize}
		Payoffs are net profits in millions of dollars.
		
		\[
		\begin{array}{lccc}
			\hline
			& S_1\,(0.50) & S_2\,(0.30) & S_3\,(0.20) \\
			\hline
			A & 34 & 16 & -2 \\
			B & 30 & 14 & 4 \\
			C & 24 & 18 & 8 \\
			\hline
		\end{array}
		\]
		
		Using expected value, compute the expected value for each plan and recommend the plan with the highest expected value.
		
		\textbf{Solution.}
		\[
		EV(A)=34(0.50)+16(0.30)+(-2)(0.20)=17+4.8-0.4=21.4
		\]
		\[
		EV(B)=30(0.50)+14(0.30)+4(0.20)=15+4.2+0.8=20.0
		\]
		\[
		EV(C)=24(0.50)+18(0.30)+8(0.20)=12+5.4+1.6=19.0
		\]
		Since \(21.4>20.0>19.0\), plan A has the highest expected value and should be recommended, although B is relatively close.
		
		% --------------------------------------------------
		
		\hypertarget{c7-cu-ex8}{}
		\item
		\subsection*{Problem 8 --- Regional Store Opening}
		
		\textbf{Problem.}
		A retailer evaluates opening a regional store. Expected value is used because the retailer wants the highest average profit over many comparable openings. Demand and competitor response are treated as independent. The states are:
		\begin{itemize}
			\item strong demand (probability \(p\)),
			\item weak demand (probability \(1-p\)),
			\item mild competitor response (probability \(q\)),
			\item aggressive competitor response (probability \(1-q\)).
		\end{itemize}
		Payoffs are net profits in millions of dollars.
		
		\[
		\begin{array}{lcc}
			\hline
			& \text{Mild}\,(q) & \text{Aggressive}\,(1-q) \\
			\hline
			\text{Strong demand}\,(p) & 44 & 18 \\
			\text{Weak demand}\,(1-p) & -4 & -16 \\
			\hline
		\end{array}
		\]
		
		Using expected value, determine when opening the store is profitable.
		
		\textbf{Solution.}
		\[
		EV=44pq+18p(1-q)-4(1-p)q-16(1-p)(1-q)
		\]
		Expand each term:
		\[
		EV=44pq+18p-18pq-4q+4pq-16+16p+16q-16pq
		\]
		Combine like terms:
		\[
		EV=(44pq-18pq+4pq-16pq)+(18p+16p)+(-4q+16q)-16
		\]
		\[
		EV=14pq+34p+12q-16
		\]
		Profitability condition:
		\[
		EV>0\Rightarrow 14pq+34p+12q-16>0
		\]
		So opening the store is profitable when \(14pq+34p+12q>16\).
		
		% --------------------------------------------------
		
		\hypertarget{c7-cu-ex9}{}
		\item
		\subsection*{Problem 9 --- Marketing Strategy Choice}
		
		\textbf{Problem.}
		A firm must choose between two marketing strategies, A and B. Expected value is used because the firm wants the highest average profit over many campaigns. Market interest and campaign execution are independent. The states are:
		\begin{itemize}
			\item high interest (probability \(p\)),
			\item low interest (probability \(1-p\)),
			\item effective execution (probability \(q\)),
			\item ineffective execution (probability \(1-q\)).
		\end{itemize}
		Payoffs are net profits in millions of dollars.
		
		\[
		\begin{array}{lcc}
			\hline
			\text{State of nature} & A & B \\
			\hline
			\text{High interest}\,(p),\,\text{Effective}\,(q) & 36 & 30 \\
			\text{High interest}\,(p),\,\text{Ineffective}\,(1-q) & 14 & 18 \\
			\text{Low interest}\,(1-p),\,\text{Effective}\,(q) & -10 & 0 \\
			\text{Low interest}\,(1-p),\,\text{Ineffective}\,(1-q) & -18 & -8 \\
			\hline
		\end{array}
		\]
		
		Using expected value, determine when strategy A yields a higher expected value than strategy B.
		
		\textbf{Solution.}
		\[
		EV_A=36pq+14p(1-q)-10(1-p)q-18(1-p)(1-q)
		\]
		\[
		EV_A=36pq+14p-14pq-10q+10pq-18+18p+18q-18pq
		\]
		\[
		EV_A=14pq+32p+8q-18
		\]
		
		\[
		EV_B=30pq+18p(1-q)+0(1-p)q-8(1-p)(1-q)
		\]
		\[
		EV_B=30pq+18p-18pq-8+8p+8q-8pq
		\]
		\[
		EV_B=4pq+26p+8q-8
		\]
		
		A better than B when:
		\[
		EV_A-EV_B>0
		\]
		\[
		(14pq+32p+8q-18)-(4pq+26p+8q-8)>0
		\]
		\[
		10pq+6p-10>0\Rightarrow 5pq+3p>5
		\]
		So strategy A has higher expected value when \(5pq+3p>5\).
		
		% --------------------------------------------------
		
		\hypertarget{c7-cu-ex10}{}
		\item
		\subsection*{Problem 10 --- Investment Portfolio Selection}
		
		\textbf{Problem.}
		An investor must choose between two portfolios, X and Y. Expected value is used because the investor wants the higher long-run average return. Economic growth and interest rates are independent. The states are:
		\begin{itemize}
			\item strong growth (probability \(p\)),
			\item weak growth (probability \(1-p\)),
			\item low interest rates (probability \(q\)),
			\item high interest rates (probability \(1-q\)).
		\end{itemize}
		Payoffs are net returns in millions of dollars.
		
		\[
		\begin{array}{lcc}
			\hline
			\text{State of nature} & X & Y \\
			\hline
			\text{Strong growth}\,(p),\,\text{Low rates}\,(q) & 34 & 30 \\
			\text{Strong growth}\,(p),\,\text{High rates}\,(1-q) & 16 & 22 \\
			\text{Weak growth}\,(1-p),\,\text{Low rates}\,(q) & 6 & 10 \\
			\text{Weak growth}\,(1-p),\,\text{High rates}\,(1-q) & -14 & -6 \\
			\hline
		\end{array}
		\]
		
		Using expected value, determine when portfolio X yields a higher expected value than portfolio Y.
		
		\textbf{Solution.}
		\[
		EV_X=34pq+16p(1-q)+6(1-p)q-14(1-p)(1-q)
		\]
		\[
		EV_X=34pq+16p-16pq+6q-6pq-14+14p+14q-14pq
		\]
		\[
		EV_X=-2pq+30p+20q-14
		\]
		
		\[
		EV_Y=30pq+22p(1-q)+10(1-p)q-6(1-p)(1-q)
		\]
		\[
		EV_Y=30pq+22p-22pq+10q-10pq-6+6p+6q-6pq
		\]
		\[
		EV_Y=-8pq+28p+16q-6
		\]
		
		Portfolio X better than Y when:
		\[
		EV_X-EV_Y>0
		\]
		\[
		(-2pq+30p+20q-14)-(-8pq+28p+16q-6)>0
		\]
		\[
		6pq+2p+4q-8>0
		\]
		\[
		3pq+p+2q>4
		\]
		Therefore, portfolio X has higher expected value when \(3pq+p+2q>4\).
		
	\end{ExamProblems}
	
\end{document}
