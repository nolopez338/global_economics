\documentclass[12pt]{article}

% Page size and tighter margins
\usepackage[a4paper,left=1.2cm,right=1.2cm,top=1.5cm,bottom=1.5cm]{geometry}

% Core packages
\usepackage{graphicx}
\usepackage{xcolor}
\usepackage{array}
\usepackage{tabularx}
\usepackage{multicol}
\usepackage[T1]{fontenc}
\usepackage[utf8]{inputenc}

\setlength{\parindent}{0pt}
\setlength{\tabcolsep}{6pt}
\renewcommand{\arraystretch}{1.15}

% Column types
\newcolumntype{Y}{>{\raggedright\arraybackslash}m{\dimexpr0.30\textwidth-2\tabcolsep-2\arrayrulewidth\relax}}
\newcolumntype{Z}{>{\raggedright\arraybackslash}m{\dimexpr0.70\textwidth-2\tabcolsep-2\arrayrulewidth\relax}}
\newcolumntype{C}[1]{>{\centering\arraybackslash}m{#1}}

% Gray subsection header box
\newcommand{\SubsectionBox}[1]{%
	\noindent\colorbox{gray!30}{%
		\parbox{\linewidth}{\textbf{#1}}%
	}\par\vspace{0.35cm}%
}

% Centered multi-line cell helper
\newcommand{\CellCenter}[1]{%
	\parbox{\linewidth}{\centering #1}%
}

\begin{document}

	% =========================
	% HEADER BOX (3 COLUMNS)
	% =========================
	\noindent
	\begin{tabularx}{\textwidth}{|C{2.8cm}|C{\dimexpr\textwidth-6cm-4\tabcolsep-4\arrayrulewidth\relax}|C{2.8cm}|}
		\hline
		\centering
		\vspace{3mm}
		\includegraphics[width=2.5cm]{../../preamble/logo.png}
		&
		\CellCenter{%
			\vspace{-5mm}
			\textbf{GLOBAL ECONOMICS}\par
			\textbf{GRADE: 10TH}\par
			\textbf{LEARNING EVIDENCE T2 L2 C6 ACTIVITY}\par
			\textbf{ANALYSIS OF DECISIONS}\par
			\textbf{TEACHER'S NAME: Nicolás López Cuéllar}
		}
		&
		\CellCenter{%
			\textbf{SECOND TERM}\par
			\textbf{2025--2026}%
		}
		\\
		\hline
	\end{tabularx}

	\vspace{0.5cm}

	% =========================
	% OBJECTIVE + CRITERIA
	% =========================
	\noindent
	\begin{tabular}{|Y|Z|}
		\hline
		{\small
			\textbf{Learning objective:} Analyze decision-making scenarios by applying minimax criteria.
		}
		&
		{\footnotesize
			\textbf{Assessment criteria:}\par
			C6: Develops decision-making strategies using probabilities and the maximum opportunity criterion.\par
		}
		\\
		\hline
	\end{tabular}

	\vspace{0.4cm}

	% =========================
	% STUDENT LINE
	% =========================
%	\noindent
%	\textbf{Student’s name:} \rule{7cm}{0.4pt}\hfill
%	\textbf{Group:} \rule{2cm}{0.4pt}\hfill
%	\textbf{Date:} \rule{3cm}{0.4pt}


	% =========================
	% ACTIVITY BODY
	% =========================
	\begin{multicols}{2}
		\SubsectionBox{Criteria assessment}\vspace{-0.25cm}
		Each assessment criterion is evaluated across the five problems. A criterion is considered passed when it is correctly activated in at least four of the five problems.

		\vspace{0.25cm}
		\SubsectionBox{1. Neighborhood bakery market setup}\vspace{-0.25cm}
		A neighborhood bakery is deciding on a weekend market setup. The bakery can expand its stall or keep a standard setup. Expected profit depends on turnout, with high turnout probability $0.60$ and low turnout probability $0.40$. Profit is measured in hundreds of dollars.

		\begin{center}
			\textit{Payoff table}\\
			\begin{tabular}{|l|c|c|}
				\hline
				Alternative &
				\begin{tabular}{@{}c@{}} High turnout \\ $(0.60)$ \end{tabular} &
				\begin{tabular}{@{}c@{}} Low turnout \\ $(0.40)$ \end{tabular} \\
				\hline
				Expand stall & 24 & 6 \\
				Standard stall & 18 & 10 \\
				\hline
			\end{tabular}
		\end{center}
		

		\vspace{0.25cm}
		\SubsectionBox{2. School fundraiser sales choice}\vspace{-0.25cm}
		A school fundraiser chooses between a booth sale and a delivery sale. Attendance can be high, medium, or low with probabilities $0.35$, $0.40$, and $0.25$. Profits are in hundreds of dollars.

		\begin{center}
			\textit{Payoff table}\\
			\begin{tabular}{|l|c|c|c|}
				\hline
				Alternative &
				\begin{tabular}{@{}c@{}} High \\ $(0.35)$ \end{tabular} &
				\begin{tabular}{@{}c@{}} Medium \\ $(0.40)$ \end{tabular} &
				\begin{tabular}{@{}c@{}} Low \\ $(0.25)$ \end{tabular} \\
				\hline
				Booth sale & 30 & 18 & 4 \\
				Delivery sale & 24 & 16 & 12 \\
				\hline
			\end{tabular}
		\end{center}
		

		\vspace{0.25cm}
		\SubsectionBox{3. Farm cooperative storage plans}\vspace{-0.25cm}
		A farm cooperative is choosing a storage plan for a harvest season. It can use Plan A, Plan B, or Plan C. Demand can be strong, moderate, or weak with probabilities $0.30$, $0.45$, and $0.25$. Profits are in hundreds of dollars.

		\begin{center}
			\textit{Payoff table}\\
			\begin{tabular}{|l|c|c|c|}
				\hline
				Alternative &
				\begin{tabular}{@{}c@{}} Strong \\ $(0.30)$ \end{tabular} &
				\begin{tabular}{@{}c@{}} Moderate \\ $(0.45)$ \end{tabular} &
				\begin{tabular}{@{}c@{}} Weak \\ $(0.25)$ \end{tabular} \\
				\hline
				Plan A & 40 & 22 & 5 \\
				Plan B & 32 & 26 & 12 \\
				Plan C & 25 & 20 & 18 \\
				\hline
			\end{tabular}
		\end{center}
		

		\vspace{0.25cm}
		\SubsectionBox{4. Community theater ticket focus}\vspace{-0.25cm}
		A community theater must choose between a matinee focus and an evening focus. Ticket demand can be strong with probability $0.55$ or weak with probability $0.45$. Profits are in hundreds of dollars.

		\begin{center}
			\textit{Payoff table}\\
			\begin{tabular}{|l|c|c|}
				\hline
				Alternative &
				\begin{tabular}{@{}c@{}} Strong demand \\ $(0.55)$ \end{tabular} &
				\begin{tabular}{@{}c@{}} Weak demand \\ $(0.45)$ \end{tabular} \\
				\hline
				Matinee focus & 20 & 8 \\
				Evening focus & 16 & 12 \\
				\hline
			\end{tabular}
		\end{center}
		

		\vspace{0.25cm}
		\SubsectionBox{5. Transit agency campaign choice}\vspace{-0.25cm}
		A transit agency is choosing between a digital campaign and a print campaign. Public interest can be high, medium, or low with probabilities $0.30$, $0.40$, and $0.30$. Profits are in hundreds of dollars.

		\begin{center}
			\textit{Payoff table}\\
			\begin{tabular}{|l|c|c|c|}
				\hline
				Alternative &
				\begin{tabular}{@{}c@{}} High \\ $(0.30)$ \end{tabular} &
				\begin{tabular}{@{}c@{}} Medium \\ $(0.40)$ \end{tabular} &
				\begin{tabular}{@{}c@{}} Low \\ $(0.30)$ \end{tabular} \\
				\hline
				Digital campaign & 28 & 14 & 6 \\
				Print campaign & 22 & 16 & 10 \\
				\hline
			\end{tabular}
		\end{center}
		

	\end{multicols}

\end{document}
