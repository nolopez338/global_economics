\makeatletter
\def\input@path{{./}{../}{../../}{preamble/}{../preamble/}{../../preamble/}}
\makeatother
% ----------------------------------------------------------
% GENERAL 

% File
\documentclass[11pt]{book}

% Margins
\usepackage[margin=1in]{geometry}

\linespread{1.2}            % Line spacing
\usepackage[utf8]{inputenc}
\usepackage[T1]{fontenc}
\usepackage{lmodern}
\usepackage{microtype}
\setlength{\parindent}{0pt}
\setlength{\parskip}{6pt}
\usepackage{booktabs}

% ----------------------------------------------------------
% TABLES
\usepackage{multicol}
\usepackage{longtable} 
\usepackage{array}
\usepackage{booktabs}
\usepackage{tabularx}
\usepackage{multirow}

% ----------------------------------------------------------
% MATHEMATICS
\usepackage{amsmath}
\usepackage{amssymb}
\usepackage{amsfonts}
\usepackage{mathtools}

% ----------------------------------------------------------
% HIDDEN CONTENT
\usepackage{ifthen}
% Define a boolean switch
\newboolean{explicaciones}
% Set the boolean switch to true or false
% Change to true to show the content

% Explanations
\newcommand{\explicacion}[2]{
	\ifthenelse{\boolean{explicaciones}}{#1}{#2}
}
\newcommand{\mostrarExplicaciones}[1]{\setboolean{explicaciones}{#1}}

% ----------------------------------------------------------
% NUMBERING

\usepackage{fancyhdr}
\pagestyle{empty} % Ensures the entire document has no page numbers

\usepackage{tocloft}
\renewcommand{\cftdot}{} % Remove dots for sections, if any
\renewcommand{\cftsecleader}{\cftdotfill{\cftdotsep}} % Remove dots for sections, if any
\cftpagenumbersoff{section} % Removes page numbers from sections
\cftpagenumbersoff{subsection} % Removes page numbers from subsections

% ----------------------------------------------------------
% IMAGES 

% General settings
\usepackage{graphicx}       % Insert images
\usepackage{float}          % Position images
% \usepackage{subfigure}      % Subfigures
\graphicspath{{imgs}}       % Image location
\usepackage{subcaption}     % Subfigures II
\usepackage{verbatim}

% Figures
\usepackage{tikz}
\usetikzlibrary{arrows.meta,positioning,trees}

% Colors
\usepackage{xcolor}     
\definecolor{popUp}{HTML}{666666}
\definecolor{popUpIn}{HTML}{CED9E0}
\definecolor{backgroundC}{HTML}{D0E8F2}
\definecolor{backgroundCC}{HTML}{FFFFFF}
\definecolor{borders}{HTML}{8c120d}
\definecolor{padding}{HTML}{77D0D7}
\definecolor{links}{HTML}{CC6F5F}

% ----------------------------------------------------------
% FRAMES

% General settings
\usepackage{tcolorbox}
\usepackage{adjustbox}          % Adjusted frame  
\setlength{\fboxrule}{3pt}  % Line width
\setlength{\fboxsep}{3pt}   % Box padding

% General frames
\usepackage{mdframed}   

\mdfdefinestyle{estiloGeneral}{    % General style
	linecolor=black,
	linewidth=1.5pt,
	roundcorner=10pt,
	backgroundcolor=backgroundC,
	innerbottommargin=0pt
}
\mdfdefinestyle{code}{          % Code style
	linecolor=black,
	linewidth=1.5pt,
	roundcorner=10pt,
	backgroundcolor=darkgray!10,
	innerbottommargin=0pt
}

% Image frame
\newtcbox{\fboxC}{
	colback=backgroundC,
	colframe=popUp,
	arc=10pt,
	boxrule=3pt,
	boxsep=0pt, % Change the padding here
	nobeforeafter
}

% ----------------------------------------------------------
% PAGE SETTINGS

% Background 
\newcommand{\background}[0]{\begin{tikzpicture}[remember picture,overlay]
		\fill[backgroundC] (-2,2) rectangle (25cm, -550);
\end{tikzpicture}}

\newcommand{\backgroundC}[0]{\begin{tikzpicture}[remember picture,overlay]
		\fill[backgroundCC] (-2,2) rectangle (25cm, -550);
\end{tikzpicture}}

% Page width 
\newcommand{\anchoPag}[0]{20cm}

% ----------------------------------------------------------
% FONT

% General
\usepackage{tgbonum}        % Font
\usepackage{listings}       % Code typesetting
\usepackage[spanish]{babel} % Load Spanish
\selectlanguage{spanish}    % Select Spanish
\usepackage{enumitem}
\usepackage{bookmark}

\setlist[itemize]{leftmargin=1.2em, itemsep=0.35em, topsep=0.35em}

% --- Table helpers ---
\newcolumntype{L}[1]{>{\raggedright\arraybackslash}p{#1}}
\newcolumntype{Y}{>{\raggedright\arraybackslash}X}
\newcolumntype{C}{>{\centering\arraybackslash}X}
\renewcommand{\arraystretch}{1.1}

% Python style
\lstdefinestyle{python}{
	language=Python,
	basicstyle=\ttfamily\small,
	commentstyle=\color{green!50!black},
	keywordstyle=\color{blue},
	numberstyle=\tiny\color{gray},
	numbers=left,
	morekeywords={>, <},
	breakatwhitespace=false,
	showstringspaces=false,
	showtabs=false,
	showspaces=false
}

% ----------------------------------------------------------
% HYPERLINKS

% General
\usepackage{hyperref}       
\hypersetup{
	colorlinks=true,
	linkcolor=links,
	filecolor=magenta,      
	urlcolor=brown,
}

% Custom commands 

% Large link
\newcommand{\bigLink}[2]{\begin{center} \fboxC{\LARGE{\href{#1}{#2}}}\end{center}}

% Small link
\newcommand{\smallLink}[2]{\begin{center}\fboxC{\href{#1}{#2}}\end{center}}

% Bold link
\newcommand{\bfLink}[2]{\href{#1}{\textbf{#2}}}


% Small URL
\newcommand{\smallUrl}[1]{\begin{center}\fboxC{\url{#1}}\end{center}}


% ----------------------------------------------------------
% CUSTOM COMMANDS FOR FIGURES

\newcommand{\espacioImagenes}[0]{-1.2cm}

% Without frame
\newcommand{\fig}[3][\espacioImagenes]{
	\hspace*{#1}
	\centering
	\includegraphics[width=#2\textwidth]{#3}
}

% With frame
\newcommand{\ffig}[2]{\begin{figure}[h]
		\hspace*{\espacioImagenes}
		\centering
		\fbox{\includegraphics[width=#1\textwidth]{#2}}
\end{figure}}

% Hyperlink with frame
\newcommand{\hfig}[3]{\begin{figure}[h]
		\hspace*{-1.4cm}
		\centering
		\color{popUp}
		\fboxC{\href{#1}{\includegraphics[width=#2\textwidth]{#3}}}
	\end{figure}
}

% Hyperlink without frame
\newcommand{\hffig}[3]{\begin{figure}[h]
		\hspace*{-1.1cm}
		\centering
		\color{popUp}
		\href{#1}{\includegraphics[width=#2\textwidth]{#3}}
	\end{figure}
}

% ----------------------------------------------------------

% Start and Contents
\newcommand{\cuadro}[1]{
	\begin{mdframed}[style=estiloGeneral]
		#1 
	\end{mdframed}
}

% Explanation video image
\newcommand{\linkExplicacion}[1]{
	\hffig{#1}{0.5}{principal/videoExplicacion}
	\vspace{-0.5cm}
}

\newcommand{\subSecLink}[2]{
	\subsubsection*{\href{#1}{\textbf{#2}}}
}

% Spacing
\newcommand{\esp}[0]{\vspace{4mm}}

% Back to start
\newcommand{\secInicio}[0]{\begin{center}\hyperref[sec:inicio]{ 
			\includegraphics[width=0.1\textwidth]{principal/up}
	}\end{center}
}


\geometry{margin=0.85in}
\AtBeginDocument{\small}

\newcommand{\ExamNameField}{\noindent\textbf{Name:}\ \rule{0.7\linewidth}{0.4pt}\par\medskip}

\newcommand{\ExamTitleBlock}[3]{%
	\begin{center}
		\Large\textbf{#1}\\
		\textbf{#2}%
		\if\relax\detokenize{#3}\relax\else\\\textbf{#3}\fi
	\end{center}
	\vspace{0.5em}
}

\newcommand{\ExamSection}[1]{\par\medskip\textbf{#1}\par\smallskip}

\newenvironment{ExamCriteria}{%
	\begin{itemize}[leftmargin=1.6em, itemsep=0.3em, topsep=0.2em]
}{%
	\end{itemize}
}

\newenvironment{ExamProblems}{%
	\begin{enumerate}[label=\textbf{P\arabic*}, leftmargin=0pt, labelsep=0.6em, itemindent=2.2em, itemsep=0.8em]
}{%
	\end{enumerate}
}

\begin{document}
	\ExamTitleBlock{10th grade}{Learning evidence T2 L2 C6 supplementary activity solutions}{}
	
	\ExamSection{Problems}
	\begin{ExamProblems}
		\item
		\subsection*{School cafeteria supply strategy under uncertain demand}
		A school cafeteria is selecting a supply strategy for next month. The options are a local farm contract, a wholesale mix, or a frozen backup plan. Student demand can be very high, medium, or low with probabilities $0.40$, $0.35$, and $0.25$. Profits are measured in hundreds of dollars.

		\begin{center}
			\textit{Payoff table}\\
			\begin{tabular}{l c c c}
				\toprule
				Alternative & Very high demand $(0.40)$ & Medium demand $(0.35)$ & Low demand $(0.25)$ \\
				\midrule
				Local farm contract & 36 & 24 & 8 \\
				Wholesale mix & 30 & 26 & 16 \\
				Frozen backup plan & 22 & 20 & 18 \\
				\bottomrule
			\end{tabular}
		\end{center}

		\subsection*{C6}
		\textbf{Maximum opportunity criterion (minimax regret).} Best payoff in each state.
		\[
		\begin{aligned}
		\text{Very high demand gives } &\max\{36,30,22\}=36,\\
		\text{Medium demand gives } &\max\{24,26,20\}=26,\\
		\text{Low demand gives } &\max\{8,16,18\}=18.
		\end{aligned}
		\]

		Regret table.
		\[
		\begin{array}{lcccc}
			\toprule
			\text{Alternative} & \text{Very high} & \text{Medium} & \text{Low} & \text{Maximum regret}\\
			\midrule
			\text{Local farm contract} & 36-36=0 & 26-24=2 & 18-8=10 & 10\\
			\text{Wholesale mix} & 36-30=6 & 26-26=0 & 18-16=2 & 6\\
			\text{Frozen backup plan} & 36-22=14 & 26-20=6 & 18-18=0 & 14\\
			\bottomrule
		\end{array}
		\]

		Minimax regret choice is \emph{Wholesale mix} because its maximum regret is $6$.

		\item
		\subsection*{Holiday bookstore inventory strategy choice}
		A neighborhood bookstore must choose an inventory strategy for a holiday month. The alternatives are aggressive stocking, balanced stocking, and conservative stocking. Sales conditions can be surge, steady, slow, or very slow with probabilities $0.25$, $0.35$, $0.25$, and $0.15$. Profits are in hundreds of dollars.

		\begin{center}
			\textit{Payoff table}\\
			\begin{tabular}{l c c c c}
				\toprule
				Alternative & Surge $(0.25)$ & Steady $(0.35)$ & Slow $(0.25)$ & Very slow $(0.15)$ \\
				\midrule
				Aggressive stocking & 50 & 30 & 12 & -4 \\
				Balanced stocking & 42 & 32 & 18 & 6 \\
				Conservative stocking & 30 & 28 & 22 & 14 \\
				\bottomrule
			\end{tabular}
		\end{center}

		\subsection*{C6}
		\textbf{Maximum opportunity criterion (minimax regret).} Best payoff in each state.
		\[
		\begin{aligned}
		\text{Surge sales gives } &\max\{50,42,30\}=50,\\
		\text{Steady sales gives } &\max\{30,32,28\}=32,\\
		\text{Slow sales gives } &\max\{12,18,22\}=22,\\
		\text{Very slow sales gives } &\max\{-4,6,14\}=14.
		\end{aligned}
		\]

		Regret table.
		\[
		\begin{array}{lccccc}
			\toprule
			\text{Alternative} & \text{Surge} & \text{Steady} & \text{Slow} & \text{Very slow} & \text{Maximum regret}\\
			\midrule
			\text{Aggressive stocking} & 50-50=0 & 32-30=2 & 22-12=10 & 14-(-4)=18 & 18\\
			\text{Balanced stocking} & 50-42=8 & 32-32=0 & 22-18=4 & 14-6=8 & 8\\
			\text{Conservative stocking} & 50-30=20 & 32-28=4 & 22-22=0 & 14-14=0 & 20\\
			\bottomrule
		\end{array}
		\]

		Minimax regret choice is \emph{Balanced stocking} because its maximum regret is $8$.

		\item
		\subsection*{Battery storage contract selection for dry season demand}
		An energy cooperative is deciding a battery storage contract for the dry season. It can choose Contract A, Contract B, Contract C, or Contract D. Electricity demand can be excellent, good, fair, or poor with probabilities $0.20$, $0.35$, $0.30$, and $0.15$. Profits are in hundreds of dollars.

		\begin{center}
			\textit{Payoff table}\\
			\begin{tabular}{l c c c c}
				\toprule
				Alternative & Excellent $(0.20)$ & Good $(0.35)$ & Fair $(0.30)$ & Poor $(0.15)$ \\
				\midrule
				Contract A & 60 & 38 & 20 & 2 \\
				Contract B & 52 & 42 & 26 & 10 \\
				Contract C & 44 & 40 & 30 & 16 \\
				Contract D & 36 & 34 & 28 & 22 \\
				\bottomrule
			\end{tabular}
		\end{center}

		\subsection*{C6}
		\textbf{Maximum opportunity criterion (minimax regret).} Best payoff in each state.
		\[
		\begin{aligned}
		\text{Excellent demand gives } &\max\{60,52,44,36\}=60,\\
		\text{Good demand gives } &\max\{38,42,40,34\}=42,\\
		\text{Fair demand gives } &\max\{20,26,30,28\}=30,\\
		\text{Poor demand gives } &\max\{2,10,16,22\}=22.
		\end{aligned}
		\]

		Regret table.
		\[
		\begin{array}{lccccc}
			\toprule
			\text{Alternative} & \text{Excellent} & \text{Good} & \text{Fair} & \text{Poor} & \text{Maximum regret}\\
			\midrule
			\text{Contract A} & 60-60=0 & 42-38=4 & 30-20=10 & 22-2=20 & 20\\
			\text{Contract B} & 60-52=8 & 42-42=0 & 30-26=4 & 22-10=12 & 12\\
			\text{Contract C} & 60-44=16 & 42-40=2 & 30-30=0 & 22-16=6 & 16\\
			\text{Contract D} & 60-36=24 & 42-34=8 & 30-28=2 & 22-22=0 & 24\\
			\bottomrule
		\end{array}
		\]

		Minimax regret choice is \emph{Contract B} because its maximum regret is $12$.

		\item
		\subsection*{Saturday program planning for community sports attendance}
		A community sports center is planning Saturday activities. It can choose a tournament focus, a mixed program, or training clinics. Attendance can be high, medium, or low with probabilities $0.50$, $0.30$, and $0.20$. Profits are in hundreds of dollars.

		\begin{center}
			\textit{Payoff table}\\
			\begin{tabular}{l c c c}
				\toprule
				Alternative & High turnout $(0.50)$ & Medium turnout $(0.30)$ & Low turnout $(0.20)$ \\
				\midrule
				Tournament focus & 34 & 18 & 2 \\
				Mixed program & 30 & 22 & 10 \\
				Training clinics & 24 & 20 & 16 \\
				\bottomrule
			\end{tabular}
		\end{center}

		\subsection*{C6}
		\textbf{Maximum opportunity criterion (minimax regret).} Best payoff in each state.
		\[
		\begin{aligned}
		\text{High turnout gives } &\max\{34,30,24\}=34,\\
		\text{Medium turnout gives } &\max\{18,22,20\}=22,\\
		\text{Low turnout gives } &\max\{2,10,16\}=16.
		\end{aligned}
		\]

		Regret table.
		\[
		\begin{array}{lcccc}
			\toprule
			\text{Alternative} & \text{High} & \text{Medium} & \text{Low} & \text{Maximum regret}\\
			\midrule
			\text{Tournament focus} & 34-34=0 & 22-18=4 & 16-2=14 & 14\\
			\text{Mixed program} & 34-30=4 & 22-22=0 & 16-10=6 & 6\\
			\text{Training clinics} & 34-24=10 & 22-20=2 & 16-16=0 & 10\\
			\bottomrule
		\end{array}
		\]

		Minimax regret choice is \emph{Mixed program} because its maximum regret is $6$.

		\item
		\subsection*{City recycling outreach strategy under variable response}
		A city recycling office is choosing an outreach strategy. The alternatives are social media campaign, school visits, door-to-door campaign, and hybrid campaign. Public response can be strong, moderate, or weak with probabilities $0.30$, $0.45$, and $0.25$. Net gains are in hundreds of dollars.

		\begin{center}
			\textit{Payoff table}\\
			\begin{tabular}{l c c c}
				\toprule
				Alternative & Strong response $(0.30)$ & Moderate response $(0.45)$ & Weak response $(0.25)$ \\
				\midrule
				Social media campaign & 40 & 20 & 4 \\
				School visits & 34 & 24 & 12 \\
				Door-to-door campaign & 28 & 26 & 18 \\
				Hybrid campaign & 38 & 25 & 14 \\
				\bottomrule
			\end{tabular}
		\end{center}

		\subsection*{C6}
		\textbf{Maximum opportunity criterion (minimax regret).} Best payoff in each state.
		\[
		\begin{aligned}
		\text{Strong response gives } &\max\{40,34,28,38\}=40,\\
		\text{Moderate response gives } &\max\{20,24,26,25\}=26,\\
		\text{Weak response gives } &\max\{4,12,18,14\}=18.
		\end{aligned}
		\]

		Regret table.
		\[
		\begin{array}{lcccc}
			\toprule
			\text{Alternative} & \text{Strong} & \text{Moderate} & \text{Weak} & \text{Maximum regret}\\
			\midrule
			\text{Social media campaign} & 40-40=0 & 26-20=6 & 18-4=14 & 14\\
			\text{School visits} & 40-34=6 & 26-24=2 & 18-12=6 & 6\\
			\text{Door-to-door campaign} & 40-28=12 & 26-26=0 & 18-18=0 & 12\\
			\text{Hybrid campaign} & 40-38=2 & 26-25=1 & 18-14=4 & 4\\
			\bottomrule
		\end{array}
		\]

		Minimax regret choice is \emph{Hybrid campaign} because its maximum regret is $4$.

		\item
		\subsection*{Vacation package allocation for coastal tourism demand}
		A coastal tourism office is deciding how to allocate vacation packages. It can choose an adventure-heavy plan, a family-balanced plan, a culture-focused plan, or a local day-trip plan. Seasonal demand can be peak, normal, off-season, or slump with probabilities $0.30$, $0.30$, $0.25$, and $0.15$. Profits are in hundreds of dollars.

		\begin{center}
			\textit{Payoff table}\\
			\begin{tabular}{l c c c c}
				\toprule
				Alternative & Peak $(0.30)$ & Normal $(0.30)$ & Off-season $(0.25)$ & Slump $(0.15)$ \\
				\midrule
				Adventure-heavy plan & 72 & 44 & 18 & -6 \\
				Family-balanced plan & 64 & 50 & 28 & 8 \\
				Culture-focused plan & 56 & 48 & 34 & 14 \\
				Local day-trip plan & 46 & 42 & 32 & 20 \\
				\bottomrule
			\end{tabular}
		\end{center}

		\subsection*{C6}
		\textbf{Maximum opportunity criterion (minimax regret).} Best payoff in each state.
		\[
		\begin{aligned}
		\text{Peak season gives } &\max\{72,64,56,46\}=72,\\
		\text{Normal season gives } &\max\{44,50,48,42\}=50,\\
		\text{Off-season gives } &\max\{18,28,34,32\}=34,\\
		\text{Slump season gives } &\max\{-6,8,14,20\}=20.
		\end{aligned}
		\]

		Regret table.
		\[
		\begin{array}{lccccc}
			\toprule
			\text{Alternative} & \text{Peak} & \text{Normal} & \text{Off-season} & \text{Slump} & \text{Maximum regret}\\
			\midrule
			\text{Adventure-heavy plan} & 72-72=0 & 50-44=6 & 34-18=16 & 20-(-6)=26 & 26\\
			\text{Family-balanced plan} & 72-64=8 & 50-50=0 & 34-28=6 & 20-8=12 & 12\\
			\text{Culture-focused plan} & 72-56=16 & 50-48=2 & 34-34=0 & 20-14=6 & 16\\
			\text{Local day-trip plan} & 72-46=26 & 50-42=8 & 34-32=2 & 20-20=0 & 26\\
			\bottomrule
		\end{array}
		\]

		Minimax regret choice is \emph{Family-balanced plan} because its maximum regret is $12$.

		\item
		\subsection*{Regional bus fleet scheduling under demand uncertainty}
		A regional bus company must select a fleet scheduling strategy for the next quarter. It can use an express-heavy plan, a balanced plan, a cost-control plan, or a flexible leasing plan. Passenger demand may be very high, high, moderate, low, or very low with probabilities $0.15$, $0.25$, $0.30$, $0.20$, and $0.10$. Profits are in hundreds of dollars.

		\begin{center}
			\textit{Payoff table}\\
			\begin{tabular}{l c c c c c}
				\toprule
				Alternative & Very high $(0.15)$ & High $(0.25)$ & Moderate $(0.30)$ & Low $(0.20)$ & Very low $(0.10)$ \\
				\midrule
				Express-heavy plan & 90 & 70 & 46 & 18 & -8 \\
				Balanced plan & 82 & 68 & 52 & 30 & 10 \\
				Cost-control plan & 70 & 60 & 50 & 36 & 22 \\
				Flexible leasing plan & 76 & 64 & 54 & 32 & 16 \\
				\bottomrule
			\end{tabular}
		\end{center}

		\subsection*{C6}
		\textbf{Maximum opportunity criterion (minimax regret).} Best payoff in each state.
		\[
		\begin{aligned}
		\text{Very high demand gives } &\max\{90,82,70,76\}=90,\\
		\text{High demand gives } &\max\{70,68,60,64\}=70,\\
		\text{Moderate demand gives } &\max\{46,52,50,54\}=54,\\
		\text{Low demand gives } &\max\{18,30,36,32\}=36,\\
		\text{Very low demand gives } &\max\{-8,10,22,16\}=22.
		\end{aligned}
		\]

		Regret table.
		\[
		\begin{array}{lcccccc}
			\toprule
			\text{Alternative} & \text{Very high} & \text{High} & \text{Moderate} & \text{Low} & \text{Very low} & \text{Maximum regret}\\
			\midrule
			\text{Express-heavy plan} & 90-90=0 & 70-70=0 & 54-46=8 & 36-18=18 & 22-(-8)=30 & 30\\
			\text{Balanced plan} & 90-82=8 & 70-68=2 & 54-52=2 & 36-30=6 & 22-10=12 & 12\\
			\text{Cost-control plan} & 90-70=20 & 70-60=10 & 54-50=4 & 36-36=0 & 22-22=0 & 20\\
			\text{Flexible leasing plan} & 90-76=14 & 70-64=6 & 54-54=0 & 36-32=4 & 22-16=6 & 14\\
			\bottomrule
		\end{array}
		\]

		Minimax regret choice is \emph{Balanced plan} because its maximum regret is $12$.
	\end{ExamProblems}
\end{document}
