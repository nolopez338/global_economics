\documentclass[11pt]{article}
\usepackage[margin=1in]{geometry}
\usepackage{amsmath,amssymb}
\usepackage{booktabs}
\usepackage{tabularx}
\usepackage[hidelinks]{hyperref}
\setlength{\parindent}{0pt}
\setlength{\parskip}{6pt}
\begin{document}
	\section*{Practice Activity 2}
	Progressive decision-rule problems that move from fully explicit revenue and cost data to mixed and fixed-component modeling.\\
	\subsection*{Contents}
	\begin{itemize}
		\item \hyperref[problem1]{Problem 1 — Food Truck Menu Upgrade}
		\item \hyperref[problem2]{Problem 2 — School Fundraiser Ticket Plan}
		\item \hyperref[problem3]{Problem 3 — City Bike-Share Pricing Strategy}
		\item \hyperref[problem4]{Problem 4 — Subscription Bundle Choices}
		\item \hyperref[problem5]{Problem 5 — Regional Distribution Plan}
		\item \hyperref[problem3a]{Problem 6 — Farmers Market Juice Stand}
		\item \hyperref[problem3b]{Problem 7 — After-School Tutoring Sessions}
		\item \hyperref[problem6]{Problem 8 — Art Fair Booth Options}
		\item \hyperref[problem7]{Problem 9 — Emergency Supply Contract}
	\end{itemize}
	
	\section*{Problem 1 — Food Truck Menu Upgrade}
	
	\label{problem1}
	
	\subsection*{Problem description}
	
	\label{problem1-problem}
	
	A food truck is choosing between two menu strategies for selling meals at lunchtime. Revenue comes from meal sales, and costs
	cover ingredients and staffing for the season.
	
	
	
	
	
	
	Two strategies are being compared:
	
	
	
	• \textbf{Expanded menu}
	
	
	
	• \textbf{Classic menu}
	
	
	
	
	
	\textbf{Given information (thousand USD):}
	
	
	
	States of nature:
	
	
	
	• Busy season (0.55)
	
	
	
	• Slow season (0.45)
	
	
	
	
	
	Revenue by state:
	
	
	
	• Busy season: Expanded 28, Classic 22
	
	
	
	• Slow season: Expanded 16, Classic 18
	
	
	
	
	
	Cost by state:
	
	
	
	• Busy season: Expanded 12, Classic 9
	
	
	
	• Slow season: Expanded 10, Classic 8
	
	
	
	
	
	
	\textbf{Question:} Compute profit for each state and decide using Maximin, Expected Value, and Maximax.
	
	\hyperref[problem1-criteria]{↑}
	
	\subsection*{Solution}
	
	\label{problem1-solution}
	
	\subsection*{Structured solution}
	
	
	
	
	
	(C2) Interpreting decision alternatives, events, consequences, and states
	+
	
	
	
	
	
	\hyperref[problem1-criteria]{↑}
	
	
	
	
	(C3) Building the payoff table from revenue and cost data
	+
	
	
	Revenue by state (thousand USD):
	
	Cost by state (thousand USD):
	
	Payoff table (profit = revenue - costs):
	\begin{center}
		\begin{tabular}{llll}
			\toprule
			State & Scenario & Expanded & Classic \\
			\midrule
			S1\newline (0.55) & Busy season\newline (0.55) & 28 - 12 = 16 & 22 - 9 = 13 \\
			S2\newline (0.45) & Slow season\newline (0.45) & 16 - 10 = 6 & 18 - 8 = 10 \\
			\bottomrule
		\end{tabular}
	\end{center}
	
	
	Final payoff table (thousand USD):
	\begin{center}
		\begin{tabular}{lll}
			\toprule
			State & Expanded & Classic \\
			\midrule
			S1\newline (0.55) & 16 & 13 \\
			S2\newline (0.45) & 6 & 10 \\
			\bottomrule
		\end{tabular}
	\end{center}
	
	
	
	
	\hyperref[problem1-criteria]{↑}
	
	
	
	
	(C4) Applying the Maximax criterion
	+
	
	
	Maximax selects the alternative with the highest best payoff.
	\[
	\begin{aligned}
		\max_i \text{Payoff}(\text{Expanded}, S_i) &= \max\{16, 6\} = 16 \\
		\max_i \text{Payoff}(\text{Classic}, S_i) &= \max\{13, 10\} = 13
	\end{aligned}
	\]
	\[
	\max\{16, 13\} = 16
	\]
	The maximax (risk-seeking) decision is \emph{expand the menu}.
	
	
	\hyperref[problem1-criteria]{↑}
	
	
	
	
	(C5) Applying the Maximin criterion
	+
	
	
	Maximin compares each alternative's worst payoff.
	\[
	\begin{aligned}
		\min_i \text{Payoff}(\text{Expanded}, S_i) &= \min\{16, 6\} = 6 \\
		\min_i \text{Payoff}(\text{Classic}, S_i) &= \min\{13, 10\} = 10
	\end{aligned}
	\]
	\[
	\max\{6, 10\} = 10
	\]
	The maximin (risk-averse) decision is \emph{keep the classic menu}.
	
	
	\hyperref[problem1-criteria]{↑}
	
	
	
	
	(C1) Formulating the decision problem for an optimal choice
	+
	
	
	The truck wants the menu that maximizes expected profit.
	\[
	\begin{aligned}
		EV_{\text{Expanded}} &= 0.55(16) + 0.45(6) \\
		&= 8.8 + 2.7 = 11.5 \\
		EV_{\text{Classic}} &= 0.55(13) + 0.45(10) \\
		&= 7.15 + 4.5 = 11.65
	\end{aligned}
	\]
	\[
	\max\{11.5, 11.65\} = 11.65
	\]
	
	Summary of recommendations across decision criteria
	\begin{itemize}
		\item Maximax favors \emph{expanded menu} (best payoff of 16).
		\item Maximin favors \emph{classic menu} (best worst-case payoff of 10).
		\item Expected value favors \emph{classic menu} (EV of 11.65).
	\end{itemize}
	
	
	
	The expanded menu offers the largest upside, while the classic menu is safer and slightly better on average.
	
	
	
	\hyperref[problem1-criteria]{↑}
	\section*{Problem 2 — School Fundraiser Ticket Plan}
	
	\label{problem2}
	
	\subsection*{Problem description}
	
	\label{problem2-problem}
	
	A school council is choosing a ticket plan for a fundraising concert. Revenue comes from ticket sales and costs include
	venue, security, and refreshments.
	
	
	
	
	
	
	Two ticket options are being compared:
	
	
	
	• \textbf{Standard tickets}
	
	
	
	• \textbf{VIP tickets}
	
	
	
	
	
	\textbf{Given information (thousand USD):}
	
	
	
	States of nature:
	
	
	
	• High attendance (0.30)
	
	
	
	• Expected attendance (0.45)
	
	
	
	• Low attendance (0.25)
	
	
	
	
	
	Revenue by state:
	
	
	
	• High: Standard 26, VIP 34
	
	
	
	• Expected: Standard 19, VIP 25
	
	
	
	• Low: Standard 13, VIP 15
	
	
	
	
	
	Cost by state:
	
	
	
	• High: Standard 10, VIP 16
	
	
	
	• Expected: Standard 10, VIP 14
	
	
	
	• Low: Standard 8, VIP 12
	
	
	
	
	
	
	\textbf{Question:} Compute profits and decide using Maximin, Expected Value, and Maximax.
	
	\hyperref[problem2-criteria]{↑}
	
	\subsection*{Solution}
	
	\label{problem2-solution}
	
	\subsection*{Structured solution}
	
	
	
	
	
	(C2) Interpreting decision alternatives, events, consequences, and states
	+
	
	
	
	
	
	\hyperref[problem2-criteria]{↑}
	
	
	
	
	(C3) Building the payoff table from revenue and cost data
	+
	
	
	Revenue by attendance state (thousand USD):
	
	Cost by attendance state (thousand USD):
	
	Payoff table (profit = revenue - costs):
	\begin{center}
		\begin{tabular}{lll}
			\toprule
			State & Standard & VIP \\
			\midrule
			S1 (High)\newline (0.30) & 26 - 10 = 16 & 34 - 16 = 18 \\
			S2 (Expected)\newline (0.45) & 19 - 10 = 9 & 25 - 14 = 11 \\
			S3 (Low)\newline (0.25) & 13 - 8 = 5 & 15 - 12 = 3 \\
			\bottomrule
		\end{tabular}
	\end{center}
	
	
	Final payoff table (thousand USD):
	\begin{center}
		\begin{tabular}{lll}
			\toprule
			State & Standard & VIP \\
			\midrule
			S1\newline (0.30) & 16 & 18 \\
			S2\newline (0.45) & 9 & 11 \\
			S3\newline (0.25) & 5 & 3 \\
			\bottomrule
		\end{tabular}
	\end{center}
	
	
	
	
	\hyperref[problem2-criteria]{↑}
	
	
	
	
	(C4) Applying the Maximax criterion
	+
	
	
	Maximax selects the alternative with the highest best payoff.
	\[
	\begin{aligned}
		\max_i \text{Payoff}(\text{Standard}, S_i) &= \max\{16, 9, 5\} = 16 \\
		\max_i \text{Payoff}(\text{VIP}, S_i) &= \max\{18, 11, 3\} = 18
	\end{aligned}
	\]
	\[
	\max\{16, 18\} = 18
	\]
	The maximax (risk-seeking) decision is \emph{VIP tickets}.
	
	
	\hyperref[problem2-criteria]{↑}
	
	
	
	
	(C5) Applying the Maximin criterion
	+
	
	
	Maximin compares each alternative's worst payoff.
	\[
	\begin{aligned}
		\min_i \text{Payoff}(\text{Standard}, S_i) &= \min\{16, 9, 5\} = 5 \\
		\min_i \text{Payoff}(\text{VIP}, S_i) &= \min\{18, 11, 3\} = 3
	\end{aligned}
	\]
	\[
	\max\{5, 3\} = 5
	\]
	The maximin (risk-averse) decision is \emph{standard tickets}.
	
	
	\hyperref[problem2-criteria]{↑}
	
	
	
	
	(C1) Formulating the decision problem for an optimal choice
	+
	
	
	The council wants the ticket plan with the highest expected profit.
	\[
	\begin{aligned}
		EV_{\text{Standard}} &= 0.30(16) + 0.45(9) + 0.25(5) \\
		&= 4.8 + 4.05 + 1.25 = 10.1 \\
		EV_{\text{VIP}} &= 0.30(18) + 0.45(11) + 0.25(3) \\
		&= 5.4 + 4.95 + 0.75 = 11.1
	\end{aligned}
	\]
	\[
	\max\{10.1, 11.1\} = 11.1
	\]
	
	Summary of recommendations across decision criteria
	\begin{itemize}
		\item Maximax favors \emph{VIP tickets} (best payoff of 18).
		\item Maximin favors \emph{standard tickets} (best worst-case payoff of 5).
		\item Expected value favors \emph{VIP tickets} (EV of 11.1).
	\end{itemize}
	
	
	
	VIP tickets have the strongest upside and average profit, while standard tickets are safer in the worst case.
	
	
	
	\hyperref[problem2-criteria]{↑}
	\section*{Problem 3 — City Bike-Share Pricing Strategy}
	
	\label{problem3}
	
	\subsection*{Problem description}
	
	\label{problem3-problem}
	
	A city bike-share program is choosing a pricing strategy for memberships. Revenue comes from rider subscriptions, and costs
	cover maintenance and station operations.
	
	
	
	
	
	
	Three pricing options are being compared:
	
	
	
	• \textbf{Premium access}
	
	
	
	• \textbf{Standard access}
	
	
	
	• \textbf{Partner access}
	
	
	
	
	
	\textbf{Given information (thousand USD):}
	
	
	
	States of nature:
	
	
	
	• High ridership (0.50)
	
	
	
	• Low ridership (0.50)
	
	
	
	
	
	Revenue by state:
	
	
	
	• High: Premium 120, Standard 100, Partner 85
	
	
	
	• Low: Premium 80, Standard 75, Partner 70
	
	
	
	
	
	Cost by state:
	
	
	
	• High: Premium 60, Standard 50, Partner 40
	
	
	
	• Low: Premium 55, Standard 45, Partner 38
	
	
	
	
	
	
	\textbf{Question:} Compute profits and decide using Maximin, Expected Value, and Maximax.
	
	\hyperref[problem3-criteria]{↑}
	
	\subsection*{Solution}
	
	\label{problem3-solution}
	
	\subsection*{Structured solution}
	
	
	
	
	
	(C2) Interpreting decision alternatives, events, consequences, and states
	+
	
	
	
	
	
	\hyperref[problem3-criteria]{↑}
	
	
	
	
	(C3) Building the payoff table from revenue and cost data
	+
	
	
	Revenue by ridership state (thousand USD):
	
	Cost by ridership state (thousand USD):
	
	Payoff table (profit = revenue - costs):
	\begin{center}
		\begin{tabular}{llll}
			\toprule
			State & Premium & Standard & Partner \\
			\midrule
			S1 (High)\newline (0.50) & 120 - 60 = 60 & 100 - 50 = 50 & 85 - 40 = 45 \\
			S2 (Low)\newline (0.50) & 80 - 55 = 25 & 75 - 45 = 30 & 70 - 38 = 32 \\
			\bottomrule
		\end{tabular}
	\end{center}
	
	
	Final payoff table (thousand USD):
	\begin{center}
		\begin{tabular}{llll}
			\toprule
			State & Premium & Standard & Partner \\
			\midrule
			S1\newline (0.50) & 60 & 50 & 45 \\
			S2\newline (0.50) & 25 & 30 & 32 \\
			\bottomrule
		\end{tabular}
	\end{center}
	
	
	
	
	\hyperref[problem3-criteria]{↑}
	
	
	
	
	(C4) Applying the Maximax criterion
	+
	
	
	Maximax compares the best payoff for each pricing option.
	\[
	\begin{aligned}
		\max_i \text{Payoff}(\text{Premium}, S_i) &= \max\{60, 25\} = 60 \\
		\max_i \text{Payoff}(\text{Standard}, S_i) &= \max\{50, 30\} = 50 \\
		\max_i \text{Payoff}(\text{Partner}, S_i) &= \max\{45, 32\} = 45
	\end{aligned}
	\]
	\[
	\max\{60, 50, 45\} = 60
	\]
	The maximax (risk-seeking) choice is \emph{premium access}.
	
	
	\hyperref[problem3-criteria]{↑}
	
	
	
	
	(C5) Applying the Maximin criterion
	+
	
	
	Maximin selects the highest worst-case payoff.
	\[
	\begin{aligned}
		\min_i \text{Payoff}(\text{Premium}, S_i) &= \min\{60, 25\} = 25 \\
		\min_i \text{Payoff}(\text{Standard}, S_i) &= \min\{50, 30\} = 30 \\
		\min_i \text{Payoff}(\text{Partner}, S_i) &= \min\{45, 32\} = 32
	\end{aligned}
	\]
	\[
	\max\{25, 30, 32\} = 32
	\]
	The maximin (risk-averse) choice is \emph{partner access}.
	
	
	\hyperref[problem3-criteria]{↑}
	
	
	
	
	(C1) Formulating the decision problem for an optimal choice
	+
	
	
	The program wants the pricing plan with the highest expected profit.
	\[
	\begin{aligned}
		EV_{\text{Premium}} &= 0.50(60) + 0.50(25) = 42.5 \\
		EV_{\text{Standard}} &= 0.50(50) + 0.50(30) = 40 \\
		EV_{\text{Partner}} &= 0.50(45) + 0.50(32) = 38.5
	\end{aligned}
	\]
	\[
	\max\{42.5, 40, 38.5\} = 42.5
	\]
	
	Summary of recommendations across decision criteria
	\begin{itemize}
		\item Maximax favors \emph{premium access} (best payoff of 60).
		\item Maximin favors \emph{partner access} (best worst-case payoff of 32).
		\item Expected value favors \emph{premium access} (EV of 42.5).
	\end{itemize}
	
	
	
	Premium access wins on upside and average profit, while partner access provides stronger downside protection.
	
	
	
	\hyperref[problem3-criteria]{↑}
	\section*{Problem 4 — Subscription Bundle Choices}
	
	\label{problem4}
	
	\subsection*{Problem description}
	
	\label{problem4-problem}
	
	A streaming service is choosing among subscription bundles. Revenue comes from subscription sales, while costs are driven by
	licensing fees that are known for each demand state.
	
	
	
	
	
	
	Strategies:
	
	
	
	1. \textbf{Premium bundle}
	
	
	
	2. \textbf{Standard bundle}
	
	
	
	3. \textbf{Budget bundle}
	
	
	
	
	
	\textbf{Given information:}
	
	
	
	States of nature:
	
	
	
	• Strong sign-ups (0.55)
	
	
	
	• Weak sign-ups (0.45)
	
	
	
	
	
	Number of subscriptions (in thousands):
	
	
	
	• Strong: Premium 12, Standard 15, Budget 18
	
	
	
	• Weak: Premium 7, Standard 9, Budget 12
	
	
	
	
	
	Price per subscription (USD):
	
	
	
	• Premium 18, Standard 14, Budget 10
	
	
	
	
	
	Cost by state (thousand USD):
	
	
	
	• Strong: Premium 140, Standard 120, Budget 95
	
	
	
	• Weak: Premium 120, Standard 105, Budget 85
	
	
	
	
	
	
	\textbf{What you must construct:} Revenue for each bundle and state, then profit.
	
	
	
	
	
	
	\textbf{Question:} Compute profits and decide using Maximin, Expected Value, and Maximax.
	
	\hyperref[problem4-criteria]{↑}
	
	\subsection*{Solution}
	
	\label{problem4-solution}
	
	\subsection*{Structured solution}
	
	
	
	
	
	(C2) Interpreting decision alternatives, events, consequences, and states
	+
	
	
	
	
	
	\hyperref[problem4-criteria]{↑}
	
	
	
	
	(C3) Building the payoff table from revenue and cost data
	+
	
	
	Subscriptions by state (thousands):
	
	Price per subscription (USD):
	
	Revenue constructed (subscriptions × price, thousand USD):
	\begin{center}
		\begin{tabular}{llll}
			\toprule
			State & Premium & Standard & Budget \\
			\midrule
			Strong\newline (0.55) & 12 × 18 = 216 & 15 × 14 = 210 & 18 × 10 = 180 \\
			Weak\newline (0.45) & 7 × 18 = 126 & 9 × 14 = 126 & 12 × 10 = 120 \\
			\bottomrule
		\end{tabular}
	\end{center}
	
	
	Cost by state (thousand USD):
	\begin{center}
		\begin{tabular}{llll}
			\toprule
			State & Premium & Standard & Budget \\
			\midrule
			Strong\newline (0.55) & 140 & 120 & 95 \\
			Weak\newline (0.45) & 120 & 105 & 85 \\
			\bottomrule
		\end{tabular}
	\end{center}
	
	
	Payoff table (profit = revenue - costs):
	\begin{center}
		\begin{tabular}{llll}
			\toprule
			State & Premium & Standard & Budget \\
			\midrule
			S1 (Strong)\newline (0.55) & 216 - 140 = 76 & 210 - 120 = 90 & 180 - 95 = 85 \\
			S2 (Weak)\newline (0.45) & 126 - 120 = 6 & 126 - 105 = 21 & 120 - 85 = 35 \\
			\bottomrule
		\end{tabular}
	\end{center}
	
	
	Final payoff table (thousand USD):
	\begin{center}
		\begin{tabular}{llll}
			\toprule
			State & Premium & Standard & Budget \\
			\midrule
			S1\newline (0.55) & 76 & 90 & 85 \\
			S2\newline (0.45) & 6 & 21 & 35 \\
			\bottomrule
		\end{tabular}
	\end{center}
	
	
	
	
	\hyperref[problem4-criteria]{↑}
	
	
	
	
	(C4) Applying the Maximax criterion
	+
	
	
	Maximax compares the best payoff for each bundle.
	\[
	\begin{aligned}
		\max_i \text{Payoff}(\text{Premium}, S_i) &= \max\{76, 6\} = 76 \\
		\max_i \text{Payoff}(\text{Standard}, S_i) &= \max\{90, 21\} = 90 \\
		\max_i \text{Payoff}(\text{Budget}, S_i) &= \max\{85, 35\} = 85
	\end{aligned}
	\]
	\[
	\max\{76, 90, 85\} = 90
	\]
	The maximax (risk-seeking) choice is the \emph{standard bundle}.
	
	
	\hyperref[problem4-criteria]{↑}
	
	
	
	
	(C5) Applying the Maximin criterion
	+
	
	
	Maximin selects the highest worst-case payoff.
	\[
	\begin{aligned}
		\min_i \text{Payoff}(\text{Premium}, S_i) &= \min\{76, 6\} = 6 \\
		\min_i \text{Payoff}(\text{Standard}, S_i) &= \min\{90, 21\} = 21 \\
		\min_i \text{Payoff}(\text{Budget}, S_i) &= \min\{85, 35\} = 35
	\end{aligned}
	\]
	\[
	\max\{6, 21, 35\} = 35
	\]
	The maximin (risk-averse) choice is the \emph{budget bundle}.
	
	
	\hyperref[problem4-criteria]{↑}
	
	
	
	
	(C1) Formulating the decision problem for an optimal choice
	+
	
	
	The service wants the bundle with the highest expected profit.
	\[
	\begin{aligned}
		EV_{\text{Premium}} &= 0.55(76) + 0.45(6) \\
		&= 41.8 + 2.7 = 44.5 \\
		EV_{\text{Standard}} &= 0.55(90) + 0.45(21) \\
		&= 49.5 + 9.45 = 58.95 \\
		EV_{\text{Budget}} &= 0.55(85) + 0.45(35) \\
		&= 46.75 + 15.75 = 62.5
	\end{aligned}
	\]
	\[
	\max\{44.5, 58.95, 62.5\} = 62.5
	\]
	
	Summary of recommendations across decision criteria
	\begin{itemize}
		\item Maximax favors the \emph{standard bundle} (best payoff of 90).
		\item Maximin favors the \emph{budget bundle} (best worst-case payoff of 35).
		\item Expected value favors the \emph{budget bundle} (EV of 62.5).
	\end{itemize}
	
	
	
	The budget bundle offers the strongest average outcome, while standard wins on upside and budget protects the downside.
	
	
	
	\hyperref[problem4-criteria]{↑}
	\section*{Problem 5 — Regional Distribution Plan}
	
	\label{problem5}
	
	\subsection*{Problem description}
	
	\label{problem5-problem}
	
	A logistics firm is choosing between two distribution plans for delivering orders. Revenue comes from delivery contracts,
	while costs depend on the number of deliveries and the cost per delivery.
	
	
	
	
	
	
	Two alternatives are being compared:
	
	
	
	• \textbf{Central depot}
	
	
	
	• \textbf{Partner carriers}
	
	
	
	
	
	\textbf{Given information:}
	
	
	
	States of nature:
	
	
	
	• High demand (0.30)
	
	
	
	• Medium demand (0.45)
	
	
	
	• Low demand (0.25)
	
	
	
	
	
	Revenue by state (thousand USD):
	
	
	
	• High: Central 245, Partner 215
	
	
	
	• Medium: Central 174, Partner 165
	
	
	
	• Low: Central 117, Partner 117
	
	
	
	
	
	Number of deliveries (thousands):
	
	
	
	• High: Central 40, Partner 40
	
	
	
	• Medium: Central 30, Partner 30
	
	
	
	• Low: Central 20, Partner 20
	
	
	
	
	
	Cost per delivery (USD):
	
	
	
	• Central: High 4.0, Medium 3.8, Low 3.6
	
	
	
	• Partner: High 3.5, Medium 3.3, Low 3.1
	
	
	
	
	
	
	\textbf{What you must construct:} Costs for each alternative and state, then profit.
	
	
	
	
	
	
	\textbf{Question:} Compute profits and decide using Maximin, Expected Value, and Maximax.
	
	\hyperref[problem5-criteria]{↑}
	
	\subsection*{Solution}
	
	\label{problem5-solution}
	
	\subsection*{Structured solution}
	
	
	
	
	
	(C2) Interpreting decision alternatives, events, consequences, and states
	+
	
	
	
	
	
	\hyperref[problem5-criteria]{↑}
	
	
	
	
	(C3) Building the payoff table from revenue and cost data
	+
	
	
	Revenue by demand state (thousand USD):
	
	Deliveries and cost per delivery:
	
	Costs constructed (deliveries × cost per delivery, thousand USD):
	\begin{center}
		\begin{tabular}{lll}
			\toprule
			State & Central & Partner \\
			\midrule
			High\newline (0.30) & 40 × 4.0 = 160 & 40 × 3.5 = 140 \\
			Medium\newline (0.45) & 30 × 3.8 = 114 & 30 × 3.3 = 99 \\
			Low\newline (0.25) & 20 × 3.6 = 72 & 20 × 3.1 = 62 \\
			\bottomrule
		\end{tabular}
	\end{center}
	
	
	Payoff table (profit = revenue - costs):
	\begin{center}
		\begin{tabular}{lll}
			\toprule
			State & Central & Partner \\
			\midrule
			S1 (High)\newline (0.30) & 245 - 160 = 85 & 215 - 140 = 75 \\
			S2 (Medium)\newline (0.45) & 174 - 114 = 60 & 165 - 99 = 66 \\
			S3 (Low)\newline (0.25) & 117 - 72 = 45 & 117 - 62 = 55 \\
			\bottomrule
		\end{tabular}
	\end{center}
	
	
	Final payoff table (thousand USD):
	\begin{center}
		\begin{tabular}{lll}
			\toprule
			State & Central & Partner \\
			\midrule
			S1\newline (0.30) & 85 & 75 \\
			S2\newline (0.45) & 60 & 66 \\
			S3\newline (0.25) & 45 & 55 \\
			\bottomrule
		\end{tabular}
	\end{center}
	
	
	
	
	\hyperref[problem5-criteria]{↑}
	
	
	
	
	(C4) Applying the Maximax criterion
	+
	
	
	Maximax compares the best profit for each plan.
	\[
	\begin{aligned}
		\max_i \text{Payoff}(\text{Central}, S_i) &= \max\{85, 60, 45\} = 85 \\
		\max_i \text{Payoff}(\text{Partner}, S_i) &= \max\{75, 66, 55\} = 75
	\end{aligned}
	\]
	\[
	\max\{85, 75\} = 85
	\]
	The maximax (risk-seeking) choice is the \emph{central depot} plan.
	
	
	\hyperref[problem5-criteria]{↑}
	
	
	
	
	(C5) Applying the Maximin criterion
	+
	
	
	Maximin picks the highest of the worst-case profits.
	\[
	\begin{aligned}
		\min_i \text{Payoff}(\text{Central}, S_i) &= \min\{85, 60, 45\} = 45 \\
		\min_i \text{Payoff}(\text{Partner}, S_i) &= \min\{75, 66, 55\} = 55
	\end{aligned}
	\]
	\[
	\max\{45, 55\} = 55
	\]
	The maximin (risk-averse) choice is the \emph{partner carriers} plan.
	
	
	\hyperref[problem5-criteria]{↑}
	
	
	
	
	(C1) Formulating the decision problem for an optimal choice
	+
	
	
	The firm compares expected profits using the state probabilities.
	\[
	\begin{aligned}
		EV_{\text{Central}} &= 0.30(85) + 0.45(60) + 0.25(45) \\
		&= 25.5 + 27 + 11.25 = 63.75 \\
		EV_{\text{Partner}} &= 0.30(75) + 0.45(66) + 0.25(55) \\
		&= 22.5 + 29.7 + 13.75 = 65.95
	\end{aligned}
	\]
	\[
	\max\{63.75, 65.95\} = 65.95
	\]
	
	Summary of recommendations across decision criteria
	\begin{itemize}
		\item Maximax favors the \emph{central depot} plan (best payoff of 85).
		\item Maximin favors the \emph{partner carriers} plan (best worst-case payoff of 55).
		\item Expected value favors the \emph{partner carriers} plan (EV of 65.95).
	\end{itemize}
	
	
	
	Partner carriers provide the best average outcome and the strongest downside protection, while the central depot has the highest upside.
	
	
	
	\hyperref[problem5-criteria]{↑}
	\section*{Problem 6 — Farmers Market Juice Stand}
	
	\label{problem3a}
	
	\subsection*{Problem description}
	
	\label{problem3a-problem}
	
	A student group is running a juice stand at a farmers market. Revenue comes from selling bottles of juice, and costs depend on
	how many bottles are produced and the cost per bottle.
	
	
	
	
	
	
	Two production options are being compared:
	
	
	
	• \textbf{Small batch}
	
	
	
	• \textbf{Large batch}
	
	
	
	
	
	\textbf{Given information:}
	
	
	
	States of nature:
	
	
	
	• High turnout (0.55)
	
	
	
	• Low turnout (0.45)
	
	
	
	
	
	Revenue by state (thousand USD):
	
	
	
	• High turnout: Small 34, Large 46
	
	
	
	• Low turnout: Small 24, Large 26
	
	
	
	
	
	Bottles sold (thousands):
	
	
	
	• High turnout: Small 8, Large 12
	
	
	
	• Low turnout: Small 6, Large 9
	
	
	
	
	
	Cost per bottle (USD):
	
	
	
	• Small 2.5, Large 2.6
	
	
	
	
	
	
	\textbf{What you must construct:} Costs for each option and state, then profit.
	
	
	
	
	
	
	\textbf{Question:} Compute profits and decide using Maximin, Expected Value, and Maximax.
	
	\hyperref[problem3a-criteria]{↑}
	
	\subsection*{Solution}
	
	\label{problem3a-solution}
	
	\subsection*{Structured solution}
	
	
	
	
	
	(C2) Interpreting decision alternatives, events, consequences, and states
	+
	
	
	
	
	
	\hyperref[problem3a-criteria]{↑}
	
	
	
	
	(C3) Building the payoff table from revenue and cost data
	+
	
	
	Revenue by turnout state (thousand USD):
	
	Bottles sold and cost per bottle:
	
	Costs constructed (bottles × cost per bottle, thousand USD):
	\begin{center}
		\begin{tabular}{lll}
			\toprule
			State & Small & Large \\
			\midrule
			High\newline (0.55) & 8 × 2.5 = 20 & 12 × 2.6 = 31.2 \\
			Low\newline (0.45) & 6 × 2.5 = 15 & 9 × 2.6 = 23.4 \\
			\bottomrule
		\end{tabular}
	\end{center}
	
	
	Payoff table (profit = revenue - costs):
	\begin{center}
		\begin{tabular}{lll}
			\toprule
			State & Small & Large \\
			\midrule
			S1 (High)\newline (0.55) & 34 - 20 = 14 & 46 - 31.2 = 14.8 \\
			S2 (Low)\newline (0.45) & 24 - 15 = 9 & 26 - 23.4 = 2.6 \\
			\bottomrule
		\end{tabular}
	\end{center}
	
	
	Final payoff table (thousand USD):
	\begin{center}
		\begin{tabular}{lll}
			\toprule
			State & Small & Large \\
			\midrule
			S1\newline (0.55) & 14 & 14.8 \\
			S2\newline (0.45) & 9 & 2.6 \\
			\bottomrule
		\end{tabular}
	\end{center}
	
	
	
	
	\hyperref[problem3a-criteria]{↑}
	
	
	
	
	(C4) Applying the Maximax criterion
	+
	
	
	Maximax selects the alternative with the highest best payoff.
	\[
	\begin{aligned}
		\max_i \text{Payoff}(\text{Small}, S_i) &= \max\{14, 9\} = 14 \\
		\max_i \text{Payoff}(\text{Large}, S_i) &= \max\{14.8, 2.6\} = 14.8
	\end{aligned}
	\]
	\[
	\max\{14, 14.8\} = 14.8
	\]
	The maximax (risk-seeking) decision is \emph{large batch}.
	
	
	\hyperref[problem3a-criteria]{↑}
	
	
	
	
	(C5) Applying the Maximin criterion
	+
	
	
	Maximin compares each alternative's worst payoff.
	\[
	\begin{aligned}
		\min_i \text{Payoff}(\text{Small}, S_i) &= \min\{14, 9\} = 9 \\
		\min_i \text{Payoff}(\text{Large}, S_i) &= \min\{14.8, 2.6\} = 2.6
	\end{aligned}
	\]
	\[
	\max\{9, 2.6\} = 9
	\]
	The maximin (risk-averse) decision is \emph{small batch}.
	
	
	\hyperref[problem3a-criteria]{↑}
	
	
	
	
	(C1) Formulating the decision problem for an optimal choice
	+
	
	
	The group wants the option with the highest expected profit.
	\[
	\begin{aligned}
		EV_{\text{Small}} &= 0.55(14) + 0.45(9) \\
		&= 7.7 + 4.05 = 11.75 \\
		EV_{\text{Large}} &= 0.55(14.8) + 0.45(2.6) \\
		&= 8.14 + 1.17 = 9.31
	\end{aligned}
	\]
	\[
	\max\{11.75, 9.31\} = 11.75
	\]
	
	Summary of recommendations across decision criteria
	\begin{itemize}
		\item Maximax favors \emph{large batch} (best payoff of 14.8).
		\item Maximin favors \emph{small batch} (best worst-case payoff of 9).
		\item Expected value favors \emph{small batch} (EV of 11.75).
	\end{itemize}
	
	
	
	The large batch has the highest upside, but the small batch provides stronger protection and the best average profit.
	
	
	
	\hyperref[problem3a-criteria]{↑}
	\section*{Problem 7 — After-School Tutoring Sessions}
	
	\label{problem3b}
	
	\subsection*{Problem description}
	
	\label{problem3b-problem}
	
	A tutoring center is choosing between two delivery modes for paid tutoring sessions. Revenue comes from session fees, and costs
	depend on the number of sessions and the cost per session.
	
	
	
	
	
	
	Two alternatives are being compared:
	
	
	
	• \textbf{In-person sessions}
	
	
	
	• \textbf{Online sessions}
	
	
	
	
	
	\textbf{Given information:}
	
	
	
	States of nature:
	
	
	
	• High enrollment (0.60)
	
	
	
	• Low enrollment (0.40)
	
	
	
	
	
	Sessions delivered (thousands):
	
	
	
	• In-person: High 2.2, Low 1.0
	
	
	
	• Online: High 2.0, Low 1.6
	
	
	
	
	
	Fee per session (USD):
	
	
	
	• In-person 35, Online 27
	
	
	
	
	
	Cost per session (USD):
	
	
	
	• In-person 20, Online 12
	
	
	
	
	
	
	\textbf{What you must construct:} Revenue and cost for each alternative and state, then profit.
	
	
	
	
	
	
	\textbf{Question:} Compute profits and decide using Maximin, Expected Value, and Maximax.
	
	\hyperref[problem3b-criteria]{↑}
	
	\subsection*{Solution}
	
	\label{problem3b-solution}
	
	\subsection*{Structured solution}
	
	
	
	
	
	(C2) Interpreting decision alternatives, events, consequences, and states
	+
	
	
	
	
	
	\hyperref[problem3b-criteria]{↑}
	
	
	
	
	(C3) Building the payoff table from revenue and cost data
	+
	
	
	Sessions delivered (thousands):
	
	Fees and costs per session (USD):
	
	Revenue constructed (sessions × fee, thousand USD):
	\begin{center}
		\begin{tabular}{lll}
			\toprule
			State & In-person & Online \\
			\midrule
			High\newline (0.60) & 2.2 × 35 = 77 & 2.0 × 27 = 54 \\
			Low\newline (0.40) & 1.0 × 35 = 35 & 1.6 × 27 = 43.2 \\
			\bottomrule
		\end{tabular}
	\end{center}
	
	
	Costs constructed (sessions × cost, thousand USD):
	\begin{center}
		\begin{tabular}{lll}
			\toprule
			State & In-person & Online \\
			\midrule
			High\newline (0.60) & 2.2 × 20 = 44 & 2.0 × 12 = 24 \\
			Low\newline (0.40) & 1.0 × 20 = 20 & 1.6 × 12 = 19.2 \\
			\bottomrule
		\end{tabular}
	\end{center}
	
	
	Payoff table (profit = revenue - costs):
	\begin{center}
		\begin{tabular}{lll}
			\toprule
			State & In-person & Online \\
			\midrule
			S1 (High)\newline (0.60) & 77 - 44 = 33 & 54 - 24 = 30 \\
			S2 (Low)\newline (0.40) & 35 - 20 = 15 & 43.2 - 19.2 = 24 \\
			\bottomrule
		\end{tabular}
	\end{center}
	
	
	Final payoff table (thousand USD):
	\begin{center}
		\begin{tabular}{lll}
			\toprule
			State & In-person & Online \\
			\midrule
			S1\newline (0.60) & 33 & 30 \\
			S2\newline (0.40) & 15 & 24 \\
			\bottomrule
		\end{tabular}
	\end{center}
	
	
	
	
	\hyperref[problem3b-criteria]{↑}
	
	
	
	
	(C4) Applying the Maximax criterion
	+
	
	
	Maximax compares the best payoff for each mode.
	\[
	\begin{aligned}
		\max_i \text{Payoff}(\text{In-person}, S_i) &= \max\{33, 15\} = 33 \\
		\max_i \text{Payoff}(\text{Online}, S_i) &= \max\{30, 24\} = 30
	\end{aligned}
	\]
	\[
	\max\{33, 30\} = 33
	\]
	The maximax (risk-seeking) choice is \emph{in-person sessions}.
	
	
	\hyperref[problem3b-criteria]{↑}
	
	
	
	
	(C5) Applying the Maximin criterion
	+
	
	
	Maximin selects the highest worst-case payoff.
	\[
	\begin{aligned}
		\min_i \text{Payoff}(\text{In-person}, S_i) &= \min\{33, 15\} = 15 \\
		\min_i \text{Payoff}(\text{Online}, S_i) &= \min\{30, 24\} = 24
	\end{aligned}
	\]
	\[
	\max\{15, 24\} = 24
	\]
	The maximin (risk-averse) choice is \emph{online sessions}.
	
	
	\hyperref[problem3b-criteria]{↑}
	
	
	
	
	(C1) Formulating the decision problem for an optimal choice
	+
	
	
	The center wants the delivery mode with the highest expected profit.
	\[
	\begin{aligned}
		EV_{\text{In-person}} &= 0.60(33) + 0.40(15) = 19.8 + 6 = 25.8 \\
		EV_{\text{Online}} &= 0.60(30) + 0.40(24) = 18 + 9.6 = 27.6
	\end{aligned}
	\]
	\[
	\max\{25.8, 27.6\} = 27.6
	\]
	
	Summary of recommendations across decision criteria
	\begin{itemize}
		\item Maximax favors \emph{in-person sessions} (best payoff of 33).
		\item Maximin favors \emph{online sessions} (best worst-case payoff of 24).
		\item Expected value favors \emph{online sessions} (EV of 27.6).
	\end{itemize}
	
	
	
	In-person sessions have the highest upside, but online sessions provide better downside protection and the strongest average profit.
	
	
	
	\hyperref[problem3b-criteria]{↑}
	\section*{Problem 8 — Art Fair Booth Options}
	
	\label{problem6}
	
	\subsection*{Problem description}
	
	\label{problem6-problem}
	
	A student art club is choosing between two booth setups for an art fair. Revenue comes from selling art kits, while costs are
	fixed setup fees that depend on the booth option chosen.
	
	
	
	
	
	
	Two booth options are being compared:
	
	
	
	• \textbf{Pop-up booth}
	
	
	
	• \textbf{Full exhibit}
	
	
	
	
	
	\textbf{Given information:}
	
	
	
	States of nature:
	
	
	
	• High foot traffic (0.60)
	
	
	
	• Low foot traffic (0.40)
	
	
	
	
	
	Kits sold (thousands):
	
	
	
	• High: Pop-up 2.4, Full exhibit 3.2
	
	
	
	• Low: Pop-up 1.4, Full exhibit 1.6
	
	
	
	
	
	Price per kit (USD):
	
	
	
	• Pop-up 15, Full exhibit 16
	
	
	
	
	
	Fixed setup cost (thousand USD):
	
	
	
	• Pop-up 18
	
	
	
	• Full exhibit 30
	
	
	
	
	
	
	\textbf{What you must construct:} Revenue for each option and state, then profit (variable revenue + fixed cost).
	
	
	
	
	
	
	\textbf{Question:} Compute profits and decide using Maximin, Expected Value, and Maximax.
	
	\hyperref[problem6-criteria]{↑}
	
	\subsection*{Solution}
	
	\label{problem6-solution}
	
	\subsection*{Structured solution}
	
	
	
	
	
	(C2) Interpreting decision alternatives, events, consequences, and states
	+
	
	
	
	
	
	\hyperref[problem6-criteria]{↑}
	
	
	
	
	(C3) Building the payoff table from revenue and cost data
	+
	
	
	Kits sold (thousands):
	
	Price per kit (USD):
	
	Revenue constructed (kits × price, thousand USD):
	\begin{center}
		\begin{tabular}{lll}
			\toprule
			State & Pop-up & Full exhibit \\
			\midrule
			High\newline (0.60) & 2.4 × 15 = 36 & 3.2 × 16 = 51.2 \\
			Low\newline (0.40) & 1.4 × 15 = 21 & 1.6 × 16 = 25.6 \\
			\bottomrule
		\end{tabular}
	\end{center}
	
	
	Fixed setup costs (thousand USD):
	\begin{center}
		\begin{tabular}{ll}
			\toprule
			Option & Cost \\
			\midrule
			Pop-up & 18 \\
			Full exhibit & 30 \\
			\bottomrule
		\end{tabular}
	\end{center}
	
	
	Payoff table (profit = revenue - fixed cost):
	\begin{center}
		\begin{tabular}{lll}
			\toprule
			State & Pop-up & Full exhibit \\
			\midrule
			S1 (High)\newline (0.60) & 36 - 18 = 18 & 51.2 - 30 = 21.2 \\
			S2 (Low)\newline (0.40) & 21 - 18 = 3 & 25.6 - 30 = -4.4 \\
			\bottomrule
		\end{tabular}
	\end{center}
	
	
	Final payoff table (thousand USD):
	\begin{center}
		\begin{tabular}{lll}
			\toprule
			State & Pop-up & Full exhibit \\
			\midrule
			S1\newline (0.60) & 18 & 21.2 \\
			S2\newline (0.40) & 3 & -4.4 \\
			\bottomrule
		\end{tabular}
	\end{center}
	
	
	
	
	\hyperref[problem6-criteria]{↑}
	
	
	
	
	(C4) Applying the Maximax criterion
	+
	
	
	Maximax compares the best payoff for each option.
	\[
	\begin{aligned}
		\max_i \text{Payoff}(\text{Pop-up}, S_i) &= \max\{18, 3\} = 18 \\
		\max_i \text{Payoff}(\text{Full exhibit}, S_i) &= \max\{21.2, -4.4\} = 21.2
	\end{aligned}
	\]
	\[
	\max\{18, 21.2\} = 21.2
	\]
	The maximax (risk-seeking) choice is the \emph{full exhibit}.
	
	
	\hyperref[problem6-criteria]{↑}
	
	
	
	
	(C5) Applying the Maximin criterion
	+
	
	
	Maximin compares each option's worst payoff.
	\[
	\begin{aligned}
		\min_i \text{Payoff}(\text{Pop-up}, S_i) &= \min\{18, 3\} = 3 \\
		\min_i \text{Payoff}(\text{Full exhibit}, S_i) &= \min\{21.2, -4.4\} = -4.4
	\end{aligned}
	\]
	\[
	\max\{3, -4.4\} = 3
	\]
	The maximin (risk-averse) choice is the \emph{pop-up booth}.
	
	
	\hyperref[problem6-criteria]{↑}
	
	
	
	
	(C1) Formulating the decision problem for an optimal choice
	+
	
	
	The club wants the option with the highest expected profit.
	\[
	\begin{aligned}
		EV_{\text{Pop-up}} &= 0.60(18) + 0.40(3) = 10.8 + 1.2 = 12 \\
		EV_{\text{Full exhibit}} &= 0.60(21.2) + 0.40(-4.4) = 12.72 - 1.76 = 10.96
	\end{aligned}
	\]
	\[
	\max\{12, 10.96\} = 12
	\]
	
	Summary of recommendations across decision criteria
	\begin{itemize}
		\item Maximax favors the \emph{full exhibit} (best payoff of 21.2).
		\item Maximin favors the \emph{pop-up booth} (best worst-case payoff of 3).
		\item Expected value favors the \emph{pop-up booth} (EV of 12).
	\end{itemize}
	
	
	
	The full exhibit offers higher upside, but the pop-up booth protects against losses and delivers the higher average profit.
	
	
	
	\hyperref[problem6-criteria]{↑}
	\section*{Problem 9 — Emergency Supply Contract}
	
	\label{problem7}
	
	\subsection*{Problem description}
	
	\label{problem7-problem}
	
	A city signs a contract for first-aid supply kits. Revenue includes a fixed service retainer paid by the city and a variable
	payment per kit delivered. Costs include a variable cost per kit plus a state-based compliance fee.
	
	
	
	
	
	
	Two suppliers are being compared:
	
	
	
	• \textbf{Local supplier}
	
	
	
	• \textbf{National supplier}
	
	
	
	
	
	\textbf{Given information:}
	
	
	
	States of nature:
	
	
	
	• Normal demand (0.50)
	
	
	
	• Surge demand (0.30)
	
	
	
	• Emergency demand (0.20)
	
	
	
	
	
	Fixed retainer paid by the city (thousand USD):
	
	
	
	• Normal 12, Surge 18, Emergency 28
	
	
	
	
	
	Kits delivered (thousands):
	
	
	
	• Local: Normal 6, Surge 9, Emergency 12
	
	
	
	• National: Normal 7, Surge 10, Emergency 13
	
	
	
	
	
	Revenue per kit (USD):
	
	
	
	• Local 10, National 8.5
	
	
	
	
	
	Cost per kit (USD):
	
	
	
	• Local 6.2, National 5.0
	
	
	
	
	
	Fixed compliance fee by state (thousand USD):
	
	
	
	• Normal 6, Surge 10, Emergency 18
	
	
	
	
	
	
	\textbf{What you must construct:} Total revenue and total cost (variable + fixed) for each alternative and state, then profit.
	
	
	
	
	
	
	\textbf{Question:} Compute profits and decide using Maximin, Expected Value, and Maximax.
	
	\hyperref[problem7-criteria]{↑}
	
	\subsection*{Solution}
	
	\label{problem7-solution}
	
	\subsection*{Structured solution}
	
	
	
	
	
	(C2) Interpreting decision alternatives, events, consequences, and states
	+
	
	
	
	
	
	\hyperref[problem7-criteria]{↑}
	
	
	
	
	(C3) Building the payoff table from revenue and cost data
	+
	
	
	Fixed retainer by state (thousand USD):
	\begin{center}
		\begin{tabular}{ll}
			\toprule
			State & Retainer \\
			\midrule
			Normal\newline (0.50) & 12 \\
			Surge\newline (0.30) & 18 \\
			Emergency\newline (0.20) & 28 \\
			\bottomrule
		\end{tabular}
	\end{center}
	
	
	Kits delivered (thousands):
	
	Revenue per kit (USD):
	\begin{center}
		\begin{tabular}{ll}
			\toprule
			Supplier & Revenue per kit \\
			\midrule
			Local & 10 \\
			National & 8.5 \\
			\bottomrule
		\end{tabular}
	\end{center}
	
	
	Variable revenue (kits × price, thousand USD):
	\begin{center}
		\begin{tabular}{lll}
			\toprule
			State & Local & National \\
			\midrule
			Normal\newline (0.50) & 6 × 10 = 60 & 7 × 8.5 = 59.5 \\
			Surge\newline (0.30) & 9 × 10 = 90 & 10 × 8.5 = 85 \\
			Emergency\newline (0.20) & 12 × 10 = 120 & 13 × 8.5 = 110.5 \\
			\bottomrule
		\end{tabular}
	\end{center}
	
	
	Total revenue (retainer + variable, thousand USD):
	\begin{center}
		\begin{tabular}{lll}
			\toprule
			State & Local & National \\
			\midrule
			Normal\newline (0.50) & 12 + 60 = 72 & 12 + 59.5 = 71.5 \\
			Surge\newline (0.30) & 18 + 90 = 108 & 18 + 85 = 103 \\
			Emergency\newline (0.20) & 28 + 120 = 148 & 28 + 110.5 = 138.5 \\
			\bottomrule
		\end{tabular}
	\end{center}
	
	
	Cost per kit (USD):
	\begin{center}
		\begin{tabular}{ll}
			\toprule
			Supplier & Cost per kit \\
			\midrule
			Local & 6.2 \\
			National & 5.0 \\
			\bottomrule
		\end{tabular}
	\end{center}
	
	
	Variable costs (kits × cost, thousand USD):
	\begin{center}
		\begin{tabular}{lll}
			\toprule
			State & Local & National \\
			\midrule
			Normal\newline (0.50) & 6 × 6.2 = 37.2 & 7 × 5.0 = 35 \\
			Surge\newline (0.30) & 9 × 6.2 = 55.8 & 10 × 5.0 = 50 \\
			Emergency\newline (0.20) & 12 × 6.2 = 74.4 & 13 × 5.0 = 65 \\
			\bottomrule
		\end{tabular}
	\end{center}
	
	
	Fixed compliance fees by state (thousand USD):
	\begin{center}
		\begin{tabular}{ll}
			\toprule
			State & Fee \\
			\midrule
			Normal\newline (0.50) & 6 \\
			Surge\newline (0.30) & 10 \\
			Emergency\newline (0.20) & 18 \\
			\bottomrule
		\end{tabular}
	\end{center}
	
	
	Total costs (variable + fixed, thousand USD):
	\begin{center}
		\begin{tabular}{lll}
			\toprule
			State & Local & National \\
			\midrule
			Normal\newline (0.50) & 37.2 + 6 = 43.2 & 35 + 6 = 41 \\
			Surge\newline (0.30) & 55.8 + 10 = 65.8 & 50 + 10 = 60 \\
			Emergency\newline (0.20) & 74.4 + 18 = 92.4 & 65 + 18 = 83 \\
			\bottomrule
		\end{tabular}
	\end{center}
	
	
	Payoff table (profit = total revenue - total costs):
	\begin{center}
		\begin{tabular}{lll}
			\toprule
			State & Local & National \\
			\midrule
			S1 (Normal)\newline (0.50) & 72 - 43.2 = 28.8 & 71.5 - 41 = 30.5 \\
			S2 (Surge)\newline (0.30) & 108 - 65.8 = 42.2 & 103 - 60 = 43 \\
			S3 (Emergency)\newline (0.20) & 148 - 92.4 = 55.6 & 138.5 - 83 = 55.5 \\
			\bottomrule
		\end{tabular}
	\end{center}
	
	
	Final payoff table (thousand USD):
	\begin{center}
		\begin{tabular}{lll}
			\toprule
			State & Local & National \\
			\midrule
			S1\newline (0.50) & 28.8 & 30.5 \\
			S2\newline (0.30) & 42.2 & 43 \\
			S3\newline (0.20) & 55.6 & 55.5 \\
			\bottomrule
		\end{tabular}
	\end{center}
	
	
	
	
	\hyperref[problem7-criteria]{↑}
	
	
	
	
	(C4) Applying the Maximax criterion
	+
	
	
	Maximax compares the best payoff for each supplier.
	\[
	\begin{aligned}
		\max_i \text{Payoff}(\text{Local}, S_i) &= \max\{28.8, 42.2, 55.6\} = 55.6 \\
		\max_i \text{Payoff}(\text{National}, S_i) &= \max\{30.5, 43, 55.5\} = 55.5
	\end{aligned}
	\]
	\[
	\max\{55.6, 55.5\} = 55.6
	\]
	The maximax (risk-seeking) choice is the \emph{local supplier}.
	
	
	\hyperref[problem7-criteria]{↑}
	
	
	
	
	(C5) Applying the Maximin criterion
	+
	
	
	Maximin selects the highest worst-case payoff.
	\[
	\begin{aligned}
		\min_i \text{Payoff}(\text{Local}, S_i) &= \min\{28.8, 42.2, 55.6\} = 28.8 \\
		\min_i \text{Payoff}(\text{National}, S_i) &= \min\{30.5, 43, 55.5\} = 30.5
	\end{aligned}
	\]
	\[
	\max\{28.8, 30.5\} = 30.5
	\]
	The maximin (risk-averse) choice is the \emph{national supplier}.
	
	
	\hyperref[problem7-criteria]{↑}
	
	
	
	
	(C1) Formulating the decision problem for an optimal choice
	+
	
	
	The city compares expected profits using the state probabilities.
	\[
	\begin{aligned}
		EV_{\text{Local}} &= 0.50(28.8) + 0.30(42.2) + 0.20(55.6) \\
		&= 14.4 + 12.66 + 11.12 = 38.18 \\
		EV_{\text{National}} &= 0.50(30.5) + 0.30(43) + 0.20(55.5) \\
		&= 15.25 + 12.9 + 11.1 = 39.25
	\end{aligned}
	\]
	\[
	\max\{38.18, 39.25\} = 39.25
	\]
	
	Summary of recommendations across decision criteria
	\begin{itemize}
		\item Maximax favors the \emph{local supplier} (best payoff of 55.6).
		\item Maximin favors the \emph{national supplier} (best worst-case payoff of 30.5).
		\item Expected value favors the \emph{national supplier} (EV of 39.25).
	\end{itemize}
	
	
	
	The local supplier has a slightly higher upside in emergencies, but the national supplier provides better protection and the highest average profit.
	
	
	
	\hyperref[problem7-criteria]{↑}
\end{document}