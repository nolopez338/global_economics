\documentclass[12pt]{article}

% Page size and tighter margins
\usepackage[a4paper,left=1.2cm,right=1.2cm,top=1.5cm,bottom=1.5cm]{geometry}

% Core packages
\usepackage{graphicx}
\usepackage{xcolor}
\usepackage{array}
\usepackage{tabularx}
\usepackage{multicol}
\usepackage[T1]{fontenc}
\usepackage[utf8]{inputenc}
\usepackage{booktabs}

\setlength{\parindent}{0pt}
\setlength{\tabcolsep}{6pt}
\renewcommand{\arraystretch}{1.15}

% Column types
\newcolumntype{Y}{>{\raggedright\arraybackslash}m{\dimexpr0.30\textwidth-2\tabcolsep-2\arrayrulewidth\relax}}
\newcolumntype{Z}{>{\raggedright\arraybackslash}m{\dimexpr0.70\textwidth-2\tabcolsep-2\arrayrulewidth\relax}}
\newcolumntype{C}[1]{>{\centering\arraybackslash}m{#1}}

% Gray subsection header box
\newcommand{\SubsectionBox}[1]{%
	\noindent\colorbox{gray!30}{%
		\parbox{\linewidth}{\textbf{#1}}%
	}\par\vspace{0.35cm}%
}

% Centered multi-line cell helper
\newcommand{\CellCenter}[1]{%
	\parbox{\linewidth}{\centering #1}%
}

\begin{document}

	\noindent
	\begin{tabularx}{\textwidth}{|C{2.8cm}|C{\dimexpr\textwidth-6cm-4\tabcolsep-4\arrayrulewidth\relax}|C{2.8cm}|}
		\hline
		\centering
		\vspace{3mm}
		\includegraphics[width=2.5cm]{../../preamble/logo.png}
		&
		\CellCenter{%
			\vspace{-5mm}
			\textbf{GLOBAL ECONOMICS}\par
			\textbf{GRADE: 10TH}\par
			\textbf{CATCH-UP ACTIVITY}\par
			\textbf{ANALYSIS OF DECISIONS}\par
			\textbf{TEACHER'S NAME: Nicolás López Cuéllar}
		}
		&
		\CellCenter{%
			\textbf{SECOND TERM}\par
			\textbf{2025--2026}%
		}
		\\
		\hline
	\end{tabularx}

	\vspace{0.5cm}

	\noindent
	\begin{tabular}{|Y|Z|}
		\hline
		{\small
			\textbf{Learning objective:} Analyze decision-making problems using probabilities and expected value to compare alternatives under uncertainty.
		}
		&
		{\footnotesize
			\textbf{Assessment criteria:}\par
			C7: Uses probabilities and expected value to analyze a decision-making problem.\par
		}
		\\
		\hline
	\end{tabular}

	\vspace{0.4cm}

	\begin{multicols}{2}
		\SubsectionBox{Criteria assessment}\vspace{-0.25cm}
		This activity evaluates criterion C7. Read each problem and complete the required decision-analysis tasks.

		\vspace{0.25cm}
		\SubsectionBox{1. Problem 1 --- Single Project under Uncertainty}\vspace{-0.25cm}
		A renewable energy firm is deciding whether to build a small solar farm. The decision is whether to build (one alternative) or not build, and management uses expected value because similar projects are repeated over time. The uncertain states are:
				\begin{itemize}
					\item \(S_1\): high electricity prices (probability \(p\)),
					\item \(S_2\): medium electricity prices (probability \(0.25\)),
					\item \(S_3\): low electricity prices (probability \(0.75-p\)).
				\end{itemize}
				Payoffs are net profits in millions of dollars from building the farm.
				
				\[
				\begin{array}{lccc}
					\hline
					& S_1\,(p) & S_2\,(0.25) & S_3\,(0.75-p) \\
					\hline
					\mbox{Build the farm} & 30 & 6 & -12 \\
					\hline
				\end{array}
				\]
				
				Use expected value and determine for which values of \(p\) building the farm is profitable.

		\vspace{0.25cm}
		\SubsectionBox{2. Problem 2 --- Technology Upgrade}\vspace{-0.25cm}
		A delivery company is considering upgrading its routing software. The choice is to upgrade or not, and expected value is appropriate because the system is used repeatedly. The uncertain states are:
				\begin{itemize}
					\item \(S_1\): fuel prices stay very high (probability \(p\)),
					\item \(S_2\): fuel prices are moderate (probability \(0.30\)),
					\item \(S_3\): fuel prices become low (probability \(0.70-p\)).
				\end{itemize}
				Payoffs are net savings in millions of dollars from upgrading.
				
				\[
				\begin{array}{lccc}
					\hline
					& S_1\,(p) & S_2\,(0.30) & S_3\,(0.70-p) \\
					\hline
					\mbox{Upgrade} & 36 & 8 & -18 \\
					\hline
				\end{array}
				\]
				
				Using expected value, determine for which values of \(p\) the upgrade is profitable.

		\vspace{0.25cm}
		\SubsectionBox{3. Problem 3 --- Two Investment Alternatives}\vspace{-0.25cm}
		An entrepreneur must choose between two investment alternatives, A and B. Expected value is used to select the higher long-run average return. The uncertain states are:
				\begin{itemize}
					\item \(S_1\): strong market growth (probability \(p\)),
					\item \(S_2\): stable market (probability \(0.25\)),
					\item \(S_3\): weak market (probability \(0.75-p\)).
				\end{itemize}
				Payoffs are net profits in millions of dollars.
				
				\[
				\begin{array}{lccc}
					\hline
					& S_1\,(p) & S_2\,(0.25) & S_3\,(0.75-p) \\
					\hline
					A & 34 & 12 & 0 \\
					B & 26 & 14 & 6 \\
					\hline
				\end{array}
				\]
				
				Using expected value, determine for which values of \(p\) option A yields a higher expected value than option B.

		\vspace{0.25cm}
		\SubsectionBox{4. Problem 4 --- Three Production Plans}\vspace{-0.25cm}
		A manufacturing firm must choose among three production plans (A, B, C). The firm uses expected value because it seeks the plan with the highest average profit over many similar quarters. The uncertain states are:
				\begin{itemize}
					\item \(S_1\): strong demand (probability \(p\)),
					\item \(S_2\): moderate demand (probability \(0.30\)),
					\item \(S_3\): weak demand (probability \(0.70-p\)).
				\end{itemize}
				Payoffs are net profits in millions of dollars.
				
				\[
				\begin{array}{lccc}
					\hline
					& S_1\,(p) & S_2\,(0.30) & S_3\,(0.70-p) \\
					\hline
					A & 44 & 14 & -10 \\
					B & 32 & 18 & 2 \\
					C & 22 & 16 & 8 \\
					\hline
				\end{array}
				\]
				
				Using expected value, determine for which values of \(p\) each plan is optimal.

		\vspace{0.25cm}
		\SubsectionBox{5. Problem 5 --- Two Alternatives with Three States}\vspace{-0.25cm}
		A shipping company must choose between two routing strategies, A and B. Expected value is used because the company wants the higher average profit over many shipments. The uncertain states are:
				\begin{itemize}
					\item \(S_1\): low congestion (probability \(p_1\)),
					\item \(S_2\): medium congestion (probability \(p_2\)),
					\item \(S_3\): high congestion (probability \(p_3=1-p_1-p_2\)).
				\end{itemize}
				Payoffs are net profits in millions of dollars.
				
				\[
				\begin{array}{lccc}
					\hline
					& S_1\,(p_1) & S_2\,(p_2) & S_3\,(1-p_1-p_2) \\
					\hline
					A & 28 & 10 & -12 \\
					B & 20 & 16 & 0 \\
					\hline
				\end{array}
				\]
				
				Using expected value, determine when strategy A yields a higher expected value than strategy B.

		\vspace{0.25cm}
		\SubsectionBox{6. Problem 6 --- Policy Choice under Three States}\vspace{-0.25cm}
		A local government compares two flood-prevention policies, A and B. Expected value is used because the city wants the highest average net benefit over many years. The uncertain states are:
				\begin{itemize}
					\item \(S_1\): heavy rainfall season (probability \(p_1\)),
					\item \(S_2\): moderate rainfall season (probability \(p_2\)),
					\item \(S_3\): dry season (probability \(1-p_1-p_2\)).
				\end{itemize}
				Payoffs are net benefits in millions of dollars.
				
				\[
				\begin{array}{lccc}
					\hline
					& S_1\,(p_1) & S_2\,(p_2) & S_3\,(1-p_1-p_2) \\
					\hline
					A & 24 & 8 & -10 \\
					B & 16 & 14 & 4 \\
					\hline
				\end{array}
				\]
				
				Using expected value, determine when policy A yields a higher expected value than policy B.

		\vspace{0.25cm}
		\SubsectionBox{7. Problem 7 --- New Service Platform Launch}\vspace{-0.25cm}
		A software company must choose among three launch plans (A, B, C) for a new service platform. Expected value is used because the company repeats similar launches and wants the highest average profit. The uncertain states are:
				\begin{itemize}
					\item \(S_1\): strong adoption (probability \(0.50\)),
					\item \(S_2\): moderate adoption (probability \(0.30\)),
					\item \(S_3\): weak adoption (probability \(0.20\)).
				\end{itemize}
				Payoffs are net profits in millions of dollars.
				
				\[
				\begin{array}{lccc}
					\hline
					& S_1\,(0.50) & S_2\,(0.30) & S_3\,(0.20) \\
					\hline
					A & 34 & 16 & -2 \\
					B & 30 & 14 & 4 \\
					C & 24 & 18 & 8 \\
					\hline
				\end{array}
				\]
				
				Using expected value, compute the expected value for each plan and recommend the plan with the highest expected value.

		\vspace{0.25cm}
		\SubsectionBox{8. Problem 8 --- Regional Store Opening}\vspace{-0.25cm}
		A retailer evaluates opening a regional store. Expected value is used because the retailer wants the highest average profit over many comparable openings. Demand and competitor response are treated as independent. The states are:
				\begin{itemize}
					\item strong demand (probability \(p\)),
					\item weak demand (probability \(1-p\)),
					\item mild competitor response (probability \(q\)),
					\item aggressive competitor response (probability \(1-q\)).
				\end{itemize}
				Payoffs are net profits in millions of dollars.
				
				\[
				\begin{array}{lcc}
					\hline
					& \mbox{Mild}\,(q) & \mbox{Aggressive}\,(1-q) \\
					\hline
					\mbox{Strong demand}\,(p) & 44 & 18 \\
					\mbox{Weak demand}\,(1-p) & -4 & -16 \\
					\hline
				\end{array}
				\]
				
				Using expected value, determine when opening the store is profitable.

		\vspace{0.25cm}
		\SubsectionBox{9. Problem 9 --- Marketing Strategy Choice}\vspace{-0.25cm}
		A firm must choose between two marketing strategies, A and B. Expected value is used because the firm wants the highest average profit over many campaigns. Market interest and campaign execution are independent. The states are:
				\begin{itemize}
					\item high interest (probability \(p\)),
					\item low interest (probability \(1-p\)),
					\item effective execution (probability \(q\)),
					\item ineffective execution (probability \(1-q\)).
				\end{itemize}
				Payoffs are net profits in millions of dollars.
				
				\[
				\begin{array}{lcc}
					\hline
					\mbox{State of nature} & A & B \\
					\hline
					\mbox{High interest}\,(p),\,\mbox{Effective}\,(q) & 36 & 30 \\
					\mbox{High interest}\,(p),\,\mbox{Ineffective}\,(1-q) & 14 & 18 \\
					\mbox{Low interest}\,(1-p),\,\mbox{Effective}\,(q) & -10 & 0 \\
					\mbox{Low interest}\,(1-p),\,\mbox{Ineffective}\,(1-q) & -18 & -8 \\
					\hline
				\end{array}
				\]
				
				Using expected value, determine when strategy A yields a higher expected value than strategy B.

		\vspace{0.25cm}
		\SubsectionBox{10. Problem 10 --- Investment Portfolio Selection}\vspace{-0.25cm}
		An investor must choose between two portfolios, X and Y. Expected value is used because the investor wants the higher long-run average return. Economic growth and interest rates are independent. The states are:
				\begin{itemize}
					\item strong growth (probability \(p\)),
					\item weak growth (probability \(1-p\)),
					\item low interest rates (probability \(q\)),
					\item high interest rates (probability \(1-q\)).
				\end{itemize}
				Payoffs are net returns in millions of dollars.
				
				\[
				\begin{array}{lcc}
					\hline
					\mbox{State of nature} & X & Y \\
					\hline
					\mbox{Strong growth}\,(p),\,\mbox{Low rates}\,(q) & 34 & 30 \\
					\mbox{Strong growth}\,(p),\,\mbox{High rates}\,(1-q) & 16 & 22 \\
					\mbox{Weak growth}\,(1-p),\,\mbox{Low rates}\,(q) & 6 & 10 \\
					\mbox{Weak growth}\,(1-p),\,\mbox{High rates}\,(1-q) & -14 & -6 \\
					\hline
				\end{array}
				\]
				
				Using expected value, determine when portfolio X yields a higher expected value than portfolio Y.

	\end{multicols}
\end{document}
