\makeatletter
\def\input@path{{./}{../}{../../}{preamble/}{../preamble/}{../../preamble/}}
\makeatother
% ----------------------------------------------------------
% GENERAL 

% File
\documentclass[11pt]{book}

% Margins
\usepackage[margin=1in]{geometry}

\linespread{1.2}            % Line spacing
\usepackage[utf8]{inputenc}
\usepackage[T1]{fontenc}
\usepackage{lmodern}
\usepackage{microtype}
\setlength{\parindent}{0pt}
\setlength{\parskip}{6pt}
\usepackage{booktabs}

% ----------------------------------------------------------
% TABLES
\usepackage{multicol}
\usepackage{longtable} 
\usepackage{array}
\usepackage{booktabs}
\usepackage{tabularx}
\usepackage{multirow}

% ----------------------------------------------------------
% MATHEMATICS
\usepackage{amsmath}
\usepackage{amssymb}
\usepackage{amsfonts}
\usepackage{mathtools}

% ----------------------------------------------------------
% HIDDEN CONTENT
\usepackage{ifthen}
% Define a boolean switch
\newboolean{explicaciones}
% Set the boolean switch to true or false
% Change to true to show the content

% Explanations
\newcommand{\explicacion}[2]{
	\ifthenelse{\boolean{explicaciones}}{#1}{#2}
}
\newcommand{\mostrarExplicaciones}[1]{\setboolean{explicaciones}{#1}}

% ----------------------------------------------------------
% NUMBERING

\usepackage{fancyhdr}
\pagestyle{empty} % Ensures the entire document has no page numbers

\usepackage{tocloft}
\renewcommand{\cftdot}{} % Remove dots for sections, if any
\renewcommand{\cftsecleader}{\cftdotfill{\cftdotsep}} % Remove dots for sections, if any
\cftpagenumbersoff{section} % Removes page numbers from sections
\cftpagenumbersoff{subsection} % Removes page numbers from subsections

% ----------------------------------------------------------
% IMAGES 

% General settings
\usepackage{graphicx}       % Insert images
\usepackage{float}          % Position images
% \usepackage{subfigure}      % Subfigures
\graphicspath{{imgs}}       % Image location
\usepackage{subcaption}     % Subfigures II
\usepackage{verbatim}

% Figures
\usepackage{tikz}
\usetikzlibrary{arrows.meta,positioning,trees}

% Colors
\usepackage{xcolor}     
\definecolor{popUp}{HTML}{666666}
\definecolor{popUpIn}{HTML}{CED9E0}
\definecolor{backgroundC}{HTML}{D0E8F2}
\definecolor{backgroundCC}{HTML}{FFFFFF}
\definecolor{borders}{HTML}{8c120d}
\definecolor{padding}{HTML}{77D0D7}
\definecolor{links}{HTML}{CC6F5F}

% ----------------------------------------------------------
% FRAMES

% General settings
\usepackage{tcolorbox}
\usepackage{adjustbox}          % Adjusted frame  
\setlength{\fboxrule}{3pt}  % Line width
\setlength{\fboxsep}{3pt}   % Box padding

% General frames
\usepackage{mdframed}   

\mdfdefinestyle{estiloGeneral}{    % General style
	linecolor=black,
	linewidth=1.5pt,
	roundcorner=10pt,
	backgroundcolor=backgroundC,
	innerbottommargin=0pt
}
\mdfdefinestyle{code}{          % Code style
	linecolor=black,
	linewidth=1.5pt,
	roundcorner=10pt,
	backgroundcolor=darkgray!10,
	innerbottommargin=0pt
}

% Image frame
\newtcbox{\fboxC}{
	colback=backgroundC,
	colframe=popUp,
	arc=10pt,
	boxrule=3pt,
	boxsep=0pt, % Change the padding here
	nobeforeafter
}

% ----------------------------------------------------------
% PAGE SETTINGS

% Background 
\newcommand{\background}[0]{\begin{tikzpicture}[remember picture,overlay]
		\fill[backgroundC] (-2,2) rectangle (25cm, -550);
\end{tikzpicture}}

\newcommand{\backgroundC}[0]{\begin{tikzpicture}[remember picture,overlay]
		\fill[backgroundCC] (-2,2) rectangle (25cm, -550);
\end{tikzpicture}}

% Page width 
\newcommand{\anchoPag}[0]{20cm}

% ----------------------------------------------------------
% FONT

% General
\usepackage{tgbonum}        % Font
\usepackage{listings}       % Code typesetting
\usepackage[spanish]{babel} % Load Spanish
\selectlanguage{spanish}    % Select Spanish
\usepackage{enumitem}
\usepackage{bookmark}

\setlist[itemize]{leftmargin=1.2em, itemsep=0.35em, topsep=0.35em}

% --- Table helpers ---
\newcolumntype{L}[1]{>{\raggedright\arraybackslash}p{#1}}
\newcolumntype{Y}{>{\raggedright\arraybackslash}X}
\newcolumntype{C}{>{\centering\arraybackslash}X}
\renewcommand{\arraystretch}{1.1}

% Python style
\lstdefinestyle{python}{
	language=Python,
	basicstyle=\ttfamily\small,
	commentstyle=\color{green!50!black},
	keywordstyle=\color{blue},
	numberstyle=\tiny\color{gray},
	numbers=left,
	morekeywords={>, <},
	breakatwhitespace=false,
	showstringspaces=false,
	showtabs=false,
	showspaces=false
}

% ----------------------------------------------------------
% HYPERLINKS

% General
\usepackage{hyperref}       
\hypersetup{
	colorlinks=true,
	linkcolor=links,
	filecolor=magenta,      
	urlcolor=brown,
}

% Custom commands 

% Large link
\newcommand{\bigLink}[2]{\begin{center} \fboxC{\LARGE{\href{#1}{#2}}}\end{center}}

% Small link
\newcommand{\smallLink}[2]{\begin{center}\fboxC{\href{#1}{#2}}\end{center}}

% Bold link
\newcommand{\bfLink}[2]{\href{#1}{\textbf{#2}}}


% Small URL
\newcommand{\smallUrl}[1]{\begin{center}\fboxC{\url{#1}}\end{center}}


% ----------------------------------------------------------
% CUSTOM COMMANDS FOR FIGURES

\newcommand{\espacioImagenes}[0]{-1.2cm}

% Without frame
\newcommand{\fig}[3][\espacioImagenes]{
	\hspace*{#1}
	\centering
	\includegraphics[width=#2\textwidth]{#3}
}

% With frame
\newcommand{\ffig}[2]{\begin{figure}[h]
		\hspace*{\espacioImagenes}
		\centering
		\fbox{\includegraphics[width=#1\textwidth]{#2}}
\end{figure}}

% Hyperlink with frame
\newcommand{\hfig}[3]{\begin{figure}[h]
		\hspace*{-1.4cm}
		\centering
		\color{popUp}
		\fboxC{\href{#1}{\includegraphics[width=#2\textwidth]{#3}}}
	\end{figure}
}

% Hyperlink without frame
\newcommand{\hffig}[3]{\begin{figure}[h]
		\hspace*{-1.1cm}
		\centering
		\color{popUp}
		\href{#1}{\includegraphics[width=#2\textwidth]{#3}}
	\end{figure}
}

% ----------------------------------------------------------

% Start and Contents
\newcommand{\cuadro}[1]{
	\begin{mdframed}[style=estiloGeneral]
		#1 
	\end{mdframed}
}

% Explanation video image
\newcommand{\linkExplicacion}[1]{
	\hffig{#1}{0.5}{principal/videoExplicacion}
	\vspace{-0.5cm}
}

\newcommand{\subSecLink}[2]{
	\subsubsection*{\href{#1}{\textbf{#2}}}
}

% Spacing
\newcommand{\esp}[0]{\vspace{4mm}}

% Back to start
\newcommand{\secInicio}[0]{\begin{center}\hyperref[sec:inicio]{ 
			\includegraphics[width=0.1\textwidth]{principal/up}
	}\end{center}
}


\geometry{margin=0.85in}
\AtBeginDocument{\small}

\newcommand{\ExamNameField}{\noindent\textbf{Name:}\ \rule{0.7\linewidth}{0.4pt}\par\medskip}

\newcommand{\ExamTitleBlock}[3]{%
	\begin{center}
		\Large\textbf{#1}\\
		\textbf{#2}%
		\if\relax\detokenize{#3}\relax\else\\\textbf{#3}\fi
	\end{center}
	\vspace{0.5em}
}

\newcommand{\ExamSection}[1]{\par\medskip\textbf{#1}\par\smallskip}

\newenvironment{ExamCriteria}{%
	\begin{itemize}[leftmargin=1.6em, itemsep=0.3em, topsep=0.2em]
}{%
	\end{itemize}
}

\newenvironment{ExamProblems}{%
	\begin{enumerate}[label=\textbf{P\arabic*}, leftmargin=0pt, labelsep=0.6em, itemindent=2.2em, itemsep=0.8em]
}{%
	\end{enumerate}
}

\begin{document}
		\ExamTitleBlock{10th grade}{Midterm Analysis of Decisions (Solutions)}{}
	
	\begin{ExamProblems}
		\item
		\subsection*{Café Menu Strategy Under Demand Uncertainty}
		A small café has budget for only one menu strategy for the coming season, so the owner must choose either launching the artisan dessert line or keeping the current menu. Foot-traffic uncertainty is summarized by two states: high traffic and low traffic, based on last year’s daily counts and a neighborhood survey that estimate a 0.55 probability of high traffic and a 0.45 probability of low traffic. The finance team estimates net profit (in thousand USD) of 42 when the dessert line meets high traffic but only 8 when traffic is low because of added fixed baking costs, while keeping the current menu yields 28 in high traffic and 20 in low traffic due to steadier costs but fewer upsell opportunities.
		
		\textbf{Question:} Make a decision on the best option based on three perspectives: safe behavior (Maximin), balanced behavior (Expected Value), and risky behavior (Maximax).
		
		\subsection*{C2}
		\begin{center}
			\begin{tabular}{l p{0.74\linewidth}}
				\toprule
				Alternatives & Launch dessert line; Keep current menu \\
				States of nature & High traffic (0.55); Low traffic (0.45) \\
				Events & Actual foot traffic after the decision \\
				Consequences & Profit in thousand USD for each alternative--state pair \\
				\bottomrule
			\end{tabular}
		\end{center}
		
		\subsection*{C3}
		\begin{center}
			\textit{Payoff table} \\
			\begin{tabular}{l c c}
				\toprule
				Alternative & High traffic (0.55) & Low traffic (0.45) \\
				\midrule
				Launch dessert line & 42 & 8 \\
				Keep menu & 28 & 20 \\
				\bottomrule
			\end{tabular}
		\end{center}
		
		\subsection*{C4}
		The maximax rule selects the alternative with the highest possible payoff.
		\[
		\begin{aligned}
			\max_i \text{Payoff}(\text{Launch}, S_i) &= \max\{42, 8\} = 42, \\
			\max_i \text{Payoff}(\text{Keep}, S_i) &= \max\{28, 20\} = 28.
		\end{aligned}
		\]
		\[
		\max\{42, 28\} = 42.
		\]
		Because 42 is the largest best payoff, the maximax (risk-seeking) recommendation is \emph{launch the dessert line}.
		
		\subsection*{C5}
		The maximin rule focuses on the worst-case payoff of each alternative.
		\[
		\begin{aligned}
			\min_i \text{Payoff}(\text{Launch}, S_i) &= \min\{42, 8\} = 8, \\
			\min_i \text{Payoff}(\text{Keep}, S_i) &= \min\{28, 20\} = 20.
		\end{aligned}
		\]
		\[
		\max\{8, 20\} = 20.
		\]
		Since 20 exceeds 8, the maximin (risk-averse) choice is \emph{keep the current menu}.
		
		\subsection*{C1}
		We choose the option that maximizes the café's profit objective. Alternatives are \textbf{launch} or \textbf{keep} the menu; states are \textbf{high} or \textbf{low traffic}. Expected values:
		\[
		\begin{aligned}
			EV_{\text{Launch}} &= 0.55(42) + 0.45(8) = 23.1 + 3.6 = 26.7, \\
			EV_{\text{Keep}} &= 0.55(28) + 0.45(20) = 15.4 + 9 = 24.4.
		\end{aligned}
		\]
		\[
		\max\{26.7, 24.4\} = 26.7.
		\]
		
		Summary of recommendations across decision criteria
		\begin{itemize}
			\item Maximax favors \emph{launching the dessert line} (highest payoff of 42).
			\item Maximin favors \emph{keeping the current menu} (best worst-case payoff of 20).
			\item Expected value favors \emph{launching the dessert line} (EV of 26.7).
		\end{itemize}
		
		Launching desserts leads on optimistic and average outcomes, while the current menu protects the downside.
	
		\item
		\subsection*{School Cafeteria Menu Choice Under Demand and Cost Uncertainty}
		A school cafeteria must pick either an expanded menu or a classic menu for the lunch period because staffing and kitchen capacity only allow one menu format. Demand uncertainty is represented by high, moderate, and low demand states derived from past sales, with probabilities of 0.60 for high demand, 0.40 for moderate demand, and 0.60 for low demand, while ingredient-cost conditions are modeled as low or high costs with probabilities of 0.55 and 0.45 from supplier quotes. Revenue (in thousand USD) for the expanded menu is 36 in high demand, 28 in moderate demand, and 20 in low demand, compared with 30, 26, and 22 for the classic menu, and ingredient costs are 12 (low) or 15 (high) for the expanded menu versus 10 (low) or 13 (high) for the classic menu, so the expanded menu earns more when demand is strong but carries higher cost exposure.
		
		\textbf{Question:} Compute profit for each state and decide using Maximin, Expected Value, and Maximax.
		
		\subsection*{C2}
		\begin{center}
			\begin{tabular}{l p{0.74\linewidth}}
				\toprule
				Alternatives & Expanded menu; Classic menu \\
				States of nature & Revenue: High demand (0.60), Moderate demand (0.40), Low demand (0.60); Cost: Low ingredient cost (0.55), High ingredient cost (0.45) \\
				Events & Demand level paired with ingredient cost conditions after the menu choice \\
				Consequences & Profit (revenue $-$ costs) for each menu and state \\
				Probabilities & Revenue: 0.60, 0.40; Cost: 0.55, 0.45 \\
				\bottomrule
			\end{tabular}
		\end{center}
		
		\subsection*{C3}
		Revenue by demand state (thousand USD):
		\begin{center}
			\begin{tabular}{l c c c}
				\toprule
				State & Probability & Expanded & Classic \\
				\midrule
				High demand & 0.60 & 36 & 30 \\
				Moderate demand & 0.40 & 28 & 26 \\
				Low demand & 0.60 & 20 & 22 \\
				\bottomrule
			\end{tabular}
		\end{center}
		
		Cost by ingredient price state (thousand USD):
		\begin{center}
			\begin{tabular}{l c c c}
				\toprule
				State & Probability & Expanded & Classic \\
				\midrule
				Low cost & 0.55 & 12 & 10 \\
				High cost & 0.45 & 15 & 13 \\
				\bottomrule
			\end{tabular}
		\end{center}
		
		Profit table (profit = revenue $-$ cost):
		\begin{center}
			\begin{tabular}{l p{0.17\linewidth} p{0.22\linewidth} p{0.22\linewidth}}
				\toprule
				State of nature & Probability & Expanded & Classic \\
				\midrule
				High demand, low cost &
				$\begin{array}{l}
					0.60 \times 0.55\\
					= 0.33
				\end{array}$ &
				$\begin{array}{l}
					36 - 12 = 24
				\end{array}$ &
				$\begin{array}{l}
					30 - 10 = 20
				\end{array}$ \\
				High demand, high cost &
				$\begin{array}{l}
					0.60 \times 0.45\\
					= 0.27
				\end{array}$ &
				$\begin{array}{l}
					36 - 15 = 21
				\end{array}$ &
				$\begin{array}{l}
					30 - 13 = 17
				\end{array}$ \\
				Moderate demand, low cost &
				$\begin{array}{l}
					0.40 \times 0.55\\
					= 0.22
				\end{array}$ &
				$\begin{array}{l}
					28 - 12 = 16
				\end{array}$ &
				$\begin{array}{l}
					26 - 10 = 16
				\end{array}$ \\
				Moderate demand, high cost &
				$\begin{array}{l}
					0.40 \times 0.45\\
					= 0.18
				\end{array}$ &
				$\begin{array}{l}
					28 - 15 = 13
				\end{array}$ &
				$\begin{array}{l}
					26 - 13 = 13
				\end{array}$ \\
				Low demand, low cost &
				$\begin{array}{l}
					0.60 \times 0.55\\
					= 0.33
				\end{array}$ &
				$\begin{array}{l}
					20 - 12 = 8
				\end{array}$ &
				$\begin{array}{l}
					22 - 10 = 12
				\end{array}$ \\
				Low demand, high cost &
				$\begin{array}{l}
					0.60 \times 0.45\\
					= 0.27
				\end{array}$ &
				$\begin{array}{l}
					20 - 15 = 5
				\end{array}$ &
				$\begin{array}{l}
					22 - 13 = 9
				\end{array}$ \\
				\bottomrule
			\end{tabular}
		\end{center}
		
		Final payoff table (thousand USD):
		\begin{center}
			\begin{tabular}{l c c c}
				\toprule
				State of nature & Probability & Expanded & Classic \\
				\midrule
				High demand, low cost & 0.33 & 24 & 20 \\
				High demand, high cost & 0.27 & 21 & 17 \\
				Moderate demand, low cost & 0.22 & 16 & 16 \\
				Moderate demand, high cost & 0.18 & 13 & 13 \\
				Low demand, low cost & 0.33 & 8 & 12 \\
				Low demand, high cost & 0.27 & 5 & 9 \\
				\bottomrule
			\end{tabular}
		\end{center}
		
		\subsection*{C4}
		Maximax selects the alternative with the highest best payoff.
		\[
		\begin{aligned}
			\max_i \text{Payoff}(\text{Expanded}, S_i) &= \max\{24, 21, 16, 13, 8, 5\} = 24, \\
			\max_i \text{Payoff}(\text{Classic}, S_i) &= \max\{20, 17, 16, 13, 12, 9\} = 20.
		\end{aligned}
		\]
		\[
		\max\{24, 20\} = 24.
		\]
		The maximax (risk-seeking) decision is \emph{expand the menu}.
		
		\subsection*{C5}
		Maximin compares each alternative's worst payoff.
		\[
		\begin{aligned}
			\min_i \text{Payoff}(\text{Expanded}, S_i) &= \min\{24, 21, 16, 13, 8, 5\} = 5, \\
			\min_i \text{Payoff}(\text{Classic}, S_i) &= \min\{20, 17, 16, 13, 12, 9\} = 9.
		\end{aligned}
		\]
		\[
		\max\{5, 9\} = 9.
		\]
		The maximin (risk-averse) decision is \emph{keep the classic menu}.
		
		\subsection*{C1}
		The truck wants the menu that maximizes expected profit.
		\[
		\begin{aligned}
			EV_{\text{Expanded}} &= 0.33(24) + 0.27(21) + 0.22(16) + 0.18(13) + 0.33(8) + 0.27(5) \\
			&= 7.92 + 5.67 + 3.52 + 2.34 + 2.64 + 1.35 = 23.44, \\
			EV_{\text{Classic}} &= 0.33(20) + 0.27(17) + 0.22(16) + 0.18(13) + 0.33(12) + 0.27(9) \\
			&= 6.60 + 4.59 + 3.52 + 2.34 + 3.96 + 2.43 = 23.44.
		\end{aligned}
		\]
		\[
		\max\{23.44, 23.44\} = 23.44.
		\]
		
		Summary of recommendations across decision criteria
		\begin{itemize}
			\item Maximax favors \emph{expanded menu} (best payoff of 24).
			\item Maximin favors \emph{classic menu} (best worst-case payoff of 9).
			\item Expected value is a tie (both EVs are 23.44).
		\end{itemize}
		
		The expanded menu offers the largest upside, while the classic menu is safer in the worst case and the expected values are equal.

		\item
		\subsection*{Logistics Model Selection Under Demand and Cost Uncertainty}
		A delivery company must choose either operating a central depot or using partner carriers because its contracts require one logistics model. Demand uncertainty is high demand (0.60) or low demand (0.40), and delivery-cost uncertainty is low (0.55) or high (0.45), based on historical order volumes and fuel invoices. Under high demand the central depot completes 800 deliveries while partners complete 750; under low demand the central depot completes 480 deliveries while partners complete 500. With the central depot the firm keeps the full contract fee of \$50 per delivery, while partner carriers share the fee, leaving \$48 per delivery in high demand and \$45 per delivery in low demand. Cost exposure reflects fixed rent, staffing, and dispatch overhead at the depot versus partner flexibility: the central depot has $18{,}000$ in costs in the low-cost state and $24{,}000$ in the high-cost state, while partners run $16{,}000$ in the low-cost state and $21{,}000$ in the high-cost state.
		
		\textbf{Question:} Compute profits and decide using Maximin, Expected Value, and Maximax.
		
		\subsection*{C2}
		\begin{center}
			\begin{tabular}{l p{0.74\linewidth}}
				\toprule
				Alternatives & Central depot; Partner carriers \\
				States of nature & Revenue: High demand (0.60), Low demand (0.40); Cost: Low delivery cost (0.55), High delivery cost (0.45) \\
				Events & Order volume paired with delivery cost conditions after the plan is chosen \\
				Consequences & Profit (revenue $-$ costs) for each plan and state \\
				Probabilities & Revenue: 0.60, 0.40; Cost: 0.55, 0.45 \\
				\bottomrule
			\end{tabular}
		\end{center}
		
		\subsection*{C3}
		Revenue by demand state (quantity $\times$ unit price = total revenue):
		\begin{center}
			\begin{tabular}{l c l l l}
				\toprule
				State & Probability & Alternative & Quantity & Unit price (USD) \\
				\midrule
				High demand & 0.60 & Central & 800 deliveries & \$50 \\
				High demand & 0.60 & Partner & 750 deliveries & \$48 \\
				Low demand & 0.40 & Central & 480 deliveries & \$50 \\
				Low demand & 0.40 & Partner & 500 deliveries & \$45 \\
				\bottomrule
			\end{tabular}
		\end{center}
		\begin{center}
			\begin{tabular}{l c l c}
				\toprule
				State & Probability & Alternative & Total revenue (USD) \\
				\midrule
				High demand & 0.60 & Central & $800 \times 50 = 40{,}000$ \\
				High demand & 0.60 & Partner & $750 \times 48 = 36{,}000$ \\
				Low demand & 0.40 & Central & $480 \times 50 = 24{,}000$ \\
				Low demand & 0.40 & Partner & $500 \times 45 = 22{,}500$ \\
				\bottomrule
			\end{tabular}
		\end{center}
		
		Cost by delivery-cost state (USD):
		\begin{center}
			\begin{tabular}{l c c c}
				\toprule
				State & Probability & Central & Partner \\
				\midrule
				Low cost & 0.55 & 18{,}000 & 16{,}000 \\
				High cost & 0.45 & 24{,}000 & 21{,}000 \\
				\bottomrule
			\end{tabular}
		\end{center}
		
		Profit table (profit = revenue $-$ costs):
		\begin{center}
			\begin{tabular}{l p{0.17\linewidth} p{0.23\linewidth} p{0.23\linewidth}}
				\toprule
				State of nature & Probability & Central & Partner \\
				\midrule
				High demand, low cost &
				$\begin{array}{l}
					0.60 \times 0.55\\
					= 0.33
				\end{array}$ &
				$\begin{array}{l}
					40{,}000 - 18{,}000\\
					= 22{,}000
				\end{array}$ &
				$\begin{array}{l}
					36{,}000 - 16{,}000\\
					= 20{,}000
				\end{array}$ \\
				High demand, high cost &
				$\begin{array}{l}
					0.60 \times 0.45\\
					= 0.27
				\end{array}$ &
				$\begin{array}{l}
					40{,}000 - 24{,}000\\
					= 16{,}000
				\end{array}$ &
				$\begin{array}{l}
					36{,}000 - 21{,}000\\
					= 15{,}000
				\end{array}$ \\
				Low demand, low cost &
				$\begin{array}{l}
					0.40 \times 0.55\\
					= 0.22
				\end{array}$ &
				$\begin{array}{l}
					24{,}000 - 18{,}000\\
					= 6{,}000
				\end{array}$ &
				$\begin{array}{l}
					22{,}500 - 16{,}000\\
					= 6{,}500
				\end{array}$ \\
				Low demand, high cost &
				$\begin{array}{l}
					0.40 \times 0.45\\
					= 0.18
				\end{array}$ &
				$\begin{array}{l}
					24{,}000 - 24{,}000\\
					= 0
				\end{array}$ &
				$\begin{array}{l}
					22{,}500 - 21{,}000\\
					= 1{,}500
				\end{array}$ \\
				\bottomrule
			\end{tabular}
		\end{center}
		
		Final payoff table (USD):
		\begin{center}
			\begin{tabular}{l c c c}
				\toprule
				State of nature & Probability & Central & Partner \\
				\midrule
				High demand, low cost & 0.33 & 22{,}000 & 20{,}000 \\
				High demand, high cost & 0.27 & 16{,}000 & 15{,}000 \\
				Low demand, low cost & 0.22 & 6{,}000 & 6{,}500 \\
				Low demand, high cost & 0.18 & 0 & 1{,}500 \\
				\bottomrule
			\end{tabular}
		\end{center}
		
		\subsection*{C4}
		Maximax compares the best profit for each plan.
		\[
		\begin{aligned}
			\max_i \text{Payoff}(\text{Central}, S_i) &= \max\{22{,}000, 16{,}000, 6{,}000, 0\} = 22{,}000, \\
			\max_i \text{Payoff}(\text{Partner}, S_i) &= \max\{20{,}000, 15{,}000, 6{,}500, 1{,}500\} = 20{,}000.
		\end{aligned}
		\]
		\[
		\max\{22{,}000, 20{,}000\} = 22{,}000.
		\]
		The maximax (risk-seeking) choice is the \emph{central depot} plan.
		
		\subsection*{C5}
		Maximin picks the highest of the worst-case profits.
		\[
		\begin{aligned}
			\min_i \text{Payoff}(\text{Central}, S_i) &= \min\{22{,}000, 16{,}000, 6{,}000, 0\} = 0, \\
			\min_i \text{Payoff}(\text{Partner}, S_i) &= \min\{20{,}000, 15{,}000, 6{,}500, 1{,}500\} = 1{,}500.
		\end{aligned}
		\]
		\[
		\max\{0, 1{,}500\} = 1{,}500.
		\]
		The maximin (risk-averse) choice is the \emph{partner carriers} plan.
		
		\subsection*{C1}
		The firm compares expected profits using the state probabilities.
		\[
		\begin{aligned}
			EV_{\text{Central}} &= 0.33(22{,}000) + 0.27(16{,}000) + 0.22(6{,}000) + 0.18(0) \\
			&= 7{,}260 + 4{,}320 + 1{,}320 + 0 = 12{,}900, \\
			EV_{\text{Partner}} &= 0.33(20{,}000) + 0.27(15{,}000) + 0.22(6{,}500) + 0.18(1{,}500) \\
			&= 6{,}600 + 4{,}050 + 1{,}430 + 270 = 12{,}350.
		\end{aligned}
		\]
		\[
		\max\{12{,}900, 12{,}350\} = 12{,}900.
		\]
		
		Summary of recommendations across decision criteria
		\begin{itemize}
			\item Maximax favors the \emph{central depot} plan (best payoff of 22{,}000).
			\item Maximin favors the \emph{partner carriers} plan (best worst-case payoff of 1{,}500).
			\item Expected value favors the \emph{central depot} plan (EV of 12{,}900).
		\end{itemize}
		
		Partner carriers provide stronger downside protection, while the central depot has the highest upside and expected value.

		\item
		\subsection*{Investment Choice Under Uncertain Market States}
		An investor is choosing between two investment options whose returns depend on whether state $S_1$ or $S_2$ occurs. What is the decision maker trying to achieve, what constraints exist in selecting a single option, what uncertainty drives the states of nature, why are options A and B the realistic choices, and why do payoffs differ across the states?
		
		\textbf{Question:} Let $p$ be the probability of state $S_1$. When is option A better than option B using Expected Value?
		
		\begin{center}
			\textit{Payoff table} \\
			\begin{tabular}{l c c}
				\toprule
				& $S_1$ $(p)$ & $S_2$ $(1-p)$ \\
				\midrule
				A & 22 & 6 \\
				B & 16 & 10 \\
				\bottomrule
			\end{tabular}
		\end{center}
		
		\subsection*{C7}
		\[
		EV(A)=22p+6(1-p)=16p+6
		\]
		\[
		EV(B)=16p+10(1-p)=6p+10
		\]
		Decision condition:
		\[
		EV(A)-EV(B)>0
		\]
		\[
		(16p+6)-(6p+10)>0 \Rightarrow 10p-4>0
		\]

		\item
		\subsection*{Production Plan Selection Under Uncertain Market Conditions}
		A firm must choose among three production plans whose profits depend on whether state $S_1$ or $S_2$ occurs. What is the decision maker trying to achieve, what constraints exist in committing to one plan, what uncertainty drives the states of nature, why are plans A, B, and C the realistic options, and why do payoffs differ across the states?
		
		\textbf{Question:} Let $p$ be the probability of state $S_1$. Determine for which values of $p$ each plan is optimal.
		
		\begin{center}
			\textit{Payoff table} \\
			\begin{tabular}{l c c}
				\toprule
				& $S_1$ $(p)$ & $S_2$ $(1-p)$ \\
				\midrule
				A & 30 & 0 \\
				B & 20 & 8 \\
				C & 14 & 12 \\
				\bottomrule
			\end{tabular}
		\end{center}
		
		\subsection*{C7}
		\[
		EV(A)=30p,\quad
		EV(B)=20p+8(1-p)=12p+8,\quad
		EV(C)=14p+12(1-p)=2p+12
		\]
		
		Pairwise comparison conditions in $>0$ form:
		\[
		EV(A)-EV(B)>0 \Rightarrow 30p-(12p+8)>0 \Rightarrow 18p-8>0
		\]
		\[
		EV(B)-EV(C)>0 \Rightarrow (12p+8)-(2p+12)>0 \Rightarrow 10p-4>0
		\]
		\[
		EV(A)-EV(C)>0 \Rightarrow 30p-(2p+12)>0 \Rightarrow 28p-12>0
		\]
	

		\item
		\subsection*{Bookstore Upgrade Decision Under Foot-Traffic Uncertainty}
		The bookstore is deciding between expanding its reading space or trying a mobile book cart because it only has budget for one upgrade this year. The main uncertainty is how many customers will visit, so the staff uses high and low foot-traffic states to capture the busiest weeks versus slower weeks. The probabilities come from last year’s daily counts and recent neighborhood survey results, which suggest a slightly higher chance of busy days. The payoffs make sense because expanding space brings in more sales and longer visits when traffic is high but leaves higher fixed costs when traffic is low, while the cart earns less in big crowds but keeps costs flexible in slow periods.
		
		\begin{center}
			\textit{Payoff table (given)} \\
			\begin{tabular}{l c c}
				\toprule
				Alternative & High traffic $(0.55)$ & Low traffic $(0.45)$ \\
				\midrule
				Expand space & 48 & 12 \\
				Mobile cart & 32 & 20 \\
				\bottomrule
			\end{tabular}
		\end{center}
		
		\textbf{Question:} Use the maximum opportunity (minimax regret) criterion to choose the best alternative. Then compare the choice to Maximax, Maximin, and Expected Value.
		
		\subsection*{C6}
		Best payoff in each state:
		\[
		\begin{aligned}
			\text{High traffic: } &\max\{48, 32\} = 48, \\
			\text{Low traffic: } &\max\{12, 20\} = 20.
		\end{aligned}
		\]
		Regret table (best payoff $-$ payoff):
		\begin{center}
			\begin{tabular}{l c c c}
				\toprule
				Alternative & High traffic $(0.55)$ & Low traffic $(0.45)$ & Maximum regret \\
				\midrule
				Expand space & $48-48=0$ & $20-12=8$ & 8 \\
				Mobile cart & $48-32=16$ & $20-20=0$ & 16 \\
				\bottomrule
			\end{tabular}
		\end{center}
		The minimax regret choice is \textbf{expand space} because it has the smaller maximum regret $(8)$.
		
		\subsection*{Comparison with other criteria}
		\begin{itemize}
			\item \textbf{Maximax:} Choose \emph{expand space} because its best payoff $(48)$ is larger than the mobile cart’s best payoff $(32)$. A risk-seeking decision-maker accepts uncertainty to chase the biggest gain, so this matches the C6 choice.
			\item \textbf{Maximin:} Choose \emph{mobile cart} because its worst payoff $(20)$ is higher than the expand option’s worst payoff $(12)$. A highly cautious decision-maker avoids low outcomes, so this differs from the C6 choice.
			\item \textbf{Expected Value:} The Expected Value favors \emph{expand space} because the higher payoff in the high-traffic state dominates when weighted by the given probabilities. This aligns with the C6 choice because it balances regret with the more likely outcome.
		\end{itemize}
	\end{ExamProblems}
	
	
	
\end{document}