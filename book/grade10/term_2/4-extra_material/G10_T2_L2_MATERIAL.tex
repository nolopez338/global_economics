\makeatletter
\def\input@path{{./}{../}{../../}{preamble/}{../preamble/}{../../preamble/}}
\makeatother
% ----------------------------------------------------------
% GENERAL 

% File
\documentclass[11pt]{book}

% Margins
\usepackage[margin=1in]{geometry}

\linespread{1.2}            % Line spacing
\usepackage[utf8]{inputenc}
\usepackage[T1]{fontenc}
\usepackage{lmodern}
\usepackage{microtype}
\setlength{\parindent}{0pt}
\setlength{\parskip}{6pt}
\usepackage{booktabs}

% ----------------------------------------------------------
% TABLES
\usepackage{multicol}
\usepackage{longtable} 
\usepackage{array}
\usepackage{booktabs}
\usepackage{tabularx}
\usepackage{multirow}

% ----------------------------------------------------------
% MATHEMATICS
\usepackage{amsmath}
\usepackage{amssymb}
\usepackage{amsfonts}
\usepackage{mathtools}

% ----------------------------------------------------------
% HIDDEN CONTENT
\usepackage{ifthen}
% Define a boolean switch
\newboolean{explicaciones}
% Set the boolean switch to true or false
% Change to true to show the content

% Explanations
\newcommand{\explicacion}[2]{
	\ifthenelse{\boolean{explicaciones}}{#1}{#2}
}
\newcommand{\mostrarExplicaciones}[1]{\setboolean{explicaciones}{#1}}

% ----------------------------------------------------------
% NUMBERING

\usepackage{fancyhdr}
\pagestyle{empty} % Ensures the entire document has no page numbers

\usepackage{tocloft}
\renewcommand{\cftdot}{} % Remove dots for sections, if any
\renewcommand{\cftsecleader}{\cftdotfill{\cftdotsep}} % Remove dots for sections, if any
\cftpagenumbersoff{section} % Removes page numbers from sections
\cftpagenumbersoff{subsection} % Removes page numbers from subsections

% ----------------------------------------------------------
% IMAGES 

% General settings
\usepackage{graphicx}       % Insert images
\usepackage{float}          % Position images
% \usepackage{subfigure}      % Subfigures
\graphicspath{{imgs}}       % Image location
\usepackage{subcaption}     % Subfigures II
\usepackage{verbatim}

% Figures
\usepackage{tikz}
\usetikzlibrary{arrows.meta,positioning,trees}

% Colors
\usepackage{xcolor}     
\definecolor{popUp}{HTML}{666666}
\definecolor{popUpIn}{HTML}{CED9E0}
\definecolor{backgroundC}{HTML}{D0E8F2}
\definecolor{backgroundCC}{HTML}{FFFFFF}
\definecolor{borders}{HTML}{8c120d}
\definecolor{padding}{HTML}{77D0D7}
\definecolor{links}{HTML}{CC6F5F}

% ----------------------------------------------------------
% FRAMES

% General settings
\usepackage{tcolorbox}
\usepackage{adjustbox}          % Adjusted frame  
\setlength{\fboxrule}{3pt}  % Line width
\setlength{\fboxsep}{3pt}   % Box padding

% General frames
\usepackage{mdframed}   

\mdfdefinestyle{estiloGeneral}{    % General style
	linecolor=black,
	linewidth=1.5pt,
	roundcorner=10pt,
	backgroundcolor=backgroundC,
	innerbottommargin=0pt
}
\mdfdefinestyle{code}{          % Code style
	linecolor=black,
	linewidth=1.5pt,
	roundcorner=10pt,
	backgroundcolor=darkgray!10,
	innerbottommargin=0pt
}

% Image frame
\newtcbox{\fboxC}{
	colback=backgroundC,
	colframe=popUp,
	arc=10pt,
	boxrule=3pt,
	boxsep=0pt, % Change the padding here
	nobeforeafter
}

% ----------------------------------------------------------
% PAGE SETTINGS

% Background 
\newcommand{\background}[0]{\begin{tikzpicture}[remember picture,overlay]
		\fill[backgroundC] (-2,2) rectangle (25cm, -550);
\end{tikzpicture}}

\newcommand{\backgroundC}[0]{\begin{tikzpicture}[remember picture,overlay]
		\fill[backgroundCC] (-2,2) rectangle (25cm, -550);
\end{tikzpicture}}

% Page width 
\newcommand{\anchoPag}[0]{20cm}

% ----------------------------------------------------------
% FONT

% General
\usepackage{tgbonum}        % Font
\usepackage{listings}       % Code typesetting
\usepackage[spanish]{babel} % Load Spanish
\selectlanguage{spanish}    % Select Spanish
\usepackage{enumitem}
\usepackage{bookmark}

\setlist[itemize]{leftmargin=1.2em, itemsep=0.35em, topsep=0.35em}

% --- Table helpers ---
\newcolumntype{L}[1]{>{\raggedright\arraybackslash}p{#1}}
\newcolumntype{Y}{>{\raggedright\arraybackslash}X}
\newcolumntype{C}{>{\centering\arraybackslash}X}
\renewcommand{\arraystretch}{1.1}

% Python style
\lstdefinestyle{python}{
	language=Python,
	basicstyle=\ttfamily\small,
	commentstyle=\color{green!50!black},
	keywordstyle=\color{blue},
	numberstyle=\tiny\color{gray},
	numbers=left,
	morekeywords={>, <},
	breakatwhitespace=false,
	showstringspaces=false,
	showtabs=false,
	showspaces=false
}

% ----------------------------------------------------------
% HYPERLINKS

% General
\usepackage{hyperref}       
\hypersetup{
	colorlinks=true,
	linkcolor=links,
	filecolor=magenta,      
	urlcolor=brown,
}

% Custom commands 

% Large link
\newcommand{\bigLink}[2]{\begin{center} \fboxC{\LARGE{\href{#1}{#2}}}\end{center}}

% Small link
\newcommand{\smallLink}[2]{\begin{center}\fboxC{\href{#1}{#2}}\end{center}}

% Bold link
\newcommand{\bfLink}[2]{\href{#1}{\textbf{#2}}}


% Small URL
\newcommand{\smallUrl}[1]{\begin{center}\fboxC{\url{#1}}\end{center}}


% ----------------------------------------------------------
% CUSTOM COMMANDS FOR FIGURES

\newcommand{\espacioImagenes}[0]{-1.2cm}

% Without frame
\newcommand{\fig}[3][\espacioImagenes]{
	\hspace*{#1}
	\centering
	\includegraphics[width=#2\textwidth]{#3}
}

% With frame
\newcommand{\ffig}[2]{\begin{figure}[h]
		\hspace*{\espacioImagenes}
		\centering
		\fbox{\includegraphics[width=#1\textwidth]{#2}}
\end{figure}}

% Hyperlink with frame
\newcommand{\hfig}[3]{\begin{figure}[h]
		\hspace*{-1.4cm}
		\centering
		\color{popUp}
		\fboxC{\href{#1}{\includegraphics[width=#2\textwidth]{#3}}}
	\end{figure}
}

% Hyperlink without frame
\newcommand{\hffig}[3]{\begin{figure}[h]
		\hspace*{-1.1cm}
		\centering
		\color{popUp}
		\href{#1}{\includegraphics[width=#2\textwidth]{#3}}
	\end{figure}
}

% ----------------------------------------------------------

% Start and Contents
\newcommand{\cuadro}[1]{
	\begin{mdframed}[style=estiloGeneral]
		#1 
	\end{mdframed}
}

% Explanation video image
\newcommand{\linkExplicacion}[1]{
	\hffig{#1}{0.5}{principal/videoExplicacion}
	\vspace{-0.5cm}
}

\newcommand{\subSecLink}[2]{
	\subsubsection*{\href{#1}{\textbf{#2}}}
}

% Spacing
\newcommand{\esp}[0]{\vspace{4mm}}

% Back to start
\newcommand{\secInicio}[0]{\begin{center}\hyperref[sec:inicio]{ 
			\includegraphics[width=0.1\textwidth]{principal/up}
	}\end{center}
}


\geometry{margin=0.85in}
\AtBeginDocument{\small}

\newcommand{\ExamNameField}{\noindent\textbf{Name:}\ \rule{0.7\linewidth}{0.4pt}\par\medskip}

\newcommand{\ExamTitleBlock}[3]{%
	\begin{center}
		\Large\textbf{#1}\\
		\textbf{#2}%
		\if\relax\detokenize{#3}\relax\else\\\textbf{#3}\fi
	\end{center}
	\vspace{0.5em}
}

\newcommand{\ExamSection}[1]{\par\medskip\textbf{#1}\par\smallskip}

\newenvironment{ExamCriteria}{%
	\begin{itemize}[leftmargin=1.6em, itemsep=0.3em, topsep=0.2em]
}{%
	\end{itemize}
}

\newenvironment{ExamProblems}{%
	\begin{enumerate}[label=\textbf{P\arabic*}, leftmargin=0pt, labelsep=0.6em, itemindent=2.2em, itemsep=0.8em]
}{%
	\end{enumerate}
}


\begin{document}
	\ExamTitleBlock{10th grade}{Learning evidence T2 L2 Decision analysis supplementary material solutions}{}
	
	\section*{Contents}
	\noindent\textbf{C6 Develops decision-making strategies using probabilities and the maximum opportunity criterion.}
	\begin{itemize}
		\item \hyperlink{c6-ex1}{Problem 1 — Community Bookstore Upgrade}
		\item \hyperlink{c6-ex2}{Problem 2 — Cooperative Packing Line Choice}
		\item \hyperlink{c6-ex3}{Problem 3 — Craft Export Shipping Plan}
		\item \hyperlink{c6-ex4}{Problem 4 — Campus Vending Options}
	\end{itemize}
	\noindent\textbf{C7 Uses probabilities and expected value to analyze a decision-making problem.}
	\begin{itemize}
		\item \hyperlink{c7-ex1}{Problem 1 — Single Project under Uncertainty}
		\item \hyperlink{c7-ex2}{Problem 2 — Technology Upgrade}
		\item \hyperlink{c7-ex3}{Problem 3 — Two Investment Alternatives}
		\item \hyperlink{c7-ex4}{Problem 4 — Three Production Plans}
		\item \hyperlink{c7-ex5}{Problem 5 — Two Alternatives with Three States}
		\item \hyperlink{c7-ex6}{Problem 6 — Policy Choice under Three States}
		\item \hyperlink{c7-ex7}{Problem 7 — New Service Platform Launch}
		\item \hyperlink{c7-ex8}{Problem 8 — Regional Store Opening}
		\item \hyperlink{c7-ex9}{Problem 9 — Marketing Strategy Choice}
		\item \hyperlink{c7-ex10}{Problem 10 — Investment Portfolio Selection}
	\end{itemize}
	
	\ExamSection{C6 Develops decision-making strategies using probabilities and the maximum opportunity criterion.}
	
	% Practice Activity 2.1 — Maximum Opportunity Criterion (Minimax Regret)
	% Problems and worked solutions extracted from the provided webpage content.
	
	% -------------------------
	% Problem 1 — Community Bookstore Upgrade
	% -------------------------
	\hypertarget{c6-ex1}{}
	\subsection*{Problem 1 — Community Bookstore Upgrade}
	
	\textbf{Problem.} A community bookstore is deciding how to update its storefront.
	
	\begin{itemize}
		\item Alternatives (profits in thousands of USD): Expand reading space; Launch a mobile book cart.
		\item States of nature and probabilities: High foot traffic ($0.55$); Low foot traffic ($0.45$).
	\end{itemize}
	
	Payoff table (given):
	\[
	\begin{array}{lcc}
		\toprule
		\text{Alternative} & \text{High traffic }(0.55) & \text{Low traffic }(0.45)\\
		\midrule
		\text{Expand space} & 48 & 12\\
		\text{Mobile cart} & 32 & 20\\
		\bottomrule
	\end{array}
	\]
	
	\textbf{Question.} Use the maximum opportunity (minimax regret) criterion to choose the best alternative. Then compare the choice to Maximax, Maximin, and Expected Value.
	
	\textbf{Solution.}
	
	\textbf{(C6) Maximum opportunity criterion (minimax regret).} Best payoff in each state:
	\[
	\begin{aligned}
		\text{High traffic: } &\max\{48, 32\} = 48\\
		\text{Low traffic: }  &\max\{12, 20\} = 20
	\end{aligned}
	\]
	
	Regret table (best payoff $-$ payoff):
	\[
	\begin{array}{lccc}
		\toprule
		\text{Alternative} & \text{High traffic} & \text{Low traffic} & \text{Maximum regret}\\
		\midrule
		\text{Expand space} & 48-48=0 & 20-12=8 & 8\\
		\text{Mobile cart}  & 48-32=16 & 20-20=0 & 16\\
		\bottomrule
	\end{array}
	\]
	
	Minimax regret choice: \emph{Expand space}, since its maximum regret is $8$ (smaller than $16$).
	
	\textbf{Comparison with other criteria.}
	\begin{itemize}
		\item \textbf{Maximax:} Choose \emph{Expand space} since $\max(48,12)=48$ is larger than $\max(32,20)=32$.
		\item \textbf{Maximin:} Choose \emph{Mobile cart} since $\min(32,20)=20$ is larger than $\min(48,12)=12$.
		\item \textbf{Expected Value:} Favors \emph{Expand space} because the higher payoff in the high-traffic state dominates when weighted by the given probabilities.
	\end{itemize}
	
	% -------------------------
	% Problem 2 — Cooperative Packing Line Choice
	% -------------------------
	\hypertarget{c6-ex2}{}
	\subsection*{Problem 2 — Cooperative Packing Line Choice}
	
	\textbf{Problem.} A dairy cooperative is choosing a packing line. The profit payoff table below was obtained from estimated revenues and costs under each order size.
	
	\begin{itemize}
		\item Alternatives: Automated packing line; Manual packing line.
		\item States of nature (order size): High order ($0.30$); Medium order ($0.45$); Low order ($0.25$).
	\end{itemize}
	
	Payoff table (profit in thousand USD, given):
	\[
	\begin{array}{lccc}
		\toprule
		\text{Alternative} & \text{High }(0.30) & \text{Medium }(0.45) & \text{Low }(0.25)\\
		\midrule
		\text{Automated} & 42 & 28 & 8\\
		\text{Manual} & 30 & 24 & 12\\
		\bottomrule
	\end{array}
	\]
	
	\textbf{Question.} Apply the maximum opportunity criterion and then compare with Maximax, Maximin, and Expected Value.
	
	\textbf{Solution.}
	
	\textbf{(C6) Maximum opportunity criterion (minimax regret).} Best payoff in each state:
	\[
	\begin{aligned}
		\text{High: } &\max\{42, 30\} = 42\\
		\text{Medium: } &\max\{28, 24\} = 28\\
		\text{Low: } &\max\{8, 12\} = 12
	\end{aligned}
	\]
	
	Regret table (best payoff $-$ payoff):
	\[
	\begin{array}{lcccc}
		\toprule
		\text{Alternative} & \text{High} & \text{Medium} & \text{Low} & \text{Maximum regret}\\
		\midrule
		\text{Automated} & 42-42=0 & 28-28=0 & 12-8=4 & 4\\
		\text{Manual} & 42-30=12 & 28-24=4 & 12-12=0 & 12\\
		\bottomrule
	\end{array}
	\]
	
	Minimax regret choice: \emph{Automated packing}, since its maximum regret is $4$ (smaller than $12$).
	
	\textbf{Comparison with other criteria.}
	\begin{itemize}
		\item \textbf{Maximax:} Choose \emph{Automated} since its best payoff is $42$ (greater than $30$).
		\item \textbf{Maximin:} Choose \emph{Manual} since its worst payoff is $12$ (greater than $8$).
		\item \textbf{Expected Value:} Favors \emph{Automated} because the high and medium outcomes carry more total probability and higher payoffs.
	\end{itemize}
	
	% -------------------------
	% Problem 3 — Craft Export Shipping Plan
	% -------------------------
	\hypertarget{c6-ex3}{}
	\subsection*{Problem 3 — Craft Export Shipping Plan}
	
	\textbf{Problem.} A cooperative that exports handmade crafts must choose a shipping plan. It can ship by air, ship by sea, or delay shipment. Demand and exchange rates combine into four states of nature:
	\begin{itemize}
		\item S1 (high demand + strong currency), $P=0.20$
		\item S2 (high demand + weak currency), $P=0.30$
		\item S3 (low demand + strong currency), $P=0.25$
		\item S4 (low demand + weak currency), $P=0.25$
	\end{itemize}
	
	Payoff table (profit in thousand USD, given):
	\[
	\begin{array}{lcccc}
		\toprule
		\text{Alternative} & \text{S1 }(0.20) & \text{S2 }(0.30) & \text{S3 }(0.25) & \text{S4 }(0.25)\\
		\midrule
		\text{Air} & 60 & 48 & 20 & 10\\
		\text{Sea} & 50 & 44 & 26 & 18\\
		\text{Delay} & 35 & 32 & 28 & 24\\
		\bottomrule
	\end{array}
	\]
	
	\textbf{Question.} Apply the maximum opportunity criterion and then compare with Maximax, Maximin, and Expected Value.
	
	\textbf{Solution.}
	
	\textbf{(C6) Maximum opportunity criterion (minimax regret).} Best payoff in each state:
	\[
	\begin{aligned}
		\text{S1: } &\max\{60, 50, 35\} = 60\\
		\text{S2: } &\max\{48, 44, 32\} = 48\\
		\text{S3: } &\max\{20, 26, 28\} = 28\\
		\text{S4: } &\max\{10, 18, 24\} = 24
	\end{aligned}
	\]
	
	Regret table (best payoff $-$ payoff):
	\[
	\begin{array}{lccccc}
		\toprule
		\text{Alternative} & \text{S1} & \text{S2} & \text{S3} & \text{S4} & \text{Maximum regret}\\
		\midrule
		\text{Air} & 60-60=0 & 48-48=0 & 28-20=8 & 24-10=14 & 14\\
		\text{Sea} & 60-50=10 & 48-44=4 & 28-26=2 & 24-18=6 & 10\\
		\text{Delay} & 60-35=25 & 48-32=16 & 28-28=0 & 24-24=0 & 25\\
		\bottomrule
	\end{array}
	\]
	
	Minimax regret choice: \emph{Ship by sea}, since its maximum regret is $10$ (smaller than $14$ and $25$).
	
	\textbf{Comparison with other criteria.}
	\begin{itemize}
		\item \textbf{Maximax:} Choose \emph{Air} since its best payoff is $60$ (largest).
		\item \textbf{Maximin:} Choose \emph{Delay} since its worst payoff is $24$ (greater than $10$ and $18$).
		\item \textbf{Expected Value:} Favors \emph{Sea} because it balances payoffs with moderate risk across the four states.
	\end{itemize}
	
	% -------------------------
	% Problem 4 — Campus Vending Options
	% -------------------------
	\hypertarget{c6-ex4}{}
	\subsection*{Problem 4 — Campus Vending Options}
	
	\textbf{Problem.} A student council is selecting a vending option for campus events. It can choose smart vending kiosks or pop-up snack stalls. Campus traffic can be high ($P=0.40$), medium ($P=0.35$), or low ($P=0.25$). The profit payoff table below was obtained from estimated revenues and costs under each scenario.
	
	Payoff table (profit in thousand USD, given):
	\[
	\begin{array}{lccc}
		\toprule
		\text{Alternative} & \text{High }(0.40) & \text{Medium }(0.35) & \text{Low }(0.25)\\
		\midrule
		\text{Kiosks} & 25 & 12 & -1\\
		\text{Pop-up stalls} & 23 & 14 & 5\\
		\bottomrule
	\end{array}
	\]
	
	\textbf{Question.} Apply the maximum opportunity criterion and then compare with Maximax, Maximin, and Expected Value.
	
	\textbf{Solution.}
	
	\textbf{(C6) Maximum opportunity criterion (minimax regret).} Best payoff in each state:
	\[
	\begin{aligned}
		\text{High: } &\max\{25, 23\} = 25\\
		\text{Medium: } &\max\{12, 14\} = 14\\
		\text{Low: } &\max\{-1, 5\} = 5
	\end{aligned}
	\]
	
	Regret table (best payoff $-$ payoff):
	\[
	\begin{array}{lcccc}
		\toprule
		\text{Alternative} & \text{High} & \text{Medium} & \text{Low} & \text{Maximum regret}\\
		\midrule
		\text{Kiosks} & 25-25=0 & 14-12=2 & 5-(-1)=6 & 6\\
		\text{Pop-up stalls} & 25-23=2 & 14-14=0 & 5-5=0 & 2\\
		\bottomrule
	\end{array}
	\]
	
	Minimax regret choice: \emph{Pop-up stalls}, since its maximum regret is $2$ (smaller than $6$).
	
	\textbf{Comparison with other criteria.}
	\begin{itemize}
		\item \textbf{Maximax:} Choose \emph{Kiosks} since its best payoff is $25$ (greater than $23$).
		\item \textbf{Maximin:} Choose \emph{Pop-up stalls} since its worst payoff is $5$ (greater than $-1$).
		\item \textbf{Expected Value:} Favors \emph{Pop-up stalls} because the medium and low outcomes have substantial probabilities and higher payoffs than kiosks in those states.
	\end{itemize}
	
	
	\ExamSection{C7 Uses probabilities and expected value to analyze a decision-making problem.}
	
	\begin{ExamProblems}
		
		\hypertarget{c7-ex1}{}
		\item
		\subsection*{Problem 1 — Single Project under Uncertainty}
		
		\textbf{Problem.}
		A firm considers launching a project. Profit depends on the state of nature.
		
		\[
		\begin{array}{lcc}
			\hline
			& S_1\,(p) & S_2\,(1-p) \\
			\hline
			\text{Project} & 18 & -4 \\
			\hline
		\end{array}
		\]
		
		Let \(p\) be the probability of state \(S_1\), and \(1-p\) the probability of state \(S_2\).  
		Determine for which values of \(p\) the project is profitable using Expected Value.
		
		\textbf{Solution.}
		Expected value:
		\[
		EV = 18\cdot p + (-4)\cdot(1-p)
		\]
		Expand:
		\[
		EV = 18p - 4(1-p) = 18p - 4 + 4p
		\]
		Combine like terms:
		\[
		EV = 22p - 4
		\]
		Profitability condition:
		\[
		EV>0 \Rightarrow 22p-4>0
		\]
		Solve:
		\[
		22p>4 \Rightarrow p>\frac{4}{22}=\frac{2}{11}
		\]
		So the project is profitable when \(p>\frac{2}{11}\).
		
		% --------------------------------------------------
		
		\hypertarget{c7-ex2}{}
		\item
		\subsection*{Problem 2 — Technology Upgrade}
		
		\textbf{Problem.}
		A company considers upgrading its technology. Profit depends on the state of nature.
		
		\[
		\begin{array}{lcc}
			\hline
			& S_1\,(p) & S_2\,(1-p) \\
			\hline
			\text{Upgrade} & 25 & -10 \\
			\hline
		\end{array}
		\]
		
		Let \(p\) be the probability of state \(S_1\).  
		Determine for which values of \(p\) the upgrade is profitable.
		
		\textbf{Solution.}
		Expected value:
		\[
		EV = 25\cdot p + (-10)\cdot(1-p)
		\]
		Expand:
		\[
		EV = 25p - 10(1-p) = 25p - 10 + 10p
		\]
		Combine like terms:
		\[
		EV = 35p - 10
		\]
		Profitability condition:
		\[
		EV>0 \Rightarrow 35p-10>0
		\]
		Solve:
		\[
		35p>10 \Rightarrow p>\frac{10}{35}=\frac{2}{7}
		\]
		So the upgrade is profitable when \(p>\frac{2}{7}\).
		
		% --------------------------------------------------
		
		\hypertarget{c7-ex3}{}
		\item
		\subsection*{Problem 3 — Two Investment Alternatives}
		
		\textbf{Problem.}
		Two investment options depend on the state of nature.
		
		\[
		\begin{array}{lcc}
			\hline
			& S_1\,(p) & S_2\,(1-p) \\
			\hline
			A & 22 & 6 \\
			B & 16 & 10 \\
			\hline
		\end{array}
		\]
		
		Let \(p\) be the probability of state \(S_1\).  
		When is option A better than option B using Expected Value.
		
		\textbf{Solution.}
		Compute expected value of A:
		\[
		EV(A)=22\cdot p + 6\cdot(1-p)
		\]
		Expand and simplify:
		\[
		EV(A)=22p + 6 - 6p = (22p-6p)+6 = 16p+6
		\]
		Compute expected value of B:
		\[
		EV(B)=16\cdot p + 10\cdot(1-p)
		\]
		Expand and simplify:
		\[
		EV(B)=16p + 10 - 10p = (16p-10p)+10 = 6p+10
		\]
		A is better than B when \(EV(A)>EV(B)\), equivalently:
		\[
		EV(A)-EV(B)>0
		\]
		Substitute:
		\[
		(16p+6)-(6p+10)>0
		\]
		Simplify:
		\[
		16p+6-6p-10>0 \Rightarrow 10p-4>0
		\]
		Solve:
		\[
		10p>4 \Rightarrow p>\frac{4}{10}=\frac{2}{5}
		\]
		So option A is better than option B when \(p>\frac{2}{5}\).
		
		% --------------------------------------------------
		
		\hypertarget{c7-ex4}{}
		\item
		\subsection*{Problem 4 — Three Production Plans}
		
		\textbf{Problem.}
		A firm must choose among three production plans.
		
		\[
		\begin{array}{lcc}
			\hline
			& S_1\,(p) & S_2\,(1-p) \\
			\hline
			A & 30 & 0 \\
			B & 20 & 8 \\
			C & 14 & 12 \\
			\hline
		\end{array}
		\]
		
		Let \(p\) be the probability of state \(S_1\).  
		Determine for which values of \(p\) each plan is optimal.
		
		\textbf{Solution.}
		Compute expected value of each plan.
		
		Plan A:
		\[
		EV(A)=30\cdot p + 0\cdot(1-p)=30p
		\]
		
		Plan B:
		\[
		EV(B)=20\cdot p + 8\cdot(1-p)
		\]
		Expand:
		\[
		EV(B)=20p + 8 - 8p = 12p+8
		\]
		
		Plan C:
		\[
		EV(C)=14\cdot p + 12\cdot(1-p)
		\]
		Expand:
		\[
		EV(C)=14p + 12 - 12p = 2p+12
		\]
		
		Now compare pairwise using \(>0\) conditions.
		
		A better than B:
		\[
		EV(A)-EV(B)>0 \Rightarrow 30p-(12p+8)>0
		\]
		Simplify:
		\[
		30p-12p-8>0 \Rightarrow 18p-8>0 \Rightarrow p>\frac{8}{18}=\frac{4}{9}
		\]
		
		B better than C:
		\[
		EV(B)-EV(C)>0 \Rightarrow (12p+8)-(2p+12)>0
		\]
		Simplify:
		\[
		12p+8-2p-12>0 \Rightarrow 10p-4>0 \Rightarrow p>\frac{4}{10}=\frac{2}{5}
		\]
		
		A better than C:
		\[
		EV(A)-EV(C)>0 \Rightarrow 30p-(2p+12)>0
		\]
		Simplify:
		\[
		30p-2p-12>0 \Rightarrow 28p-12>0 \Rightarrow p>\frac{12}{28}=\frac{3}{7}
		\]
		
		Determine optimal plan by intervals:
		\begin{itemize}
			\item If \(p<\frac{2}{5}\), then \(EV(C)>EV(B)\). Also \(EV(C)>EV(A)\) because \(EV(A)-EV(C)>0\) requires \(p>\frac{3}{7}\). So for \(p<\frac{2}{5}\), plan C is optimal.
			\item If \(\frac{2}{5}<p<\frac{4}{9}\), then \(EV(B)>EV(C)\) and \(EV(A)<EV(B)\) (since \(p<\frac{4}{9}\)). So in this interval, plan B is optimal.
			\item If \(p>\frac{4}{9}\), then \(EV(A)>EV(B)\) and also \(EV(A)>EV(C)\) (since \(\frac{4}{9}>\frac{3}{7}\)). So for \(p>\frac{4}{9}\), plan A is optimal.
		\end{itemize}
		
		% --------------------------------------------------
		
		\hypertarget{c7-ex5}{}
		\item
		\subsection*{Problem 5 — Two Alternatives with Three States}
		
		\textbf{Problem.}
		Two strategies depend on three states of nature.
		
		\[
		\begin{array}{lccc}
			\hline
			& S_1\,(p_1) & S_2\,(p_2) & S_3\,(1-p_1-p_2) \\
			\hline
			A & 20 & 10 & -5 \\
			B & 12 & 14 & 0 \\
			\hline
		\end{array}
		\]
		
		Let \(p_1=P(S_1)\), \(p_2=P(S_2)\), and \(1-p_1-p_2=P(S_3)\).  
		When is strategy A better than strategy B using Expected Value.
		
		\textbf{Solution.}
		Compute expected value of A:
		\[
		EV(A)=20p_1 + 10p_2 + (-5)(1-p_1-p_2)
		\]
		Expand the last term:
		\[
		EV(A)=20p_1 + 10p_2 -5 + 5p_1 + 5p_2
		\]
		Combine like terms:
		\[
		EV(A)=(20p_1+5p_1) + (10p_2+5p_2) - 5 = 25p_1 + 15p_2 - 5
		\]
		Compute expected value of B:
		\[
		EV(B)=12p_1 + 14p_2 + 0\cdot(1-p_1-p_2)=12p_1+14p_2
		\]
		A better than B when:
		\[
		EV(A)-EV(B)>0
		\]
		Substitute:
		\[
		(25p_1+15p_2-5)-(12p_1+14p_2)>0
		\]
		Simplify:
		\[
		25p_1+15p_2-5-12p_1-14p_2>0
		\Rightarrow 13p_1+p_2-5>0
		\]
		So strategy A is better than strategy B when \(13p_1+p_2-5>0\).
		
		% --------------------------------------------------
		
		\hypertarget{c7-ex6}{}
		\item
		\subsection*{Problem 6 — Policy Choice under Three States}
		
		\textbf{Problem.}
		A government compares two policies under three states of nature.
		
		\[
		\begin{array}{lccc}
			\hline
			& S_1\,(p_1) & S_2\,(p_2) & S_3\,(1-p_1-p_2) \\
			\hline
			A & 18 & 8 & -6 \\
			B & 10 & 12 & 2 \\
			\hline
		\end{array}
		\]
		
		Let \(p_1=P(S_1)\), \(p_2=P(S_2)\), and \(1-p_1-p_2=P(S_3)\).  
		When is policy A better than policy B.
		
		\textbf{Solution.}
		Compute expected value of A:
		\[
		EV(A)=18p_1 + 8p_2 + (-6)(1-p_1-p_2)
		\]
		Expand:
		\[
		EV(A)=18p_1 + 8p_2 -6 + 6p_1 + 6p_2
		\]
		Combine:
		\[
		EV(A)=(18p_1+6p_1) + (8p_2+6p_2) - 6 = 24p_1 + 14p_2 - 6
		\]
		Compute expected value of B:
		\[
		EV(B)=10p_1 + 12p_2 + 2(1-p_1-p_2)
		\]
		Expand:
		\[
		EV(B)=10p_1 + 12p_2 + 2 - 2p_1 - 2p_2
		\]
		Combine:
		\[
		EV(B)=(10p_1-2p_1) + (12p_2-2p_2) + 2 = 8p_1 + 10p_2 + 2
		\]
		A better than B when:
		\[
		EV(A)-EV(B)>0
		\]
		Substitute:
		\[
		(24p_1+14p_2-6)-(8p_1+10p_2+2)>0
		\]
		Simplify:
		\[
		24p_1+14p_2-6-8p_1-10p_2-2>0
		\Rightarrow 16p_1+4p_2-8>0
		\]
		Divide by \(4\) (positive, inequality direction unchanged):
		\[
		4p_1+p_2-2>0
		\]
		So policy A is better than policy B when \(4p_1+p_2-2>0\).
		
		% --------------------------------------------------
		
		\hypertarget{c7-ex7}{}
		\item
		\subsection*{Problem 7 — New Service Platform Launch}
		
		\textbf{Problem.}
		A company considers launching a digital service. Profit depends on two independent factors:
		customer adoption and system reliability.
		
		\[
		\begin{array}{lcc}
			\hline
			& \text{Stable}\,(q) & \text{Unstable}\,(1-q) \\
			\hline
			\text{High adoption}\,(p) & 28 & 6 \\
			\text{Low adoption}\,(1-p) & -4 & -12 \\
			\hline
		\end{array}
		\]
		
		Let \(p\) be the probability of high adoption and \(q\) the probability of system stability.  
		Find \(EV(p,q)\) and determine when launching is profitable.
		
		\textbf{Solution.}
		By independence, joint probabilities:
		\[
		P(\text{High,Stable})=pq,\quad
		P(\text{High,Unstable})=p(1-q),\quad
		P(\text{Low,Stable})=(1-p)q,\quad
		P(\text{Low,Unstable})=(1-p)(1-q)
		\]
		Expected value:
		\[
		EV=28(pq)+6\bigl(p(1-q)\bigr)+(-4)\bigl((1-p)q\bigr)+(-12)\bigl((1-p)(1-q)\bigr)
		\]
		Expand term by term:
		\[
		EV=28pq + 6p - 6pq - 4q + 4pq - 12(1-p-q+pq)
		\]
		Expand the last product:
		\[
		-12(1-p-q+pq)=-12 + 12p + 12q - 12pq
		\]
		Combine all terms:
		\[
		EV=(28pq-6pq+4pq-12pq) + (6p+12p) + (-4q+12q) - 12
		\]
		\[
		EV=(14pq) + 18p + 8q - 12
		\]
		Profitability condition:
		\[
		EV>0 \Rightarrow 14pq+18p+8q-12>0
		\]
		
		% --------------------------------------------------
		
		\hypertarget{c7-ex8}{}
		\item
		\subsection*{Problem 8 — Regional Store Opening}
		
		\textbf{Problem.}
		A retailer evaluates opening a new store. Profit depends on demand and competitor response.
		
		\[
		\begin{array}{lcc}
			\hline
			& \text{Mild}\,(q) & \text{Aggressive}\,(1-q) \\
			\hline
			\text{Strong demand}\,(p) & 35 & 12 \\
			\text{Weak demand}\,(1-p) & -5 & -20 \\
			\hline
		\end{array}
		\]
		
		Let \(p\) be the probability of strong demand and \(q\) the probability of a mild response.  
		Determine when opening the store is profitable.
		
		\textbf{Solution.}
		Joint probabilities:
		\[
		P(\text{Strong,Mild})=pq,\quad
		P(\text{Strong,Aggressive})=p(1-q),\quad
		P(\text{Weak,Mild})=(1-p)q,\quad
		P(\text{Weak,Aggressive})=(1-p)(1-q)
		\]
		Expected value:
		\[
		EV=35(pq)+12\bigl(p(1-q)\bigr)+(-5)\bigl((1-p)q\bigr)+(-20)\bigl((1-p)(1-q)\bigr)
		\]
		Expand:
		\[
		EV=35pq + 12p - 12pq - 5q + 5pq - 20(1-p-q+pq)
		\]
		Expand last term:
		\[
		-20(1-p-q+pq)=-20 + 20p + 20q - 20pq
		\]
		Combine:
		\[
		EV=(35pq-12pq+5pq-20pq) + (12p+20p) + (-5q+20q) - 20
		\]
		\[
		EV=8pq + 32p + 15q - 20
		\]
		Profitability condition:
		\[
		EV>0 \Rightarrow 8pq+32p+15q-20>0
		\]
		
		% --------------------------------------------------
		
		\hypertarget{c7-ex9}{}
		\item
		\subsection*{Problem 9 — Marketing Strategy Choice}
		
		\textbf{Problem.}
		A firm must choose between two marketing strategies. Outcomes depend on market interest
		and campaign execution.
		
		\[
		\begin{array}{lcc}
			\hline
			\text{State of nature} & A & B \\
			\hline
			\text{High interest}\,(p),\,\text{Effective}\,(q) & 30 & 22 \\
			\text{High interest}\,(p),\,\text{Ineffective}\,(1-q) & 10 & 14 \\
			\text{Low interest}\,(1-p),\,\text{Effective}\,(q) & -6 & 4 \\
			\text{Low interest}\,(1-p),\,\text{Ineffective}\,(1-q) & -14 & -8 \\
			\hline
		\end{array}
		\]
		
		Let \(p\) be the probability of high interest and \(q\) the probability of effective execution.  
		When is strategy A better than strategy B.
		
		\textbf{Solution.}
		Compute \(EV_A\) by weighting each payoff:
		\[
		EV_A=30(pq)+10\bigl(p(1-q)\bigr)+(-6)\bigl((1-p)q\bigr)+(-14)\bigl((1-p)(1-q)\bigr)
		\]
		Expand:
		\[
		EV_A=30pq + 10p - 10pq - 6q + 6pq - 14(1-p-q+pq)
		\]
		Expand last term:
		\[
		-14(1-p-q+pq)=-14 + 14p + 14q - 14pq
		\]
		Combine:
		\[
		EV_A=(30pq-10pq+6pq-14pq) + (10p+14p) + (-6q+14q) - 14
		\]
		\[
		EV_A=12pq + 24p + 8q - 14
		\]
		
		Compute \(EV_B\):
		\[
		EV_B=22(pq)+14\bigl(p(1-q)\bigr)+4\bigl((1-p)q\bigr)+(-8)\bigl((1-p)(1-q)\bigr)
		\]
		Expand:
		\[
		EV_B=22pq + 14p - 14pq + 4q - 4pq - 8(1-p-q+pq)
		\]
		Expand last term:
		\[
		-8(1-p-q+pq)=-8 + 8p + 8q - 8pq
		\]
		Combine:
		\[
		EV_B=(22pq-14pq-4pq-8pq) + (14p+8p) + (4q+8q) - 8
		\]
		\[
		EV_B=-4pq + 22p + 12q - 8
		\]
		
		Strategy A better than B when:
		\[
		EV_A-EV_B>0
		\]
		Substitute:
		\[
		(12pq+24p+8q-14)-(-4pq+22p+12q-8)>0
		\]
		Simplify:
		\[
		12pq+24p+8q-14+4pq-22p-12q+8>0
		\Rightarrow 16pq+2p-4q-6>0
		\]
		
		% --------------------------------------------------
		
		\hypertarget{c7-ex10}{}
		\item
		\subsection*{Problem 10 — Investment Portfolio Selection}
		
		\textbf{Problem.}
		An investor must choose between two portfolios. Outcomes depend on economic growth
		and interest rates.
		
		\[
		\begin{array}{lcc}
			\hline
			\text{State of nature} & X & Y \\
			\hline
			\text{Strong growth}\,(p),\,\text{Low rates}\,(q) & 26 & 20 \\
			\text{Strong growth}\,(p),\,\text{High rates}\,(1-q) & 12 & 16 \\
			\text{Weak growth}\,(1-p),\,\text{Low rates}\,(q) & 2 & 6 \\
			\text{Weak growth}\,(1-p),\,\text{High rates}\,(1-q) & -10 & -4 \\
			\hline
		\end{array}
		\]
		
		Let \(p\) be the probability of strong growth and \(q\) the probability of low interest rates.  
		When is portfolio X better than portfolio Y.
		
		\textbf{Solution.}
		Compute \(EV_X\):
		\[
		EV_X=26(pq)+12\bigl(p(1-q)\bigr)+2\bigl((1-p)q\bigr)+(-10)\bigl((1-p)(1-q)\bigr)
		\]
		Expand:
		\[
		EV_X=26pq + 12p - 12pq + 2q - 2pq - 10(1-p-q+pq)
		\]
		Expand last term:
		\[
		-10(1-p-q+pq)=-10 + 10p + 10q - 10pq
		\]
		Combine:
		\[
		EV_X=(26pq-12pq-2pq-10pq) + (12p+10p) + (2q+10q) - 10
		\]
		\[
		EV_X=2pq + 22p + 12q - 10
		\]
		
		Compute \(EV_Y\):
		\[
		EV_Y=20(pq)+16\bigl(p(1-q)\bigr)+6\bigl((1-p)q\bigr)+(-4)\bigl((1-p)(1-q)\bigr)
		\]
		Expand:
		\[
		EV_Y=20pq + 16p - 16pq + 6q - 6pq - 4(1-p-q+pq)
		\]
		Expand last term:
		\[
		-4(1-p-q+pq)=-4 + 4p + 4q - 4pq
		\]
		Combine:
		\[
		EV_Y=(20pq-16pq-6pq-4pq) + (16p+4p) + (6q+4q) - 4
		\]
		\[
		EV_Y=-6pq + 20p + 10q - 4
		\]
		
		Portfolio X better than Y when:
		\[
		EV_X-EV_Y>0
		\]
		Substitute:
		\[
		(2pq+22p+12q-10)-(-6pq+20p+10q-4)>0
		\]
		Simplify:
		\[
		2pq+22p+12q-10+6pq-20p-10q+4>0
		\Rightarrow 8pq+2p+2q-6>0
		\]
		
	\end{ExamProblems}
	
\end{document}
