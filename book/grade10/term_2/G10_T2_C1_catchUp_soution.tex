\makeatletter
\def\input@path{{./}{../}{../../}{preamble/}{../preamble/}{../../preamble/}}
\makeatother
% ----------------------------------------------------------
% GENERAL 

% File
\documentclass[11pt]{book}

% Margins
\usepackage[margin=1in]{geometry}

\linespread{1.2}            % Line spacing
\usepackage[utf8]{inputenc}
\usepackage[T1]{fontenc}
\usepackage{lmodern}
\usepackage{microtype}
\setlength{\parindent}{0pt}
\setlength{\parskip}{6pt}
\usepackage{booktabs}

% ----------------------------------------------------------
% TABLES
\usepackage{multicol}
\usepackage{longtable} 
\usepackage{array}
\usepackage{booktabs}
\usepackage{tabularx}
\usepackage{multirow}

% ----------------------------------------------------------
% MATHEMATICS
\usepackage{amsmath}
\usepackage{amssymb}
\usepackage{amsfonts}
\usepackage{mathtools}

% ----------------------------------------------------------
% HIDDEN CONTENT
\usepackage{ifthen}
% Define a boolean switch
\newboolean{explicaciones}
% Set the boolean switch to true or false
% Change to true to show the content

% Explanations
\newcommand{\explicacion}[2]{
	\ifthenelse{\boolean{explicaciones}}{#1}{#2}
}
\newcommand{\mostrarExplicaciones}[1]{\setboolean{explicaciones}{#1}}

% ----------------------------------------------------------
% NUMBERING

\usepackage{fancyhdr}
\pagestyle{empty} % Ensures the entire document has no page numbers

\usepackage{tocloft}
\renewcommand{\cftdot}{} % Remove dots for sections, if any
\renewcommand{\cftsecleader}{\cftdotfill{\cftdotsep}} % Remove dots for sections, if any
\cftpagenumbersoff{section} % Removes page numbers from sections
\cftpagenumbersoff{subsection} % Removes page numbers from subsections

% ----------------------------------------------------------
% IMAGES 

% General settings
\usepackage{graphicx}       % Insert images
\usepackage{float}          % Position images
% \usepackage{subfigure}      % Subfigures
\graphicspath{{imgs}}       % Image location
\usepackage{subcaption}     % Subfigures II
\usepackage{verbatim}

% Figures
\usepackage{tikz}
\usetikzlibrary{arrows.meta,positioning,trees}

% Colors
\usepackage{xcolor}     
\definecolor{popUp}{HTML}{666666}
\definecolor{popUpIn}{HTML}{CED9E0}
\definecolor{backgroundC}{HTML}{D0E8F2}
\definecolor{backgroundCC}{HTML}{FFFFFF}
\definecolor{borders}{HTML}{8c120d}
\definecolor{padding}{HTML}{77D0D7}
\definecolor{links}{HTML}{CC6F5F}

% ----------------------------------------------------------
% FRAMES

% General settings
\usepackage{tcolorbox}
\usepackage{adjustbox}          % Adjusted frame  
\setlength{\fboxrule}{3pt}  % Line width
\setlength{\fboxsep}{3pt}   % Box padding

% General frames
\usepackage{mdframed}   

\mdfdefinestyle{estiloGeneral}{    % General style
	linecolor=black,
	linewidth=1.5pt,
	roundcorner=10pt,
	backgroundcolor=backgroundC,
	innerbottommargin=0pt
}
\mdfdefinestyle{code}{          % Code style
	linecolor=black,
	linewidth=1.5pt,
	roundcorner=10pt,
	backgroundcolor=darkgray!10,
	innerbottommargin=0pt
}

% Image frame
\newtcbox{\fboxC}{
	colback=backgroundC,
	colframe=popUp,
	arc=10pt,
	boxrule=3pt,
	boxsep=0pt, % Change the padding here
	nobeforeafter
}

% ----------------------------------------------------------
% PAGE SETTINGS

% Background 
\newcommand{\background}[0]{\begin{tikzpicture}[remember picture,overlay]
		\fill[backgroundC] (-2,2) rectangle (25cm, -550);
\end{tikzpicture}}

\newcommand{\backgroundC}[0]{\begin{tikzpicture}[remember picture,overlay]
		\fill[backgroundCC] (-2,2) rectangle (25cm, -550);
\end{tikzpicture}}

% Page width 
\newcommand{\anchoPag}[0]{20cm}

% ----------------------------------------------------------
% FONT

% General
\usepackage{tgbonum}        % Font
\usepackage{listings}       % Code typesetting
\usepackage[spanish]{babel} % Load Spanish
\selectlanguage{spanish}    % Select Spanish
\usepackage{enumitem}
\usepackage{bookmark}

\setlist[itemize]{leftmargin=1.2em, itemsep=0.35em, topsep=0.35em}

% --- Table helpers ---
\newcolumntype{L}[1]{>{\raggedright\arraybackslash}p{#1}}
\newcolumntype{Y}{>{\raggedright\arraybackslash}X}
\newcolumntype{C}{>{\centering\arraybackslash}X}
\renewcommand{\arraystretch}{1.1}

% Python style
\lstdefinestyle{python}{
	language=Python,
	basicstyle=\ttfamily\small,
	commentstyle=\color{green!50!black},
	keywordstyle=\color{blue},
	numberstyle=\tiny\color{gray},
	numbers=left,
	morekeywords={>, <},
	breakatwhitespace=false,
	showstringspaces=false,
	showtabs=false,
	showspaces=false
}

% ----------------------------------------------------------
% HYPERLINKS

% General
\usepackage{hyperref}       
\hypersetup{
	colorlinks=true,
	linkcolor=links,
	filecolor=magenta,      
	urlcolor=brown,
}

% Custom commands 

% Large link
\newcommand{\bigLink}[2]{\begin{center} \fboxC{\LARGE{\href{#1}{#2}}}\end{center}}

% Small link
\newcommand{\smallLink}[2]{\begin{center}\fboxC{\href{#1}{#2}}\end{center}}

% Bold link
\newcommand{\bfLink}[2]{\href{#1}{\textbf{#2}}}


% Small URL
\newcommand{\smallUrl}[1]{\begin{center}\fboxC{\url{#1}}\end{center}}


% ----------------------------------------------------------
% CUSTOM COMMANDS FOR FIGURES

\newcommand{\espacioImagenes}[0]{-1.2cm}

% Without frame
\newcommand{\fig}[3][\espacioImagenes]{
	\hspace*{#1}
	\centering
	\includegraphics[width=#2\textwidth]{#3}
}

% With frame
\newcommand{\ffig}[2]{\begin{figure}[h]
		\hspace*{\espacioImagenes}
		\centering
		\fbox{\includegraphics[width=#1\textwidth]{#2}}
\end{figure}}

% Hyperlink with frame
\newcommand{\hfig}[3]{\begin{figure}[h]
		\hspace*{-1.4cm}
		\centering
		\color{popUp}
		\fboxC{\href{#1}{\includegraphics[width=#2\textwidth]{#3}}}
	\end{figure}
}

% Hyperlink without frame
\newcommand{\hffig}[3]{\begin{figure}[h]
		\hspace*{-1.1cm}
		\centering
		\color{popUp}
		\href{#1}{\includegraphics[width=#2\textwidth]{#3}}
	\end{figure}
}

% ----------------------------------------------------------

% Start and Contents
\newcommand{\cuadro}[1]{
	\begin{mdframed}[style=estiloGeneral]
		#1 
	\end{mdframed}
}

% Explanation video image
\newcommand{\linkExplicacion}[1]{
	\hffig{#1}{0.5}{principal/videoExplicacion}
	\vspace{-0.5cm}
}

\newcommand{\subSecLink}[2]{
	\subsubsection*{\href{#1}{\textbf{#2}}}
}

% Spacing
\newcommand{\esp}[0]{\vspace{4mm}}

% Back to start
\newcommand{\secInicio}[0]{\begin{center}\hyperref[sec:inicio]{ 
			\includegraphics[width=0.1\textwidth]{principal/up}
	}\end{center}
}


\geometry{margin=0.85in}
\AtBeginDocument{\small}

\newcommand{\ExamNameField}{\noindent\textbf{Name:}\ \rule{0.7\linewidth}{0.4pt}\par\medskip}

\newcommand{\ExamTitleBlock}[3]{%
	\begin{center}
		\Large\textbf{#1}\\
		\textbf{#2}%
		\if\relax\detokenize{#3}\relax\else\\\textbf{#3}\fi
	\end{center}
	\vspace{0.5em}
}

\newcommand{\ExamSection}[1]{\par\medskip\textbf{#1}\par\smallskip}

\newenvironment{ExamCriteria}{%
	\begin{itemize}[leftmargin=1.6em, itemsep=0.3em, topsep=0.2em]
}{%
	\end{itemize}
}

\newenvironment{ExamProblems}{%
	\begin{enumerate}[label=\textbf{P\arabic*}, leftmargin=0pt, labelsep=0.6em, itemindent=2.2em, itemsep=0.8em]
}{%
	\end{enumerate}
}

\begin{document}
	\ExamTitleBlock{10th grade}{Term 2 C1 Catch-Up Activity (Solutions)}{}
	
	\ExamSection{Problems}
	\begin{ExamProblems}
		\item
		\subsection*{Problem description}
		A student entrepreneur is evaluating a short-term retail opportunity that depends heavily on how customer traffic develops during the weekend. The decision is straightforward in structure but uncertain in outcome, because actual demand can shift quickly based on consumer mood and local conditions. Management therefore needs a probability-based framework to connect each demand scenario with its corresponding financial consequence and to justify the final choice on expected performance rather than intuition.

		\begin{center}
			\textit{Payoff table} \par
			\begin{tabular}{l c c}
				\toprule
				State of nature & Probability & Rent kiosk \\
				\midrule
				High demand & 0.60 & 36 \\
				Low demand & 0.40 & 10 \\
				\bottomrule
			\end{tabular}
		\end{center}

		Which alternative should management choose according to the Expected Value criterion?

		\subsection*{Solution}
		\textbf{Step 1 --- Apply the Expected Value formula.}
		\[
		EV(A_i)=\sum [\text{Payoff}\times\text{Probability}]
		\]
		For the kiosk decision:
		\[
		EV(\text{Rent kiosk})=(36\times0.60)+(10\times0.40)
		\]
		\[
		EV(\text{Rent kiosk})=21.6+4.0=25.6
		\]
		
		\textbf{Step 2 --- Compare expected values.}
		There is only one alternative, so its expected value is $25.6$.
		
		\textbf{Step 3 --- Final decision in business language.}
		Based on the expected value criterion, the entrepreneur should proceed with renting the kiosk because the project yields an expected profit of $25.6$ hundred dollars under the stated demand probabilities.		\item
		\subsection*{Problem description}
		A coffee shop chain must select a bean purchasing approach for the next month while demand remains uncertain. One alternative emphasizes product quality and brand positioning, while the other emphasizes sourcing flexibility and cost control. Because the market can evolve in more than one direction, leadership must evaluate how each plan performs across possible demand states and use expected value reasoning to support a disciplined, forward-looking decision.

		\begin{center}
			\textit{Payoff table} \par
			\begin{tabular}{l c c c}
				\toprule
				State of nature & Probability & Plan A & Plan B \\
				\midrule
				High demand & 0.55 & 84 & 72 \\
				Low demand & 0.45 & 22 & 34 \\
				\bottomrule
			\end{tabular}
		\end{center}

		Which alternative should management choose according to the Expected Value criterion?

		\subsection*{Solution}
		\textbf{Step 1 --- Apply the Expected Value formula.}
		\[
		EV(A_i)=\sum [\text{Payoff}\times\text{Probability}]
		\]
		\[
		EV(\text{Plan A})=(84\times0.55)+(22\times0.45)
		\]
		\[
		EV(\text{Plan A})=46.2+9.9=56.1
		\]
		\[
		EV(\text{Plan B})=(72\times0.55)+(34\times0.45)
		\]
		\[
		EV(\text{Plan B})=39.6+15.3=54.9
		\]
		
		\textbf{Step 2 --- Compare expected values.}
		$EV(\text{Plan A})=56.1$ and $EV(\text{Plan B})=54.9$, so Plan A has the higher expected value.
		
		\textbf{Step 3 --- Final decision in business language.}
		Using the expected value criterion, the chain should adopt Plan A because it provides the highest expected monthly profit under the current demand outlook.		\item
		\subsection*{Problem description}
		A small factory is planning seasonal production and must choose between two operating modes with different cost structures and responsiveness profiles. Market conditions may strengthen, remain stable, or soften, and each state creates a different profit implication for each mode. Since management cannot know in advance which condition will occur, the decision should be anchored in expected value so that uncertainty is incorporated systematically into the production strategy.

		\begin{center}
			\textit{Payoff table} \par
			\begin{tabular}{l c c c}
				\toprule
				State of nature & Probability & Mode A & Mode B \\
				\midrule
				Strong market & 0.30 & 95 & 88 \\
				Stable market & 0.45 & 60 & 66 \\
				Weak market & 0.25 & 18 & 30 \\
				\bottomrule
			\end{tabular}
		\end{center}

		Which alternative should management choose according to the Expected Value criterion?

		\subsection*{Solution}
		\textbf{Step 1 --- Apply the Expected Value formula.}
		\[
		EV(A_i)=\sum [\text{Payoff}\times\text{Probability}]
		\]
		\[
		EV(\text{Mode A})=(95\times0.30)+(60\times0.45)+(18\times0.25)
		\]
		\[
		EV(\text{Mode A})=28.5+27.0+4.5=60.0
		\]
		\[
		EV(\text{Mode B})=(88\times0.30)+(66\times0.45)+(30\times0.25)
		\]
		\[
		EV(\text{Mode B})=26.4+29.7+7.5=63.6
		\]
		
		\textbf{Step 2 --- Compare expected values.}
		$EV(\text{Mode A})=60.0$ and $EV(\text{Mode B})=63.6$, so Mode B is superior on expected value.
		
		\textbf{Step 3 --- Final decision in business language.}
		The factory should select Mode B because it maximizes expected seasonal profit once each market condition is weighted by its probability.		\item
		\subsection*{Problem description}
		A delivery startup is selecting a fleet strategy for the coming quarter in an environment where fuel conditions are uncertain and can materially affect operating margins. Each strategic option offers a different balance between control, flexibility, and exposure to cost volatility. To choose responsibly, the firm must evaluate outcomes across the plausible fuel-cost states and rely on expected value to identify the alternative with the strongest overall economic justification.

		\begin{center}
			\textit{Payoff table} \par
			\begin{tabular}{l c c c c}
				\toprule
				State of nature & Probability & Strategy A & Strategy B & Strategy C \\
				\midrule
				Low fuel cost & 0.25 & 120 & 108 & 96 \\
				Medium fuel cost & 0.50 & 86 & 92 & 88 \\
				High fuel cost & 0.25 & 44 & 58 & 70 \\
				\bottomrule
			\end{tabular}
		\end{center}

		Which alternative should management choose according to the Expected Value criterion?

		\subsection*{Solution}
		\textbf{Step 1 --- Apply the Expected Value formula.}
		\[
		EV(A_i)=\sum [\text{Payoff}\times\text{Probability}]
		\]
		\[
		EV(\text{Strategy A})=(120\times0.25)+(86\times0.50)+(44\times0.25)
		\]
		\[
		EV(\text{Strategy A})=30+43+11=84.0
		\]
		\[
		EV(\text{Strategy B})=(108\times0.25)+(92\times0.50)+(58\times0.25)
		\]
		\[
		EV(\text{Strategy B})=27+46+14.5=87.5
		\]
		\[
		EV(\text{Strategy C})=(96\times0.25)+(88\times0.50)+(70\times0.25)
		\]
		\[
		EV(\text{Strategy C})=24+44+17.5=85.5
		\]
		
		\textbf{Step 2 --- Compare expected values.}
		$EV(\text{A})=84.0$, $EV(\text{B})=87.5$, and $EV(\text{C})=85.5$; Strategy B is the largest.
		
		\textbf{Step 3 --- Final decision in business language.}
		Under the expected value criterion, the startup should choose Strategy B because it delivers the highest expected quarterly profit after accounting for fuel-cost uncertainty.		\item
		\subsection*{Problem description}
		A supermarket is defining an inventory policy for imported fruit while facing uncertainty in supply-chain reliability. Management recognizes that logistics conditions can range from smooth to severely disrupted, and each situation affects availability, waste risk, and revenue potential differently. Because no single scenario is guaranteed, the firm must compare policies using probability-weighted outcomes and select the option that best supports resilient profit performance.

		\begin{center}
			\textit{Payoff table} \par
			\begin{tabular}{l c c c c}
				\toprule
				State of nature & Probability & Policy A & Policy B & Policy C \\
				\midrule
				Very smooth supply & 0.20 & 74 & 82 & 86 \\
				Smooth supply & 0.35 & 68 & 76 & 78 \\
				Disrupted supply & 0.30 & 36 & 50 & 58 \\
				Severely disrupted supply & 0.15 & 8 & 24 & 34 \\
				\bottomrule
			\end{tabular}
		\end{center}

		Which alternative should management choose according to the Expected Value criterion?

		\subsection*{Solution}
		\textbf{Step 1 --- Apply the Expected Value formula.}
		\[
		EV(A_i)=\sum [\text{Payoff}\times\text{Probability}]
		\]
		\[
		EV(\text{Policy A})=(74\times0.20)+(68\times0.35)+(36\times0.30)+(8\times0.15)
		\]
		\[
		EV(\text{Policy A})=14.8+23.8+10.8+1.2=50.6
		\]
		\[
		EV(\text{Policy B})=(82\times0.20)+(76\times0.35)+(50\times0.30)+(24\times0.15)
		\]
		\[
		EV(\text{Policy B})=16.4+26.6+15.0+3.6=61.6
		\]
		\[
		EV(\text{Policy C})=(86\times0.20)+(78\times0.35)+(58\times0.30)+(34\times0.15)
		\]
		\[
		EV(\text{Policy C})=17.2+27.3+17.4+5.1=67.0
		\]
		
		\textbf{Step 2 --- Compare expected values.}
		$EV(\text{Policy A})=50.6$, $EV(\text{Policy B})=61.6$, and $EV(\text{Policy C})=67.0$.
		The highest expected value is Policy C.
		
		\textbf{Step 3 --- Final decision in business language.}
		The supermarket should implement Policy C, since it provides the strongest expected profit performance across the full range of supply-chain outcomes.		\item
		\subsection*{Problem description}
		An electronics retailer is choosing a launch format for a new device in a season where demand intensity may vary significantly. Each format reflects a different channel strategy and operational commitment, leading to distinct financial outcomes under different market conditions. Decision-makers therefore need to assess each format across plausible seasonal states and use expected value analysis to justify the launch approach with the strongest overall return profile.

		\begin{center}
			\textit{Payoff table} \par
			\begin{tabular}{l c c c c c}
				\toprule
				State of nature & Probability & Format A & Format B & Format C & Format D \\
				\midrule
				Boom season & 0.20 & 150 & 132 & 144 & 120 \\
				Good season & 0.30 & 110 & 118 & 124 & 108 \\
				Moderate season & 0.30 & 72 & 82 & 94 & 88 \\
				Slow season & 0.20 & 30 & 48 & 62 & 66 \\
				\bottomrule
			\end{tabular}
		\end{center}

		Which alternative should management choose according to the Expected Value criterion?

		\subsection*{Solution}
		\textbf{Step 1 --- Apply the Expected Value formula.}
		\[
		EV(A_i)=\sum [\text{Payoff}\times\text{Probability}]
		\]
		\[
		EV(\text{Format A})=(150\times0.20)+(110\times0.30)+(72\times0.30)+(30\times0.20)
		\]
		\[
		EV(\text{Format A})=30+33+21.6+6=90.6
		\]
		\[
		EV(\text{Format B})=(132\times0.20)+(118\times0.30)+(82\times0.30)+(48\times0.20)
		\]
		\[
		EV(\text{Format B})=26.4+35.4+24.6+9.6=96.0
		\]
		\[
		EV(\text{Format C})=(144\times0.20)+(124\times0.30)+(94\times0.30)+(62\times0.20)
		\]
		\[
		EV(\text{Format C})=28.8+37.2+28.2+12.4=106.6
		\]
		\[
		EV(\text{Format D})=(120\times0.20)+(108\times0.30)+(88\times0.30)+(66\times0.20)
		\]
		\[
		EV(\text{Format D})=24+32.4+26.4+13.2=96.0
		\]
		
		\textbf{Step 2 --- Compare expected values.}
		Format C has the highest expected value at $106.6$, above Formats A, B, and D.
		
		\textbf{Step 3 --- Final decision in business language.}
		The retailer should launch through Format C because it maximizes expected profit when seasonal demand probabilities are incorporated into the decision.		\item
		\subsection*{Problem description}
		A grain exporter must choose among competing shipping contracts while freight market conditions remain uncertain. Contract design influences both upside potential and downside protection as freight tightness changes across the planning horizon. Since management cannot predict a single freight outcome with certainty, it should evaluate all contracts under the full set of plausible states and apply expected value logic to support a defensible commercial decision.

		\begin{center}
			\textit{Payoff table}
			\begin{tabular}{l c c c c c}
				\toprule
				State of nature & Probability & Contract A & Contract B & Contract C & Contract D \\
				\midrule
				Very favorable freight & 0.15 & 160 & 152 & 146 & 132 \\
				Favorable freight & 0.25 & 138 & 140 & 136 & 126 \\
				Neutral freight & 0.30 & 102 & 114 & 120 & 116 \\
				Tight freight & 0.20 & 66 & 80 & 92 & 98 \\
				Very tight freight & 0.10 & 24 & 40 & 54 & 70 \\
				\bottomrule
			\end{tabular}
		\end{center}

		Which alternative should management choose according to the Expected Value criterion?

		\subsection*{Solution}
		\textbf{Step 1 --- Apply the Expected Value formula.}
		\[
		EV(A_i)=\sum [\text{Payoff}\times\text{Probability}]
		\]
		\[
		EV(\text{Contract A})=(160\times0.15)+(138\times0.25)+(102\times0.30)+(66\times0.20)+(24\times0.10)
		\]
		\[
		EV(\text{Contract A})=24+34.5+30.6+13.2+2.4=104.7
		\]
		\[
		EV(\text{Contract B})=(152\times0.15)+(140\times0.25)+(114\times0.30)+(80\times0.20)+(40\times0.10)
		\]
		\[
		EV(\text{Contract B})=22.8+35+34.2+16+4=112.0
		\]
		\[
		EV(\text{Contract C})=(146\times0.15)+(136\times0.25)+(120\times0.30)+(92\times0.20)+(54\times0.10)
		\]
		\[
		EV(\text{Contract C})=21.9+34+36+18.4+5.4=115.7
		\]
		\[
		EV(\text{Contract D})=(132\times0.15)+(126\times0.25)+(116\times0.30)+(98\times0.20)+(70\times0.10)
		\]
		\[
		EV(\text{Contract D})=19.8+31.5+34.8+19.6+7=112.7
		\]
		
		\textbf{Step 2 --- Compare expected values.}
		Contract C has the largest expected value at $115.7$.
		
		\textbf{Step 3 --- Final decision in business language.}
		The exporter should award the shipping business to Contract C, as this option maximizes expected profit across projected freight market scenarios.		\item
		\subsection*{Problem description}
		A fashion retailer is evaluating expansion plans for urban markets where demand can shift across several intensity levels. Each plan reflects a different growth posture, creating distinct exposure to both strong and weak market outcomes. To align strategy with financial discipline, management should compare alternatives across all plausible demand states and use expected value as the core criterion for selecting the expansion path.

		\begin{center}
			\textit{Payoff table} \par
			\begin{tabular}{l c c c c c c}
				\toprule
				State of nature & Probability & Plan A & Plan B & Plan C & Plan D & Plan E \\
				\midrule
				Very high demand & 0.12 & 190 & 182 & 176 & 168 & 156 \\
				High demand & 0.23 & 150 & 154 & 152 & 146 & 140 \\
				Medium demand & 0.30 & 108 & 120 & 128 & 126 & 124 \\
				Low demand & 0.22 & 58 & 76 & 90 & 98 & 104 \\
				Very low demand & 0.13 & 12 & 34 & 52 & 64 & 76 \\
				\bottomrule
			\end{tabular}
		\end{center}

		Which alternative should management choose according to the Expected Value criterion?

		\subsection*{Solution}
		\textbf{Step 1 --- Apply the Expected Value formula.}
		\[
		EV(A_i)=\sum [\text{Payoff}\times\text{Probability}]
		\]
		\[
		EV(\text{Plan A})=(190\times0.12)+(150\times0.23)+(108\times0.30)+(58\times0.22)+(12\times0.13)
		\]
		\[
		EV(\text{Plan A})=22.8+34.5+32.4+12.76+1.56=104.02
		\]
		\[
		EV(\text{Plan B})=(182\times0.12)+(154\times0.23)+(120\times0.30)+(76\times0.22)+(34\times0.13)
		\]
		\[
		EV(\text{Plan B})=21.84+35.42+36+16.72+4.42=114.4
		\]
		\[
		EV(\text{Plan C})=(176\times0.12)+(152\times0.23)+(128\times0.30)+(90\times0.22)+(52\times0.13)
		\]
		\[
		EV(\text{Plan C})=21.12+34.96+38.4+19.8+6.76=121.04
		\]
		\[
		EV(\text{Plan D})=(168\times0.12)+(146\times0.23)+(126\times0.30)+(98\times0.22)+(64\times0.13)
		\]
		\[
		EV(\text{Plan D})=20.16+33.58+37.8+21.56+8.32=121.42
		\]
		\[
		EV(\text{Plan E})=(156\times0.12)+(140\times0.23)+(124\times0.30)+(104\times0.22)+(76\times0.13)
		\]
		\[
		EV(\text{Plan E})=18.72+32.2+37.2+22.88+9.88=120.88
		\]
		
		\textbf{Step 2 --- Compare expected values.}
		Plan D has the highest expected value at $121.42$, slightly above Plans C and E.
		
		\textbf{Step 3 --- Final decision in business language.}
		The retailer should execute Plan D, because it offers the best expected profit once demand uncertainty is translated into probability-weighted outcomes.		\item
		\subsection*{Problem description}
		A regional energy distributor is selecting a pricing package in a context where weather-driven demand can change materially during the operating period. Each package offers a different trade-off between high-demand capture and low-demand protection. Because weather patterns are uncertain and financially significant, management needs a probability-weighted evaluation of outcomes to determine which package provides the most robust expected economic result.

		\begin{center}
			\textit{Payoff table}
			\begin{tabular}{l c c c c c c}
				\toprule
				State of nature & Probability & Package A & Package B & Package C & Package D & Package E \\
				\midrule
				Extreme cold & 0.10 & 220 & 210 & 204 & 196 & 186 \\
				Cold & 0.20 & 184 & 186 & 182 & 176 & 170 \\
				Mild & 0.25 & 142 & 152 & 158 & 156 & 152 \\
				Warm & 0.20 & 96 & 110 & 122 & 128 & 130 \\
				Hot & 0.15 & 58 & 74 & 88 & 98 & 106 \\
				Extreme hot & 0.10 & 22 & 40 & 54 & 68 & 82 \\
				\bottomrule
			\end{tabular}
		\end{center}

		Which alternative should management choose according to the Expected Value criterion?

		\subsection*{Solution}
		\textbf{Step 1 --- Apply the Expected Value formula.}
		\[
		EV(A_i)=\sum [\text{Payoff}\times\text{Probability}]
		\]
		\[
		EV(\text{Package A})=(220\times0.10)+(184\times0.20)+(142\times0.25)+(96\times0.20)+(58\times0.15)+(22\times0.10)
		\]
		\[
		EV(\text{Package A})=22+36.8+35.5+19.2+8.7+2.2=124.4
		\]
		\[
		EV(\text{Package B})=(210\times0.10)+(186\times0.20)+(152\times0.25)+(110\times0.20)+(74\times0.15)+(40\times0.10)
		\]
		\[
		EV(\text{Package B})=21+37.2+38+22+11.1+4=133.3
		\]
		\[
		EV(\text{Package C})=(204\times0.10)+(182\times0.20)+(158\times0.25)+(122\times0.20)+(88\times0.15)+(54\times0.10)
		\]
		\[
		EV(\text{Package C})=20.4+36.4+39.5+24.4+13.2+5.4=139.3
		\]
		\[
		EV(\text{Package D})=(196\times0.10)+(176\times0.20)+(156\times0.25)+(128\times0.20)+(98\times0.15)+(68\times0.10)
		\]
		\[
		EV(\text{Package D})=19.6+35.2+39+25.6+14.7+6.8=140.9
		\]
		\[
		EV(\text{Package E})=(186\times0.10)+(170\times0.20)+(152\times0.25)+(130\times0.20)+(106\times0.15)+(82\times0.10)
		\]
		\[
		EV(\text{Package E})=18.6+34+38+26+15.9+8.2=140.7
		\]
		
		\textbf{Step 2 --- Compare expected values.}
		Package D has the highest expected value ($140.9$), narrowly above Package E.
		
		\textbf{Step 3 --- Final decision in business language.}
		The distributor should choose Package D because it generates the strongest expected return under the probability distribution of weather-driven demand conditions.		\item
		\subsection*{Problem description}
		A national logistics group is deciding on a network design while macro-economic conditions may evolve through expansion, slowdown, contraction, and recovery patterns. Each design presents a different balance between growth capacity and resilience under weaker environments. Given this broad uncertainty set, executives should compare all designs across the plausible states of nature and rely on expected value analysis to choose the configuration with the strongest expected performance.

		\begin{center}
			\textit{Payoff table}
			\begin{tabular}{l c c c c c c c}
				\toprule
				State of nature & Probability & Design A & Design B & Design C & Design D & Design E & Design F \\
				\midrule
				Rapid expansion & 0.14 & 260 & 248 & 242 & 234 & 226 & 214 \\
				Steady growth & 0.20 & 220 & 222 & 218 & 212 & 206 & 198 \\
				Flat market & 0.24 & 170 & 182 & 188 & 190 & 188 & 184 \\
				Mild contraction & 0.16 & 118 & 132 & 146 & 156 & 162 & 164 \\
				Strong contraction & 0.12 & 70 & 86 & 102 & 118 & 130 & 142 \\
				Recovery transition & 0.14 & 136 & 148 & 158 & 164 & 168 & 170 \\
				\bottomrule
			\end{tabular}
		\end{center}

		Which alternative should management choose according to the Expected Value criterion?

		\subsection*{Solution}
		\textbf{Step 1 --- Apply the Expected Value formula.}
		\[
		EV(A_i)=\sum [\text{Payoff}\times\text{Probability}]
		\]
		\[
		EV(\text{Design A})=(260\times0.14)+(220\times0.20)+(170\times0.24)+(118\times0.16)+(70\times0.12)+(136\times0.14)
		\]
		\[
		EV(\text{Design A})=36.4+44+40.8+18.88+8.4+19.04=167.52
		\]
		\[
		EV(\text{Design B})=(248\times0.14)+(222\times0.20)+(182\times0.24)+(132\times0.16)+(86\times0.12)+(148\times0.14)
		\]
		\[
		EV(\text{Design B})=34.72+44.4+43.68+21.12+10.32+20.72=174.96
		\]
		\[
		EV(\text{Design C})=(242\times0.14)+(218\times0.20)+(188\times0.24)+(146\times0.16)+(102\times0.12)+(158\times0.14)
		\]
		\[
		EV(\text{Design C})=33.88+43.6+45.12+23.36+12.24+22.12=180.32
		\]
		\[
		EV(\text{Design D})=(234\times0.14)+(212\times0.20)+(190\times0.24)+(156\times0.16)+(118\times0.12)+(164\times0.14)
		\]
		\[
		EV(\text{Design D})=32.76+42.4+45.6+24.96+14.16+22.96=182.84
		\]
		\[
		EV(\text{Design E})=(226\times0.14)+(206\times0.20)+(188\times0.24)+(162\times0.16)+(130\times0.12)+(168\times0.14)
		\]
		\[
		EV(\text{Design E})=31.64+41.2+45.12+25.92+15.6+23.52=183.0
		\]
		\[
		EV(\text{Design F})=(214\times0.14)+(198\times0.20)+(184\times0.24)+(164\times0.16)+(142\times0.12)+(170\times0.14)
		\]
		\[
		EV(\text{Design F})=29.96+39.6+44.16+26.24+17.04+23.8=180.8
		\]
		
		\textbf{Step 2 --- Compare expected values.}
		Among the six designs, Design E has the largest expected value at $183.0$.
		
		\textbf{Step 3 --- Final decision in business language.}
		Using the expected value criterion, management should implement Design E because it provides the highest expected profit across the projected macro-economic distribution.
		\end{ExamProblems}
\end{document}
