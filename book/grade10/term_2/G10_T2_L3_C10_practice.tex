\makeatletter
\def\input@path{{./}{../}{../../}{preamble/}{../preamble/}{../../preamble/}}
\makeatother
% ----------------------------------------------------------
% GENERAL 

% File
\documentclass[11pt]{book}

% Margins
\usepackage[margin=1in]{geometry}

\linespread{1.2}            % Line spacing
\usepackage[utf8]{inputenc}
\usepackage[T1]{fontenc}
\usepackage{lmodern}
\usepackage{microtype}
\setlength{\parindent}{0pt}
\setlength{\parskip}{6pt}
\usepackage{booktabs}

% ----------------------------------------------------------
% TABLES
\usepackage{multicol}
\usepackage{longtable} 
\usepackage{array}
\usepackage{booktabs}
\usepackage{tabularx}
\usepackage{multirow}

% ----------------------------------------------------------
% MATHEMATICS
\usepackage{amsmath}
\usepackage{amssymb}
\usepackage{amsfonts}
\usepackage{mathtools}

% ----------------------------------------------------------
% HIDDEN CONTENT
\usepackage{ifthen}
% Define a boolean switch
\newboolean{explicaciones}
% Set the boolean switch to true or false
% Change to true to show the content

% Explanations
\newcommand{\explicacion}[2]{
	\ifthenelse{\boolean{explicaciones}}{#1}{#2}
}
\newcommand{\mostrarExplicaciones}[1]{\setboolean{explicaciones}{#1}}

% ----------------------------------------------------------
% NUMBERING

\usepackage{fancyhdr}
\pagestyle{empty} % Ensures the entire document has no page numbers

\usepackage{tocloft}
\renewcommand{\cftdot}{} % Remove dots for sections, if any
\renewcommand{\cftsecleader}{\cftdotfill{\cftdotsep}} % Remove dots for sections, if any
\cftpagenumbersoff{section} % Removes page numbers from sections
\cftpagenumbersoff{subsection} % Removes page numbers from subsections

% ----------------------------------------------------------
% IMAGES 

% General settings
\usepackage{graphicx}       % Insert images
\usepackage{float}          % Position images
% \usepackage{subfigure}      % Subfigures
\graphicspath{{imgs}}       % Image location
\usepackage{subcaption}     % Subfigures II
\usepackage{verbatim}

% Figures
\usepackage{tikz}
\usetikzlibrary{arrows.meta,positioning,trees}

% Colors
\usepackage{xcolor}     
\definecolor{popUp}{HTML}{666666}
\definecolor{popUpIn}{HTML}{CED9E0}
\definecolor{backgroundC}{HTML}{D0E8F2}
\definecolor{backgroundCC}{HTML}{FFFFFF}
\definecolor{borders}{HTML}{8c120d}
\definecolor{padding}{HTML}{77D0D7}
\definecolor{links}{HTML}{CC6F5F}

% ----------------------------------------------------------
% FRAMES

% General settings
\usepackage{tcolorbox}
\usepackage{adjustbox}          % Adjusted frame  
\setlength{\fboxrule}{3pt}  % Line width
\setlength{\fboxsep}{3pt}   % Box padding

% General frames
\usepackage{mdframed}   

\mdfdefinestyle{estiloGeneral}{    % General style
	linecolor=black,
	linewidth=1.5pt,
	roundcorner=10pt,
	backgroundcolor=backgroundC,
	innerbottommargin=0pt
}
\mdfdefinestyle{code}{          % Code style
	linecolor=black,
	linewidth=1.5pt,
	roundcorner=10pt,
	backgroundcolor=darkgray!10,
	innerbottommargin=0pt
}

% Image frame
\newtcbox{\fboxC}{
	colback=backgroundC,
	colframe=popUp,
	arc=10pt,
	boxrule=3pt,
	boxsep=0pt, % Change the padding here
	nobeforeafter
}

% ----------------------------------------------------------
% PAGE SETTINGS

% Background 
\newcommand{\background}[0]{\begin{tikzpicture}[remember picture,overlay]
		\fill[backgroundC] (-2,2) rectangle (25cm, -550);
\end{tikzpicture}}

\newcommand{\backgroundC}[0]{\begin{tikzpicture}[remember picture,overlay]
		\fill[backgroundCC] (-2,2) rectangle (25cm, -550);
\end{tikzpicture}}

% Page width 
\newcommand{\anchoPag}[0]{20cm}

% ----------------------------------------------------------
% FONT

% General
\usepackage{tgbonum}        % Font
\usepackage{listings}       % Code typesetting
\usepackage[spanish]{babel} % Load Spanish
\selectlanguage{spanish}    % Select Spanish
\usepackage{enumitem}
\usepackage{bookmark}

\setlist[itemize]{leftmargin=1.2em, itemsep=0.35em, topsep=0.35em}

% --- Table helpers ---
\newcolumntype{L}[1]{>{\raggedright\arraybackslash}p{#1}}
\newcolumntype{Y}{>{\raggedright\arraybackslash}X}
\newcolumntype{C}{>{\centering\arraybackslash}X}
\renewcommand{\arraystretch}{1.1}

% Python style
\lstdefinestyle{python}{
	language=Python,
	basicstyle=\ttfamily\small,
	commentstyle=\color{green!50!black},
	keywordstyle=\color{blue},
	numberstyle=\tiny\color{gray},
	numbers=left,
	morekeywords={>, <},
	breakatwhitespace=false,
	showstringspaces=false,
	showtabs=false,
	showspaces=false
}

% ----------------------------------------------------------
% HYPERLINKS

% General
\usepackage{hyperref}       
\hypersetup{
	colorlinks=true,
	linkcolor=links,
	filecolor=magenta,      
	urlcolor=brown,
}

% Custom commands 

% Large link
\newcommand{\bigLink}[2]{\begin{center} \fboxC{\LARGE{\href{#1}{#2}}}\end{center}}

% Small link
\newcommand{\smallLink}[2]{\begin{center}\fboxC{\href{#1}{#2}}\end{center}}

% Bold link
\newcommand{\bfLink}[2]{\href{#1}{\textbf{#2}}}


% Small URL
\newcommand{\smallUrl}[1]{\begin{center}\fboxC{\url{#1}}\end{center}}


% ----------------------------------------------------------
% CUSTOM COMMANDS FOR FIGURES

\newcommand{\espacioImagenes}[0]{-1.2cm}

% Without frame
\newcommand{\fig}[3][\espacioImagenes]{
	\hspace*{#1}
	\centering
	\includegraphics[width=#2\textwidth]{#3}
}

% With frame
\newcommand{\ffig}[2]{\begin{figure}[h]
		\hspace*{\espacioImagenes}
		\centering
		\fbox{\includegraphics[width=#1\textwidth]{#2}}
\end{figure}}

% Hyperlink with frame
\newcommand{\hfig}[3]{\begin{figure}[h]
		\hspace*{-1.4cm}
		\centering
		\color{popUp}
		\fboxC{\href{#1}{\includegraphics[width=#2\textwidth]{#3}}}
	\end{figure}
}

% Hyperlink without frame
\newcommand{\hffig}[3]{\begin{figure}[h]
		\hspace*{-1.1cm}
		\centering
		\color{popUp}
		\href{#1}{\includegraphics[width=#2\textwidth]{#3}}
	\end{figure}
}

% ----------------------------------------------------------

% Start and Contents
\newcommand{\cuadro}[1]{
	\begin{mdframed}[style=estiloGeneral]
		#1 
	\end{mdframed}
}

% Explanation video image
\newcommand{\linkExplicacion}[1]{
	\hffig{#1}{0.5}{principal/videoExplicacion}
	\vspace{-0.5cm}
}

\newcommand{\subSecLink}[2]{
	\subsubsection*{\href{#1}{\textbf{#2}}}
}

% Spacing
\newcommand{\esp}[0]{\vspace{4mm}}

% Back to start
\newcommand{\secInicio}[0]{\begin{center}\hyperref[sec:inicio]{ 
			\includegraphics[width=0.1\textwidth]{principal/up}
	}\end{center}
}


\geometry{margin=0.85in}
\AtBeginDocument{\small}

\newcommand{\ExamNameField}{\noindent\textbf{Name:}\ \rule{0.7\linewidth}{0.4pt}\par\medskip}

\newcommand{\ExamTitleBlock}[3]{%
	\begin{center}
		\Large\textbf{#1}\\
		\textbf{#2}%
		\if\relax\detokenize{#3}\relax\else\\\textbf{#3}\fi
	\end{center}
	\vspace{0.5em}
}

\newcommand{\ExamSection}[1]{\par\medskip\textbf{#1}\par\smallskip}

\newenvironment{ExamCriteria}{%
	\begin{itemize}[leftmargin=1.6em, itemsep=0.3em, topsep=0.2em]
}{%
	\end{itemize}
}

\newenvironment{ExamProblems}{%
	\begin{enumerate}[label=\textbf{P\arabic*}, leftmargin=0pt, labelsep=0.6em, itemindent=2.2em, itemsep=0.8em]
}{%
	\end{enumerate}
}

\begin{document}
	\ExamTitleBlock{10th grade}{Term 2 -- Lesson 3 MATERIAL: Decision Analysis with Bayes, Maximin, and Minimax Regret (Worked Solutions)}{}
	
	\ExamSection{C10: Full end-to-end decision analysis}
	\begin{ExamProblems}
		\item
		\subsection*{Problem description}
		A food-processing company must choose one packaging strategy for next quarter:
		$P$ (premium line), $Q$ (standard line), or $R$ (hybrid line).
		Demand can be high ($0.60$) or low ($0.40$).
		Operating cost has two states: stable input costs ($0.65$) and costly input spike ($0.35$).
		Revenues (thousand USD):
		\begin{itemize}
			\item High demand: $P=500,\ Q=540,\ R=520$.
			\item Low demand: $P=320,\ Q=300,\ R=310$.
		\end{itemize}
		Costs (thousand USD):
		\begin{itemize}
			\item $P$: stable $190$, spike $255$.
			\item $Q$: stable $220$, spike $290$.
			\item $R$: stable $205$, spike $270$.
		\end{itemize}
		The stable-cost values are lower because packaging film contracts lock in most input prices, while the spike state captures imported resin and freight increases.
		Build the payoff table, apply EV, Maximin, and Minimax Regret, produce ordered rankings, then use a staged decision tree to choose a final strategy.

		\subsection*{C10}
		Step 1. Build the payoff table from narrative data.
		Joint state probabilities:
		\[
		\Pr(H,S)=0.60\cdot 0.65=0.39,\quad \Pr(H,C)=0.60\cdot 0.35=0.21,\quad \Pr(L,S)=0.40\cdot 0.65=0.26,\quad \Pr(L,C)=0.40\cdot 0.35=0.14.
		\]
		Payoff = revenue $-$ cost.
		\begin{center}
			\begin{tabular}{l c c c c}
				\toprule
				State of nature & Probability & $P$ & $Q$ & $R$ \\
				\midrule
				High demand, stable cost & 0.39 & $500-190=310$ & $540-220=320$ & $520-205=315$ \\
				High demand, costly spike & 0.21 & $500-255=245$ & $540-290=250$ & $520-270=250$ \\
				Low demand, stable cost & 0.26 & $320-190=130$ & $300-220=80$ & $310-205=105$ \\
				Low demand, costly spike & 0.14 & $320-255=65$ & $300-290=10$ & $310-270=40$ \\
				\bottomrule
			\end{tabular}
		\end{center}

		Step 2. Bayes criterion (EV).
		\[
		EV(P)=208.95,\quad EV(Q)=199.00,\quad EV(R)=201.95.
		\]
		EV ranking: $P>R>Q$.

		Step 3. Maximin criterion.
		\[
		\min(P)=65,\quad \min(Q)=10,\quad \min(R)=40.
		\]
		Maximin ranking: $P>R>Q$.

		Step 4. Minimax Regret criterion.
		State best payoffs: $\{320,250,130,65\}$.
		\begin{center}
			\begin{tabular}{l c c c c c}
				\toprule
				Alternative & $HS$ & $HC$ & $LS$ & $LC$ & Max regret \\
				\midrule
				$P$ & 10 & 5 & 0 & 0 & 10 \\
				$Q$ & 0 & 0 & 50 & 55 & 55 \\
				$R$ & 5 & 0 & 25 & 25 & 25 \\
				\bottomrule
			\end{tabular}
		\end{center}
		Minimax Regret ranking (lower is better): $P<R<Q$.

		Step 5. Staged decision tree and final choice.
		Filtering rule: keep best 2 EV options, then choose smaller Minimax Regret.
		\begin{center}
			\begin{tabular}{l}
				\toprule
				Decision tree \\
				\midrule
				$D_0$: $\{P,Q,R\}$ \\
				$\rightarrow D_1$ (EV filter): keep $\{P,R\}$ \\
				$\rightarrow D_2$ (Minimax Regret): compare $P(10)$ vs $R(25)$ \\
				Leaves: $P$, $R$ \\
				Final leaf selected: $P$ \\
				\bottomrule
			\end{tabular}
		\end{center}
		Final explanation: $P$ stays strongest across expected value, worst-case floor, and regret, so the staged process ends at $P$.

		\newpage
		\item
		\subsection*{Problem description}
		An online retailer must choose one pricing strategy among five options $S_1,S_2,S_3,S_4,S_5$.
		Demand states are optimistic ($0.45$), regular ($0.35$), and weak ($0.20$).
		Payoffs (thousand USD):
		\begin{center}
			\begin{tabular}{l c c c c c c}
				\toprule
				State & Probability & $S_1$ & $S_2$ & $S_3$ & $S_4$ & $S_5$ \\
				\midrule
				Optimistic & 0.45 & 150 & 140 & 170 & 130 & 160 \\
				Regular & 0.35 & 90 & 100 & 80 & 110 & 95 \\
				Weak & 0.20 & 20 & 45 & -10 & 60 & 35 \\
				\bottomrule
			\end{tabular}
		\end{center}
		Apply EV, Maximin, and Minimax Regret step by step, generate full ordered rankings, and then apply this staged rule:
		keep best 4 EV options $\rightarrow$ keep best 3 Maximin options $\rightarrow$ choose smallest Minimax Regret.

		\subsection*{C10}
		Step 1. EV calculations.
		\[
		\begin{aligned}
		EV(S_1)&=103.00,\\
		EV(S_2)&=107.00,\\
		EV(S_3)&=102.50,\\
		EV(S_4)&=109.00,\\
		EV(S_5)&=112.25.
		\end{aligned}
		\]
		EV ranking:
		\[
		S_5\,(112.25)>S_4\,(109.00)>S_2\,(107.00)>S_1\,(103.00)>S_3\,(102.50).
		\]

		Step 2. Maximin calculations.
		\[
		\min(S_1)=20,\ \min(S_2)=45,\ \min(S_3)=-10,\ \min(S_4)=60,\ \min(S_5)=35.
		\]
		Maximin ranking:
		\[
		S_4\,(60)>S_2\,(45)>S_5\,(35)>S_1\,(20)>S_3\,(-10).
		\]

		Step 3. Minimax Regret calculations.
		Best state payoffs:
		\[
		\max(\text{Optimistic})=170,\quad \max(\text{Regular})=110,\quad \max(\text{Weak})=60.
		\]
		Regret table:
		\begin{center}
			\begin{tabular}{l c c c c}
				\toprule
				Alternative & Optimistic & Regular & Weak & Max regret \\
				\midrule
				$S_1$ & 20 & 20 & 40 & 40 \\
				$S_2$ & 30 & 10 & 15 & 30 \\
				$S_3$ & 0 & 30 & 70 & 70 \\
				$S_4$ & 40 & 0 & 0 & 40 \\
				$S_5$ & 10 & 15 & 25 & 25 \\
				\bottomrule
			\end{tabular}
		\end{center}
		Minimax Regret ranking (lower is better):
		\[
		S_5\,(25)>S_2\,(30)>S_1\,(40)=S_4\,(40)>S_3\,(70).
		\]

		Step 4. Decision tree for the staged rule.
		\begin{itemize}
			\item EV top 4: $\{S_5,S_4,S_2,S_1\}$.
			\item On this subset, Maximin top 3: $S_4(60),S_2(45),S_5(35)$, so keep $\{S_4,S_2,S_5\}$.
			\item On this subset, Minimax Regret values are $S_4(40),S_2(30),S_5(25)$.
		\end{itemize}
		\begin{center}
			\begin{tabular}{l}
				\toprule
				Decision tree \\
				\midrule
				$D_0$: $\{S_1,S_2,S_3,S_4,S_5\}$ \\
				$\rightarrow D_1$ (EV filter): $\{S_5,S_4,S_2,S_1\}$ \\
				$\rightarrow D_2$ (Maximin filter): $\{S_4,S_2,S_5\}$ \\
				$\rightarrow D_3$ (Minimax Regret choice): compare $S_4(40),S_2(30),S_5(25)$ \\
				Leaves: $S_4$, $S_2$, $S_5$ \\
				Final leaf selected: $S_5$ \\
				\bottomrule
			\end{tabular}
		\end{center}

		Step 5. Final conclusion.
		The final decision is $S_5$ because it leads EV and has the smallest maximum regret among the finalists.
	\end{ExamProblems}
\end{document}
