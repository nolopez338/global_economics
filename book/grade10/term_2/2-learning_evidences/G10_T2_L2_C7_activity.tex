\documentclass[12pt]{article}

% Page size and tighter margins
\usepackage[a4paper,left=1.2cm,right=1.2cm,top=1.5cm,bottom=1.5cm]{geometry}

% Core packages
\usepackage{graphicx}
\usepackage{xcolor}
\usepackage{array}
\usepackage{tabularx}
\usepackage{multicol}
\usepackage{amsmath}
\usepackage[T1]{fontenc}
\usepackage[utf8]{inputenc}

% Tighter itemize spacing
\usepackage{enumitem}
\setlist[itemize]{noitemsep, topsep=0pt, parsep=0pt, partopsep=0pt}

\setlength{\parindent}{0pt}
\setlength{\tabcolsep}{6pt}
\renewcommand{\arraystretch}{1.15}

% Column types
\newcolumntype{Y}{>{\raggedright\arraybackslash}m{\dimexpr0.55\textwidth-2\tabcolsep-2\arrayrulewidth\relax}}
\newcolumntype{Z}{>{\raggedright\arraybackslash}m{\dimexpr0.45\textwidth-2\tabcolsep-2\arrayrulewidth\relax}}
\newcolumntype{C}[1]{>{\centering\arraybackslash}m{#1}}

% Gray subsection header box
\newcommand{\SubsectionBox}[1]{%
	\noindent\colorbox{gray!30}{%
		\parbox{\linewidth}{\textbf{#1}}%
	}\par\vspace{0.35cm}%
}

% Centered multi-line cell helper
\newcommand{\CellCenter}[1]{%
	\parbox{\linewidth}{\centering #1}%
}

\begin{document}
	
	% =========================
	% HEADER BOX
	% =========================
	\noindent
	\begin{tabularx}{\textwidth}{|C{2.8cm}|C{\dimexpr\textwidth-6cm-4\tabcolsep-4\arrayrulewidth\relax}|C{2.8cm}|}
		\hline
		\raisebox{-1cm}{\includegraphics[width=2cm]{../../preamble/logo.png}}
		&
		\CellCenter{%
			\textbf{}\par
			\textbf{GLOBAL ECONOMICS}\par
			\textbf{GRADE: 10TH}\par
			\textbf{LEARNING EVIDENCE T2 L2 C7 ACTIVITY}\par
			\textbf{ANALYSIS OF DECISIONS}\par
			\textbf{TEACHER'S NAME: Nicolás López Cuéllar}
		}
		&
		\CellCenter{%
			\vspace{5mm}
			\textbf{SECOND TERM}\par
			\textbf{2025--2026}
		}
		\\
		\hline
	\end{tabularx}
	
	\vspace{0.5cm}
	
	% =========================
	% OBJECTIVE + CRITERIA
	% =========================
	\noindent
	\begin{tabular}{|Y|Z|}
		\hline
		{\small
			\textbf{Learning objective:} Analyze decision-making problems using probabilities, compute and compare expected values, and select decisions based on the expected value criterion.
		}
		&
		{\footnotesize
			\textbf{Assessment criteria:}\par
			C7: Uses probabilities and expected value to analyze a decision-making problem.
		}
		\\
		\hline
	\end{tabular}
	
	\vspace{0.4cm}
	
	\begin{multicols}{2}
		
		\SubsectionBox{Criteria assessment}\vspace{-0.25cm}
		Criterion C7 is assessed in every problem. It is considered passed when it is correctly activated in at least seven of the ten problems.
		\vspace{0.25cm}
		\SubsectionBox{1. Single Project under Uncertainty}\vspace{-0.25cm}
		\textbf{Problem.}
		A renewable energy firm is deciding whether to build a small solar farm. The decision is whether to build or not build, and management uses expected value to decide because they expect similar projects to be repeated over time. The payoffs represent net profits in millions of dollars.
		
		\[
		\begin{array}{lcc}
			\hline
			& S_1\,(p) & S_2\,(1-p) \\
			\hline
			\text{Build the farm} & 24 & -6 \\
			\hline
		\end{array}
		\]
		
		Use expected value as the decision rule and determine for which values of \(p\) building the farm is profitable.
		
		\vspace{0.25cm}
		\SubsectionBox{2. Technology Upgrade}\vspace{-0.25cm}
		\textbf{Problem.}
		A delivery company is considering upgrading its routing software. The choice is to upgrade or not, and expected value is appropriate because the company plans to use the same software for many delivery cycles. Payoffs are net savings in millions of dollars.
		
		\[
		\begin{array}{lcc}
			\hline
			& S_1\,(p) & S_2\,(1-p) \\
			\hline
			\text{Upgrade} & 32 & -14 \\
			\hline
		\end{array}
		\]
		
		Using expected value, determine for which values of \(p\) the upgrade is profitable.
		
		\vspace{0.25cm}
		\SubsectionBox{3. Two Investment Alternatives}\vspace{-0.25cm}
		\textbf{Problem.}
		An entrepreneur must choose between two investment alternatives, A and B. Expected value is used because the entrepreneur wants the option with the higher long-run average return. Payoffs are net profits in millions of dollars.
		
		\[
		\begin{array}{lcc}
			\hline
			& S_1\,(p) & S_2\,(1-p) \\
			\hline
			A & 28 & 4 \\
			B & 18 & 12 \\
			\hline
		\end{array}
		\]
		
		Using expected value, determine for which values of \(p\) option A yields a higher expected value than option B.
		
		\vspace{0.25cm}
		\SubsectionBox{4. Three Production Plans}\vspace{-0.25cm}
		\textbf{Problem.}
		A manufacturing firm must choose among three production plans (A, B, C) for the next quarter. The firm uses expected value because it seeks the plan with the highest average profit across many similar quarters. Payoffs are net profits in millions of dollars.
		
		\[
		\begin{array}{lcc}
			\hline
			& S_1\,(p) & S_2\,(1-p) \\
			\hline
			A & 40 & -6 \\
			B & 26 & 6 \\
			C & 18 & 12 \\
			\hline
		\end{array}
		\]
		
		Using expected value, determine for which values of \(p\) each plan is optimal.
		
		\vspace{0.25cm}
		\SubsectionBox{5. Two Alternatives with Three States}\vspace{-0.25cm}
		\textbf{Problem.}
		A shipping company must choose between two routing strategies, A and B. Expected value is the decision criterion because the company wants the route with the higher average profit over many shipments. Payoffs are net profits in millions of dollars.
		
		\[
		\begin{array}{lccc}
			\hline
			& S_1\,(p_1) & S_2\,(p_2) & S_3\,(1-p_1-p_2) \\
			\hline
			A & 24 & 8 & -9 \\
			B & 16 & 14 & 2 \\
			\hline
		\end{array}
		\]
		
		Using expected value, determine when strategy A yields a higher expected value than strategy B.
		
		\vspace{0.25cm}
		\SubsectionBox{6. Policy Choice under Three States}\vspace{-0.25cm}
		\textbf{Problem.}
		A local government compares two flood-prevention policies, A and B. Expected value is used because the city plans to choose the policy that yields the highest average net benefit over many years. Payoffs are net benefits in millions of dollars.
		
		\[
		\begin{array}{lccc}
			\hline
			& S_1\,(p_1) & S_2\,(p_2) & S_3\,(1-p_1-p_2) \\
			\hline
			A & 20 & 6 & -8 \\
			B & 12 & 10 & 4 \\
			\hline
		\end{array}
		\]
		
		Using expected value, determine when policy A yields a higher expected value than policy B.
		
		\vspace{0.25cm}
		\SubsectionBox{7. New Service Platform Launch}\vspace{-0.25cm}
		\textbf{Problem.}
		A software company must choose among three launch plans (A, B, C) for a new service platform. Expected value is used because the company plans to repeat similar launches and wants the highest average profit across many comparable decisions. Payoffs are net profits in millions of dollars.
		
		\[
		\begin{array}{lcc}
			\hline
			& S_1\,(p) & S_2\,(1-p) \\
			\hline
			A & 30 & 4 \\
			B & 22 & 10 \\
			C & 16 & 14 \\
			\hline
		\end{array}
		\]
		
		Using expected value, determine for which values of \(p\) each launch plan is optimal.
		
		\vspace{0.25cm}
		\SubsectionBox{8. Regional Store Opening}\vspace{-0.25cm}
		\textbf{Problem.}
		A retailer evaluates opening a regional store. Expected value is appropriate because the retailer wants the highest average profit across multiple comparable openings. Payoffs are net profits in millions of dollars.
		
		\[
		\begin{array}{lcc}
			\hline
			& \text{Mild}\,(q) & \text{Aggressive}\,(1-q) \\
			\hline
			\text{Strong}\,(p) & 40 & 14 \\
			\text{Weak}\,(1-p) & -2 & -18 \\
			\hline
		\end{array}
		\]
		
		Using expected value, determine when opening the store is profitable.
		
		\vspace{0.25cm}
		\SubsectionBox{9. Marketing Strategy Choice}\vspace{-0.25cm}
		\textbf{Problem.}
		A firm must choose between two marketing strategies, A and B. Expected value is used because the firm wants the strategy with the highest average profit across many campaigns. Payoffs are net profits in millions of dollars.
		
		\[
		\begin{array}{lcc}
			\hline
			\text{State of nature} & A & B \\
			\hline
			\text{High interest}\,(p),\,\text{Effective}\,(q) & 34 & 26 \\
			\text{High interest}\,(p),\,\text{Ineffective}\,(1-q) & 12 & 16 \\
			\text{Low interest}\,(1-p),\,\text{Effective}\,(q) & -8 & 2 \\
			\text{Low interest}\,(1-p),\,\text{Ineffective}\,(1-q) & -16 & -6 \\
			\hline
		\end{array}
		\]
		
		Using expected value, determine when strategy A yields a higher expected value than strategy B.
		
		\vspace{0.25cm}
		\SubsectionBox{10. Investment Portfolio Selection}\vspace{-0.25cm}
		\textbf{Problem.}
		An investor must choose between two portfolios, X and Y. Expected value is used because the investor wants the option with the higher long-run average return. Payoffs are net returns in millions of dollars.
		
		\[
		\begin{array}{lcc}
			\hline
			\text{State of nature} & X & Y \\
			\hline
			\text{Strong growth}\,(p),\,\text{Low rates}\,(q) & 30 & 24 \\
			\text{Strong growth}\,(p),\,\text{High rates}\,(1-q) & 14 & 18 \\
			\text{Weak growth}\,(1-p),\,\text{Low rates}\,(q) & 4 & 8 \\
			\text{Weak growth}\,(1-p),\,\text{High rates}\,(1-q) & -12 & -6 \\
			\hline
		\end{array}
		\]
		
		Using expected value, determine when portfolio X yields a higher expected value than portfolio Y.
		
	\end{multicols}
	
\end{document}
