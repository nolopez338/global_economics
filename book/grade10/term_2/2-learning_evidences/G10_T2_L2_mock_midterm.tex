\documentclass[12pt]{article}

% Page size and tighter margins
\usepackage[a4paper,left=1.2cm,right=1.2cm,top=1.5cm,bottom=1.5cm]{geometry}

% Core packages
\usepackage{graphicx}
\usepackage{xcolor}
\usepackage{array}
\usepackage{tabularx}
\usepackage{multicol}
\usepackage{amsmath}
\usepackage[T1]{fontenc}
\usepackage[utf8]{inputenc}

% Tighter itemize spacing
\usepackage{enumitem}
\setlist[itemize]{noitemsep, topsep=0pt, parsep=0pt, partopsep=0pt}

\setlength{\parindent}{0pt}
\setlength{\tabcolsep}{6pt}
\renewcommand{\arraystretch}{1.15}

% Column types
\newcolumntype{Y}{>{\raggedright\arraybackslash}m{\dimexpr0.25\textwidth-2\tabcolsep-2\arrayrulewidth\relax}}
\newcolumntype{Z}{>{\raggedright\arraybackslash}m{\dimexpr0.75\textwidth-2\tabcolsep-2\arrayrulewidth\relax}}
\newcolumntype{C}[1]{>{\centering\arraybackslash}m{#1}}

% Gray subsection header box
\newcommand{\SubsectionBox}[1]{%
	\noindent\colorbox{gray!30}{%
		\parbox{\linewidth}{\textbf{#1}}%
	}\par\vspace{0.35cm}%
}

% Centered multi-line cell helper
\newcommand{\CellCenter}[1]{%
	\parbox{\linewidth}{\centering #1}%
}

\begin{document}
	
	% =========================
	% HEADER BOX
	% =========================
	\noindent
	\begin{tabularx}{\textwidth}{|C{2.8cm}|C{\dimexpr\textwidth-6cm-4\tabcolsep-4\arrayrulewidth\relax}|C{2.8cm}|}
		\hline
		\raisebox{-1cm}{\includegraphics[width=2cm]{../../preamble/logo.png}}
		&
		\CellCenter{%
			\textbf{}\par
			\textbf{GLOBAL ECONOMICS}\par
			\textbf{GRADE: 10TH}\par
			\textbf{SECOND TERM MIDTERM MOCK}\par
			\textbf{ANALYSIS OF DECISIONS PROBLEMS}\par
			\textbf{TEACHER'S NAME: Nicolás López Cuéllar}
		}
		&
		\CellCenter{%
			\vspace{5mm}
			\textbf{SECOND TERM}\par
			\textbf{2025--2026}
		}
		\\
		\hline
	\end{tabularx}
	
	\vspace{0.5cm}
	
	% =========================
	% OBJECTIVE + CRITERIA
	% =========================
	\noindent
	\begin{tabular}{|Y|Z|}
		\hline
		{\small
			\textbf{Learning objective:} Analyze decision-making problems under uncertainty by constructing payoff structures and applying decision criteria, including expected value.
		}
		&
		{\footnotesize
			\textbf{Assessment criteria:}\par
			C1: Describes the way a problem can be formulated for optimal decision making. Evaluates expected value. \par
			C2: Interprets decision alternatives, events, consequences, and states of nature.\par
			C3: Builds a payoff table from a description of the problem.\par
			C4: Explains the maximax criterion for decision making without probabilities.\par
			C5: Summarizes the maximin criterion for decision making without probabilities. \par 
			C6: Develops decision-making strategies using probabilities and the maximum opportunity criterion.\par
			C7: Uses probabilities and expected value to analyze a decision-making problem.\par
		}
		\\
		\hline
	\end{tabular}
	
	\vspace{0.4cm}
	
	\begin{multicols}{2}
		
		\SubsectionBox{Criteria assessment}\vspace{-0.25cm}
		To pass a criterion, Criteria C1–C7 must each be correctly applied in at least two problems.
		
		\vspace{0.25cm}
		\SubsectionBox{(C2,C3) 1. Problem description}\vspace{-0.25cm}
		A bike-sharing cooperative must choose a station layout for the next quarter: Layout A (dense stations),
		Layout B (hub stations), or Layout C (hybrid stations). Trip revenue depends on ridership demand,
		which can be high or low. The probability of high demand is $0.60$ and low demand is $0.40$.
		Maintenance costs depend on parts prices, which can be low, medium, or high with probabilities
		$0.50$, $0.30$, and $0.20$. Revenue is in thousands of dollars. If demand is high, revenue is 520 for Layout A,
		560 for Layout B, and 545 for Layout C. If demand is low, revenue is 300 for Layout A, 320 for Layout B,
		and 315 for Layout C. Maintenance costs are 190 (low), 230 (medium), and 275 (high) for Layout A;
		220 (low), 255 (medium), and 300 (high) for Layout B; and 205 (low), 245 (medium), and 285 (high) for Layout C.
		Construct the payoff table.
		
		\vspace{0.25cm}
		\SubsectionBox{(C2,C3) 2. Logistics Network Decision Under Demand and Cost Uncertainty}\vspace{-0.25cm}
		A logistics firm must choose one network model for the next year: Model A (in-house hub),
		Model B (partner network), or Model C (hybrid cross-dock). Demand uncertainty is high demand (0.55)
		or low demand (0.45), and delivery-cost uncertainty is low (0.50) or high (0.50).
		Under high demand, Model A completes 960 shipments, Model B completes 900 shipments, and Model C completes 940 shipments.
		Under low demand, Model A completes 580 shipments, Model B completes 620 shipments, and Model C completes 600 shipments.
		Model A earns \$46 per shipment, Model B earns \$43 in high demand and \$41 in low demand,
		and Model C earns \$44 per shipment. Cost exposure reflects annual operating costs:
		Model A has \$20{,}000 (low cost) and \$26{,}000 (high cost),
		Model B has \$17{,}200 (low cost) and \$22{,}400 (high cost), and
		Model C has \$18{,}600 (low cost) and \$24{,}000 (high cost).
		Construct the payoff table.
		
		\vspace{0.25cm}
		\SubsectionBox{(C1,C4,C5,C6) 3. Holiday Inventory Plan Selection Under Expanded Demand Uncertainty}\vspace{-0.25cm}
		A retail chain must select one of five inventory plans for the holiday period: Plan A (aggressive stock),
		Plan B (balanced stock), Plan C (data-driven mix), Plan D (conservative stock), or Plan E (minimal stock).
		Demand can be strong, moderate, or weak with probabilities $0.30$, $0.45$, and $0.25$.
		The final payoff table (in thousands of dollars) is given below. Use Maximax, Maximin, Minimax Regret,
		and Expected Value to select a plan.
		
		\begin{center}
			\textit{Payoff table} \\
			\begin{tabular}{l c c c c c c}
				\hline
				Demand & Prob. & A & B & C & D & E \\
				\hline
				Strong & 0.30 & 120 & 110 & 98 & 88 & 76 \\
				Moderate & 0.45 & 72 & 78 & 82 & 74 & 68 \\
				Weak & 0.25 & 20 & 34 & 46 & 52 & 58 \\
				\hline
			\end{tabular}
		\end{center}
		
		\vspace{0.25cm}
		\SubsectionBox{(C1,C4,C5,C6) 4. Routing System Choice Under Fuel and Congestion Uncertainty}\vspace{-0.25cm}
		A shipping company must choose between four routing systems for the next season: System A, System B,
		System C, or System D. Four states of nature summarize fuel and congestion conditions
		with probabilities $0.15$, $0.35$, $0.25$, and $0.25$. The final payoff table (in thousands of dollars)
		is given below. Use Maximax, Maximin, Minimax Regret, and Expected Value to select a system.
		
		\begin{center}
			\textit{Payoff table} \\
			\begin{tabular}{l c c c c c}
				\hline
				State of nature & Prob. & A & B & C & D \\
				\hline
				State $S_1$ & 0.15 & 55 & 50 & 48 & 44 \\
				State $S_2$ & 0.35 & 34 & 36 & 32 & 30 \\
				State $S_3$ & 0.25 & 18 & 22 & 24 & 20 \\
				State $S_4$ & 0.25 & -10 & 4 & 8 & 12 \\
				\hline
			\end{tabular}
		\end{center}
		
		\vspace{0.25cm}
		\SubsectionBox{(C7) 5. Expected Value Comparison of Production Plans with Three States}\vspace{-0.25cm}
		A manufacturer must select one production plan, labeled A, B, or C, before knowing which market condition will occur.
		There are three possible states of nature: state $S_1$, which occurs with probability $\frac{p}{2}$,
		state $S_2$, which occurs with probability $\frac{p}{2}$, and state $S_3$, which occurs with probability $1-p$.
		Each production plan generates a different profit depending on the realized state of nature, as summarized in the payoff table below.
		The objective is to determine which production plan maximizes expected profit as a function of $p$.
		
		\begin{center}
			\textit{Payoff table} \\
			\begin{tabular}{l c c c}
				\hline
				& $S_1$ $\left(\frac{p}{2}\right)$ & $S_2$ $\left(\frac{p}{2}\right)$ & $S_3$ $(1-p)$ \\
				\hline
				A & 44 & 24 & 6 \\
				B & 30 & 30 & 14 \\
				C & 18 & 20 & 22 \\
				\hline
			\end{tabular}
		\end{center}
		
		\vspace{0.25cm}
		\SubsectionBox{(C7) 6. Expected Value Comparison of Zoning Options with Three Alternatives}\vspace{-0.25cm}
		A city council must choose among three zoning options, labeled A, B, and C, before knowing which future condition will occur.
		There are two possible states of nature: state $S_1$, which occurs with probability $p$, and state $S_2$, which occurs with probability $1-p$.
		Each zoning option generates a different net return depending on the realized state of nature, as shown in the payoff table below.
		The objective is to determine which zoning option yields the higher expected return as a function of $p$.
		
		\begin{center}
			\textit{Payoff table} \\
			\begin{tabular}{l c c}
				\hline
				& $S_1$ $(p)$ & $S_2$ $(1-p)$ \\
				\hline
				A & 34 & 8 \\
				B & 24 & 14 \\
				C & 18 & 18 \\
				\hline
			\end{tabular}
		\end{center}
		
	\end{multicols}
	
\end{document}
