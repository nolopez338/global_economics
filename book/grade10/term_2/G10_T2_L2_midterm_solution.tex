\makeatletter
\def\input@path{{./}{../}{../../}{preamble/}{../preamble/}{../../preamble/}}
\makeatother
% ----------------------------------------------------------
% GENERAL 

% File
\documentclass[11pt]{book}

% Margins
\usepackage[margin=1in]{geometry}

\linespread{1.2}            % Line spacing
\usepackage[utf8]{inputenc}
\usepackage[T1]{fontenc}
\usepackage{lmodern}
\usepackage{microtype}
\setlength{\parindent}{0pt}
\setlength{\parskip}{6pt}
\usepackage{booktabs}

% ----------------------------------------------------------
% TABLES
\usepackage{multicol}
\usepackage{longtable} 
\usepackage{array}
\usepackage{booktabs}
\usepackage{tabularx}
\usepackage{multirow}

% ----------------------------------------------------------
% MATHEMATICS
\usepackage{amsmath}
\usepackage{amssymb}
\usepackage{amsfonts}
\usepackage{mathtools}

% ----------------------------------------------------------
% HIDDEN CONTENT
\usepackage{ifthen}
% Define a boolean switch
\newboolean{explicaciones}
% Set the boolean switch to true or false
% Change to true to show the content

% Explanations
\newcommand{\explicacion}[2]{
	\ifthenelse{\boolean{explicaciones}}{#1}{#2}
}
\newcommand{\mostrarExplicaciones}[1]{\setboolean{explicaciones}{#1}}

% ----------------------------------------------------------
% NUMBERING

\usepackage{fancyhdr}
\pagestyle{empty} % Ensures the entire document has no page numbers

\usepackage{tocloft}
\renewcommand{\cftdot}{} % Remove dots for sections, if any
\renewcommand{\cftsecleader}{\cftdotfill{\cftdotsep}} % Remove dots for sections, if any
\cftpagenumbersoff{section} % Removes page numbers from sections
\cftpagenumbersoff{subsection} % Removes page numbers from subsections

% ----------------------------------------------------------
% IMAGES 

% General settings
\usepackage{graphicx}       % Insert images
\usepackage{float}          % Position images
% \usepackage{subfigure}      % Subfigures
\graphicspath{{imgs}}       % Image location
\usepackage{subcaption}     % Subfigures II
\usepackage{verbatim}

% Figures
\usepackage{tikz}
\usetikzlibrary{arrows.meta,positioning,trees}

% Colors
\usepackage{xcolor}     
\definecolor{popUp}{HTML}{666666}
\definecolor{popUpIn}{HTML}{CED9E0}
\definecolor{backgroundC}{HTML}{D0E8F2}
\definecolor{backgroundCC}{HTML}{FFFFFF}
\definecolor{borders}{HTML}{8c120d}
\definecolor{padding}{HTML}{77D0D7}
\definecolor{links}{HTML}{CC6F5F}

% ----------------------------------------------------------
% FRAMES

% General settings
\usepackage{tcolorbox}
\usepackage{adjustbox}          % Adjusted frame  
\setlength{\fboxrule}{3pt}  % Line width
\setlength{\fboxsep}{3pt}   % Box padding

% General frames
\usepackage{mdframed}   

\mdfdefinestyle{estiloGeneral}{    % General style
	linecolor=black,
	linewidth=1.5pt,
	roundcorner=10pt,
	backgroundcolor=backgroundC,
	innerbottommargin=0pt
}
\mdfdefinestyle{code}{          % Code style
	linecolor=black,
	linewidth=1.5pt,
	roundcorner=10pt,
	backgroundcolor=darkgray!10,
	innerbottommargin=0pt
}

% Image frame
\newtcbox{\fboxC}{
	colback=backgroundC,
	colframe=popUp,
	arc=10pt,
	boxrule=3pt,
	boxsep=0pt, % Change the padding here
	nobeforeafter
}

% ----------------------------------------------------------
% PAGE SETTINGS

% Background 
\newcommand{\background}[0]{\begin{tikzpicture}[remember picture,overlay]
		\fill[backgroundC] (-2,2) rectangle (25cm, -550);
\end{tikzpicture}}

\newcommand{\backgroundC}[0]{\begin{tikzpicture}[remember picture,overlay]
		\fill[backgroundCC] (-2,2) rectangle (25cm, -550);
\end{tikzpicture}}

% Page width 
\newcommand{\anchoPag}[0]{20cm}

% ----------------------------------------------------------
% FONT

% General
\usepackage{tgbonum}        % Font
\usepackage{listings}       % Code typesetting
\usepackage[spanish]{babel} % Load Spanish
\selectlanguage{spanish}    % Select Spanish
\usepackage{enumitem}
\usepackage{bookmark}

\setlist[itemize]{leftmargin=1.2em, itemsep=0.35em, topsep=0.35em}

% --- Table helpers ---
\newcolumntype{L}[1]{>{\raggedright\arraybackslash}p{#1}}
\newcolumntype{Y}{>{\raggedright\arraybackslash}X}
\newcolumntype{C}{>{\centering\arraybackslash}X}
\renewcommand{\arraystretch}{1.1}

% Python style
\lstdefinestyle{python}{
	language=Python,
	basicstyle=\ttfamily\small,
	commentstyle=\color{green!50!black},
	keywordstyle=\color{blue},
	numberstyle=\tiny\color{gray},
	numbers=left,
	morekeywords={>, <},
	breakatwhitespace=false,
	showstringspaces=false,
	showtabs=false,
	showspaces=false
}

% ----------------------------------------------------------
% HYPERLINKS

% General
\usepackage{hyperref}       
\hypersetup{
	colorlinks=true,
	linkcolor=links,
	filecolor=magenta,      
	urlcolor=brown,
}

% Custom commands 

% Large link
\newcommand{\bigLink}[2]{\begin{center} \fboxC{\LARGE{\href{#1}{#2}}}\end{center}}

% Small link
\newcommand{\smallLink}[2]{\begin{center}\fboxC{\href{#1}{#2}}\end{center}}

% Bold link
\newcommand{\bfLink}[2]{\href{#1}{\textbf{#2}}}


% Small URL
\newcommand{\smallUrl}[1]{\begin{center}\fboxC{\url{#1}}\end{center}}


% ----------------------------------------------------------
% CUSTOM COMMANDS FOR FIGURES

\newcommand{\espacioImagenes}[0]{-1.2cm}

% Without frame
\newcommand{\fig}[3][\espacioImagenes]{
	\hspace*{#1}
	\centering
	\includegraphics[width=#2\textwidth]{#3}
}

% With frame
\newcommand{\ffig}[2]{\begin{figure}[h]
		\hspace*{\espacioImagenes}
		\centering
		\fbox{\includegraphics[width=#1\textwidth]{#2}}
\end{figure}}

% Hyperlink with frame
\newcommand{\hfig}[3]{\begin{figure}[h]
		\hspace*{-1.4cm}
		\centering
		\color{popUp}
		\fboxC{\href{#1}{\includegraphics[width=#2\textwidth]{#3}}}
	\end{figure}
}

% Hyperlink without frame
\newcommand{\hffig}[3]{\begin{figure}[h]
		\hspace*{-1.1cm}
		\centering
		\color{popUp}
		\href{#1}{\includegraphics[width=#2\textwidth]{#3}}
	\end{figure}
}

% ----------------------------------------------------------

% Start and Contents
\newcommand{\cuadro}[1]{
	\begin{mdframed}[style=estiloGeneral]
		#1 
	\end{mdframed}
}

% Explanation video image
\newcommand{\linkExplicacion}[1]{
	\hffig{#1}{0.5}{principal/videoExplicacion}
	\vspace{-0.5cm}
}

\newcommand{\subSecLink}[2]{
	\subsubsection*{\href{#1}{\textbf{#2}}}
}

% Spacing
\newcommand{\esp}[0]{\vspace{4mm}}

% Back to start
\newcommand{\secInicio}[0]{\begin{center}\hyperref[sec:inicio]{ 
			\includegraphics[width=0.1\textwidth]{principal/up}
	}\end{center}
}


\geometry{margin=0.85in}
\AtBeginDocument{\small}

\newcommand{\ExamNameField}{\noindent\textbf{Name:}\ \rule{0.7\linewidth}{0.4pt}\par\medskip}

\newcommand{\ExamTitleBlock}[3]{%
	\begin{center}
		\Large\textbf{#1}\\
		\textbf{#2}%
		\if\relax\detokenize{#3}\relax\else\\\textbf{#3}\fi
	\end{center}
	\vspace{0.5em}
}

\newcommand{\ExamSection}[1]{\par\medskip\textbf{#1}\par\smallskip}

\newenvironment{ExamCriteria}{%
	\begin{itemize}[leftmargin=1.6em, itemsep=0.3em, topsep=0.2em]
}{%
	\end{itemize}
}

\newenvironment{ExamProblems}{%
	\begin{enumerate}[label=\textbf{P\arabic*}, leftmargin=0pt, labelsep=0.6em, itemindent=2.2em, itemsep=0.8em]
}{%
	\end{enumerate}
}

\begin{document}
	\ExamTitleBlock{10th grade}{Midterm Analysis of Decisions (Solutions)}{}
	
	\ExamSection{Problems}
	\begin{ExamProblems}
		\item
		\subsection*{Bakery Product Bundle Selection Under Multiple Demand Levels}
		A neighborhood bakery must select one of three seasonal bundles for the next month: Bundle A (artisan focus),
		Bundle B (balanced assortment), or Bundle C (value pack). Demand can be very high, high, medium, or low.
		Based on prior months, the probability of very high demand is $0.20$, high demand is $0.35$, medium demand is $0.30$,
		and low demand is $0.15$. Profits are measured in hundreds of dollars. If demand is very high, Bundle A yields 110,
		Bundle B yields 95, and Bundle C yields 80. If demand is high, Bundle A yields 88, Bundle B yields 76, and Bundle C yields 60.
		If demand is medium, Bundle A yields 52, Bundle B yields 58, and Bundle C yields 46. If demand is low, Bundle A yields $-6$,
		Bundle B yields 24, and Bundle C yields 32. Construct the payoff table.
		
		\subsection*{C2}
		\begin{center}
			\begin{tabular}{l p{0.74\linewidth}}
				\toprule
				Decision Alternatives & Bundle A (artisan), Bundle B (balanced), Bundle C (value pack) \\
				States of Nature & Demand very high, high, medium, low with probabilities $0.20$, $0.35$, $0.30$, $0.15$ \\
				Events & Realized demand level during the month \\
				Consequences & Profit in hundreds of dollars after all costs \\
				\bottomrule
			\end{tabular}
		\end{center}
		\textbf{Population parameters}
		
		Treating each bundle's four payoff outcomes as the full population:
		\[
		\mu=\frac{1}{n}\sum_{i=1}^{n}x_i,\qquad
		\sigma^2=\frac{1}{n}\sum_{i=1}^{n}(x_i-\mu)^2,\qquad
		\sigma=\sqrt{\sigma^2}.
		\]
		\[
		\begin{aligned}
		\text{Bundle A: }&\mu=61,\ \sigma^2=1925,\ \sigma\approx 43.87,\\
		\text{Bundle B: }&\mu=63.25,\ \sigma^2=684.6875,\ \sigma\approx 26.17,\\
		\text{Bundle C: }&\mu=54.5,\ \sigma^2=314.75,\ \sigma\approx 17.74.
		\end{aligned}
		\]
		
		\textbf{Sample statistics}
		
		Using the same payoff lists as a sample of size $n=4$:
		\[
		\bar{x}=\frac{1}{n}\sum_{i=1}^{n}x_i,\qquad
		s^2=\frac{1}{n-1}\sum_{i=1}^{n}(x_i-\bar{x})^2,\qquad
		s=\sqrt{s^2}.
		\]
		\[
		\begin{aligned}
		\text{Bundle A: }&\bar{x}=61,\ s^2\approx 2566.67,\ s\approx 50.66,\\
		\text{Bundle B: }&\bar{x}=63.25,\ s^2\approx 912.92,\ s\approx 30.21,\\
		\text{Bundle C: }&\bar{x}=54.5,\ s^2\approx 419.67,\ s\approx 20.49.
		\end{aligned}
		\]
		
		\subsection*{C3}
		Payoff = revenue $-$ cost. In this problem, the profits for each bundle are already provided and directly represent the corresponding payoffs.
		\begin{center}
			\textit{Payoff table} \\
			\begin{tabular}{l c c c c}
				\toprule
				State of nature & Probability & Bundle A & Bundle B & Bundle C \\
				\midrule
				Very high demand & 0.20 & 110 & 95 & 80 \\
				High demand & 0.35 & 88 & 76 & 60 \\
				Medium demand & 0.30 & 52 & 58 & 46 \\
				Low demand & 0.15 & -6 & 24 & 32 \\
				\bottomrule
			\end{tabular}
		\end{center}
		
		\item
		\subsection*{Problem description}
		A bike-sharing cooperative must choose a station layout for the next quarter: Layout A (dense stations)
		or Layout B (hub stations). Trip revenue depends on ridership demand, which can be high or low.
		The probability of high demand is $0.65$ and low demand is $0.35$. Maintenance costs depend on parts prices,
		which can be low, medium, or high with probabilities $0.40$, $0.35$, and $0.25$. Revenue is in thousands of dollars.
		If demand is high, revenue is 500 for Layout A and 540 for Layout B. If demand is low, revenue is 280 for Layout A
		and 300 for Layout B. Maintenance costs are 180 (low), 220 (medium), and 260 (high) for Layout A, and 210 (low),
		250 (medium), and 290 (high) for Layout B. Construct the payoff table.
		
		\subsection*{C2}
		\begin{center}
			\begin{tabular}{l p{0.74\linewidth}}
				\toprule
				Decision Alternatives & Layout A (dense), Layout B (hub) \\
				States of Nature & Demand high, low with parts prices low, medium, or high \\
				Events & Realized demand level paired with parts price conditions in the quarter \\
				Consequences & Profit in thousands of dollars from revenue minus maintenance costs \\
				\bottomrule
			\end{tabular}
		\end{center}
		\textbf{Population parameters}
		
		From the six payoff outcomes in the full state space for each layout:
		\[
		\mu=\frac{1}{n}\sum_{i=1}^{n}x_i,\qquad
		\sigma^2=\frac{1}{n}\sum_{i=1}^{n}(x_i-\mu)^2,\qquad
		\sigma=\sqrt{\sigma^2}.
		\]
		\[
		\begin{aligned}
		\text{Layout A: }&\mu=170,\ \sigma^2\approx 13166.67,\ \sigma\approx 114.75,\\
		\text{Layout B: }&\mu=170,\ \sigma^2\approx 15466.67,\ \sigma\approx 124.37.
		\end{aligned}
		\]
		
		\textbf{Sample statistics}
		
		Using the same six payoffs as a sample $(n=6)$:
		\[
		\bar{x}=\frac{1}{n}\sum_{i=1}^{n}x_i,\qquad
		s^2=\frac{1}{n-1}\sum_{i=1}^{n}(x_i-\bar{x})^2,\qquad
		s=\sqrt{s^2}.
		\]
		\[
		\begin{aligned}
		\text{Layout A: }&\bar{x}=170,\ s^2=15800,\ s\approx 125.70,\\
		\text{Layout B: }&\bar{x}=170,\ s^2=18560,\ s\approx 136.24.
		\end{aligned}
		\]
		
		\subsection*{C3}
		Payoff = revenue $-$ cost, where revenue is determined by demand and cost is determined by parts prices.
		\begin{center}
			\begin{minipage}[t]{0.48\linewidth}
				\textit{Revenue parameters by layout}
				\begin{center}
					\begin{tabular}{l c c}
						\toprule
						Layout & High & Low \\
						\midrule
						Probabilities & 0.65 & 0.35 \\
						Layout A & 500 & 280 \\
						Layout B & 540 & 300 \\
						\bottomrule
					\end{tabular}
				\end{center}
			\end{minipage}
			\hfill
			\begin{minipage}[t]{0.48\linewidth}
				\textit{Cost parameters by layout}
				\begin{center}
					\begin{tabular}{l c c c}
						\toprule
						Layout & Low & Medium & High \\
						\midrule
						Probabilities & 0.40 & 0.35 & 0.25 \\
						Layout A & 180 & 220 & 260 \\
						Layout B & 210 & 250 & 290 \\
						\bottomrule
					\end{tabular}
				\end{center}
			\end{minipage}
		\end{center}
		Profit is computed as Revenue minus Cost for each alternative and state.
		
		\begin{center}
			\textit{Profit parameters by layout}
			\begin{tabular}{l p{0.17\linewidth} p{0.2\linewidth} p{0.2\linewidth}}
				\toprule
				State of nature & Probability & Layout A & Layout B \\
				\midrule
				High demand, low cost &
				$\begin{array}{l}
					0.65 \cdot 0.40\\
					= 0.26
				\end{array}$ &
				$\begin{array}{l}
					500 - 180\\
					= 320
				\end{array}$ &
				$\begin{array}{l}
					540 - 210\\
					= 330
				\end{array}$ \\
				High demand, medium cost &
				$\begin{array}{l}
					0.65 \cdot 0.35\\
					= 0.2275
				\end{array}$ &
				$\begin{array}{l}
					500 - 220\\
					= 280
				\end{array}$ &
				$\begin{array}{l}
					540 - 250\\
					= 290
				\end{array}$ \\
				High demand, high cost &
				$\begin{array}{l}
					0.65 \cdot 0.25\\
					= 0.1625
				\end{array}$ &
				$\begin{array}{l}
					500 - 260\\
					= 240
				\end{array}$ &
				$\begin{array}{l}
					540 - 290\\
					= 250
				\end{array}$ \\
				Low demand, low cost &
				$\begin{array}{l}
					0.35 \cdot 0.40\\
					= 0.14
				\end{array}$ &
				$\begin{array}{l}
					280 - 180\\
					= 100
				\end{array}$ &
				$\begin{array}{l}
					300 - 210\\
					= 90
				\end{array}$ \\
				Low demand, medium cost &
				$\begin{array}{l}
					0.35 \cdot 0.35\\
					= 0.1225
				\end{array}$ &
				$\begin{array}{l}
					280 - 220\\
					= 60
				\end{array}$ &
				$\begin{array}{l}
					300 - 250\\
					= 50
				\end{array}$ \\
				Low demand, high cost &
				$\begin{array}{l}
					0.35 \cdot 0.25\\
					= 0.0875
				\end{array}$ &
				$\begin{array}{l}
					280 - 260\\
					= 20
				\end{array}$ &
				$\begin{array}{l}
					300 - 290\\
					= 10
				\end{array}$ \\
				\bottomrule
			\end{tabular}
		\end{center}
		\begin{center}
			\textit{Payoff table} \\
			\begin{tabular}{l c c c}
				\toprule
				State of nature & Probability & Layout A & Layout B \\
				\midrule
				High demand, low cost & 0.26 & 320 & 330 \\
				High demand, medium cost & 0.2275 & 280 & 290 \\
				High demand, high cost & 0.1625 & 240 & 250 \\
				Low demand, low cost & 0.14 & 100 & 90 \\
				Low demand, medium cost & 0.1225 & 60 & 50 \\
				Low demand, high cost & 0.0875 & 20 & 10 \\
				\bottomrule
			\end{tabular}
		\end{center}
		
		\item
		\subsection*{Shipping Hub Decision Under Demand and Cost Uncertainty}
		A package shipping firm must choose either operating an in-house hub or outsourcing to regional partners for the next year.
		Demand uncertainty is high demand (0.60) or low demand (0.40), and delivery-cost uncertainty is low (0.55) or high (0.45).
		Under high demand the in-house hub completes 900 shipments while partners complete 850; under low demand the in-house hub completes
		520 shipments while partners complete 560. The in-house hub earns \$45 per shipment, while partners earn \$42 per shipment in high demand
		and \$40 per shipment in low demand. Cost exposure reflects annual operating costs: the in-house hub has \$18{,}000 in the low-cost state
		and \$24{,}000 in the high-cost state, while partners incur \$16{,}500 in the low-cost state and \$21{,}000 in the high-cost state.
		Construct the payoff table.
		
		\subsection*{C2}
		\begin{center}
			\begin{tabular}{l p{0.74\linewidth}}
				\toprule
				Decision Alternatives & In-house hub; Regional partners \\
				States of Nature & Demand high, low with delivery costs low or high \\
				Events & Order volume paired with delivery cost conditions after the plan is chosen \\
				Consequences & Profit (revenue $-$ costs) for each plan and state \\
				Probabilities & Revenue: 0.60, 0.40; Cost: 0.55, 0.45 \\
				\bottomrule
			\end{tabular}
		\end{center}
		\textbf{Population parameters}
		
		Using each alternative's four payoff outcomes across all states of nature as the population:
		\[
		\mu=\frac{1}{n}\sum_{i=1}^{n}x_i,\qquad
		\sigma^2=\frac{1}{n}\sum_{i=1}^{n}(x_i-\mu)^2,\qquad
		\sigma=\sqrt{\sigma^2}.
		\]
		\[
		\begin{aligned}
		\text{In-house: }&\mu=10{,}950,\ \sigma^2=82{,}102{,}500,\ \sigma\approx 9{,}061.04,\\
		\text{Partner: }&\mu=10{,}300,\ \sigma^2=49{,}285{,}000,\ \sigma\approx 7{,}020.33.
		\end{aligned}
		\]
		
		\textbf{Sample statistics}
		
		Treating those same four payoffs as a sample $(n=4)$:
		\[
		\bar{x}=\frac{1}{n}\sum_{i=1}^{n}x_i,\qquad
		s^2=\frac{1}{n-1}\sum_{i=1}^{n}(x_i-\bar{x})^2,\qquad
		s=\sqrt{s^2}.
		\]
		\[
		\begin{aligned}
		\text{In-house: }&\bar{x}=10{,}950,\ s^2=109{,}470{,}000,\ s\approx 10{,}462.79,\\
		\text{Partner: }&\bar{x}=10{,}300,\ s^2\approx 65{,}713{,}333.33,\ s\approx 8{,}106.38.
		\end{aligned}
		\]
		
		\subsection*{C3}
		Revenue by demand state (quantity $\times$ unit price = total revenue):
		\begin{center}
			\begin{tabular}{l c l l l}
				\toprule
				State & Probability & Alternative & Quantity & Unit price (USD) \\
				\midrule
				High demand & 0.60 & In-house & 900 shipments & \$45 \\
				High demand & 0.60 & Partner & 850 shipments & \$42 \\
				Low demand & 0.40 & In-house & 520 shipments & \$45 \\
				Low demand & 0.40 & Partner & 560 shipments & \$40 \\
				\bottomrule
			\end{tabular}
		\end{center}
		\begin{center}
			\begin{tabular}{l c l c}
				\toprule
				State & Probability & Alternative & Total revenue (USD) \\
				\midrule
				High demand & 0.60 & In-house & $900 \times 45 = 40{,}500$ \\
				High demand & 0.60 & Partner & $850 \times 42 = 35{,}700$ \\
				Low demand & 0.40 & In-house & $520 \times 45 = 23{,}400$ \\
				Low demand & 0.40 & Partner & $560 \times 40 = 22{,}400$ \\
				\bottomrule
			\end{tabular}
		\end{center}
		
		Cost by delivery-cost state (USD):
		\begin{center}
			\begin{tabular}{l c c c}
				\toprule
				State & Probability & In-house & Partner \\
				\midrule
				Low cost & 0.55 & 18{,}000 & 16{,}500 \\
				High cost & 0.45 & 24{,}000 & 21{,}000 \\
				\bottomrule
			\end{tabular}
		\end{center}
		
		Profit table (profit = revenue $-$ costs):
		\begin{center}
			\begin{tabular}{l p{0.17\linewidth} p{0.23\linewidth} p{0.23\linewidth}}
				\toprule
				State of nature & Probability & In-house & Partner \\
				\midrule
				High demand, low cost &
				$\begin{array}{l}
					0.60 \times 0.55\\
					= 0.33
				\end{array}$ &
				$\begin{array}{l}
					40{,}500 - 18{,}000\\
					= 22{,}500
				\end{array}$ &
				$\begin{array}{l}
					35{,}700 - 16{,}500\\
					= 19{,}200
				\end{array}$ \\
				High demand, high cost &
				$\begin{array}{l}
					0.60 \times 0.45\\
					= 0.27
				\end{array}$ &
				$\begin{array}{l}
					40{,}500 - 24{,}000\\
					= 16{,}500
				\end{array}$ &
				$\begin{array}{l}
					35{,}700 - 21{,}000\\
					= 14{,}700
				\end{array}$ \\
				Low demand, low cost &
				$\begin{array}{l}
					0.40 \times 0.55\\
					= 0.22
				\end{array}$ &
				$\begin{array}{l}
					23{,}400 - 18{,}000\\
					= 5{,}400
				\end{array}$ &
				$\begin{array}{l}
					22{,}400 - 16{,}500\\
					= 5{,}900
				\end{array}$ \\
				Low demand, high cost &
				$\begin{array}{l}
					0.40 \times 0.45\\
					= 0.18
				\end{array}$ &
				$\begin{array}{l}
					23{,}400 - 24{,}000\\
					= -600
				\end{array}$ &
				$\begin{array}{l}
					22{,}400 - 21{,}000\\
					= 1{,}400
				\end{array}$ \\
				\bottomrule
			\end{tabular}
		\end{center}
		
		Final payoff table (USD):
		\begin{center}
			\begin{tabular}{l c c c}
				\toprule
				State of nature & Probability & In-house & Partner \\
				\midrule
				High demand, low cost & 0.33 & 22{,}500 & 19{,}200 \\
				High demand, high cost & 0.27 & 16{,}500 & 14{,}700 \\
				Low demand, low cost & 0.22 & 5{,}400 & 5{,}900 \\
				Low demand, high cost & 0.18 & -600 & 1{,}400 \\
				\bottomrule
			\end{tabular}
		\end{center}
		
		\item
		\subsection*{Community Theater Ticketing Strategy Under Demand Uncertainty}
		A community theater must choose one of three ticketing strategies for the upcoming season: Strategy A, Strategy B,
		or Strategy C. Demand can be strong or weak with probabilities $0.60$ and $0.40$. The final payoff table (in thousands of dollars)
		is given below. Use Maximax, Maximin, Minimax Regret, and Expected Value to select a strategy.
		
		\begin{center}
			\textit{Payoff table} \\
			\begin{tabular}{l c c c c}
				\toprule
				State of nature & Probability & Strategy A & Strategy B & Strategy C \\
				\midrule
				Strong demand & 0.60 & 60 & 52 & 45 \\
				Weak demand & 0.40 & 10 & 18 & 25 \\
				\bottomrule
			\end{tabular}
		\end{center}
		
		\subsection*{C4}
		\[
		\begin{aligned}
		\max(\text{Strategy A}) &= \max\{60,10\} = 60,\\
		\max(\text{Strategy B}) &= \max\{52,18\} = 52,\\
		\max(\text{Strategy C}) &= \max\{45,25\} = 45.
		\end{aligned}
		\]
		\[
		\max\{\max(\text{Strategy A}), \max(\text{Strategy B}), \max(\text{Strategy C})\}
		= \max\{60, 52, 45\}
		= 60.
		\]
		The Maximax choice is Strategy A because it has the highest possible payoff.
		
		\subsection*{C5}
		\[
		\begin{aligned}
		\min(\text{Strategy A}) &= \min\{60,10\} = 10,\\
		\min(\text{Strategy B}) &= \min\{52,18\} = 18,\\
		\min(\text{Strategy C}) &= \min\{45,25\} = 25.
		\end{aligned}
		\]
		\[
		\max\{\min(\text{Strategy A}), \min(\text{Strategy B}), \min(\text{Strategy C})\}
		= \max\{10, 18, 25\}
		= 25.
		\]
		The Maximin choice is Strategy C because it has the largest worst-case payoff.
		
		\subsection*{C6}
		Best payoff in each state:
		\[
		\begin{aligned}
			\text{Strong demand: } &\max\{60, 52, 45\} = 60, \\
			\text{Weak demand: } &\max\{10, 18, 25\} = 25.
		\end{aligned}
		\]
		Regret table (best payoff $-$ payoff):
		\begin{center}
			\begin{tabular}{l c c c c}
				\toprule
				Alternative & Strong demand $(0.60)$ & Weak demand $(0.40)$ & Maximum regret \\
				\midrule
				Strategy A & $60-60=0$ & $25-10=15$ & 15 \\
				Strategy B & $60-52=8$ & $25-18=7$ & 8 \\
				Strategy C & $60-45=15$ & $25-25=0$ & 15 \\
				\bottomrule
			\end{tabular}
		\end{center}
		The minimax regret choice is Strategy B because it has the smallest maximum regret $(8)$.
		
		\subsection*{C1}
		\[
		\begin{aligned}
		EV_A &= 0.60(60)+0.40(10)=36+4=40,\\
		EV_B &= 0.60(52)+0.40(18)=31.2+7.2=38.4,\\
		EV_C &= 0.60(45)+0.40(25)=27+10=37.
		\end{aligned}
		\]
		The expected value criterion chooses Strategy A because $40$ is the largest expected payoff.
		
		\item
		\subsection*{Holiday Inventory Plan Selection Under Demand Uncertainty}
		A retail chain must select one of three inventory plans for the holiday period: Plan A, Plan B, or Plan C.
		Demand can be strong, moderate, or weak with probabilities $0.30$, $0.50$, and $0.20$. The final payoff table
		(in thousands of dollars) is given below. Use Maximax, Maximin, Minimax Regret, and Expected Value to select a plan.
		
		\begin{center}
			\textit{Payoff table} \\
			\begin{tabular}{l c c c c}
				\toprule
				State of nature & Probability & Plan A & Plan B & Plan C \\
				\midrule
				Strong demand & 0.30 & 80 & 70 & 60 \\
				Moderate demand & 0.50 & 50 & 55 & 45 \\
				Weak demand & 0.20 & 10 & 30 & 40 \\
				\bottomrule
			\end{tabular}
		\end{center}
		
		\subsection*{C4}
		\[
		\begin{aligned}
		\max(\text{Plan A}) &= \max\{80,50,10\} = 80,\\
		\max(\text{Plan B}) &= \max\{70,55,30\} = 70,\\
		\max(\text{Plan C}) &= \max\{60,45,40\} = 60.
		\end{aligned}
		\]
		\[
		\max\{\max(\text{Plan A}), \max(\text{Plan B}), \max(\text{Plan C})\}
		= \max\{80, 70, 60\}
		= 80.
		\]
		The Maximax choice is Plan A because it has the highest best payoff.
		
		\subsection*{C5}
		\[
		\begin{aligned}
		\min(\text{Plan A}) &= \min\{80,50,10\} = 10,\\
		\min(\text{Plan B}) &= \min\{70,55,30\} = 30,\\
		\min(\text{Plan C}) &= \min\{60,45,40\} = 40.
		\end{aligned}
		\]
		\[
		\max\{\min(\text{Plan A}), \min(\text{Plan B}), \min(\text{Plan C})\}
		= \max\{10, 30, 40\}
		= 40.
		\]
		The Maximin choice is Plan C because it has the best worst-case payoff.
		
		\subsection*{C6}
		Best payoff in each state:
		\[
		\begin{aligned}
			\text{Strong demand: } &\max\{80, 70, 60\} = 80, \\
			\text{Moderate demand: } &\max\{50, 55, 45\} = 55, \\
			\text{Weak demand: } &\max\{10, 30, 40\} = 40.
		\end{aligned}
		\]
		Regret table (best payoff $-$ payoff):
		\begin{center}
			\begin{tabular}{l c c c c}
				\toprule
				Alternative & Strong $(0.30)$ & Moderate $(0.50)$ & Weak $(0.20)$ & Maximum regret \\
				\midrule
				Plan A & $80-80=0$ & $55-50=5$ & $40-10=30$ & 30 \\
				Plan B & $80-70=10$ & $55-55=0$ & $40-30=10$ & 10 \\
				Plan C & $80-60=20$ & $55-45=10$ & $40-40=0$ & 20 \\
				\bottomrule
			\end{tabular}
		\end{center}
		The minimax regret choice is Plan B because it has the smallest maximum regret $(10)$.
		
		\subsection*{C1}
		\[
		\begin{aligned}
		EV_A &= 0.30(80)+0.50(50)+0.20(10)=24+25+2=51,\\
		EV_B &= 0.30(70)+0.50(55)+0.20(30)=21+27.5+6=54.5,\\
		EV_C &= 0.30(60)+0.50(45)+0.20(40)=18+22.5+8=48.5.
		\end{aligned}
		\]
		The expected value criterion chooses Plan B because $54.5$ is the largest expected payoff.
		
		\item
		\subsection*{Routing System Choice Under Fuel and Congestion Uncertainty}
		A shipping company must choose between two routing systems for the next season: System A or System B.
		Four states of nature summarize fuel and congestion conditions with probabilities $0.25$, $0.30$, $0.20$, and $0.25$.
		The final payoff table (in thousands of dollars) is given below. Use Maximax, Maximin, Minimax Regret, and Expected Value to select a system.
		
		\begin{center}
			\textit{Payoff table} \\
			\begin{tabular}{l c c c}
				\toprule
				State of nature & Probability & System A & System B \\
				\midrule
				State $S_1$ & 0.25 & 40 & 35 \\
				State $S_2$ & 0.30 & 30 & 28 \\
				State $S_3$ & 0.20 & 15 & 20 \\
				State $S_4$ & 0.25 & -5 & 5 \\
				\bottomrule
			\end{tabular}
		\end{center}
		
		\subsection*{C4}
		\[
		\begin{aligned}
		\max(\text{System A}) &= \max\{40,30,15,-5\} = 40,\\
		\max(\text{System B}) &= \max\{35,28,20,5\} = 35.
		\end{aligned}
		\]
		\[
		\max\{\max(\text{System A}), \max(\text{System B})\}
		= \max\{40, 35\}
		= 40.
		\]
		The Maximax choice is System A because it has the highest possible payoff.
		
		\subsection*{C5}
		\[
		\begin{aligned}
		\min(\text{System A}) &= \min\{40,30,15,-5\} = -5,\\
		\min(\text{System B}) &= \min\{35,28,20,5\} = 5.
		\end{aligned}
		\]
		\[
		\max\{\min(\text{System A}), \min(\text{System B})\}
		= \max\{-5, 5\}
		= 5.
		\]
		The Maximin choice is System B because it has the best worst-case payoff.
		
		\subsection*{C6}
		Best payoff in each state:
		\[
		\begin{aligned}
			\text{State } S_1: &\max\{40, 35\} = 40, \\
			\text{State } S_2: &\max\{30, 28\} = 30, \\
			\text{State } S_3: &\max\{15, 20\} = 20, \\
			\text{State } S_4: &\max\{-5, 5\} = 5.
		\end{aligned}
		\]
		Regret table (best payoff $-$ payoff):
		\begin{center}
			\begin{tabular}{l c c c c c}
				\toprule
				Alternative & $S_1$ $(0.25)$ & $S_2$ $(0.30)$ & $S_3$ $(0.20)$ & $S_4$ $(0.25)$ & Maximum regret \\
				\midrule
				System A & $40-40=0$ & $30-30=0$ & $20-15=5$ & $5-(-5)=10$ & 10 \\
				System B & $40-35=5$ & $30-28=2$ & $20-20=0$ & $5-5=0$ & 5 \\
				\bottomrule
			\end{tabular}
		\end{center}
		The minimax regret choice is System B because it has the smallest maximum regret $(5)$.
		
		\subsection*{C1}
		\[
		\begin{aligned}
		EV_A &= 0.25(40)+0.30(30)+0.20(15)+0.25(-5)=10+9+3-1.25=20.75,\\
		EV_B &= 0.25(35)+0.30(28)+0.20(20)+0.25(5)=8.75+8.4+4+1.25=22.4.
		\end{aligned}
		\]
		The expected value criterion chooses System B because $22.4$ is the largest expected payoff.
		
		\item
		\subsection*{Expected Value Comparison of Marketing Options}
		A publisher must choose between two marketing options, A and B, before knowing which market condition will occur.
		There are two possible states of nature: state $S_1$, which occurs with probability $p$, and state $S_2$, which occurs with probability $1-p$.
		The objective is to select the option that yields the higher expected profit, measured in consistent monetary units.
		The payoff table below summarizes the profit associated with each option under each state of nature.
		\begin{center}
			\textit{Payoff table} \\
			\begin{tabular}{l c c}
				\toprule
				& $S_1$ $(p)$ & $S_2$ $(1-p)$ \\
				\midrule
				A & 24 & 12 \\
				B & 18 & 16 \\
				\bottomrule
			\end{tabular}
		\end{center}
		
		\subsection*{C7}
		\[
		EV(A)=24p+12(1-p)=12p+12
		\]
		\[
		EV(B)=18p+16(1-p)=2p+16
		\]
		Decision condition:
		\[
		EV(A)-EV(B)>0
		\]
		\[
		(12p+12)-(2p+16)>0 \Rightarrow 10p-4>0
		\]
		Option A is better when $p>0.4$, and option B is better when $p \le 0.4$.
		
		\item
		\subsection*{Expected Value Comparison of Production Plans}
		A manufacturer must select one production plan, labeled A, B, or C, before knowing which market condition will occur.
		There are two possible states of nature: state $S_1$, which occurs with probability $p$, and state $S_2$, which occurs with probability $1-p$.
		Each production plan generates a different profit depending on the realized state of nature, as summarized in the payoff table below.
		The objective is to determine which production plan maximizes expected profit as a function of $p$.
		
		\begin{center}
			\textit{Payoff table} \\
			\begin{tabular}{l c c}
				\toprule
				& $S_1$ $(p)$ & $S_2$ $(1-p)$ \\
				\midrule
				A & 32 & 4 \\
				B & 20 & 14 \\
				C & 12 & 10 \\
				\bottomrule
			\end{tabular}
		\end{center}
		
		\subsection*{C7}
		\[
		EV(A)=32p+4(1-p)=28p+4,\quad
		EV(B)=20p+14(1-p)=6p+14,\quad
		EV(C)=12p+10(1-p)=2p+10
		\]
		
		Pairwise comparison conditions in $>0$ form:
		\[
		EV(A)-EV(B)>0 \Rightarrow (28p+4)-(6p+14)>0 \Rightarrow 22p-10>0
		\]
		\[
		EV(B)-EV(C)>0 \Rightarrow (6p+14)-(2p+10)>0 \Rightarrow 4p+4>0
		\]
		\[
		EV(A)-EV(C)>0 \Rightarrow (28p+4)-(2p+10)>0 \Rightarrow 26p-6>0
		\]
		Plan A is optimal when $p>\frac{5}{11}$, Plan B is optimal when $p \le \frac{5}{11}$, and Plan C is never optimal because $EV(B) > EV(C)$ for all $p$.
		
		\item
		\subsection*{Expected Value Comparison of Zoning Options}
		A city council must choose between two zoning options, labeled A and B, before knowing which future condition will occur.
		There are two possible states of nature: state $S_1$, which occurs with probability $p$, and state $S_2$, which occurs with probability $1-p$.
		Each zoning option generates a different net return depending on the realized state of nature, as shown in the payoff table below.
		The objective is to determine which zoning option yields the higher expected return as a function of $p$.
		
		\begin{center}
			\textit{Payoff table} \\
			\begin{tabular}{l c c}
				\toprule
				& $S_1$ $(p)$ & $S_2$ $(1-p)$ \\
				\midrule
				A & 26 & 6 \\
				B & 20 & 12 \\
				\bottomrule
			\end{tabular}
		\end{center}
		
		\subsection*{C7}
		\[
		EV(A)=26p+6(1-p)=20p+6
		\]
		\[
		EV(B)=20p+12(1-p)=8p+12
		\]
		Decision condition:
		\[
		EV(A)-EV(B)>0
		\]
		\[
		(20p+6)-(8p+12)>0 \Rightarrow 12p-6>0
		\]
		Option A is better when $p>0.5$, and option B is better when $p \le 0.5$.
	\end{ExamProblems}
\end{document}
