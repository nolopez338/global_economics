\documentclass[11pt]{article}

\usepackage[margin=1in]{geometry}

\usepackage{amsmath,amssymb}

\usepackage{booktabs}

\usepackage{hyperref}

\hypersetup{hidelinks}

\setlength{\parindent}{0pt}

\setlength{\parskip}{6pt}

\begin{document}
	
	\section*{Practice Activity 1}
	
	Problems and detailed worked solutions on Expected Value, Maximin, and Maximax.
	
	\section*{Contents}
	
	\hypertarget{table-of-contents}{}
	
	\begin{itemize}
		\item \hyperlink{problem1}{Problem 1 — Local Café Investment}
		\item \hyperlink{problem2}{Problem 2 — Green Energy Production Mix}
		\item \hyperlink{problem3}{Problem 3 — Global Shipping Network Design}
		\item \hyperlink{problem4}{Problem 4 — Community Bakery Lunch Boxes}
		\item \hyperlink{problem5}{Problem 5 — Rural Internet Subscriptions}
		\item \hyperlink{problem6}{Problem 6 — Coffee Export Routes}
		\item \hyperlink{problem7}{Problem 7 — Lake Ferry Tourism}
		\item \hyperlink{problem8}{Problem 8 — Cold-Storage Contracts}
		\item \hyperlink{problem9}{Problem 9 — Solar Installation Bids}
		\item \hyperlink{problem10}{Problem 10 — Air Cargo Charters}
		\item \hyperlink{problem11}{Problem 11 — Seafood Export Licenses}
	\end{itemize}
	
	
	\section*{Problem 1 — Local Café Investment}
	
	\hypertarget{problem1}{}
	
	\hypertarget{problem1-criteria}{}
	
	
	\begin{itemize}
		\item \hyperlink{problem1-problem}{Problem description}
		\item \hyperlink{problem1-c2}{(C2) Interpreting decision alternatives, events, consequences, and states}
		\item \hyperlink{problem1-c3}{(C3) Building the payoff table}
		\item \hyperlink{problem1-c4}{(C4) Applying the Maximax criterion}
		\item \hyperlink{problem1-c5}{(C5) Applying the Maximin criterion}
		\item \hyperlink{problem1-c1}{(C1) Formulating the decision problem for an optimal choice using expected value}
	\end{itemize}
	
	
	\subsection*{Problem description}
	
	\hypertarget{problem1-problem}{}
	
	A small café is deciding whether to launch a new line of artisan desserts. The transaction is daily café sales, where
	profit is earned from each dessert plate sold.
	
	
	
	
	
	
	The owner has two options:
	
	
	
	
	
	
	• \textbf{Launch the dessert line}, with profit outcomes depending on foot traffic:
	– High traffic $\to 42$ thousand USD
	– Low traffic $\to 8$ thousand USD
	
	
	
	
	• \textbf{Keep the current menu}, with profit outcomes:
	– High traffic $\to 28$ thousand USD
	– Low traffic $\to 20$ thousand USD
	
	
	
	
	
	
	Foot-traffic probabilities:
	• High traffic $\to 0.55$
	• Low traffic $\to 0.45$
	
	
	
	
	
	\textbf{Question:} Make a decision on the best option based on three perspectives:
	safe behavior (Maximin), balanced behavior (Expected Value), and risky behavior (Maximax).
	
	\subsection*{Solution}
	
	\hypertarget{problem1-solution}{}
	
	\paragraph{Structured solution}
	
	
	(C2) Interpreting decision alternatives, events, consequences, and states
	+
	
	
	\begin{center}
		\begin{tabular}{ll}
			\toprule
			Element & Description \\
			\midrule
			Alternatives & Launch dessert line; Keep current menu \\
			States of nature & High traffic
			($0.55$)
			Low traffic
			($0.45$) \\
			Events & Actual foot traffic after the decision \\
			Consequences & Profit in thousand USD for each alternative–state pair \\
			\bottomrule
		\end{tabular}
	\end{center}
	
	
	
	
	
	
	(C3) Building the payoff table
	+
	
	
	\begin{center}
		\begin{tabular}{lll}
			\toprule
			Alternative & High traffic
			($0.55$) & Low traffic
			($0.45$) \\
			\midrule
			Launch dessert line & 42 & 8 \\
			Keep menu & 28 & 20 \\
			\bottomrule
		\end{tabular}
	\end{center}
	
	
	
	
	
	
	(C4) Applying the Maximax criterion
	+
	
	
	The maximax rule selects the alternative with the highest possible payoff.
	\[
	\begin{aligned}
		\max_i \text{Payoff}(\text{Launch}, S_i) \&= \max\{42, 8\} = 42 \\
		\max_i \text{Payoff}(\text{Keep}, S_i) \&= \max\{28, 20\} = 28
	\end{aligned}
	\]
	\[
	\max\{42, 28\} = 42
	\]
Because $42$ is the largest best payoff, the maximax (risk-seeking) recommendation is \emph{launch the dessert line}.
	
	
	
	
	
	(C5) Applying the Maximin criterion
	+
	
	
	The maximin rule focuses on the worst-case payoff of each alternative.
	\[
	\begin{aligned}
		\min_i \text{Payoff}(\text{Launch}, S_i) \&= \min\{42, 8\} = 8 \\
		\min_i \text{Payoff}(\text{Keep}, S_i) \&= \min\{28, 20\} = 20
	\end{aligned}
	\]
	\[
	\max\{8, 20\} = 20
	\]
Since $20$ exceeds $8$, the maximin (risk-averse) choice is \emph{keep the current menu}.
	
	
	
	
	
	(C1) Formulating the decision problem for an optimal choice using expected value
	+
	
	
	We choose the option that maximizes the café's profit objective. Alternatives are \textbf{launch} or
	\textbf{keep} the menu; states are \textbf{high} or \textbf{low traffic}. Expected values:
	\[
	\begin{aligned}
		\mathrm{EV}_{\text{Launch}} \&= 0.55(42) + 0.45(8) = 23.1 + 3.6 = 26.7 \\
		\mathrm{EV}_{\text{Keep}} \&= 0.55(28) + 0.45(20) = 15.4 + 9 = 24.4
	\end{aligned}
	\]
	\[
	\max\{26.7, 24.4\} = 26.7
	\]
	
	Summary of recommendations across decision criteria
	\begin{itemize}
		\item Maximax favors \emph{launching the dessert line} (highest payoff of $42$).
		\item Maximin favors \emph{keeping the current menu} (best worst-case payoff of $20$).
		\item Expected value favors \emph{launching the dessert line} (EV of $26.7$).
	\end{itemize}
	
	
	Launching desserts leads on optimistic and average outcomes, while the current menu protects the downside.
	
	\subsection*{(C2) Interpreting decision alternatives, events, consequences, and states}
	
	\hypertarget{problem1-c2}{}
	
	\begin{center}
		\begin{tabular}{ll}
			\toprule
			Element & Description \\
			\midrule
			Alternatives & Launch dessert line; Keep current menu \\
			States of nature & High traffic
			($0.55$)
			Low traffic
			($0.45$) \\
			Events & Actual foot traffic after the decision \\
			Consequences & Profit in thousand USD for each alternative–state pair \\
			\bottomrule
		\end{tabular}
	\end{center}
	
	\subsection*{(C3) Building the payoff table}
	
	\hypertarget{problem1-c3}{}
	
	\begin{center}
		\begin{tabular}{lll}
			\toprule
			Alternative & High traffic
			($0.55$) & Low traffic
			($0.45$) \\
			\midrule
			Launch dessert line & 42 & 8 \\
			Keep menu & 28 & 20 \\
			\bottomrule
		\end{tabular}
	\end{center}
	
	\subsection*{(C4) Applying the Maximax criterion}
	
	\hypertarget{problem1-c4}{}
	
	The maximax rule selects the alternative with the highest possible payoff.
	\[
	\begin{aligned}
		\max_i \text{Payoff}(\text{Launch}, S_i) \&= \max\{42, 8\} = 42 \\
		\max_i \text{Payoff}(\text{Keep}, S_i) \&= \max\{28, 20\} = 28
	\end{aligned}
	\]
	\[
	\max\{42, 28\} = 42
	\]
Because $42$ is the largest best payoff, the maximax (risk-seeking) recommendation is \emph{launch the dessert line}.
	
	\subsection*{(C5) Applying the Maximin criterion}
	
	\hypertarget{problem1-c5}{}
	
	The maximin rule focuses on the worst-case payoff of each alternative.
	\[
	\begin{aligned}
		\min_i \text{Payoff}(\text{Launch}, S_i) \&= \min\{42, 8\} = 8 \\
		\min_i \text{Payoff}(\text{Keep}, S_i) \&= \min\{28, 20\} = 20
	\end{aligned}
	\]
	\[
	\max\{8, 20\} = 20
	\]
Since $20$ exceeds $8$, the maximin (risk-averse) choice is \emph{keep the current menu}.
	
	\subsection*{(C1) Formulating the decision problem for an optimal choice using expected value}
	
	\hypertarget{problem1-c1}{}
	
	We choose the option that maximizes the café's profit objective. Alternatives are \textbf{launch} or
	\textbf{keep} the menu; states are \textbf{high} or \textbf{low traffic}. Expected values:
	\[
	\begin{aligned}
		\mathrm{EV}_{\text{Launch}} \&= 0.55(42) + 0.45(8) = 23.1 + 3.6 = 26.7 \\
		\mathrm{EV}_{\text{Keep}} \&= 0.55(28) + 0.45(20) = 15.4 + 9 = 24.4
	\end{aligned}
	\]
	\[
	\max\{26.7, 24.4\} = 26.7
	\]
	
	Summary of recommendations across decision criteria
	\begin{itemize}
		\item Maximax favors \emph{launching the dessert line} (highest payoff of $42$).
		\item Maximin favors \emph{keeping the current menu} (best worst-case payoff of $20$).
		\item Expected value favors \emph{launching the dessert line} (EV of $26.7$).
	\end{itemize}
	
	Launching desserts leads on optimistic and average outcomes, while the current menu protects the downside.
	
	\section*{Problem 2 — Green Energy Production Mix}
	
	\hypertarget{problem2}{}
	
	\hypertarget{problem2-criteria}{}
	
	
	\begin{itemize}
		\item \hyperlink{problem2-problem}{Problem description}
		\item \hyperlink{problem2-c2}{(C2) Interpreting decision alternatives, events, consequences, and states}
		\item \hyperlink{problem2-c3}{(C3) Building the payoff table}
		\item \hyperlink{problem2-c4}{(C4) Applying the Maximax criterion}
		\item \hyperlink{problem2-c5}{(C5) Applying the Maximin criterion}
		\item \hyperlink{problem2-c1}{(C1) Formulating the decision problem for an optimal choice using expected value}
	\end{itemize}
	
	
	\subsection*{Problem description}
	
	\hypertarget{problem2-problem}{}
	
	A renewable-energy company must select an electricity generation mix. The transaction is annual electricity sales,
	where profit is earned per megawatt-hour generated and sold.
	
	
	
	
	
	
	Three weather states are possible:
	
	
	
	
	
	
	• Windy
	• Sunny
	• Cloudy
	
	
	
	
	
	
	The company is comparing three strategies:
	
	1. Build wind turbines
	2. Build solar farms
	3. Build a balanced hybrid system
	
	
	
	
	
	
	Expected profits (in thousand USD):
	
	
	
	
	• Wind turbines $\to 90$ (Windy), $45$ (Sunny), $30$ (Cloudy)
	• Solar farms $\to 35$ (Windy), $95$ (Sunny), $25$ (Cloudy)
	• Hybrid system $\to 70$ (Windy), $65$ (Sunny), $60$ (Cloudy)
	
	
	
	
	
	
	Weather probabilities:
	• Windy $\to 0.30$
	• Sunny $\to 0.45$
	• Cloudy $\to 0.25$
	
	
	
	
	
	\textbf{Question:} Make a decision on the best strategy based on three perspectives:
	safe behavior (Maximin), balanced behavior (Expected Value), and risky behavior (Maximax).
	
	\subsection*{Solution}
	
	\hypertarget{problem2-solution}{}
	
	\paragraph{Structured solution}
	
	
	(C2) Interpreting decision alternatives, events, consequences, and states
	+
	
	
	\begin{center}
		\begin{tabular}{ll}
			\toprule
			Element & Description \\
			\midrule
			Alternatives & Wind turbines; Solar farms; Hybrid system \\
			States of nature & Windy
			($0.30$)
			Sunny
			($0.45$)
			Cloudy
			($0.25$) \\
			Events & Weather outcome affecting generation \\
			Consequences & Profit in thousand USD for each strategy–weather combination \\
			\bottomrule
		\end{tabular}
	\end{center}
	
	
	
	
	
	
	(C3) Building the payoff table
	+
	
	
	\begin{center}
		\begin{tabular}{llll}
			\toprule
			Strategy & Windy
			($0.30$) & Sunny
			($0.45$) & Cloudy
			($0.25$) \\
			\midrule
			Wind turbines & 90 & 45 & 30 \\
			Solar farms & 35 & 95 & 25 \\
			Hybrid system & 70 & 65 & 60 \\
			\bottomrule
		\end{tabular}
	\end{center}
	
	
	
	
	
	
	(C4) Applying the Maximax criterion
	+
	
	
	Maximax compares the highest payoff for each alternative.
	\[
	\begin{aligned}
		\max_i \text{Payoff}(\text{Wind}, S_i) \&= \max\{90, 45, 30\} = 90 \\
		\max_i \text{Payoff}(\text{Solar}, S_i) \&= \max\{35, 95, 25\} = 95 \\
		\max_i \text{Payoff}(\text{Hybrid}, S_i) \&= \max\{70, 65, 60\} = 70
	\end{aligned}
	\]
	\[
	\max\{90, 95, 70\} = 95
	\]
The largest best payoff is $95$, so the maximax decision is \emph{build solar farms}.
	
	
	
	
	
	(C5) Applying the Maximin criterion
	+
	
	
	Maximin compares the worst payoff for each alternative.
	\[
	\begin{aligned}
		\min_i \text{Payoff}(\text{Wind}, S_i) \&= \min\{90, 45, 30\} = 30 \\
		\min_i \text{Payoff}(\text{Solar}, S_i) \&= \min\{35, 95, 25\} = 25 \\
		\min_i \text{Payoff}(\text{Hybrid}, S_i) \&= \min\{70, 65, 60\} = 60
	\end{aligned}
	\]
	\[
	\max\{30, 25, 60\} = 60
	\]
	The maximin recommendation is the \emph{hybrid system}.
	
	
	
	
	
	(C1) Formulating the decision problem for an optimal choice using expected value
	+
	
	
	The firm selects the generation mix that maximizes profit. Alternatives are \textbf{wind},
	\textbf{solar}, or \textbf{hybrid} and the uncertain states are \textbf{windy},
	\textbf{sunny}, or \textbf{cloudy}. Expected values:
	\[
	\begin{aligned}
		\mathrm{EV}_{\text{Wind}} \&= 0.30(90) + 0.45(45) + 0.25(30) = 27 + 20.25 + 7.5 = 54.75 \\
		\mathrm{EV}_{\text{Solar}} \&= 0.30(35) + 0.45(95) + 0.25(25) = 10.5 + 42.75 + 6.25 = 59.5 \\
		\mathrm{EV}_{\text{Hybrid}} \&= 0.30(70) + 0.45(65) + 0.25(60) = 21 + 29.25 + 15 = 65.25
	\end{aligned}
	\]
	\[
	\max\{54.75, 59.5, 65.25\} = 65.25
	\]
	
	Summary of recommendations across decision criteria
	\begin{itemize}
		\item Maximax favors \emph{solar farms} (highest payoff of $95$).
		\item Maximin favors the \emph{hybrid system} (best worst-case payoff of $60$).
		\item Expected value favors the \emph{hybrid system} (EV of $65.25$).
	\end{itemize}
	
	
	Solar has the best upside, but the hybrid option balances downside protection with the strongest average profit.
	
	\subsection*{(C2) Interpreting decision alternatives, events, consequences, and states}
	
	\hypertarget{problem2-c2}{}
	
	\begin{center}
		\begin{tabular}{ll}
			\toprule
			Element & Description \\
			\midrule
			Alternatives & Wind turbines; Solar farms; Hybrid system \\
			States of nature & Windy
			($0.30$)
			Sunny
			($0.45$)
			Cloudy
			($0.25$) \\
			Events & Weather outcome affecting generation \\
			Consequences & Profit in thousand USD for each strategy–weather combination \\
			\bottomrule
		\end{tabular}
	\end{center}
	
	\subsection*{(C3) Building the payoff table}
	
	\hypertarget{problem2-c3}{}
	
	\begin{center}
		\begin{tabular}{llll}
			\toprule
			Strategy & Windy
			($0.30$) & Sunny
			($0.45$) & Cloudy
			($0.25$) \\
			\midrule
			Wind turbines & 90 & 45 & 30 \\
			Solar farms & 35 & 95 & 25 \\
			Hybrid system & 70 & 65 & 60 \\
			\bottomrule
		\end{tabular}
	\end{center}
	
	\subsection*{(C4) Applying the Maximax criterion}
	
	\hypertarget{problem2-c4}{}
	
	Maximax compares the highest payoff for each alternative.
	\[
	\begin{aligned}
		\max_i \text{Payoff}(\text{Wind}, S_i) \&= \max\{90, 45, 30\} = 90 \\
		\max_i \text{Payoff}(\text{Solar}, S_i) \&= \max\{35, 95, 25\} = 95 \\
		\max_i \text{Payoff}(\text{Hybrid}, S_i) \&= \max\{70, 65, 60\} = 70
	\end{aligned}
	\]
	\[
	\max\{90, 95, 70\} = 95
	\]
The largest best payoff is $95$, so the maximax decision is \emph{build solar farms}.
	
	\subsection*{(C5) Applying the Maximin criterion}
	
	\hypertarget{problem2-c5}{}
	
	Maximin compares the worst payoff for each alternative.
	\[
	\begin{aligned}
		\min_i \text{Payoff}(\text{Wind}, S_i) \&= \min\{90, 45, 30\} = 30 \\
		\min_i \text{Payoff}(\text{Solar}, S_i) \&= \min\{35, 95, 25\} = 25 \\
		\min_i \text{Payoff}(\text{Hybrid}, S_i) \&= \min\{70, 65, 60\} = 60
	\end{aligned}
	\]
	\[
	\max\{30, 25, 60\} = 60
	\]
	The maximin recommendation is the \emph{hybrid system}.
	
	\subsection*{(C1) Formulating the decision problem for an optimal choice using expected value}
	
	\hypertarget{problem2-c1}{}
	
	The firm selects the generation mix that maximizes profit. Alternatives are \textbf{wind},
	\textbf{solar}, or \textbf{hybrid} and the uncertain states are \textbf{windy},
	\textbf{sunny}, or \textbf{cloudy}. Expected values:
	\[
	\begin{aligned}
		\mathrm{EV}_{\text{Wind}} \&= 0.30(90) + 0.45(45) + 0.25(30) = 27 + 20.25 + 7.5 = 54.75 \\
		\mathrm{EV}_{\text{Solar}} \&= 0.30(35) + 0.45(95) + 0.25(25) = 10.5 + 42.75 + 6.25 = 59.5 \\
		\mathrm{EV}_{\text{Hybrid}} \&= 0.30(70) + 0.45(65) + 0.25(60) = 21 + 29.25 + 15 = 65.25
	\end{aligned}
	\]
	\[
	\max\{54.75, 59.5, 65.25\} = 65.25
	\]
	
	Summary of recommendations across decision criteria
	\begin{itemize}
		\item Maximax favors \emph{solar farms} (highest payoff of $95$).
		\item Maximin favors the \emph{hybrid system} (best worst-case payoff of $60$).
		\item Expected value favors the \emph{hybrid system} (EV of $65.25$).
	\end{itemize}
	
	Solar has the best upside, but the hybrid option balances downside protection with the strongest average profit.
	
	\section*{Problem 3 — Global Shipping Network Design}
	
	\hypertarget{problem3}{}
	
	\hypertarget{problem3-criteria}{}
	
	
	\begin{itemize}
		\item \hyperlink{problem3-problem}{Problem description}
		\item \hyperlink{problem3-c2}{(C2) Interpreting decision alternatives, events, consequences, and states}
		\item \hyperlink{problem3-c3}{(C3) Building the payoff table}
		\item \hyperlink{problem3-c4}{(C4) Applying the Maximax criterion}
		\item \hyperlink{problem3-c5}{(C5) Applying the Maximin criterion}
		\item \hyperlink{problem3-c1}{(C1) Formulating the decision problem for an optimal choice using expected value}
	\end{itemize}
	
	
	\subsection*{Problem description}
	
	\hypertarget{problem3-problem}{}
	
	A multinational logistics firm is choosing between two shipping network designs for long-term freight contracts. The
	transaction is global freight services, where profit is earned per contract delivered on schedule.
	
	
	
	
	
	
	Management is considering:
	
	
	
	
	
	
	• \textbf{Centralized mega-hub network}
	• \textbf{Regional multi-hub network}
	
	
	
	
	The firm models four trade conditions:
	• Trade boom ($0.20$)
	• Stable trade ($0.35$)
	• Moderate disruptions ($0.30$)
	• Severe disruptions ($0.15$)
	
	
	
	
	
	
	Expected profits (in thousand USD) are:
	• Centralized network $\to 220, 140, 40, -30$
	• Regional network $\to 180, 160, 110, 60$
	
	
	
	
	
	\textbf{Question:} Make a decision on the best configuration based on three perspectives:
	safe behavior (Maximin), balanced behavior (Expected Value), and risky behavior (Maximax).
	
	\subsection*{Solution}
	
	\hypertarget{problem3-solution}{}
	
	\paragraph{Structured solution}
	
	
	(C2) Interpreting decision alternatives, events, consequences, and states
	+
	
	
	\begin{center}
		\begin{tabular}{ll}
			\toprule
			Element & Description \\
			\midrule
			Alternatives & Centralized mega-hub; Regional multi-hub \\
			States of nature & \shortstack{Boom ($0.20$)\\Stable ($0.35$)\\Moderate disruptions ($0.30$)\\Severe disruptions ($0.15$)} \\
			Events & Trade conditions realized during the planning horizon \\
			Consequences & Profit in thousand USD for each network–state pair \\
			\bottomrule
		\end{tabular}
	\end{center}
	
	
	
	
	
	
	(C3) Building the payoff table
	+
	
	
	\begin{center}
		\begin{tabular}{lllll}
			\toprule
			Network & Boom
			($0.20$) & Stable
			($0.35$) & \shortstack{Moderate\\disruptions\\($0.30$)} & \shortstack{Severe\\disruptions\\($0.15$)} \\
			\midrule
			Centralized mega-hub & 220 & 140 & 40 & -30 \\
			Regional multi-hub & 180 & 160 & 110 & 60 \\
			\bottomrule
		\end{tabular}
	\end{center}
	
	
	
	
	
	
	(C4) Applying the Maximax criterion
	+
	
	
	Maximax selects the alternative with the highest possible payoff.
	\[
	\begin{aligned}
		\max_i \text{Payoff}(\text{Centralized}, S_i) \&= \max\{220, 140, 40, -30\} = 220 \\
		\max_i \text{Payoff}(\text{Regional}, S_i) \&= \max\{180, 160, 110, 60\} = 180
	\end{aligned}
	\]
	\[
	\max\{220, 180\} = 220
	\]
	The maximax choice is the \emph{centralized mega-hub network}.
	
	
	
	
	
	(C5) Applying the Maximin criterion
	+
	
	
	Maximin compares the worst payoff for each alternative.
	\[
	\begin{aligned}
		\min_i \text{Payoff}(\text{Centralized}, S_i) \&= \min\{220, 140, 40, -30\} = -30 \\
		\min_i \text{Payoff}(\text{Regional}, S_i) \&= \min\{180, 160, 110, 60\} = 60
	\end{aligned}
	\]
	\[
	\max\{-30, 60\} = 60
	\]
	The maximin recommendation is the \emph{regional multi-hub network}.
	
	
	
	
	
	(C1) Formulating the decision problem for an optimal choice using expected value
	+
	
	
	The firm chooses the network that maximizes long-run profit. Alternatives are \textbf{centralized} or
	\textbf{regional} and the states are \textbf{boom}, \textbf{stable}, \textbf{moderate},
	and \textbf{severe disruptions}. Expected values:
	\[
	\begin{aligned}
		\mathrm{EV}_{\text{Centralized}} \&= 0.20(220) + 0.35(140) + 0.30(40) + 0.15(-30) \\
		\&= 44 + 49 + 12 - 4.5 = 100.5 \\
		\mathrm{EV}_{\text{Regional}} \&= 0.20(180) + 0.35(160) + 0.30(110) + 0.15(60) \\
		\&= 36 + 56 + 33 + 9 = 134
	\end{aligned}
	\]
	\[
	\max\{100.5, 134\} = 134
	\]
	
	Summary of recommendations across decision criteria
	\begin{itemize}
		\item Maximax favors the \emph{centralized mega-hub network} (highest payoff of $220$).
		\item Maximin favors the \emph{regional multi-hub network} (best worst-case payoff of $60$).
		\item Expected value favors the \emph{regional multi-hub network} (EV of $134$).
	\end{itemize}
	
	
	Centralization offers peak upside, but the regional option delivers stronger protection and higher average profit.
	
	\subsection*{(C2) Interpreting decision alternatives, events, consequences, and states}
	
	\hypertarget{problem3-c2}{}
	
	\begin{center}
		\begin{tabular}{ll}
			\toprule
			Element & Description \\
			\midrule
			Alternatives & Centralized mega-hub; Regional multi-hub \\
			States of nature & \shortstack{Boom ($0.20$)\\Stable ($0.35$)\\Moderate disruptions ($0.30$)\\Severe disruptions ($0.15$)} \\
			Events & Trade conditions realized during the planning horizon \\
			Consequences & Profit in thousand USD for each network–state pair \\
			\bottomrule
		\end{tabular}
	\end{center}
	
	\subsection*{(C3) Building the payoff table}
	
	\hypertarget{problem3-c3}{}
	
	\begin{center}
		\begin{tabular}{lllll}
			\toprule
			Network & Boom
			($0.20$) & Stable
			($0.35$) & \shortstack{Moderate\\disruptions\\($0.30$)} & \shortstack{Severe\\disruptions\\($0.15$)} \\
			\midrule
			Centralized mega-hub & 220 & 140 & 40 & -30 \\
			Regional multi-hub & 180 & 160 & 110 & 60 \\
			\bottomrule
		\end{tabular}
	\end{center}
	
	\subsection*{(C4) Applying the Maximax criterion}
	
	\hypertarget{problem3-c4}{}
	
	Maximax selects the alternative with the highest possible payoff.
	\[
	\begin{aligned}
		\max_i \text{Payoff}(\text{Centralized}, S_i) \&= \max\{220, 140, 40, -30\} = 220 \\
		\max_i \text{Payoff}(\text{Regional}, S_i) \&= \max\{180, 160, 110, 60\} = 180
	\end{aligned}
	\]
	\[
	\max\{220, 180\} = 220
	\]
	The maximax choice is the \emph{centralized mega-hub network}.
	
	\subsection*{(C5) Applying the Maximin criterion}
	
	\hypertarget{problem3-c5}{}
	
	Maximin compares the worst payoff for each alternative.
	\[
	\begin{aligned}
		\min_i \text{Payoff}(\text{Centralized}, S_i) \&= \min\{220, 140, 40, -30\} = -30 \\
		\min_i \text{Payoff}(\text{Regional}, S_i) \&= \min\{180, 160, 110, 60\} = 60
	\end{aligned}
	\]
	\[
	\max\{-30, 60\} = 60
	\]
	The maximin recommendation is the \emph{regional multi-hub network}.
	
	\subsection*{(C1) Formulating the decision problem for an optimal choice using expected value}
	
	\hypertarget{problem3-c1}{}
	
	The firm chooses the network that maximizes long-run profit. Alternatives are \textbf{centralized} or
	\textbf{regional} and the states are \textbf{boom}, \textbf{stable}, \textbf{moderate},
	and \textbf{severe disruptions}. Expected values:
	\[
	\begin{aligned}
		\mathrm{EV}_{\text{Centralized}} \&= 0.20(220) + 0.35(140) + 0.30(40) + 0.15(-30) \\
		\&= 44 + 49 + 12 - 4.5 = 100.5 \\
		\mathrm{EV}_{\text{Regional}} \&= 0.20(180) + 0.35(160) + 0.30(110) + 0.15(60) \\
		\&= 36 + 56 + 33 + 9 = 134
	\end{aligned}
	\]
	\[
	\max\{100.5, 134\} = 134
	\]
	
	Summary of recommendations across decision criteria
	\begin{itemize}
		\item Maximax favors the \emph{centralized mega-hub network} (highest payoff of $220$).
		\item Maximin favors the \emph{regional multi-hub network} (best worst-case payoff of $60$).
		\item Expected value favors the \emph{regional multi-hub network} (EV of $134$).
	\end{itemize}
	
	Centralization offers peak upside, but the regional option delivers stronger protection and higher average profit.
	
	\section*{Problem 4 — Community Bakery Lunch Boxes}
	
	\hypertarget{problem4}{}
	
	\hypertarget{problem4-criteria}{}
	
	
	\begin{itemize}
		\item \hyperlink{problem4-problem}{Problem description}
		\item \hyperlink{problem4-c2}{(C2) Interpreting decision alternatives, events, consequences, and states}
		\item \hyperlink{problem4-c3}{(C3) Building the payoff table}
		\item \hyperlink{problem4-c4}{(C4) Applying the Maximax criterion}
		\item \hyperlink{problem4-c5}{(C5) Applying the Maximin criterion}
		\item \hyperlink{problem4-c1}{(C1) Formulating the decision problem for an optimal choice using expected value}
	\end{itemize}
	
	
	\subsection*{Problem description}
	
	\hypertarget{problem4-problem}{}
	
	A community bakery is bidding on office lunch-box contracts. The transaction is meal box sales, where profit is earned
	per lunch box delivered.
	
	
	
	
	
	
	The bakery can choose one pricing strategy:
	
	
	
	
	
	
	• \textbf{Standard packaging} with a profit of $0.04$ thousand USD per lunch box
	• \textbf{Premium packaging} with a profit of $0.07$ thousand USD per lunch box
	
	
	
	
	
	
	Demand states:
	• High office demand ($0.60$)
	• Low office demand ($0.40$)
	
	
	
	
	
	
	Expected number of lunch boxes:
	• Standard packaging $\to 900$ (High), $550$ (Low)
	• Premium packaging $\to 650$ (High), $250$ (Low)
	
	
	
	
	
	\textbf{Question:} Compute profits and choose the best option using Maximin, Expected Value, and Maximax.
	
	\subsection*{Solution}
	
	\hypertarget{problem4-solution}{}
	
	\paragraph{Structured solution}
	
	
	(C2) Interpreting decision alternatives, events, consequences, and states
	+
	
	
	\begin{center}
		\begin{tabular}{ll}
			\toprule
			Element & Description \\
			\midrule
			Alternatives & Standard packaging; Premium packaging \\
			States of nature & High demand
			($0.60$)
			Low demand
			($0.40$) \\
			Events & Actual number of lunch-box contracts realized \\
			Consequences & Profit in thousand USD, based on boxes $\times$ profit per box \\
			\bottomrule
		\end{tabular}
	\end{center}
	
	
	
	
	
	
	(C3) Building the payoff table
	+
	
	
	Profit calculations (thousand USD):
	\begin{itemize}
		\item Standard: High $\to 900 \times 0.04 = 36$; Low $\to 550 \times 0.04 = 22$.
		\item Premium: High $\to 650 \times 0.07 = 45.5$; Low $\to 250 \times 0.07 = 17.5$.
	\end{itemize}
	
	\begin{center}
		\begin{tabular}{lll}
			\toprule
			Alternative & High demand
			($0.60$) & Low demand
			($0.40$) \\
			\midrule
			Standard packaging & 36 & 22 \\
			Premium packaging & 45.5 & 17.5 \\
			\bottomrule
		\end{tabular}
	\end{center}
	
	
	
	
	
	
	(C4) Applying the Maximax criterion
	+
	
	
	Maximax compares the best outcomes.
	\[
	\begin{aligned}
		\max_i \text{Payoff}(\text{Standard}, S_i) \&= \max\{36, 22\} = 36 \\
		\max_i \text{Payoff}(\text{Premium}, S_i) \&= \max\{45.5, 17.5\} = 45.5
	\end{aligned}
	\]
	\[
	\max\{36, 45.5\} = 45.5
	\]
	The maximax choice is \emph{premium packaging}.
	
	
	
	
	
	(C5) Applying the Maximin criterion
	+
	
	
	Maximin compares the worst outcomes.
	\[
	\begin{aligned}
		\min_i \text{Payoff}(\text{Standard}, S_i) \&= \min\{36, 22\} = 22 \\
		\min_i \text{Payoff}(\text{Premium}, S_i) \&= \min\{45.5, 17.5\} = 17.5
	\end{aligned}
	\]
	\[
	\max\{22, 17.5\} = 22
	\]
	The maximin recommendation is \emph{standard packaging}.
	
	
	
	
	
	(C1) Formulating the decision problem for an optimal choice using expected value
	+
	
	
	We select the pricing strategy that maximizes expected profit. Expected values:
	\[
	\begin{aligned}
		\mathrm{EV}_{\text{Standard}} \&= 0.60(36) + 0.40(22) = 21.6 + 8.8 = 30.4 \\
		\mathrm{EV}_{\text{Premium}} \&= 0.60(45.5) + 0.40(17.5) = 27.3 + 7 = 34.3
	\end{aligned}
	\]
	\[
	\max\{30.4, 34.3\} = 34.3
	\]
	
	Summary of recommendations across decision criteria
	\begin{itemize}
		\item Maximax favors \emph{premium packaging} (highest payoff of $45.5$).
		\item Maximin favors \emph{standard packaging} (best worst-case payoff of $22$).
		\item Expected value favors \emph{premium packaging} (EV of $34.3$).
	\end{itemize}
	
	
	Premium packaging leads on optimistic and average profits, while standard packaging limits downside risk.
	
	\subsection*{(C2) Interpreting decision alternatives, events, consequences, and states}
	
	\hypertarget{problem4-c2}{}
	
	\begin{center}
		\begin{tabular}{ll}
			\toprule
			Element & Description \\
			\midrule
			Alternatives & Standard packaging; Premium packaging \\
			States of nature & High demand
			($0.60$)
			Low demand
			($0.40$) \\
			Events & Actual number of lunch-box contracts realized \\
			Consequences & Profit in thousand USD, based on boxes $\times$ profit per box \\
			\bottomrule
		\end{tabular}
	\end{center}
	
	\subsection*{(C3) Building the payoff table}
	
	\hypertarget{problem4-c3}{}
	
	Profit calculations (thousand USD):
	\begin{itemize}
		\item Standard: High $\to 900 \times 0.04 = 36$; Low $\to 550 \times 0.04 = 22$.
		\item Premium: High $\to 650 \times 0.07 = 45.5$; Low $\to 250 \times 0.07 = 17.5$.
	\end{itemize}
	\begin{center}
		\begin{tabular}{lll}
			\toprule
			Alternative & High demand
			($0.60$) & Low demand
			($0.40$) \\
			\midrule
			Standard packaging & 36 & 22 \\
			Premium packaging & 45.5 & 17.5 \\
			\bottomrule
		\end{tabular}
	\end{center}
	
	\subsection*{(C4) Applying the Maximax criterion}
	
	\hypertarget{problem4-c4}{}
	
	Maximax compares the best outcomes.
	\[
	\begin{aligned}
		\max_i \text{Payoff}(\text{Standard}, S_i) \&= \max\{36, 22\} = 36 \\
		\max_i \text{Payoff}(\text{Premium}, S_i) \&= \max\{45.5, 17.5\} = 45.5
	\end{aligned}
	\]
	\[
	\max\{36, 45.5\} = 45.5
	\]
	The maximax choice is \emph{premium packaging}.
	
	\subsection*{(C5) Applying the Maximin criterion}
	
	\hypertarget{problem4-c5}{}
	
	Maximin compares the worst outcomes.
	\[
	\begin{aligned}
		\min_i \text{Payoff}(\text{Standard}, S_i) \&= \min\{36, 22\} = 22 \\
		\min_i \text{Payoff}(\text{Premium}, S_i) \&= \min\{45.5, 17.5\} = 17.5
	\end{aligned}
	\]
	\[
	\max\{22, 17.5\} = 22
	\]
	The maximin recommendation is \emph{standard packaging}.
	
	\subsection*{(C1) Formulating the decision problem for an optimal choice using expected value}
	
	\hypertarget{problem4-c1}{}
	
	We select the pricing strategy that maximizes expected profit. Expected values:
	\[
	\begin{aligned}
		\mathrm{EV}_{\text{Standard}} \&= 0.60(36) + 0.40(22) = 21.6 + 8.8 = 30.4 \\
		\mathrm{EV}_{\text{Premium}} \&= 0.60(45.5) + 0.40(17.5) = 27.3 + 7 = 34.3
	\end{aligned}
	\]
	\[
	\max\{30.4, 34.3\} = 34.3
	\]
	
	Summary of recommendations across decision criteria
	\begin{itemize}
		\item Maximax favors \emph{premium packaging} (highest payoff of $45.5$).
		\item Maximin favors \emph{standard packaging} (best worst-case payoff of $22$).
		\item Expected value favors \emph{premium packaging} (EV of $34.3$).
	\end{itemize}
	
	Premium packaging leads on optimistic and average profits, while standard packaging limits downside risk.
	
	\section*{Problem 5 — Rural Internet Subscriptions}
	
	\hypertarget{problem5}{}
	
	\hypertarget{problem5-criteria}{}
	
	\begin{itemize}
		\item \hyperlink{problem5-problem}{Problem description}
		\item \hyperlink{problem5-c2}{(C2) Interpreting decision alternatives, events, consequences, and states}
		\item \hyperlink{problem5-c3}{(C3) Building the payoff table}
		\item \hyperlink{problem5-c4}{(C4) Applying the Maximax criterion}
		\item \hyperlink{problem5-c5}{(C5) Applying the Maximin criterion}
		\item \hyperlink{problem5-c1}{(C1) Formulating the decision problem for an optimal choice using expected value}
	\end{itemize}
	
	
	\subsection*{Problem description}
	
	\hypertarget{problem5-problem}{}
	
	A rural internet provider is selecting a subscription plan mix. The transaction is monthly broadband subscriptions,
	where profit is earned per subscription sold.
	
	
	
	
	
	
	Decision alternatives:
	
	
	
	
	
	
	• \textbf{Basic plan} with profit of $0.03$ thousand USD per subscription
	• \textbf{Plus plan} with profit of $0.05$ thousand USD per subscription
	• \textbf{Pro plan} with profit of $0.08$ thousand USD per subscription
	
	
	
	
	
	
	Enrollment states:
	• High enrollment ($0.55$)
	• Low enrollment ($0.45$)
	
	
	
	
	
	
	Expected subscriptions sold:
	• Basic $\to 1400$ (High), $900$ (Low)
	• Plus $\to 1000$ (High), $700$ (Low)
	• Pro $\to 600$ (High), $300$ (Low)
	
	
	
	
	
	\textbf{Question:} Compute profits and choose the best plan using Maximin, Expected Value, and Maximax.
	
	\subsection*{Solution}
	
	\hypertarget{problem5-solution}{}
	
	\paragraph{Structured solution}
	
	
	(C2) Interpreting decision alternatives, events, consequences, and states
	+
	
	
	\begin{center}
		\begin{tabular}{ll}
			\toprule
			Element & Description \\
			\midrule
			Alternatives & Basic plan; Plus plan; Pro plan \\
			States of nature & High enrollment
			($0.55$)
			Low enrollment
			($0.45$) \\
			Events & Actual subscription uptake after plan selection \\
			Consequences & Profit in thousand USD based on subscriptions $\times$ profit per subscription \\
			\bottomrule
		\end{tabular}
	\end{center}
	
	
	
	
	
	
	(C3) Building the payoff table
	+
	
	
	Profit calculations (thousand USD):
	\begin{itemize}
		\item Basic: High $\to 1400 \times 0.03 = 42$; Low $\to 900 \times 0.03 = 27$.
		\item Plus: High $\to 1000 \times 0.05 = 50$; Low $\to 700 \times 0.05 = 35$.
		\item Pro: High $\to 600 \times 0.08 = 48$; Low $\to 300 \times 0.08 = 24$.
	\end{itemize}
	
	\begin{center}
		\begin{tabular}{lll}
			\toprule
			Alternative & High enrollment
			($0.55$) & Low enrollment
			($0.45$) \\
			\midrule
			Basic plan & 42 & 27 \\
			Plus plan & 50 & 35 \\
			Pro plan & 48 & 24 \\
			\bottomrule
		\end{tabular}
	\end{center}
	
	
	
	
	
	
	(C4) Applying the Maximax criterion
	+
	
	
	Maximax compares the best outcomes.
	\[
	\begin{aligned}
		\max_i \text{Payoff}(\text{Basic}, S_i) \&= \max\{42, 27\} = 42 \\
		\max_i \text{Payoff}(\text{Plus}, S_i) \&= \max\{50, 35\} = 50 \\
		\max_i \text{Payoff}(\text{Pro}, S_i) \&= \max\{48, 24\} = 48
	\end{aligned}
	\]
	\[
	\max\{42, 50, 48\} = 50
	\]
	The maximax choice is the \emph{Plus plan}.
	
	
	
	
	
	(C5) Applying the Maximin criterion
	+
	
	
	Maximin compares the worst outcomes.
	\[
	\begin{aligned}
		\min_i \text{Payoff}(\text{Basic}, S_i) \&= \min\{42, 27\} = 27 \\
		\min_i \text{Payoff}(\text{Plus}, S_i) \&= \min\{50, 35\} = 35 \\
		\min_i \text{Payoff}(\text{Pro}, S_i) \&= \min\{48, 24\} = 24
	\end{aligned}
	\]
	\[
	\max\{27, 35, 24\} = 35
	\]
	The maximin recommendation is the \emph{Plus plan}.
	
	
	
	
	
	(C1) Formulating the decision problem for an optimal choice using expected value
	+
	
	
	We select the plan mix that maximizes expected profit. Expected values:
	\[
	\begin{aligned}
		\mathrm{EV}_{\text{Basic}} \&= 0.55(42) + 0.45(27) = 23.1 + 12.15 = 35.25 \\
		\mathrm{EV}_{\text{Plus}} \&= 0.55(50) + 0.45(35) = 27.5 + 15.75 = 43.25 \\
		\mathrm{EV}_{\text{Pro}} \&= 0.55(48) + 0.45(24) = 26.4 + 10.8 = 37.2
	\end{aligned}
	\]
	\[
	\max\{35.25, 43.25, 37.2\} = 43.25
	\]
	
	Summary of recommendations across decision criteria
	\begin{itemize}
		\item Maximax favors the \emph{Plus plan} (highest payoff of $50$).
		\item Maximin favors the \emph{Plus plan} (best worst-case payoff of $35$).
		\item Expected value favors the \emph{Plus plan} (EV of $43.25$).
	\end{itemize}
	
	
	The Plus plan balances a strong upside with the best downside protection and highest average profit.
	
	\subsection*{(C2) Interpreting decision alternatives, events, consequences, and states}
	
	\hypertarget{problem5-c2}{}
	
	\begin{center}
		\begin{tabular}{ll}
			\toprule
			Element & Description \\
			\midrule
			Alternatives & Basic plan; Plus plan; Pro plan \\
			States of nature & High enrollment
			($0.55$)
			Low enrollment
			($0.45$) \\
			Events & Actual subscription uptake after plan selection \\
			Consequences & Profit in thousand USD based on subscriptions $\times$ profit per subscription \\
			\bottomrule
		\end{tabular}
	\end{center}
	
	\subsection*{(C3) Building the payoff table}
	
	\hypertarget{problem5-c3}{}
	
	Profit calculations (thousand USD):
	\begin{itemize}
		\item Basic: High $\to 1400 \times 0.03 = 42$; Low $\to 900 \times 0.03 = 27$.
		\item Plus: High $\to 1000 \times 0.05 = 50$; Low $\to 700 \times 0.05 = 35$.
		\item Pro: High $\to 600 \times 0.08 = 48$; Low $\to 300 \times 0.08 = 24$.
	\end{itemize}
	\begin{center}
		\begin{tabular}{lll}
			\toprule
			Alternative & High enrollment
			($0.55$) & Low enrollment
			($0.45$) \\
			\midrule
			Basic plan & 42 & 27 \\
			Plus plan & 50 & 35 \\
			Pro plan & 48 & 24 \\
			\bottomrule
		\end{tabular}
	\end{center}
	
	\subsection*{(C4) Applying the Maximax criterion}
	
	\hypertarget{problem5-c4}{}
	
	Maximax compares the best outcomes.
	\[
	\begin{aligned}
		\max_i \text{Payoff}(\text{Basic}, S_i) \&= \max\{42, 27\} = 42 \\
		\max_i \text{Payoff}(\text{Plus}, S_i) \&= \max\{50, 35\} = 50 \\
		\max_i \text{Payoff}(\text{Pro}, S_i) \&= \max\{48, 24\} = 48
	\end{aligned}
	\]
	\[
	\max\{42, 50, 48\} = 50
	\]
	The maximax choice is the \emph{Plus plan}.
	
	\subsection*{(C5) Applying the Maximin criterion}
	
	\hypertarget{problem5-c5}{}
	
	Maximin compares the worst outcomes.
	\[
	\begin{aligned}
		\min_i \text{Payoff}(\text{Basic}, S_i) \&= \min\{42, 27\} = 27 \\
		\min_i \text{Payoff}(\text{Plus}, S_i) \&= \min\{50, 35\} = 35 \\
		\min_i \text{Payoff}(\text{Pro}, S_i) \&= \min\{48, 24\} = 24
	\end{aligned}
	\]
	\[
	\max\{27, 35, 24\} = 35
	\]
	The maximin recommendation is the \emph{Plus plan}.
	
	\subsection*{(C1) Formulating the decision problem for an optimal choice using expected value}
	
	\hypertarget{problem5-c1}{}
	
	We select the plan mix that maximizes expected profit. Expected values:
	\[
	\begin{aligned}
		\mathrm{EV}_{\text{Basic}} \&= 0.55(42) + 0.45(27) = 23.1 + 12.15 = 35.25 \\
		\mathrm{EV}_{\text{Plus}} \&= 0.55(50) + 0.45(35) = 27.5 + 15.75 = 43.25 \\
		\mathrm{EV}_{\text{Pro}} \&= 0.55(48) + 0.45(24) = 26.4 + 10.8 = 37.2
	\end{aligned}
	\]
	\[
	\max\{35.25, 43.25, 37.2\} = 43.25
	\]
	
	Summary of recommendations across decision criteria
	\begin{itemize}
		\item Maximax favors the \emph{Plus plan} (highest payoff of $50$).
		\item Maximin favors the \emph{Plus plan} (best worst-case payoff of $35$).
		\item Expected value favors the \emph{Plus plan} (EV of $43.25$).
	\end{itemize}
	
	The Plus plan balances a strong upside with the best downside protection and highest average profit.
	
	\section*{Problem 6 — Coffee Export Routes}
	
	\hypertarget{problem6}{}
	
	\hypertarget{problem6-criteria}{}
	
	\begin{itemize}
		\item \hyperlink{problem6-problem}{Problem description}
		\item \hyperlink{problem6-c2}{(C2) Interpreting decision alternatives, events, consequences, and states}
		\item \hyperlink{problem6-c3}{(C3) Building the payoff table}
		\item \hyperlink{problem6-c4}{(C4) Applying the Maximax criterion}
		\item \hyperlink{problem6-c5}{(C5) Applying the Maximin criterion}
		\item \hyperlink{problem6-c1}{(C1) Formulating the decision problem for an optimal choice using expected value}
	\end{itemize}
	
	
	\subsection*{Problem description}
	
	\hypertarget{problem6-problem}{}
	
	A coffee exporter must choose a shipping route. The transaction is exporting coffee shipments, where profit is earned
	per shipment delivered.
	
	
	
	
	
	
	Decision alternatives:
	• \textbf{Route through Port A}
	• \textbf{Route through Port B}
	
	
	
	
	Exchange-rate states:
	• Favorable rate ($0.50$)
	• Unfavorable rate ($0.50$)
	
	
	
	
	
	
	Profit per shipment depends on the exchange rate:
	• Favorable $\to 6$ thousand USD per shipment
	• Unfavorable $\to 2$ thousand USD per shipment
	
	
	
	
	
	
	Expected shipments:
	• Port A $\to 40$ (Favorable), $28$ (Unfavorable)
	• Port B $\to 32$ (Favorable), $30$ (Unfavorable)
	
	
	
	
	
	\textbf{Question:} Compute profits and choose the best route using Maximin, Expected Value, and Maximax.
	
	\subsection*{Solution}
	
	\hypertarget{problem6-solution}{}
	
	\paragraph{Structured solution}
	
	
	(C2) Interpreting decision alternatives, events, consequences, and states
	+
	
	
	\begin{center}
		\begin{tabular}{ll}
			\toprule
			Element & Description \\
			\midrule
			Alternatives & Port A route; Port B route \\
			States of nature & Favorable rate
			($0.50$)
			Unfavorable rate
			($0.50$) \\
			Events & Exchange-rate level that affects profit per shipment \\
			Consequences & Profit in thousand USD based on shipments $\times$ profit per shipment \\
			\bottomrule
		\end{tabular}
	\end{center}
	
	
	
	
	
	
	(C3) Building the payoff table
	+
	
	
	Profit calculations (thousand USD):
	\begin{itemize}
		\item Port A: Favorable $\to 40 \times 6 = 240$; Unfavorable $\to 28 \times 2 = 56$.
		\item Port B: Favorable $\to 32 \times 6 = 192$; Unfavorable $\to 30 \times 2 = 60$.
	\end{itemize}
	
	\begin{center}
		\begin{tabular}{lll}
			\toprule
			Route & Favorable rate
			($0.50$) & Unfavorable rate
			($0.50$) \\
			\midrule
			Port A & 240 & 56 \\
			Port B & 192 & 60 \\
			\bottomrule
		\end{tabular}
	\end{center}
	
	
	
	
	
	
	(C4) Applying the Maximax criterion
	+
	
	
	Maximax compares the best outcomes.
	\[
	\begin{aligned}
		\max_i \text{Payoff}(\text{Port A}, S_i) \&= \max\{240, 56\} = 240 \\
		\max_i \text{Payoff}(\text{Port B}, S_i) \&= \max\{192, 60\} = 192
	\end{aligned}
	\]
	\[
	\max\{240, 192\} = 240
	\]
	The maximax choice is \emph{routing through Port A}.
	
	
	
	
	
	(C5) Applying the Maximin criterion
	+
	
	
	Maximin compares the worst outcomes.
	\[
	\begin{aligned}
		\min_i \text{Payoff}(\text{Port A}, S_i) \&= \min\{240, 56\} = 56 \\
		\min_i \text{Payoff}(\text{Port B}, S_i) \&= \min\{192, 60\} = 60
	\end{aligned}
	\]
	\[
	\max\{56, 60\} = 60
	\]
	The maximin recommendation is \emph{routing through Port B}.
	
	
	
	
	
	(C1) Formulating the decision problem for an optimal choice using expected value
	+
	
	
	We choose the route that maximizes expected profit. Expected values:
	\[
	\begin{aligned}
		\mathrm{EV}_{\text{Port A}} \&= 0.50(240) + 0.50(56) = 120 + 28 = 148 \\
		\mathrm{EV}_{\text{Port B}} \&= 0.50(192) + 0.50(60) = 96 + 30 = 126
	\end{aligned}
	\]
	\[
	\max\{148, 126\} = 148
	\]
	
	Summary of recommendations across decision criteria
	\begin{itemize}
		\item Maximax favors \emph{Port A} (highest payoff of $240$).
		\item Maximin favors \emph{Port B} (best worst-case payoff of $60$).
		\item Expected value favors \emph{Port A} (EV of $148$).
	\end{itemize}
	
	
	Port A offers higher upside and average profit, while Port B provides slightly better downside protection.
	
	\subsection*{(C2) Interpreting decision alternatives, events, consequences, and states}
	
	\hypertarget{problem6-c2}{}
	
	\begin{center}
		\begin{tabular}{ll}
			\toprule
			Element & Description \\
			\midrule
			Alternatives & Port A route; Port B route \\
			States of nature & Favorable rate
			($0.50$)
			Unfavorable rate
			($0.50$) \\
			Events & Exchange-rate level that affects profit per shipment \\
			Consequences & Profit in thousand USD based on shipments $\times$ profit per shipment \\
			\bottomrule
		\end{tabular}
	\end{center}
	
	\subsection*{(C3) Building the payoff table}
	
	\hypertarget{problem6-c3}{}
	
	Profit calculations (thousand USD):
	\begin{itemize}
		\item Port A: Favorable $\to 40 \times 6 = 240$; Unfavorable $\to 28 \times 2 = 56$.
		\item Port B: Favorable $\to 32 \times 6 = 192$; Unfavorable $\to 30 \times 2 = 60$.
	\end{itemize}
	\begin{center}
		\begin{tabular}{lll}
			\toprule
			Route & Favorable rate
			($0.50$) & Unfavorable rate
			($0.50$) \\
			\midrule
			Port A & 240 & 56 \\
			Port B & 192 & 60 \\
			\bottomrule
		\end{tabular}
	\end{center}
	
	\subsection*{(C4) Applying the Maximax criterion}
	
	\hypertarget{problem6-c4}{}
	
	Maximax compares the best outcomes.
	\[
	\begin{aligned}
		\max_i \text{Payoff}(\text{Port A}, S_i) \&= \max\{240, 56\} = 240 \\
		\max_i \text{Payoff}(\text{Port B}, S_i) \&= \max\{192, 60\} = 192
	\end{aligned}
	\]
	\[
	\max\{240, 192\} = 240
	\]
	The maximax choice is \emph{routing through Port A}.
	
	\subsection*{(C5) Applying the Maximin criterion}
	
	\hypertarget{problem6-c5}{}
	
	Maximin compares the worst outcomes.
	\[
	\begin{aligned}
		\min_i \text{Payoff}(\text{Port A}, S_i) \&= \min\{240, 56\} = 56 \\
		\min_i \text{Payoff}(\text{Port B}, S_i) \&= \min\{192, 60\} = 60
	\end{aligned}
	\]
	\[
	\max\{56, 60\} = 60
	\]
	The maximin recommendation is \emph{routing through Port B}.
	
	\subsection*{(C1) Formulating the decision problem for an optimal choice using expected value}
	
	\hypertarget{problem6-c1}{}
	
	We choose the route that maximizes expected profit. Expected values:
	\[
	\begin{aligned}
		\mathrm{EV}_{\text{Port A}} \&= 0.50(240) + 0.50(56) = 120 + 28 = 148 \\
		\mathrm{EV}_{\text{Port B}} \&= 0.50(192) + 0.50(60) = 96 + 30 = 126
	\end{aligned}
	\]
	\[
	\max\{148, 126\} = 148
	\]
	
	Summary of recommendations across decision criteria
	\begin{itemize}
		\item Maximax favors \emph{Port A} (highest payoff of $240$).
		\item Maximin favors \emph{Port B} (best worst-case payoff of $60$).
		\item Expected value favors \emph{Port A} (EV of $148$).
	\end{itemize}
	
	Port A offers higher upside and average profit, while Port B provides slightly better downside protection.
	
	\section*{Problem 7 — Lake Ferry Tourism}
	
	\hypertarget{problem7}{}
	
	\hypertarget{problem7-criteria}{}
	
	\begin{itemize}
		\item \hyperlink{problem7-problem}{Problem description}
		\item \hyperlink{problem7-c2}{(C2) Interpreting decision alternatives, events, consequences, and states}
		\item \hyperlink{problem7-c3}{(C3) Building the payoff table}
		\item \hyperlink{problem7-c4}{(C4) Applying the Maximax criterion}
		\item \hyperlink{problem7-c5}{(C5) Applying the Maximin criterion}
		\item \hyperlink{problem7-c1}{(C1) Formulating the decision problem for an optimal choice using expected value}
	\end{itemize}
	
	
	\subsection*{Problem description}
	
	\hypertarget{problem7-problem}{}
	
	A lake tourism operator is choosing boat sizes for sightseeing tours. The transaction is tour ticket sales, where profit
	is earned per ticket sold.
	
	
	
	
	
	
	Alternatives:
	• \textbf{Operate small ferries}
	• \textbf{Operate large ferries}
	
	
	
	
	Seasonal states:
	• Peak season ($0.40$)
	• Normal season ($0.35$)
	• Rainy season ($0.25$)
	
	
	
	
	
	
	Profit per ticket depends on the season:
	• Peak $\to 0.05$ thousand USD
	• Normal $\to 0.04$ thousand USD
	• Rainy $\to 0.03$ thousand USD
	
	
	
	
	
	
	Expected tickets sold:
	• Small ferries $\to 2200$ (Peak), $1600$ (Normal), $1000$ (Rainy)
	• Large ferries $\to 2600$ (Peak), $1800$ (Normal), $800$ (Rainy)
	
	
	
	
	
	\textbf{Question:} Compute profits and choose the best fleet size using Maximin, Expected Value, and Maximax.
	
	\subsection*{Solution}
	
	\hypertarget{problem7-solution}{}
	
	\paragraph{Structured solution}
	
	
	(C2) Interpreting decision alternatives, events, consequences, and states
	+
	
	
	\begin{center}
		\begin{tabular}{ll}
			\toprule
			Element & Description \\
			\midrule
			Alternatives & Small ferries; Large ferries \\
			States of nature & Peak
			($0.40$)
			Normal
			($0.35$)
			Rainy
			($0.25$) \\
			Events & Tourism season and ticket price level \\
			Consequences & Profit in thousand USD based on tickets $\times$ profit per ticket \\
			\bottomrule
		\end{tabular}
	\end{center}
	
	
	
	
	
	
	(C3) Building the payoff table
	+
	
	
	Profit calculations (thousand USD):
	\begin{itemize}
		\item Small ferries: Peak $\to 2200 \times 0.05 = 110$; Normal $\to 1600 \times 0.04 = 64$; Rainy $\to 1000 \times 0.03 = 30$.
		\item Large ferries: Peak $\to 2600 \times 0.05 = 130$; Normal $\to 1800 \times 0.04 = 72$; Rainy $\to 800 \times 0.03 = 24$.
	\end{itemize}
	
	\begin{center}
		\begin{tabular}{llll}
			\toprule
			Alternative & Peak
			($0.40$) & Normal
			($0.35$) & Rainy
			($0.25$) \\
			\midrule
			Small ferries & 110 & 64 & 30 \\
			Large ferries & 130 & 72 & 24 \\
			\bottomrule
		\end{tabular}
	\end{center}
	
	
	
	
	
	
	(C4) Applying the Maximax criterion
	+
	
	
	Maximax compares the best outcomes.
	\[
	\begin{aligned}
		\max_i \text{Payoff}(\text{Small}, S_i) \&= \max\{110, 64, 30\} = 110 \\
		\max_i \text{Payoff}(\text{Large}, S_i) \&= \max\{130, 72, 24\} = 130
	\end{aligned}
	\]
	\[
	\max\{110, 130\} = 130
	\]
	The maximax choice is \emph{large ferries}.
	
	
	
	
	
	(C5) Applying the Maximin criterion
	+
	
	
	Maximin compares the worst outcomes.
	\[
	\begin{aligned}
		\min_i \text{Payoff}(\text{Small}, S_i) \&= \min\{110, 64, 30\} = 30 \\
		\min_i \text{Payoff}(\text{Large}, S_i) \&= \min\{130, 72, 24\} = 24
	\end{aligned}
	\]
	\[
	\max\{30, 24\} = 30
	\]
	The maximin recommendation is \emph{small ferries}.
	
	
	
	
	
	(C1) Formulating the decision problem for an optimal choice using expected value
	+
	
	
	We select the fleet size that maximizes expected profit. Expected values:
	\[
	\begin{aligned}
		\mathrm{EV}_{\text{Small}} \&= 0.40(110) + 0.35(64) + 0.25(30) = 44 + 22.4 + 7.5 = 73.9 \\
		\mathrm{EV}_{\text{Large}} \&= 0.40(130) + 0.35(72) + 0.25(24) = 52 + 25.2 + 6 = 83.2
	\end{aligned}
	\]
	\[
	\max\{73.9, 83.2\} = 83.2
	\]
	
	Summary of recommendations across decision criteria
	\begin{itemize}
		\item Maximax favors \emph{large ferries} (highest payoff of $130$).
		\item Maximin favors \emph{small ferries} (best worst-case payoff of $30$).
		\item Expected value favors \emph{large ferries} (EV of $83.2$).
	\end{itemize}
	
	
	Large ferries provide higher upside and average profit, but small ferries protect the rainy-season downside.
	
	\subsection*{(C2) Interpreting decision alternatives, events, consequences, and states}
	
	\hypertarget{problem7-c2}{}
	
	\begin{center}
		\begin{tabular}{ll}
			\toprule
			Element & Description \\
			\midrule
			Alternatives & Small ferries; Large ferries \\
			States of nature & Peak
			($0.40$)
			Normal
			($0.35$)
			Rainy
			($0.25$) \\
			Events & Tourism season and ticket price level \\
			Consequences & Profit in thousand USD based on tickets $\times$ profit per ticket \\
			\bottomrule
		\end{tabular}
	\end{center}
	
	\subsection*{(C3) Building the payoff table}
	
	\hypertarget{problem7-c3}{}
	
	Profit calculations (thousand USD):
	\begin{itemize}
		\item Small ferries: Peak $\to 2200 \times 0.05 = 110$; Normal $\to 1600 \times 0.04 = 64$; Rainy $\to 1000 \times 0.03 = 30$.
		\item Large ferries: Peak $\to 2600 \times 0.05 = 130$; Normal $\to 1800 \times 0.04 = 72$; Rainy $\to 800 \times 0.03 = 24$.
	\end{itemize}
	\begin{center}
		\begin{tabular}{llll}
			\toprule
			Alternative & Peak
			($0.40$) & Normal
			($0.35$) & Rainy
			($0.25$) \\
			\midrule
			Small ferries & 110 & 64 & 30 \\
			Large ferries & 130 & 72 & 24 \\
			\bottomrule
		\end{tabular}
	\end{center}
	
	\subsection*{(C4) Applying the Maximax criterion}
	
	\hypertarget{problem7-c4}{}
	
	Maximax compares the best outcomes.
	\[
	\begin{aligned}
		\max_i \text{Payoff}(\text{Small}, S_i) \&= \max\{110, 64, 30\} = 110 \\
		\max_i \text{Payoff}(\text{Large}, S_i) \&= \max\{130, 72, 24\} = 130
	\end{aligned}
	\]
	\[
	\max\{110, 130\} = 130
	\]
	The maximax choice is \emph{large ferries}.
	
	\subsection*{(C5) Applying the Maximin criterion}
	
	\hypertarget{problem7-c5}{}
	
	Maximin compares the worst outcomes.
	\[
	\begin{aligned}
		\min_i \text{Payoff}(\text{Small}, S_i) \&= \min\{110, 64, 30\} = 30 \\
		\min_i \text{Payoff}(\text{Large}, S_i) \&= \min\{130, 72, 24\} = 24
	\end{aligned}
	\]
	\[
	\max\{30, 24\} = 30
	\]
	The maximin recommendation is \emph{small ferries}.
	
	\subsection*{(C1) Formulating the decision problem for an optimal choice using expected value}
	
	\hypertarget{problem7-c1}{}
	
	We select the fleet size that maximizes expected profit. Expected values:
	\[
	\begin{aligned}
		\mathrm{EV}_{\text{Small}} \&= 0.40(110) + 0.35(64) + 0.25(30) = 44 + 22.4 + 7.5 = 73.9 \\
		\mathrm{EV}_{\text{Large}} \&= 0.40(130) + 0.35(72) + 0.25(24) = 52 + 25.2 + 6 = 83.2
	\end{aligned}
	\]
	\[
	\max\{73.9, 83.2\} = 83.2
	\]
	
	Summary of recommendations across decision criteria
	\begin{itemize}
		\item Maximax favors \emph{large ferries} (highest payoff of $130$).
		\item Maximin favors \emph{small ferries} (best worst-case payoff of $30$).
		\item Expected value favors \emph{large ferries} (EV of $83.2$).
	\end{itemize}
	
	Large ferries provide higher upside and average profit, but small ferries protect the rainy-season downside.
	
	\section*{Problem 8 — Cold-Storage Contracts}
	
	\hypertarget{problem8}{}
	
	\hypertarget{problem8-criteria}{}
	
	\begin{itemize}
		\item \hyperlink{problem8-problem}{Problem description}
		\item \hyperlink{problem8-c2}{(C2) Interpreting decision alternatives, events, consequences, and states}
		\item \hyperlink{problem8-c3}{(C3) Building the payoff table}
		\item \hyperlink{problem8-c4}{(C4) Applying the Maximax criterion}
		\item \hyperlink{problem8-c5}{(C5) Applying the Maximin criterion}
		\item \hyperlink{problem8-c1}{(C1) Formulating the decision problem for an optimal choice using expected value}
	\end{itemize}
	
	
	\subsection*{Problem description}
	
	\hypertarget{problem8-problem}{}
	
	A logistics company is leasing cold-storage space for agricultural pallets. The transaction is storage contracts,
	where profit is earned per pallet stored plus a fixed profit associated with the chosen facility.
	
	
	
	
	
	
	Decision alternatives:
	• \textbf{Small facility} (fixed profit $40$ thousand USD, profit $0.18$ thousand USD per pallet)
	• \textbf{Medium facility} (fixed profit $25$ thousand USD, profit $0.12$ thousand USD per pallet)
	• \textbf{Large facility} (fixed profit $10$ thousand USD, profit $0.08$ thousand USD per pallet)
	
	
	
	
	
	
	Demand states:
	• High storage demand ($0.55$)
	• Low storage demand ($0.45$)
	
	
	
	
	
	
	Expected pallets stored:
	• Small $\to 300$ (High), $220$ (Low)
	• Medium $\to 600$ (High), $420$ (Low)
	• Large $\to 900$ (High), $700$ (Low)
	
	
	
	
	
	\textbf{Question:} Compute profits and choose the best facility using Maximin, Expected Value, and Maximax.
	
	\subsection*{Solution}
	
	\hypertarget{problem8-solution}{}
	
	\paragraph{Structured solution}
	
	
	(C2) Interpreting decision alternatives, events, consequences, and states
	+
	
	
	\begin{center}
		\begin{tabular}{ll}
			\toprule
			Element & Description \\
			\midrule
			Alternatives & Small facility; Medium facility; Large facility \\
			States of nature & High demand
			($0.55$)
			Low demand
			($0.45$) \\
			Events & Storage demand realized after the lease decision \\
			Consequences & Profit in thousand USD $= \text{pallets} \times \text{profit per pallet} + \text{fixed profit}$ \\
			\bottomrule
		\end{tabular}
	\end{center}
	
	
	
	
	
	
	(C3) Building the payoff table
	+
	
	
	Profit calculations (thousand USD):
	\begin{itemize}
		\item Small: High $\to 300 \times 0.18 + 40 = 94$; Low $\to 220 \times 0.18 + 40 = 79.6$.
		\item Medium: High $\to 600 \times 0.12 + 25 = 97$; Low $\to 420 \times 0.12 + 25 = 75.4$.
		\item Large: High $\to 900 \times 0.08 + 10 = 82$; Low $\to 700 \times 0.08 + 10 = 66$.
	\end{itemize}
	
	\begin{center}
		\begin{tabular}{lll}
			\toprule
			Alternative & High demand
			($0.55$) & Low demand
			($0.45$) \\
			\midrule
			Small facility & 94 & 79.6 \\
			Medium facility & 97 & 75.4 \\
			Large facility & 82 & 66 \\
			\bottomrule
		\end{tabular}
	\end{center}
	
	
	
	
	
	
	(C4) Applying the Maximax criterion
	+
	
	
	Maximax compares the best outcomes.
	\[
	\begin{aligned}
		\max_i \text{Payoff}(\text{Small}, S_i) \&= \max\{94, 79.6\} = 94 \\
		\max_i \text{Payoff}(\text{Medium}, S_i) \&= \max\{97, 75.4\} = 97 \\
		\max_i \text{Payoff}(\text{Large}, S_i) \&= \max\{82, 66\} = 82
	\end{aligned}
	\]
	\[
	\max\{94, 97, 82\} = 97
	\]
	The maximax choice is the \emph{medium facility}.
	
	
	
	
	
	(C5) Applying the Maximin criterion
	+
	
	
	Maximin compares the worst outcomes.
	\[
	\begin{aligned}
		\min_i \text{Payoff}(\text{Small}, S_i) \&= \min\{94, 79.6\} = 79.6 \\
		\min_i \text{Payoff}(\text{Medium}, S_i) \&= \min\{97, 75.4\} = 75.4 \\
		\min_i \text{Payoff}(\text{Large}, S_i) \&= \min\{82, 66\} = 66
	\end{aligned}
	\]
	\[
	\max\{79.6, 75.4, 66\} = 79.6
	\]
	The maximin recommendation is the \emph{small facility}.
	
	
	
	
	
	(C1) Formulating the decision problem for an optimal choice using expected value
	+
	
	
	We select the facility that maximizes expected profit. Expected values:
	\[
	\begin{aligned}
		\mathrm{EV}_{\text{Small}} \&= 0.55(94) + 0.45(79.6) = 51.7 + 35.82 = 87.52 \\
		\mathrm{EV}_{\text{Medium}} \&= 0.55(97) + 0.45(75.4) = 53.35 + 33.93 = 87.28 \\
		\mathrm{EV}_{\text{Large}} \&= 0.55(82) + 0.45(66) = 45.1 + 29.7 = 74.8
	\end{aligned}
	\]
	\[
	\max\{87.52, 87.28, 74.8\} = 87.52
	\]
	
	Summary of recommendations across decision criteria
	\begin{itemize}
		\item Maximax favors the \emph{medium facility} (highest payoff of $97$).
		\item Maximin favors the \emph{small facility} (best worst-case payoff of $79.6$).
		\item Expected value favors the \emph{small facility} (EV of $87.52$).
	\end{itemize}
	
	
	The medium facility has the best upside, while the small facility provides the best downside protection and slightly
	higher expected profit.
	
	\subsection*{(C2) Interpreting decision alternatives, events, consequences, and states}
	
	\hypertarget{problem8-c2}{}
	
	\begin{center}
		\begin{tabular}{ll}
			\toprule
			Element & Description \\
			\midrule
			Alternatives & Small facility; Medium facility; Large facility \\
			States of nature & High demand
			($0.55$)
			Low demand
			($0.45$) \\
			Events & Storage demand realized after the lease decision \\
			Consequences & Profit in thousand USD $= \text{pallets} \times \text{profit per pallet} + \text{fixed profit}$ \\
			\bottomrule
		\end{tabular}
	\end{center}
	
	\subsection*{(C3) Building the payoff table}
	
	\hypertarget{problem8-c3}{}
	
	Profit calculations (thousand USD):
	\begin{itemize}
		\item Small: High $\to 300 \times 0.18 + 40 = 94$; Low $\to 220 \times 0.18 + 40 = 79.6$.
		\item Medium: High $\to 600 \times 0.12 + 25 = 97$; Low $\to 420 \times 0.12 + 25 = 75.4$.
		\item Large: High $\to 900 \times 0.08 + 10 = 82$; Low $\to 700 \times 0.08 + 10 = 66$.
	\end{itemize}
	\begin{center}
		\begin{tabular}{lll}
			\toprule
			Alternative & High demand
			($0.55$) & Low demand
			($0.45$) \\
			\midrule
			Small facility & 94 & 79.6 \\
			Medium facility & 97 & 75.4 \\
			Large facility & 82 & 66 \\
			\bottomrule
		\end{tabular}
	\end{center}
	
	\subsection*{(C4) Applying the Maximax criterion}
	
	\hypertarget{problem8-c4}{}
	
	Maximax compares the best outcomes.
	\[
	\begin{aligned}
		\max_i \text{Payoff}(\text{Small}, S_i) \&= \max\{94, 79.6\} = 94 \\
		\max_i \text{Payoff}(\text{Medium}, S_i) \&= \max\{97, 75.4\} = 97 \\
		\max_i \text{Payoff}(\text{Large}, S_i) \&= \max\{82, 66\} = 82
	\end{aligned}
	\]
	\[
	\max\{94, 97, 82\} = 97
	\]
	The maximax choice is the \emph{medium facility}.
	
	\subsection*{(C5) Applying the Maximin criterion}
	
	\hypertarget{problem8-c5}{}
	
	Maximin compares the worst outcomes.
	\[
	\begin{aligned}
		\min_i \text{Payoff}(\text{Small}, S_i) \&= \min\{94, 79.6\} = 79.6 \\
		\min_i \text{Payoff}(\text{Medium}, S_i) \&= \min\{97, 75.4\} = 75.4 \\
		\min_i \text{Payoff}(\text{Large}, S_i) \&= \min\{82, 66\} = 66
	\end{aligned}
	\]
	\[
	\max\{79.6, 75.4, 66\} = 79.6
	\]
	The maximin recommendation is the \emph{small facility}.
	
	\subsection*{(C1) Formulating the decision problem for an optimal choice using expected value}
	
	\hypertarget{problem8-c1}{}
	
	We select the facility that maximizes expected profit. Expected values:
	\[
	\begin{aligned}
		\mathrm{EV}_{\text{Small}} \&= 0.55(94) + 0.45(79.6) = 51.7 + 35.82 = 87.52 \\
		\mathrm{EV}_{\text{Medium}} \&= 0.55(97) + 0.45(75.4) = 53.35 + 33.93 = 87.28 \\
		\mathrm{EV}_{\text{Large}} \&= 0.55(82) + 0.45(66) = 45.1 + 29.7 = 74.8
	\end{aligned}
	\]
	\[
	\max\{87.52, 87.28, 74.8\} = 87.52
	\]
	
	Summary of recommendations across decision criteria
	\begin{itemize}
		\item Maximax favors the \emph{medium facility} (highest payoff of $97$).
		\item Maximin favors the \emph{small facility} (best worst-case payoff of $79.6$).
		\item Expected value favors the \emph{small facility} (EV of $87.52$).
	\end{itemize}
	
	The medium facility has the best upside, while the small facility provides the best downside protection and slightly
	higher expected profit.
	
	\section*{Problem 9 — Solar Installation Bids}
	
	\hypertarget{problem9}{}
	
	\hypertarget{problem9-criteria}{}
	
	
	\begin{itemize}
		\item \hyperlink{problem9-problem}{Problem description}
		\item \hyperlink{problem9-c2}{(C2) Interpreting decision alternatives, events, consequences, and states}
		\item \hyperlink{problem9-c3}{(C3) Building the payoff table}
		\item \hyperlink{problem9-c4}{(C4) Applying the Maximax criterion}
		\item \hyperlink{problem9-c5}{(C5) Applying the Maximin criterion}
		\item \hyperlink{problem9-c1}{(C1) Formulating the decision problem for an optimal choice using expected value}
	\end{itemize}
	
	
	\subsection*{Problem description}
	
	\hypertarget{problem9-problem}{}
	
	A solar company is bidding on installation contracts. The transaction is installing solar systems, where profit is earned
	per installation plus a fixed profit adjustment from policy conditions.
	
	
	
	
	
	
	Decision alternatives:
	• \textbf{Residential focus} (profit $0.9$ thousand USD per installation)
	• \textbf{Commercial focus} (profit $1.6$ thousand USD per installation)
	• \textbf{Mixed portfolio} (profit $1.2$ thousand USD per installation)
	
	
	
	
	
	
	Policy states with fixed profit components:
	• Rebate continues ($0.55$) $\to +20$ thousand USD
	• Rebate ends ($0.45$) $\to -10$ thousand USD
	
	
	
	
	
	
	Expected installations:
	• Residential $\to 80$ (Rebate), $50$ (No rebate)
	• Commercial $\to 55$ (Rebate), $40$ (No rebate)
	• Mixed $\to 70$ (Rebate), $45$ (No rebate)
	
	
	
	
	
	\textbf{Question:} Compute profits and choose the best bid strategy using Maximin, Expected Value, and Maximax.
	
	\subsection*{Solution}
	
	\hypertarget{problem9-solution}{}
	
	\paragraph{Structured solution}
	
	
	(C2) Interpreting decision alternatives, events, consequences, and states
	+
	
	
	\begin{center}
		\begin{tabular}{ll}
			\toprule
			Element & Description \\
			\midrule
			Alternatives & Residential; Commercial; Mixed portfolio \\
			States of nature & Rebate continues
			($0.55$)
			Rebate ends
			($0.45$) \\
			Events & Policy outcome affecting the fixed profit adjustment \\
			Consequences & Profit in thousand USD $= \text{installs} \times \text{profit per install} + \text{fixed profit}$ \\
			\bottomrule
		\end{tabular}
	\end{center}
	
	
	
	
	
	
	(C3) Building the payoff table
	+
	
	
	Profit calculations (thousand USD):
	\begin{itemize}
		\item Residential: Rebate $\to 80 \times 0.9 + 20 = 92$; No rebate $\to 50 \times 0.9 - 10 = 35$.
		\item Commercial: Rebate $\to 55 \times 1.6 + 20 = 108$; No rebate $\to 40 \times 1.6 - 10 = 54$.
		\item Mixed: Rebate $\to 70 \times 1.2 + 20 = 104$; No rebate $\to 45 \times 1.2 - 10 = 44$.
	\end{itemize}
	
	\begin{center}
		\begin{tabular}{lll}
			\toprule
			Alternative & Rebate continues
			($0.55$) & Rebate ends
			($0.45$) \\
			\midrule
			Residential focus & 92 & 35 \\
			Commercial focus & 108 & 54 \\
			Mixed portfolio & 104 & 44 \\
			\bottomrule
		\end{tabular}
	\end{center}
	
	
	
	
	
	
	(C4) Applying the Maximax criterion
	+
	
	
	Maximax compares the best outcomes.
	\[
	\begin{aligned}
		\max_i \text{Payoff}(\text{Residential}, S_i) \&= \max\{92, 35\} = 92 \\
		\max_i \text{Payoff}(\text{Commercial}, S_i) \&= \max\{108, 54\} = 108 \\
		\max_i \text{Payoff}(\text{Mixed}, S_i) \&= \max\{104, 44\} = 104
	\end{aligned}
	\]
	\[
	\max\{92, 108, 104\} = 108
	\]
	The maximax choice is the \emph{commercial focus}.
	
	
	
	
	
	(C5) Applying the Maximin criterion
	+
	
	
	Maximin compares the worst outcomes.
	\[
	\begin{aligned}
		\min_i \text{Payoff}(\text{Residential}, S_i) \&= \min\{92, 35\} = 35 \\
		\min_i \text{Payoff}(\text{Commercial}, S_i) \&= \min\{108, 54\} = 54 \\
		\min_i \text{Payoff}(\text{Mixed}, S_i) \&= \min\{104, 44\} = 44
	\end{aligned}
	\]
	\[
	\max\{35, 54, 44\} = 54
	\]
	The maximin recommendation is the \emph{commercial focus}.
	
	
	
	
	
	(C1) Formulating the decision problem for an optimal choice using expected value
	+
	
	
	We select the bid strategy that maximizes expected profit. Expected values:
	\[
	\begin{aligned}
		\mathrm{EV}_{\text{Residential}} \&= 0.55(92) + 0.45(35) = 50.6 + 15.75 = 66.35 \\
		\mathrm{EV}_{\text{Commercial}} \&= 0.55(108) + 0.45(54) = 59.4 + 24.3 = 83.7 \\
		\mathrm{EV}_{\text{Mixed}} \&= 0.55(104) + 0.45(44) = 57.2 + 19.8 = 77
	\end{aligned}
	\]
	\[
	\max\{66.35, 83.7, 77\} = 83.7
	\]
	
	Summary of recommendations across decision criteria
	\begin{itemize}
		\item Maximax favors the \emph{commercial focus} (highest payoff of $108$).
		\item Maximin favors the \emph{commercial focus} (best worst-case payoff of $54$).
		\item Expected value favors the \emph{commercial focus} (EV of $83.7$).
	\end{itemize}
	
	
	The commercial focus leads on upside, downside protection, and average profit when the rebate is uncertain.
	
	\subsection*{(C2) Interpreting decision alternatives, events, consequences, and states}
	
	\hypertarget{problem9-c2}{}
	
	\begin{center}
		\begin{tabular}{ll}
			\toprule
			Element & Description \\
			\midrule
			Alternatives & Residential; Commercial; Mixed portfolio \\
			States of nature & Rebate continues
			($0.55$)
			Rebate ends
			($0.45$) \\
			Events & Policy outcome affecting the fixed profit adjustment \\
			Consequences & Profit in thousand USD $= \text{installs} \times \text{profit per install} + \text{fixed profit}$ \\
			\bottomrule
		\end{tabular}
	\end{center}
	
	\subsection*{(C3) Building the payoff table}
	
	\hypertarget{problem9-c3}{}
	
	Profit calculations (thousand USD):
	\begin{itemize}
		\item Residential: Rebate $\to 80 \times 0.9 + 20 = 92$; No rebate $\to 50 \times 0.9 - 10 = 35$.
		\item Commercial: Rebate $\to 55 \times 1.6 + 20 = 108$; No rebate $\to 40 \times 1.6 - 10 = 54$.
		\item Mixed: Rebate $\to 70 \times 1.2 + 20 = 104$; No rebate $\to 45 \times 1.2 - 10 = 44$.
	\end{itemize}
	\begin{center}
		\begin{tabular}{lll}
			\toprule
			Alternative & Rebate continues
			($0.55$) & Rebate ends
			($0.45$) \\
			\midrule
			Residential focus & 92 & 35 \\
			Commercial focus & 108 & 54 \\
			Mixed portfolio & 104 & 44 \\
			\bottomrule
		\end{tabular}
	\end{center}
	
	\subsection*{(C4) Applying the Maximax criterion}
	
	\hypertarget{problem9-c4}{}
	
	Maximax compares the best outcomes.
	\[
	\begin{aligned}
		\max_i \text{Payoff}(\text{Residential}, S_i) \&= \max\{92, 35\} = 92 \\
		\max_i \text{Payoff}(\text{Commercial}, S_i) \&= \max\{108, 54\} = 108 \\
		\max_i \text{Payoff}(\text{Mixed}, S_i) \&= \max\{104, 44\} = 104
	\end{aligned}
	\]
	\[
	\max\{92, 108, 104\} = 108
	\]
	The maximax choice is the \emph{commercial focus}.
	
	\subsection*{(C5) Applying the Maximin criterion}
	
	\hypertarget{problem9-c5}{}
	
	Maximin compares the worst outcomes.
	\[
	\begin{aligned}
		\min_i \text{Payoff}(\text{Residential}, S_i) \&= \min\{92, 35\} = 35 \\
		\min_i \text{Payoff}(\text{Commercial}, S_i) \&= \min\{108, 54\} = 54 \\
		\min_i \text{Payoff}(\text{Mixed}, S_i) \&= \min\{104, 44\} = 44
	\end{aligned}
	\]
	\[
	\max\{35, 54, 44\} = 54
	\]
	The maximin recommendation is the \emph{commercial focus}.
	
	\subsection*{(C1) Formulating the decision problem for an optimal choice using expected value}
	
	\hypertarget{problem9-c1}{}
	
	We select the bid strategy that maximizes expected profit. Expected values:
	\[
	\begin{aligned}
		\mathrm{EV}_{\text{Residential}} \&= 0.55(92) + 0.45(35) = 50.6 + 15.75 = 66.35 \\
		\mathrm{EV}_{\text{Commercial}} \&= 0.55(108) + 0.45(54) = 59.4 + 24.3 = 83.7 \\
		\mathrm{EV}_{\text{Mixed}} \&= 0.55(104) + 0.45(44) = 57.2 + 19.8 = 77
	\end{aligned}
	\]
	\[
	\max\{66.35, 83.7, 77\} = 83.7
	\]
	
	Summary of recommendations across decision criteria
	\begin{itemize}
		\item Maximax favors the \emph{commercial focus} (highest payoff of $108$).
		\item Maximin favors the \emph{commercial focus} (best worst-case payoff of $54$).
		\item Expected value favors the \emph{commercial focus} (EV of $83.7$).
	\end{itemize}
	
	The commercial focus leads on upside, downside protection, and average profit when the rebate is uncertain.
	
	\section*{Problem 10 — Air Cargo Charters}
	
	\hypertarget{problem10}{}
	
	\hypertarget{problem10-criteria}{}

	
	\begin{itemize}
		\item \hyperlink{problem10-problem}{Problem description}
		\item \hyperlink{problem10-c2}{(C2) Interpreting decision alternatives, events, consequences, and states}
		\item \hyperlink{problem10-c3}{(C3) Building the payoff table}
		\item \hyperlink{problem10-c4}{(C4) Applying the Maximax criterion}
		\item \hyperlink{problem10-c5}{(C5) Applying the Maximin criterion}
		\item \hyperlink{problem10-c1}{(C1) Formulating the decision problem for an optimal choice using expected value}
	\end{itemize}
	
	
	\subsection*{Problem description}
	
	\hypertarget{problem10-problem}{}
	
	An air cargo firm is choosing a charter fleet. The transaction is cargo charters, where profit is earned per charter
	plus a fixed profit component tied to the chosen aircraft lease.
	
	
	
	
	
	
	Alternatives:
	• \textbf{Lease narrow-body aircraft} (fixed profit $30$ thousand USD)
	• \textbf{Lease wide-body aircraft} (fixed profit $50$ thousand USD)
	
	
	
	
	
	
	Fuel-price states:
	• Low fuel cost ($0.30$)
	• Medium fuel cost ($0.45$)
	• High fuel cost ($0.25$)
	
	
	
	
	
	
	Profit per charter depends on fuel prices:
	• Low fuel $\to 8$ thousand USD
	• Medium fuel $\to 5$ thousand USD
	• High fuel $\to 2$ thousand USD
	
	
	
	
	
	
	Expected charters:
	• Narrow-body $\to 20$ (Low), $18$ (Medium), $15$ (High)
	• Wide-body $\to 14$ (Low), $12$ (Medium), $9$ (High)
	
	
	
	
	
	\textbf{Question:} Compute profits and choose the best lease using Maximin, Expected Value, and Maximax.
	
	\subsection*{Solution}
	
	\hypertarget{problem10-solution}{}
	
	\paragraph{Structured solution}
	
	
	(C2) Interpreting decision alternatives, events, consequences, and states
	+
	
	
	\begin{center}
		\begin{tabular}{ll}
			\toprule
			Element & Description \\
			\midrule
			Alternatives & Narrow-body lease; Wide-body lease \\
			States of nature & Low fuel
			($0.30$)
			Medium fuel
			($0.45$)
			High fuel
			($0.25$) \\
			Events & Fuel prices affecting profit per charter \\
			Consequences & Profit in thousand USD $= \text{charters} \times \text{profit per charter} + \text{fixed profit}$ \\
			\bottomrule
		\end{tabular}
	\end{center}
	
	
	
	
	
	
	(C3) Building the payoff table
	+
	
	
	Profit calculations (thousand USD):
	\begin{itemize}
		\item Narrow-body: Low $\to 20 \times 8 + 30 = 190$; Medium $\to 18 \times 5 + 30 = 120$; High $\to 15 \times 2 + 30 = 60$.
		\item Wide-body: Low $\to 14 \times 8 + 50 = 162$; Medium $\to 12 \times 5 + 50 = 110$; High $\to 9 \times 2 + 50 = 68$.
	\end{itemize}
	
	\begin{center}
		\begin{tabular}{llll}
			\toprule
			Alternative & Low fuel
			($0.30$) & Medium fuel
			($0.45$) & High fuel
			($0.25$) \\
			\midrule
			Narrow-body lease & 190 & 120 & 60 \\
			Wide-body lease & 162 & 110 & 68 \\
			\bottomrule
		\end{tabular}
	\end{center}
	
	
	
	
	
	
	(C4) Applying the Maximax criterion
	+
	
	
	Maximax compares the best outcomes.
	\[
	\begin{aligned}
		\max_i \text{Payoff}(\text{Narrow}, S_i) \&= \max\{190, 120, 60\} = 190 \\
		\max_i \text{Payoff}(\text{Wide}, S_i) \&= \max\{162, 110, 68\} = 162
	\end{aligned}
	\]
	\[
	\max\{190, 162\} = 190
	\]
	The maximax choice is the \emph{narrow-body lease}.
	
	
	
	
	
	(C5) Applying the Maximin criterion
	+
	
	
	Maximin compares the worst outcomes.
	\[
	\begin{aligned}
		\min_i \text{Payoff}(\text{Narrow}, S_i) \&= \min\{190, 120, 60\} = 60 \\
		\min_i \text{Payoff}(\text{Wide}, S_i) \&= \min\{162, 110, 68\} = 68
	\end{aligned}
	\]
	\[
	\max\{60, 68\} = 68
	\]
	The maximin recommendation is the \emph{wide-body lease}.
	
	
	
	
	
	(C1) Formulating the decision problem for an optimal choice using expected value
	+
	
	
	We select the lease that maximizes expected profit. Expected values:
	\[
	\begin{aligned}
		\mathrm{EV}_{\text{Narrow}} \&= 0.30(190) + 0.45(120) + 0.25(60) = 57 + 54 + 15 = 126 \\
		\mathrm{EV}_{\text{Wide}} \&= 0.30(162) + 0.45(110) + 0.25(68) = 48.6 + 49.5 + 17 = 115.1
	\end{aligned}
	\]
	\[
	\max\{126, 115.1\} = 126
	\]
	
	Summary of recommendations across decision criteria
	\begin{itemize}
		\item Maximax favors the \emph{narrow-body lease} (highest payoff of $190$).
		\item Maximin favors the \emph{wide-body lease} (best worst-case payoff of $68$).
		\item Expected value favors the \emph{narrow-body lease} (EV of $126$).
	\end{itemize}
	
	
	Narrow-body aircraft provide stronger upside and expected profit, while wide-body leases offer better downside
	protection when fuel costs spike.
	
	\subsection*{(C2) Interpreting decision alternatives, events, consequences, and states}
	
	\hypertarget{problem10-c2}{}
	
	\begin{center}
		\begin{tabular}{ll}
			\toprule
			Element & Description \\
			\midrule
			Alternatives & Narrow-body lease; Wide-body lease \\
			States of nature & Low fuel
			($0.30$)
			Medium fuel
			($0.45$)
			High fuel
			($0.25$) \\
			Events & Fuel prices affecting profit per charter \\
			Consequences & Profit in thousand USD $= \text{charters} \times \text{profit per charter} + \text{fixed profit}$ \\
			\bottomrule
		\end{tabular}
	\end{center}
	
	\subsection*{(C3) Building the payoff table}
	
	\hypertarget{problem10-c3}{}
	
	Profit calculations (thousand USD):
	\begin{itemize}
		\item Narrow-body: Low $\to 20 \times 8 + 30 = 190$; Medium $\to 18 \times 5 + 30 = 120$; High $\to 15 \times 2 + 30 = 60$.
		\item Wide-body: Low $\to 14 \times 8 + 50 = 162$; Medium $\to 12 \times 5 + 50 = 110$; High $\to 9 \times 2 + 50 = 68$.
	\end{itemize}
	\begin{center}
		\begin{tabular}{llll}
			\toprule
			Alternative & Low fuel
			($0.30$) & Medium fuel
			($0.45$) & High fuel
			($0.25$) \\
			\midrule
			Narrow-body lease & 190 & 120 & 60 \\
			Wide-body lease & 162 & 110 & 68 \\
			\bottomrule
		\end{tabular}
	\end{center}
	
	\subsection*{(C4) Applying the Maximax criterion}
	
	\hypertarget{problem10-c4}{}
	
	Maximax compares the best outcomes.
	\[
	\begin{aligned}
		\max_i \text{Payoff}(\text{Narrow}, S_i) \&= \max\{190, 120, 60\} = 190 \\
		\max_i \text{Payoff}(\text{Wide}, S_i) \&= \max\{162, 110, 68\} = 162
	\end{aligned}
	\]
	\[
	\max\{190, 162\} = 190
	\]
	The maximax choice is the \emph{narrow-body lease}.
	
	\subsection*{(C5) Applying the Maximin criterion}
	
	\hypertarget{problem10-c5}{}
	
	Maximin compares the worst outcomes.
	\[
	\begin{aligned}
		\min_i \text{Payoff}(\text{Narrow}, S_i) \&= \min\{190, 120, 60\} = 60 \\
		\min_i \text{Payoff}(\text{Wide}, S_i) \&= \min\{162, 110, 68\} = 68
	\end{aligned}
	\]
	\[
	\max\{60, 68\} = 68
	\]
	The maximin recommendation is the \emph{wide-body lease}.
	
	\subsection*{(C1) Formulating the decision problem for an optimal choice using expected value}
	
	\hypertarget{problem10-c1}{}
	
	We select the lease that maximizes expected profit. Expected values:
	\[
	\begin{aligned}
		\mathrm{EV}_{\text{Narrow}} \&= 0.30(190) + 0.45(120) + 0.25(60) = 57 + 54 + 15 = 126 \\
		\mathrm{EV}_{\text{Wide}} \&= 0.30(162) + 0.45(110) + 0.25(68) = 48.6 + 49.5 + 17 = 115.1
	\end{aligned}
	\]
	\[
	\max\{126, 115.1\} = 126
	\]
	
	Summary of recommendations across decision criteria
	\begin{itemize}
		\item Maximax favors the \emph{narrow-body lease} (highest payoff of $190$).
		\item Maximin favors the \emph{wide-body lease} (best worst-case payoff of $68$).
		\item Expected value favors the \emph{narrow-body lease} (EV of $126$).
	\end{itemize}
	
	Narrow-body aircraft provide stronger upside and expected profit, while wide-body leases offer better downside
	protection when fuel costs spike.
	
	\section*{Problem 11 — Seafood Export Licenses}
	
	\hypertarget{problem11}{}
	
	\hypertarget{problem11-criteria}{}
	
	
	\begin{itemize}
		\item \hyperlink{problem11-problem}{Problem description}
		\item \hyperlink{problem11-c2}{(C2) Interpreting decision alternatives, events, consequences, and states}
		\item \hyperlink{problem11-c3}{(C3) Building the payoff table}
		\item \hyperlink{problem11-c4}{(C4) Applying the Maximax criterion}
		\item \hyperlink{problem11-c5}{(C5) Applying the Maximin criterion}
		\item \hyperlink{problem11-c1}{(C1) Formulating the decision problem for an optimal choice using expected value}
	\end{itemize}
	
	
	\subsection*{Problem description}
	
	\hypertarget{problem11-problem}{}
	
	A seafood exporter is choosing a license focus. The transaction is exporting seafood crates, where profit is earned per
	crate plus a fixed profit adjustment tied to seasonal conditions.
	
	
	
	
	
	
	Alternatives:
	• \textbf{Frozen fillet focus}
	• \textbf{Fresh export focus}
	
	
	
	
	Seasonal states with fixed profit components:
	• Strong tourism season ($0.40$) $\to +30$ thousand USD
	• Normal season ($0.35$) $\to +10$ thousand USD
	• Stormy season ($0.25$) $\to -15$ thousand USD
	
	
	
	
	
	
	Profit per crate depends on the season:
	• Strong $\to 0.7$ thousand USD
	• Normal $\to 0.5$ thousand USD
	• Stormy $\to 0.2$ thousand USD
	
	
	
	
	
	
	Expected crates exported:
	• Frozen fillet $\to 120$ (Strong), $100$ (Normal), $80$ (Stormy)
	• Fresh export $\to 90$ (Strong), $110$ (Normal), $70$ (Stormy)
	
	
	
	
	
	\textbf{Question:} Compute profits and choose the best focus using Maximin, Expected Value, and Maximax.
	
	\subsection*{Solution}
	
	\hypertarget{problem11-solution}{}
	
	\paragraph{Structured solution}
	
	
	(C2) Interpreting decision alternatives, events, consequences, and states
	+
	
	
	\begin{center}
		\begin{tabular}{ll}
			\toprule
			Element & Description \\
			\midrule
			Alternatives & Frozen fillet focus; Fresh export focus \\
			States of nature & Strong
			($0.40$)
			Normal
			($0.35$)
			Stormy
			($0.25$) \\
			Events & Seasonal conditions affecting prices and fixed profit adjustments \\
			Consequences & Profit in thousand USD $= \text{crates} \times \text{profit per crate} + \text{fixed profit}$ \\
			\bottomrule
		\end{tabular}
	\end{center}
	
	
	
	
	
	
	(C3) Building the payoff table
	+
	
	
	Profit calculations (thousand USD):
	\begin{itemize}
		\item Frozen fillet: Strong $\to 120 \times 0.7 + 30 = 114$; Normal $\to 100 \times 0.5 + 10 = 60$; Stormy $\to 80 \times 0.2 - 15 = 1$.
		\item Fresh export: Strong $\to 90 \times 0.7 + 30 = 93$; Normal $\to 110 \times 0.5 + 10 = 65$; Stormy $\to 70 \times 0.2 - 15 = -1$.
	\end{itemize}
	
	\begin{center}
		\begin{tabular}{llll}
			\toprule
			Alternative & Strong
			($0.40$) & Normal
			($0.35$) & Stormy
			($0.25$) \\
			\midrule
			Frozen fillet focus & 114 & 60 & 1 \\
			Fresh export focus & 93 & 65 & -1 \\
			\bottomrule
		\end{tabular}
	\end{center}
	
	
	
	
	
	
	(C4) Applying the Maximax criterion
	+
	
	
	Maximax compares the best outcomes.
	\[
	\begin{aligned}
		\max_i \text{Payoff}(\text{Frozen}, S_i) \&= \max\{114, 60, 1\} = 114 \\
		\max_i \text{Payoff}(\text{Fresh}, S_i) \&= \max\{93, 65, -1\} = 93
	\end{aligned}
	\]
	\[
	\max\{114, 93\} = 114
	\]
	The maximax choice is the \emph{frozen fillet focus}.
	
	
	
	
	
	(C5) Applying the Maximin criterion
	+
	
	
	Maximin compares the worst outcomes.
	\[
	\begin{aligned}
		\min_i \text{Payoff}(\text{Frozen}, S_i) \&= \min\{114, 60, 1\} = 1 \\
		\min_i \text{Payoff}(\text{Fresh}, S_i) \&= \min\{93, 65, -1\} = -1
	\end{aligned}
	\]
	\[
	\max\{1, -1\} = 1
	\]
	The maximin recommendation is the \emph{frozen fillet focus}.
	
	
	
	
	
	(C1) Formulating the decision problem for an optimal choice using expected value
	+
	
	
	We select the license focus that maximizes expected profit. Expected values:
	\[
	\begin{aligned}
		\mathrm{EV}_{\text{Frozen}} \&= 0.40(114) + 0.35(60) + 0.25(1) = 45.6 + 21 + 0.25 = 66.85 \\
		\mathrm{EV}_{\text{Fresh}} \&= 0.40(93) + 0.35(65) + 0.25(-1) = 37.2 + 22.75 - 0.25 = 59.7
	\end{aligned}
	\]
	\[
	\max\{66.85, 59.7\} = 66.85
	\]
	
	Summary of recommendations across decision criteria
	\begin{itemize}
		\item Maximax favors the \emph{frozen fillet focus} (highest payoff of $114$).
		\item Maximin favors the \emph{frozen fillet focus} (best worst-case payoff of $1$).
		\item Expected value favors the \emph{frozen fillet focus} (EV of $66.85$).
	\end{itemize}
	
	
	Frozen exports provide stronger upside and downside protection, leading to the highest average profit.
	
	\subsection*{(C2) Interpreting decision alternatives, events, consequences, and states}
	
	\hypertarget{problem11-c2}{}
	
	\begin{center}
		\begin{tabular}{ll}
			\toprule
			Element & Description \\
			\midrule
			Alternatives & Frozen fillet focus; Fresh export focus \\
			States of nature & Strong
			($0.40$)
			Normal
			($0.35$)
			Stormy
			($0.25$) \\
			Events & Seasonal conditions affecting prices and fixed profit adjustments \\
			Consequences & Profit in thousand USD $= \text{crates} \times \text{profit per crate} + \text{fixed profit}$ \\
			\bottomrule
		\end{tabular}
	\end{center}
	
	\subsection*{(C3) Building the payoff table}
	
	\hypertarget{problem11-c3}{}
	
	Profit calculations (thousand USD):
	\begin{itemize}
		\item Frozen fillet: Strong $\to 120 \times 0.7 + 30 = 114$; Normal $\to 100 \times 0.5 + 10 = 60$; Stormy $\to 80 \times 0.2 - 15 = 1$.
		\item Fresh export: Strong $\to 90 \times 0.7 + 30 = 93$; Normal $\to 110 \times 0.5 + 10 = 65$; Stormy $\to 70 \times 0.2 - 15 = -1$.
	\end{itemize}
	\begin{center}
		\begin{tabular}{llll}
			\toprule
			Alternative & Strong
			($0.40$) & Normal
			($0.35$) & Stormy
			($0.25$) \\
			\midrule
			Frozen fillet focus & 114 & 60 & 1 \\
			Fresh export focus & 93 & 65 & -1 \\
			\bottomrule
		\end{tabular}
	\end{center}
	
	\subsection*{(C4) Applying the Maximax criterion}
	
	\hypertarget{problem11-c4}{}
	
	Maximax compares the best outcomes.
	\[
	\begin{aligned}
		\max_i \text{Payoff}(\text{Frozen}, S_i) \&= \max\{114, 60, 1\} = 114 \\
		\max_i \text{Payoff}(\text{Fresh}, S_i) \&= \max\{93, 65, -1\} = 93
	\end{aligned}
	\]
	\[
	\max\{114, 93\} = 114
	\]
	The maximax choice is the \emph{frozen fillet focus}.
	
	\subsection*{(C5) Applying the Maximin criterion}
	
	\hypertarget{problem11-c5}{}
	
	Maximin compares the worst outcomes.
	\[
	\begin{aligned}
		\min_i \text{Payoff}(\text{Frozen}, S_i) \&= \min\{114, 60, 1\} = 1 \\
		\min_i \text{Payoff}(\text{Fresh}, S_i) \&= \min\{93, 65, -1\} = -1
	\end{aligned}
	\]
	\[
	\max\{1, -1\} = 1
	\]
	The maximin recommendation is the \emph{frozen fillet focus}.
	
	\subsection*{(C1) Formulating the decision problem for an optimal choice using expected value}
	
	\hypertarget{problem11-c1}{}
	
	We select the license focus that maximizes expected profit. Expected values:
	\[
	\begin{aligned}
		\mathrm{EV}_{\text{Frozen}} \&= 0.40(114) + 0.35(60) + 0.25(1) = 45.6 + 21 + 0.25 = 66.85 \\
		\mathrm{EV}_{\text{Fresh}} \&= 0.40(93) + 0.35(65) + 0.25(-1) = 37.2 + 22.75 - 0.25 = 59.7
	\end{aligned}
	\]
	\[
	\max\{66.85, 59.7\} = 66.85
	\]
	
	Summary of recommendations across decision criteria
	\begin{itemize}
		\item Maximax favors the \emph{frozen fillet focus} (highest payoff of $114$).
		\item Maximin favors the \emph{frozen fillet focus} (best worst-case payoff of $1$).
		\item Expected value favors the \emph{frozen fillet focus} (EV of $66.85$).
	\end{itemize}
	
	Frozen exports provide stronger upside and downside protection, leading to the highest average profit.
	
\end{document}
