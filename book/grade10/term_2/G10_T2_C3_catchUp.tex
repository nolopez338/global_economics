\documentclass[12pt]{article}

% Page size and tighter margins
\usepackage[a4paper,left=1.2cm,right=1.2cm,top=1.5cm,bottom=1.5cm]{geometry}

% Core packages
\usepackage{graphicx}
\usepackage{xcolor}
\usepackage{array}
\usepackage{tabularx}
\usepackage{multicol}
\usepackage{amsmath}
\usepackage[T1]{fontenc}
\usepackage[utf8]{inputenc}
\usepackage{booktabs}

\setlength{\parindent}{0pt}
\setlength{\tabcolsep}{6pt}
\renewcommand{\arraystretch}{1.15}

% Column types
\newcolumntype{Y}{>{\raggedright\arraybackslash}m{\dimexpr0.30\textwidth-2\tabcolsep-2\arrayrulewidth\relax}}
\newcolumntype{Z}{>{\raggedright\arraybackslash}m{\dimexpr0.70\textwidth-2\tabcolsep-2\arrayrulewidth\relax}}
\newcolumntype{C}[1]{>{\centering\arraybackslash}m{#1}}

% Gray subsection header box
\newcommand{\SubsectionBox}[1]{%
	\noindent\colorbox{gray!30}{%
		\parbox{\linewidth}{\textbf{#1}}%
	}\par\vspace{0.35cm}%
}

% Centered multi-line cell helper
\newcommand{\CellCenter}[1]{%
	\parbox{\linewidth}{\centering #1}%
}

\begin{document}

	\noindent
	\begin{tabularx}{\textwidth}{|C{2.8cm}|C{\dimexpr\textwidth-6cm-4\tabcolsep-4\arrayrulewidth\relax}|C{2.8cm}|}
		\hline
		\centering
		\vspace{3mm}
		\includegraphics[width=2.5cm]{../../preamble/logo.png}
		&
		\CellCenter{%
			\vspace{-5mm}
			\textbf{GLOBAL ECONOMICS}\par
			\textbf{GRADE: 10TH}\par
			\textbf{CATCH-UP ACTIVITY}\par
			\textbf{ANALYSIS OF DECISIONS}\par
			\textbf{TEACHER'S NAME: Nicolás López Cuéllar}
		}
		&
		\CellCenter{%
			\textbf{SECOND TERM}\par
			\textbf{2025--2026}%
		}
		\\
		\hline
	\end{tabularx}

	\vspace{0.5cm}

	\noindent
	\begin{tabular}{|Y|Z|}
		\hline
		{\small
			\textbf{Learning objective:} Construct payoff tables from narrative decision problems by organizing alternatives, states of nature, and consequences.
		}
		&
		{\footnotesize
			\textbf{Assessment criteria:}\par
			C3: Builds a payoff table from a description of the problem.\par
		}
		\\
		\hline
	\end{tabular}

	\vspace{0.4cm}

	\begin{multicols}{2}
		\SubsectionBox{Criteria assessment}\vspace{-0.25cm}
		This activity evaluates criterion C3. Read each problem and complete the required decision-analysis tasks.

		\vspace{0.25cm}
		\SubsectionBox{1. Problem description}\vspace{-0.25cm}
		A neighborhood juice bar must choose one of two monthly operating formats: Format A (premium menu)
				or Format B (standard menu). Market demand for the month can be high or low.
				
				The projected payoff values are already net profits in thousands of dollars, so no separate revenue or cost calculation is required.
				Under high demand, the estimated profit is 96 for Format A and 78 for Format B. Under low demand, the estimated profit is 18 for Format A and 36 for Format B.
				
				Management must construct the payoff table using these direct profit entries for each format--demand combination.
				\subsection*{Given data}
				\begin{center}
					\begin{tabular}{l c c}
						\toprule
						Alternative & High demand & Low demand \\
						\midrule
						Format A & 96 & 18 \\
						Format B & 78 & 36 \\
						\bottomrule
					\end{tabular}
				\end{center}

		\vspace{0.25cm}
		\SubsectionBox{2. Problem description}\vspace{-0.25cm}
		A retail snack distributor is selecting one of three inventory strategies for a one-week city event: Strategy A (premium assortment),
				Strategy B (balanced assortment), or Strategy C (economy assortment). Demand can be strong, moderate, or weak.
				
				All numerical values are already estimated profits in thousands of dollars.
				For strong demand, profits are 140 (A), 122 (B), and 105 (C). For moderate demand, profits are 88 (A), 94 (B), and 76 (C).
				For weak demand, profits are 12 (A), 38 (B), and 49 (C).
				
				The firm must construct the payoff table directly from these profit outcomes for each strategy and demand state.
				\subsection*{Given data}
				\begin{center}
					\begin{tabular}{l c c c}
						\toprule
						Alternative & Strong demand & Moderate demand & Weak demand \\
						\midrule
						Strategy A & 140 & 88 & 12 \\
						Strategy B & 122 & 94 & 38 \\
						Strategy C & 105 & 76 & 49 \\
						\bottomrule
					\end{tabular}
				\end{center}

		\vspace{0.25cm}
		\SubsectionBox{3. Problem description}\vspace{-0.25cm}
		A local garment manufacturer must choose between two production plans for the next season: Plan A (premium shirts)
				or Plan B (basic shirts). Market demand can be very high, high, medium, or low.
				
				For each plan, the per-unit amount is a profit contribution per shirt: Plan A earns \$18 per shirt and Plan B earns \$14 per shirt.
				The state-dependent quantities are expected shirt sales volumes:
				\begin{itemize}
					\item Very high demand: Plan A sells 9{,}000 shirts, Plan B sells 10{,}200 shirts.
					\item High demand: Plan A sells 7{,}200 shirts, Plan B sells 8{,}300 shirts.
					\item Medium demand: Plan A sells 5{,}000 shirts, Plan B sells 6{,}000 shirts.
					\item Low demand: Plan A sells 2{,}600 shirts, Plan B sells 3{,}400 shirts.
				\end{itemize}
				
				The manufacturer must first compute profit for each state using
				\(\text{Profit}=\text{Quantity sold}\times\text{Profit per unit}\), then construct the payoff table in thousands of dollars.
				\subsection*{Given data}
				\begin{center}
					\begin{tabular}{l c c c c}
						\toprule
						Alternative & Very high & High & Medium & Low \\
						\midrule
						Plan A quantity & 9{,}000 & 7{,}200 & 5{,}000 & 2{,}600 \\
						Plan B quantity & 10{,}200 & 8{,}300 & 6{,}000 & 3{,}400 \\
						\bottomrule
					\end{tabular}
				\end{center}
				Profit per unit: Plan A = \$18, Plan B = \$14.

		\vspace{0.25cm}
		\SubsectionBox{4. Problem description}\vspace{-0.25cm}
		A regional publisher must choose among three magazine distribution models for the coming month: Model A (city kiosks),
				Model B (subscription contracts), or Model C (mixed channel). Demand can be high or low.
				
				For each model, the fixed component is a monthly lump-sum profit contribution, and the per-copy amount is a variable contribution per copy sold.
				Model A: fixed component \$24{,}000 and contribution \$6.5 per copy.
				Model B: fixed component \$15{,}000 and contribution \$5.8 per copy.
				Model C: fixed component \$19{,}000 and contribution \$6.1 per copy.
				
				The quantity values are expected copies sold under each demand state:
				High demand: A = 18{,}000, B = 21{,}000, C = 19{,}500.
				Low demand: A = 9{,}500, B = 12{,}000, C = 10{,}700.
				
				Management must compute profit for each alternative using
				\(\text{Profit}=\text{Fixed component}+(\text{Quantity}\times\text{Contribution per copy})\), and then construct the payoff table in thousands of dollars.
				\subsection*{Given data}
				\begin{center}
					\begin{tabular}{l c c}
						\toprule
						Alternative & High demand copies & Low demand copies \\
						\midrule
						Model A & 18{,}000 & 9{,}500 \\
						Model B & 21{,}000 & 12{,}000 \\
						Model C & 19{,}500 & 10{,}700 \\
						\bottomrule
					\end{tabular}
				\end{center}

		\vspace{0.25cm}
		\SubsectionBox{5. Problem description}\vspace{-0.25cm}
		A commercial printing firm must choose between two production contracts for a peak exam-season cycle: Contract A or Contract B.
				Demand may be strong, normal, or weak.
				
				For Contract A, the listed numbers are already profits in thousands of dollars: 112 under strong demand, 74 under normal demand, and 26 under weak demand.
				For Contract B, profit is not given directly and must be computed from a lump-sum fixed component plus a variable margin per booklet sold:
				\[
				\text{Profit}_B = 20{,}000 + (\text{Booklets sold} \times 4.8)
				\]
				For Contract B, the quantity values are expected booklet sales: 18{,}000 (strong), 12{,}000 (normal), and 7{,}500 (weak).
				
				The firm must compute Contract B profits first, then construct the payoff table in thousands of dollars with all profits arranged by demand state.
				\subsection*{Given data}
				\begin{center}
					\begin{tabular}{l c c c}
						\toprule
						Alternative & Strong & Normal & Weak \\
						\midrule
						Contract A (direct profit) & 112 & 74 & 26 \\
						Contract B booklets sold & 18{,}000 & 12{,}000 & 7{,}500 \\
						\bottomrule
					\end{tabular}
				\end{center}

		\vspace{0.25cm}
		\SubsectionBox{6. Problem description}\vspace{-0.25cm}
		A local bus operator must choose between Fleet A and Fleet B for a passenger transport contract.
				Revenue can be high or low, while fuel and operating costs can be low, medium, or high.
				
				All figures are given directly in thousands of dollars and are separated into revenue values and cost values.
				Revenue states:
				Fleet A revenue is 420 in the high-revenue state and 290 in the low-revenue state.
				Fleet B revenue is 450 in the high-revenue state and 300 in the low-revenue state.
				
				Cost states:
				Fleet A cost is 170 in the low-cost state, 205 in the medium-cost state, and 240 in the high-cost state.
				Fleet B cost is 185 in the low-cost state, 220 in the medium-cost state, and 260 in the high-cost state.
				
				The operator must construct the payoff table by combining each revenue state with each cost state and applying
				\(\text{Profit}=\text{Revenue}-\text{Cost}\).
				\subsection*{Given data tables}
				\begin{center}
					\begin{minipage}[t]{0.48\linewidth}
						\textit{Revenue data (given directly)}
						\begin{center}
							\begin{tabular}{l c c}
								\toprule
								Fleet & High revenue & Low revenue \\
								\midrule
								Fleet A & 420 & 290 \\
								Fleet B & 450 & 300 \\
								\bottomrule
							\end{tabular}
						\end{center}
					\end{minipage}
					\hfill
					\begin{minipage}[t]{0.48\linewidth}
						\textit{Cost data (given directly)}
						\begin{center}
							\begin{tabular}{l c c c}
								\toprule
								Fleet & Low cost & Medium cost & High cost \\
								\midrule
								Fleet A & 170 & 205 & 240 \\
								Fleet B & 185 & 220 & 260 \\
								\bottomrule
							\end{tabular}
						\end{center}
					\end{minipage}
				\end{center}

		\vspace{0.25cm}
		\SubsectionBox{7. Problem description}\vspace{-0.25cm}
		A contract catering company must choose among three lunch-service plans: Plan A, Plan B, or Plan C.
				Revenue depends on quantity sold and selling price per meal, while cost values are given directly.
				Revenue has two demand states (high and low), and cost has two states (low cost and high cost).
				
				Revenue-input quantities and prices are:
				Plan A sells 14{,}000 meals in high demand or 9{,}000 meals in low demand, at a price of \$12 per meal.
				Plan B sells 15{,}500 meals in high demand or 10{,}500 meals in low demand, at a price of \$11.5 per meal.
				Plan C sells 13{,}200 meals in high demand or 8{,}800 meals in low demand, at a price of \$12.4 per meal.
				
				Cost values are already provided in thousand USD:
				Plan A cost is 102 (low-cost state) and 128 (high-cost state);
				Plan B cost is 108 (low-cost state) and 134 (high-cost state);
				Plan C cost is 98 (low-cost state) and 124 (high-cost state).
				
				The firm must first compute revenue from
				\(\text{Revenue}=\text{Quantity}\times\text{Price per meal}\), then construct the payoff table using
				\(\text{Profit}=\text{Revenue}-\text{Cost}\) for every revenue--cost state pair.
				\subsection*{Given data tables}
				\begin{center}
					\textit{Revenue-side quantities and prices}
					\begin{center}
						\begin{tabular}{l c c c}
							\toprule
							Plan & High & Low & P per meal \\
							\midrule
							A & 14{,}000 & 9{,}000 & 12.0 \\
							B & 15{,}500 & 10{,}500 & 11.5 \\
							C & 13{,}200 & 8{,}800 & 12.4 \\
							\bottomrule
						\end{tabular}
					\end{center}
					
					\vspace{0.5cm}
					
					\textit{Cost-side data (given directly)}
					\begin{center}
						\begin{tabular}{l c c}
							\toprule
							Plan & Low cost & High cost \\
							\midrule
							A & 102 & 128 \\
							B & 108 & 134 \\
							C & 98 & 124 \\
							\bottomrule
						\end{tabular}
					\end{center}
				\end{center}
				
		\vspace{0.25cm}
		\SubsectionBox{8. Problem description}\vspace{-0.25cm}
		A private training provider must choose between Program A and Program B.
				Revenue is given directly for three enrollment-demand states (high, medium, and low), while costs must be computed from participant counts and unit cost.
				
				Revenue values (thousand USD) are:
				Program A revenue is 360 (high), 280 (medium), and 210 (low).
				Program B revenue is 390 (high), 300 (medium), and 230 (low).
				
				Cost inputs are:
				For Program A, participant quantity is 1{,}000 in the low-cost state and 1{,}300 in the high-cost state, with unit cost \$120 per participant.
				For Program B, participant quantity is 1{,}050 in the low-cost state and 1{,}350 in the high-cost state, with unit cost \$125 per participant.
				
				Management must first compute costs using
				\(\text{Cost}=\text{Quantity}\times\text{Unit cost}\), then construct the payoff table in thousands of dollars with
				\(\text{Profit}=\text{Revenue}-\text{Cost}\).
				\subsection*{Given data tables}
				\begin{center}
					\textit{Revenue-side data (given directly)}
					\begin{center}
						\begin{tabular}{l c c c}
							\toprule
							Program & High & Medium & Low \\
							\midrule
							A & 360 & 280 & 210 \\
							B & 390 & 300 & 230 \\
							\bottomrule
						\end{tabular}
					\end{center}
					
					\vspace{0.5cm}
					
					\textit{Cost-side quantity and unit cost}
					\begin{center}
						\begin{tabular}{l c c c}
							\toprule
							Program & Qty (low cost) & Qty (high cost) & Unit cost \\
							\midrule
							A & 1{,}000 & 1{,}300 & 120 \\
							B & 1{,}050 & 1{,}350 & 125 \\
							\bottomrule
						\end{tabular}
					\end{center}
				\end{center}
				
		\vspace{0.25cm}
		\SubsectionBox{9. Problem description}\vspace{-0.25cm}
		A small electronics assembler must choose one of three sourcing contracts: Contract A, Contract B, or Contract C.
				For each contract, both revenue and cost must be calculated from quantities and unit rates.
				Revenue has two demand states (high and medium), and cost has three component-price states (low, medium, high).
				
				Revenue-side inputs are:
				\begin{itemize}
					\item Contract A: quantity sold is 3{,}200 units in high demand and 2{,}500 units in medium demand; selling price is \$210 per unit.
					\item Contract B: quantity sold is 3{,}450 units in high demand and 2{,}650 units in medium demand; selling price is \$205 per unit.
					\item Contract C: quantity sold is 3{,}300 units in high demand and 2{,}600 units in medium demand; selling price is \$208 per unit.
				\end{itemize}
				Cost-side inputs are:
				\begin{itemize}
					\item Contract A: cost quantity is 3{,}100 units in each cost state, with unit cost \$128 (low), \$136 (medium), and \$145 (high).
					\item Contract B: cost quantity is 3{,}300 units in each cost state, with unit cost \$124 (low), \$133 (medium), and \$142 (high).
					\item Contract C: cost quantity is 3{,}180 units in each cost state, with unit cost \$126 (low), \$135 (medium), and \$144 (high).
				\end{itemize}
				
				The assembler must compute revenue and cost first using
				\(\text{Revenue}=q\times p\) and \(\text{Cost}=q\times c\), then construct the payoff table in thousands of dollars from
				\(\text{Profit}=\text{Revenue}-\text{Cost}\).
				\subsection*{Given data tables}
				\begin{center}
					\textit{Revenue computation inputs}
					\begin{center}
						\begin{tabular}{l c c c}
							\toprule
							Contract & Qty high & Qty medium & Price \\
							\midrule
							A & 3{,}200 & 2{,}500 & 210 \\
							B & 3{,}450 & 2{,}650 & 205 \\
							C & 3{,}300 & 2{,}600 & 208 \\
							\bottomrule
						\end{tabular}
					\end{center}
					
					\vspace{0.5cm}
					
					\textit{Cost computation inputs}
					\begin{center}
						\begin{tabular}{l c c c}
							\toprule
							Contract $\backslash$ cost & low & med & high \\
							\midrule
							A ($q=3{,}100$) & 128 & 136 & 145 \\
							B ($q=3{,}300$) & 124 & 133 & 142 \\
							C ($q=3{,}180$) & 126 & 135 & 144 \\
							\bottomrule
						\end{tabular}
					\end{center}
				\end{center}
				
		\vspace{0.25cm}
		\SubsectionBox{10. Problem description}\vspace{-0.25cm}
		An agricultural exporter must choose among three logistics plans: Plan X, Plan Y, or Plan Z.
				Both revenue and cost include a fixed component plus a variable component linked to activity volume.
				Revenue has three demand states (high, medium, low), and cost has two operating states (normal and disrupted).
				
				Revenue models are:
				\[
				\begin{aligned}
				R_X &= 150{,}000 + (q\times52),\\
				R_Y &= 130{,}000 + (q\times55),\\
				R_Z &= 145{,}000 + (q\times53).
				\end{aligned}
				\]
				In these formulas, the fixed terms (150{,}000, 130{,}000, 145{,}000) are fixed revenue components, and 52, 55, and 53 are variable revenue per unit.
				Demand-state quantities are:
				X: high 9{,}000, medium 7{,}000, low 5{,}000;
				Y: high 8{,}700, medium 6{,}900, low 5{,}300;
				Z: high 8{,}900, medium 6{,}950, low 5{,}150.
				
				Cost models are:
				\[
				\begin{aligned}
				C_X &= 120{,}000 + (n\times27),\\
				C_Y &= 105{,}000 + (n\times29),\\
				C_Z &= 112{,}000 + (n\times28).
				\end{aligned}
				\]
				Here, the fixed terms (120{,}000, 105{,}000, 112{,}000) are fixed costs, and 27, 29, and 28 are variable cost per unit of activity.
				Cost-state activity quantities are:
				X: normal 8{,}200, disrupted 9{,}100;
				Y: normal 8{,}000, disrupted 9{,}000;
				Z: normal 8{,}100, disrupted 9{,}050.
				
				The exporter must compute revenue and cost for each state first, then construct the payoff table in thousands of dollars using
				\(\text{Profit}=\text{Revenue}-\text{Cost}\).
				\subsection*{Given data tables}
				\begin{center}
					\textit{Revenue side (fixed + variable)}
					\begin{center}
						\begin{tabular}{l c c c}
							\toprule
							Plan & High $q$ & Medium $q$ & Low $q$ \\
							\midrule
							X & 9{,}000 & 7{,}000 & 5{,}000 \\
							Y & 8{,}700 & 6{,}900 & 5{,}300 \\
							Z & 8{,}900 & 6{,}950 & 5{,}150 \\
							\bottomrule
						\end{tabular}
					\end{center}
					
					\vspace{0.5cm}
					
					\textit{Cost side (fixed + variable)}
					\begin{center}
						\begin{tabular}{l c c}
							\toprule
							Plan & Normal $n$ & Disrupted $n$ \\
							\midrule
							X & 8{,}200 & 9{,}100 \\
							Y & 8{,}000 & 9{,}000 \\
							Z & 8{,}100 & 9{,}050 \\
							\bottomrule
						\end{tabular}
					\end{center}
				\end{center}
	\end{multicols}
\end{document}
