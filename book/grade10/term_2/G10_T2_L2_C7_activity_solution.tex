\makeatletter
\def\input@path{{./}{../}{../../}{preamble/}{../preamble/}{../../preamble/}}
\makeatother
% ----------------------------------------------------------
% GENERAL 

% File
\documentclass[11pt]{book}

% Margins
\usepackage[margin=1in]{geometry}

\linespread{1.2}            % Line spacing
\usepackage[utf8]{inputenc}
\usepackage[T1]{fontenc}
\usepackage{lmodern}
\usepackage{microtype}
\setlength{\parindent}{0pt}
\setlength{\parskip}{6pt}
\usepackage{booktabs}

% ----------------------------------------------------------
% TABLES
\usepackage{multicol}
\usepackage{longtable} 
\usepackage{array}
\usepackage{booktabs}
\usepackage{tabularx}
\usepackage{multirow}

% ----------------------------------------------------------
% MATHEMATICS
\usepackage{amsmath}
\usepackage{amssymb}
\usepackage{amsfonts}
\usepackage{mathtools}

% ----------------------------------------------------------
% HIDDEN CONTENT
\usepackage{ifthen}
% Define a boolean switch
\newboolean{explicaciones}
% Set the boolean switch to true or false
% Change to true to show the content

% Explanations
\newcommand{\explicacion}[2]{
	\ifthenelse{\boolean{explicaciones}}{#1}{#2}
}
\newcommand{\mostrarExplicaciones}[1]{\setboolean{explicaciones}{#1}}

% ----------------------------------------------------------
% NUMBERING

\usepackage{fancyhdr}
\pagestyle{empty} % Ensures the entire document has no page numbers

\usepackage{tocloft}
\renewcommand{\cftdot}{} % Remove dots for sections, if any
\renewcommand{\cftsecleader}{\cftdotfill{\cftdotsep}} % Remove dots for sections, if any
\cftpagenumbersoff{section} % Removes page numbers from sections
\cftpagenumbersoff{subsection} % Removes page numbers from subsections

% ----------------------------------------------------------
% IMAGES 

% General settings
\usepackage{graphicx}       % Insert images
\usepackage{float}          % Position images
% \usepackage{subfigure}      % Subfigures
\graphicspath{{imgs}}       % Image location
\usepackage{subcaption}     % Subfigures II
\usepackage{verbatim}

% Figures
\usepackage{tikz}
\usetikzlibrary{arrows.meta,positioning,trees}

% Colors
\usepackage{xcolor}     
\definecolor{popUp}{HTML}{666666}
\definecolor{popUpIn}{HTML}{CED9E0}
\definecolor{backgroundC}{HTML}{D0E8F2}
\definecolor{backgroundCC}{HTML}{FFFFFF}
\definecolor{borders}{HTML}{8c120d}
\definecolor{padding}{HTML}{77D0D7}
\definecolor{links}{HTML}{CC6F5F}

% ----------------------------------------------------------
% FRAMES

% General settings
\usepackage{tcolorbox}
\usepackage{adjustbox}          % Adjusted frame  
\setlength{\fboxrule}{3pt}  % Line width
\setlength{\fboxsep}{3pt}   % Box padding

% General frames
\usepackage{mdframed}   

\mdfdefinestyle{estiloGeneral}{    % General style
	linecolor=black,
	linewidth=1.5pt,
	roundcorner=10pt,
	backgroundcolor=backgroundC,
	innerbottommargin=0pt
}
\mdfdefinestyle{code}{          % Code style
	linecolor=black,
	linewidth=1.5pt,
	roundcorner=10pt,
	backgroundcolor=darkgray!10,
	innerbottommargin=0pt
}

% Image frame
\newtcbox{\fboxC}{
	colback=backgroundC,
	colframe=popUp,
	arc=10pt,
	boxrule=3pt,
	boxsep=0pt, % Change the padding here
	nobeforeafter
}

% ----------------------------------------------------------
% PAGE SETTINGS

% Background 
\newcommand{\background}[0]{\begin{tikzpicture}[remember picture,overlay]
		\fill[backgroundC] (-2,2) rectangle (25cm, -550);
\end{tikzpicture}}

\newcommand{\backgroundC}[0]{\begin{tikzpicture}[remember picture,overlay]
		\fill[backgroundCC] (-2,2) rectangle (25cm, -550);
\end{tikzpicture}}

% Page width 
\newcommand{\anchoPag}[0]{20cm}

% ----------------------------------------------------------
% FONT

% General
\usepackage{tgbonum}        % Font
\usepackage{listings}       % Code typesetting
\usepackage[spanish]{babel} % Load Spanish
\selectlanguage{spanish}    % Select Spanish
\usepackage{enumitem}
\usepackage{bookmark}

\setlist[itemize]{leftmargin=1.2em, itemsep=0.35em, topsep=0.35em}

% --- Table helpers ---
\newcolumntype{L}[1]{>{\raggedright\arraybackslash}p{#1}}
\newcolumntype{Y}{>{\raggedright\arraybackslash}X}
\newcolumntype{C}{>{\centering\arraybackslash}X}
\renewcommand{\arraystretch}{1.1}

% Python style
\lstdefinestyle{python}{
	language=Python,
	basicstyle=\ttfamily\small,
	commentstyle=\color{green!50!black},
	keywordstyle=\color{blue},
	numberstyle=\tiny\color{gray},
	numbers=left,
	morekeywords={>, <},
	breakatwhitespace=false,
	showstringspaces=false,
	showtabs=false,
	showspaces=false
}

% ----------------------------------------------------------
% HYPERLINKS

% General
\usepackage{hyperref}       
\hypersetup{
	colorlinks=true,
	linkcolor=links,
	filecolor=magenta,      
	urlcolor=brown,
}

% Custom commands 

% Large link
\newcommand{\bigLink}[2]{\begin{center} \fboxC{\LARGE{\href{#1}{#2}}}\end{center}}

% Small link
\newcommand{\smallLink}[2]{\begin{center}\fboxC{\href{#1}{#2}}\end{center}}

% Bold link
\newcommand{\bfLink}[2]{\href{#1}{\textbf{#2}}}


% Small URL
\newcommand{\smallUrl}[1]{\begin{center}\fboxC{\url{#1}}\end{center}}


% ----------------------------------------------------------
% CUSTOM COMMANDS FOR FIGURES

\newcommand{\espacioImagenes}[0]{-1.2cm}

% Without frame
\newcommand{\fig}[3][\espacioImagenes]{
	\hspace*{#1}
	\centering
	\includegraphics[width=#2\textwidth]{#3}
}

% With frame
\newcommand{\ffig}[2]{\begin{figure}[h]
		\hspace*{\espacioImagenes}
		\centering
		\fbox{\includegraphics[width=#1\textwidth]{#2}}
\end{figure}}

% Hyperlink with frame
\newcommand{\hfig}[3]{\begin{figure}[h]
		\hspace*{-1.4cm}
		\centering
		\color{popUp}
		\fboxC{\href{#1}{\includegraphics[width=#2\textwidth]{#3}}}
	\end{figure}
}

% Hyperlink without frame
\newcommand{\hffig}[3]{\begin{figure}[h]
		\hspace*{-1.1cm}
		\centering
		\color{popUp}
		\href{#1}{\includegraphics[width=#2\textwidth]{#3}}
	\end{figure}
}

% ----------------------------------------------------------

% Start and Contents
\newcommand{\cuadro}[1]{
	\begin{mdframed}[style=estiloGeneral]
		#1 
	\end{mdframed}
}

% Explanation video image
\newcommand{\linkExplicacion}[1]{
	\hffig{#1}{0.5}{principal/videoExplicacion}
	\vspace{-0.5cm}
}

\newcommand{\subSecLink}[2]{
	\subsubsection*{\href{#1}{\textbf{#2}}}
}

% Spacing
\newcommand{\esp}[0]{\vspace{4mm}}

% Back to start
\newcommand{\secInicio}[0]{\begin{center}\hyperref[sec:inicio]{ 
			\includegraphics[width=0.1\textwidth]{principal/up}
	}\end{center}
}


\geometry{margin=0.85in}
\AtBeginDocument{\small}

\newcommand{\ExamNameField}{\noindent\textbf{Name:}\ \rule{0.7\linewidth}{0.4pt}\par\medskip}

\newcommand{\ExamTitleBlock}[3]{%
	\begin{center}
		\Large\textbf{#1}\\
		\textbf{#2}%
		\if\relax\detokenize{#3}\relax\else\\\textbf{#3}\fi
	\end{center}
	\vspace{0.5em}
}

\newcommand{\ExamSection}[1]{\par\medskip\textbf{#1}\par\smallskip}

\newenvironment{ExamCriteria}{%
	\begin{itemize}[leftmargin=1.6em, itemsep=0.3em, topsep=0.2em]
}{%
	\end{itemize}
}

\newenvironment{ExamProblems}{%
	\begin{enumerate}[label=\textbf{P\arabic*}, leftmargin=0pt, labelsep=0.6em, itemindent=2.2em, itemsep=0.8em]
}{%
	\end{enumerate}
}


\begin{document}
	\ExamTitleBlock{10th grade}{Learning evidence T2 L2 Decision analysis C7 activity solutions}{}
	
	\section*{Contents}
	\noindent\textbf{C7 Uses probabilities and expected value to analyze a decision-making problem.}
	\begin{itemize}
		\item \hyperlink{c7-ex1}{Problem 1 — Single Project under Uncertainty}
		\item \hyperlink{c7-ex2}{Problem 2 — Technology Upgrade}
		\item \hyperlink{c7-ex3}{Problem 3 — Two Investment Alternatives}
		\item \hyperlink{c7-ex4}{Problem 4 — Three Production Plans}
		\item \hyperlink{c7-ex5}{Problem 5 — Two Alternatives with Three States}
		\item \hyperlink{c7-ex6}{Problem 6 — Policy Choice under Three States}
		\item \hyperlink{c7-ex7}{Problem 7 — New Service Platform Launch}
		\item \hyperlink{c7-ex8}{Problem 8 — Regional Store Opening}
		\item \hyperlink{c7-ex9}{Problem 9 — Marketing Strategy Choice}
		\item \hyperlink{c7-ex10}{Problem 10 — Investment Portfolio Selection}
	\end{itemize}
	
	\ExamSection{C7 Uses probabilities and expected value to analyze a decision-making problem.}
	
	\begin{ExamProblems}
		
		\hypertarget{c7-ex1}{}
		\item
		\subsection*{Problem 1 — Single Project under Uncertainty}
		
		\textbf{Problem.}
		A renewable energy firm is deciding whether to build a small solar farm. The decision is whether to build (one alternative) or not build, and management uses expected value to decide because they expect similar projects to be repeated over time. The uncertain states of nature are:
		\begin{itemize}
			\item \(S_1\): high electricity prices next year (probability \(p\) from a market forecast),
			\item \(S_2\): low electricity prices next year (probability \(1-p\)).
		\end{itemize}
		The payoffs are net profits in millions of dollars from building the farm.
		
		\[
		\begin{array}{lcc}
			\hline
			& S_1\,(p) & S_2\,(1-p) \\
			\hline
			\text{Build the farm} & 24 & -6 \\
			\hline
		\end{array}
		\]
		
		Use expected value as the decision rule and determine for which values of \(p\) building the farm is profitable (expected value positive).
		
		\textbf{Solution.}
		Expected value:
		\[
		EV = 24\cdot p + (-6)\cdot(1-p)
		\]
		Expand:
		\[
		EV = 24p - 6(1-p) = 24p - 6 + 6p
		\]
		Combine like terms:
		\[
		EV = 30p - 6
		\]
		Profitability condition:
		\[
		EV>0 \Rightarrow 30p-6>0
		\]
		Solve:
		\[
		30p>6 \Rightarrow p>\frac{6}{30}=\frac{1}{5}
		\]
		So the solar farm is profitable when \(p>\frac{1}{5}\).
		
		% --------------------------------------------------
		
		\hypertarget{c7-ex2}{}
		\item
		\subsection*{Problem 2 — Technology Upgrade}
		
		\textbf{Problem.}
		A delivery company is considering upgrading its routing software. The choice is to upgrade or not, and expected value is appropriate because the company plans to use the same software for many delivery cycles. The uncertain states are:
		\begin{itemize}
			\item \(S_1\): fuel prices stay high (probability \(p\) estimated from energy reports),
			\item \(S_2\): fuel prices fall (probability \(1-p\)).
		\end{itemize}
		Payoffs are net savings in millions of dollars from upgrading.
		
		\[
		\begin{array}{lcc}
			\hline
			& S_1\,(p) & S_2\,(1-p) \\
			\hline
			\text{Upgrade} & 32 & -14 \\
			\hline
		\end{array}
		\]
		
		Using expected value, determine for which values of \(p\) the upgrade is profitable.
		
		\textbf{Solution.}
		Expected value:
		\[
		EV = 32\cdot p + (-14)\cdot(1-p)
		\]
		Expand:
		\[
		EV = 32p - 14(1-p) = 32p - 14 + 14p
		\]
		Combine like terms:
		\[
		EV = 46p - 14
		\]
		Profitability condition:
		\[
		EV>0 \Rightarrow 46p-14>0
		\]
		Solve:
		\[
		46p>14 \Rightarrow p>\frac{14}{46}=\frac{7}{23}
		\]
		So the upgrade is profitable when \(p>\frac{7}{23}\).
		
		% --------------------------------------------------
		
		\hypertarget{c7-ex3}{}
		\item
		\subsection*{Problem 3 — Two Investment Alternatives}
		
		\textbf{Problem.}
		An entrepreneur must choose between two investment alternatives, A and B. Expected value is used because the entrepreneur wants the option with the higher long-run average return. The uncertain states are:
		\begin{itemize}
			\item \(S_1\): strong market growth (probability \(p\) based on analyst forecasts),
			\item \(S_2\): weak market growth (probability \(1-p\)).
		\end{itemize}
		Payoffs are net profits in millions of dollars.
		
		\[
		\begin{array}{lcc}
			\hline
			& S_1\,(p) & S_2\,(1-p) \\
			\hline
			A & 28 & 4 \\
			B & 18 & 12 \\
			\hline
		\end{array}
		\]
		
		Using expected value, determine for which values of \(p\) option A yields a higher expected value than option B.
		
		\textbf{Solution.}
		Compute expected value of A:
		\[
		EV(A)=28\cdot p + 4\cdot(1-p)
		\]
		Expand and simplify:
		\[
		EV(A)=28p + 4 - 4p = (28p-4p)+4 = 24p+4
		\]
		Compute expected value of B:
		\[
		EV(B)=18\cdot p + 12\cdot(1-p)
		\]
		Expand and simplify:
		\[
		EV(B)=18p + 12 - 12p = (18p-12p)+12 = 6p+12
		\]
		A is better than B when \(EV(A)>EV(B)\), equivalently:
		\[
		EV(A)-EV(B)>0
		\]
		Substitute:
		\[
		(24p+4)-(6p+12)>0
		\]
		Simplify:
		\[
		24p+4-6p-12>0 \Rightarrow 18p-8>0
		\]
		Solve:
		\[
		18p>8 \Rightarrow p>\frac{8}{18}=\frac{4}{9}
		\]
		So option A is better than option B when \(p>\frac{4}{9}\).
		
		% --------------------------------------------------
		
		\hypertarget{c7-ex4}{}
		\item
		\subsection*{Problem 4 — Three Production Plans}
		
		\textbf{Problem.}
		A manufacturing firm must choose among three production plans (A, B, C) for the next quarter. The firm uses expected value because it seeks the plan with the highest average profit across many similar quarters. The uncertain states are:
		\begin{itemize}
			\item \(S_1\): strong demand (probability \(p\) from the sales forecast),
			\item \(S_2\): weak demand (probability \(1-p\)).
		\end{itemize}
		Payoffs are net profits in millions of dollars.
		
		\[
		\begin{array}{lcc}
			\hline
			& S_1\,(p) & S_2\,(1-p) \\
			\hline
			A & 40 & -6 \\
			B & 26 & 6 \\
			C & 18 & 12 \\
			\hline
		\end{array}
		\]
		
		Using expected value, determine for which values of \(p\) each plan is optimal.
		
		\textbf{Solution.}
		Compute expected value of each plan.
		
		Plan A:
		\[
		EV(A)=40\cdot p + (-6)\cdot(1-p)
		\]
		Expand:
		\[
		EV(A)=40p - 6 + 6p = 46p - 6
		\]
		
		Plan B:
		\[
		EV(B)=26\cdot p + 6\cdot(1-p)
		\]
		Expand:
		\[
		EV(B)=26p + 6 - 6p = 20p+6
		\]
		
		Plan C:
		\[
		EV(C)=18\cdot p + 12\cdot(1-p)
		\]
		Expand:
		\[
		EV(C)=18p + 12 - 12p = 6p+12
		\]
		
		Now compare pairwise using \(>0\) conditions.
		
		A better than B:
		\[
		EV(A)-EV(B)>0 \Rightarrow (46p-6)-(20p+6)>0
		\]
		Simplify:
		\[
		46p-6-20p-6>0 \Rightarrow 26p-12>0 \Rightarrow p>\frac{12}{26}=\frac{6}{13}
		\]
		
		B better than C:
		\[
		EV(B)-EV(C)>0 \Rightarrow (20p+6)-(6p+12)>0
		\]
		Simplify:
		\[
		20p+6-6p-12>0 \Rightarrow 14p-6>0 \Rightarrow p>\frac{6}{14}=\frac{3}{7}
		\]
		
		A better than C:
		\[
		EV(A)-EV(C)>0 \Rightarrow (46p-6)-(6p+12)>0
		\]
		Simplify:
		\[
		46p-6-6p-12>0 \Rightarrow 40p-18>0 \Rightarrow p>\frac{18}{40}=\frac{9}{20}
		\]
		
		Determine optimal plan by intervals:
		\begin{itemize}
			\item If \(p<\frac{3}{7}\), then \(EV(C)>EV(B)\) and \(EV(C)>EV(A)\) because \(p<\frac{9}{20}\). So for \(p<\frac{3}{7}\), plan C is optimal.
			\item If \(\frac{3}{7}<p<\frac{6}{13}\), then \(EV(B)>EV(C)\) and \(EV(A)<EV(B)\). So in this interval, plan B is optimal.
			\item If \(p>\frac{6}{13}\), then \(EV(A)>EV(B)\) and also \(EV(A)>EV(C)\) (since \(\frac{6}{13}>\frac{9}{20}\)). So for \(p>\frac{6}{13}\), plan A is optimal.
		\end{itemize}
		
		% --------------------------------------------------
		
		\hypertarget{c7-ex5}{}
		\item
		\subsection*{Problem 5 — Two Alternatives with Three States}
		
		\textbf{Problem.}
		A shipping company must choose between two routing strategies, A and B. Expected value is the decision criterion because the company wants the route with the higher average profit over many shipments. The uncertain states are:
		\begin{itemize}
			\item \(S_1\): low congestion (probability \(p_1\) from traffic data),
			\item \(S_2\): moderate congestion (probability \(p_2\)),
			\item \(S_3\): severe congestion (probability \(1-p_1-p_2\)).
		\end{itemize}
		Payoffs are net profits in millions of dollars.
		
		\[
		\begin{array}{lccc}
			\hline
			& S_1\,(p_1) & S_2\,(p_2) & S_3\,(1-p_1-p_2) \\
			\hline
			A & 24 & 8 & -9 \\
			B & 16 & 14 & 2 \\
			\hline
		\end{array}
		\]
		
		Using expected value, determine when strategy A yields a higher expected value than strategy B.
		
		\textbf{Solution.}
		Compute expected value of A:
		\[
		EV(A)=24p_1 + 8p_2 + (-9)(1-p_1-p_2)
		\]
		Expand the last term:
		\[
		EV(A)=24p_1 + 8p_2 -9 + 9p_1 + 9p_2
		\]
		Combine like terms:
		\[
		EV(A)=(24p_1+9p_1) + (8p_2+9p_2) - 9 = 33p_1 + 17p_2 - 9
		\]
		Compute expected value of B:
		\[
		EV(B)=16p_1 + 14p_2 + 2\cdot(1-p_1-p_2)
		\]
		Expand:
		\[
		EV(B)=16p_1 + 14p_2 + 2 - 2p_1 - 2p_2
		\]
		Combine:
		\[
		EV(B)=(16p_1-2p_1) + (14p_2-2p_2) + 2 = 14p_1 + 12p_2 + 2
		\]
		A better than B when:
		\[
		EV(A)-EV(B)>0
		\]
		Substitute:
		\[
		(33p_1+17p_2-9)-(14p_1+12p_2+2)>0
		\]
		Simplify:
		\[
		33p_1+17p_2-9-14p_1-12p_2-2>0
		\Rightarrow 19p_1+5p_2-11>0
		\]
		So strategy A is better than strategy B when \(19p_1+5p_2-11>0\).
		
		% --------------------------------------------------
		
		\hypertarget{c7-ex6}{}
		\item
		\subsection*{Problem 6 — Policy Choice under Three States}
		
		\textbf{Problem.}
		A local government compares two flood-prevention policies, A and B. Expected value is used because the city plans to choose the policy that yields the highest average net benefit over many years. The uncertain states are:
		\begin{itemize}
			\item \(S_1\): heavy rainfall season (probability \(p_1\) from climate models),
			\item \(S_2\): moderate rainfall season (probability \(p_2\)),
			\item \(S_3\): dry season (probability \(1-p_1-p_2\)).
		\end{itemize}
		Payoffs are net benefits in millions of dollars.
		
		\[
		\begin{array}{lccc}
			\hline
			& S_1\,(p_1) & S_2\,(p_2) & S_3\,(1-p_1-p_2) \\
			\hline
			A & 20 & 6 & -8 \\
			B & 12 & 10 & 4 \\
			\hline
		\end{array}
		\]
		
		Using expected value, determine when policy A yields a higher expected value than policy B.
		
		\textbf{Solution.}
		Compute expected value of A:
		\[
		EV(A)=20p_1 + 6p_2 + (-8)(1-p_1-p_2)
		\]
		Expand:
		\[
		EV(A)=20p_1 + 6p_2 -8 + 8p_1 + 8p_2
		\]
		Combine:
		\[
		EV(A)=(20p_1+8p_1) + (6p_2+8p_2) - 8 = 28p_1 + 14p_2 - 8
		\]
		Compute expected value of B:
		\[
		EV(B)=12p_1 + 10p_2 + 4(1-p_1-p_2)
		\]
		Expand:
		\[
		EV(B)=12p_1 + 10p_2 + 4 - 4p_1 - 4p_2
		\]
		Combine:
		\[
		EV(B)=(12p_1-4p_1) + (10p_2-4p_2) + 4 = 8p_1 + 6p_2 + 4
		\]
		A better than B when:
		\[
		EV(A)-EV(B)>0
		\]
		Substitute:
		\[
		(28p_1+14p_2-8)-(8p_1+6p_2+4)>0
		\]
		Simplify:
		\[
		28p_1+14p_2-8-8p_1-6p_2-4>0
		\Rightarrow 20p_1+8p_2-12>0
		\]
		Divide by \(4\) (positive, inequality direction unchanged):
		\[
		5p_1+2p_2-3>0
		\]
		So policy A is better than policy B when \(5p_1+2p_2-3>0\).
		
		% --------------------------------------------------
		
		\hypertarget{c7-ex7}{}
		\item
		\subsection*{Problem 7 — New Service Platform Launch}
		
		\textbf{Problem.}
		A software company must choose among three launch plans (A, B, C) for a new service platform. Expected value is used because the company plans to repeat similar launches and wants the highest average profit. The uncertain states are:
		\begin{itemize}
			\item \(S_1\): strong early adoption (probability \(0.6\) based on pilot data),
			\item \(S_2\): modest early adoption (probability \(0.4\)).
		\end{itemize}
		Payoffs are net profits in millions of dollars.
		
		\[
		\begin{array}{lcc}
			\hline
			& S_1\,(0.6) & S_2\,(0.4) \\
			\hline
			A & 30 & 4 \\
			B & 22 & 10 \\
			C & 16 & 14 \\
			\hline
		\end{array}
		\]
		
		Using expected value, compute the expected value for each plan and recommend the plan with the highest expected value.
		
		\textbf{Solution.}
		Compute expected value of each plan.
		
		Plan A:
		\[
		EV(A)=30\cdot 0.6 + 4\cdot 0.4
		\]
		\[
		EV(A)=18 + 1.6 = 19.6
		\]
		
		Plan B:
		\[
		EV(B)=22\cdot 0.6 + 10\cdot 0.4
		\]
		\[
		EV(B)=13.2 + 4 = 17.2
		\]
		
		Plan C:
		\[
		EV(C)=16\cdot 0.6 + 14\cdot 0.4
		\]
		\[
		EV(C)=9.6 + 5.6 = 15.2
		\]
		
		Compare the expected values: \(19.6>17.2>15.2\). Therefore, plan A has the highest expected value, so plan A is recommended.
		
		% --------------------------------------------------
		
		\hypertarget{c7-ex8}{}
		\item
		\subsection*{Problem 8 — Regional Store Opening}
		
		\textbf{Problem.}
		A retailer evaluates opening a regional store. Expected value is appropriate because the retailer wants the highest average profit across multiple comparable openings. The uncertain factors are demand and competitor response, assumed independent based on separate studies. The states are:
		\begin{itemize}
			\item strong demand (probability \(p\) from market research),
			\item weak demand (probability \(1-p\)),
			\item mild competitor response (probability \(q\)),
			\item aggressive competitor response (probability \(1-q\)).
		\end{itemize}
		Payoffs are net profits in millions of dollars.
		
		\[
		\begin{array}{lcc}
			\hline
			& \text{Mild}\,(q) & \text{Aggressive}\,(1-q) \\
			\hline
			\text{Strong demand}\,(p) & 40 & 14 \\
			\text{Weak demand}\,(1-p) & -2 & -18 \\
			\hline
		\end{array}
		\]
		
		Using expected value, determine when opening the store is profitable.
		
		\textbf{Solution.}
		Joint probabilities:
		\[
		P(\text{Strong,Mild})=pq,\quad
		P(\text{Strong,Aggressive})=p(1-q),\quad
		P(\text{Weak,Mild})=(1-p)q,\quad
		P(\text{Weak,Aggressive})=(1-p)(1-q)
		\]
		Expected value:
		\[
		EV=40(pq)+14\bigl(p(1-q)\bigr)+(-2)\bigl((1-p)q\bigr)+(-18)\bigl((1-p)(1-q)\bigr)
		\]
		Expand:
		\[
		EV=40pq + 14p - 14pq - 2q + 2pq - 18(1-p-q+pq)
		\]
		Expand last term:
		\[
		-18(1-p-q+pq)=-18 + 18p + 18q - 18pq
		\]
		Combine:
		\[
		EV=(40pq-14pq+2pq-18pq) + (14p+18p) + (-2q+18q) - 18
		\]
		\[
		EV=10pq + 32p + 16q - 18
		\]
		Profitability condition:
		\[
		EV>0 \Rightarrow 10pq+32p+16q-18>0
		\]
		
		% --------------------------------------------------
		
		\hypertarget{c7-ex9}{}
		\item
		\subsection*{Problem 9 — Marketing Strategy Choice}
		
		\textbf{Problem.}
		A firm must choose between two marketing strategies, A and B. Expected value is used because the firm wants the strategy with the highest average profit across many campaigns. Market interest and campaign execution are assumed independent because they are driven by separate teams. The states are:
		\begin{itemize}
			\item high interest (probability \(p\) from survey data),
			\item low interest (probability \(1-p\)),
			\item effective execution (probability \(q\) from past performance),
			\item ineffective execution (probability \(1-q\)).
		\end{itemize}
		Payoffs are net profits in millions of dollars.
		
		\[
		\begin{array}{lcc}
			\hline
			\text{State of nature} & A & B \\
			\hline
			\text{High interest}\,(p),\,\text{Effective}\,(q) & 34 & 26 \\
			\text{High interest}\,(p),\,\text{Ineffective}\,(1-q) & 12 & 16 \\
			\text{Low interest}\,(1-p),\,\text{Effective}\,(q) & -8 & 2 \\
			\text{Low interest}\,(1-p),\,\text{Ineffective}\,(1-q) & -16 & -6 \\
			\hline
		\end{array}
		\]
		
		Using expected value, determine when strategy A yields a higher expected value than strategy B.
		
		\textbf{Solution.}
		Compute \(EV_A\) by weighting each payoff:
		\[
		EV_A=34(pq)+12\bigl(p(1-q)\bigr)+(-8)\bigl((1-p)q\bigr)+(-16)\bigl((1-p)(1-q)\bigr)
		\]
		Expand:
		\[
		EV_A=34pq + 12p - 12pq - 8q + 8pq - 16(1-p-q+pq)
		\]
		Expand last term:
		\[
		-16(1-p-q+pq)=-16 + 16p + 16q - 16pq
		\]
		Combine:
		\[
		EV_A=(34pq-12pq+8pq-16pq) + (12p+16p) + (-8q+16q) - 16
		\]
		\[
		EV_A=14pq + 28p + 8q - 16
		\]
		
		Compute \(EV_B\):
		\[
		EV_B=26(pq)+16\bigl(p(1-q)\bigr)+2\bigl((1-p)q\bigr)+(-6)\bigl((1-p)(1-q)\bigr)
		\]
		Expand:
		\[
		EV_B=26pq + 16p - 16pq + 2q - 2pq - 6(1-p-q+pq)
		\]
		Expand last term:
		\[
		-6(1-p-q+pq)=-6 + 6p + 6q - 6pq
		\]
		Combine:
		\[
		EV_B=(26pq-16pq-2pq-6pq) + (16p+6p) + (2q+6q) - 6
		\]
		\[
		EV_B=2pq + 22p + 8q - 6
		\]
		
		Strategy A better than B when:
		\[
		EV_A-EV_B>0
		\]
		Substitute:
		\[
		(14pq+28p+8q-16)-(2pq+22p+8q-6)>0
		\]
		Simplify:
		\[
		14pq+28p+8q-16-2pq-22p-8q+6>0
		\Rightarrow 12pq+6p-10>0
		\]
		
		% --------------------------------------------------
		
		\hypertarget{c7-ex10}{}
		\item
		\subsection*{Problem 10 — Investment Portfolio Selection}
		
		\textbf{Problem.}
		An investor must choose between two portfolios, X and Y. Expected value is used because the investor wants the option with the higher long-run average return. Economic growth and interest rates are treated as independent based on separate macroeconomic forecasts. The states are:
		\begin{itemize}
			\item strong growth (probability \(p\)),
			\item weak growth (probability \(1-p\)),
			\item low interest rates (probability \(q\)),
			\item high interest rates (probability \(1-q\)).
		\end{itemize}
		Payoffs are net returns in millions of dollars.
		
		\[
		\begin{array}{lcc}
			\hline
			\text{State of nature} & X & Y \\
			\hline
			\text{Strong growth}\,(p),\,\text{Low rates}\,(q) & 30 & 24 \\
			\text{Strong growth}\,(p),\,\text{High rates}\,(1-q) & 14 & 18 \\
			\text{Weak growth}\,(1-p),\,\text{Low rates}\,(q) & 4 & 8 \\
			\text{Weak growth}\,(1-p),\,\text{High rates}\,(1-q) & -12 & -6 \\
			\hline
		\end{array}
		\]
		
		Using expected value, determine when portfolio X yields a higher expected value than portfolio Y.
		
		\textbf{Solution.}
		Compute \(EV_X\):
		\[
		EV_X=30(pq)+14\bigl(p(1-q)\bigr)+4\bigl((1-p)q\bigr)+(-12)\bigl((1-p)(1-q)\bigr)
		\]
		Expand:
		\[
		EV_X=30pq + 14p - 14pq + 4q - 4pq - 12(1-p-q+pq)
		\]
		Expand last term:
		\[
		-12(1-p-q+pq)=-12 + 12p + 12q - 12pq
		\]
		Combine:
		\[
		EV_X=(30pq-14pq-4pq-12pq) + (14p+12p) + (4q+12q) - 12
		\]
		\[
		EV_X=26p + 16q - 12
		\]
		
		Compute \(EV_Y\):
		\[
		EV_Y=24(pq)+18\bigl(p(1-q)\bigr)+8\bigl((1-p)q\bigr)+(-6)\bigl((1-p)(1-q)\bigr)
		\]
		Expand:
		\[
		EV_Y=24pq + 18p - 18pq + 8q - 8pq - 6(1-p-q+pq)
		\]
		Expand last term:
		\[
		-6(1-p-q+pq)=-6 + 6p + 6q - 6pq
		\]
		Combine:
		\[
		EV_Y=(24pq-18pq-8pq-6pq) + (18p+6p) + (8q+6q) - 6
		\]
		\[
		EV_Y=-8pq + 24p + 14q - 6
		\]
		
		Portfolio X better than Y when:
		\[
		EV_X-EV_Y>0
		\]
		Substitute:
		\[
		(26p+16q-12)-(-8pq+24p+14q-6)>0
		\]
		Simplify:
		\[
		26p+16q-12+8pq-24p-14q+6>0
		\Rightarrow 8pq+2p+2q-6>0
		\]
		
	\end{ExamProblems}

\end{document}
