\makeatletter
\def\input@path{{./}{../}{../../}{preamble/}{../preamble/}{../../preamble/}}
\makeatother
% ----------------------------------------------------------
% GENERAL 

% File
\documentclass[11pt]{book}

% Margins
\usepackage[margin=1in]{geometry}

\linespread{1.2}            % Line spacing
\usepackage[utf8]{inputenc}
\usepackage[T1]{fontenc}
\usepackage{lmodern}
\usepackage{microtype}
\setlength{\parindent}{0pt}
\setlength{\parskip}{6pt}
\usepackage{booktabs}

% ----------------------------------------------------------
% TABLES
\usepackage{multicol}
\usepackage{longtable} 
\usepackage{array}
\usepackage{booktabs}
\usepackage{tabularx}
\usepackage{multirow}

% ----------------------------------------------------------
% MATHEMATICS
\usepackage{amsmath}
\usepackage{amssymb}
\usepackage{amsfonts}
\usepackage{mathtools}

% ----------------------------------------------------------
% HIDDEN CONTENT
\usepackage{ifthen}
% Define a boolean switch
\newboolean{explicaciones}
% Set the boolean switch to true or false
% Change to true to show the content

% Explanations
\newcommand{\explicacion}[2]{
	\ifthenelse{\boolean{explicaciones}}{#1}{#2}
}
\newcommand{\mostrarExplicaciones}[1]{\setboolean{explicaciones}{#1}}

% ----------------------------------------------------------
% NUMBERING

\usepackage{fancyhdr}
\pagestyle{empty} % Ensures the entire document has no page numbers

\usepackage{tocloft}
\renewcommand{\cftdot}{} % Remove dots for sections, if any
\renewcommand{\cftsecleader}{\cftdotfill{\cftdotsep}} % Remove dots for sections, if any
\cftpagenumbersoff{section} % Removes page numbers from sections
\cftpagenumbersoff{subsection} % Removes page numbers from subsections

% ----------------------------------------------------------
% IMAGES 

% General settings
\usepackage{graphicx}       % Insert images
\usepackage{float}          % Position images
% \usepackage{subfigure}      % Subfigures
\graphicspath{{imgs}}       % Image location
\usepackage{subcaption}     % Subfigures II
\usepackage{verbatim}

% Figures
\usepackage{tikz}
\usetikzlibrary{arrows.meta,positioning,trees}

% Colors
\usepackage{xcolor}     
\definecolor{popUp}{HTML}{666666}
\definecolor{popUpIn}{HTML}{CED9E0}
\definecolor{backgroundC}{HTML}{D0E8F2}
\definecolor{backgroundCC}{HTML}{FFFFFF}
\definecolor{borders}{HTML}{8c120d}
\definecolor{padding}{HTML}{77D0D7}
\definecolor{links}{HTML}{CC6F5F}

% ----------------------------------------------------------
% FRAMES

% General settings
\usepackage{tcolorbox}
\usepackage{adjustbox}          % Adjusted frame  
\setlength{\fboxrule}{3pt}  % Line width
\setlength{\fboxsep}{3pt}   % Box padding

% General frames
\usepackage{mdframed}   

\mdfdefinestyle{estiloGeneral}{    % General style
	linecolor=black,
	linewidth=1.5pt,
	roundcorner=10pt,
	backgroundcolor=backgroundC,
	innerbottommargin=0pt
}
\mdfdefinestyle{code}{          % Code style
	linecolor=black,
	linewidth=1.5pt,
	roundcorner=10pt,
	backgroundcolor=darkgray!10,
	innerbottommargin=0pt
}

% Image frame
\newtcbox{\fboxC}{
	colback=backgroundC,
	colframe=popUp,
	arc=10pt,
	boxrule=3pt,
	boxsep=0pt, % Change the padding here
	nobeforeafter
}

% ----------------------------------------------------------
% PAGE SETTINGS

% Background 
\newcommand{\background}[0]{\begin{tikzpicture}[remember picture,overlay]
		\fill[backgroundC] (-2,2) rectangle (25cm, -550);
\end{tikzpicture}}

\newcommand{\backgroundC}[0]{\begin{tikzpicture}[remember picture,overlay]
		\fill[backgroundCC] (-2,2) rectangle (25cm, -550);
\end{tikzpicture}}

% Page width 
\newcommand{\anchoPag}[0]{20cm}

% ----------------------------------------------------------
% FONT

% General
\usepackage{tgbonum}        % Font
\usepackage{listings}       % Code typesetting
\usepackage[spanish]{babel} % Load Spanish
\selectlanguage{spanish}    % Select Spanish
\usepackage{enumitem}
\usepackage{bookmark}

\setlist[itemize]{leftmargin=1.2em, itemsep=0.35em, topsep=0.35em}

% --- Table helpers ---
\newcolumntype{L}[1]{>{\raggedright\arraybackslash}p{#1}}
\newcolumntype{Y}{>{\raggedright\arraybackslash}X}
\newcolumntype{C}{>{\centering\arraybackslash}X}
\renewcommand{\arraystretch}{1.1}

% Python style
\lstdefinestyle{python}{
	language=Python,
	basicstyle=\ttfamily\small,
	commentstyle=\color{green!50!black},
	keywordstyle=\color{blue},
	numberstyle=\tiny\color{gray},
	numbers=left,
	morekeywords={>, <},
	breakatwhitespace=false,
	showstringspaces=false,
	showtabs=false,
	showspaces=false
}

% ----------------------------------------------------------
% HYPERLINKS

% General
\usepackage{hyperref}       
\hypersetup{
	colorlinks=true,
	linkcolor=links,
	filecolor=magenta,      
	urlcolor=brown,
}

% Custom commands 

% Large link
\newcommand{\bigLink}[2]{\begin{center} \fboxC{\LARGE{\href{#1}{#2}}}\end{center}}

% Small link
\newcommand{\smallLink}[2]{\begin{center}\fboxC{\href{#1}{#2}}\end{center}}

% Bold link
\newcommand{\bfLink}[2]{\href{#1}{\textbf{#2}}}


% Small URL
\newcommand{\smallUrl}[1]{\begin{center}\fboxC{\url{#1}}\end{center}}


% ----------------------------------------------------------
% CUSTOM COMMANDS FOR FIGURES

\newcommand{\espacioImagenes}[0]{-1.2cm}

% Without frame
\newcommand{\fig}[3][\espacioImagenes]{
	\hspace*{#1}
	\centering
	\includegraphics[width=#2\textwidth]{#3}
}

% With frame
\newcommand{\ffig}[2]{\begin{figure}[h]
		\hspace*{\espacioImagenes}
		\centering
		\fbox{\includegraphics[width=#1\textwidth]{#2}}
\end{figure}}

% Hyperlink with frame
\newcommand{\hfig}[3]{\begin{figure}[h]
		\hspace*{-1.4cm}
		\centering
		\color{popUp}
		\fboxC{\href{#1}{\includegraphics[width=#2\textwidth]{#3}}}
	\end{figure}
}

% Hyperlink without frame
\newcommand{\hffig}[3]{\begin{figure}[h]
		\hspace*{-1.1cm}
		\centering
		\color{popUp}
		\href{#1}{\includegraphics[width=#2\textwidth]{#3}}
	\end{figure}
}

% ----------------------------------------------------------

% Start and Contents
\newcommand{\cuadro}[1]{
	\begin{mdframed}[style=estiloGeneral]
		#1 
	\end{mdframed}
}

% Explanation video image
\newcommand{\linkExplicacion}[1]{
	\hffig{#1}{0.5}{principal/videoExplicacion}
	\vspace{-0.5cm}
}

\newcommand{\subSecLink}[2]{
	\subsubsection*{\href{#1}{\textbf{#2}}}
}

% Spacing
\newcommand{\esp}[0]{\vspace{4mm}}

% Back to start
\newcommand{\secInicio}[0]{\begin{center}\hyperref[sec:inicio]{ 
			\includegraphics[width=0.1\textwidth]{principal/up}
	}\end{center}
}


\geometry{margin=0.85in}
\AtBeginDocument{\small}

\newcommand{\ExamNameField}{\noindent\textbf{Name:}\ \rule{0.7\linewidth}{0.4pt}\par\medskip}

\newcommand{\ExamTitleBlock}[3]{%
	\begin{center}
		\Large\textbf{#1}\\
		\textbf{#2}%
		\if\relax\detokenize{#3}\relax\else\\\textbf{#3}\fi
	\end{center}
	\vspace{0.5em}
}

\newcommand{\ExamSection}[1]{\par\medskip\textbf{#1}\par\smallskip}

\newenvironment{ExamCriteria}{%
	\begin{itemize}[leftmargin=1.6em, itemsep=0.3em, topsep=0.2em]
}{%
	\end{itemize}
}

\newenvironment{ExamProblems}{%
	\begin{enumerate}[label=\textbf{P\arabic*}, leftmargin=0pt, labelsep=0.6em, itemindent=2.2em, itemsep=0.8em]
}{%
	\end{enumerate}
}

\begin{document}
	\ExamTitleBlock{10th grade}{Term 2 C4 Catch-Up Activity (Solutions)}{}
	
	\ExamSection{Problems}
	\begin{ExamProblems}
		\item
		\subsection*{Problem description}
		A student entrepreneur is evaluating a short-term retail opportunity that depends heavily on how customer traffic develops during the weekend. The decision is straightforward in structure but uncertain in outcome, because actual demand can shift quickly based on consumer mood and local conditions. Management therefore needs a probability-based framework to connect each demand scenario with its corresponding financial consequence and to justify the final choice on expected performance rather than intuition.

		\begin{center}
			\textit{Payoff table} \par
			\begin{tabular}{l c c}
				\toprule
				State of nature & Probability & Rent kiosk \\
				\midrule
				High demand & 0.60 & 36 \\
				Low demand & 0.40 & 10 \\
				\bottomrule
			\end{tabular}
		\end{center}

		Which alternative should be selected according to the maximax criterion?

		\subsection*{Solution}
		\textbf{Step 1 --- Identify the maximum payoff for each alternative.}
		For Rent kiosk: $\max=36$.
		
		\textbf{Step 2 --- Select the alternative with the highest maximum payoff.}
		The largest maximum payoff is $ 36 $, so the maximax choice is Rent kiosk.
		
		\textbf{Step 3 --- Final decision in business language.}
		The maximax criterion is optimistic because it focuses on the best-case outcome for each alternative and selects the option with the greatest upside potential.
		\item
		\subsection*{Problem description}
		A coffee shop chain must select a bean purchasing approach for the next month while demand remains uncertain. One alternative emphasizes product quality and brand positioning, while the other emphasizes sourcing flexibility and cost control. Because the market can evolve in more than one direction, leadership must evaluate how each plan performs across possible demand states and use expected value reasoning to support a disciplined, forward-looking decision.

		\begin{center}
			\textit{Payoff table} \par
			\begin{tabular}{l c c c}
				\toprule
				State of nature & Probability & Plan A & Plan B \\
				\midrule
				High demand & 0.55 & 84 & 72 \\
				Low demand & 0.45 & 22 & 34 \\
				\bottomrule
			\end{tabular}
		\end{center}

		Which alternative should be selected according to the maximax criterion?

		\subsection*{Solution}
		\textbf{Step 1 --- Identify the maximum payoff for each alternative.}
		For Plan A: $\max=84$.
		For Plan B: $\max=72$.
		
		\textbf{Step 2 --- Select the alternative with the highest maximum payoff.}
		The largest maximum payoff is $ 84 $, so the maximax choice is Plan A.
		
		\textbf{Step 3 --- Final decision in business language.}
		The maximax criterion is optimistic because it focuses on the best-case outcome for each alternative and selects the option with the greatest upside potential.
		\item
		\subsection*{Problem description}
		A small factory is planning seasonal production and must choose between two operating modes with different cost structures and responsiveness profiles. Market conditions may strengthen, remain stable, or soften, and each state creates a different profit implication for each mode. Since management cannot know in advance which condition will occur, the decision should be anchored in expected value so that uncertainty is incorporated systematically into the production strategy.

		\begin{center}
			\textit{Payoff table} \par
			\begin{tabular}{l c c c}
				\toprule
				State of nature & Probability & Mode A & Mode B \\
				\midrule
				Strong market & 0.30 & 95 & 88 \\
				Stable market & 0.45 & 60 & 66 \\
				Weak market & 0.25 & 18 & 30 \\
				\bottomrule
			\end{tabular}
		\end{center}

		Which alternative should be selected according to the maximax criterion?

		\subsection*{Solution}
		\textbf{Step 1 --- Identify the maximum payoff for each alternative.}
		For Mode A: $\max=95$.
		For Mode B: $\max=88$.
		
		\textbf{Step 2 --- Select the alternative with the highest maximum payoff.}
		The largest maximum payoff is $ 95 $, so the maximax choice is Mode A.
		
		\textbf{Step 3 --- Final decision in business language.}
		The maximax criterion is optimistic because it focuses on the best-case outcome for each alternative and selects the option with the greatest upside potential.
		\item
		\subsection*{Problem description}
		A delivery startup is selecting a fleet strategy for the coming quarter in an environment where fuel conditions are uncertain and can materially affect operating margins. Each strategic option offers a different balance between control, flexibility, and exposure to cost volatility. To choose responsibly, the firm must evaluate outcomes across the plausible fuel-cost states and rely on expected value to identify the alternative with the strongest overall economic justification.

		\begin{center}
			\textit{Payoff table} \par
			\begin{tabular}{l c c c c}
				\toprule
				State of nature & Probability & Strategy A & Strategy B & Strategy C \\
				\midrule
				Low fuel cost & 0.25 & 120 & 108 & 96 \\
				Medium fuel cost & 0.50 & 86 & 92 & 88 \\
				High fuel cost & 0.25 & 44 & 58 & 70 \\
				\bottomrule
			\end{tabular}
		\end{center}

		Which alternative should be selected according to the maximax criterion?

		\subsection*{Solution}
		\textbf{Step 1 --- Identify the maximum payoff for each alternative.}
		For Strategy A: $\max=120$.
		For Strategy B: $\max=108$.
		For Strategy C: $\max=96$.
		
		\textbf{Step 2 --- Select the alternative with the highest maximum payoff.}
		The largest maximum payoff is $ 120 $, so the maximax choice is Strategy A.
		
		\textbf{Step 3 --- Final decision in business language.}
		The maximax criterion is optimistic because it focuses on the best-case outcome for each alternative and selects the option with the greatest upside potential.
		\item
		\subsection*{Problem description}
		A supermarket is defining an inventory policy for imported fruit while facing uncertainty in supply-chain reliability. Management recognizes that logistics conditions can range from smooth to severely disrupted, and each situation affects availability, waste risk, and revenue potential differently. Because no single scenario is guaranteed, the firm must compare policies using probability-weighted outcomes and select the option that best supports resilient profit performance.

		\begin{center}
			\textit{Payoff table} \par
			\begin{tabular}{l c c c c}
				\toprule
				State of nature & Probability & Policy A & Policy B & Policy C \\
				\midrule
				Very smooth supply & 0.20 & 74 & 82 & 86 \\
				Smooth supply & 0.35 & 68 & 76 & 78 \\
				Disrupted supply & 0.30 & 36 & 50 & 58 \\
				Severely disrupted supply & 0.15 & 8 & 24 & 34 \\
				\bottomrule
			\end{tabular}
		\end{center}

		Which alternative should be selected according to the maximax criterion?

		\subsection*{Solution}
		\textbf{Step 1 --- Identify the maximum payoff for each alternative.}
		For Policy A: $\max=74$.
		For Policy B: $\max=82$.
		For Policy C: $\max=86$.
		
		\textbf{Step 2 --- Select the alternative with the highest maximum payoff.}
		The largest maximum payoff is $ 86 $, so the maximax choice is Policy C.
		
		\textbf{Step 3 --- Final decision in business language.}
		The maximax criterion is optimistic because it focuses on the best-case outcome for each alternative and selects the option with the greatest upside potential.
		\item
		\subsection*{Problem description}
		An electronics retailer is choosing a launch format for a new device in a season where demand intensity may vary significantly. Each format reflects a different channel strategy and operational commitment, leading to distinct financial outcomes under different market conditions. Decision-makers therefore need to assess each format across plausible seasonal states and use expected value analysis to justify the launch approach with the strongest overall return profile.

		\begin{center}
			\textit{Payoff table} \par
			\begin{tabular}{l c c c c c}
				\toprule
				State of nature & Probability & Format A & Format B & Format C & Format D \\
				\midrule
				Boom season & 0.20 & 150 & 132 & 144 & 120 \\
				Good season & 0.30 & 110 & 118 & 124 & 108 \\
				Moderate season & 0.30 & 72 & 82 & 94 & 88 \\
				Slow season & 0.20 & 30 & 48 & 62 & 66 \\
				\bottomrule
			\end{tabular}
		\end{center}

		Which alternative should be selected according to the maximax criterion?

		\subsection*{Solution}
		\textbf{Step 1 --- Identify the maximum payoff for each alternative.}
		For Format A: $\max=150$.
		For Format B: $\max=132$.
		For Format C: $\max=144$.
		For Format D: $\max=120$.
		
		\textbf{Step 2 --- Select the alternative with the highest maximum payoff.}
		The largest maximum payoff is $ 150 $, so the maximax choice is Format A.
		
		\textbf{Step 3 --- Final decision in business language.}
		The maximax criterion is optimistic because it focuses on the best-case outcome for each alternative and selects the option with the greatest upside potential.
		\item
		\subsection*{Problem description}
		A grain exporter must choose among competing shipping contracts while freight market conditions remain uncertain. Contract design influences both upside potential and downside protection as freight tightness changes across the planning horizon. Since management cannot predict a single freight outcome with certainty, it should evaluate all contracts under the full set of plausible states and apply expected value logic to support a defensible commercial decision.

		\begin{center}
			\textit{Payoff table}
			\begin{tabular}{l c c c c c}
				\toprule
				State of nature & Probability & Contract A & Contract B & Contract C & Contract D \\
				\midrule
				Very favorable freight & 0.15 & 160 & 152 & 146 & 132 \\
				Favorable freight & 0.25 & 138 & 140 & 136 & 126 \\
				Neutral freight & 0.30 & 102 & 114 & 120 & 116 \\
				Tight freight & 0.20 & 66 & 80 & 92 & 98 \\
				Very tight freight & 0.10 & 24 & 40 & 54 & 70 \\
				\bottomrule
			\end{tabular}
		\end{center}

		Which alternative should be selected according to the maximax criterion?

		\subsection*{Solution}
		\textbf{Step 1 --- Identify the maximum payoff for each alternative.}
		For Contract A: $\max=160$.
		For Contract B: $\max=152$.
		For Contract C: $\max=146$.
		For Contract D: $\max=132$.
		
		\textbf{Step 2 --- Select the alternative with the highest maximum payoff.}
		The largest maximum payoff is $ 160 $, so the maximax choice is Contract A.
		
		\textbf{Step 3 --- Final decision in business language.}
		The maximax criterion is optimistic because it focuses on the best-case outcome for each alternative and selects the option with the greatest upside potential.
		\item
		\subsection*{Problem description}
		A fashion retailer is evaluating expansion plans for urban markets where demand can shift across several intensity levels. Each plan reflects a different growth posture, creating distinct exposure to both strong and weak market outcomes. To align strategy with financial discipline, management should compare alternatives across all plausible demand states and use expected value as the core criterion for selecting the expansion path.

		\begin{center}
			\textit{Payoff table} \par
			\begin{tabular}{l c c c c c c}
				\toprule
				State of nature & Probability & Plan A & Plan B & Plan C & Plan D & Plan E \\
				\midrule
				Very high demand & 0.12 & 190 & 182 & 176 & 168 & 156 \\
				High demand & 0.23 & 150 & 154 & 152 & 146 & 140 \\
				Medium demand & 0.30 & 108 & 120 & 128 & 126 & 124 \\
				Low demand & 0.22 & 58 & 76 & 90 & 98 & 104 \\
				Very low demand & 0.13 & 12 & 34 & 52 & 64 & 76 \\
				\bottomrule
			\end{tabular}
		\end{center}

		Which alternative should be selected according to the maximax criterion?

		\subsection*{Solution}
		\textbf{Step 1 --- Identify the maximum payoff for each alternative.}
		For Plan A: $\max=190$.
		For Plan B: $\max=182$.
		For Plan C: $\max=176$.
		For Plan D: $\max=168$.
		For Plan E: $\max=156$.
		
		\textbf{Step 2 --- Select the alternative with the highest maximum payoff.}
		The largest maximum payoff is $ 190 $, so the maximax choice is Plan A.
		
		\textbf{Step 3 --- Final decision in business language.}
		The maximax criterion is optimistic because it focuses on the best-case outcome for each alternative and selects the option with the greatest upside potential.
		\item
		\subsection*{Problem description}
		A regional energy distributor is selecting a pricing package in a context where weather-driven demand can change materially during the operating period. Each package offers a different trade-off between high-demand capture and low-demand protection. Because weather patterns are uncertain and financially significant, management needs a probability-weighted evaluation of outcomes to determine which package provides the most robust expected economic result.

		\begin{center}
			\textit{Payoff table}
			\begin{tabular}{l c c c c c c}
				\toprule
				State of nature & Probability & Package A & Package B & Package C & Package D & Package E \\
				\midrule
				Extreme cold & 0.10 & 220 & 210 & 204 & 196 & 186 \\
				Cold & 0.20 & 184 & 186 & 182 & 176 & 170 \\
				Mild & 0.25 & 142 & 152 & 158 & 156 & 152 \\
				Warm & 0.20 & 96 & 110 & 122 & 128 & 130 \\
				Hot & 0.15 & 58 & 74 & 88 & 98 & 106 \\
				Extreme hot & 0.10 & 22 & 40 & 54 & 68 & 82 \\
				\bottomrule
			\end{tabular}
		\end{center}

		Which alternative should be selected according to the maximax criterion?

		\subsection*{Solution}
		\textbf{Step 1 --- Identify the maximum payoff for each alternative.}
		For Package A: $\max=220$.
		For Package B: $\max=210$.
		For Package C: $\max=204$.
		For Package D: $\max=196$.
		For Package E: $\max=186$.
		
		\textbf{Step 2 --- Select the alternative with the highest maximum payoff.}
		The largest maximum payoff is $ 220 $, so the maximax choice is Package A.
		
		\textbf{Step 3 --- Final decision in business language.}
		The maximax criterion is optimistic because it focuses on the best-case outcome for each alternative and selects the option with the greatest upside potential.
		\item
		\subsection*{Problem description}
		A national logistics group is deciding on a network design while macro-economic conditions may evolve through expansion, slowdown, contraction, and recovery patterns. Each design presents a different balance between growth capacity and resilience under weaker environments. Given this broad uncertainty set, executives should compare all designs across the plausible states of nature and rely on expected value analysis to choose the configuration with the strongest expected performance.

		\begin{center}
			\textit{Payoff table}
			\begin{tabular}{l c c c c c c c}
				\toprule
				State of nature & Probability & Design A & Design B & Design C & Design D & Design E & Design F \\
				\midrule
				Rapid expansion & 0.14 & 260 & 248 & 242 & 234 & 226 & 214 \\
				Steady growth & 0.20 & 220 & 222 & 218 & 212 & 206 & 198 \\
				Flat market & 0.24 & 170 & 182 & 188 & 190 & 188 & 184 \\
				Mild contraction & 0.16 & 118 & 132 & 146 & 156 & 162 & 164 \\
				Strong contraction & 0.12 & 70 & 86 & 102 & 118 & 130 & 142 \\
				Recovery transition & 0.14 & 136 & 148 & 158 & 164 & 168 & 170 \\
				\bottomrule
			\end{tabular}
		\end{center}

		Which alternative should be selected according to the maximax criterion?

		\subsection*{Solution}
		\textbf{Step 1 --- Identify the maximum payoff for each alternative.}
		For Design A: $\max=260$.
		For Design B: $\max=248$.
		For Design C: $\max=242$.
		For Design D: $\max=234$.
		For Design E: $\max=226$.
		For Design F: $\max=214$.
		
		\textbf{Step 2 --- Select the alternative with the highest maximum payoff.}
		The largest maximum payoff is $ 260 $, so the maximax choice is Design A.
		
		\textbf{Step 3 --- Final decision in business language.}
		The maximax criterion is optimistic because it focuses on the best-case outcome for each alternative and selects the option with the greatest upside potential.
	\end{ExamProblems}
\end{document}
