\documentclass[12pt]{article}

% Page size and tighter margins
\usepackage[a4paper,left=1.2cm,right=1.2cm,top=1.5cm,bottom=1.5cm]{geometry}

% Core packages
\usepackage{graphicx}
\usepackage{xcolor}
\usepackage{array}
\usepackage{tabularx}
\usepackage{multicol}
\usepackage[T1]{fontenc}
\usepackage[utf8]{inputenc}
\usepackage{booktabs}
\usepackage{amsmath}

\setlength{\parindent}{0pt}
\setlength{\tabcolsep}{6pt}
\renewcommand{\arraystretch}{1.15}

% Column types
\newcolumntype{Y}{>{\raggedright\arraybackslash}m{\dimexpr0.30\textwidth-2\tabcolsep-2\arrayrulewidth\relax}}
\newcolumntype{Z}{>{\raggedright\arraybackslash}m{\dimexpr0.70\textwidth-2\tabcolsep-2\arrayrulewidth\relax}}
\newcolumntype{C}[1]{>{\centering\arraybackslash}m{#1}}

% Gray subsection header box
\newcommand{\SubsectionBox}[1]{%
	\noindent\colorbox{gray!30}{%
		\parbox{\dimexpr\linewidth-2\fboxsep\relax}{\textbf{#1}}%
	}\par\vspace{0.35cm}%
}

% Centered multi-line cell helper
\newcommand{\CellCenter}[1]{%
	\parbox{\linewidth}{\centering #1}%
}

\begin{document}

	\noindent
		\begin{tabularx}{\textwidth}{|C{2.8cm}|X|C{2.8cm}|}
		\hline
		\centering
		\vspace{3mm}
		\includegraphics[width=2.5cm]{../../preamble/logo.png}
		&
		\CellCenter{%
			\vspace{-5mm}
			\textbf{GLOBAL ECONOMICS}\par
			\textbf{GRADE: 10TH}\par
			\textbf{CATCH-UP ACTIVITY}\par
			\textbf{ANALYSIS OF DECISIONS}\par
			\textbf{TEACHER'S NAME: Nicolás López Cuéllar}
		}
		&
		\CellCenter{%
			\textbf{SECOND TERM}\par
			\textbf{2025--2026}%
		}
		\\
		\hline
	\end{tabularx}

	\vspace{0.5cm}

	\noindent
	\begin{tabular}{|Y|Z|}
		\hline
		{\small
			\textbf{Learning objective:} Interpret decision alternatives, events, consequences, and states of nature in applied decision-making scenarios.
		}
		&
		{\footnotesize
			\textbf{Assessment criteria:}\par
			C2: Interprets decision alternatives, events, consequences, and states of nature.\par
		}
		\\
		\hline
	\end{tabular}

	\vspace{0.4cm}

	\begin{multicols}{2}
		\SubsectionBox{Criteria assessment}\vspace{-0.25cm}
		This activity evaluates criterion C2. Read each problem and complete the required decision-analysis tasks.

		\vspace{0.25cm}
		\SubsectionBox{1. Problem description}\vspace{-0.25cm}
		A bottled tea company must choose one of two monthly launch options for a new flavor line:
				Option A (premium glass packaging) or Option B (standard recyclable packaging).
				Actual demand for the month is uncertain and can be high or low.
				Estimated net profits (thousand USD) are already available:
				A gives 98 under high demand and 20 under low demand;
				B gives 80 under high demand and 37 under low demand.

		\vspace{0.25cm}
		\SubsectionBox{2. Problem description}\vspace{-0.25cm}
		A regional home-goods wholesaler must choose one of three stocking policies for a holiday campaign:
				Policy A (premium assortment), Policy B (balanced assortment), or Policy C (value assortment).
				Demand can be strong, moderate, or weak.
				Net profits (thousand USD) are estimated as follows:
				A: 142, 90, 14; B: 124, 96, 40; C: 107, 78, 50.

		\vspace{0.25cm}
		\SubsectionBox{3. Problem description}\vspace{-0.25cm}
		A sportswear producer must choose one of two monthly production mixes:
				Mix A (performance line) or Mix B (basic line).
				Demand may be very high, high, medium, or low.
				Profit equals units sold times contribution per unit.
				Contribution per unit is \$19 for Mix A and \$15 for Mix B.
				Expected units sold are:
				A: 8{,}800 (very high), 7{,}100 (high), 4{,}900 (medium), 2{,}500 (low);
				B: 10{,}000, 8{,}100, 5{,}900, 3{,}300 respectively.

		\vspace{0.25cm}
		\SubsectionBox{4. Problem description}\vspace{-0.25cm}
		A digital news publisher must select one of three distribution plans for next month:
				Plan A (direct subscriptions), Plan B (platform partnership), or Plan C (hybrid channel).
				Demand can be high or low.
				Profit model: \(\text{Profit}=\text{Fixed component}+(\text{Subscribers}\times\text{Contribution per subscriber})\).
				Parameters:
				A: fixed \$25{,}000, contribution \$6.4;
				B: fixed \$16{,}000, contribution \$5.9;
				C: fixed \$20{,}000, contribution \$6.0.
				Subscribers by state:
				High --- A 17{,}500, B 20{,}800, C 19{,}200;
				Low --- A 9{,}200, B 11{,}700, C 10{,}500.

		\vspace{0.25cm}
		\SubsectionBox{5. Problem description}\vspace{-0.25cm}
		A packaging supplier must choose between two contract structures for a seasonal client:
				Contract A or Contract B. Demand may be strong, normal, or weak.
				Contract A profits are given directly (thousand USD): 114, 76, 28.
				Contract B uses
				\(\text{Profit}_B=22{,}000+(\text{Units sold}\times4.6)\),
				with units sold 17{,}500 (strong), 11{,}800 (normal), 7{,}200 (weak).

		\vspace{0.25cm}
		\SubsectionBox{6. Problem description}\vspace{-0.25cm}
		An urban courier company must select one of two fleet plans for a quarterly delivery contract:
				Fleet A or Fleet B.
				Revenue can be high or low; operating cost can be low, medium, or high.
				All values are given directly in thousand USD.
				Revenue: A = 430 (high), 295 (low); B = 458 (high), 305 (low).
				Costs: A = 175, 210, 246 for low/medium/high cost;
				B = 188, 224, 264 for low/medium/high cost.
				Profit equals revenue minus cost.

		\vspace{0.25cm}
		\SubsectionBox{7. Problem description}\vspace{-0.25cm}
		A corporate meal-service provider must choose one of three service plans: A, B, or C.
				Meal demand can be high or low, while input costs can be low or high.
				Revenue is computed from quantity and price; costs are given directly.
				Revenue inputs:
				A: 13{,}800 meals (high), 8{,}900 (low), price \$12.2;
				B: 15{,}200, 10{,}300, price \$11.6;
				C: 13{,}000, 8{,}600, price \$12.5.
				Cost data (thousand USD):
				A: 103 (low cost), 129 (high cost);
				B: 109, 136;
				C: 99, 125.
				Profit = revenue $-$ cost.

		\vspace{0.25cm}
		\SubsectionBox{8. Problem description}\vspace{-0.25cm}
		A vocational training operator must choose between Program A and Program B.
				Tuition revenue is given directly for enrollment-demand states high, medium, and low.
				Costs depend on instructor rates (low-cost or high-cost state) and must be computed from participant counts.
				Revenue (thousand USD):
				A: 365, 285, 212;
				B: 395, 305, 233.
				Cost model: \(\text{Cost}=\text{Participants}\times\text{Unit cost}\).
				A: 980 participants (low-cost state) and 1{,}280 (high-cost state), unit cost \$122.
				B: 1{,}040 and 1{,}340, unit cost \$126.
				Profit = revenue $-$ cost.

		\vspace{0.25cm}
		\SubsectionBox{9. Problem description}\vspace{-0.25cm}
		A smart-appliance firm must choose one of three release configurations:
				Configuration P, Q, or R.
				Both revenue and cost include fixed and variable components.
				Demand can be high, medium, or low; operating condition can be normal or disrupted.
				Revenue models (USD):
				\(R_P=148{,}000+52q\), \(R_Q=132{,}000+55q\), \(R_R=144{,}000+53q\).
				Demand quantities:
				P: 8{,}700, 6{,}900, 5{,}000;
				Q: 8{,}500, 6{,}800, 5{,}200;
				R: 8{,}800, 6{,}950, 5{,}100.
				Cost models (USD):
				\(C_P=121{,}000+27n\), \(C_Q=106{,}000+29n\), \(C_R=113{,}000+28n\).
				Cost activity quantities:
				P: 8{,}100 (normal), 9{,}000 (disrupted);
				Q: 7{,}900, 8{,}900;
				R: 8{,}000, 9{,}000.

		\vspace{0.25cm}
		\SubsectionBox{10. Problem description}\vspace{-0.25cm}
		A specialty coffee chain must decide between two expansion formats for a district rollout:
				Format A (large stores) or Format B (compact stores).
				Demand can be high or low, and ingredient costs can be stable or volatile.
				Projected net profits are already estimated in thousand USD:
				Format A: 210 (high, stable), 184 (high, volatile), 74 (low, stable), 48 (low, volatile).
				Format B: 198 (high, stable), 170 (high, volatile), 92 (low, stable), 64 (low, volatile).

	\end{multicols}
\end{document}
