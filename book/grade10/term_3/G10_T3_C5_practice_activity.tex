\makeatletter
\def\input@path{{./}{../}{../../}{preamble/}{../preamble/}{../../preamble/}}
\makeatother
% ----------------------------------------------------------
% GENERAL 

% File
\documentclass[11pt]{book}

% Margins
\usepackage[margin=1in]{geometry}

\linespread{1.2}            % Line spacing
\usepackage[utf8]{inputenc}
\usepackage[T1]{fontenc}
\usepackage{lmodern}
\usepackage{microtype}
\setlength{\parindent}{0pt}
\setlength{\parskip}{6pt}
\usepackage{booktabs}

% ----------------------------------------------------------
% TABLES
\usepackage{multicol}
\usepackage{longtable} 
\usepackage{array}
\usepackage{booktabs}
\usepackage{tabularx}
\usepackage{multirow}

% ----------------------------------------------------------
% MATHEMATICS
\usepackage{amsmath}
\usepackage{amssymb}
\usepackage{amsfonts}
\usepackage{mathtools}

% ----------------------------------------------------------
% HIDDEN CONTENT
\usepackage{ifthen}
% Define a boolean switch
\newboolean{explicaciones}
% Set the boolean switch to true or false
% Change to true to show the content

% Explanations
\newcommand{\explicacion}[2]{
	\ifthenelse{\boolean{explicaciones}}{#1}{#2}
}
\newcommand{\mostrarExplicaciones}[1]{\setboolean{explicaciones}{#1}}

% ----------------------------------------------------------
% NUMBERING

\usepackage{fancyhdr}
\pagestyle{empty} % Ensures the entire document has no page numbers

\usepackage{tocloft}
\renewcommand{\cftdot}{} % Remove dots for sections, if any
\renewcommand{\cftsecleader}{\cftdotfill{\cftdotsep}} % Remove dots for sections, if any
\cftpagenumbersoff{section} % Removes page numbers from sections
\cftpagenumbersoff{subsection} % Removes page numbers from subsections

% ----------------------------------------------------------
% IMAGES 

% General settings
\usepackage{graphicx}       % Insert images
\usepackage{float}          % Position images
% \usepackage{subfigure}      % Subfigures
\graphicspath{{imgs}}       % Image location
\usepackage{subcaption}     % Subfigures II
\usepackage{verbatim}

% Figures
\usepackage{tikz}
\usetikzlibrary{arrows.meta,positioning,trees}

% Colors
\usepackage{xcolor}     
\definecolor{popUp}{HTML}{666666}
\definecolor{popUpIn}{HTML}{CED9E0}
\definecolor{backgroundC}{HTML}{D0E8F2}
\definecolor{backgroundCC}{HTML}{FFFFFF}
\definecolor{borders}{HTML}{8c120d}
\definecolor{padding}{HTML}{77D0D7}
\definecolor{links}{HTML}{CC6F5F}

% ----------------------------------------------------------
% FRAMES

% General settings
\usepackage{tcolorbox}
\usepackage{adjustbox}          % Adjusted frame  
\setlength{\fboxrule}{3pt}  % Line width
\setlength{\fboxsep}{3pt}   % Box padding

% General frames
\usepackage{mdframed}   

\mdfdefinestyle{estiloGeneral}{    % General style
	linecolor=black,
	linewidth=1.5pt,
	roundcorner=10pt,
	backgroundcolor=backgroundC,
	innerbottommargin=0pt
}
\mdfdefinestyle{code}{          % Code style
	linecolor=black,
	linewidth=1.5pt,
	roundcorner=10pt,
	backgroundcolor=darkgray!10,
	innerbottommargin=0pt
}

% Image frame
\newtcbox{\fboxC}{
	colback=backgroundC,
	colframe=popUp,
	arc=10pt,
	boxrule=3pt,
	boxsep=0pt, % Change the padding here
	nobeforeafter
}

% ----------------------------------------------------------
% PAGE SETTINGS

% Background 
\newcommand{\background}[0]{\begin{tikzpicture}[remember picture,overlay]
		\fill[backgroundC] (-2,2) rectangle (25cm, -550);
\end{tikzpicture}}

\newcommand{\backgroundC}[0]{\begin{tikzpicture}[remember picture,overlay]
		\fill[backgroundCC] (-2,2) rectangle (25cm, -550);
\end{tikzpicture}}

% Page width 
\newcommand{\anchoPag}[0]{20cm}

% ----------------------------------------------------------
% FONT

% General
\usepackage{tgbonum}        % Font
\usepackage{listings}       % Code typesetting
\usepackage[spanish]{babel} % Load Spanish
\selectlanguage{spanish}    % Select Spanish
\usepackage{enumitem}
\usepackage{bookmark}

\setlist[itemize]{leftmargin=1.2em, itemsep=0.35em, topsep=0.35em}

% --- Table helpers ---
\newcolumntype{L}[1]{>{\raggedright\arraybackslash}p{#1}}
\newcolumntype{Y}{>{\raggedright\arraybackslash}X}
\newcolumntype{C}{>{\centering\arraybackslash}X}
\renewcommand{\arraystretch}{1.1}

% Python style
\lstdefinestyle{python}{
	language=Python,
	basicstyle=\ttfamily\small,
	commentstyle=\color{green!50!black},
	keywordstyle=\color{blue},
	numberstyle=\tiny\color{gray},
	numbers=left,
	morekeywords={>, <},
	breakatwhitespace=false,
	showstringspaces=false,
	showtabs=false,
	showspaces=false
}

% ----------------------------------------------------------
% HYPERLINKS

% General
\usepackage{hyperref}       
\hypersetup{
	colorlinks=true,
	linkcolor=links,
	filecolor=magenta,      
	urlcolor=brown,
}

% Custom commands 

% Large link
\newcommand{\bigLink}[2]{\begin{center} \fboxC{\LARGE{\href{#1}{#2}}}\end{center}}

% Small link
\newcommand{\smallLink}[2]{\begin{center}\fboxC{\href{#1}{#2}}\end{center}}

% Bold link
\newcommand{\bfLink}[2]{\href{#1}{\textbf{#2}}}


% Small URL
\newcommand{\smallUrl}[1]{\begin{center}\fboxC{\url{#1}}\end{center}}


% ----------------------------------------------------------
% CUSTOM COMMANDS FOR FIGURES

\newcommand{\espacioImagenes}[0]{-1.2cm}

% Without frame
\newcommand{\fig}[3][\espacioImagenes]{
	\hspace*{#1}
	\centering
	\includegraphics[width=#2\textwidth]{#3}
}

% With frame
\newcommand{\ffig}[2]{\begin{figure}[h]
		\hspace*{\espacioImagenes}
		\centering
		\fbox{\includegraphics[width=#1\textwidth]{#2}}
\end{figure}}

% Hyperlink with frame
\newcommand{\hfig}[3]{\begin{figure}[h]
		\hspace*{-1.4cm}
		\centering
		\color{popUp}
		\fboxC{\href{#1}{\includegraphics[width=#2\textwidth]{#3}}}
	\end{figure}
}

% Hyperlink without frame
\newcommand{\hffig}[3]{\begin{figure}[h]
		\hspace*{-1.1cm}
		\centering
		\color{popUp}
		\href{#1}{\includegraphics[width=#2\textwidth]{#3}}
	\end{figure}
}

% ----------------------------------------------------------

% Start and Contents
\newcommand{\cuadro}[1]{
	\begin{mdframed}[style=estiloGeneral]
		#1 
	\end{mdframed}
}

% Explanation video image
\newcommand{\linkExplicacion}[1]{
	\hffig{#1}{0.5}{principal/videoExplicacion}
	\vspace{-0.5cm}
}

\newcommand{\subSecLink}[2]{
	\subsubsection*{\href{#1}{\textbf{#2}}}
}

% Spacing
\newcommand{\esp}[0]{\vspace{4mm}}

% Back to start
\newcommand{\secInicio}[0]{\begin{center}\hyperref[sec:inicio]{ 
			\includegraphics[width=0.1\textwidth]{principal/up}
	}\end{center}
}


\geometry{margin=0.85in}
\AtBeginDocument{\small}

\newcommand{\ExamNameField}{\noindent\textbf{Name:}\ \rule{0.7\linewidth}{0.4pt}\par\medskip}

\newcommand{\ExamTitleBlock}[3]{%
	\begin{center}
		\Large\textbf{#1}\\
		\textbf{#2}%
		\if\relax\detokenize{#3}\relax\else\\\textbf{#3}\fi
	\end{center}
	\vspace{0.5em}
}

\newcommand{\ExamSection}[1]{\par\medskip\textbf{#1}\par\smallskip}

\newenvironment{ExamCriteria}{%
	\begin{itemize}[leftmargin=1.6em, itemsep=0.3em, topsep=0.2em]
}{%
	\end{itemize}
}

\newenvironment{ExamProblems}{%
	\begin{enumerate}[label=\textbf{P\arabic*}, leftmargin=0pt, labelsep=0.6em, itemindent=2.2em, itemsep=0.8em]
}{%
	\end{enumerate}
}

\begin{document}
\ExamTitleBlock{10th grade}{Term 3 Practice Activity: C5 Standard Normal Table}{}
\ExamSection{C5 Uses the normal standard distribution and its tables to calculate probabilities in context situations.}

\noindent\textbf{Standard normal table (values of \(\Phi(z)\)).} Use this same table for all problems in this section.
\begin{center}
\begin{tabular}{c|cccccccccc}
	\toprule
	$z$ & 0.00 & 0.01 & 0.02 & 0.03 & 0.04 & 0.05 & 0.06 & 0.07 & 0.08 & 0.09 \\
	\midrule
	0.0 & 0.5000 & 0.5040 & 0.5080 & 0.5120 & 0.5160 & 0.5199 & 0.5239 & 0.5279 & 0.5319 & 0.5359 \\
	0.1 & 0.5398 & 0.5438 & 0.5478 & 0.5517 & 0.5557 & 0.5596 & 0.5636 & 0.5675 & 0.5714 & 0.5753 \\
	0.2 & 0.5793 & 0.5832 & 0.5871 & 0.5910 & 0.5948 & 0.5987 & 0.6026 & 0.6064 & 0.6103 & 0.6141 \\
	0.3 & 0.6179 & 0.6217 & 0.6255 & 0.6293 & 0.6331 & 0.6368 & 0.6406 & 0.6443 & 0.6480 & 0.6517 \\
	0.4 & 0.6554 & 0.6591 & 0.6628 & 0.6664 & 0.6700 & 0.6736 & 0.6772 & 0.6808 & 0.6844 & 0.6879 \\
	0.5 & 0.6915 & 0.6950 & 0.6985 & 0.7019 & 0.7054 & 0.7088 & 0.7123 & 0.7157 & 0.7190 & 0.7224 \\
	0.6 & 0.7257 & 0.7291 & 0.7324 & 0.7357 & 0.7389 & 0.7422 & 0.7454 & 0.7486 & 0.7517 & 0.7549 \\
	0.7 & 0.7580 & 0.7611 & 0.7642 & 0.7673 & 0.7704 & 0.7734 & 0.7764 & 0.7794 & 0.7823 & 0.7852 \\
	0.8 & 0.7881 & 0.7910 & 0.7939 & 0.7967 & 0.7995 & 0.8023 & 0.8051 & 0.8078 & 0.8106 & 0.8133 \\
	0.9 & 0.8159 & 0.8186 & 0.8212 & 0.8238 & 0.8264 & 0.8289 & 0.8315 & 0.8340 & 0.8365 & 0.8389 \\
	1.0 & 0.8413 & 0.8438 & 0.8461 & 0.8485 & 0.8508 & 0.8531 & 0.8554 & 0.8577 & 0.8599 & 0.8621 \\
	1.1 & 0.8643 & 0.8665 & 0.8686 & 0.8708 & 0.8729 & 0.8749 & 0.8770 & 0.8790 & 0.8810 & 0.8830 \\
	1.2 & 0.8849 & 0.8869 & 0.8888 & 0.8907 & 0.8925 & 0.8944 & 0.8962 & 0.8980 & 0.8997 & 0.9015 \\
	1.3 & 0.9032 & 0.9049 & 0.9066 & 0.9082 & 0.9099 & 0.9115 & 0.9131 & 0.9147 & 0.9162 & 0.9177 \\
	1.4 & 0.9192 & 0.9207 & 0.9222 & 0.9236 & 0.9251 & 0.9265 & 0.9279 & 0.9292 & 0.9306 & 0.9319 \\
	1.5 & 0.9332 & 0.9345 & 0.9357 & 0.9370 & 0.9382 & 0.9394 & 0.9406 & 0.9418 & 0.9429 & 0.9441 \\
	1.6 & 0.9452 & 0.9463 & 0.9474 & 0.9484 & 0.9495 & 0.9505 & 0.9515 & 0.9525 & 0.9535 & 0.9545 \\
	1.7 & 0.9554 & 0.9564 & 0.9573 & 0.9582 & 0.9591 & 0.9599 & 0.9608 & 0.9616 & 0.9625 & 0.9633 \\
	1.8 & 0.9641 & 0.9649 & 0.9656 & 0.9664 & 0.9671 & 0.9678 & 0.9686 & 0.9693 & 0.9699 & 0.9706 \\
	1.9 & 0.9713 & 0.9719 & 0.9726 & 0.9732 & 0.9738 & 0.9744 & 0.9750 & 0.9756 & 0.9761 & 0.9767 \\
	2.0 & 0.9772 & 0.9778 & 0.9783 & 0.9788 & 0.9793 & 0.9798 & 0.9803 & 0.9808 & 0.9812 & 0.9817 \\
	\bottomrule
\end{tabular}
\end{center}

\begin{ExamProblems}

\item
\subsection*{Problem 1 --- Standard Normal Table Practice IV}

\textbf{Problem.}
Let \(Z\sim N(0,1)\).

\textbf{Question.} Use the table to find \(P(Z<0.35)\) and \(P(Z>1.40)\).

\textbf{Solution.}
From the table, \(\Phi(0.35)=0.6368\), so
\[
P(Z<0.35)=0.6368.
\]
Also \(\Phi(1.40)=0.9192\), hence
\[
P(Z>1.40)=1-\Phi(1.40)=1-0.9192=0.0808.
\]
\textbf{Interpretation.} About 63.7\% of values are below 0.35, and 8.1\% are above 1.40.

% --------------------------------------------------

\item
\subsection*{Problem 2 --- Central Band with a Negative Bound}

\textbf{Problem.}
Let \(Z\sim N(0,1)\).

\textbf{Question.} Use the table to compute \(P(-0.55\le Z\le 1.15)\).

\textbf{Solution.}
From the table, \(\Phi(1.15)=0.8749\) and \(\Phi(0.55)=0.7088\).
Using symmetry,
\[
\Phi(-0.55)=1-\Phi(0.55)=1-0.7088=0.2912.
\]
Therefore,
\[
P(-0.55\le Z\le 1.15)=\Phi(1.15)-\Phi(-0.55)=0.8749-0.2912=0.5837.
\]
\textbf{Interpretation.} About 58.4\% of values lie between -0.55 and 1.15.

% --------------------------------------------------

\item
\subsection*{Problem 3 --- Two-Tail Union I}

\textbf{Problem.}
Let \(Z\sim N(0,1)\).

\textbf{Question.} Find \(P(Z\le -0.95\ \text{or}\ Z\ge 0.85)\).

\textbf{Solution.}
From the table, \(\Phi(0.95)=0.8289\) and \(\Phi(0.85)=0.8023\).
So
\[
P(Z\le -0.95)=1-\Phi(0.95)=1-0.8289=0.1711,
\]
\[
P(Z\ge 0.85)=1-\Phi(0.85)=1-0.8023=0.1977.
\]
The two events are disjoint, hence
\[
P(Z\le -0.95\ \text{or}\ Z\ge 0.85)=0.1711+0.1977=0.3688.
\]
\textbf{Interpretation.} About 36.9\% of values are in these two outer tails.

% --------------------------------------------------

\item
\subsection*{Problem 4 --- Complement with Absolute Value}

\textbf{Problem.}
Let \(Z\sim N(0,1)\).

\textbf{Question.} Find \(P(|Z|\le 1.05)\) and \(P(|Z|\ge 1.05)\).

\textbf{Solution.}
From the table, \(\Phi(1.05)=0.8531\).
Using symmetry,
\[
\Phi(-1.05)=1-\Phi(1.05)=1-0.8531=0.1469.
\]
Then
\[
P(|Z|\le 1.05)=P(-1.05\le Z\le 1.05)=\Phi(1.05)-\Phi(-1.05)=0.8531-0.1469=0.7062.
\]
Using complements,
\[
P(|Z|\ge 1.05)=1-P(|Z|\le 1.05)=1-0.7062=0.2938.
\]
\textbf{Interpretation.} About 70.6\% of values are within 1.05 standard units of 0, and 29.4\% are outside.

% --------------------------------------------------

\item
\subsection*{Problem 5 --- Interval Across Zero}

\textbf{Problem.}
Let \(Z\sim N(0,1)\).

\textbf{Question.} Use the table to compute \(P(-1.65\le Z\le 0.70)\).

\textbf{Solution.}
From the table, \(\Phi(0.70)=0.7580\) and \(\Phi(1.65)=0.9505\).
By symmetry,
\[
\Phi(-1.65)=1-\Phi(1.65)=1-0.9505=0.0495.
\]
So
\[
P(-1.65\le Z\le 0.70)=\Phi(0.70)-\Phi(-1.65)=0.7580-0.0495=0.7085.
\]
\textbf{Interpretation.} About 70.9\% of values lie from -1.65 up to 0.70.

% --------------------------------------------------

\item
\subsection*{Problem 6 --- Two-Tail Union II}

\textbf{Problem.}
Let \(Z\sim N(0,1)\).

\textbf{Question.} Find \(P(Z\le -1.20\ \text{or}\ Z\ge 0.30)\).

\textbf{Solution.}
From the table, \(\Phi(1.20)=0.8849\) and \(\Phi(0.30)=0.6179\).
Thus
\[
P(Z\le -1.20)=1-\Phi(1.20)=1-0.8849=0.1151,
\]
\[
P(Z\ge 0.30)=1-\Phi(0.30)=1-0.6179=0.3821.
\]
Therefore,
\[
P(Z\le -1.20\ \text{or}\ Z\ge 0.30)=0.1151+0.3821=0.4972.
\]
\textbf{Interpretation.} The probability in these two tails is about 49.7\%.

% --------------------------------------------------

\item
\subsection*{Problem 7 --- Upper Slice and Upper Tail}

\textbf{Problem.}
Let \(Z\sim N(0,1)\).

\textbf{Question.} Calculate \(P(0.45<Z<1.90)\) and \(P(Z>1.90)\).

\textbf{Solution.}
From the table, \(\Phi(1.90)=0.9713\) and \(\Phi(0.45)=0.6736\).
So
\[
P(0.45<Z<1.90)=\Phi(1.90)-\Phi(0.45)=0.9713-0.6736=0.2977.
\]
Also,
\[
P(Z>1.90)=1-\Phi(1.90)=1-0.9713=0.0287.
\]
\textbf{Interpretation.} About 29.8\% of values lie between 0.45 and 1.90, while only 2.9\% lie above 1.90.

% --------------------------------------------------

\item
\subsection*{Problem 8 --- Interval with a Small Negative Bound}

\textbf{Problem.}
Let \(Z\sim N(0,1)\).

\textbf{Question.} Use the table to find \(P(-0.25\le Z\le 1.35)\).

\textbf{Solution.}
From the table, \(\Phi(1.35)=0.9115\) and \(\Phi(0.25)=0.5987\).
Using symmetry,
\[
\Phi(-0.25)=1-\Phi(0.25)=1-0.5987=0.4013.
\]
Hence
\[
P(-0.25\le Z\le 1.35)=\Phi(1.35)-\Phi(-0.25)=0.9115-0.4013=0.5102.
\]
\textbf{Interpretation.} About 51.0\% of values are between -0.25 and 1.35.

% --------------------------------------------------

\item
\subsection*{Problem 9 --- Two-Tail Union III}

\textbf{Problem.}
Let \(Z\sim N(0,1)\).

\textbf{Question.} Determine \(P(Z\le -1.80\ \text{or}\ Z\ge 0.65)\).

\textbf{Solution.}
From the table, \(\Phi(1.80)=0.9641\) and \(\Phi(0.65)=0.7422\).
So
\[
P(Z\le -1.80)=1-\Phi(1.80)=1-0.9641=0.0359,
\]
\[
P(Z\ge 0.65)=1-\Phi(0.65)=1-0.7422=0.2578.
\]
Then
\[
P(Z\le -1.80\ \text{or}\ Z\ge 0.65)=0.0359+0.2578=0.2937.
\]
\textbf{Interpretation.} About 29.4\% of values fall in these two tails.

% --------------------------------------------------

\item
\subsection*{Problem 10 --- Symmetry Near Zero}

\textbf{Problem.}
Let \(Z\sim N(0,1)\).

\textbf{Question.} Find \(P(Z<-1.40)\) and \(P(-1.40\le Z\le 0)\).

\textbf{Solution.}
From the table, \(\Phi(1.40)=0.9192\) and \(\Phi(0)=0.5000\).
Using symmetry,
\[
P(Z<-1.40)=\Phi(-1.40)=1-\Phi(1.40)=1-0.9192=0.0808.
\]
Then
\[
P(-1.40\le Z\le 0)=\Phi(0)-\Phi(-1.40)=0.5000-0.0808=0.4192.
\]
\textbf{Interpretation.} About 8.1\% of values lie below -1.40, and 41.9\% lie between -1.40 and 0.

\end{ExamProblems}
\end{document}
