\makeatletter
\def\input@path{{./}{../}{../../}{preamble/}{../preamble/}{../../preamble/}}
\makeatother
% ----------------------------------------------------------
% GENERAL 

% File
\documentclass[11pt]{book}

% Margins
\usepackage[margin=1in]{geometry}

\linespread{1.2}            % Line spacing
\usepackage[utf8]{inputenc}
\usepackage[T1]{fontenc}
\usepackage{lmodern}
\usepackage{microtype}
\setlength{\parindent}{0pt}
\setlength{\parskip}{6pt}
\usepackage{booktabs}

% ----------------------------------------------------------
% TABLES
\usepackage{multicol}
\usepackage{longtable} 
\usepackage{array}
\usepackage{booktabs}
\usepackage{tabularx}
\usepackage{multirow}

% ----------------------------------------------------------
% MATHEMATICS
\usepackage{amsmath}
\usepackage{amssymb}
\usepackage{amsfonts}
\usepackage{mathtools}

% ----------------------------------------------------------
% HIDDEN CONTENT
\usepackage{ifthen}
% Define a boolean switch
\newboolean{explicaciones}
% Set the boolean switch to true or false
% Change to true to show the content

% Explanations
\newcommand{\explicacion}[2]{
	\ifthenelse{\boolean{explicaciones}}{#1}{#2}
}
\newcommand{\mostrarExplicaciones}[1]{\setboolean{explicaciones}{#1}}

% ----------------------------------------------------------
% NUMBERING

\usepackage{fancyhdr}
\pagestyle{empty} % Ensures the entire document has no page numbers

\usepackage{tocloft}
\renewcommand{\cftdot}{} % Remove dots for sections, if any
\renewcommand{\cftsecleader}{\cftdotfill{\cftdotsep}} % Remove dots for sections, if any
\cftpagenumbersoff{section} % Removes page numbers from sections
\cftpagenumbersoff{subsection} % Removes page numbers from subsections

% ----------------------------------------------------------
% IMAGES 

% General settings
\usepackage{graphicx}       % Insert images
\usepackage{float}          % Position images
% \usepackage{subfigure}      % Subfigures
\graphicspath{{imgs}}       % Image location
\usepackage{subcaption}     % Subfigures II
\usepackage{verbatim}

% Figures
\usepackage{tikz}
\usetikzlibrary{arrows.meta,positioning,trees}

% Colors
\usepackage{xcolor}     
\definecolor{popUp}{HTML}{666666}
\definecolor{popUpIn}{HTML}{CED9E0}
\definecolor{backgroundC}{HTML}{D0E8F2}
\definecolor{backgroundCC}{HTML}{FFFFFF}
\definecolor{borders}{HTML}{8c120d}
\definecolor{padding}{HTML}{77D0D7}
\definecolor{links}{HTML}{CC6F5F}

% ----------------------------------------------------------
% FRAMES

% General settings
\usepackage{tcolorbox}
\usepackage{adjustbox}          % Adjusted frame  
\setlength{\fboxrule}{3pt}  % Line width
\setlength{\fboxsep}{3pt}   % Box padding

% General frames
\usepackage{mdframed}   

\mdfdefinestyle{estiloGeneral}{    % General style
	linecolor=black,
	linewidth=1.5pt,
	roundcorner=10pt,
	backgroundcolor=backgroundC,
	innerbottommargin=0pt
}
\mdfdefinestyle{code}{          % Code style
	linecolor=black,
	linewidth=1.5pt,
	roundcorner=10pt,
	backgroundcolor=darkgray!10,
	innerbottommargin=0pt
}

% Image frame
\newtcbox{\fboxC}{
	colback=backgroundC,
	colframe=popUp,
	arc=10pt,
	boxrule=3pt,
	boxsep=0pt, % Change the padding here
	nobeforeafter
}

% ----------------------------------------------------------
% PAGE SETTINGS

% Background 
\newcommand{\background}[0]{\begin{tikzpicture}[remember picture,overlay]
		\fill[backgroundC] (-2,2) rectangle (25cm, -550);
\end{tikzpicture}}

\newcommand{\backgroundC}[0]{\begin{tikzpicture}[remember picture,overlay]
		\fill[backgroundCC] (-2,2) rectangle (25cm, -550);
\end{tikzpicture}}

% Page width 
\newcommand{\anchoPag}[0]{20cm}

% ----------------------------------------------------------
% FONT

% General
\usepackage{tgbonum}        % Font
\usepackage{listings}       % Code typesetting
\usepackage[spanish]{babel} % Load Spanish
\selectlanguage{spanish}    % Select Spanish
\usepackage{enumitem}
\usepackage{bookmark}

\setlist[itemize]{leftmargin=1.2em, itemsep=0.35em, topsep=0.35em}

% --- Table helpers ---
\newcolumntype{L}[1]{>{\raggedright\arraybackslash}p{#1}}
\newcolumntype{Y}{>{\raggedright\arraybackslash}X}
\newcolumntype{C}{>{\centering\arraybackslash}X}
\renewcommand{\arraystretch}{1.1}

% Python style
\lstdefinestyle{python}{
	language=Python,
	basicstyle=\ttfamily\small,
	commentstyle=\color{green!50!black},
	keywordstyle=\color{blue},
	numberstyle=\tiny\color{gray},
	numbers=left,
	morekeywords={>, <},
	breakatwhitespace=false,
	showstringspaces=false,
	showtabs=false,
	showspaces=false
}

% ----------------------------------------------------------
% HYPERLINKS

% General
\usepackage{hyperref}       
\hypersetup{
	colorlinks=true,
	linkcolor=links,
	filecolor=magenta,      
	urlcolor=brown,
}

% Custom commands 

% Large link
\newcommand{\bigLink}[2]{\begin{center} \fboxC{\LARGE{\href{#1}{#2}}}\end{center}}

% Small link
\newcommand{\smallLink}[2]{\begin{center}\fboxC{\href{#1}{#2}}\end{center}}

% Bold link
\newcommand{\bfLink}[2]{\href{#1}{\textbf{#2}}}


% Small URL
\newcommand{\smallUrl}[1]{\begin{center}\fboxC{\url{#1}}\end{center}}


% ----------------------------------------------------------
% CUSTOM COMMANDS FOR FIGURES

\newcommand{\espacioImagenes}[0]{-1.2cm}

% Without frame
\newcommand{\fig}[3][\espacioImagenes]{
	\hspace*{#1}
	\centering
	\includegraphics[width=#2\textwidth]{#3}
}

% With frame
\newcommand{\ffig}[2]{\begin{figure}[h]
		\hspace*{\espacioImagenes}
		\centering
		\fbox{\includegraphics[width=#1\textwidth]{#2}}
\end{figure}}

% Hyperlink with frame
\newcommand{\hfig}[3]{\begin{figure}[h]
		\hspace*{-1.4cm}
		\centering
		\color{popUp}
		\fboxC{\href{#1}{\includegraphics[width=#2\textwidth]{#3}}}
	\end{figure}
}

% Hyperlink without frame
\newcommand{\hffig}[3]{\begin{figure}[h]
		\hspace*{-1.1cm}
		\centering
		\color{popUp}
		\href{#1}{\includegraphics[width=#2\textwidth]{#3}}
	\end{figure}
}

% ----------------------------------------------------------

% Start and Contents
\newcommand{\cuadro}[1]{
	\begin{mdframed}[style=estiloGeneral]
		#1 
	\end{mdframed}
}

% Explanation video image
\newcommand{\linkExplicacion}[1]{
	\hffig{#1}{0.5}{principal/videoExplicacion}
	\vspace{-0.5cm}
}

\newcommand{\subSecLink}[2]{
	\subsubsection*{\href{#1}{\textbf{#2}}}
}

% Spacing
\newcommand{\esp}[0]{\vspace{4mm}}

% Back to start
\newcommand{\secInicio}[0]{\begin{center}\hyperref[sec:inicio]{ 
			\includegraphics[width=0.1\textwidth]{principal/up}
	}\end{center}
}


\geometry{margin=0.85in}
\AtBeginDocument{\small}

\newcommand{\ExamNameField}{\noindent\textbf{Name:}\ \rule{0.7\linewidth}{0.4pt}\par\medskip}

\newcommand{\ExamTitleBlock}[3]{%
	\begin{center}
		\Large\textbf{#1}\\
		\textbf{#2}%
		\if\relax\detokenize{#3}\relax\else\\\textbf{#3}\fi
	\end{center}
	\vspace{0.5em}
}

\newcommand{\ExamSection}[1]{\par\medskip\textbf{#1}\par\smallskip}

\newenvironment{ExamCriteria}{%
	\begin{itemize}[leftmargin=1.6em, itemsep=0.3em, topsep=0.2em]
}{%
	\end{itemize}
}

\newenvironment{ExamProblems}{%
	\begin{enumerate}[label=\textbf{P\arabic*}, leftmargin=0pt, labelsep=0.6em, itemindent=2.2em, itemsep=0.8em]
}{%
	\end{enumerate}
}


\begin{document}
\ExamTitleBlock{10th grade}{Term 3 Practice Activity: C4 Variance and Standard Deviation}{}
\ExamSection{C4 Interprets the concept of variance and standard deviation of a probability function.}

\begin{ExamProblems}

\item
\subsection*{Problem 1 — Streaming Session Length (Uniform)}

\textbf{Problem.}
A student's daily streaming time \(X\) (minutes) is modeled by a continuous uniform distribution on \([2,10]\):
\[
X\sim \text{Uniform}(2,10).
\]

\textbf{Question.} Find \(E[X]\), \(\mathrm{Var}(X)\), and the standard deviation. Then interpret the spread.

\textbf{Solution.}
For \(X\sim\text{Uniform}(a,b)\):
\[
E[X]=\frac{a+b}{2},\qquad \mathrm{Var}(X)=\frac{(b-a)^2}{12}.
\]
Here \(a=2\), \(b=10\):
\[
E[X]=\frac{2+10}{2}=6,
\]
\[
\mathrm{Var}(X)=\frac{(10-2)^2}{12}=\frac{8^2}{12}=\frac{64}{12}=\frac{16}{3}.
\]
Standard deviation:
\[
\sigma=\sqrt{\frac{16}{3}}=\frac{4}{\sqrt{3}}\approx 2.31.
\]

\textbf{Interpretation.}
The variance \(\frac{16}{3}\) measures squared spread around 6 minutes, and a standard deviation of about 2.31 minutes means session times are typically about 2 to 3 minutes away from the mean.

% --------------------------------------------------

\item
\subsection*{Problem 2 — Test Score Model (Normal)}

\textbf{Problem.}
A math quiz score \(S\) is normally distributed with mean 72 points and standard deviation 6 points:
\[
S\sim N(72,6^2).
\]

\textbf{Question.} State the expected value, variance, and standard deviation. Interpret the spread.

\textbf{Solution.}
For a normal distribution \(N(\mu,\sigma^2)\):
\[
E[S]=\mu,\qquad \mathrm{Var}(S)=\sigma^2,\qquad \text{SD}=\sigma.
\]
So:
\[
E[S]=72,\qquad \mathrm{Var}(S)=6^2=36,\qquad \sigma=6.
\]

\textbf{Interpretation.}
The variance 36 shows squared score variability around 72, and the standard deviation of 6 points means scores are commonly around 6 points above or below the mean.

% --------------------------------------------------

\item
\subsection*{Problem 3 — Delivery Distance (Uniform)}

\textbf{Problem.}
A local delivery app models one-trip distance \(D\) (km) as
\[
D\sim \text{Uniform}(0,6).
\]

\textbf{Question.} Compute \(E[D]\), \(\mathrm{Var}(D)\), and the standard deviation. Interpret the spread.

\textbf{Solution.}
Using uniform formulas:
\[
E[D]=\frac{0+6}{2}=3,
\]
\[
\mathrm{Var}(D)=\frac{(6-0)^2}{12}=\frac{36}{12}=3.
\]
Standard deviation:
\[
\sigma=\sqrt{3}\approx 1.73.
\]

\textbf{Interpretation.}
The variance 3 \(\text{km}^2\) and standard deviation about 1.73 km describe how much trip distances vary around the average distance of 3 km.

% --------------------------------------------------

\item
\subsection*{Problem 4 — Refund Amounts (Discrete)}

\textbf{Problem.}
A store's random refund amount \(R\) (dollars) has this probability distribution:
\[
\begin{array}{c|cccc}
R & 0 & 5 & 10 & 15 \\\hline
P(R=r) & 0.10 & 0.30 & 0.40 & 0.20
\end{array}
\]

\textbf{Question.} Find \(E[R]\), \(\mathrm{Var}(R)\), and the standard deviation. Interpret the spread.

\textbf{Solution.}
Use summation formulas:
\[
E[R]=\sum r\,P(R=r),\qquad \mathrm{Var}(R)=E[R^2]-\big(E[R]\big)^2.
\]
First,
\[
E[R]=0(0.10)+5(0.30)+10(0.40)+15(0.20)=0+1.5+4+3=8.5.
\]
Now,
\[
E[R^2]=0^2(0.10)+5^2(0.30)+10^2(0.40)+15^2(0.20)
=0+7.5+40+45=92.5.
\]
So,
\[
\mathrm{Var}(R)=92.5-(8.5)^2=92.5-72.25=20.25.
\]
Standard deviation:
\[
\sigma=\sqrt{20.25}=4.5.
\]

\textbf{Interpretation.}
The standard deviation of \$4.50 means refund values typically differ from the mean refund (\$8.50) by about four to five dollars.

% --------------------------------------------------

\item
\subsection*{Problem 5 — Study Session Duration (Uniform)}

\textbf{Problem.}
A Grade 10 student studies between 4 and 16 minutes for a quick review activity, modeled by
\[
T\sim \text{Uniform}(4,16).
\]

\textbf{Question.} Compute \(E[T]\), \(\mathrm{Var}(T)\), and the standard deviation. Interpret the spread.

\textbf{Solution.}
\[
E[T]=\frac{4+16}{2}=10,
\]
\[
\mathrm{Var}(T)=\frac{(16-4)^2}{12}=\frac{12^2}{12}=12.
\]
Standard deviation:
\[
\sigma=\sqrt{12}=2\sqrt{3}\approx 3.46.
\]

\textbf{Interpretation.}
The variance 12 \(\text{min}^2\) and standard deviation about 3.46 minutes indicate moderate spread around the 10-minute average review time.

% --------------------------------------------------

\item
\subsection*{Problem 6 — Temperature Drift (Linear Density)}

\textbf{Problem.}
A classroom temperature drift \(X\) (in \(^\circ\)C above baseline) follows a symmetric triangular model on \([0,12]\) with peak at 6. For this model, use the known results:
\[
E[X]=\frac{a+b}{2},\qquad \mathrm{Var}(X)=\frac{(b-a)^2}{24}.
\]

\textbf{Question.} Compute \(E[X]\), \(\mathrm{Var}(X)\), and the standard deviation. Interpret the spread.

\textbf{Solution.}
Here \(a=0\), \(b=12\), so
\[
E[X]=\frac{0+12}{2}=6.
\]
Using the stated triangular-variance formula:
\[
\mathrm{Var}(X)=\frac{(12-0)^2}{24}=\frac{144}{24}=6.
\]
Standard deviation:
\[
\sigma=\sqrt{6}\approx 2.45.
\]

\textbf{Interpretation.}
The standard deviation of about \(2.45^\circ\)C means typical temperature drift is around 2 to 3 degrees from the mean drift of \(6^\circ\)C.

% --------------------------------------------------

\item
\subsection*{Problem 7 — Package Weight (Uniform)}

\textbf{Problem.}
A shipping center models package weight \(W\) (kg) by
\[
W\sim \text{Uniform}(1,9).
\]

\textbf{Question.} Find \(E[W]\), \(\mathrm{Var}(W)\), and the standard deviation. Interpret the spread.

\textbf{Solution.}
\[
E[W]=\frac{1+9}{2}=5,
\]
\[
\mathrm{Var}(W)=\frac{(9-1)^2}{12}=\frac{64}{12}=\frac{16}{3}.
\]
Standard deviation:
\[
\sigma=\sqrt{\frac{16}{3}}\approx 2.31.
\]

\textbf{Interpretation.}
The variance \(\frac{16}{3}\) and standard deviation about 2.31 kg show that package weights are often a couple of kilograms away from the 5 kg average.

% --------------------------------------------------

\item
\subsection*{Problem 8 — Battery Life (Normal)}

\textbf{Problem.}
A phone battery life \(B\) (hours) is modeled by
\[
B\sim N(18,2.5^2).
\]

\textbf{Question.} State \(E[B]\), \(\mathrm{Var}(B)\), and the standard deviation. Interpret the spread.

\textbf{Solution.}
From the normal form \(N(\mu,\sigma^2)\):
\[
E[B]=\mu=18,
\]
\[
\mathrm{Var}(B)=\sigma^2=(2.5)^2=6.25,
\]
\[
\text{SD}=\sigma=2.5.
\]

\textbf{Interpretation.}
A standard deviation of 2.5 hours means battery life commonly varies by about 2.5 hours around the 18-hour average.

% --------------------------------------------------

\item
\subsection*{Problem 9 — Ride-share Earnings (Uniform)}

\textbf{Problem.}
A student driver's one-trip earning \(E\) (dollars) is modeled as
\[
E\sim \text{Uniform}(20,32).
\]

\textbf{Question.} Compute the expected value, variance, and standard deviation. Interpret the spread.

\textbf{Solution.}
\[
E[E]=\frac{20+32}{2}=26,
\]
\[
\mathrm{Var}(E)=\frac{(32-20)^2}{12}=\frac{12^2}{12}=12.
\]
Standard deviation:
\[
\sigma=\sqrt{12}=2\sqrt{3}\approx 3.46.
\]

\textbf{Interpretation.}
The standard deviation of about \$3.46 tells us trip earnings are typically around three to four dollars from the mean earning of \$26.

% --------------------------------------------------

\item
\subsection*{Problem 10 — App Usage by Time Block (Piecewise Constant)}

\textbf{Problem.}
Daily app usage time \(U\) (hours) has piecewise constant density:
\[
f(u)=\begin{cases}
\frac{1}{8}, & 1\le u<3,\\[4pt]
\frac{3}{8}, & 3\le u\le 5.
\end{cases}
\]
This means probability \(\frac{1}{4}\) on \([1,3]\) and probability \(\frac{3}{4}\) on \([3,5]\).

\textbf{Question.} Find \(E[U]\), \(\mathrm{Var}(U)\), and standard deviation using weighted averages and finite sums. Interpret the spread.

\textbf{Solution.}
Let block \(A=[1,3]\) with \(P(A)=\frac{1}{4}\), and block \(B=[3,5]\) with \(P(B)=\frac{3}{4}\).
Within each block, usage is uniform.

Conditional means:
\[
E[U\mid A]=\frac{1+3}{2}=2,
\qquad
E[U\mid B]=\frac{3+5}{2}=4.
\]
So
\[
E[U]=\frac{1}{4}(2)+\frac{3}{4}(4)=\frac{1}{2}+3=\frac{7}{2}=3.5.
\]

Conditional variances (uniform width 2 in each block):
\[
\mathrm{Var}(U\mid A)=\mathrm{Var}(U\mid B)=\frac{2^2}{12}=\frac{1}{3}.
\]
Use
\[
\mathrm{Var}(U)=E[\mathrm{Var}(U\mid \text{block})]+\mathrm{Var}(E[U\mid \text{block}]).
\]
First part:
\[
E[\mathrm{Var}(U\mid \text{block})]=\frac{1}{4}\cdot\frac{1}{3}+\frac{3}{4}\cdot\frac{1}{3}=\frac{1}{3}.
\]
Second part:
\[
\mathrm{Var}(E[U\mid \text{block}])=
\frac{1}{4}(2-3.5)^2+\frac{3}{4}(4-3.5)^2
=\frac{1}{4}(2.25)+\frac{3}{4}(0.25)=0.75=\frac{3}{4}.
\]
Therefore,
\[
\mathrm{Var}(U)=\frac{1}{3}+\frac{3}{4}=\frac{13}{12}.
\]
Standard deviation:
\[
\sigma=\sqrt{\frac{13}{12}}\approx 1.04.
\]

\textbf{Interpretation.}
The variance \(\frac{13}{12}\) and standard deviation about 1.04 hours describe the typical spread of daily app usage around the mean of 3.5 hours.

\end{ExamProblems}
\end{document}
