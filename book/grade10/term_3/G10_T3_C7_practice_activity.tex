\makeatletter
\def\input@path{{./}{../}{../../}{preamble/}{../preamble/}{../../preamble/}}
\makeatother
% ----------------------------------------------------------
% GENERAL 

% File
\documentclass[11pt]{book}

% Margins
\usepackage[margin=1in]{geometry}

\linespread{1.2}            % Line spacing
\usepackage[utf8]{inputenc}
\usepackage[T1]{fontenc}
\usepackage{lmodern}
\usepackage{microtype}
\setlength{\parindent}{0pt}
\setlength{\parskip}{6pt}
\usepackage{booktabs}

% ----------------------------------------------------------
% TABLES
\usepackage{multicol}
\usepackage{longtable} 
\usepackage{array}
\usepackage{booktabs}
\usepackage{tabularx}
\usepackage{multirow}

% ----------------------------------------------------------
% MATHEMATICS
\usepackage{amsmath}
\usepackage{amssymb}
\usepackage{amsfonts}
\usepackage{mathtools}

% ----------------------------------------------------------
% HIDDEN CONTENT
\usepackage{ifthen}
% Define a boolean switch
\newboolean{explicaciones}
% Set the boolean switch to true or false
% Change to true to show the content

% Explanations
\newcommand{\explicacion}[2]{
	\ifthenelse{\boolean{explicaciones}}{#1}{#2}
}
\newcommand{\mostrarExplicaciones}[1]{\setboolean{explicaciones}{#1}}

% ----------------------------------------------------------
% NUMBERING

\usepackage{fancyhdr}
\pagestyle{empty} % Ensures the entire document has no page numbers

\usepackage{tocloft}
\renewcommand{\cftdot}{} % Remove dots for sections, if any
\renewcommand{\cftsecleader}{\cftdotfill{\cftdotsep}} % Remove dots for sections, if any
\cftpagenumbersoff{section} % Removes page numbers from sections
\cftpagenumbersoff{subsection} % Removes page numbers from subsections

% ----------------------------------------------------------
% IMAGES 

% General settings
\usepackage{graphicx}       % Insert images
\usepackage{float}          % Position images
% \usepackage{subfigure}      % Subfigures
\graphicspath{{imgs}}       % Image location
\usepackage{subcaption}     % Subfigures II
\usepackage{verbatim}

% Figures
\usepackage{tikz}
\usetikzlibrary{arrows.meta,positioning,trees}

% Colors
\usepackage{xcolor}     
\definecolor{popUp}{HTML}{666666}
\definecolor{popUpIn}{HTML}{CED9E0}
\definecolor{backgroundC}{HTML}{D0E8F2}
\definecolor{backgroundCC}{HTML}{FFFFFF}
\definecolor{borders}{HTML}{8c120d}
\definecolor{padding}{HTML}{77D0D7}
\definecolor{links}{HTML}{CC6F5F}

% ----------------------------------------------------------
% FRAMES

% General settings
\usepackage{tcolorbox}
\usepackage{adjustbox}          % Adjusted frame  
\setlength{\fboxrule}{3pt}  % Line width
\setlength{\fboxsep}{3pt}   % Box padding

% General frames
\usepackage{mdframed}   

\mdfdefinestyle{estiloGeneral}{    % General style
	linecolor=black,
	linewidth=1.5pt,
	roundcorner=10pt,
	backgroundcolor=backgroundC,
	innerbottommargin=0pt
}
\mdfdefinestyle{code}{          % Code style
	linecolor=black,
	linewidth=1.5pt,
	roundcorner=10pt,
	backgroundcolor=darkgray!10,
	innerbottommargin=0pt
}

% Image frame
\newtcbox{\fboxC}{
	colback=backgroundC,
	colframe=popUp,
	arc=10pt,
	boxrule=3pt,
	boxsep=0pt, % Change the padding here
	nobeforeafter
}

% ----------------------------------------------------------
% PAGE SETTINGS

% Background 
\newcommand{\background}[0]{\begin{tikzpicture}[remember picture,overlay]
		\fill[backgroundC] (-2,2) rectangle (25cm, -550);
\end{tikzpicture}}

\newcommand{\backgroundC}[0]{\begin{tikzpicture}[remember picture,overlay]
		\fill[backgroundCC] (-2,2) rectangle (25cm, -550);
\end{tikzpicture}}

% Page width 
\newcommand{\anchoPag}[0]{20cm}

% ----------------------------------------------------------
% FONT

% General
\usepackage{tgbonum}        % Font
\usepackage{listings}       % Code typesetting
\usepackage[spanish]{babel} % Load Spanish
\selectlanguage{spanish}    % Select Spanish
\usepackage{enumitem}
\usepackage{bookmark}

\setlist[itemize]{leftmargin=1.2em, itemsep=0.35em, topsep=0.35em}

% --- Table helpers ---
\newcolumntype{L}[1]{>{\raggedright\arraybackslash}p{#1}}
\newcolumntype{Y}{>{\raggedright\arraybackslash}X}
\newcolumntype{C}{>{\centering\arraybackslash}X}
\renewcommand{\arraystretch}{1.1}

% Python style
\lstdefinestyle{python}{
	language=Python,
	basicstyle=\ttfamily\small,
	commentstyle=\color{green!50!black},
	keywordstyle=\color{blue},
	numberstyle=\tiny\color{gray},
	numbers=left,
	morekeywords={>, <},
	breakatwhitespace=false,
	showstringspaces=false,
	showtabs=false,
	showspaces=false
}

% ----------------------------------------------------------
% HYPERLINKS

% General
\usepackage{hyperref}       
\hypersetup{
	colorlinks=true,
	linkcolor=links,
	filecolor=magenta,      
	urlcolor=brown,
}

% Custom commands 

% Large link
\newcommand{\bigLink}[2]{\begin{center} \fboxC{\LARGE{\href{#1}{#2}}}\end{center}}

% Small link
\newcommand{\smallLink}[2]{\begin{center}\fboxC{\href{#1}{#2}}\end{center}}

% Bold link
\newcommand{\bfLink}[2]{\href{#1}{\textbf{#2}}}


% Small URL
\newcommand{\smallUrl}[1]{\begin{center}\fboxC{\url{#1}}\end{center}}


% ----------------------------------------------------------
% CUSTOM COMMANDS FOR FIGURES

\newcommand{\espacioImagenes}[0]{-1.2cm}

% Without frame
\newcommand{\fig}[3][\espacioImagenes]{
	\hspace*{#1}
	\centering
	\includegraphics[width=#2\textwidth]{#3}
}

% With frame
\newcommand{\ffig}[2]{\begin{figure}[h]
		\hspace*{\espacioImagenes}
		\centering
		\fbox{\includegraphics[width=#1\textwidth]{#2}}
\end{figure}}

% Hyperlink with frame
\newcommand{\hfig}[3]{\begin{figure}[h]
		\hspace*{-1.4cm}
		\centering
		\color{popUp}
		\fboxC{\href{#1}{\includegraphics[width=#2\textwidth]{#3}}}
	\end{figure}
}

% Hyperlink without frame
\newcommand{\hffig}[3]{\begin{figure}[h]
		\hspace*{-1.1cm}
		\centering
		\color{popUp}
		\href{#1}{\includegraphics[width=#2\textwidth]{#3}}
	\end{figure}
}

% ----------------------------------------------------------

% Start and Contents
\newcommand{\cuadro}[1]{
	\begin{mdframed}[style=estiloGeneral]
		#1 
	\end{mdframed}
}

% Explanation video image
\newcommand{\linkExplicacion}[1]{
	\hffig{#1}{0.5}{principal/videoExplicacion}
	\vspace{-0.5cm}
}

\newcommand{\subSecLink}[2]{
	\subsubsection*{\href{#1}{\textbf{#2}}}
}

% Spacing
\newcommand{\esp}[0]{\vspace{4mm}}

% Back to start
\newcommand{\secInicio}[0]{\begin{center}\hyperref[sec:inicio]{ 
			\includegraphics[width=0.1\textwidth]{principal/up}
	}\end{center}
}


\geometry{margin=0.85in}
\AtBeginDocument{\small}

\newcommand{\ExamNameField}{\noindent\textbf{Name:}\ \rule{0.7\linewidth}{0.4pt}\par\medskip}

\newcommand{\ExamTitleBlock}[3]{%
	\begin{center}
		\Large\textbf{#1}\\
		\textbf{#2}%
		\if\relax\detokenize{#3}\relax\else\\\textbf{#3}\fi
	\end{center}
	\vspace{0.5em}
}

\newcommand{\ExamSection}[1]{\par\medskip\textbf{#1}\par\smallskip}

\newenvironment{ExamCriteria}{%
	\begin{itemize}[leftmargin=1.6em, itemsep=0.3em, topsep=0.2em]
}{%
	\end{itemize}
}

\newenvironment{ExamProblems}{%
	\begin{enumerate}[label=\textbf{P\arabic*}, leftmargin=0pt, labelsep=0.6em, itemindent=2.2em, itemsep=0.8em]
}{%
	\end{enumerate}
}


\begin{document}
\ExamTitleBlock{10th grade}{Term 3 Practice Activity: C7 Inverse Normal Distribution}{}

\ExamSection{C7 Solves problems through the concept of inverse normal distribution.}

\begin{ExamProblems}

\item
\subsection*{Problem 1 — Top 5\% Exam Scores}

\textbf{Problem.}
A large exam has scores modeled by
\[
X \sim N(72, 9^2).
\]

\textbf{Question.} Identify the target probability before using inverse normal, then find the score \(x\) that marks the top 5\% of students.

\textbf{Solution.}
Top 5\% means
\[
P(X>x)=0.05 \Rightarrow P(X\le x)=0.95.
\]
Standardize:
\[
P\!\left(Z\le \frac{x-72}{9}\right)=0.95.
\]
From the inverse normal table, \(z_{0.95}\approx 1.645\).
So
\[
\frac{x-72}{9}=1.645 \Rightarrow x=72+1.645(9)=86.805.
\]
\[
\boxed{x\approx 86.8}
\]
\textbf{Interpretation.} A score of about 86.8 is the cutoff so only about 5\% of students score higher.

% --------------------------------------------------

\item
\subsection*{Problem 2 — Bottom 10\% Commute Times}

\textbf{Problem.}
Morning commute times (minutes) are modeled by
\[
X \sim N(38, 7^2).
\]

\textbf{Question.} Identify the target probability before using inverse normal, then find the time \(x\) below which the bottom 10\% of commutes fall.

\textbf{Solution.}
Bottom 10\% means
\[
P(X\le x)=0.10.
\]
Convert to standard normal:
\[
P\!\left(Z\le \frac{x-38}{7}\right)=0.10.
\]
Inverse normal value: \(z_{0.10}\approx -1.28\).
Thus
\[
\frac{x-38}{7}=-1.28 \Rightarrow x=38-1.28(7)=29.04.
\]
\[
\boxed{x\approx 29.0\text{ min}}
\]
\textbf{Interpretation.} About 10\% of commutes are 29.0 minutes or shorter.

% --------------------------------------------------

\item
\subsection*{Problem 3 — Central 95\% Device Battery Life}

\textbf{Problem.}
Battery life (hours) for a device model follows
\[
X \sim N(20, 2.5^2).
\]

\textbf{Question.} Identify the target probabilities before using inverse normal, then find the interval \([a,b]\) containing the middle 95\% of battery lives.

\textbf{Solution.}
Middle 95\% leaves 2.5\% in each tail:
\[
P(X\le a)=0.025,\qquad P(X\le b)=0.975.
\]
Standardize each endpoint:
\[
\frac{a-20}{2.5}=z_{0.025},\qquad \frac{b-20}{2.5}=z_{0.975}.
\]
From inverse normal values, \(z_{0.025}\approx -1.96\), \(z_{0.975}\approx 1.96\).
So
\[
a=20-1.96(2.5)=15.10,\qquad b=20+1.96(2.5)=24.90.
\]
\[
\boxed{[a,b]\approx[15.1,24.9]\text{ h}}
\]
\textbf{Interpretation.} About 95\% of batteries last between 15.1 h and 24.9 h.

% --------------------------------------------------

\item
\subsection*{Problem 4 — Bottom 20\% Streaming Speeds}

\textbf{Problem.}
Home streaming speed (Mbps) is modeled by
\[
X \sim N(85, 12^2).
\]

\textbf{Question.} Identify the target probability before using inverse normal, then find the speed \(x\) that separates the bottom 20\% of connections.

\textbf{Solution.}
Bottom 20\% means
\[
P(X\le x)=0.20.
\]
Convert to \(Z\):
\[
P\!\left(Z\le \frac{x-85}{12}\right)=0.20.
\]
Inverse normal value: \(z_{0.20}\approx -0.84\).
Then
\[
\frac{x-85}{12}=-0.84 \Rightarrow x=85-0.84(12)=74.92.
\]
\[
\boxed{x\approx 74.9\text{ Mbps}}
\]
\textbf{Interpretation.} Roughly 20\% of homes have streaming speed below about 74.9 Mbps.

% --------------------------------------------------

\item
\subsection*{Problem 5 — Top 2\% Plant Heights}

\textbf{Problem.}
The height of plants in a greenhouse (cm) follows
\[
X \sim N(48, 6^2).
\]

\textbf{Question.} Identify the target probability before using inverse normal, then find the height \(x\) so that only the tallest 2\% are above it.

\textbf{Solution.}
Top 2\% means
\[
P(X>x)=0.02 \Rightarrow P(X\le x)=0.98.
\]
Standardize:
\[
P\!\left(Z\le \frac{x-48}{6}\right)=0.98.
\]
From the inverse normal table, \(z_{0.98}\approx 2.05\).
So
\[
\frac{x-48}{6}=2.05 \Rightarrow x=48+2.05(6)=60.30.
\]
\[
\boxed{x\approx 60.3\text{ cm}}
\]
\textbf{Interpretation.} Plants taller than about 60.3 cm are in the tallest 2\%.

% --------------------------------------------------

\item
\subsection*{Problem 6 — Bottom 1\% Package Delivery Times}

\textbf{Problem.}
Delivery times for a service (hours) are modeled by
\[
X \sim N(36, 5^2).
\]

\textbf{Question.} Identify the target probability before using inverse normal, then find the delivery time \(x\) below which the fastest 1\% of deliveries occur.

\textbf{Solution.}
Bottom 1\% means
\[
P(X\le x)=0.01.
\]
Convert to a standard normal statement:
\[
P\!\left(Z\le \frac{x-36}{5}\right)=0.01.
\]
Inverse normal value: \(z_{0.01}\approx -2.33\).
Then
\[
\frac{x-36}{5}=-2.33 \Rightarrow x=36-2.33(5)=24.35.
\]
\[
\boxed{x\approx 24.4\text{ h}}
\]
\textbf{Interpretation.} About 1\% of packages are delivered in 24.4 hours or less.

% --------------------------------------------------

\item
\subsection*{Problem 7 — Middle 80\% Fitness Run Times}

\textbf{Problem.}
A 2 km run time (minutes) for students is modeled by
\[
X \sim N(11.8, 1.4^2).
\]

\textbf{Question.} Identify the target probabilities before using inverse normal, then find the interval \([a,b]\) containing the middle 80\% of run times.

\textbf{Solution.}
Middle 80\% leaves 10\% in each tail:
\[
P(X\le a)=0.10,\qquad P(X\le b)=0.90.
\]
Standardizing gives
\[
\frac{a-11.8}{1.4}=z_{0.10},\qquad \frac{b-11.8}{1.4}=z_{0.90}.
\]
From inverse normal values: \(z_{0.10}\approx -1.28\), \(z_{0.90}\approx 1.28\).
Hence
\[
a=11.8-1.28(1.4)=10.008,\qquad b=11.8+1.28(1.4)=13.592.
\]
\[
\boxed{[a,b]\approx[10.0,13.6]\text{ min}}
\]
\textbf{Interpretation.} About 80\% of students finish the 2 km run between 10.0 and 13.6 minutes.

% --------------------------------------------------

\item
\subsection*{Problem 8 — Top 10\% Product Lifetime}

\textbf{Problem.}
A light bulb lifetime (hours) is modeled by
\[
X \sim N(1200, 150^2).
\]

\textbf{Question.} Identify the target probability before using inverse normal, then find the lifetime \(x\) that marks the top 10\% longest-lasting bulbs.

\textbf{Solution.}
Top 10\% means
\[
P(X>x)=0.10 \Rightarrow P(X\le x)=0.90.
\]
Convert to standard normal:
\[
P\!\left(Z\le \frac{x-1200}{150}\right)=0.90.
\]
Inverse value: \(z_{0.90}\approx 1.28\).
So
\[
\frac{x-1200}{150}=1.28 \Rightarrow x=1200+1.28(150)=1392.
\]
\[
\boxed{x\approx 1392\text{ h}}
\]
\textbf{Interpretation.} Only about 10\% of bulbs last longer than roughly 1392 hours.

% --------------------------------------------------

\item
\subsection*{Problem 9 — Middle 90\% Manufacturing Tolerance Width}

\textbf{Problem.}
A metal rod diameter (mm) is modeled by
\[
X \sim N(10.00, 0.08^2).
\]

\textbf{Question.} Identify the target probabilities before using inverse normal, then find the central 90\% interval \([a,b]\) for acceptable diameters.

\textbf{Solution.}
Middle 90\% leaves 5\% in each tail:
\[
P(X\le a)=0.05,\qquad P(X\le b)=0.95.
\]
Standardize:
\[
\frac{a-10.00}{0.08}=z_{0.05},\qquad \frac{b-10.00}{0.08}=z_{0.95}.
\]
Inverse normal values: \(z_{0.05}\approx -1.645\), \(z_{0.95}\approx 1.645\).
Then
\[
a=10.00-1.645(0.08)=9.8684,
\qquad
b=10.00+1.645(0.08)=10.1316.
\]
\[
\boxed{[a,b]\approx[9.868,10.132]\text{ mm}}
\]
\textbf{Interpretation.} About 90\% of rod diameters are expected between 9.868 mm and 10.132 mm.

% --------------------------------------------------

\item
\subsection*{Problem 10 — Middle 96\% Chemistry Scores}

\textbf{Problem.}
Chemistry quiz scores are modeled by
\[
X \sim N(64, 10^2).
\]

\textbf{Question.} Identify the target probabilities before using inverse normal, then find the score interval \([a,b]\) containing the middle 96\% of students.

\textbf{Solution.}
Middle 96\% leaves 2\% in each tail:
\[
P(X\le a)=0.02,\qquad P(X\le b)=0.98.
\]
Convert to standard normal equations:
\[
\frac{a-64}{10}=z_{0.02},\qquad \frac{b-64}{10}=z_{0.98}.
\]
From inverse normal values, \(z_{0.02}\approx -2.05\), \(z_{0.98}\approx 2.05\).
So
\[
a=64-2.05(10)=43.5,
\qquad
b=64+2.05(10)=84.5.
\]
\[
\boxed{[a,b]\approx[43.5,84.5]}
\]
\textbf{Interpretation.} About 96\% of students score between roughly 43.5 and 84.5 points.

\end{ExamProblems}

\end{document}
