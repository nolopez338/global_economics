\makeatletter
\def\input@path{{./}{../}{../../}{preamble/}{../preamble/}{../../preamble/}}
\makeatother
% ----------------------------------------------------------
% GENERAL 

% File
\documentclass[11pt]{book}

% Margins
\usepackage[margin=1in]{geometry}

\linespread{1.2}            % Line spacing
\usepackage[utf8]{inputenc}
\usepackage[T1]{fontenc}
\usepackage{lmodern}
\usepackage{microtype}
\setlength{\parindent}{0pt}
\setlength{\parskip}{6pt}
\usepackage{booktabs}

% ----------------------------------------------------------
% TABLES
\usepackage{multicol}
\usepackage{longtable} 
\usepackage{array}
\usepackage{booktabs}
\usepackage{tabularx}
\usepackage{multirow}

% ----------------------------------------------------------
% MATHEMATICS
\usepackage{amsmath}
\usepackage{amssymb}
\usepackage{amsfonts}
\usepackage{mathtools}

% ----------------------------------------------------------
% HIDDEN CONTENT
\usepackage{ifthen}
% Define a boolean switch
\newboolean{explicaciones}
% Set the boolean switch to true or false
% Change to true to show the content

% Explanations
\newcommand{\explicacion}[2]{
	\ifthenelse{\boolean{explicaciones}}{#1}{#2}
}
\newcommand{\mostrarExplicaciones}[1]{\setboolean{explicaciones}{#1}}

% ----------------------------------------------------------
% NUMBERING

\usepackage{fancyhdr}
\pagestyle{empty} % Ensures the entire document has no page numbers

\usepackage{tocloft}
\renewcommand{\cftdot}{} % Remove dots for sections, if any
\renewcommand{\cftsecleader}{\cftdotfill{\cftdotsep}} % Remove dots for sections, if any
\cftpagenumbersoff{section} % Removes page numbers from sections
\cftpagenumbersoff{subsection} % Removes page numbers from subsections

% ----------------------------------------------------------
% IMAGES 

% General settings
\usepackage{graphicx}       % Insert images
\usepackage{float}          % Position images
% \usepackage{subfigure}      % Subfigures
\graphicspath{{imgs}}       % Image location
\usepackage{subcaption}     % Subfigures II
\usepackage{verbatim}

% Figures
\usepackage{tikz}
\usetikzlibrary{arrows.meta,positioning,trees}

% Colors
\usepackage{xcolor}     
\definecolor{popUp}{HTML}{666666}
\definecolor{popUpIn}{HTML}{CED9E0}
\definecolor{backgroundC}{HTML}{D0E8F2}
\definecolor{backgroundCC}{HTML}{FFFFFF}
\definecolor{borders}{HTML}{8c120d}
\definecolor{padding}{HTML}{77D0D7}
\definecolor{links}{HTML}{CC6F5F}

% ----------------------------------------------------------
% FRAMES

% General settings
\usepackage{tcolorbox}
\usepackage{adjustbox}          % Adjusted frame  
\setlength{\fboxrule}{3pt}  % Line width
\setlength{\fboxsep}{3pt}   % Box padding

% General frames
\usepackage{mdframed}   

\mdfdefinestyle{estiloGeneral}{    % General style
	linecolor=black,
	linewidth=1.5pt,
	roundcorner=10pt,
	backgroundcolor=backgroundC,
	innerbottommargin=0pt
}
\mdfdefinestyle{code}{          % Code style
	linecolor=black,
	linewidth=1.5pt,
	roundcorner=10pt,
	backgroundcolor=darkgray!10,
	innerbottommargin=0pt
}

% Image frame
\newtcbox{\fboxC}{
	colback=backgroundC,
	colframe=popUp,
	arc=10pt,
	boxrule=3pt,
	boxsep=0pt, % Change the padding here
	nobeforeafter
}

% ----------------------------------------------------------
% PAGE SETTINGS

% Background 
\newcommand{\background}[0]{\begin{tikzpicture}[remember picture,overlay]
		\fill[backgroundC] (-2,2) rectangle (25cm, -550);
\end{tikzpicture}}

\newcommand{\backgroundC}[0]{\begin{tikzpicture}[remember picture,overlay]
		\fill[backgroundCC] (-2,2) rectangle (25cm, -550);
\end{tikzpicture}}

% Page width 
\newcommand{\anchoPag}[0]{20cm}

% ----------------------------------------------------------
% FONT

% General
\usepackage{tgbonum}        % Font
\usepackage{listings}       % Code typesetting
\usepackage[spanish]{babel} % Load Spanish
\selectlanguage{spanish}    % Select Spanish
\usepackage{enumitem}
\usepackage{bookmark}

\setlist[itemize]{leftmargin=1.2em, itemsep=0.35em, topsep=0.35em}

% --- Table helpers ---
\newcolumntype{L}[1]{>{\raggedright\arraybackslash}p{#1}}
\newcolumntype{Y}{>{\raggedright\arraybackslash}X}
\newcolumntype{C}{>{\centering\arraybackslash}X}
\renewcommand{\arraystretch}{1.1}

% Python style
\lstdefinestyle{python}{
	language=Python,
	basicstyle=\ttfamily\small,
	commentstyle=\color{green!50!black},
	keywordstyle=\color{blue},
	numberstyle=\tiny\color{gray},
	numbers=left,
	morekeywords={>, <},
	breakatwhitespace=false,
	showstringspaces=false,
	showtabs=false,
	showspaces=false
}

% ----------------------------------------------------------
% HYPERLINKS

% General
\usepackage{hyperref}       
\hypersetup{
	colorlinks=true,
	linkcolor=links,
	filecolor=magenta,      
	urlcolor=brown,
}

% Custom commands 

% Large link
\newcommand{\bigLink}[2]{\begin{center} \fboxC{\LARGE{\href{#1}{#2}}}\end{center}}

% Small link
\newcommand{\smallLink}[2]{\begin{center}\fboxC{\href{#1}{#2}}\end{center}}

% Bold link
\newcommand{\bfLink}[2]{\href{#1}{\textbf{#2}}}


% Small URL
\newcommand{\smallUrl}[1]{\begin{center}\fboxC{\url{#1}}\end{center}}


% ----------------------------------------------------------
% CUSTOM COMMANDS FOR FIGURES

\newcommand{\espacioImagenes}[0]{-1.2cm}

% Without frame
\newcommand{\fig}[3][\espacioImagenes]{
	\hspace*{#1}
	\centering
	\includegraphics[width=#2\textwidth]{#3}
}

% With frame
\newcommand{\ffig}[2]{\begin{figure}[h]
		\hspace*{\espacioImagenes}
		\centering
		\fbox{\includegraphics[width=#1\textwidth]{#2}}
\end{figure}}

% Hyperlink with frame
\newcommand{\hfig}[3]{\begin{figure}[h]
		\hspace*{-1.4cm}
		\centering
		\color{popUp}
		\fboxC{\href{#1}{\includegraphics[width=#2\textwidth]{#3}}}
	\end{figure}
}

% Hyperlink without frame
\newcommand{\hffig}[3]{\begin{figure}[h]
		\hspace*{-1.1cm}
		\centering
		\color{popUp}
		\href{#1}{\includegraphics[width=#2\textwidth]{#3}}
	\end{figure}
}

% ----------------------------------------------------------

% Start and Contents
\newcommand{\cuadro}[1]{
	\begin{mdframed}[style=estiloGeneral]
		#1 
	\end{mdframed}
}

% Explanation video image
\newcommand{\linkExplicacion}[1]{
	\hffig{#1}{0.5}{principal/videoExplicacion}
	\vspace{-0.5cm}
}

\newcommand{\subSecLink}[2]{
	\subsubsection*{\href{#1}{\textbf{#2}}}
}

% Spacing
\newcommand{\esp}[0]{\vspace{4mm}}

% Back to start
\newcommand{\secInicio}[0]{\begin{center}\hyperref[sec:inicio]{ 
			\includegraphics[width=0.1\textwidth]{principal/up}
	}\end{center}
}


\geometry{margin=0.85in}
\AtBeginDocument{\small}

\newcommand{\ExamNameField}{\noindent\textbf{Name:}\ \rule{0.7\linewidth}{0.4pt}\par\medskip}

\newcommand{\ExamTitleBlock}[3]{%
	\begin{center}
		\Large\textbf{#1}\\
		\textbf{#2}%
		\if\relax\detokenize{#3}\relax\else\\\textbf{#3}\fi
	\end{center}
	\vspace{0.5em}
}

\newcommand{\ExamSection}[1]{\par\medskip\textbf{#1}\par\smallskip}

\newenvironment{ExamCriteria}{%
	\begin{itemize}[leftmargin=1.6em, itemsep=0.3em, topsep=0.2em]
}{%
	\end{itemize}
}

\newenvironment{ExamProblems}{%
	\begin{enumerate}[label=\textbf{P\arabic*}, leftmargin=0pt, labelsep=0.6em, itemindent=2.2em, itemsep=0.8em]
}{%
	\end{enumerate}
}


\begin{document}
\ExamTitleBlock{10th grade}{Term 3 Practice Activity: C2 Uniform Random Variables}{}
\ExamSection{C2 Explains how a uniform random variable works.}

\begin{ExamProblems}

\item
\subsection*{Problem 1 — Bus Stop Waiting Time}

\textbf{Problem.}
A learner waits for a bus that can arrive at any time between 0 and 12 minutes after reaching the stop.
Let $X\sim \text{Uniform}(0,12)$.

\textbf{Question.}
Find:
\begin{itemize}
    \item[(a)] $P(X<5)$,
    \item[(b)] $P(3\le X\le 9)$.
Interpret each result in context.
\end{itemize}

\textbf{Solution.}
The total interval length is
\[
b-a=12-0=12.
\]
For a uniform variable,
\[
P(\text{event})=\frac{\text{event interval length}}{\text{total length}}.
\]
\[
P(X<5)=\frac{5-0}{12}=\frac{5}{12}\approx 0.417.
\]
\[
P(3\le X\le 9)=\frac{9-3}{12}=\frac{6}{12}=\frac{1}{2}=0.5.
\]

\textbf{Interpretation.}
There is a $\frac{5}{12}$ chance the wait is under 5 minutes, and a $\frac{1}{2}$ chance the wait is between 3 and 9 minutes.

% --------------------------------------------------

\item
\subsection*{Problem 2 — App Download Duration}

\textbf{Problem.}
A file download takes a random time between 2 and 10 minutes.
Let $X\sim \text{Uniform}(2,10)$.

\textbf{Question.}
Find:
\begin{itemize}
    \item[(a)] $P(4\le X\le 7)$,
    \item[(b)] $P(X>8)$.
Interpret each result.
\end{itemize}

\textbf{Solution.}
The total interval length is
\[
b-a=10-2=8.
\]
\[
P(4\le X\le 7)=\frac{7-4}{8}=\frac{3}{8}=0.375.
\]
\[
P(X>8)=\frac{10-8}{8}=\frac{2}{8}=\frac{1}{4}=0.25.
\]

\textbf{Interpretation.}
The download is between 4 and 7 minutes with probability $\frac{3}{8}$, and longer than 8 minutes with probability $\frac{1}{4}$.

% --------------------------------------------------

\item
\subsection*{Problem 3 — Morning Temperature Range}

\textbf{Problem.}
In a city, early-morning temperature (in $^\circ$C) is modeled as uniform from $-3$ to $5$.
Let $X\sim \text{Uniform}(-3,5)$.

\textbf{Question.}
Find:
\begin{itemize}
    \item[(a)] $P(-1\le X\le 2)$,
    \item[(b)] $P(X<0\ \text{or}\ X>4)$.
Interpret each probability.
\end{itemize}

\textbf{Solution.}
The total interval length is
\[
b-a=5-(-3)=8.
\]
\[
P(-1\le X\le 2)=\frac{2-(-1)}{8}=\frac{3}{8}=0.375.
\]
For the union of disjoint intervals:
\[
P(X<0\ \text{or}\ X>4)=\frac{0-(-3)}{8}+\frac{5-4}{8}=\frac{3}{8}+\frac{1}{8}=\frac{1}{2}=0.5.
\]

\textbf{Interpretation.}
There is a $\frac{3}{8}$ chance the temperature is between $-1^\circ$C and $2^\circ$C, and a $\frac{1}{2}$ chance it is below $0^\circ$C or above $4^\circ$C.

% --------------------------------------------------

\item
\subsection*{Problem 4 — Delivery Arrival Minute}

\textbf{Problem.}
A delivery may arrive any time between minute 100 and minute 160 of a tracking window.
Let $X\sim \text{Uniform}(100,160)$.

\textbf{Question.}
Find:
\begin{itemize}
    \item[(a)] $P(115\le X\le 145)$,
    \item[(b)] $P(X<110\ \text{or}\ X>150)$.
Interpret both answers.
\end{itemize}

\textbf{Solution.}
The total interval length is
\[
b-a=160-100=60.
\]
\[
P(115\le X\le 145)=\frac{145-115}{60}=\frac{30}{60}=\frac{1}{2}=0.5.
\]
\[
P(X<110\ \text{or}\ X>150)=\frac{110-100}{60}+\frac{160-150}{60}=\frac{10}{60}+\frac{10}{60}=\frac{20}{60}=\frac{1}{3}.
\]

\textbf{Interpretation.}
The package has probability $\frac{1}{2}$ of arriving between minutes 115 and 145, and probability $\frac{1}{3}$ of arriving very early or very late in the window.

% --------------------------------------------------

\item
\subsection*{Problem 5 — Practice Session Length}

\textbf{Problem.}
A student's after-school practice session can last between 30 and 90 minutes, uniformly.
Let $X\sim \text{Uniform}(30,90)$.

\textbf{Question.}
Find:
\begin{itemize}
    \item[(a)] $P(X\le 50)$,
    \item[(b)] $P(45\le X\le 75)$.
Interpret each result.
\end{itemize}

\textbf{Solution.}
The total interval length is
\[
b-a=90-30=60.
\]
\[
P(X\le 50)=\frac{50-30}{60}=\frac{20}{60}=\frac{1}{3}.
\]
\[
P(45\le X\le 75)=\frac{75-45}{60}=\frac{30}{60}=\frac{1}{2}=0.5.
\]

\textbf{Interpretation.}
There is a $\frac{1}{3}$ chance practice ends by 50 minutes, and a $\frac{1}{2}$ chance it lasts from 45 to 75 minutes.

% --------------------------------------------------

\item
\subsection*{Problem 6 — Commute Time with Complement}

\textbf{Problem.}
A student's commute time (minutes) is uniformly distributed from 5 to 25.
Let $X\sim \text{Uniform}(5,25)$.

\textbf{Question.}
Find:
\begin{itemize}
    \item[(a)] $P(10\le X\le 20)$,
    \item[(b)] $P(X<10\ \text{or}\ X>20)$ using complement reasoning.
Interpret each probability.
\end{itemize}

\textbf{Solution.}
The total interval length is
\[
b-a=25-5=20.
\]
\[
P(10\le X\le 20)=\frac{20-10}{20}=\frac{10}{20}=\frac{1}{2}.
\]
Using complement:
\[
P(X<10\ \text{or}\ X>20)=1-P(10\le X\le 20)=1-\frac{1}{2}=\frac{1}{2}.
\]

\textbf{Interpretation.}
The commute is between 10 and 20 minutes half the time, and outside that range the other half.

% --------------------------------------------------

\item
\subsection*{Problem 7 — Online Game Match Duration}

\textbf{Problem.}
An online game match lasts between 40 and 70 minutes, uniformly.
Let $X\sim \text{Uniform}(40,70)$.

\textbf{Question.}
Find:
\begin{itemize}
    \item[(a)] $P(X>55)$,
    \item[(b)] $P(45\le X\le 60)$.
Interpret both answers.
\end{itemize}

\textbf{Solution.}
The total interval length is
\[
b-a=70-40=30.
\]
\[
P(X>55)=\frac{70-55}{30}=\frac{15}{30}=\frac{1}{2}=0.5.
\]
\[
P(45\le X\le 60)=\frac{60-45}{30}=\frac{15}{30}=\frac{1}{2}=0.5.
\]

\textbf{Interpretation.}
A match has a $\frac{1}{2}$ chance to last more than 55 minutes, and also a $\frac{1}{2}$ chance to last from 45 to 60 minutes.

% --------------------------------------------------

\item
\subsection*{Problem 8 — Daily Minimum Temperature}

\textbf{Problem.}
The minimum temperature (in $^\circ$C) is modeled as uniform from $-8$ to $4$.
Let $X\sim \text{Uniform}(-8,4)$.

\textbf{Question.}
Find:
\begin{itemize}
    \item[(a)] $P(-5\le X\le 1)$,
    \item[(b)] $P(X>-2)$ using complement reasoning.
Interpret each result.
\end{itemize}

\textbf{Solution.}
The total interval length is
\[
b-a=4-(-8)=12.
\]
\[
P(-5\le X\le 1)=\frac{1-(-5)}{12}=\frac{6}{12}=\frac{1}{2}=0.5.
\]
Use the complement of $X\le -2$:
\[
P(X>-2)=1-P(X\le -2)=1-\frac{-2-(-8)}{12}=1-\frac{6}{12}=\frac{1}{2}.
\]

\textbf{Interpretation.}
There is a $\frac{1}{2}$ chance the minimum temperature is from $-5^\circ$C to $1^\circ$C, and a $\frac{1}{2}$ chance it is above $-2^\circ$C.

% --------------------------------------------------

\item
\subsection*{Problem 9 — Video Call Start Minute}

\textbf{Problem.}
A video call starts at a random minute between 50 and 110 minutes after 3:00 p.m.
Let $X\sim \text{Uniform}(50,110)$.

\textbf{Question.}
Find:
\begin{itemize}
    \item[(a)] $P(70\le X\le 95)$,
    \item[(b)] $P(X<65\ \text{or}\ X>100)$.
Interpret both probabilities.
\end{itemize}

\textbf{Solution.}
The total interval length is
\[
b-a=110-50=60.
\]
\[
P(70\le X\le 95)=\frac{95-70}{60}=\frac{25}{60}=\frac{5}{12}.
\]
For the disjoint union:
\[
P(X<65\ \text{or}\ X>100)=\frac{65-50}{60}+\frac{110-100}{60}=\frac{15}{60}+\frac{10}{60}=\frac{25}{60}=\frac{5}{12}.
\]

\textbf{Interpretation.}
The call starts between minutes 70 and 95 with probability $\frac{5}{12}$, and has the same probability of starting very early or very late.

% --------------------------------------------------

\item
\subsection*{Problem 10 — Streaming Buffer Time}

\textbf{Problem.}
A streaming app buffer time is uniformly distributed between 0 and 15 seconds.
Let $X\sim \text{Uniform}(0,15)$.

\textbf{Question.}
Find:
\begin{itemize}
    \item[(a)] $P(X\ge 12)$,
    \item[(b)] $P(4\le X\le 10)$.
Interpret each answer in context.
\end{itemize}

\textbf{Solution.}
The total interval length is
\[
b-a=15-0=15.
\]
\[
P(X\ge 12)=\frac{15-12}{15}=\frac{3}{15}=\frac{1}{5}=0.2.
\]
\[
P(4\le X\le 10)=\frac{10-4}{15}=\frac{6}{15}=\frac{2}{5}=0.4.
\]

\textbf{Interpretation.}
There is a $\frac{1}{5}$ chance buffering lasts at least 12 seconds, and a $\frac{2}{5}$ chance it lasts between 4 and 10 seconds.

\end{ExamProblems}
\end{document}
