\makeatletter
\def\input@path{{./}{../}{../../}{preamble/}{../preamble/}{../../preamble/}}
\makeatother
% ----------------------------------------------------------
% GENERAL 

% File
\documentclass[11pt]{book}

% Margins
\usepackage[margin=1in]{geometry}

\linespread{1.2}            % Line spacing
\usepackage[utf8]{inputenc}
\usepackage[T1]{fontenc}
\usepackage{lmodern}
\usepackage{microtype}
\setlength{\parindent}{0pt}
\setlength{\parskip}{6pt}
\usepackage{booktabs}

% ----------------------------------------------------------
% TABLES
\usepackage{multicol}
\usepackage{longtable} 
\usepackage{array}
\usepackage{booktabs}
\usepackage{tabularx}
\usepackage{multirow}

% ----------------------------------------------------------
% MATHEMATICS
\usepackage{amsmath}
\usepackage{amssymb}
\usepackage{amsfonts}
\usepackage{mathtools}

% ----------------------------------------------------------
% HIDDEN CONTENT
\usepackage{ifthen}
% Define a boolean switch
\newboolean{explicaciones}
% Set the boolean switch to true or false
% Change to true to show the content

% Explanations
\newcommand{\explicacion}[2]{
	\ifthenelse{\boolean{explicaciones}}{#1}{#2}
}
\newcommand{\mostrarExplicaciones}[1]{\setboolean{explicaciones}{#1}}

% ----------------------------------------------------------
% NUMBERING

\usepackage{fancyhdr}
\pagestyle{empty} % Ensures the entire document has no page numbers

\usepackage{tocloft}
\renewcommand{\cftdot}{} % Remove dots for sections, if any
\renewcommand{\cftsecleader}{\cftdotfill{\cftdotsep}} % Remove dots for sections, if any
\cftpagenumbersoff{section} % Removes page numbers from sections
\cftpagenumbersoff{subsection} % Removes page numbers from subsections

% ----------------------------------------------------------
% IMAGES 

% General settings
\usepackage{graphicx}       % Insert images
\usepackage{float}          % Position images
% \usepackage{subfigure}      % Subfigures
\graphicspath{{imgs}}       % Image location
\usepackage{subcaption}     % Subfigures II
\usepackage{verbatim}

% Figures
\usepackage{tikz}
\usetikzlibrary{arrows.meta,positioning,trees}

% Colors
\usepackage{xcolor}     
\definecolor{popUp}{HTML}{666666}
\definecolor{popUpIn}{HTML}{CED9E0}
\definecolor{backgroundC}{HTML}{D0E8F2}
\definecolor{backgroundCC}{HTML}{FFFFFF}
\definecolor{borders}{HTML}{8c120d}
\definecolor{padding}{HTML}{77D0D7}
\definecolor{links}{HTML}{CC6F5F}

% ----------------------------------------------------------
% FRAMES

% General settings
\usepackage{tcolorbox}
\usepackage{adjustbox}          % Adjusted frame  
\setlength{\fboxrule}{3pt}  % Line width
\setlength{\fboxsep}{3pt}   % Box padding

% General frames
\usepackage{mdframed}   

\mdfdefinestyle{estiloGeneral}{    % General style
	linecolor=black,
	linewidth=1.5pt,
	roundcorner=10pt,
	backgroundcolor=backgroundC,
	innerbottommargin=0pt
}
\mdfdefinestyle{code}{          % Code style
	linecolor=black,
	linewidth=1.5pt,
	roundcorner=10pt,
	backgroundcolor=darkgray!10,
	innerbottommargin=0pt
}

% Image frame
\newtcbox{\fboxC}{
	colback=backgroundC,
	colframe=popUp,
	arc=10pt,
	boxrule=3pt,
	boxsep=0pt, % Change the padding here
	nobeforeafter
}

% ----------------------------------------------------------
% PAGE SETTINGS

% Background 
\newcommand{\background}[0]{\begin{tikzpicture}[remember picture,overlay]
		\fill[backgroundC] (-2,2) rectangle (25cm, -550);
\end{tikzpicture}}

\newcommand{\backgroundC}[0]{\begin{tikzpicture}[remember picture,overlay]
		\fill[backgroundCC] (-2,2) rectangle (25cm, -550);
\end{tikzpicture}}

% Page width 
\newcommand{\anchoPag}[0]{20cm}

% ----------------------------------------------------------
% FONT

% General
\usepackage{tgbonum}        % Font
\usepackage{listings}       % Code typesetting
\usepackage[spanish]{babel} % Load Spanish
\selectlanguage{spanish}    % Select Spanish
\usepackage{enumitem}
\usepackage{bookmark}

\setlist[itemize]{leftmargin=1.2em, itemsep=0.35em, topsep=0.35em}

% --- Table helpers ---
\newcolumntype{L}[1]{>{\raggedright\arraybackslash}p{#1}}
\newcolumntype{Y}{>{\raggedright\arraybackslash}X}
\newcolumntype{C}{>{\centering\arraybackslash}X}
\renewcommand{\arraystretch}{1.1}

% Python style
\lstdefinestyle{python}{
	language=Python,
	basicstyle=\ttfamily\small,
	commentstyle=\color{green!50!black},
	keywordstyle=\color{blue},
	numberstyle=\tiny\color{gray},
	numbers=left,
	morekeywords={>, <},
	breakatwhitespace=false,
	showstringspaces=false,
	showtabs=false,
	showspaces=false
}

% ----------------------------------------------------------
% HYPERLINKS

% General
\usepackage{hyperref}       
\hypersetup{
	colorlinks=true,
	linkcolor=links,
	filecolor=magenta,      
	urlcolor=brown,
}

% Custom commands 

% Large link
\newcommand{\bigLink}[2]{\begin{center} \fboxC{\LARGE{\href{#1}{#2}}}\end{center}}

% Small link
\newcommand{\smallLink}[2]{\begin{center}\fboxC{\href{#1}{#2}}\end{center}}

% Bold link
\newcommand{\bfLink}[2]{\href{#1}{\textbf{#2}}}


% Small URL
\newcommand{\smallUrl}[1]{\begin{center}\fboxC{\url{#1}}\end{center}}


% ----------------------------------------------------------
% CUSTOM COMMANDS FOR FIGURES

\newcommand{\espacioImagenes}[0]{-1.2cm}

% Without frame
\newcommand{\fig}[3][\espacioImagenes]{
	\hspace*{#1}
	\centering
	\includegraphics[width=#2\textwidth]{#3}
}

% With frame
\newcommand{\ffig}[2]{\begin{figure}[h]
		\hspace*{\espacioImagenes}
		\centering
		\fbox{\includegraphics[width=#1\textwidth]{#2}}
\end{figure}}

% Hyperlink with frame
\newcommand{\hfig}[3]{\begin{figure}[h]
		\hspace*{-1.4cm}
		\centering
		\color{popUp}
		\fboxC{\href{#1}{\includegraphics[width=#2\textwidth]{#3}}}
	\end{figure}
}

% Hyperlink without frame
\newcommand{\hffig}[3]{\begin{figure}[h]
		\hspace*{-1.1cm}
		\centering
		\color{popUp}
		\href{#1}{\includegraphics[width=#2\textwidth]{#3}}
	\end{figure}
}

% ----------------------------------------------------------

% Start and Contents
\newcommand{\cuadro}[1]{
	\begin{mdframed}[style=estiloGeneral]
		#1 
	\end{mdframed}
}

% Explanation video image
\newcommand{\linkExplicacion}[1]{
	\hffig{#1}{0.5}{principal/videoExplicacion}
	\vspace{-0.5cm}
}

\newcommand{\subSecLink}[2]{
	\subsubsection*{\href{#1}{\textbf{#2}}}
}

% Spacing
\newcommand{\esp}[0]{\vspace{4mm}}

% Back to start
\newcommand{\secInicio}[0]{\begin{center}\hyperref[sec:inicio]{ 
			\includegraphics[width=0.1\textwidth]{principal/up}
	}\end{center}
}


\geometry{margin=0.85in}
\AtBeginDocument{\small}

\newcommand{\ExamNameField}{\noindent\textbf{Name:}\ \rule{0.7\linewidth}{0.4pt}\par\medskip}

\newcommand{\ExamTitleBlock}[3]{%
	\begin{center}
		\Large\textbf{#1}\\
		\textbf{#2}%
		\if\relax\detokenize{#3}\relax\else\\\textbf{#3}\fi
	\end{center}
	\vspace{0.5em}
}

\newcommand{\ExamSection}[1]{\par\medskip\textbf{#1}\par\smallskip}

\newenvironment{ExamCriteria}{%
	\begin{itemize}[leftmargin=1.6em, itemsep=0.3em, topsep=0.2em]
}{%
	\end{itemize}
}

\newenvironment{ExamProblems}{%
	\begin{enumerate}[label=\textbf{P\arabic*}, leftmargin=0pt, labelsep=0.6em, itemindent=2.2em, itemsep=0.8em]
}{%
	\end{enumerate}
}


\begin{document}
	\ExamTitleBlock{10th grade}{Learning evidence T3 Continuous probability and finance supplementary material solutions}{}
	
	\section*{Contents}
	\noindent\textbf{C1 Describes probabilities in continuous variables as a density function.}
	\begin{itemize}
		\item \hyperlink{c1-ex1}{Problem 1 — Factory Dust Model}
		\item \hyperlink{c1-ex2}{Problem 2 — Cooling Time Density}
		\item \hyperlink{c1-ex3}{Problem 3 — River Flow Index}
	\end{itemize}
	\noindent\textbf{C2 Explains how a uniform random variable works.}
	\begin{itemize}
		\item \hyperlink{c2-ex1}{Problem 1 — Bus Arrival Wait Time}
		\item \hyperlink{c2-ex2}{Problem 2 — Random Phone Call Time}
		\item \hyperlink{c2-ex3}{Problem 3 — Choosing a Seat Row}
	\end{itemize}
	\noindent\textbf{C3 Expresses how to work with distributions that have a linear behaviour.}
	\begin{itemize}
		\item \hyperlink{c3-ex1}{Problem 1 — Battery Life with Increasing Density}
		\item \hyperlink{c3-ex2}{Problem 2 — Package Weights with Decreasing Density}
		\item \hyperlink{c3-ex3}{Problem 3 — Temperature Drift}
	\end{itemize}
	\noindent\textbf{C4 Interprets the concept of variance and standard deviation of a probability function.}
	\begin{itemize}
		\item \hyperlink{c4-ex1}{Problem 1 — Uniform Ticket Refunds}
		\item \hyperlink{c4-ex2}{Problem 2 — Normal Battery Voltage}
		\item \hyperlink{c4-ex3}{Problem 3 — Uniform Download Time}
	\end{itemize}
	\noindent\textbf{C5 Uses the normal standard distribution and its tables to calculate probabilities in context situations.}
	\begin{itemize}
		\item \hyperlink{c5-ex1}{Problem 1 — Standard Normal Table Practice I}
		\item \hyperlink{c5-ex2}{Problem 2 — Standard Normal Table Practice II}
		\item \hyperlink{c5-ex3}{Problem 3 — Standard Normal Table Practice III}
	\end{itemize}
	\noindent\textbf{C6 Employs the standardisation of the normal variable in real-life problems.}
	\begin{itemize}
		\item \hyperlink{c6-ex1}{Problem 1 — Test Scores Standardisation}
		\item \hyperlink{c6-ex2}{Problem 2 — Fruit Mass Quality Control}
		\item \hyperlink{c6-ex3}{Problem 3 — Shipping Times}
	\end{itemize}
	\noindent\textbf{C7 Solves problems through the concept of inverse normal distribution.}
	\begin{itemize}
		\item \hyperlink{c7-ex1}{Problem 1 — Top 5\% Cable Lengths}
		\item \hyperlink{c7-ex2}{Problem 2 — Bottom 10\% Commute Times}
		\item \hyperlink{c7-ex3}{Problem 3 — Middle 90\% Tire Lifetimes}
	\end{itemize}
	\noindent\textbf{C8 Summarizes the elements and characteristics that an investment portfolio may have.}
	\begin{itemize}
		\item \hyperlink{c8-ex1}{Problem 1 — Portfolio Returns as a Uniform Model}
		\item \hyperlink{c8-ex2}{Problem 2 — Mixing Two Return Distributions}
		\item \hyperlink{c8-ex3}{Problem 3 — Interpreting a Normal Return Model}
	\end{itemize}
	\noindent\textbf{C9 Establishes that the risk of a portfolio is determined by its probability distribution.}
	\begin{itemize}
		\item \hyperlink{c9-ex1}{Problem 1 — Two Portfolios with the Same Expected Return}
		\item \hyperlink{c9-ex2}{Problem 2 — Comparing Spread in Continuous Models}
		\item \hyperlink{c9-ex3}{Problem 3 — Shape and Tail Risk}
	\end{itemize}
	\noindent\textbf{C10 Evaluates finance-context situations with normal distribution and other continuous random variables.}
	\begin{itemize}
		\item \hyperlink{c10-ex1}{Problem 1 — Monthly Return Probabilities}
		\item \hyperlink{c10-ex2}{Problem 2 — Fees and Returns Together}
	\end{itemize}
	
	\ExamSection{C1 Describes probabilities in continuous variables as a density function.}
	
	\begin{ExamProblems}
		
		\hypertarget{c1-ex1}{}
		\item
		\subsection*{Problem 1 — Factory Dust Model}
		
		\textbf{Problem.}
		A factory models the amount of dust (in grams) collected in a filter per hour by a continuous random variable
		\(X\) supported on \([0,5]\). The density is defined as
		\[
		f(x)=k,\quad 0\le x\le 5.
		\]
		
		\textbf{Question.} Find the constant \(k\) that makes \(f(x)\) a valid density. Then compute
		\(P(1\le X\le 4)\). Interpret the result.
		
		\textbf{Solution.}
		For a constant density on \([0,5]\), the graph is a rectangle of width 5 and height \(k\).
		Its area must be 1:
		\[
		5k=1\Rightarrow k=\frac{1}{5}.
		\]
		Now \(P(1\le X\le 4)\) is the area of a rectangle with width \(4-1=3\) and height \(\frac{1}{5}\):
		\[
		P(1\le X\le 4)=3\cdot\frac{1}{5}=\frac{3}{5}=0.60.
		\]
		\textbf{Interpretation.} There is a 60\% chance the dust amount is between 1 g and 4 g.
		
		% --------------------------------------------------
		
		\hypertarget{c1-ex2}{}
		\item
		\subsection*{Problem 2 — Cooling Time Density}
		
		\textbf{Problem.}
		A science class observes the cooling time (in minutes) of a heated metal rod. The random variable
		\(T\) is continuous on \([2, 8]\) with density
		\[
		f(t)=k(t-2),\quad 2\le t\le 8.
		\]
		
		\textbf{Question.} Find \(k\) so that \(f(t)\) is a valid density, then find \(P(T>6)\).
		
		\textbf{Solution.}
		The graph of \(f(t)=k(t-2)\) on \([2,8]\) is a triangle:
		it starts at 0 when \(t=2\) and reaches height \(6k\) at \(t=8\).
		So total area is
		\[
		\frac{1}{2}(\text{base})(\text{height})=\frac{1}{2}(6)(6k)=18k=1,
		\]
		which gives \(k=\frac{1}{18}\).
		
		For \(P(T>6)\), use the region from 6 to 8. There the shape is a trapezoid with
		left height \(f(6)=\frac{1}{18}(4)=\frac{2}{9}\), right height \(f(8)=\frac{1}{18}(6)=\frac{1}{3}\), and width 2:
		\[
		P(T>6)=\frac{\left(\frac{2}{9}+\frac{1}{3}\right)}{2}\cdot 2
		=\frac{2}{9}+\frac{1}{3}=\frac{5}{9}.
		\]
		\textbf{Interpretation.} About 55.6\% of the rods take longer than 6 minutes to cool.
		
		% --------------------------------------------------
		
		\hypertarget{c1-ex3}{}
		\item
		\subsection*{Problem 3 — River Flow Index}
		
		\textbf{Problem.}
		A river flow index \(F\) (unitless) is modeled on \([0,3]\) by a piecewise-constant density:
		\[
		f(x)=\begin{cases}
		\frac{1}{6}, & 0\le x<1,\\
		\frac{1}{3}, & 1\le x\le 3.
		\end{cases}
		\]
		
		\textbf{Question.} Verify that \(f(x)\) is a valid density, then compute \(P(F\le 2)\).
		
		\textbf{Solution.}
		Check total area by rectangles:
		\[
		\left(1\cdot\frac{1}{6}\right)+\left(2\cdot\frac{1}{3}\right)=\frac{1}{6}+\frac{4}{6}=1,
		\]
		so the density is valid.
		
		Now
		\[
		P(F\le 2)=\left(1\cdot\frac{1}{6}\right)+\left(1\cdot\frac{1}{3}\right)=\frac{1}{6}+\frac{1}{3}=\frac{1}{2}=0.50.
		\]
		\textbf{Interpretation.} There is a 50\% chance the index is at most 2.
		
	\end{ExamProblems}
	
	\ExamSection{C2 Explains how a uniform random variable works.}
	
	\begin{ExamProblems}
		
		\hypertarget{c2-ex1}{}
		\item
		\subsection*{Problem 1 — Bus Arrival Wait Time}
		
		\textbf{Problem.}
		A bus arrives uniformly at any time between 0 and 12 minutes after a student reaches the stop.
		Let \(W\) be the waiting time (minutes). Then \(W\sim\text{Uniform}(0,12)\).
		
		\textbf{Question.} Find \(P(W<5)\) and \(P(4\le W\le 9)\). Interpret both results.
		
		\textbf{Solution.}
		For a uniform model, probability is interval length divided by total length.
		The total length is \(12\).
		
		First probability:
		\[
		P(W<5)=\frac{5-0}{12} = \frac{5}{12}.
		\]
		Second probability:
		\[
		P(4\le W\le 9)=\frac{9-4}{12}=\frac{5}{12}.
		\]
		\textbf{Interpretation.} There is a \(\frac{5}{12}\) (about 41.7\%) chance the wait is under 5 minutes,
		and the same chance that the wait is between 4 and 9 minutes.
		
		% --------------------------------------------------
		
		\hypertarget{c2-ex2}{}
		\item
		\subsection*{Problem 2 — Random Phone Call Time}
		
		\textbf{Problem.}
		A student receives a phone call at a random time between 18:00 and 19:00.
		Let \(T\sim\text{Uniform}(0,60)\) be the minutes after 18:00 when the call arrives.
		
		\textbf{Question.} Find \(P(10\le T\le 25)\) and \(P(T<10\ \text{or}\ T>50)\).
		
		\textbf{Solution.}
		The density is constant: \(f(t)=\frac{1}{60}\) for \(0\le t\le 60\).
		
		First probability:
		\[
		P(10\le T\le 25)=\frac{25-10}{60}=\frac{15}{60}=0.25.
		\]
		For the union of intervals:
		\[
		P(T<10\ \text{or}\ T>50)=\frac{10}{60}+\frac{10}{60}=\frac{20}{60}=\frac{1}{3}.
		\]
		\textbf{Interpretation.} There is a 25\% chance the call arrives between 18:10 and 18:25,
		and a one-third chance it arrives in the first 10 minutes or last 10 minutes.
		
		% --------------------------------------------------
		
		\hypertarget{c2-ex3}{}
		\item
		\subsection*{Problem 3 — Choosing a Seat Row}
		
		\textbf{Problem.}
		A theater row is numbered from 1 to 20, and a random seat is chosen uniformly along the row.
		Let \(S\sim\text{Uniform}(1,20)\) represent the seat number (treated as continuous for modelling).
		
		\textbf{Question.} Find \(P(5\le S\le 12)\) and \(P(S\le 4\ \text{or}\ 16\le S\le 20)\).
		
		\textbf{Solution.}
		The uniform density is \(f(s)=\frac{1}{19}\) for \(1\le s\le 20\).
		
		First probability:
		\[
		P(5\le S\le 12)=\frac{12-5}{19}=\frac{7}{19}\approx 0.368.
		\]
		Second probability:
		\[
		P(S\le 4\ \text{or}\ 16\le S\le 20)=\frac{4-1}{19}+\frac{20-16}{19}
		=\frac{3}{19}+\frac{4}{19}=\frac{7}{19}.
		\]
		\textbf{Interpretation.} The chance of a seat between rows 5 and 12 is about 36.8\%,
		and the same chance for the extreme front or back rows.
		
	\end{ExamProblems}
	
	\ExamSection{C3 Expresses how to work with distributions that have a linear behaviour.}
	
	\begin{ExamProblems}
		
		\hypertarget{c3-ex1}{}
		\item
		\subsection*{Problem 1 — Battery Life with Increasing Density}
		
		\textbf{Problem.}
		A battery's lifetime (hours) is modeled by a continuous random variable \(X\) on \([0, 6]\) with
		linear density
		\[
		f(x)=a+bx,\quad 0\le x\le 6,
		\]
		and \(f(0)=0\). That means the density starts at 0 and increases linearly.
		
		\textbf{Question.} Find \(f(x)\) explicitly, then compute \(P(X\le 3)\). Interpret the result.
		
		\textbf{Solution.}
		The condition \(f(0)=0\) gives \(a=0\), so \(f(x)=bx\).
		Its graph on \([0,6]\) is a triangle with base 6 and height \(f(6)=6b\).
		Total area must be 1:
		\[
		\frac{1}{2}(6)(6b)=18b=1\Rightarrow b=\frac{1}{18}.
		\]
		So \(f(x)=\frac{1}{18}x\).
		
		Now \(P(X\le 3)\) is the area of the small similar triangle from 0 to 3.
		Its height at \(x=3\) is \(f(3)=\frac{3}{18}=\frac{1}{6}\), so
		\[
		P(X\le 3)=\frac{1}{2}(3)\left(\frac{1}{6}\right)=\frac{1}{4}.
		\]
		\textbf{Interpretation.} Only 25\% of batteries last 3 hours or less, so longer lives are more likely.
		
		% --------------------------------------------------
		
		\hypertarget{c3-ex2}{}
		\item
		\subsection*{Problem 2 — Package Weights with Decreasing Density}
		
		\textbf{Problem.}
		A delivery service models package weight (kg) by a linear density on \([0, 4]\):
		\[
		f(w)=a-bw,\quad 0\le w\le 4,
		\]
		with \(f(4)=0\) so heavier packages are less common.
		
		\textbf{Question.} Determine \(f(w)\), then find \(P(1\le W\le 3)\). Interpret the result.
		
		\textbf{Solution.}
		From \(f(4)=0\), we have \(a-4b=0\Rightarrow a=4b\).
		The graph of \(f(w)=a-bw\) on \([0,4]\) is a triangle with base 4 and height \(a\).
		Total area is 1:
		\[
		\frac{1}{2}(4)(a)=1\Rightarrow 2a=1\Rightarrow a=\frac{1}{2}.
		\]
		Then \(a=4b\Rightarrow b=\frac{1}{8}\), so
		\[
		f(w)=\frac{1}{2}-\frac{1}{8}w.
		\]
		
		Now find \(P(1\le W\le 3)\) as trapezoid area on \([1,3]\).
		Heights are \(f(1)=\frac{3}{8}\) and \(f(3)=\frac{1}{8}\), width is 2:
		\[
		P(1\le W\le 3)=\frac{\left(\frac{3}{8}+\frac{1}{8}\right)}{2}\cdot 2=\frac{1}{2}.
		\]
		\textbf{Interpretation.} About 50\% of packages weigh between 1 kg and 3 kg.
		
		% --------------------------------------------------
		
		\hypertarget{c3-ex3}{}
		\item
		\subsection*{Problem 3 — Temperature Drift}
		
		\textbf{Problem.}
		A sensor records a temperature drift \(Y\) (\(^\circ\)C) over an hour, modeled on \([0,4]\) by
		\[
		f(y)=a+by,\quad 0\le y\le 4,
		\]
		with \(f(0)=\frac{1}{8}\) and \(f(4)=\frac{3}{8}\).
		
		\textbf{Question.} Find \(a\) and \(b\), then compare \(P(0\le Y\le 1)\) with \(P(3\le Y\le 4)\).
		
		\textbf{Solution.}
		The conditions give \(a=\frac{1}{8}\) and \(\frac{1}{8}+4b=\frac{3}{8}\Rightarrow b=\frac{1}{16}\).
		So
		\[
		f(y)=\frac{1}{8}+\frac{1}{16}y.
		\]
		The full graph on \([0,4]\) is a trapezoid with heights \(\frac{1}{8}\) and \(\frac{3}{8}\), width 4:
		\[
		\text{total area}=\frac{\left(\frac{1}{8}+\frac{3}{8}\right)}{2}\cdot 4=1,
		\]
		so it is a valid density.
		
		For \([0,1]\), use a trapezoid with heights \(f(0)=\frac{1}{8}\), \(f(1)=\frac{3}{16}\), width 1:
		\[
		P(0\le Y\le 1)=\frac{\left(\frac{1}{8}+\frac{3}{16}\right)}{2}\cdot 1=\frac{5}{32}.
		\]
		For \([3,4]\), heights are \(f(3)=\frac{5}{16}\), \(f(4)=\frac{3}{8}\), width 1:
		\[
		P(3\le Y\le 4)=\frac{\left(\frac{5}{16}+\frac{3}{8}\right)}{2}\cdot 1=\frac{11}{32}.
		\]
		\textbf{Interpretation.} The interval \([3,4]\) is more likely because the density increases with \(y\).
		
	\end{ExamProblems}
	
	\ExamSection{C4 Interprets the concept of variance and standard deviation of a probability function.}
	
	\begin{ExamProblems}
		
		\hypertarget{c4-ex1}{}
		\item
		\subsection*{Problem 1 — Uniform Ticket Refunds}
		
		\textbf{Problem.}
		A theater issues random refunds uniformly between \(\$0\) and \(\$20\) when a show is delayed.
		Let \(R\sim\text{Uniform}(0,20)\) denote the refund amount.
		
		\textbf{Question.} Compute \(E[R]\), \(\mathrm{Var}(R)\), and the standard deviation. Interpret the spread.
		
		\textbf{Solution.}
		For a uniform \([a,b]\):
		\[
		E[R]=\frac{a+b}{2},\quad \mathrm{Var}(R)=\frac{(b-a)^2}{12}.
		\]
		Here \(a=0\), \(b=20\), so
		\[
		E[R]=\frac{0+20}{2}=10,\quad \mathrm{Var}(R)=\frac{20^2}{12}=\frac{400}{12}=\frac{100}{3}.
		\]
		Standard deviation:
		\[
		\sigma=\sqrt{\frac{100}{3}}\approx 5.77.
		\]
		\textbf{Interpretation.} The variance \(\frac{100}{3}\) measures the average squared spread around \(10\),
		while the standard deviation of about \$5.77 is the typical distance from the mean in dollars.
		
		% --------------------------------------------------
		
		\hypertarget{c4-ex2}{}
		\item
		\subsection*{Problem 2 — Normal Battery Voltage}
		
		\textbf{Problem.}
		The voltage of a battery cell is approximately normal with mean 1.5 V and standard deviation 0.08 V.
		Let \(V\sim N(1.5,0.08^2)\).
		
		\textbf{Question.} State the variance and standard deviation. Explain what each tells you
		about voltage variability.
		
		\textbf{Solution.}
		For a normal distribution, the variance is \(\sigma^2\) and the standard deviation is \(\sigma\).
		Here
		\[
		\mathrm{Var}(V)=0.08^2=0.0064,\qquad \sigma=0.08\text{ V}.
		\]
		\textbf{Interpretation.} The variance (0.0064) describes the average squared deviation from 1.5 V,
		while the standard deviation (0.08 V) is the typical size of a voltage fluctuation in volts.
		
		% --------------------------------------------------
		
		\hypertarget{c4-ex3}{}
		\item
		\subsection*{Problem 3 — Uniform Download Time}
		
		\textbf{Problem.}
		A file download time \(T\) (minutes) is uniformly distributed between 6 and 14 minutes.
		Let \(T\sim\text{Uniform}(6,14)\).
		
		\textbf{Question.} Compute the variance and standard deviation, and explain what they mean
		about the consistency of download times.
		
		\textbf{Solution.}
		For a uniform \([a,b]\),
		\[
		\mathrm{Var}(T)=\frac{(b-a)^2}{12}.
		\]
		Here \(b-a=8\), so
		\[
		\mathrm{Var}(T)=\frac{8^2}{12}=\frac{64}{12}=\frac{16}{3}.
		\]
		The standard deviation is
		\[
		\sigma=\sqrt{\frac{16}{3}}\approx 2.31\text{ minutes}.
		\]
		\textbf{Interpretation.} The variance \(\frac{16}{3}\) measures squared spread around the mean,
		while the standard deviation of about 2.31 minutes tells us typical variation in actual time.
		
	\end{ExamProblems}
	
	\ExamSection{C5 Uses the normal standard distribution and its tables to calculate probabilities in context situations.}
	
	\noindent\textbf{Standard normal table (values of \(\Phi(z)\)).} Use this same table for all problems in this section.
	\begin{center}
	\begin{tabular}{c|cccccccccc}
		\toprule
		$z$ & 0.00 & 0.01 & 0.02 & 0.03 & 0.04 & 0.05 & 0.06 & 0.07 & 0.08 & 0.09 \\
		\midrule
		0.0 & 0.5000 & 0.5040 & 0.5080 & 0.5120 & 0.5160 & 0.5199 & 0.5239 & 0.5279 & 0.5319 & 0.5359 \\
		0.1 & 0.5398 & 0.5438 & 0.5478 & 0.5517 & 0.5557 & 0.5596 & 0.5636 & 0.5675 & 0.5714 & 0.5753 \\
		0.2 & 0.5793 & 0.5832 & 0.5871 & 0.5910 & 0.5948 & 0.5987 & 0.6026 & 0.6064 & 0.6103 & 0.6141 \\
		0.3 & 0.6179 & 0.6217 & 0.6255 & 0.6293 & 0.6331 & 0.6368 & 0.6406 & 0.6443 & 0.6480 & 0.6517 \\
		0.4 & 0.6554 & 0.6591 & 0.6628 & 0.6664 & 0.6700 & 0.6736 & 0.6772 & 0.6808 & 0.6844 & 0.6879 \\
		0.5 & 0.6915 & 0.6950 & 0.6985 & 0.7019 & 0.7054 & 0.7088 & 0.7123 & 0.7157 & 0.7190 & 0.7224 \\
		0.6 & 0.7257 & 0.7291 & 0.7324 & 0.7357 & 0.7389 & 0.7422 & 0.7454 & 0.7486 & 0.7517 & 0.7549 \\
		0.7 & 0.7580 & 0.7611 & 0.7642 & 0.7673 & 0.7704 & 0.7734 & 0.7764 & 0.7794 & 0.7823 & 0.7852 \\
		0.8 & 0.7881 & 0.7910 & 0.7939 & 0.7967 & 0.7995 & 0.8023 & 0.8051 & 0.8078 & 0.8106 & 0.8133 \\
		0.9 & 0.8159 & 0.8186 & 0.8212 & 0.8238 & 0.8264 & 0.8289 & 0.8315 & 0.8340 & 0.8365 & 0.8389 \\
		1.0 & 0.8413 & 0.8438 & 0.8461 & 0.8485 & 0.8508 & 0.8531 & 0.8554 & 0.8577 & 0.8599 & 0.8621 \\
		1.1 & 0.8643 & 0.8665 & 0.8686 & 0.8708 & 0.8729 & 0.8749 & 0.8770 & 0.8790 & 0.8810 & 0.8830 \\
		1.2 & 0.8849 & 0.8869 & 0.8888 & 0.8907 & 0.8925 & 0.8944 & 0.8962 & 0.8980 & 0.8997 & 0.9015 \\
		1.3 & 0.9032 & 0.9049 & 0.9066 & 0.9082 & 0.9099 & 0.9115 & 0.9131 & 0.9147 & 0.9162 & 0.9177 \\
		1.4 & 0.9192 & 0.9207 & 0.9222 & 0.9236 & 0.9251 & 0.9265 & 0.9279 & 0.9292 & 0.9306 & 0.9319 \\
		1.5 & 0.9332 & 0.9345 & 0.9357 & 0.9370 & 0.9382 & 0.9394 & 0.9406 & 0.9418 & 0.9429 & 0.9441 \\
		1.6 & 0.9452 & 0.9463 & 0.9474 & 0.9484 & 0.9495 & 0.9505 & 0.9515 & 0.9525 & 0.9535 & 0.9545 \\
		1.7 & 0.9554 & 0.9564 & 0.9573 & 0.9582 & 0.9591 & 0.9599 & 0.9608 & 0.9616 & 0.9625 & 0.9633 \\
		1.8 & 0.9641 & 0.9649 & 0.9656 & 0.9664 & 0.9671 & 0.9678 & 0.9686 & 0.9693 & 0.9699 & 0.9706 \\
		1.9 & 0.9713 & 0.9719 & 0.9726 & 0.9732 & 0.9738 & 0.9744 & 0.9750 & 0.9756 & 0.9761 & 0.9767 \\
		2.0 & 0.9772 & 0.9778 & 0.9783 & 0.9788 & 0.9793 & 0.9798 & 0.9803 & 0.9808 & 0.9812 & 0.9817 \\
		\bottomrule
	\end{tabular}
	\end{center}
	
	\begin{ExamProblems}
		
		\hypertarget{c5-ex1}{}
		\item
		\subsection*{Problem 1 — Standard Normal Table Practice I}
		
		\textbf{Problem.}
		Let \(Z\sim N(0,1)\).
		
		\textbf{Question.} Use the standard normal table to find \(P(Z<1.25)\) and \(P(Z>0.60)\).
		
		\textbf{Solution.}
		From the table, \(\Phi(1.25)=0.8944\).
		So
		\[
		P(Z<1.25)=0.8944.
		\]
		Also \(\Phi(0.60)=0.7257\), so
		\[
		P(Z>0.60)=1-\Phi(0.60)=1-0.7257=0.2743.
		\]
		\textbf{Interpretation.} About 89.4\% of values lie below 1.25, and 27.4\% lie above 0.60.
		
		% --------------------------------------------------
		
		\hypertarget{c5-ex2}{}
		\item
		\subsection*{Problem 2 — Standard Normal Table Practice II}
		
		\textbf{Problem.}
		Let \(Z\sim N(0,1)\).
		
		\textbf{Question.} Use the table to compute \(P(-0.40\le Z\le 1.10)\).
		
		\textbf{Solution.}
		From the table, \(\Phi(1.10)=0.8643\) and \(\Phi(0.40)=0.6554\).
		Because \(\Phi(-0.40)=1-\Phi(0.40)=0.3446\),
		\[
		P(-0.40\le Z\le 1.10)=\Phi(1.10)-\Phi(-0.40)=0.8643-0.3446=0.5197.
		\]
		\textbf{Interpretation.} About 52.0\% of standard normal values lie between -0.40 and 1.10.
		
		% --------------------------------------------------
		
		\hypertarget{c5-ex3}{}
		\item
		\subsection*{Problem 3 — Standard Normal Table Practice III}
		
		\textbf{Problem.}
		Let \(Z\sim N(0,1)\).
		
		\textbf{Question.} Use the table to find \(P(Z\le -1.30\ \text{or}\ Z\ge 0.80)\).
		
		\textbf{Solution.}
		From the table, \(\Phi(1.30)=0.9032\) and \(\Phi(0.80)=0.7881\).
		So
		\[
		P(Z\le -1.30)=1-\Phi(1.30)=0.0968,
		\qquad P(Z\ge 0.80)=1-\Phi(0.80)=0.2119.
		\]
		Thus
		\[
		P(Z\le -1.30\ \text{or}\ Z\ge 0.80)=0.0968+0.2119=0.3087.
		\]
		\textbf{Interpretation.} About 30.9\% of values lie in the two tails outside [-1.30, 0.80].
		
	\end{ExamProblems}
	
	\ExamSection{C6 Employs the standardisation of the normal variable in real-life problems.}
	
	\begin{ExamProblems}
		
		\hypertarget{c6-ex1}{}
		\item
		\subsection*{Problem 1 — Test Scores Standardisation}
		
		\textbf{Problem.}
		A math test score \(X\) is normally distributed with mean 70 and standard deviation 8.
		
		\textbf{Question.} Find the probability that a randomly chosen student scores above 85.
		Show the standardisation step clearly.
		
		\textbf{Solution.}
		Standardize:
		\[
		Z=\frac{X-70}{8}.
		\]
		Then
		\[
		P(X>85)=P\left(Z>\frac{85-70}{8}\right)=P(Z>1.875).
		\]
		From the standard normal table, \(\Phi(1.88)\approx 0.9699\).
		So
		\[
		P(Z>1.875)\approx 1-0.9699=0.0301.
		\]
		\textbf{Interpretation.} Only about 3\% of students score above 85.
		
		% --------------------------------------------------
		
		\hypertarget{c6-ex2}{}
		\item
		\subsection*{Problem 2 — Fruit Mass Quality Control}
		
		\textbf{Problem.}
		The mass of apples from an orchard is normally distributed with mean 180 g
		and standard deviation 12 g. Apples are considered "large" if they weigh more than 200 g.
		
		\textbf{Question.} Find the probability that an apple is classified as large.
		
		\textbf{Solution.}
		Standardize:
		\[
		Z=\frac{X-180}{12}.
		\]
		Compute
		\[
		P(X>200)=P\left(Z>\frac{200-180}{12}\right)=P(Z>1.67\text{ (approx)}).
		\]
		From the table, \(\Phi(1.67)\approx 0.9525\).
		Thus
		\[
		P(X>200)\approx 1-0.9525=0.0475.
		\]
		\textbf{Interpretation.} About 4.8\% of apples are large.
		
		% --------------------------------------------------
		
		\hypertarget{c6-ex3}{}
		\item
		\subsection*{Problem 3 — Shipping Times}
		
		\textbf{Problem.}
		Shipping times for a package are normally distributed with mean 3.4 days and
		standard deviation 0.5 days.
		
		\textbf{Question.} Find the probability that a package arrives between 3 and 4 days.
		Show the standardisation step before using the table.
		
		\textbf{Solution.}
		Let \(X\) be shipping time. Standardize the endpoints:
		\[
		Z_1=\frac{3-3.4}{0.5}=-0.8,\qquad Z_2=\frac{4-3.4}{0.5}=1.2.
		\]
		Thus
		\[
		P(3\le X\le 4)=P(-0.8\le Z\le 1.2)=\Phi(1.2)-\Phi(-0.8).
		\]
		From the table, \(\Phi(1.2)=0.8849\) and \(\Phi(-0.8)=1-\Phi(0.8)=1-0.7881=0.2119\).
		So
		\[
		P(3\le X\le 4)=0.8849-0.2119=0.6730.
		\]
		\textbf{Interpretation.} About 67.3\% of packages arrive between 3 and 4 days.
		
	\end{ExamProblems}
	
	\ExamSection{C7 Solves problems through the concept of inverse normal distribution.}
	
	\begin{ExamProblems}
		
		\hypertarget{c7-ex1}{}
		\item
		\subsection*{Problem 1 — Top 5\% Cable Lengths}
		
		\textbf{Problem.}
		Cables are manufactured with lengths \(L\sim N(50\text{ cm}, 1.2^2\text{ cm}^2)\).
		
		\textbf{Question.} Find the length \(\ell\) such that only 5\% of cables are longer than \(\ell\).
		State the target probability before using the inverse normal table.
		
		\textbf{Solution.}
		"Top 5\%" means
		\[
		P(L>\ell)=0.05\quad\Rightarrow\quad P(L\le \ell)=0.95.
		\]
		Convert to a standard normal value:
		\[
		P\left(Z\le \frac{\ell-50}{1.2}\right)=0.95.
		\]
		From the inverse normal table, \(z_{0.95}\approx 1.645\).
		So
		\[
		\frac{\ell-50}{1.2}=1.645\Rightarrow \ell=50+1.2(1.645)\approx 51.97\text{ cm}.
		\]
		\textbf{Interpretation.} About 5\% of cables are longer than 52.0 cm.
		
		% --------------------------------------------------
		
		\hypertarget{c7-ex2}{}
		\item
		\subsection*{Problem 2 — Bottom 10\% Commute Times}
		
		\textbf{Problem.}
		Commute times to school are normally distributed with mean 25 minutes and
		standard deviation 6 minutes.
		
		\textbf{Question.} Find the time \(t\) such that 10\% of commutes are shorter than \(t\).
		
		\textbf{Solution.}
		"Bottom 10\%" means
		\[
		P(T\le t)=0.10.
		\]
		Standardize:
		\[
		P\left(Z\le \frac{t-25}{6}\right)=0.10.
		\]
		From the inverse normal table, \(z_{0.10}\approx -1.28\).
		So
		\[
		\frac{t-25}{6}=-1.28\Rightarrow t=25-1.28(6)=25-7.68=17.32\text{ minutes}.
		\]
		\textbf{Interpretation.} About 10\% of students arrive in 17.3 minutes or less.
		
		% --------------------------------------------------
		
		\hypertarget{c7-ex3}{}
		\item
		\subsection*{Problem 3 — Middle 90\% Tire Lifetimes}
		
		\textbf{Problem.}
		A tire model has lifetimes \(L\sim N(40{,}000\text{ km}, 4{,}000^2\text{ km}^2)\).
		
		\textbf{Question.} Find the interval of lifetimes that contains the middle 90\% of tires.
		
		\textbf{Solution.}
		The middle 90\% leaves 5\% in each tail, so we need the 5th and 95th percentiles.
		Let \(a\) be the 5th percentile and \(b\) the 95th percentile. Then
		\[
		P(L\le a)=0.05,\qquad P(L\le b)=0.95.
		\]
		From inverse normal values, \(z_{0.05}\approx -1.645\) and \(z_{0.95}\approx 1.645\).
		Standardize:
		\[
		\frac{a-40{,}000}{4{,}000}=-1.645,\qquad \frac{b-40{,}000}{4{,}000}=1.645.
		\]
		So
		\[
		a=40{,}000-1.645(4{,}000)=33{,}420\text{ km},\qquad b=40{,}000+1.645(4{,}000)=46{,}580\text{ km}.
		\]
		\textbf{Interpretation.} About 90\% of tires last between roughly 33{,}400 km and 46{,}600 km.
		
	\end{ExamProblems}
	
	\ExamSection{C8 Summarizes the elements and characteristics that an investment portfolio may have.}
	
	\begin{ExamProblems}
		
		\hypertarget{c8-ex1}{}
		\item
		\subsection*{Problem 1 — Portfolio Returns as a Uniform Model}
		
		\textbf{Problem.}
		A student savings portfolio has monthly returns modeled as a uniform distribution
		between -1\% and 5\%. Let \(R\sim\text{Uniform}(-0.01,0.05)\).
		
		\textbf{Question.} Find the probability of a non-negative return and describe what the uniform model
		says about the portfolio's variability.
		
		\textbf{Solution.}
		The density is constant on the interval, so
		\[
		P(R\ge 0)=\frac{0.05-0}{0.05-(-0.01)}=\frac{0.05}{0.06}=\frac{5}{6}\approx 0.833.
		\]
		\textbf{Interpretation.} The uniform model treats all returns between -1\% and 5\% as equally likely,
		so variability is spread evenly across that range. About 83.3\% of months are non-negative.
		
		% --------------------------------------------------
		
		\hypertarget{c8-ex2}{}
		\item
		\subsection*{Problem 2 — Mixing Two Return Distributions}
		
		\textbf{Problem.}
		A club invests in two funds. Fund A has returns modeled by a linear density on \([0\%,8\%]\)
		with endpoints \(f_A(0)=0\) and \(f_A(0.08)=25\). Fund B has returns modeled by \(N(4\%,2\%^2)\).
		
		\textbf{Question.} For Fund A, verify that this is a valid density and compute \(P(0.02\le R_A\le 0.06)\).
		Then explain qualitatively how the two distributions differ in shape and what that means for the portfolio.
		
		\textbf{Solution.}
		For Fund A, the density graph is a triangle with base \(0.08\) and height \(25\), so
		\[
		\text{total area}=\frac{1}{2}(0.08)(25)=1,
		\]
		which confirms a valid density.
		
		Because the line rises from 0 to 25, the slope is \(25/0.08=312.5\), so
		\[
		f_A(r)=312.5r.
		\]
		Now use trapezoid area on \([0.02,0.06]\).
		Heights: \(f_A(0.02)=6.25\), \(f_A(0.06)=18.75\), width \(0.04\):
		\[
		P(0.02\le R_A\le 0.06)=\frac{(6.25+18.75)}{2}\cdot 0.04=12.5\cdot 0.04=0.50.
		\]
		\textbf{Interpretation.} Fund A has a linear density that rises with return, so higher returns are more likely.
		Fund B is bell-shaped around 4\%, so most outcomes cluster near the mean with fewer extreme values.
		The portfolio elements differ in shape and spread, affecting both expected outcomes and variability.
		
		% --------------------------------------------------
		
		\hypertarget{c8-ex3}{}
		\item
		\subsection*{Problem 3 — Interpreting a Normal Return Model}
		
		\textbf{Problem.}
		A scholarship fund models annual returns as \(R\sim N(0.06,0.03^2)\).
		
		\textbf{Question.} Estimate the probability that the return is between 3\% and 9\%, and explain
		what the standard deviation tells you about the portfolio.
		
		\textbf{Solution.}
		Standardize:
		\[
		Z_1=\frac{0.03-0.06}{0.03}=-1,\qquad Z_2=\frac{0.09-0.06}{0.03}=1.
		\]
		Thus
		\[
		P(0.03\le R\le 0.09)=P(-1\le Z\le 1)=\Phi(1)-\Phi(-1).
		\]
		From the table, \(\Phi(1)=0.8413\) and \(\Phi(-1)=1-\Phi(1)=0.1587\).
		So
		\[
		P(0.03\le R\le 0.09)=0.8413-0.1587=0.6826.
		\]
		\textbf{Interpretation.} About 68\% of returns fall within one standard deviation of the mean,
		so the standard deviation (3\%) summarizes typical yearly variation for this portfolio.
		
	\end{ExamProblems}
	
	\ExamSection{C9 Establishes that the risk of a portfolio is determined by its probability distribution.}
	
	\begin{ExamProblems}
		
		\hypertarget{c9-ex1}{}
		\item
		\subsection*{Problem 1 — Two Portfolios with the Same Expected Return}
		
		\textbf{Problem.}
		Two portfolios have the following annual return distributions:
		\[
		\begin{array}{lccc}
		\toprule
		\text{Return} & -4\% & 6\% & 16\% \\
		\midrule
		\text{Portfolio A} & 0.25 & 0.50 & 0.25 \\
		\text{Portfolio B} & 0.10 & 0.80 & 0.10 \\
		\bottomrule
		\end{array}
		\]
		
		\textbf{Question.} Compute the expected return and variance for each portfolio and decide
		which one is riskier.
		
		\textbf{Solution.}
		Expected return for A:
		\[
		E[R_A]=0.25(-0.04)+0.50(0.06)+0.25(0.16)= -0.01+0.03+0.04=0.06.
		\]
		Expected return for B:
		\[
		E[R_B]=0.10(-0.04)+0.80(0.06)+0.10(0.16)= -0.004+0.048+0.016=0.06.
		\]
		Both have the same expected return (6\%). Now compute variance.
		
		For A:
		\[
		E[R_A^2]=0.25(0.04^2)+0.50(0.06^2)+0.25(0.16^2)
		=0.25(0.0016)+0.50(0.0036)+0.25(0.0256)
		=0.0004+0.0018+0.0064=0.0086.
		\]
		\[
		\mathrm{Var}(R_A)=0.0086-(0.06)^2=0.0086-0.0036=0.0050.
		\]
		For B:
		\[
		E[R_B^2]=0.10(0.04^2)+0.80(0.06^2)+0.10(0.16^2)
		=0.10(0.0016)+0.80(0.0036)+0.10(0.0256)
		=0.00016+0.00288+0.00256=0.0056.
		\]
		\[
		\mathrm{Var}(R_B)=0.0056-(0.06)^2=0.0056-0.0036=0.0020.
		\]
		\textbf{Interpretation.} Portfolio A has a larger variance, so its distribution is more spread out
		and it is riskier, even though both have the same expected return.
		
		% --------------------------------------------------
		
		\hypertarget{c9-ex2}{}
		\item
		\subsection*{Problem 2 — Comparing Spread in Continuous Models}
		
		\textbf{Problem.}
		Portfolio C has returns modeled as \(R_C\sim\text{Uniform}(-0.02,0.08)\).
		Portfolio D has returns modeled as \(R_D\sim N(0.03,0.02^2)\).
		
		\textbf{Question.} Compute the variance for each portfolio and explain which one is riskier
		based on distributional spread.
		
		\textbf{Solution.}
		For Portfolio C (uniform),
		\[
		\mathrm{Var}(R_C)=\frac{(0.08-(-0.02))^2}{12}=\frac{0.10^2}{12}=\frac{0.01}{12}\approx 0.000833.
		\]
		For Portfolio D (normal), the variance is \(\sigma^2=0.02^2=0.0004\).
		
		\textbf{Interpretation.} Portfolio C has a larger variance, so its distribution is wider.
		That wider spread means more variability in returns, which indicates higher risk.
		
		% --------------------------------------------------
		
		\hypertarget{c9-ex3}{}
		\item
		\subsection*{Problem 3 — Shape and Tail Risk}
		
		\textbf{Problem.}
		Two portfolios have continuous return models on \([-5\%,15\%]\).
		Portfolio E has a linear density that increases with return, with
		\(f_E(-0.05)=0\) and \(f_E(0.15)=10\).
		Portfolio F has a linear density that decreases with return, with
		\(f_F(-0.05)=10\) and \(f_F(0.15)=0\).
		
		\textbf{Question.} Verify that each is a valid density, then compare the probability of a loss
		(\(R<0\)) for the two portfolios. Explain what this says about risk.
		
		\textbf{Solution.}
		For each portfolio, the density graph is a triangle with base \(0.20\) and height \(10\), so
		\[
		\text{total area}=\frac{1}{2}(0.20)(10)=1.
		\]
		So both are valid densities.
		
		For Portfolio E, \(R<0\) is the left small triangle from \(-0.05\) to 0.
		The height at 0 is halfway up the line, so \(f_E(0)=5\).
		Thus
		\[
		P_E(R<0)=\frac{1}{2}(0.05)(5)=0.125.
		\]
		For Portfolio F, \(R<0\) is a trapezoid on \([-0.05,0]\) with heights
		\(f_F(-0.05)=10\) and \(f_F(0)=7.5\), width \(0.05\):
		\[
		P_F(R<0)=\frac{(10+7.5)}{2}\cdot 0.05=8.75\cdot 0.05=0.4375.
		\]
		\textbf{Interpretation.} Portfolio F puts more probability on low returns and has a much higher chance
		of a loss. The shape of the distribution (increasing vs. decreasing) directly affects risk.
		
	\end{ExamProblems}
	
	\ExamSection{C10 Evaluates finance-context situations with normal distribution and other continuous random variables.}
	
	\begin{ExamProblems}
		
		\hypertarget{c10-ex1}{}
		\item
		\subsection*{Problem 1 — Monthly Return Probabilities}
		
		\textbf{Problem.}
		A mutual fund's monthly return \(R\) is approximately normal with mean 1.2\% and
		standard deviation 3\%. Assume \(R\sim N(0.012,0.03^2)\).
		
		\textbf{Question.} Find the probability that the fund has a loss in a month
		(\(R<0\)). Interpret the result.
		
		\textbf{Solution.}
		Standardize:
		\[
		Z=\frac{R-0.012}{0.03}.
		\]
		Then
		\[
		P(R<0)=P\left(Z<\frac{0-0.012}{0.03}\right)=P(Z<-0.40).
		\]
		From the table, \(\Phi(-0.40)=1-\Phi(0.40)\approx 1-0.6554=0.3446\).
		\textbf{Interpretation.} There is about a 34\% chance of a negative return in a month.
		
		% --------------------------------------------------
		
		\hypertarget{c10-ex2}{}
		\item
		\subsection*{Problem 2 — Fees and Returns Together}
		
		\textbf{Problem.}
		An investment platform charges a monthly fee \(F\) (in dollars) with this distribution:
		\[
		P(F=2)=0.25,\quad P(F=5)=0.50,\quad P(F=8)=0.25.
		\]
		The monthly portfolio return \(R\) (in dollars)
		is modeled by a normal distribution with mean \(\$25\) and standard deviation \(\$10\).
		
		\textbf{Question.} Find the probability that the net gain \(G=R-F\) exceeds \(\$20\).
		Clearly define the random variables and interpret the result.
		
		\textbf{Solution.}
		Define \(R\sim N(25,10^2)\) and fee outcomes \(F\in\{2,5,8\}\) with the stated probabilities.
		We want
		\[
		P(G>20)=P(R-F>20).
		\]
		Use a weighted sum over the three fee values:
		\[
		P(G>20)=0.25\,P(R>22)+0.50\,P(R>25)+0.25\,P(R>28).
		\]
		Standardize each term:
		\[
		P(R>22)=P\left(Z>\frac{22-25}{10}\right)=P(Z>-0.3)=0.6179,
		\]
		\[
		P(R>25)=P(Z>0)=0.5000,
		\]
		\[
		P(R>28)=P\left(Z>\frac{28-25}{10}\right)=P(Z>0.3)=0.3821.
		\]
		Now compute:
		\[
		P(G>20)=0.25(0.6179)+0.50(0.5000)+0.25(0.3821)
		=0.154475+0.250000+0.095525=0.5000.
		\]
		\textbf{Interpretation.} There is a 50\% chance that the net gain exceeds \$20
		when both return and fee variability are considered.
		
	\end{ExamProblems}
	
\end{document}
