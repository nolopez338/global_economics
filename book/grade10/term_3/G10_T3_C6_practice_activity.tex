\makeatletter
\def\input@path{{./}{../}{../../}{preamble/}{../preamble/}{../../preamble/}}
\makeatother
% ----------------------------------------------------------
% GENERAL 

% File
\documentclass[11pt]{book}

% Margins
\usepackage[margin=1in]{geometry}

\linespread{1.2}            % Line spacing
\usepackage[utf8]{inputenc}
\usepackage[T1]{fontenc}
\usepackage{lmodern}
\usepackage{microtype}
\setlength{\parindent}{0pt}
\setlength{\parskip}{6pt}
\usepackage{booktabs}

% ----------------------------------------------------------
% TABLES
\usepackage{multicol}
\usepackage{longtable} 
\usepackage{array}
\usepackage{booktabs}
\usepackage{tabularx}
\usepackage{multirow}

% ----------------------------------------------------------
% MATHEMATICS
\usepackage{amsmath}
\usepackage{amssymb}
\usepackage{amsfonts}
\usepackage{mathtools}

% ----------------------------------------------------------
% HIDDEN CONTENT
\usepackage{ifthen}
% Define a boolean switch
\newboolean{explicaciones}
% Set the boolean switch to true or false
% Change to true to show the content

% Explanations
\newcommand{\explicacion}[2]{
	\ifthenelse{\boolean{explicaciones}}{#1}{#2}
}
\newcommand{\mostrarExplicaciones}[1]{\setboolean{explicaciones}{#1}}

% ----------------------------------------------------------
% NUMBERING

\usepackage{fancyhdr}
\pagestyle{empty} % Ensures the entire document has no page numbers

\usepackage{tocloft}
\renewcommand{\cftdot}{} % Remove dots for sections, if any
\renewcommand{\cftsecleader}{\cftdotfill{\cftdotsep}} % Remove dots for sections, if any
\cftpagenumbersoff{section} % Removes page numbers from sections
\cftpagenumbersoff{subsection} % Removes page numbers from subsections

% ----------------------------------------------------------
% IMAGES 

% General settings
\usepackage{graphicx}       % Insert images
\usepackage{float}          % Position images
% \usepackage{subfigure}      % Subfigures
\graphicspath{{imgs}}       % Image location
\usepackage{subcaption}     % Subfigures II
\usepackage{verbatim}

% Figures
\usepackage{tikz}
\usetikzlibrary{arrows.meta,positioning,trees}

% Colors
\usepackage{xcolor}     
\definecolor{popUp}{HTML}{666666}
\definecolor{popUpIn}{HTML}{CED9E0}
\definecolor{backgroundC}{HTML}{D0E8F2}
\definecolor{backgroundCC}{HTML}{FFFFFF}
\definecolor{borders}{HTML}{8c120d}
\definecolor{padding}{HTML}{77D0D7}
\definecolor{links}{HTML}{CC6F5F}

% ----------------------------------------------------------
% FRAMES

% General settings
\usepackage{tcolorbox}
\usepackage{adjustbox}          % Adjusted frame  
\setlength{\fboxrule}{3pt}  % Line width
\setlength{\fboxsep}{3pt}   % Box padding

% General frames
\usepackage{mdframed}   

\mdfdefinestyle{estiloGeneral}{    % General style
	linecolor=black,
	linewidth=1.5pt,
	roundcorner=10pt,
	backgroundcolor=backgroundC,
	innerbottommargin=0pt
}
\mdfdefinestyle{code}{          % Code style
	linecolor=black,
	linewidth=1.5pt,
	roundcorner=10pt,
	backgroundcolor=darkgray!10,
	innerbottommargin=0pt
}

% Image frame
\newtcbox{\fboxC}{
	colback=backgroundC,
	colframe=popUp,
	arc=10pt,
	boxrule=3pt,
	boxsep=0pt, % Change the padding here
	nobeforeafter
}

% ----------------------------------------------------------
% PAGE SETTINGS

% Background 
\newcommand{\background}[0]{\begin{tikzpicture}[remember picture,overlay]
		\fill[backgroundC] (-2,2) rectangle (25cm, -550);
\end{tikzpicture}}

\newcommand{\backgroundC}[0]{\begin{tikzpicture}[remember picture,overlay]
		\fill[backgroundCC] (-2,2) rectangle (25cm, -550);
\end{tikzpicture}}

% Page width 
\newcommand{\anchoPag}[0]{20cm}

% ----------------------------------------------------------
% FONT

% General
\usepackage{tgbonum}        % Font
\usepackage{listings}       % Code typesetting
\usepackage[spanish]{babel} % Load Spanish
\selectlanguage{spanish}    % Select Spanish
\usepackage{enumitem}
\usepackage{bookmark}

\setlist[itemize]{leftmargin=1.2em, itemsep=0.35em, topsep=0.35em}

% --- Table helpers ---
\newcolumntype{L}[1]{>{\raggedright\arraybackslash}p{#1}}
\newcolumntype{Y}{>{\raggedright\arraybackslash}X}
\newcolumntype{C}{>{\centering\arraybackslash}X}
\renewcommand{\arraystretch}{1.1}

% Python style
\lstdefinestyle{python}{
	language=Python,
	basicstyle=\ttfamily\small,
	commentstyle=\color{green!50!black},
	keywordstyle=\color{blue},
	numberstyle=\tiny\color{gray},
	numbers=left,
	morekeywords={>, <},
	breakatwhitespace=false,
	showstringspaces=false,
	showtabs=false,
	showspaces=false
}

% ----------------------------------------------------------
% HYPERLINKS

% General
\usepackage{hyperref}       
\hypersetup{
	colorlinks=true,
	linkcolor=links,
	filecolor=magenta,      
	urlcolor=brown,
}

% Custom commands 

% Large link
\newcommand{\bigLink}[2]{\begin{center} \fboxC{\LARGE{\href{#1}{#2}}}\end{center}}

% Small link
\newcommand{\smallLink}[2]{\begin{center}\fboxC{\href{#1}{#2}}\end{center}}

% Bold link
\newcommand{\bfLink}[2]{\href{#1}{\textbf{#2}}}


% Small URL
\newcommand{\smallUrl}[1]{\begin{center}\fboxC{\url{#1}}\end{center}}


% ----------------------------------------------------------
% CUSTOM COMMANDS FOR FIGURES

\newcommand{\espacioImagenes}[0]{-1.2cm}

% Without frame
\newcommand{\fig}[3][\espacioImagenes]{
	\hspace*{#1}
	\centering
	\includegraphics[width=#2\textwidth]{#3}
}

% With frame
\newcommand{\ffig}[2]{\begin{figure}[h]
		\hspace*{\espacioImagenes}
		\centering
		\fbox{\includegraphics[width=#1\textwidth]{#2}}
\end{figure}}

% Hyperlink with frame
\newcommand{\hfig}[3]{\begin{figure}[h]
		\hspace*{-1.4cm}
		\centering
		\color{popUp}
		\fboxC{\href{#1}{\includegraphics[width=#2\textwidth]{#3}}}
	\end{figure}
}

% Hyperlink without frame
\newcommand{\hffig}[3]{\begin{figure}[h]
		\hspace*{-1.1cm}
		\centering
		\color{popUp}
		\href{#1}{\includegraphics[width=#2\textwidth]{#3}}
	\end{figure}
}

% ----------------------------------------------------------

% Start and Contents
\newcommand{\cuadro}[1]{
	\begin{mdframed}[style=estiloGeneral]
		#1 
	\end{mdframed}
}

% Explanation video image
\newcommand{\linkExplicacion}[1]{
	\hffig{#1}{0.5}{principal/videoExplicacion}
	\vspace{-0.5cm}
}

\newcommand{\subSecLink}[2]{
	\subsubsection*{\href{#1}{\textbf{#2}}}
}

% Spacing
\newcommand{\esp}[0]{\vspace{4mm}}

% Back to start
\newcommand{\secInicio}[0]{\begin{center}\hyperref[sec:inicio]{ 
			\includegraphics[width=0.1\textwidth]{principal/up}
	}\end{center}
}


\geometry{margin=0.85in}
\AtBeginDocument{\small}

\newcommand{\ExamNameField}{\noindent\textbf{Name:}\ \rule{0.7\linewidth}{0.4pt}\par\medskip}

\newcommand{\ExamTitleBlock}[3]{%
	\begin{center}
		\Large\textbf{#1}\\
		\textbf{#2}%
		\if\relax\detokenize{#3}\relax\else\\\textbf{#3}\fi
	\end{center}
	\vspace{0.5em}
}

\newcommand{\ExamSection}[1]{\par\medskip\textbf{#1}\par\smallskip}

\newenvironment{ExamCriteria}{%
	\begin{itemize}[leftmargin=1.6em, itemsep=0.3em, topsep=0.2em]
}{%
	\end{itemize}
}

\newenvironment{ExamProblems}{%
	\begin{enumerate}[label=\textbf{P\arabic*}, leftmargin=0pt, labelsep=0.6em, itemindent=2.2em, itemsep=0.8em]
}{%
	\end{enumerate}
}


\begin{document}
\ExamTitleBlock{10th grade}{Term 3 Practice Activity: C6 Standardisation in Context}{}

\ExamSection{C6 Employs the standardisation of the normal variable in real-life problems.}

\begin{ExamProblems}

\item
\subsection*{Problem 1 — Exam Performance Benchmark}

\textbf{Problem.}
Mathematics test marks at a school are modeled by a normal variable
\(X\sim N(70,8^2)\), where marks are out of 100.

\textbf{Question.}
Find the probability that a randomly selected student scores above 82 marks.
Show the standardisation step clearly.

\textbf{Solution.}
Use
\[
Z=\frac{X-\mu}{\sigma}=\frac{X-70}{8}.
\]
Then
\[
P(X>82)=P\!\left(Z>\frac{82-70}{8}\right)=P(Z>1.50).
\]
From the standard normal table, \(\Phi(1.50)=0.9332\).
So
\[
P(Z>1.50)=1-\Phi(1.50)=1-0.9332=0.0668.
\]
\textbf{Interpretation.}
About 6.68\% of students are expected to score above 82.

% --------------------------------------------------

\item
\subsection*{Problem 2 — Delivery Time Reliability}

\textbf{Problem.}
Delivery times for online orders are normally distributed as
\(X\sim N(56,5^2)\), where \(X\) is measured in minutes.

\textbf{Question.}
Find the probability that an order is delivered in less than 50 minutes.
Show the standardisation step clearly.

\textbf{Solution.}
Standardise with
\[
Z=\frac{X-56}{5}.
\]
Then
\[
P(X<50)=P\!\left(Z<\frac{50-56}{5}\right)=P(Z<-1.20).
\]
Using symmetry with the table value \(\Phi(1.20)=0.8849\):
\[
\Phi(-1.20)=1-\Phi(1.20)=1-0.8849=0.1151.
\]
Hence
\[
P(X<50)=0.1151.
\]
\textbf{Interpretation.}
There is an 11.51\% chance a delivery takes less than 50 minutes.

% --------------------------------------------------

\item
\subsection*{Problem 3 — Apple Mass Acceptance Band}

\textbf{Problem.}
The mass of apples in a packing center follows
\(X\sim N(182,8^2)\), with \(X\) in grams.

\textbf{Question.}
Find the probability that an apple has mass between 170 g and 194 g.
Show the standardisation step clearly.

\textbf{Solution.}
Use
\[
Z=\frac{X-182}{8}.
\]
For the bounds:
\[
z_1=\frac{170-182}{8}=-1.50,
\qquad
z_2=\frac{194-182}{8}=1.50.
\]
So
\[
P(170\le X\le 194)=P(-1.50\le Z\le 1.50)=\Phi(1.50)-\Phi(-1.50).
\]
From the table, \(\Phi(1.50)=0.9332\), and
\[
\Phi(-1.50)=1-\Phi(1.50)=1-0.9332=0.0668.
\]
Therefore
\[
P(170\le X\le 194)=0.9332-0.0668=0.8664.
\]
\textbf{Interpretation.}
About 86.64\% of apples are expected to be between 170 g and 194 g.

% --------------------------------------------------

\item
\subsection*{Problem 4 — Commute Duration Window}

\textbf{Problem.}
Students' commute times are modeled by
\(X\sim N(30,4^2)\), where \(X\) is in minutes.

\textbf{Question.}
Find the probability that a student commute lasts from 24 to 36 minutes.
Show the standardisation step clearly.

\textbf{Solution.}
Standardise:
\[
Z=\frac{X-30}{4}.
\]
Then
\[
z_1=\frac{24-30}{4}=-1.50,
\qquad
z_2=\frac{36-30}{4}=1.50.
\]
Hence
\[
P(24\le X\le 36)=P(-1.50\le Z\le 1.50)=\Phi(1.50)-\Phi(-1.50).
\]
Using table values, \(\Phi(1.50)=0.9332\) and
\(\Phi(-1.50)=1-0.9332=0.0668\):
\[
P(24\le X\le 36)=0.9332-0.0668=0.8664.
\]
\textbf{Interpretation.}
There is an 86.64\% chance a commute is between 24 and 36 minutes.

% --------------------------------------------------

\item
\subsection*{Problem 5 — Battery Life Threshold}

\textbf{Problem.}
A phone battery life test result is modeled by
\(X\sim N(10,0.8^2)\), where \(X\) is in hours of use.

\textbf{Question.}
Find the probability that a battery lasts longer than 11.2 hours.
Show the standardisation step clearly.

\textbf{Solution.}
Use
\[
Z=\frac{X-10}{0.8}.
\]
Then
\[
P(X>11.2)=P\!\left(Z>\frac{11.2-10}{0.8}\right)=P(Z>1.50).
\]
From the table, \(\Phi(1.50)=0.9332\). So
\[
P(Z>1.50)=1-0.9332=0.0668.
\]
Therefore
\[
P(X>11.2)=0.0668.
\]
\textbf{Interpretation.}
Only about 6.68\% of batteries are expected to exceed 11.2 hours.

% --------------------------------------------------

\item
\subsection*{Problem 6 — Streaming Download Speed Band}

\textbf{Problem.}
Home streaming download speed is modeled by
\(X\sim N(44,5^2)\), with \(X\) measured in Mbps.

\textbf{Question.}
Find the probability that the speed is between 38 Mbps and 50 Mbps.
Show the standardisation step clearly.

\textbf{Solution.}
Standardise:
\[
Z=\frac{X-44}{5}.
\]
Endpoint z-scores:
\[
z_1=\frac{38-44}{5}=-1.20,
\qquad
z_2=\frac{50-44}{5}=1.20.
\]
So
\[
P(38\le X\le 50)=P(-1.20\le Z\le 1.20)=\Phi(1.20)-\Phi(-1.20).
\]
From the table, \(\Phi(1.20)=0.8849\), and
\[
\Phi(-1.20)=1-\Phi(1.20)=1-0.8849=0.1151.
\]
Thus
\[
P(38\le X\le 50)=0.8849-0.1151=0.7698.
\]
\textbf{Interpretation.}
About 76.98\% of download-speed readings lie between 38 Mbps and 50 Mbps.

% --------------------------------------------------

\item
\subsection*{Problem 7 — Coffee Shop Queue Extremes}

\textbf{Problem.}
Queue waiting time at a coffee shop is modeled by
\(X\sim N(8,2.5^2)\), where \(X\) is in minutes.

\textbf{Question.}
Find the probability that a customer waits at most 4 minutes or at least 12 minutes.
Show the standardisation step clearly.

\textbf{Solution.}
Use
\[
Z=\frac{X-8}{2.5}.
\]
Compute z-scores:
\[
z_a=\frac{4-8}{2.5}=-1.60,
\qquad
z_b=\frac{12-8}{2.5}=1.60.
\]
Then
\[
P(X\le 4\ \text{or}\ X\ge 12)=P(Z\le -1.60)+P(Z\ge 1.60).
\]
From the table, \(\Phi(1.60)=0.9452\), so
\[
P(Z\le -1.60)=\Phi(-1.60)=1-0.9452=0.0548,
\]
\[
P(Z\ge 1.60)=1-\Phi(1.60)=1-0.9452=0.0548.
\]
Therefore
\[
P(X\le 4\ \text{or}\ X\ge 12)=0.0548+0.0548=0.1096.
\]
\textbf{Interpretation.}
There is a 10.96\% chance of a very short or very long queue wait.

% --------------------------------------------------

\item
\subsection*{Problem 8 — Daily Screen-On Time}

\textbf{Problem.}
A student's daily phone screen-on time is modeled by
\(X\sim N(3.6,1.0^2)\), where \(X\) is in hours.

\textbf{Question.}
Find the probability that screen-on time is between 2.4 h and 4.8 h.
Show the standardisation step clearly.

\textbf{Solution.}
Standardise:
\[
Z=\frac{X-3.6}{1.0}.
\]
Then
\[
z_1=\frac{2.4-3.6}{1.0}=-1.20,
\qquad
z_2=\frac{4.8-3.6}{1.0}=1.20.
\]
So
\[
P(2.4\le X\le 4.8)=P(-1.20\le Z\le 1.20)=\Phi(1.20)-\Phi(-1.20).
\]
Using \(\Phi(1.20)=0.8849\) and \(\Phi(-1.20)=1-0.8849=0.1151\):
\[
P(2.4\le X\le 4.8)=0.8849-0.1151=0.7698.
\]
\textbf{Interpretation.}
The model predicts a 76.98\% chance that daily screen time is between 2.4 and 4.8 hours.

% --------------------------------------------------

\item
\subsection*{Problem 9 — Daily Step Count Extremes}

\textbf{Problem.}
A teen's daily step count is modeled by
\(X\sim N(9000,1250^2)\), where \(X\) is in steps.

\textbf{Question.}
Find the probability that the step count is at most 6800 steps or at least 11\,200 steps.
Show the standardisation step clearly.

\textbf{Solution.}
Use
\[
Z=\frac{X-9000}{1250}.
\]
Compute:
\[
z_a=\frac{6800-9000}{1250}=-1.76,
\qquad
z_b=\frac{11200-9000}{1250}=1.76.
\]
Therefore
\[
P(X\le 6800\ \text{or}\ X\ge 11200)=P(Z\le -1.76)+P(Z\ge 1.76).
\]
From the table, \(\Phi(1.76)=0.9608\). Hence
\[
P(Z\le -1.76)=\Phi(-1.76)=1-0.9608=0.0392,
\]
\[
P(Z\ge 1.76)=1-\Phi(1.76)=1-0.9608=0.0392.
\]
So
\[
P(X\le 6800\ \text{or}\ X\ge 11200)=0.0392+0.0392=0.0784.
\]
\textbf{Interpretation.}
There is a 7.84\% chance of a very low or very high daily step count.

% --------------------------------------------------

\item
\subsection*{Problem 10 — Classroom Temperature Comfort Range}

\textbf{Problem.}
Hourly classroom temperature readings are modeled by
\(X\sim N(22.0,1.25^2)\), where \(X\) is in \(^\circ\)C.

\textbf{Question.}
Find the probability that a reading is between 20.5\(^\circ\)C and 24.0\(^\circ\)C.
Show the standardisation step clearly.

\textbf{Solution.}
Standardise with
\[
Z=\frac{X-22.0}{1.25}.
\]
Compute z-scores:
\[
z_1=\frac{20.5-22.0}{1.25}=-1.20,
\qquad
z_2=\frac{24.0-22.0}{1.25}=1.60.
\]
So
\[
P(20.5\le X\le 24.0)=P(-1.20\le Z\le 1.60)=\Phi(1.60)-\Phi(-1.20).
\]
From the table, \(\Phi(1.60)=0.9452\) and \(\Phi(-1.20)=1-\Phi(1.20)=1-0.8849=0.1151\).
Thus
\[
P(20.5\le X\le 24.0)=0.9452-0.1151=0.8301.
\]
\textbf{Interpretation.}
The probability that the room temperature stays in this comfort range is 83.01\%.

\end{ExamProblems}

\end{document}
