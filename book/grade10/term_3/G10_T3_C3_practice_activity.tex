\makeatletter
\def\input@path{{./}{../}{../../}{preamble/}{../preamble/}{../../preamble/}}
\makeatother
% ----------------------------------------------------------
% GENERAL 

% File
\documentclass[11pt]{book}

% Margins
\usepackage[margin=1in]{geometry}

\linespread{1.2}            % Line spacing
\usepackage[utf8]{inputenc}
\usepackage[T1]{fontenc}
\usepackage{lmodern}
\usepackage{microtype}
\setlength{\parindent}{0pt}
\setlength{\parskip}{6pt}
\usepackage{booktabs}

% ----------------------------------------------------------
% TABLES
\usepackage{multicol}
\usepackage{longtable} 
\usepackage{array}
\usepackage{booktabs}
\usepackage{tabularx}
\usepackage{multirow}

% ----------------------------------------------------------
% MATHEMATICS
\usepackage{amsmath}
\usepackage{amssymb}
\usepackage{amsfonts}
\usepackage{mathtools}

% ----------------------------------------------------------
% HIDDEN CONTENT
\usepackage{ifthen}
% Define a boolean switch
\newboolean{explicaciones}
% Set the boolean switch to true or false
% Change to true to show the content

% Explanations
\newcommand{\explicacion}[2]{
	\ifthenelse{\boolean{explicaciones}}{#1}{#2}
}
\newcommand{\mostrarExplicaciones}[1]{\setboolean{explicaciones}{#1}}

% ----------------------------------------------------------
% NUMBERING

\usepackage{fancyhdr}
\pagestyle{empty} % Ensures the entire document has no page numbers

\usepackage{tocloft}
\renewcommand{\cftdot}{} % Remove dots for sections, if any
\renewcommand{\cftsecleader}{\cftdotfill{\cftdotsep}} % Remove dots for sections, if any
\cftpagenumbersoff{section} % Removes page numbers from sections
\cftpagenumbersoff{subsection} % Removes page numbers from subsections

% ----------------------------------------------------------
% IMAGES 

% General settings
\usepackage{graphicx}       % Insert images
\usepackage{float}          % Position images
% \usepackage{subfigure}      % Subfigures
\graphicspath{{imgs}}       % Image location
\usepackage{subcaption}     % Subfigures II
\usepackage{verbatim}

% Figures
\usepackage{tikz}
\usetikzlibrary{arrows.meta,positioning,trees}

% Colors
\usepackage{xcolor}     
\definecolor{popUp}{HTML}{666666}
\definecolor{popUpIn}{HTML}{CED9E0}
\definecolor{backgroundC}{HTML}{D0E8F2}
\definecolor{backgroundCC}{HTML}{FFFFFF}
\definecolor{borders}{HTML}{8c120d}
\definecolor{padding}{HTML}{77D0D7}
\definecolor{links}{HTML}{CC6F5F}

% ----------------------------------------------------------
% FRAMES

% General settings
\usepackage{tcolorbox}
\usepackage{adjustbox}          % Adjusted frame  
\setlength{\fboxrule}{3pt}  % Line width
\setlength{\fboxsep}{3pt}   % Box padding

% General frames
\usepackage{mdframed}   

\mdfdefinestyle{estiloGeneral}{    % General style
	linecolor=black,
	linewidth=1.5pt,
	roundcorner=10pt,
	backgroundcolor=backgroundC,
	innerbottommargin=0pt
}
\mdfdefinestyle{code}{          % Code style
	linecolor=black,
	linewidth=1.5pt,
	roundcorner=10pt,
	backgroundcolor=darkgray!10,
	innerbottommargin=0pt
}

% Image frame
\newtcbox{\fboxC}{
	colback=backgroundC,
	colframe=popUp,
	arc=10pt,
	boxrule=3pt,
	boxsep=0pt, % Change the padding here
	nobeforeafter
}

% ----------------------------------------------------------
% PAGE SETTINGS

% Background 
\newcommand{\background}[0]{\begin{tikzpicture}[remember picture,overlay]
		\fill[backgroundC] (-2,2) rectangle (25cm, -550);
\end{tikzpicture}}

\newcommand{\backgroundC}[0]{\begin{tikzpicture}[remember picture,overlay]
		\fill[backgroundCC] (-2,2) rectangle (25cm, -550);
\end{tikzpicture}}

% Page width 
\newcommand{\anchoPag}[0]{20cm}

% ----------------------------------------------------------
% FONT

% General
\usepackage{tgbonum}        % Font
\usepackage{listings}       % Code typesetting
\usepackage[spanish]{babel} % Load Spanish
\selectlanguage{spanish}    % Select Spanish
\usepackage{enumitem}
\usepackage{bookmark}

\setlist[itemize]{leftmargin=1.2em, itemsep=0.35em, topsep=0.35em}

% --- Table helpers ---
\newcolumntype{L}[1]{>{\raggedright\arraybackslash}p{#1}}
\newcolumntype{Y}{>{\raggedright\arraybackslash}X}
\newcolumntype{C}{>{\centering\arraybackslash}X}
\renewcommand{\arraystretch}{1.1}

% Python style
\lstdefinestyle{python}{
	language=Python,
	basicstyle=\ttfamily\small,
	commentstyle=\color{green!50!black},
	keywordstyle=\color{blue},
	numberstyle=\tiny\color{gray},
	numbers=left,
	morekeywords={>, <},
	breakatwhitespace=false,
	showstringspaces=false,
	showtabs=false,
	showspaces=false
}

% ----------------------------------------------------------
% HYPERLINKS

% General
\usepackage{hyperref}       
\hypersetup{
	colorlinks=true,
	linkcolor=links,
	filecolor=magenta,      
	urlcolor=brown,
}

% Custom commands 

% Large link
\newcommand{\bigLink}[2]{\begin{center} \fboxC{\LARGE{\href{#1}{#2}}}\end{center}}

% Small link
\newcommand{\smallLink}[2]{\begin{center}\fboxC{\href{#1}{#2}}\end{center}}

% Bold link
\newcommand{\bfLink}[2]{\href{#1}{\textbf{#2}}}


% Small URL
\newcommand{\smallUrl}[1]{\begin{center}\fboxC{\url{#1}}\end{center}}


% ----------------------------------------------------------
% CUSTOM COMMANDS FOR FIGURES

\newcommand{\espacioImagenes}[0]{-1.2cm}

% Without frame
\newcommand{\fig}[3][\espacioImagenes]{
	\hspace*{#1}
	\centering
	\includegraphics[width=#2\textwidth]{#3}
}

% With frame
\newcommand{\ffig}[2]{\begin{figure}[h]
		\hspace*{\espacioImagenes}
		\centering
		\fbox{\includegraphics[width=#1\textwidth]{#2}}
\end{figure}}

% Hyperlink with frame
\newcommand{\hfig}[3]{\begin{figure}[h]
		\hspace*{-1.4cm}
		\centering
		\color{popUp}
		\fboxC{\href{#1}{\includegraphics[width=#2\textwidth]{#3}}}
	\end{figure}
}

% Hyperlink without frame
\newcommand{\hffig}[3]{\begin{figure}[h]
		\hspace*{-1.1cm}
		\centering
		\color{popUp}
		\href{#1}{\includegraphics[width=#2\textwidth]{#3}}
	\end{figure}
}

% ----------------------------------------------------------

% Start and Contents
\newcommand{\cuadro}[1]{
	\begin{mdframed}[style=estiloGeneral]
		#1 
	\end{mdframed}
}

% Explanation video image
\newcommand{\linkExplicacion}[1]{
	\hffig{#1}{0.5}{principal/videoExplicacion}
	\vspace{-0.5cm}
}

\newcommand{\subSecLink}[2]{
	\subsubsection*{\href{#1}{\textbf{#2}}}
}

% Spacing
\newcommand{\esp}[0]{\vspace{4mm}}

% Back to start
\newcommand{\secInicio}[0]{\begin{center}\hyperref[sec:inicio]{ 
			\includegraphics[width=0.1\textwidth]{principal/up}
	}\end{center}
}


\geometry{margin=0.85in}
\AtBeginDocument{\small}

\newcommand{\ExamNameField}{\noindent\textbf{Name:}\ \rule{0.7\linewidth}{0.4pt}\par\medskip}

\newcommand{\ExamTitleBlock}[3]{%
	\begin{center}
		\Large\textbf{#1}\\
		\textbf{#2}%
		\if\relax\detokenize{#3}\relax\else\\\textbf{#3}\fi
	\end{center}
	\vspace{0.5em}
}

\newcommand{\ExamSection}[1]{\par\medskip\textbf{#1}\par\smallskip}

\newenvironment{ExamCriteria}{%
	\begin{itemize}[leftmargin=1.6em, itemsep=0.3em, topsep=0.2em]
}{%
	\end{itemize}
}

\newenvironment{ExamProblems}{%
	\begin{enumerate}[label=\textbf{P\arabic*}, leftmargin=0pt, labelsep=0.6em, itemindent=2.2em, itemsep=0.8em]
}{%
	\end{enumerate}
}


\begin{document}
\ExamTitleBlock{10th grade}{Term 3 Practice Activity: C3 Linear Densities (No integration)}{}
\ExamSection{C3 Expresses how to work with distributions that have a linear behaviour.}

\begin{ExamProblems}

\item
\subsection*{Problem 1 — Phone Battery Life Ramp-Up}

\textbf{Problem.}
A phone repair shop models battery lifetime $X$ (hours) on $[0,8]$ with a linear density
\[
f(x)=a+bx,\quad 0\le x\le 8,
\]
and $f(0)=0$.

\textbf{Question.}
(a) Determine $a$ and $b$ and write $f(x)$ explicitly. (b) Compute $P(X\le 4)$. (c) Interpret the result.

\textbf{Solution.}
Because $f(0)=0$, we get $a=0$, so the graph is an increasing triangle: $f(x)=bx$ on $[0,8]$.
The full triangle has base $8$ and height $f(8)=8b$, so
\[
\frac{1}{2}(8)(8b)=1\Rightarrow 32b=1\Rightarrow b=\frac{1}{32}.
\]
Hence
\[
f(x)=\frac{x}{32},\quad 0\le x\le 8.
\]
Now $P(X\le 4)$ is the area of the smaller triangle from $0$ to $4$.
Its height is $f(4)=\frac{4}{32}=\frac18$, so
\[
P(X\le 4)=\frac12(4)\left(\frac18\right)=\frac14=0.25.
\]

\textbf{Interpretation.}
There is a $\frac14$ chance (25\%) that a battery lasts at most 4 hours.

% --------------------------------------------------

\item
\subsection*{Problem 2 — Delivery Weight Taper}

\textbf{Problem.}
A courier company models parcel weight $W$ (kg) on $[0,6]$ by
\[
f(w)=a-bw,\quad 0\le w\le 6,
\]
with $f(6)=0$.

\textbf{Question.}
(a) Find $a$ and $b$. (b) Compute $P(2\le W\le 5)$. (c) Interpret the result.

\textbf{Solution.}
Since $f(6)=0$, $a-6b=0\Rightarrow a=6b$.
The graph is a decreasing triangle with base $6$ and height $a$, so
\[
\frac12(6)(a)=1\Rightarrow 3a=1\Rightarrow a=\frac13.
\]
Then $b=\frac{a}{6}=\frac{1}{18}$, so
\[
f(w)=\frac13-\frac{w}{18}.
\]
For $P(2\le W\le 5)$, use trapezoid area on $[2,5]$.
Heights: $f(2)=\frac{2}{9}$ and $f(5)=\frac{1}{18}$, width $=3$:
\[
P(2\le W\le 5)=\frac{\left(\frac{2}{9}+\frac{1}{18}\right)}{2}\cdot 3
=\frac{\frac{5}{18}}{2}\cdot 3=\frac{5}{12}\approx 0.417.
\]

\textbf{Interpretation.}
About $\frac{5}{12}$ of parcels are expected to weigh between 2 kg and 5 kg.

% --------------------------------------------------

\item
\subsection*{Problem 3 — Sensor Drift Build-Up}

\textbf{Problem.}
A lab models sensor drift $D$ (units) on $[0,10]$ with
\[
f(d)=kd,\quad 0\le d\le 10.
\]

\textbf{Question.}
(a) Determine $k$. (b) Compute $P(6\le D\le 10)$. (c) Interpret the probability.

\textbf{Solution.}
The graph is an increasing triangle from $(0,0)$ to $(10,10k)$.
Total area is
\[
\frac12(10)(10k)=50k=1\Rightarrow k=\frac{1}{50}.
\]
So $f(d)=\frac{d}{50}$.
Now $P(6\le D\le 10)$ is a trapezoid with width $4$ and heights
\[
f(6)=\frac{6}{50}=\frac{3}{25},\qquad f(10)=\frac{10}{50}=\frac15.
\]
Therefore
\[
P(6\le D\le 10)=\frac{\left(\frac{3}{25}+\frac15\right)}{2}\cdot 4
=\frac{\frac{8}{25}}{2}\cdot 4=\frac{16}{25}=0.64.
\]

\textbf{Interpretation.}
There is a 64\% chance the drift is between 6 and 10 units, so larger drift values are more common.

% --------------------------------------------------

\item
\subsection*{Problem 4 — Charging Time Drop-Off}

\textbf{Problem.}
Charging time $T$ (minutes) for quick top-ups is modeled on $[0,5]$ by
\[
f(t)=a-bt,\quad 0\le t\le 5,
\]
with $f(5)=0$.

\textbf{Question.}
(a) Find $a$ and $b$. (b) Compute $P(T<2)$. (c) Interpret the result.

\textbf{Solution.}
From $f(5)=0$, $a-5b=0\Rightarrow a=5b$.
The graph is a decreasing triangle with base $5$ and height $a$.
Using area $=1$:
\[
\frac12(5)(a)=1\Rightarrow \frac{5a}{2}=1\Rightarrow a=\frac{2}{5}.
\]
Then $b=\frac{a}{5}=\frac{2}{25}$, hence
\[
f(t)=\frac25-\frac{2}{25}t.
\]
Now $P(T<2)$ is a trapezoid on $[0,2]$ with heights
\[
f(0)=\frac25,\qquad f(2)=\frac25-\frac{4}{25}=\frac{6}{25},
\]
width $2$:
\[
P(T<2)=\frac{\left(\frac25+\frac{6}{25}\right)}{2}\cdot 2
=\frac{16}{25}=0.64.
\]

\textbf{Interpretation.}
Short charging times (under 2 minutes in this model) occur with probability $\frac{16}{25}$.

% --------------------------------------------------

\item
\subsection*{Problem 5 — Commute Delay with Nonzero Ends}

\textbf{Problem.}
Commute delay $C$ (minutes) is modeled on $[0,10]$ by a linear density
\[
f(c)=a+bc,\quad 0\le c\le 10,
\]
with endpoint heights $f(0)=\frac{1}{20}$ and $f(10)=\frac{3}{20}$.

\textbf{Question.}
(a) Find $a$ and $b$. (b) Compute $P(2\le C\le 6)$. (c) Interpret the result.

\textbf{Solution.}
From $f(0)=\frac{1}{20}$, we get $a=\frac{1}{20}$.
Using $f(10)=\frac{3}{20}$:
\[
\frac{1}{20}+10b=\frac{3}{20}\Rightarrow 10b=\frac{2}{20}=\frac{1}{10}\Rightarrow b=\frac{1}{100}.
\]
So
\[
f(c)=\frac{1}{20}+\frac{c}{100}.
\]
The full graph is a trapezoid (nonzero endpoints) with heights $\frac{1}{20}$ and $\frac{3}{20}$, width $10$:
\[
\frac{\left(\frac{1}{20}+\frac{3}{20}\right)}{2}\cdot 10=1,
\]
so it is valid.
For $[2,6]$, heights are
\[
f(2)=\frac{7}{100},\qquad f(6)=\frac{11}{100},
\]
width $4$:
\[
P(2\le C\le 6)=\frac{\left(\frac{7}{100}+\frac{11}{100}\right)}{2}\cdot 4
=\frac{18}{100}\cdot 2=\frac{9}{25}=0.36.
\]

\textbf{Interpretation.}
A delay between 2 and 6 minutes has probability $\frac{9}{25}$.

% --------------------------------------------------

\item
\subsection*{Problem 6 — App Session Fade-Out}

\textbf{Problem.}
An app's session time $S$ (minutes) is modeled on $[0,12]$ by
\[
f(s)=k(12-s),\quad 0\le s\le 12.
\]

\textbf{Question.}
(a) Determine $k$. (b) Compare $P(0\le S\le 3)$ and $P(9\le S\le 12)$. (c) Interpret.

\textbf{Solution.}
This is a decreasing triangle ending at 0 when $s=12$.
Base is $12$, height is $f(0)=12k$:
\[
\frac12(12)(12k)=72k=1\Rightarrow k=\frac{1}{72}.
\]
So
\[
f(s)=\frac{12-s}{72}.
\]
For $[0,3]$, heights are $f(0)=\frac16$ and $f(3)=\frac18$, width $3$:
\[
P(0\le S\le 3)=\frac{\left(\frac16+\frac18\right)}{2}\cdot 3
=\frac{7}{16}.
\]
For $[9,12]$, heights are $f(9)=\frac{1}{24}$ and $f(12)=0$, width $3$:
\[
P(9\le S\le 12)=\frac{\left(\frac{1}{24}+0\right)}{2}\cdot 3
=\frac{1}{16}.
\]
Hence $P(0\le S\le 3)>P(9\le S\le 12)$.

\textbf{Interpretation.}
Very short sessions are much more likely than very long sessions in this model.

% --------------------------------------------------

\item
\subsection*{Problem 7 — Streaming Buffer Time with Baseline}

\textbf{Problem.}
A streaming app models buffer wait time $B$ (seconds) on $[0,8]$ with
\[
f(b)=a-b_1 b,\quad 0\le b\le 8,
\]
and endpoint conditions $f(0)=\frac14$, $f(8)=\frac{1}{20}$.

\textbf{Question.}
(a) Determine $a$ and $b_1$. (b) Find $P(2\le B\le 6)$. (c) Interpret the result.

\textbf{Solution.}
From $f(0)=\frac14$, $a=\frac14$.
Using $f(8)=\frac{1}{20}$:
\[
\frac14-8b_1=\frac{1}{20}\Rightarrow 8b_1=\frac14-\frac{1}{20}=\frac15
\Rightarrow b_1=\frac{1}{40}.
\]
Thus
\[
f(b)=\frac14-\frac{b}{40}.
\]
The full graph is a trapezoid with heights $\frac14$ and $\frac{1}{20}$ and width $8$:
\[
\frac{\left(\frac14+\frac{1}{20}\right)}{2}\cdot 8=1.
\]
For $[2,6]$, heights are
\[
f(2)=\frac15,\qquad f(6)=\frac{1}{10},
\]
width $4$:
\[
P(2\le B\le 6)=\frac{\left(\frac15+\frac{1}{10}\right)}{2}\cdot 4
=\frac{3}{20}\cdot 4=\frac{3}{5}=0.60.
\]

\textbf{Interpretation.}
There is a 60\% chance that buffering lasts between 2 and 6 seconds.

% --------------------------------------------------

\item
\subsection*{Problem 8 — Fridge Temperature Rise During Defrost}

\textbf{Problem.}
During a short defrost cycle, the temperature rise $Y$ ($^\circ$C) is modeled on $[0,6]$ by
\[
f(y)=ky,\quad 0\le y\le 6.
\]

\textbf{Question.}
(a) Find $k$ and write the density explicitly. (b) Compute $P(1\le Y\le 4)$. (c) Interpret.

\textbf{Solution.}
The graph is an increasing triangle from $(0,0)$ to $(6,6k)$.
Using total area $=1$:
\[
\frac12(6)(6k)=18k=1\Rightarrow k=\frac{1}{18}.
\]
So
\[
f(y)=\frac{y}{18}.
\]
For $P(1\le Y\le 4)$, use trapezoid area on $[1,4]$.
Heights are
\[
f(1)=\frac{1}{18},\qquad f(4)=\frac{2}{9},
\]
width $3$:
\[
P(1\le Y\le 4)=\frac{\left(\frac{1}{18}+\frac{2}{9}\right)}{2}\cdot 3
=\frac{\frac{5}{18}}{2}\cdot 3=\frac{5}{12}\approx 0.417.
\]

\textbf{Interpretation.}
There is a $\frac{5}{12}$ chance that the temperature rise is between $1^\circ$C and $4^\circ$C.

% --------------------------------------------------

% --------------------------------------------------

\item
\subsection*{Problem 9 — Packaging Fill Level with Two Linear Segments}

\textbf{Problem.}
A filling machine outputs fill level $F$ (litres) on $[0,8]$ with piecewise linear density
\[
f(x)=\begin{cases}
\dfrac{x}{16}, & 0\le x\le 4,\\
\dfrac{8-x}{16}, & 4\le x\le 8.
\end{cases}
\]

\textbf{Question.}
(a) Verify it is a valid density using area geometry. (b) Compute $P(2\le F\le 6)$. (c) Interpret.

\textbf{Solution.}
The graph is a triangle made from two linear segments, peaking at $x=4$ with height $f(4)=\frac14$.
Total area:
\[
\frac12(8)\left(\frac14\right)=1,
\]
so it is a valid density.
For $P(2\le F\le 6)$, use symmetry or subtract side triangles.
The full area is 1. The left excluded triangle on $[0,2]$ has base $2$, height $f(2)=\frac18$:
\[
A_L=\frac12(2)\left(\frac18\right)=\frac18.
\]
The right excluded triangle on $[6,8]$ has base $2$, height $f(6)=\frac18$:
\[
A_R=\frac12(2)\left(\frac18\right)=\frac18.
\]
Hence
\[
P(2\le F\le 6)=1-\frac18-\frac18=\frac34=0.75.
\]

\textbf{Interpretation.}
There is a 75\% chance the fill level is between 2 L and 6 L.

% --------------------------------------------------

\item
\subsection*{Problem 10 — Drone Delivery Delay with Uneven Peak}

\textbf{Problem.}
Drone delay $Z$ (minutes) is modeled on $[0,8]$ by
\[
f(z)=\begin{cases}
kz, & 0\le z\le 4,\\
\dfrac{k}{2}(8-z), & 4\le z\le 8.
\end{cases}
\]

\textbf{Question.}
(a) Determine $k$. (b) Find $P(2\le Z\le 6)$. (c) Interpret the probability.

\textbf{Solution.}
The graph is piecewise linear: rising from 0 to $4k$ at $z=4$, then falling to 0 at $z=8$ with a gentler slope.
Use area of two triangles:
\[
\text{Area on }[0,4]=\frac12(4)(4k)=8k,
\]
\[
\text{Area on }[4,8]=\frac12(4)(2k)=4k.
\]
Total area $8k+4k=12k=1$, so
\[
k=\frac{1}{12}.
\]
Now compute $P(2\le Z\le 6)$ by splitting the interval.
On $[2,4]$, heights are
\[
f(2)=\frac{1}{6},\qquad f(4)=\frac{1}{3},
\]
width $2$, so
\[
A_1=\frac{\left(\frac16+\frac13\right)}{2}\cdot 2=\frac12.
\]
On $[4,6]$, heights are
\[
f(4)=\frac{1}{6},\qquad f(6)=\frac{1}{12},
\]
width $2$, so
\[
A_2=\frac{\left(\frac16+\frac{1}{12}\right)}{2}\cdot 2=\frac14.
\]
Therefore
\[
P(2\le Z\le 6)=A_1+A_2=\frac34=0.75.
\]

\textbf{Interpretation.}
About 75\% of deliveries have delays between 2 and 6 minutes.

\end{ExamProblems}
\end{document}
