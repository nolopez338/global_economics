\makeatletter
\def\input@path{{./}{../}{../../}{preamble/}{../preamble/}{../../preamble/}}
\makeatother
% ----------------------------------------------------------
% GENERAL 

% File
\documentclass[11pt]{book}

% Margins
\usepackage[margin=1in]{geometry}

\linespread{1.2}            % Line spacing
\usepackage[utf8]{inputenc}
\usepackage[T1]{fontenc}
\usepackage{lmodern}
\usepackage{microtype}
\setlength{\parindent}{0pt}
\setlength{\parskip}{6pt}
\usepackage{booktabs}

% ----------------------------------------------------------
% TABLES
\usepackage{multicol}
\usepackage{longtable} 
\usepackage{array}
\usepackage{booktabs}
\usepackage{tabularx}
\usepackage{multirow}

% ----------------------------------------------------------
% MATHEMATICS
\usepackage{amsmath}
\usepackage{amssymb}
\usepackage{amsfonts}
\usepackage{mathtools}

% ----------------------------------------------------------
% HIDDEN CONTENT
\usepackage{ifthen}
% Define a boolean switch
\newboolean{explicaciones}
% Set the boolean switch to true or false
% Change to true to show the content

% Explanations
\newcommand{\explicacion}[2]{
	\ifthenelse{\boolean{explicaciones}}{#1}{#2}
}
\newcommand{\mostrarExplicaciones}[1]{\setboolean{explicaciones}{#1}}

% ----------------------------------------------------------
% NUMBERING

\usepackage{fancyhdr}
\pagestyle{empty} % Ensures the entire document has no page numbers

\usepackage{tocloft}
\renewcommand{\cftdot}{} % Remove dots for sections, if any
\renewcommand{\cftsecleader}{\cftdotfill{\cftdotsep}} % Remove dots for sections, if any
\cftpagenumbersoff{section} % Removes page numbers from sections
\cftpagenumbersoff{subsection} % Removes page numbers from subsections

% ----------------------------------------------------------
% IMAGES 

% General settings
\usepackage{graphicx}       % Insert images
\usepackage{float}          % Position images
% \usepackage{subfigure}      % Subfigures
\graphicspath{{imgs}}       % Image location
\usepackage{subcaption}     % Subfigures II
\usepackage{verbatim}

% Figures
\usepackage{tikz}
\usetikzlibrary{arrows.meta,positioning,trees}

% Colors
\usepackage{xcolor}     
\definecolor{popUp}{HTML}{666666}
\definecolor{popUpIn}{HTML}{CED9E0}
\definecolor{backgroundC}{HTML}{D0E8F2}
\definecolor{backgroundCC}{HTML}{FFFFFF}
\definecolor{borders}{HTML}{8c120d}
\definecolor{padding}{HTML}{77D0D7}
\definecolor{links}{HTML}{CC6F5F}

% ----------------------------------------------------------
% FRAMES

% General settings
\usepackage{tcolorbox}
\usepackage{adjustbox}          % Adjusted frame  
\setlength{\fboxrule}{3pt}  % Line width
\setlength{\fboxsep}{3pt}   % Box padding

% General frames
\usepackage{mdframed}   

\mdfdefinestyle{estiloGeneral}{    % General style
	linecolor=black,
	linewidth=1.5pt,
	roundcorner=10pt,
	backgroundcolor=backgroundC,
	innerbottommargin=0pt
}
\mdfdefinestyle{code}{          % Code style
	linecolor=black,
	linewidth=1.5pt,
	roundcorner=10pt,
	backgroundcolor=darkgray!10,
	innerbottommargin=0pt
}

% Image frame
\newtcbox{\fboxC}{
	colback=backgroundC,
	colframe=popUp,
	arc=10pt,
	boxrule=3pt,
	boxsep=0pt, % Change the padding here
	nobeforeafter
}

% ----------------------------------------------------------
% PAGE SETTINGS

% Background 
\newcommand{\background}[0]{\begin{tikzpicture}[remember picture,overlay]
		\fill[backgroundC] (-2,2) rectangle (25cm, -550);
\end{tikzpicture}}

\newcommand{\backgroundC}[0]{\begin{tikzpicture}[remember picture,overlay]
		\fill[backgroundCC] (-2,2) rectangle (25cm, -550);
\end{tikzpicture}}

% Page width 
\newcommand{\anchoPag}[0]{20cm}

% ----------------------------------------------------------
% FONT

% General
\usepackage{tgbonum}        % Font
\usepackage{listings}       % Code typesetting
\usepackage[spanish]{babel} % Load Spanish
\selectlanguage{spanish}    % Select Spanish
\usepackage{enumitem}
\usepackage{bookmark}

\setlist[itemize]{leftmargin=1.2em, itemsep=0.35em, topsep=0.35em}

% --- Table helpers ---
\newcolumntype{L}[1]{>{\raggedright\arraybackslash}p{#1}}
\newcolumntype{Y}{>{\raggedright\arraybackslash}X}
\newcolumntype{C}{>{\centering\arraybackslash}X}
\renewcommand{\arraystretch}{1.1}

% Python style
\lstdefinestyle{python}{
	language=Python,
	basicstyle=\ttfamily\small,
	commentstyle=\color{green!50!black},
	keywordstyle=\color{blue},
	numberstyle=\tiny\color{gray},
	numbers=left,
	morekeywords={>, <},
	breakatwhitespace=false,
	showstringspaces=false,
	showtabs=false,
	showspaces=false
}

% ----------------------------------------------------------
% HYPERLINKS

% General
\usepackage{hyperref}       
\hypersetup{
	colorlinks=true,
	linkcolor=links,
	filecolor=magenta,      
	urlcolor=brown,
}

% Custom commands 

% Large link
\newcommand{\bigLink}[2]{\begin{center} \fboxC{\LARGE{\href{#1}{#2}}}\end{center}}

% Small link
\newcommand{\smallLink}[2]{\begin{center}\fboxC{\href{#1}{#2}}\end{center}}

% Bold link
\newcommand{\bfLink}[2]{\href{#1}{\textbf{#2}}}


% Small URL
\newcommand{\smallUrl}[1]{\begin{center}\fboxC{\url{#1}}\end{center}}


% ----------------------------------------------------------
% CUSTOM COMMANDS FOR FIGURES

\newcommand{\espacioImagenes}[0]{-1.2cm}

% Without frame
\newcommand{\fig}[3][\espacioImagenes]{
	\hspace*{#1}
	\centering
	\includegraphics[width=#2\textwidth]{#3}
}

% With frame
\newcommand{\ffig}[2]{\begin{figure}[h]
		\hspace*{\espacioImagenes}
		\centering
		\fbox{\includegraphics[width=#1\textwidth]{#2}}
\end{figure}}

% Hyperlink with frame
\newcommand{\hfig}[3]{\begin{figure}[h]
		\hspace*{-1.4cm}
		\centering
		\color{popUp}
		\fboxC{\href{#1}{\includegraphics[width=#2\textwidth]{#3}}}
	\end{figure}
}

% Hyperlink without frame
\newcommand{\hffig}[3]{\begin{figure}[h]
		\hspace*{-1.1cm}
		\centering
		\color{popUp}
		\href{#1}{\includegraphics[width=#2\textwidth]{#3}}
	\end{figure}
}

% ----------------------------------------------------------

% Start and Contents
\newcommand{\cuadro}[1]{
	\begin{mdframed}[style=estiloGeneral]
		#1 
	\end{mdframed}
}

% Explanation video image
\newcommand{\linkExplicacion}[1]{
	\hffig{#1}{0.5}{principal/videoExplicacion}
	\vspace{-0.5cm}
}

\newcommand{\subSecLink}[2]{
	\subsubsection*{\href{#1}{\textbf{#2}}}
}

% Spacing
\newcommand{\esp}[0]{\vspace{4mm}}

% Back to start
\newcommand{\secInicio}[0]{\begin{center}\hyperref[sec:inicio]{ 
			\includegraphics[width=0.1\textwidth]{principal/up}
	}\end{center}
}


\geometry{margin=0.85in}
\AtBeginDocument{\small}

\newcommand{\ExamNameField}{\noindent\textbf{Name:}\ \rule{0.7\linewidth}{0.4pt}\par\medskip}

\newcommand{\ExamTitleBlock}[3]{%
	\begin{center}
		\Large\textbf{#1}\\
		\textbf{#2}%
		\if\relax\detokenize{#3}\relax\else\\\textbf{#3}\fi
	\end{center}
	\vspace{0.5em}
}

\newcommand{\ExamSection}[1]{\par\medskip\textbf{#1}\par\smallskip}

\newenvironment{ExamCriteria}{%
	\begin{itemize}[leftmargin=1.6em, itemsep=0.3em, topsep=0.2em]
}{%
	\end{itemize}
}

\newenvironment{ExamProblems}{%
	\begin{enumerate}[label=\textbf{P\arabic*}, leftmargin=0pt, labelsep=0.6em, itemindent=2.2em, itemsep=0.8em]
}{%
	\end{enumerate}
}


\begin{document}
	\ExamTitleBlock{10th grade}{Term 3 Practice Activity: C1 Density Functions (No integration)}{}
	
	\ExamSection{C1 Describes probabilities in continuous variables as a density function.}
	
	\begin{ExamProblems}
		

		\item
		\subsection*{Problem 1 — App Notifications per Hour}
		
		\textbf{Problem.}
		Let $X$ be the number of app notifications a student receives in one hour. The probability mass function is
		\[
		\begin{array}{c|cccc}
		x & 0 & 1 & 2 & 3\\
		\hline
		P(X=x) & \frac{1}{10} & \frac{3}{10} & \frac{2}{5} & \frac{1}{5}
		\end{array}
		\]
		
		\textbf{Question.}
		(a) Verify this is a valid probability distribution.
		(b) Compute $P(X\le 2)$.
		(c) Interpret the probability in context.
		
		\textbf{Solution.}
		Check the total probability:
		\[
		\frac{1}{10}+\frac{3}{10}+\frac{2}{5}+\frac{1}{5}=\frac{1}{10}+\frac{3}{10}+\frac{4}{10}+\frac{2}{10}=1.
		\]
		So it is a valid probability distribution.
		\[
		P(X\le 2)=P(X=0)+P(X=1)+P(X=2)=\frac{1}{10}+\frac{3}{10}+\frac{2}{5}=\frac{4}{5}.
		\]
		\textbf{Interpretation.} There is a $\frac{4}{5}$ chance the student gets at most 2 notifications in an hour.
		
		% --------------------------------------------------
		
		\item
		\subsection*{Problem 2 — Defective Items in a Batch of 3}
		
		\textbf{Problem.}
		A quality check records $X$, the number of defective items in a batch of 3. Suppose
		\[
		\begin{array}{c|cccc}
		x & 0 & 1 & 2 & 3\\
		\hline
		P(X=x) & \frac{1}{2} & \frac{3}{10} & \frac{1}{10} & \frac{1}{10}
		\end{array}
		\]
		
		\textbf{Question.}
		(a) Verify this is a valid probability distribution.
		(b) Compute $P(1\le X\le 2)$.
		(c) Interpret the probability in context.
		
		\textbf{Solution.}
		First, add all probabilities:
		\[
		\frac{1}{2}+\frac{3}{10}+\frac{1}{10}+\frac{1}{10}=\frac{5}{10}+\frac{3}{10}+\frac{1}{10}+\frac{1}{10}=1.
		\]
		So the distribution is valid.
		\[
		P(1\le X\le 2)=P(X=1)+P(X=2)=\frac{3}{10}+\frac{1}{10}=\frac{2}{5}.
		\]
		\textbf{Interpretation.} The probability of getting 1 or 2 defective items in a batch is $\frac{2}{5}$.
		
		% --------------------------------------------------
		
		\item
		\subsection*{Problem 3 — Late Deliveries in One Day}
		
		\textbf{Problem.}
		Let $X$ be the number of late deliveries made by a shop in one day. The probability mass function is
		\[
		\begin{array}{c|ccccc}
		x & 0 & 1 & 2 & 3 & 4\\
		\hline
		P(X=x) & \frac{1}{4} & \frac{3}{10} & \frac{1}{5} & \frac{3}{20} & \frac{1}{10}
		\end{array}
		\]
		
		\textbf{Question.}
		(a) Verify this is a valid probability distribution.
		(b) Compute $P(X\ge 2)$.
		(c) Interpret the probability in context.
		
		\textbf{Solution.}
		Check the sum of probabilities:
		\[
		\frac{1}{4}+\frac{3}{10}+\frac{1}{5}+\frac{3}{20}+\frac{1}{10}
		=\frac{5}{20}+\frac{6}{20}+\frac{4}{20}+\frac{3}{20}+\frac{2}{20}=1.
		\]
		Hence it is valid.
		\[
		P(X\ge 2)=P(X=2)+P(X=3)+P(X=4)=\frac{1}{5}+\frac{3}{20}+\frac{1}{10}=\frac{9}{20}.
		\]
		\textbf{Interpretation.} There is a $\frac{9}{20}$ chance of having at least 2 late deliveries in a day.
		
		% --------------------------------------------------
		
		\item
		\subsection*{Problem 4 — Customers in a 10-Minute Window}
		
		\textbf{Problem.}
		A café tracks $X$, the number of customers arriving in a 10-minute window.
		\[
		\begin{array}{c|cccc}
		x & 1 & 2 & 3 & 4\\
		\hline
		P(X=x) & 0.2 & 0.35 & 0.3 & 0.15
		\end{array}
		\]
		
		\textbf{Question.}
		(a) Verify this is a valid probability distribution.
		(b) Compute $P(X\ne 4)$ using a complement.
		(c) Interpret the probability in context.
		
		\textbf{Solution.}
		Add the probabilities:
		\[
		0.2+0.35+0.3+0.15=1.
		\]
		So this is a valid probability distribution.
		\[
		P(X\ne 4)=1-P(X=4)=1-0.15=0.85.
		\]
		\textbf{Interpretation.} The chance that the café does not receive exactly 4 customers in that window is $0.85$.
		
		% --------------------------------------------------
		
		\item
		\subsection*{Problem 5 — Correct Answers on a Short Quiz}
		
		\textbf{Problem.}
		Let $X$ be the number of correct answers on a 4-question quiz.
		\[
		\begin{array}{c|ccccc}
		x & 0 & 1 & 2 & 3 & 4\\
		\hline
		P(X=x) & 0.05 & 0.15 & 0.4 & 0.25 & 0.15
		\end{array}
		\]
		
		\textbf{Question.}
		(a) Verify this is a valid probability distribution.
		(b) Compute $P(X\ge 3)$.
		(c) Interpret the probability in context.
		
		\textbf{Solution.}
		Sum all probabilities:
		\[
		0.05+0.15+0.4+0.25+0.15=1.
		\]
		So the distribution is valid.
		\[
		P(X\ge 3)=P(X=3)+P(X=4)=0.25+0.15=0.4.
		\]
		\textbf{Interpretation.} The probability of scoring at least 3 correct answers is $0.4$.
		
		% --------------------------------------------------
		\item
		\subsection*{Problem 6 — Water Tank Fill Level}
		
		\textbf{Problem.}
		A sensor records the fill level $X$ (in meters) of a water tank during a short period.
		Assume $X$ is continuous on $[0,5]$ with density
		\[
		f(x)=k,\quad 0\le x\le 5.
		\]
		
		\textbf{Question.} 
		(a) Determine $k$ so the model is a valid density.
		(b) Compute $P(1\le X\le 4)$.
		(c) Interpret the probability in context.
		
		\textbf{Solution.}
		The graph is a rectangle with width $5$ and height $k$, so total area is
		\[
		5k=1\Rightarrow k=\frac{1}{5}.
		\]
		Now $P(1\le X\le 4)$ is a rectangle of width $3$ and height $\frac{1}{5}$:
		\[
		P(1\le X\le 4)=3\cdot\frac{1}{5}=\frac{3}{5}=0.6.
		\]
		\textbf{Interpretation.} There is a $\frac{3}{5}$ chance that the fill level is between 1 m and 4 m.
		
		% --------------------------------------------------
		
		\item
		\subsection*{Problem 7 — Ramp Travel Time}
		
		\textbf{Problem.}
		The travel time $T$ (in seconds) for a robot to cross a ramp is modeled on $[2,8]$ by
		\[
		f(t)=k(t-2),\quad 2\le t\le 8.
		\]
		
		\textbf{Question.}
		(a) Determine $k$ so $f(t)$ is a valid density.
		(b) Compute $P(T>6)$.
		(c) Interpret the result.
		
		\textbf{Solution.}
		The graph from $2$ to $8$ is a triangle with base $6$ and height $f(8)=6k$.
		So total area is
		\[
		\frac{1}{2}(6)(6k)=18k=1\Rightarrow k=\frac{1}{18}.
		\]
		For $T>6$, use the trapezoid on $[6,8]$.
		Its left height is $f(6)=\frac{1}{18}(4)=\frac{2}{9}$, right height is $f(8)=\frac{1}{3}$, width is $2$:
		\[
		P(T>6)=\frac{\left(\frac{2}{9}+\frac{1}{3}\right)}{2}\cdot 2
		=\frac{2}{9}+\frac{1}{3}=\frac{5}{9}.
		\]
		\textbf{Interpretation.} The chance the robot takes more than 6 s is $\frac{5}{9}$.
		
		% --------------------------------------------------
		
		\item
		\subsection*{Problem 8 — Air Quality Index Block Model}
		
		\textbf{Problem.}
		An air-quality indicator $A$ is modeled on $[0,4]$ with piecewise-constant density
		\[
		f(a)=\begin{cases}
		k, & 0\le a<1,\\
		2k, & 1\le a<3,\\
		k, & 3\le a\le 4.
		\end{cases}
		\]
		
		\textbf{Question.}
		(a) Find $k$.
		(b) Compute $P(0.5\le A\le 2.5)$.
		(c) Interpret the probability.
		
		\textbf{Solution.}
		Add rectangle areas for total area 1:
		\[
		(1\cdot k)+(2\cdot 2k)+(1\cdot k)=6k=1\Rightarrow k=\frac{1}{6}.
		\]
		Now split $[0.5,2.5]$:
		from $0.5$ to $1$ area is $0.5\cdot\frac{1}{6}=\frac{1}{12}$,
		and from $1$ to $2.5$ area is $1.5\cdot\frac{1}{3}=\frac{1}{2}$.
		So
		\[
		P(0.5\le A\le 2.5)=\frac{1}{12}+\frac{1}{2}=\frac{7}{12}.
		\]
		\textbf{Interpretation.} The model gives a $\frac{7}{12}$ chance that $A$ lies between 0.5 and 2.5.
		
		% --------------------------------------------------
		
		\item
		\subsection*{Problem 9 — Symmetric Delivery Window}
		
		\textbf{Problem.}
		A delivery offset $Y$ (hours from schedule) is modeled on $[-1,3]$ with a triangular density
		that rises to a peak at $y=1$:
		\[
		f(y)=\begin{cases}
		k(y+1), & -1\le y\le 1,\\
		k(3-y), & 1< y\le 3.
		\end{cases}
		\]
		
		\textbf{Question.}
		(a) Determine $k$.
		(b) Compute $P(0\le Y\le 2)$.
		(c) Interpret your answer.
		
		\textbf{Solution.}
		The whole graph is one triangle with base $4$ (from $-1$ to $3$) and height $f(1)=2k$.
		Thus
		\[
		\frac{1}{2}(4)(2k)=4k=1\Rightarrow k=\frac{1}{4}.
		\]
		For $0\le Y\le 2$, use two equal trapezoids:
		on $[0,1]$ heights are $f(0)=\frac{1}{4}$ and $f(1)=\frac{1}{2}$,
		so area is $\frac{\left(\frac{1}{4}+\frac{1}{2}\right)}{2}\cdot 1=\frac{3}{8}$.
		By symmetry, area on $[1,2]$ is also $\frac{3}{8}$.
		Hence
		\[
		P(0\le Y\le 2)=\frac{3}{8}+\frac{3}{8}=\frac{3}{4}.
		\]
		\textbf{Interpretation.} There is a $\frac{3}{4}$ chance the offset is between 0 and 2 hours.
		
		% --------------------------------------------------
		
		\item
		\subsection*{Problem 10 — Machine Vibration Level}
		
		\textbf{Problem.}
		A vibration measure $V$ is continuous on $[0,6]$ with density
		\[
		f(v)=\begin{cases}
		k, & 0\le v<2,\\
		2k, & 2\le v\le 6.
		\end{cases}
		\]
		
		\textbf{Question.}
		(a) Find $k$.
		(b) Compute $P(V\ge 3)$.
		(c) Interpret the result.
		
		\textbf{Solution.}
		Total area:
		\[
		(2\cdot k)+(4\cdot 2k)=2k+8k=10k=1\Rightarrow k=\frac{1}{10}.
		\]
		For $V\ge 3$, only the second block is used from $3$ to $6$.
		Width is $3$, height is $2k=\frac{1}{5}$:
		\[
		P(V\ge 3)=3\cdot\frac{1}{5}=\frac{3}{5}.
		\]
		\textbf{Interpretation.} The chance the vibration level is at least 3 is $\frac{3}{5}$.
		
		% --------------------------------------------------
		
		\item
		\subsection*{Problem 11 — Straight-Line Cooling Error}
		
		\textbf{Problem.}
		A cooling error variable $E$ is modeled on $[1,5]$ by
		\[
		f(e)=k(5-e),\quad 1\le e\le 5.
		\]
		
		\textbf{Question.}
		(a) Determine $k$.
		(b) Compute $P(2\le E\le 4)$.
		(c) Interpret the probability.
		
		\textbf{Solution.}
		The full graph is a triangle decreasing from height $f(1)=4k$ to 0 at $e=5$.
		Base is $4$, so
		\[
		\frac{1}{2}(4)(4k)=8k=1\Rightarrow k=\frac{1}{8}.
		\]
		For $2\le E\le 4$, the shape is a trapezoid with
		left height $f(2)=\frac{3}{8}$, right height $f(4)=\frac{1}{8}$, width $2$:
		\[
		P(2\le E\le 4)=\frac{\left(\frac{3}{8}+\frac{1}{8}\right)}{2}\cdot 2
		=\frac{1}{2}.
		\]
		\textbf{Interpretation.} There is a $\frac{1}{2}$ chance the error is between 2 and 4 units.
		
		% --------------------------------------------------
		
		\item
		\subsection*{Problem 12 — Traffic Gap with Flat Peak}
		
		\textbf{Problem.}
		Let $G$ (seconds) be a traffic gap modeled on $[0,6]$ by
		\[
		f(g)=\begin{cases}
		kg, & 0\le g<2,\\
		2k, & 2\le g<4,\\
		k(6-g), & 4\le g\le 6.
		\end{cases}
		\]
		
		\textbf{Question.}
		(a) Find $k$.
		(b) Compute $P(1\le G\le 5)$.
		(c) Interpret in context.
		
		\textbf{Solution.}
		Compute total area by pieces:
		left triangle on $[0,2]$: $\frac{1}{2}(2)(2k)=2k$,
		middle rectangle on $[2,4]$: $(2)(2k)=4k$,
		right triangle on $[4,6]$: $\frac{1}{2}(2)(2k)=2k$.
		So
		\[
		2k+4k+2k=8k=1\Rightarrow k=\frac{1}{8}.
		\]
		Now $P(1\le G\le 5)$ equals total area minus two small end triangles:
		each end triangle has base $1$ and height $\frac{1}{8}$, so each area is
		$\frac{1}{2}(1)\left(\frac{1}{8}\right)=\frac{1}{16}$.
		Hence
		\[
		P(1\le G\le 5)=1-\frac{1}{16}-\frac{1}{16}=\frac{7}{8}.
		\]
		\textbf{Interpretation.} A gap between 1 s and 5 s occurs with probability $\frac{7}{8}$.
		
		% --------------------------------------------------
		
		\item
		\subsection*{Problem 13 — Quality Score Trapezoid}
		
		\textbf{Problem.}
		A quality score $Q$ is modeled on $[2,8]$ by a line that starts above zero:
		\[
		f(q)=k\left(1+\frac{q-2}{3}\right),\quad 2\le q\le 8.
		\]
		This is linear, with $f(2)=k$ and $f(8)=3k$.
		
		\textbf{Question.}
		(a) Determine $k$.
		(b) Compute $P(4\le Q\le 7)$.
		(c) Interpret the result.
		
		\textbf{Solution.}
		The full graph on $[2,8]$ is a trapezoid with parallel sides $k$ and $3k$, width $6$:
		\[
		\text{Area}=\frac{k+3k}{2}\cdot 6=12k=1\Rightarrow k=\frac{1}{12}.
		\]
		For $[4,7]$, heights are
		\[
		f(4)=\frac{1}{12}\left(1+\frac{2}{3}\right)=\frac{5}{36},\qquad
		f(7)=\frac{1}{12}\left(1+\frac{5}{3}\right)=\frac{2}{9}.
		\]
		Width is $3$, so trapezoid area is
		\[
		P(4\le Q\le 7)=\frac{\frac{5}{36}+\frac{2}{9}}{2}\cdot 3
		=\frac{\frac{5}{36}+\frac{8}{36}}{2}\cdot 3
		=\frac{13}{24}.
		\]
		\textbf{Interpretation.} The probability that the quality score lies from 4 to 7 is $\frac{13}{24}$.
		
		% --------------------------------------------------
		
		\item
		\subsection*{Problem 14 — Centered Error Triangle}
		
		\textbf{Problem.}
		A measurement error $Z$ is modeled on $[-2,2]$ by
		\[
		f(z)=\begin{cases}
		k(z+2), & -2\le z\le 0,\\
		k(2-z), & 0< z\le 2.
		\end{cases}
		\]
		
		\textbf{Question.}
		(a) Find $k$.
		(b) Compute $P(-1\le Z\le 1)$.
		(c) Interpret in context.
		
		\textbf{Solution.}
		The graph is a symmetric triangle with base $4$ and peak height $f(0)=2k$.
		So
		\[
		\frac{1}{2}(4)(2k)=4k=1\Rightarrow k=\frac{1}{4}.
		\]
		For $-1\le Z\le 1$, use symmetry:
		compute area on $[-1,0]$ and double it.
		On $[-1,0]$, heights are $f(-1)=\frac{1}{4}$ and $f(0)=\frac{1}{2}$ with width $1$:
		\[
		\text{Area}[-1,0]=\frac{\frac{1}{4}+\frac{1}{2}}{2}\cdot 1=\frac{3}{8}.
		\]
		Therefore
		\[
		P(-1\le Z\le 1)=2\cdot\frac{3}{8}=\frac{3}{4}.
		\]
		\textbf{Interpretation.} The chance the error stays within 1 unit of zero is $\frac{3}{4}$.
		
		% --------------------------------------------------
		
		\item
		\subsection*{Problem 15 — Three-Level Service Time}
		
		\textbf{Problem.}
		A service time variable $S$ (minutes) is modeled on $[0,10]$ by
		\[
		f(s)=\begin{cases}
		k, & 0\le s<4,\\
		\frac{k}{2}, & 4\le s<8,\\
		\frac{3k}{2}, & 8\le s\le 10.
		\end{cases}
		\]
		
		\textbf{Question.}
		(a) Determine $k$.
		(b) Compute $P(S<7)$.
		(c) Interpret the result.
		
		\textbf{Solution.}
		Use rectangle areas for total area 1:
		\[
		(4)(k)+(4)\left(\frac{k}{2}\right)+(2)\left(\frac{3k}{2}\right)
		=4k+2k+3k=9k=1,
		\]
		so
		\[
		k=\frac{1}{9}.
		\]
		For $S<7$, include $[0,4]$ and $[4,7]$:
		\[
		P(S<7)=4\left(\frac{1}{9}\right)+3\left(\frac{1}{18}\right)
		=\frac{4}{9}+\frac{1}{6}=\frac{11}{18}.
		\]
		\textbf{Interpretation.} The model predicts a $\frac{11}{18}$ chance that service takes less than 7 minutes.
		
	\end{ExamProblems}
	
\end{document}
