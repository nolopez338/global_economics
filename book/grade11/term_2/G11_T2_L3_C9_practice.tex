\makeatletter
\def\input@path{{./}{../}{../../}{preamble/}{../preamble/}{../../preamble/}}
\makeatother
% ----------------------------------------------------------
% GENERAL 

% File
\documentclass[11pt]{book}

% Margins
\usepackage[margin=1in]{geometry}

\linespread{1.2}            % Line spacing
\usepackage[utf8]{inputenc}
\usepackage[T1]{fontenc}
\usepackage{lmodern}
\usepackage{microtype}
\setlength{\parindent}{0pt}
\setlength{\parskip}{6pt}
\usepackage{booktabs}

% ----------------------------------------------------------
% TABLES
\usepackage{multicol}
\usepackage{longtable} 
\usepackage{array}
\usepackage{booktabs}
\usepackage{tabularx}
\usepackage{multirow}

% ----------------------------------------------------------
% MATHEMATICS
\usepackage{amsmath}
\usepackage{amssymb}
\usepackage{amsfonts}
\usepackage{mathtools}

% ----------------------------------------------------------
% HIDDEN CONTENT
\usepackage{ifthen}
% Define a boolean switch
\newboolean{explicaciones}
% Set the boolean switch to true or false
% Change to true to show the content

% Explanations
\newcommand{\explicacion}[2]{
	\ifthenelse{\boolean{explicaciones}}{#1}{#2}
}
\newcommand{\mostrarExplicaciones}[1]{\setboolean{explicaciones}{#1}}

% ----------------------------------------------------------
% NUMBERING

\usepackage{fancyhdr}
\pagestyle{empty} % Ensures the entire document has no page numbers

\usepackage{tocloft}
\renewcommand{\cftdot}{} % Remove dots for sections, if any
\renewcommand{\cftsecleader}{\cftdotfill{\cftdotsep}} % Remove dots for sections, if any
\cftpagenumbersoff{section} % Removes page numbers from sections
\cftpagenumbersoff{subsection} % Removes page numbers from subsections

% ----------------------------------------------------------
% IMAGES 

% General settings
\usepackage{graphicx}       % Insert images
\usepackage{float}          % Position images
% \usepackage{subfigure}      % Subfigures
\graphicspath{{imgs}}       % Image location
\usepackage{subcaption}     % Subfigures II
\usepackage{verbatim}

% Figures
\usepackage{tikz}
\usetikzlibrary{arrows.meta,positioning,trees}

% Colors
\usepackage{xcolor}     
\definecolor{popUp}{HTML}{666666}
\definecolor{popUpIn}{HTML}{CED9E0}
\definecolor{backgroundC}{HTML}{D0E8F2}
\definecolor{backgroundCC}{HTML}{FFFFFF}
\definecolor{borders}{HTML}{8c120d}
\definecolor{padding}{HTML}{77D0D7}
\definecolor{links}{HTML}{CC6F5F}

% ----------------------------------------------------------
% FRAMES

% General settings
\usepackage{tcolorbox}
\usepackage{adjustbox}          % Adjusted frame  
\setlength{\fboxrule}{3pt}  % Line width
\setlength{\fboxsep}{3pt}   % Box padding

% General frames
\usepackage{mdframed}   

\mdfdefinestyle{estiloGeneral}{    % General style
	linecolor=black,
	linewidth=1.5pt,
	roundcorner=10pt,
	backgroundcolor=backgroundC,
	innerbottommargin=0pt
}
\mdfdefinestyle{code}{          % Code style
	linecolor=black,
	linewidth=1.5pt,
	roundcorner=10pt,
	backgroundcolor=darkgray!10,
	innerbottommargin=0pt
}

% Image frame
\newtcbox{\fboxC}{
	colback=backgroundC,
	colframe=popUp,
	arc=10pt,
	boxrule=3pt,
	boxsep=0pt, % Change the padding here
	nobeforeafter
}

% ----------------------------------------------------------
% PAGE SETTINGS

% Background 
\newcommand{\background}[0]{\begin{tikzpicture}[remember picture,overlay]
		\fill[backgroundC] (-2,2) rectangle (25cm, -550);
\end{tikzpicture}}

\newcommand{\backgroundC}[0]{\begin{tikzpicture}[remember picture,overlay]
		\fill[backgroundCC] (-2,2) rectangle (25cm, -550);
\end{tikzpicture}}

% Page width 
\newcommand{\anchoPag}[0]{20cm}

% ----------------------------------------------------------
% FONT

% General
\usepackage{tgbonum}        % Font
\usepackage{listings}       % Code typesetting
\usepackage[spanish]{babel} % Load Spanish
\selectlanguage{spanish}    % Select Spanish
\usepackage{enumitem}
\usepackage{bookmark}

\setlist[itemize]{leftmargin=1.2em, itemsep=0.35em, topsep=0.35em}

% --- Table helpers ---
\newcolumntype{L}[1]{>{\raggedright\arraybackslash}p{#1}}
\newcolumntype{Y}{>{\raggedright\arraybackslash}X}
\newcolumntype{C}{>{\centering\arraybackslash}X}
\renewcommand{\arraystretch}{1.1}

% Python style
\lstdefinestyle{python}{
	language=Python,
	basicstyle=\ttfamily\small,
	commentstyle=\color{green!50!black},
	keywordstyle=\color{blue},
	numberstyle=\tiny\color{gray},
	numbers=left,
	morekeywords={>, <},
	breakatwhitespace=false,
	showstringspaces=false,
	showtabs=false,
	showspaces=false
}

% ----------------------------------------------------------
% HYPERLINKS

% General
\usepackage{hyperref}       
\hypersetup{
	colorlinks=true,
	linkcolor=links,
	filecolor=magenta,      
	urlcolor=brown,
}

% Custom commands 

% Large link
\newcommand{\bigLink}[2]{\begin{center} \fboxC{\LARGE{\href{#1}{#2}}}\end{center}}

% Small link
\newcommand{\smallLink}[2]{\begin{center}\fboxC{\href{#1}{#2}}\end{center}}

% Bold link
\newcommand{\bfLink}[2]{\href{#1}{\textbf{#2}}}


% Small URL
\newcommand{\smallUrl}[1]{\begin{center}\fboxC{\url{#1}}\end{center}}


% ----------------------------------------------------------
% CUSTOM COMMANDS FOR FIGURES

\newcommand{\espacioImagenes}[0]{-1.2cm}

% Without frame
\newcommand{\fig}[3][\espacioImagenes]{
	\hspace*{#1}
	\centering
	\includegraphics[width=#2\textwidth]{#3}
}

% With frame
\newcommand{\ffig}[2]{\begin{figure}[h]
		\hspace*{\espacioImagenes}
		\centering
		\fbox{\includegraphics[width=#1\textwidth]{#2}}
\end{figure}}

% Hyperlink with frame
\newcommand{\hfig}[3]{\begin{figure}[h]
		\hspace*{-1.4cm}
		\centering
		\color{popUp}
		\fboxC{\href{#1}{\includegraphics[width=#2\textwidth]{#3}}}
	\end{figure}
}

% Hyperlink without frame
\newcommand{\hffig}[3]{\begin{figure}[h]
		\hspace*{-1.1cm}
		\centering
		\color{popUp}
		\href{#1}{\includegraphics[width=#2\textwidth]{#3}}
	\end{figure}
}

% ----------------------------------------------------------

% Start and Contents
\newcommand{\cuadro}[1]{
	\begin{mdframed}[style=estiloGeneral]
		#1 
	\end{mdframed}
}

% Explanation video image
\newcommand{\linkExplicacion}[1]{
	\hffig{#1}{0.5}{principal/videoExplicacion}
	\vspace{-0.5cm}
}

\newcommand{\subSecLink}[2]{
	\subsubsection*{\href{#1}{\textbf{#2}}}
}

% Spacing
\newcommand{\esp}[0]{\vspace{4mm}}

% Back to start
\newcommand{\secInicio}[0]{\begin{center}\hyperref[sec:inicio]{ 
			\includegraphics[width=0.1\textwidth]{principal/up}
	}\end{center}
}


\geometry{margin=0.85in}
\AtBeginDocument{\small}

\newcommand{\ExamNameField}{\noindent\textbf{Name:}\ \rule{0.7\linewidth}{0.4pt}\par\medskip}

\newcommand{\ExamTitleBlock}[3]{%
	\begin{center}
		\Large\textbf{#1}\\
		\textbf{#2}%
		\if\relax\detokenize{#3}\relax\else\\\textbf{#3}\fi
	\end{center}
	\vspace{0.5em}
}

\newcommand{\ExamSection}[1]{\par\medskip\textbf{#1}\par\smallskip}

\newenvironment{ExamCriteria}{%
	\begin{itemize}[leftmargin=1.6em, itemsep=0.3em, topsep=0.2em]
}{%
	\end{itemize}
}

\newenvironment{ExamProblems}{%
	\begin{enumerate}[label=\textbf{P\arabic*}, leftmargin=0pt, labelsep=0.6em, itemindent=2.2em, itemsep=0.8em]
}{%
	\end{enumerate}
}

\begin{document}
	\ExamSection{C9: Confidence interval for the mean when variance is unknown}
	\begin{ExamProblems}
		\item
		\subsection*{Problem description}
		A retail bank studies the mean monthly card spending (hundreds of USD) of new digital-account clients.
		From a large random sample, the summary statistics are: $n=36$, $\bar{x}=52.40$, $s=8.10$.
		Construct both a 95\% t-confidence interval and a 95\% z-confidence interval for the population mean, then compare them.
		
		\subsection*{C9}
		Step 1: Identify parameter and statistics
		\[
		\mu=\text{population mean monthly card spending},\quad n=36,\quad \bar{x}=52.40,\quad s=8.10
		\]
		\[
		df=n-1=35,\quad n\ge 30\text{ (large sample)}
		\]
		Step 2: Construct t-interval
		\[
		SE=\frac{s}{\sqrt{n}}=\frac{8.10}{\sqrt{36}}=1.35,\qquad t^*=t_{0.025,35}\approx 2.030
		\]
		\[
		\mu\in \bar{x}\pm t^*SE=52.40\pm 2.030(1.35)=52.40\pm 2.74=[49.66,\,55.14]
		\]
		Step 3: Construct z-interval
		\[
		z^*=z_{0.025}=1.960,\qquad \mu\in \bar{x}\pm z^*SE=52.40\pm 1.960(1.35)=52.40\pm 2.65=[49.75,\,55.05]
		\]
		Step 4: Numerical comparison
		\[
			t^*=2.030>1.960=z^*,\quad E_t=2.74>2.65=E_z
		\]
		\[
			\text{Width}_t=2(2.74)=5.48,\qquad \text{Width}_z=2(2.65)=5.30
		\]
		\[
			\text{Absolute difference}=\text{Width}_t-\text{Width}_z=5.48-5.30=0.18
		\]
		\[
			\text{Relative difference (base }z\text{-interval)}=\frac{\text{Width}_t-\text{Width}_z}{\text{Width}_z}\times 100
			=\frac{5.48-5.30}{5.30}\times 100=3.40\%
		\]
		The t-interval is 3.40\% wider than the z-interval, using the z-interval width as the base reference.
		Step 5: Conceptual explanation of convergence
		Because $df=35$ is large, the t distribution is very close to the standard normal distribution.
		As $df$ increases, $t(df)\to N(0,1)$, so the t and z intervals become very similar.
		The normal approximation is reasonable here, while the t-interval remains the exact method for unknown variance.
		Step 6: Interpretation in economic/financial context
		Both intervals indicate that the bank's mean monthly card spending is around 50 to 55 hundred USD.
		Using the exact t method, a 95\% confidence interval is $[49.66,55.14]$ hundred USD.
		
		\newpage
		\item
		\subsection*{Problem description}
		An import company estimates the mean daily shipping cost per container (thousand USD).
		From a large random sample, the summary statistics are: $n=64$, $\bar{x}=1.84$, $s=0.56$.
		Construct both a 90\% t-confidence interval and a 90\% z-confidence interval for the population mean, then compare them.
		
		\subsection*{C9}
		Step 1: Identify parameter and statistics
		\[
		\mu=\text{population mean daily shipping cost per container},\quad n=64,\quad \bar{x}=1.84,\quad s=0.56
		\]
		\[
		df=63,\quad n\ge 30\text{ (large sample)}
		\]
		Step 2: Construct t-interval
		\[
		SE=\frac{0.56}{\sqrt{64}}=0.07,\qquad t^*=t_{0.05,63}\approx 1.669
		\]
		\[
		\mu\in 1.84\pm 1.669(0.07)=1.84\pm 0.117=[1.723,\,1.957]
		\]
		Step 3: Construct z-interval
		\[
		z^*=z_{0.05}=1.645,\qquad \mu\in 1.84\pm 1.645(0.07)=1.84\pm 0.115=[1.725,\,1.955]
		\]
		Step 4: Numerical comparison
		\[
			t^*=1.669>1.645=z^*,\quad E_t=0.117>0.115=E_z
		\]
		\[
			\text{Width}_t=0.234,\qquad \text{Width}_z=0.230
		\]
		\[
			\text{Absolute difference}=\text{Width}_t-\text{Width}_z=0.234-0.230=0.004
		\]
		\[
			\text{Relative difference (base }z\text{-interval)}=\frac{\text{Width}_t-\text{Width}_z}{\text{Width}_z}\times 100
			=\frac{0.234-0.230}{0.230}\times 100=1.74\%
		\]
		The t-interval is 1.74\% wider than the z-interval, using the z-interval width as the base reference.
		Step 5: Conceptual explanation of convergence
		With $df=63$, the t critical value is already very close to the z critical value.
		This reflects convergence: as $df$ increases, $t(df)$ converges to $N(0,1)$.
		Therefore, the normal approximation is reasonable in this large-sample setting.
		Step 6: Interpretation in economic/financial context
		The company can estimate mean shipping cost at about 1.72 to 1.96 thousand USD per container.
		Using the exact t method, the 90\% confidence interval is $[1.723,1.957]$ thousand USD.
		
		\newpage
		\item
		\subsection*{Problem description}
		An economist estimates the mean monthly fuel expense (thousand USD) for urban delivery firms.
		From a large random sample, the summary statistics are: $n=49$, $\bar{x}=27.30$, $s=4.90$.
		Construct both a 99\% t-confidence interval and a 99\% z-confidence interval for the population mean, then compare them.
		
		\subsection*{C9}
		Step 1: Identify parameter and statistics
		\[
		\mu=\text{population mean monthly fuel expense},\quad n=49,\quad \bar{x}=27.30,\quad s=4.90
		\]
		\[
		df=48,\quad n\ge 30\text{ (large sample)}
		\]
		Step 2: Construct t-interval
		\[
		SE=\frac{4.90}{\sqrt{49}}=0.70,\qquad t^*=t_{0.005,48}\approx 2.682
		\]
		\[
		\mu\in 27.30\pm 2.682(0.70)=27.30\pm 1.88=[25.42,\,29.18]
		\]
		Step 3: Construct z-interval
		\[
		z^*=z_{0.005}=2.576,\qquad \mu\in 27.30\pm 2.576(0.70)=27.30\pm 1.80=[25.50,\,29.10]
		\]
		Step 4: Numerical comparison
		\[
			t^*=2.682>2.576=z^*,\quad E_t=1.88>1.80=E_z
		\]
		\[
			\text{Width}_t=3.76,\qquad \text{Width}_z=3.60
		\]
		\[
			\text{Absolute difference}=\text{Width}_t-\text{Width}_z=3.76-3.60=0.16
		\]
		\[
			\text{Relative difference (base }z\text{-interval)}=\frac{\text{Width}_t-\text{Width}_z}{\text{Width}_z}\times 100
			=\frac{3.76-3.60}{3.60}\times 100=4.44\%
		\]
		The t-interval is 4.44\% wider than the z-interval, using the z-interval width as the base reference.
		Step 5: Conceptual explanation of convergence
		Even at 99\% confidence, the two intervals are close because the sample is large.
		As degrees of freedom increase, the t distribution converges to the standard normal distribution, $t(df)\to N(0,1)$.
		So the z interval is a reasonable approximation, while the t interval remains the exact choice when $\sigma$ is unknown.
		Step 6: Interpretation in economic/financial context
		The estimated mean monthly fuel expense is around 25.4 to 29.2 thousand USD.
		Using the exact method, the 99\% t-confidence interval is $[25.42,29.18]$ thousand USD.
	\end{ExamProblems}
\end{document}
