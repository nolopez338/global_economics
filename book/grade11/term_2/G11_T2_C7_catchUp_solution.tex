\makeatletter
\def\input@path{{./}{../}{../../}{preamble/}{../preamble/}{../../preamble/}}
\makeatother
% ----------------------------------------------------------
% GENERAL 

% File
\documentclass[11pt]{book}

% Margins
\usepackage[margin=1in]{geometry}

\linespread{1.2}            % Line spacing
\usepackage[utf8]{inputenc}
\usepackage[T1]{fontenc}
\usepackage{lmodern}
\usepackage{microtype}
\setlength{\parindent}{0pt}
\setlength{\parskip}{6pt}
\usepackage{booktabs}

% ----------------------------------------------------------
% TABLES
\usepackage{multicol}
\usepackage{longtable} 
\usepackage{array}
\usepackage{booktabs}
\usepackage{tabularx}
\usepackage{multirow}

% ----------------------------------------------------------
% MATHEMATICS
\usepackage{amsmath}
\usepackage{amssymb}
\usepackage{amsfonts}
\usepackage{mathtools}

% ----------------------------------------------------------
% HIDDEN CONTENT
\usepackage{ifthen}
% Define a boolean switch
\newboolean{explicaciones}
% Set the boolean switch to true or false
% Change to true to show the content

% Explanations
\newcommand{\explicacion}[2]{
	\ifthenelse{\boolean{explicaciones}}{#1}{#2}
}
\newcommand{\mostrarExplicaciones}[1]{\setboolean{explicaciones}{#1}}

% ----------------------------------------------------------
% NUMBERING

\usepackage{fancyhdr}
\pagestyle{empty} % Ensures the entire document has no page numbers

\usepackage{tocloft}
\renewcommand{\cftdot}{} % Remove dots for sections, if any
\renewcommand{\cftsecleader}{\cftdotfill{\cftdotsep}} % Remove dots for sections, if any
\cftpagenumbersoff{section} % Removes page numbers from sections
\cftpagenumbersoff{subsection} % Removes page numbers from subsections

% ----------------------------------------------------------
% IMAGES 

% General settings
\usepackage{graphicx}       % Insert images
\usepackage{float}          % Position images
% \usepackage{subfigure}      % Subfigures
\graphicspath{{imgs}}       % Image location
\usepackage{subcaption}     % Subfigures II
\usepackage{verbatim}

% Figures
\usepackage{tikz}
\usetikzlibrary{arrows.meta,positioning,trees}

% Colors
\usepackage{xcolor}     
\definecolor{popUp}{HTML}{666666}
\definecolor{popUpIn}{HTML}{CED9E0}
\definecolor{backgroundC}{HTML}{D0E8F2}
\definecolor{backgroundCC}{HTML}{FFFFFF}
\definecolor{borders}{HTML}{8c120d}
\definecolor{padding}{HTML}{77D0D7}
\definecolor{links}{HTML}{CC6F5F}

% ----------------------------------------------------------
% FRAMES

% General settings
\usepackage{tcolorbox}
\usepackage{adjustbox}          % Adjusted frame  
\setlength{\fboxrule}{3pt}  % Line width
\setlength{\fboxsep}{3pt}   % Box padding

% General frames
\usepackage{mdframed}   

\mdfdefinestyle{estiloGeneral}{    % General style
	linecolor=black,
	linewidth=1.5pt,
	roundcorner=10pt,
	backgroundcolor=backgroundC,
	innerbottommargin=0pt
}
\mdfdefinestyle{code}{          % Code style
	linecolor=black,
	linewidth=1.5pt,
	roundcorner=10pt,
	backgroundcolor=darkgray!10,
	innerbottommargin=0pt
}

% Image frame
\newtcbox{\fboxC}{
	colback=backgroundC,
	colframe=popUp,
	arc=10pt,
	boxrule=3pt,
	boxsep=0pt, % Change the padding here
	nobeforeafter
}

% ----------------------------------------------------------
% PAGE SETTINGS

% Background 
\newcommand{\background}[0]{\begin{tikzpicture}[remember picture,overlay]
		\fill[backgroundC] (-2,2) rectangle (25cm, -550);
\end{tikzpicture}}

\newcommand{\backgroundC}[0]{\begin{tikzpicture}[remember picture,overlay]
		\fill[backgroundCC] (-2,2) rectangle (25cm, -550);
\end{tikzpicture}}

% Page width 
\newcommand{\anchoPag}[0]{20cm}

% ----------------------------------------------------------
% FONT

% General
\usepackage{tgbonum}        % Font
\usepackage{listings}       % Code typesetting
\usepackage[spanish]{babel} % Load Spanish
\selectlanguage{spanish}    % Select Spanish
\usepackage{enumitem}
\usepackage{bookmark}

\setlist[itemize]{leftmargin=1.2em, itemsep=0.35em, topsep=0.35em}

% --- Table helpers ---
\newcolumntype{L}[1]{>{\raggedright\arraybackslash}p{#1}}
\newcolumntype{Y}{>{\raggedright\arraybackslash}X}
\newcolumntype{C}{>{\centering\arraybackslash}X}
\renewcommand{\arraystretch}{1.1}

% Python style
\lstdefinestyle{python}{
	language=Python,
	basicstyle=\ttfamily\small,
	commentstyle=\color{green!50!black},
	keywordstyle=\color{blue},
	numberstyle=\tiny\color{gray},
	numbers=left,
	morekeywords={>, <},
	breakatwhitespace=false,
	showstringspaces=false,
	showtabs=false,
	showspaces=false
}

% ----------------------------------------------------------
% HYPERLINKS

% General
\usepackage{hyperref}       
\hypersetup{
	colorlinks=true,
	linkcolor=links,
	filecolor=magenta,      
	urlcolor=brown,
}

% Custom commands 

% Large link
\newcommand{\bigLink}[2]{\begin{center} \fboxC{\LARGE{\href{#1}{#2}}}\end{center}}

% Small link
\newcommand{\smallLink}[2]{\begin{center}\fboxC{\href{#1}{#2}}\end{center}}

% Bold link
\newcommand{\bfLink}[2]{\href{#1}{\textbf{#2}}}


% Small URL
\newcommand{\smallUrl}[1]{\begin{center}\fboxC{\url{#1}}\end{center}}


% ----------------------------------------------------------
% CUSTOM COMMANDS FOR FIGURES

\newcommand{\espacioImagenes}[0]{-1.2cm}

% Without frame
\newcommand{\fig}[3][\espacioImagenes]{
	\hspace*{#1}
	\centering
	\includegraphics[width=#2\textwidth]{#3}
}

% With frame
\newcommand{\ffig}[2]{\begin{figure}[h]
		\hspace*{\espacioImagenes}
		\centering
		\fbox{\includegraphics[width=#1\textwidth]{#2}}
\end{figure}}

% Hyperlink with frame
\newcommand{\hfig}[3]{\begin{figure}[h]
		\hspace*{-1.4cm}
		\centering
		\color{popUp}
		\fboxC{\href{#1}{\includegraphics[width=#2\textwidth]{#3}}}
	\end{figure}
}

% Hyperlink without frame
\newcommand{\hffig}[3]{\begin{figure}[h]
		\hspace*{-1.1cm}
		\centering
		\color{popUp}
		\href{#1}{\includegraphics[width=#2\textwidth]{#3}}
	\end{figure}
}

% ----------------------------------------------------------

% Start and Contents
\newcommand{\cuadro}[1]{
	\begin{mdframed}[style=estiloGeneral]
		#1 
	\end{mdframed}
}

% Explanation video image
\newcommand{\linkExplicacion}[1]{
	\hffig{#1}{0.5}{principal/videoExplicacion}
	\vspace{-0.5cm}
}

\newcommand{\subSecLink}[2]{
	\subsubsection*{\href{#1}{\textbf{#2}}}
}

% Spacing
\newcommand{\esp}[0]{\vspace{4mm}}

% Back to start
\newcommand{\secInicio}[0]{\begin{center}\hyperref[sec:inicio]{ 
			\includegraphics[width=0.1\textwidth]{principal/up}
	}\end{center}
}


\geometry{margin=0.85in}
\AtBeginDocument{\small}

\newcommand{\ExamNameField}{\noindent\textbf{Name:}\ \rule{0.7\linewidth}{0.4pt}\par\medskip}

\newcommand{\ExamTitleBlock}[3]{%
	\begin{center}
		\Large\textbf{#1}\\
		\textbf{#2}%
		\if\relax\detokenize{#3}\relax\else\\\textbf{#3}\fi
	\end{center}
	\vspace{0.5em}
}

\newcommand{\ExamSection}[1]{\par\medskip\textbf{#1}\par\smallskip}

\newenvironment{ExamCriteria}{%
	\begin{itemize}[leftmargin=1.6em, itemsep=0.3em, topsep=0.2em]
}{%
	\end{itemize}
}

\newenvironment{ExamProblems}{%
	\begin{enumerate}[label=\textbf{P\arabic*}, leftmargin=0pt, labelsep=0.6em, itemindent=2.2em, itemsep=0.8em]
}{%
	\end{enumerate}
}

\begin{document}
\ExamTitleBlock{11th grade}{Term 2 Catch-up Solution: C7 (Confidence Intervals for a Proportion)}{}

\subsection*{Problem 1}
\subsection*{Problem description}
A digital bank wants to estimate the proportion of young adult clients (ages 20--29) who use automatic monthly savings transfers. In a random sample of \(n=120\) clients, \(x=66\) report using automatic transfers. Construct and interpret a \(90\%\) confidence interval for the population proportion.

\subsection*{C7}
We construct a confidence interval for a population proportion:
\[
\hat p \pm z^*\sqrt{\frac{\hat p(1-\hat p)}{n}}
\]

Step 1: Sample proportion
\[
\hat p=\frac{x}{n}=\frac{66}{120}=0.55
\]
Check conditions:
\[
n\hat p=120(0.55)=66\ge 10,\qquad n(1-\hat p)=120(0.45)=54\ge 10
\]

Step 2: Standard error
\[
SE=\sqrt{\frac{0.55(0.45)}{120}}\approx 0.0454
\]

Step 3: Margin of error for \(90\%\) confidence \((z^*=1.645)\)
\[
E=1.645(0.0454)\approx 0.0747
\]

Step 4: Confidence interval
\[
0.55\pm 0.0747\Rightarrow [0.475,\,0.625]
\]

Interpretation: We are 90\% confident that between 47.5\% and 62.5\% of young adult clients use automatic monthly savings transfers.

\subsection*{Problem 2}
\subsection*{Problem description}
A rideshare platform wants to estimate the proportion of drivers in their 20s who bought private accident insurance this year. In a sample of \(n=200\) drivers, \(x=92\) purchased insurance. Construct \(90\%\), \(95\%\), and \(99\%\) confidence intervals for the population proportion, and give a brief interpretation.

\subsection*{C7}
Step 1: Sample proportion
\[
\hat p=\frac{92}{200}=0.46
\]
Conditions:
\[
n\hat p=92\ge 10,\qquad n(1-\hat p)=108\ge 10
\]

Step 2: Standard error
\[
SE=\sqrt{\frac{0.46(0.54)}{200}}\approx 0.0352
\]

Step 3: Margins of error
\[
E_{90}=1.645(0.0352)\approx 0.0579
\]
\[
E_{95}=1.96(0.0352)\approx 0.0690
\]
\[
E_{99}=2.576(0.0352)\approx 0.0907
\]

Step 4: Confidence intervals
\[
90\%:[0.46\pm 0.0579]\Rightarrow [0.402,\,0.518]
\]
\[
95\%:[0.46\pm 0.0690]\Rightarrow [0.391,\,0.529]
\]
\[
99\%:[0.46\pm 0.0907]\Rightarrow [0.369,\,0.551]
\]

Interpretation: As confidence increases, the interval becomes wider. All three intervals suggest the insured proportion is around the mid-40\% range.

\subsection*{Problem 3}
\subsection*{Problem description}
A fintech app wants to estimate the proportion of users in their 20s who pay credit card balances in full each month. In a sample of \(n=260\) users, \(x=182\) report paying in full. Construct and interpret a \(95\%\) confidence interval. Then answer: is a benchmark value of \(p=0.75\) plausible based on this interval?

\subsection*{C7}
Step 1: Sample proportion
\[
\hat p=\frac{182}{260}=0.70
\]
Conditions:
\[
n\hat p=182\ge 10,\qquad n(1-\hat p)=78\ge 10
\]

Step 2: Standard error
\[
SE=\sqrt{\frac{0.70(0.30)}{260}}\approx 0.0284
\]

Step 3: Margin of error \((95\%,\ z^*=1.96)\)
\[
E=1.96(0.0284)\approx 0.0557
\]

Step 4: Confidence interval
\[
0.70\pm 0.0557\Rightarrow [0.644,\,0.756]
\]

Interpretation: We are 95\% confident that between 64.4\% and 75.6\% of similar users pay their balances in full monthly.

Analysis question: Is \(p=0.75\) plausible?

Answer: Yes. The value 0.75 is inside the confidence interval, so it is plausible given this sample at the 95\% confidence level.

\subsection*{Problem 4}
\subsection*{Problem description}
A micro-investing platform is evaluating a support program. It wants the proportion of young clients who invest every month to be at least \(60\%\). After one cycle, a random sample of \(n=320\) clients finds \(x=173\) monthly investors. Construct a \(98\%\) confidence interval for the population proportion and decide whether the \(60\%\) benchmark is plausible. Justify your decision using the interval bounds.

\subsection*{C7}
Step 1: Sample proportion
\[
\hat p=\frac{173}{320}=0.5406
\]
Conditions:
\[
n\hat p=173\ge 10,\qquad n(1-\hat p)=147\ge 10
\]

Step 2: Standard error
\[
SE=\sqrt{\frac{0.5406(0.4594)}{320}}\approx 0.0279
\]

Step 3: Margin of error \((98\%,\ z^*=2.326)\)
\[
E=2.326(0.0279)\approx 0.0649
\]

Step 4: Confidence interval
\[
0.5406\pm 0.0649\Rightarrow [0.476,\,0.606]
\]

Interpretation: We are 98\% confident that the true monthly-investor proportion is between 47.6\% and 60.6\%.

Analysis question: Is the benchmark \(p\ge 0.60\) plausible, and what decision should the platform make?

Answer: The value 0.60 is inside the interval but very near the upper bound. So \(60\%\) is still plausible, but only marginally. A careful decision is that the program has not clearly established a proportion above 60\%; the platform should continue support and collect more data before claiming the benchmark is comfortably met.

\subsection*{Problem 5}
\subsection*{Problem description}
A budgeting startup compares two groups of users in their 20s.
\begin{itemize}
  \item Group A (users who completed a financial bootcamp): \(n=250\), \(x=170\) keep an emergency fund.
  \item Group B (users without bootcamp): \(n=250\), \(x=142\) keep an emergency fund.
\end{itemize}
Construct a \(95\%\) confidence interval for each group proportion. Compare the intervals and discuss what overlap or non-overlap suggests about relative plausibility of higher emergency-fund behavior.

\subsection*{C7}
For each group:
\[
\hat p \pm 1.96\sqrt{\frac{\hat p(1-\hat p)}{n}}
\]

\textbf{Group A}
\[
\hat p_A=\frac{170}{250}=0.68
\]
\[
SE_A=\sqrt{\frac{0.68(0.32)}{250}}\approx 0.0295,
\quad E_A=1.96(0.0295)\approx 0.0577
\]
\[
\text{CI}_{A,95\%}=0.68\pm 0.0577\Rightarrow [0.622,\,0.738]
\]

\textbf{Group B}
\[
\hat p_B=\frac{142}{250}=0.568
\]
\[
SE_B=\sqrt{\frac{0.568(0.432)}{250}}\approx 0.0313,
\quad E_B=1.96(0.0313)\approx 0.0613
\]
\[
\text{CI}_{B,95\%}=0.568\pm 0.0613\Rightarrow [0.507,\,0.629]
\]

Interpretation: Group A is estimated around 62.2\% to 73.8\%, while Group B is around 50.7\% to 62.9\%.

Analysis question: What does the overlap pattern suggest?

Answer: The intervals have only a small overlap region (about 62.2\% to 62.9\%). This pattern supports that higher emergency-fund behavior is more plausible in Group A, although there is still some shared plausible range.

\subsection*{Problem 6}
\subsection*{Problem description}
A national survey studies young workers and compares two sectors.
\begin{itemize}
  \item Sector 1 (gig logistics): \(n=300\), \(x=126\) contributed to retirement savings last month.
  \item Sector 2 (junior office jobs): \(n=280\), \(x=151\) contributed last month.
\end{itemize}
Construct \(90\%\) confidence intervals for both sector proportions. Which sector appears safer in terms of consistently higher retirement-saving participation, and which estimate is more uncertain?

\subsection*{C7}
\textbf{Sector 1}
\[
\hat p_1=\frac{126}{300}=0.42,
\quad SE_1=\sqrt{\frac{0.42(0.58)}{300}}\approx 0.0285
\]
\[
E_1=1.645(0.0285)\approx 0.0469
\]
\[
\text{CI}_{1,90\%}=0.42\pm 0.0469\Rightarrow [0.373,\,0.467]
\]

\textbf{Sector 2}
\[
\hat p_2=\frac{151}{280}=0.5393,
\quad SE_2=\sqrt{\frac{0.5393(0.4607)}{280}}\approx 0.0298
\]
\[
E_2=1.645(0.0298)\approx 0.0490
\]
\[
\text{CI}_{2,90\%}=0.5393\pm 0.0490\Rightarrow [0.490,\,0.588]
\]

Interpretation: Sector 2 has a clearly higher interval center and higher bounds.

Analysis questions:
\begin{itemize}
  \item Which sector appears safer for higher participation?
  \item Which estimate is more uncertain?
\end{itemize}

Answers:
\begin{itemize}
  \item Sector 2 appears safer because its entire interval is shifted upward relative to Sector 1.
  \item Sector 2 has slightly greater uncertainty here because its interval width (about 0.098) is a bit wider than Sector 1 (about 0.094).
\end{itemize}

\subsection*{Problem 7}
\subsection*{Problem description}
A payment app tracks budgeting-rule usage (the 50-30-20 rule) among users in their 20s in two cities.
\begin{itemize}
  \item City A: \(n=360\), \(x=198\) use the rule.
  \item City B: \(n=340\), \(x=170\) use the rule.
\end{itemize}
For each city, construct \(90\%\) and \(95\%\) confidence intervals. Then explain how the decision about whether City A is above \(55\%\) changes when the confidence level increases.

\subsection*{C7}
\textbf{City A}
\[
\hat p_A=\frac{198}{360}=0.55,
\quad SE_A=\sqrt{\frac{0.55(0.45)}{360}}\approx 0.0262
\]
\[
E_{A,90}=1.645(0.0262)\approx 0.0431,
\quad E_{A,95}=1.96(0.0262)\approx 0.0514
\]
\[
\text{CI}_{A,90\%}=[0.507,\,0.593],
\quad \text{CI}_{A,95\%}=[0.499,\,0.601]
\]

\textbf{City B}
\[
\hat p_B=\frac{170}{340}=0.50,
\quad SE_B=\sqrt{\frac{0.50(0.50)}{340}}\approx 0.0271
\]
\[
E_{B,90}=1.645(0.0271)\approx 0.0446,
\quad E_{B,95}=1.96(0.0271)\approx 0.0531
\]
\[
\text{CI}_{B,90\%}=[0.455,\,0.545],
\quad \text{CI}_{B,95\%}=[0.447,\,0.553]
\]

Interpretation: City A is centered higher than City B at both confidence levels.

Analysis question: How does the decision about \(p_A>0.55\) change?

Answer: At both confidence levels, 0.55 is inside City A's interval. Moving from 90\% to 95\% widens the interval and makes the claim less sharp, so we still cannot say the true proportion is clearly above 55\%; we can only say values around 55\% remain plausible.

\subsection*{Problem 8}
\subsection*{Problem description}
An online broker compares two periods for the same age group.
\begin{itemize}
  \item Period 1: \(n=420\), \(x=231\) users made at least one ETF purchase.
  \item Period 2: \(n=410\), \(x=254\) users made at least one ETF purchase.
\end{itemize}
Construct a \(95\%\) confidence interval for each period. Compare uncertainty and discuss whether adoption appears to have increased.

\subsection*{C7}
\textbf{Period 1}
\[
\hat p_1=\frac{231}{420}=0.55,
\quad SE_1=\sqrt{\frac{0.55(0.45)}{420}}\approx 0.0243
\]
\[
E_1=1.96(0.0243)\approx 0.0476
\]
\[
\text{CI}_{1,95\%}=[0.502,\,0.598]
\]

\textbf{Period 2}
\[
\hat p_2=\frac{254}{410}=0.6195,
\quad SE_2=\sqrt{\frac{0.6195(0.3805)}{410}}\approx 0.0240
\]
\[
E_2=1.96(0.0240)\approx 0.0470
\]
\[
\text{CI}_{2,95\%}=[0.573,\,0.666]
\]

Interpretation: Period 2 is centered higher, and both intervals have similar width.

Analysis questions:
\begin{itemize}
  \item Which period has greater uncertainty?
  \item Does adoption appear to have increased?
\end{itemize}

Answers:
\begin{itemize}
  \item Uncertainty is very similar; Period 1 is slightly wider.
  \item Yes, adoption appears to have increased because Period 2's interval is mostly above Period 1 and the overlap is small.
\end{itemize}

\subsection*{Problem 9}
\subsection*{Problem description}
A policy team evaluates three training channels that teach debt-management skills to young adults.
\begin{itemize}
  \item Channel A: \(n=280\), \(x=168\) participants reduced debt for three consecutive months.
  \item Channel B: \(n=300\), \(x=171\) participants reduced debt for three consecutive months.
  \item Channel C: \(n=260\), \(x=130\) participants reduced debt for three consecutive months.
\end{itemize}
Construct \(90\%\) and \(99\%\) confidence intervals for all three channels. Then recommend which channel should be prioritized if the target is to keep the true success proportion around or above \(60\%\), considering uncertainty.

\subsection*{C7}
\textbf{Channel A}
\[
\hat p_A=\frac{168}{280}=0.60,
\quad SE_A=\sqrt{\frac{0.60(0.40)}{280}}\approx 0.0293
\]
\[
E_{A,90}=1.645(0.0293)\approx 0.0482,
\quad E_{A,99}=2.576(0.0293)\approx 0.0755
\]
\[
\text{CI}_{A,90\%}=[0.552,\,0.648],
\quad \text{CI}_{A,99\%}=[0.524,\,0.676]
\]

\textbf{Channel B}
\[
\hat p_B=\frac{171}{300}=0.57,
\quad SE_B=\sqrt{\frac{0.57(0.43)}{300}}\approx 0.0286
\]
\[
E_{B,90}=1.645(0.0286)\approx 0.0470,
\quad E_{B,99}=2.576(0.0286)\approx 0.0736
\]
\[
\text{CI}_{B,90\%}=[0.523,\,0.617],
\quad \text{CI}_{B,99\%}=[0.496,\,0.644]
\]

\textbf{Channel C}
\[
\hat p_C=\frac{130}{260}=0.50,
\quad SE_C=\sqrt{\frac{0.50(0.50)}{260}}\approx 0.0310
\]
\[
E_{C,90}=1.645(0.0310)\approx 0.0510,
\quad E_{C,99}=2.576(0.0310)\approx 0.0799
\]
\[
\text{CI}_{C,90\%}=[0.449,\,0.551],
\quad \text{CI}_{C,99\%}=[0.420,\,0.580]
\]

Interpretation: Channel A is centered highest and remains closest to the 60\% target even at 99\% confidence.

Analysis question: Which channel should be prioritized and why?

Answer: Channel A should be prioritized. Its intervals are consistently higher than the others, and its 90\% interval is entirely near or above the target threshold. Channel B is moderate, while Channel C is clearly below target in most plausible values.

\subsection*{Problem 10}
\subsection*{Problem description}
A youth-finance nonprofit pilots three versions of a savings habit program.
\begin{itemize}
  \item Program X: \(n=350\), \(x=221\) participants stayed on a weekly savings plan for 12 weeks.
  \item Program Y: \(n=330\), \(x=191\) participants stayed on plan.
  \item Program Z: \(n=340\), \(x=180\) participants stayed on plan.
\end{itemize}
Construct \(95\%\) and \(99\%\) confidence intervals for each program. Then give a decision recommendation if funding should prioritize programs with a plausibly highest retention proportion and relatively stable uncertainty.

\subsection*{C7}
\textbf{Program X}
\[
\hat p_X=\frac{221}{350}=0.6314,
\quad SE_X=\sqrt{\frac{0.6314(0.3686)}{350}}\approx 0.0258
\]
\[
E_{X,95}=1.96(0.0258)\approx 0.0506,
\quad E_{X,99}=2.576(0.0258)\approx 0.0665
\]
\[
\text{CI}_{X,95\%}=[0.581,\,0.682],
\quad \text{CI}_{X,99\%}=[0.565,\,0.698]
\]

\textbf{Program Y}
\[
\hat p_Y=\frac{191}{330}=0.5788,
\quad SE_Y=\sqrt{\frac{0.5788(0.4212)}{330}}\approx 0.0272
\]
\[
E_{Y,95}=1.96(0.0272)\approx 0.0533,
\quad E_{Y,99}=2.576(0.0272)\approx 0.0701
\]
\[
\text{CI}_{Y,95\%}=[0.525,\,0.632],
\quad \text{CI}_{Y,99\%}=[0.509,\,0.649]
\]

\textbf{Program Z}
\[
\hat p_Z=\frac{180}{340}=0.5294,
\quad SE_Z=\sqrt{\frac{0.5294(0.4706)}{340}}\approx 0.0271
\]
\[
E_{Z,95}=1.96(0.0271)\approx 0.0531,
\quad E_{Z,99}=2.576(0.0271)\approx 0.0698
\]
\[
\text{CI}_{Z,95\%}=[0.476,\,0.582],
\quad \text{CI}_{Z,99\%}=[0.460,\,0.599]
\]

Interpretation: Program X has the highest center and higher interval bounds at both confidence levels.

Analysis questions:
\begin{itemize}
  \item Which program is most plausible as the strongest retention option?
  \item How does changing from 95\% to 99\% affect the confidence in the recommendation?
\end{itemize}

Answers:
\begin{itemize}
  \item Program X is the strongest recommendation because both its 95\% and 99\% intervals are shifted above Y and Z.
  \item Moving to 99\% widens all intervals, increasing caution. Even so, X remains highest, so the recommendation is robust under a stricter confidence level.
\end{itemize}

\end{document}
