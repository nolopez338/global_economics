\documentclass[12pt]{article}

% Page size and tighter margins
\usepackage[a4paper,left=1.2cm,right=1.2cm,top=1.5cm,bottom=1.5cm]{geometry}

% Core packages
\usepackage{graphicx}
\usepackage{xcolor}
\usepackage{array}
\usepackage{tabularx}
\usepackage{multicol}
\usepackage[T1]{fontenc}
\usepackage[utf8]{inputenc}

\setlength{\parindent}{0pt}
\setlength{\tabcolsep}{6pt}
\renewcommand{\arraystretch}{1.15}

% Column types
\newcolumntype{Y}{>{\raggedright\arraybackslash}m{\dimexpr0.40\textwidth-2\tabcolsep-2\arrayrulewidth\relax}}
\newcolumntype{Z}{>{\raggedright\arraybackslash}m{\dimexpr0.60\textwidth-2\tabcolsep-2\arrayrulewidth\relax}}
\newcolumntype{C}[1]{>{\centering\arraybackslash}m{#1}}

% Gray subsection header box
\newcommand{\SubsectionBox}[1]{%
	\noindent\colorbox{gray!30}{%
		\parbox{\linewidth}{\textbf{#1}}%
	}\par\vspace{0.35cm}%
}

% Centered multi-line cell helper
\newcommand{\CellCenter}[1]{%
	\parbox{\linewidth}{\centering #1}%
}

\begin{document}

	% =========================
	% HEADER BOX (3 COLUMNS)
	% =========================
	\noindent
	\begin{tabularx}{\textwidth}{|C{2.8cm}|C{\dimexpr\textwidth-6cm-4\tabcolsep-4\arrayrulewidth\relax}|C{2.8cm}|}
		\hline
		\centering
		\vspace{3mm}
		\includegraphics[width=2.5cm]{../../preamble/logo.png}
		&
		\CellCenter{%
			\vspace{-5mm}
			\textbf{GLOBAL ECONOMICS}\par
			\textbf{GRADE: 11TH}\par
			\textbf{LEARNING EVIDENCE T2 L2 C7 ACTIVITY}\par
			\textbf{CONFIDENCE INTERVALS FOR PROPORTIONS}\par
			\textbf{TEACHER'S NAME: Nicolás López Cuéllar}
		}
		&
		\CellCenter{%
			\textbf{SECOND TERM}\par
			\textbf{2025--2026}%
		}
		\\
		\hline
	\end{tabularx}

	\vspace{0.5cm}

	% =========================
	% OBJECTIVE + CRITERIA
	% =========================
	\noindent
	\begin{tabular}{|Y|Z|}
		\hline
		{\small
			\textbf{Learning objective:} Construct and interpret confidence intervals for a population proportion in context.
		}
		&
		{\footnotesize
			\textbf{Assessment criteria:}\par
			C7: Constructs the confidence interval for a proportion.\par
		}
		\\
		\hline
	\end{tabular}


	% =========================
	% EXAM BODY
	% =========================
	\begin{multicols}{2}
		\SubsectionBox{Criteria assessment}\vspace{-0.25cm}
		Criterion C7 is assessed in every problem. It is considered passed when it is correctly activated in at least six of the seven problems.
		
		\vspace{0.25cm}
		\SubsectionBox{1. Local Transport Satisfaction}\vspace{-0.25cm}
		A city transport office wants to estimate the proportion of all city commuters who are satisfied with the reliability of the bus system so it can evaluate service quality. The office randomly surveys 100 commuters, and 58 commuters in the sample say they are satisfied.

		Using this sample information, construct and interpret a 90\% confidence interval for the population proportion of all city commuters who are satisfied with the bus system.
		
		\vspace{0.25cm}
		\SubsectionBox{2. Digital Payment Adoption}\vspace{-0.25cm}
		A business association is studying the proportion of all small retailers in the region that accept digital payments to understand technology adoption in retail. It surveys 180 small retailers, and 81 of those retailers report that they accept digital payments.

		Based on the survey, construct and interpret a 95\% confidence interval for the population proportion of small retailers who accept digital payments.
		
		\vspace{0.25cm}
		\SubsectionBox{3. Water-Saving Appliance Use}\vspace{-0.25cm}
		An environmental agency wants to estimate the proportion of all households in the province that use water-saving appliances in order to plan conservation programs. It surveys 250 households, and 190 of them report using water-saving appliances.

		Use the survey results to construct and interpret a 95\% confidence interval for the population proportion of households that use water-saving appliances.
		
		\vspace{0.25cm}
		\SubsectionBox{4. Export Quality Compliance}\vspace{-0.25cm}
		A trade ministry monitors export quality and wants to estimate the proportion of all export shipments that fail to meet a required quality standard. In a recent inspection of 320 export shipments, 68 shipments fail the standard. The ministry has publicly stated that no more than 20\% of shipments fail.

		Construct a 98\% confidence interval for the population proportion of shipments that fail the standard, and use the interval to evaluate whether the ministry’s claim is plausible at the 98\% confidence level.
		
		\vspace{0.25cm}
		\SubsectionBox{5. Financial Literacy Completion}\vspace{-0.25cm}
		A national education board wants to estimate the proportion of all students who would pass a financial literacy assessment to judge the effectiveness of its curriculum. The board tests 500 students, and 275 of them pass. The board’s policy goal is that at least 60\% of students should pass.

		Construct and interpret a 99\% confidence interval for the population proportion of students who pass, and use it to assess whether the policy statement is supported at the 99\% confidence level.
	
		\vspace{0.25cm}
		\SubsectionBox{6. Regional Health Insurance Coverage}\vspace{-0.25cm}
		A health ministry needs an estimate of the proportion of all adults in the region who have health insurance to guide funding and policy decisions. It surveys 600 adults, and 312 report having health insurance.

		Construct and interpret 90\%, 95\%, and 98\% confidence intervals for the population proportion of insured adults, and explain how the confidence level affects the precision of the estimate in this policy context.
	
		\vspace{0.25cm}
		\SubsectionBox{7. Food Safety Inspection Compliance}\vspace{-0.25cm}
		A national agency inspects 850 facilities and finds 102 noncompliant. A previous inspection one year earlier found 120 noncompliant facilities out of 900.

		Construct a 99\% confidence interval for the current inspection and compare it with the previous one.

	\end{multicols}

\end{document}
