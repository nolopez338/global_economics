\documentclass[12pt]{article}
\usepackage[a4paper,left=1.2cm,right=1.2cm,top=1.5cm,bottom=1.5cm]{geometry}
\usepackage{graphicx}
\usepackage{xcolor}
\usepackage{array}
\usepackage{tabularx}
\usepackage{amsmath}
\usepackage{multicol}
\usepackage[T1]{fontenc}
\usepackage[utf8]{inputenc}
\setlength{\parindent}{0pt}
\setlength{\tabcolsep}{6pt}
\renewcommand{\arraystretch}{1.15}
\newcolumntype{Y}{>{\raggedright\arraybackslash}m{\dimexpr0.55\textwidth-2\tabcolsep-2\arrayrulewidth\relax}}
\newcolumntype{Z}{>{\raggedright\arraybackslash}m{\dimexpr0.45\textwidth-2\tabcolsep-2\arrayrulewidth\relax}}
\newcolumntype{C}[1]{>{\centering\arraybackslash}m{#1}}
\newcommand{\SubsectionBox}[1]{%
\noindent\colorbox{gray!30}{\parbox{\linewidth}{\textbf{#1}}}\par\vspace{0.35cm}%
}
\newcommand{\CellCenter}[1]{\parbox{\linewidth}{\centering #1}}

\begin{document}

\noindent
\begin{tabularx}{\textwidth}{|C{2.8cm}|C{\dimexpr\textwidth-6cm-4\tabcolsep-4\arrayrulewidth\relax}|C{2.8cm}|}
\hline
\centering\vspace{3mm}\includegraphics[width=2.5cm]{../../preamble/logo.png}&
\CellCenter{\vspace{-5mm}\textbf{GLOBAL ECONOMICS}\par\textbf{GRADE: 11TH}\par\textbf{CATCH-UP}\par\textbf{CRITERION C1}\par\textbf{TEACHER'S NAME: Nicolás López Cuéllar}}&
\CellCenter{\textbf{SECOND TERM}\par\textbf{2025--2026}}\\
\hline
\end{tabularx}

\vspace{0.5cm}
\noindent
\begin{tabular}{|Y|Z|}
\hline
{\footnotesize\textbf{Learning objective:} Compute sample mean and sample standard deviation from sample data in confidence interval planning contexts.}&
{\footnotesize\textbf{Assessment criteria:}\par C1: Compute the sample mean and sample standard deviation.}\\
\hline
\end{tabular}

\begin{multicols}{2}
	\SubsectionBox{Criteria assessment}\vspace{-0.25cm}
	Each assessment criterion is evaluated across the problems in this catch-up exam. A criterion is considered passed when it is correctly activated in 9 of the 10 problems of this activity.

	\vspace{0.25cm}
\SubsectionBox{Problem 1}
\subsection*{Digital Studio Weekly Payouts}
A group of Grade 11 students manages a small digital-product studio that sells logo templates and social media packs. They tracked one random sample of \textbf{14 weekly net payouts} (USD) during a semester pilot to understand whether their business cash flow is stable enough to support reinvestment decisions.

		\textbf{Sample data (USD):}
		\[
		\begin{aligned}
			118,\ 124,\ 131,\ 127,\ 136,\ 142,\ 125,\ 139,\\
			133,\ 121,\ 145,\ 129,\ 137,\ 126
		\end{aligned}
		\]

		Compute $\bar{x}$, $s^2$, and $s$. Then write a short interpretation of what the variability says about weekly income risk for this studio.

\vspace{0.5cm}
\SubsectionBox{Problem 2}
\subsection*{Income Channel Volatility Comparison}
A student entrepreneur receives income from two channels: short-form video monetization and freelance thumbnail design. She collected two independent samples of weekly net income (USD) to decide which channel is more predictable before choosing where to invest more time next month.

		\textbf{Sample A (Video monetization, USD):}
		\[62,\ 75,\ 68,\ 91,\ 57,\ 84,\ 73,\ 95,\ 66,\ 88\]

		\textbf{Sample B (Freelance design, USD):}
		\[54,\ 58,\ 61,\ 63,\ 57,\ 60,\ 56,\ 65,\ 59,\ 62,\ 58\]

		For each sample, compute the mean and sample standard deviation. Based on your results, identify which channel appears more volatile and explain why this matters for planning reliable monthly cash flow.

\vspace{0.5cm}
\SubsectionBox{Problem 3}
\subsection*{Referral Bonus Dispersion Analysis}
A teen fintech app tracks referral bonuses paid to student ambassadors in one week, and management wants to estimate both the average bonus and the spread before adjusting the incentive plan. In this sample, a bonus of 8 USD was paid to 3 students, 12 USD was paid to 5 students, 16 USD was paid to 7 students, 20 USD was paid to 4 students, and 24 USD was paid to 2 students. Treat all of these observations as one sample, compute the sample mean and sample standard deviation, and briefly comment on whether the bonus structure is tightly clustered or widely dispersed.

\vspace{0.5cm}
\SubsectionBox{Problem 4}
\subsection*{Delivery Route Tip Volatility}
A weekend delivery team led by senior students compared tip income per shift for two independent routes to determine which route provides more stable earnings for transport planning. In Sample A for Route A, tip income was 6 USD in 2 shifts, 10 USD in 4 shifts, 14 USD in 6 shifts, 18 USD in 5 shifts, and 22 USD in 3 shifts. In Sample B for Route B, tip income was 5 USD in 4 shifts, 9 USD in 6 shifts, 13 USD in 5 shifts, 17 USD in 3 shifts, and 21 USD in 2 shifts. For each sample, compute the sample mean and sample standard deviation using grouped-data formulas, then compare the two routes in terms of tip-income volatility.

\vspace{0.5cm}
\SubsectionBox{Problem 5}
\subsection*{Subscription Earnings Dispersion}
A student-run study-notes subscription channel grouped weekly net earnings to review volatility before setting savings targets. Over the period observed, 3 weeks fell in the 40--59 USD interval, 6 weeks fell in the 60--79 USD interval, 8 weeks fell in the 80--99 USD interval, 5 weeks fell in the 100--119 USD interval, and 2 weeks fell in the 120--139 USD interval. Using class midpoints, approximate the sample mean and sample standard deviation, and explain what the dispersion implies for setting a realistic weekly savings contribution.

\vspace{0.5cm}
\SubsectionBox{Problem 6}
\subsection*{Portfolio Strategy Stability Comparison}
A youth investment club compared two independent grouped samples of monthly portfolio returns from different micro-investment strategies before allocating new capital. In Sample A for Strategy A, 3 months were in the 20--39 USD interval, 5 months were in the 40--59 USD interval, 7 months were in the 60--79 USD interval, 4 months were in the 80--99 USD interval, and 1 month was in the 100--119 USD interval. In Sample B for Strategy B, 2 months were in the 10--29 USD interval, 4 months were in the 30--49 USD interval, 6 months were in the 50--69 USD interval, 5 months were in the 70--89 USD interval, and 3 months were in the 90--109 USD interval. Using class midpoints, compute the sample mean and sample standard deviation for each strategy, then compare which strategy appears more stable for conservative student investors.

\vspace{0.5cm}
\SubsectionBox{Problem 7}
\subsection*{Planner Sales Outlier Impact}
A student creator who sells printable planners online summarized most weeks as grouped regular-week net sales, but two viral-promotion weeks were clear outliers that must be included in the same sample for risk analysis. In the regular-week data, sales of 70 USD occurred in 4 weeks, 85 USD occurred in 6 weeks, 100 USD occurred in 5 weeks, and 115 USD occurred in 3 weeks, and the additional outlier observations were 185 USD and 210 USD. Using all observations together as one sample, compute $\bar{x}$ and $s$, then explain how the outliers influence measured dispersion and why that matters for next month's budgeting.

\vspace{0.5cm}
\SubsectionBox{Problem 8}
\subsection*{Streamer Income Outlier Impact}
A 17-year-old gaming streamer grouped normal weekly monetization income by intervals, while two sponsorship weeks were recorded as clear outliers, and the full set is used to evaluate volatility before committing to fixed monthly expenses. In the grouped regular weeks, 4 observations were in the 30--49 USD interval, 7 observations were in the 50--69 USD interval, 6 observations were in the 70--89 USD interval, and 3 observations were in the 90--109 USD interval, and the two additional outlier observations were 160 USD and 190 USD. Using class midpoints for the grouped data and including the outliers explicitly, compute the sample mean and sample standard deviation, and interpret what this volatility indicates about the risk of depending on streaming as a primary income source.

\vspace{0.5cm}
\SubsectionBox{Problem 9}
\subsection*{Late Payment Probability Variability}
A student-built homework app tracked the monthly probability of late payment over eight recent months to evaluate consistency in payment behavior. The observed probabilities were 12\%, 18\%, 10\%, 16\%, 14\%, 20\%, 11\%, and 19\%. Treat these probability values as one sample dataset, compute the sample mean and sample standard deviation, and interpret what the spread says about month-to-month payment risk.

\vspace{0.5cm}
\SubsectionBox{Problem 10}
\subsection*{Sale Probability Variability}
A student startup tracked the weekly probability of making at least one sale from social-media outreach over seven weeks. The observed probabilities were 0.35, 0.28, 0.32, 0.30, 0.27, 0.33, and 0.29. Treat these probability values as one sample dataset, compute the sample mean and sample standard deviation, and interpret what this implies about weekly sales-consistency risk.

\end{multicols}

\end{document}
