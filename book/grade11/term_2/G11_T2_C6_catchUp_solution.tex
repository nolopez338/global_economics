\makeatletter
\def\input@path{{./}{../}{../../}{preamble/}{../preamble/}{../../preamble/}}
\makeatother
% ----------------------------------------------------------
% GENERAL 

% File
\documentclass[11pt]{book}

% Margins
\usepackage[margin=1in]{geometry}

\linespread{1.2}            % Line spacing
\usepackage[utf8]{inputenc}
\usepackage[T1]{fontenc}
\usepackage{lmodern}
\usepackage{microtype}
\setlength{\parindent}{0pt}
\setlength{\parskip}{6pt}
\usepackage{booktabs}

% ----------------------------------------------------------
% TABLES
\usepackage{multicol}
\usepackage{longtable} 
\usepackage{array}
\usepackage{booktabs}
\usepackage{tabularx}
\usepackage{multirow}

% ----------------------------------------------------------
% MATHEMATICS
\usepackage{amsmath}
\usepackage{amssymb}
\usepackage{amsfonts}
\usepackage{mathtools}

% ----------------------------------------------------------
% HIDDEN CONTENT
\usepackage{ifthen}
% Define a boolean switch
\newboolean{explicaciones}
% Set the boolean switch to true or false
% Change to true to show the content

% Explanations
\newcommand{\explicacion}[2]{
	\ifthenelse{\boolean{explicaciones}}{#1}{#2}
}
\newcommand{\mostrarExplicaciones}[1]{\setboolean{explicaciones}{#1}}

% ----------------------------------------------------------
% NUMBERING

\usepackage{fancyhdr}
\pagestyle{empty} % Ensures the entire document has no page numbers

\usepackage{tocloft}
\renewcommand{\cftdot}{} % Remove dots for sections, if any
\renewcommand{\cftsecleader}{\cftdotfill{\cftdotsep}} % Remove dots for sections, if any
\cftpagenumbersoff{section} % Removes page numbers from sections
\cftpagenumbersoff{subsection} % Removes page numbers from subsections

% ----------------------------------------------------------
% IMAGES 

% General settings
\usepackage{graphicx}       % Insert images
\usepackage{float}          % Position images
% \usepackage{subfigure}      % Subfigures
\graphicspath{{imgs}}       % Image location
\usepackage{subcaption}     % Subfigures II
\usepackage{verbatim}

% Figures
\usepackage{tikz}
\usetikzlibrary{arrows.meta,positioning,trees}

% Colors
\usepackage{xcolor}     
\definecolor{popUp}{HTML}{666666}
\definecolor{popUpIn}{HTML}{CED9E0}
\definecolor{backgroundC}{HTML}{D0E8F2}
\definecolor{backgroundCC}{HTML}{FFFFFF}
\definecolor{borders}{HTML}{8c120d}
\definecolor{padding}{HTML}{77D0D7}
\definecolor{links}{HTML}{CC6F5F}

% ----------------------------------------------------------
% FRAMES

% General settings
\usepackage{tcolorbox}
\usepackage{adjustbox}          % Adjusted frame  
\setlength{\fboxrule}{3pt}  % Line width
\setlength{\fboxsep}{3pt}   % Box padding

% General frames
\usepackage{mdframed}   

\mdfdefinestyle{estiloGeneral}{    % General style
	linecolor=black,
	linewidth=1.5pt,
	roundcorner=10pt,
	backgroundcolor=backgroundC,
	innerbottommargin=0pt
}
\mdfdefinestyle{code}{          % Code style
	linecolor=black,
	linewidth=1.5pt,
	roundcorner=10pt,
	backgroundcolor=darkgray!10,
	innerbottommargin=0pt
}

% Image frame
\newtcbox{\fboxC}{
	colback=backgroundC,
	colframe=popUp,
	arc=10pt,
	boxrule=3pt,
	boxsep=0pt, % Change the padding here
	nobeforeafter
}

% ----------------------------------------------------------
% PAGE SETTINGS

% Background 
\newcommand{\background}[0]{\begin{tikzpicture}[remember picture,overlay]
		\fill[backgroundC] (-2,2) rectangle (25cm, -550);
\end{tikzpicture}}

\newcommand{\backgroundC}[0]{\begin{tikzpicture}[remember picture,overlay]
		\fill[backgroundCC] (-2,2) rectangle (25cm, -550);
\end{tikzpicture}}

% Page width 
\newcommand{\anchoPag}[0]{20cm}

% ----------------------------------------------------------
% FONT

% General
\usepackage{tgbonum}        % Font
\usepackage{listings}       % Code typesetting
\usepackage[spanish]{babel} % Load Spanish
\selectlanguage{spanish}    % Select Spanish
\usepackage{enumitem}
\usepackage{bookmark}

\setlist[itemize]{leftmargin=1.2em, itemsep=0.35em, topsep=0.35em}

% --- Table helpers ---
\newcolumntype{L}[1]{>{\raggedright\arraybackslash}p{#1}}
\newcolumntype{Y}{>{\raggedright\arraybackslash}X}
\newcolumntype{C}{>{\centering\arraybackslash}X}
\renewcommand{\arraystretch}{1.1}

% Python style
\lstdefinestyle{python}{
	language=Python,
	basicstyle=\ttfamily\small,
	commentstyle=\color{green!50!black},
	keywordstyle=\color{blue},
	numberstyle=\tiny\color{gray},
	numbers=left,
	morekeywords={>, <},
	breakatwhitespace=false,
	showstringspaces=false,
	showtabs=false,
	showspaces=false
}

% ----------------------------------------------------------
% HYPERLINKS

% General
\usepackage{hyperref}       
\hypersetup{
	colorlinks=true,
	linkcolor=links,
	filecolor=magenta,      
	urlcolor=brown,
}

% Custom commands 

% Large link
\newcommand{\bigLink}[2]{\begin{center} \fboxC{\LARGE{\href{#1}{#2}}}\end{center}}

% Small link
\newcommand{\smallLink}[2]{\begin{center}\fboxC{\href{#1}{#2}}\end{center}}

% Bold link
\newcommand{\bfLink}[2]{\href{#1}{\textbf{#2}}}


% Small URL
\newcommand{\smallUrl}[1]{\begin{center}\fboxC{\url{#1}}\end{center}}


% ----------------------------------------------------------
% CUSTOM COMMANDS FOR FIGURES

\newcommand{\espacioImagenes}[0]{-1.2cm}

% Without frame
\newcommand{\fig}[3][\espacioImagenes]{
	\hspace*{#1}
	\centering
	\includegraphics[width=#2\textwidth]{#3}
}

% With frame
\newcommand{\ffig}[2]{\begin{figure}[h]
		\hspace*{\espacioImagenes}
		\centering
		\fbox{\includegraphics[width=#1\textwidth]{#2}}
\end{figure}}

% Hyperlink with frame
\newcommand{\hfig}[3]{\begin{figure}[h]
		\hspace*{-1.4cm}
		\centering
		\color{popUp}
		\fboxC{\href{#1}{\includegraphics[width=#2\textwidth]{#3}}}
	\end{figure}
}

% Hyperlink without frame
\newcommand{\hffig}[3]{\begin{figure}[h]
		\hspace*{-1.1cm}
		\centering
		\color{popUp}
		\href{#1}{\includegraphics[width=#2\textwidth]{#3}}
	\end{figure}
}

% ----------------------------------------------------------

% Start and Contents
\newcommand{\cuadro}[1]{
	\begin{mdframed}[style=estiloGeneral]
		#1 
	\end{mdframed}
}

% Explanation video image
\newcommand{\linkExplicacion}[1]{
	\hffig{#1}{0.5}{principal/videoExplicacion}
	\vspace{-0.5cm}
}

\newcommand{\subSecLink}[2]{
	\subsubsection*{\href{#1}{\textbf{#2}}}
}

% Spacing
\newcommand{\esp}[0]{\vspace{4mm}}

% Back to start
\newcommand{\secInicio}[0]{\begin{center}\hyperref[sec:inicio]{ 
			\includegraphics[width=0.1\textwidth]{principal/up}
	}\end{center}
}


\geometry{margin=0.85in}
\AtBeginDocument{\small}

\newcommand{\ExamNameField}{\noindent\textbf{Name:}\ \rule{0.7\linewidth}{0.4pt}\par\medskip}

\newcommand{\ExamTitleBlock}[3]{%
	\begin{center}
		\Large\textbf{#1}\\
		\textbf{#2}%
		\if\relax\detokenize{#3}\relax\else\\\textbf{#3}\fi
	\end{center}
	\vspace{0.5em}
}

\newcommand{\ExamSection}[1]{\par\medskip\textbf{#1}\par\smallskip}

\newenvironment{ExamCriteria}{%
	\begin{itemize}[leftmargin=1.6em, itemsep=0.3em, topsep=0.2em]
}{%
	\end{itemize}
}

\newenvironment{ExamProblems}{%
	\begin{enumerate}[label=\textbf{P\arabic*}, leftmargin=0pt, labelsep=0.6em, itemindent=2.2em, itemsep=0.8em]
}{%
	\end{enumerate}
}

\begin{document}
	\ExamTitleBlock{11th grade}{Term 2 Catch-Up: C6 Required Sample Size (Solutions)}{}

	\ExamSection{Evaluated criteria}
	\begin{ExamCriteria}
		\item C6: Determine the required sample size to achieve a target margin of error.
	\end{ExamCriteria}

	\ExamSection{Problems}
	\begin{ExamProblems}
		\item
		\subsection*{Weekly Spending Sample Size}
		A digital-payment study wants to estimate mean weekly spending by first-job workers. Long-run records show $\sigma = 96$ USD. The team wants 95\% confidence with target margin of error $E = 12$ USD. Determine the minimum required sample size.

		\subsection*{C6}
		Step 1: Identify the target error and confidence level. Here $E = 12$ and 95\% confidence gives $z^* = 1.96$.

		Step 2: Substitute into the required-sample-size formula.
		\[
		n = \left(\frac{z^*\sigma}{E}\right)^2
		= \left(\frac{1.96 \cdot 96}{12}\right)^2
		= \left(15.68\right)^2
		= 245.8624
		\]

		Step 3: Round up to keep the error at or below target. The minimum sample size is $n = 246$. This verifies Criterion 6 because the rounded sample size satisfies the requested margin of error at the stated confidence level.

		\item
		\subsection*{Utilities Spending Sample Size}
		A youth housing survey estimates mean monthly utilities spending. The long-run standard deviation is known as $\sigma = 54$ USD. The planner requires 90\% confidence and margin of error $E = 6$ USD. Determine the minimum required sample size.

		\subsection*{C6}
		Step 1: Identify inputs. The target margin is $E = 6$, and for 90\% confidence, $z^* = 1.645$.

		Step 2: Apply the formula.
		\[
		n = \left(\frac{z^*\sigma}{E}\right)^2
		= \left(\frac{1.645 \cdot 54}{6}\right)^2
		= \left(14.805\right)^2
		= 219.1880
		\]

		Step 3: Round up. The minimum required sample size is $n = 220$. This verifies Criterion 6 because the computed sample size is rounded upward to guarantee the target precision.

		\item
		\subsection*{Designer Earnings Sample Planning}
		A freelance-income team wants the mean weekly platform earnings for young designers. Long-run variability is $\sigma = 125$ USD. The team wants a 95\% confidence estimate with margin of error $E = 20$ USD. Extra context says there are 9{,}800 active designers and last week\'s median earning was 410 USD. Determine the minimum required sample size.

		\subsection*{C6}
		Step 1: Identify only quantities needed for C6. Use $\sigma = 125$, $E = 20$, and for 95\% confidence $z^* = 1.96$. The active-user count and median are irrelevant to this formula.

		Step 2: Compute the required sample size.
		\[
		n = \left(\frac{z^*\sigma}{E}\right)^2
		= \left(\frac{1.96 \cdot 125}{20}\right)^2
		= \left(12.25\right)^2
		= 150.0625
		\]

		Step 3: Round up. The minimum sample size is $n = 151$. This verifies Criterion 6 because the solution uses the correct planning inputs and upward rounding.

		\item
		\subsection*{Ride-Share Interval Width Planning}
		A transportation budget survey estimates mean monthly ride-share spending by university graduates. The known long-run spread is $\sigma = 72$ USD. The required confidence interval width is $W = 18$ USD at 95\% confidence. Determine the minimum required sample size.

		\subsection*{C6}
		Step 1: Convert width to margin of error and identify $z^*$. Since $W = 2E$,
		\[
		E = \frac{W}{2} = \frac{18}{2} = 9
		\]
		For 95\% confidence, $z^* = 1.96$.

		Step 2: Apply the formula with the converted margin.
		\[
		n = \left(\frac{z^*\sigma}{E}\right)^2
		= \left(\frac{1.96 \cdot 72}{9}\right)^2
		= \left(15.68\right)^2
		= 245.8624
		\]

		Step 3: Round up. The minimum sample size is $n = 246$. This verifies Criterion 6 because width was correctly converted to margin of error before sample-size planning.

		\item
		\subsection*{Confidence Level Margin Tradeoff}
		An e-commerce savings study tracks mean monthly discount use by young adults. Long-run standard deviation is $\sigma = 84$ USD. The team wants the same target margin $E = 10$ USD under two confidence levels: 90\% and 95\%. Compute both required sample sizes and compare.

		\subsection*{C6}
		Step 1: Identify shared and changing quantities. Here $E = 10$ and $\sigma = 84$ are fixed. Use $z^* = 1.645$ for 90\% and $z^* = 1.96$ for 95\% confidence.

		Step 2: Compute each required sample size.
		\[
		n_{90} = \left(\frac{1.645 \cdot 84}{10}\right)^2
		= \left(13.818\right)^2
		= 190.9371
		\]
		\[
		n_{95} = \left(\frac{1.96 \cdot 84}{10}\right)^2
		= \left(16.464\right)^2
		= 271.0644
		\]

		Step 3: Round up and compare. The minimum sample sizes are $n_{90} = 191$ and $n_{95} = 272$, so 95\% confidence needs more observations. This verifies Criterion 6 because both confidence-level plans are computed and checked with correct upward rounding.

		\item
		\subsection*{Delivery Cost Precision Upgrade}
		A food-delivery cost survey already has $n_{\text{current}} = 140$ users and currently reports margin of error $E_{\text{current}} = 11$ USD at 95\% confidence. The organization now wants margin of error $E_{\text{new}} = 8$ USD at the same confidence level. Determine the required total sample size and additional observations needed.

		\subsection*{C6}
		Step 1: Identify the target setup. The desired margin is $E_{\text{new}} = 8$ at 95\% confidence, so $z^* = 1.96$. Use current information to recover $\sigma$:
		\[
		\sigma = \frac{E_{\text{current}}\sqrt{n_{\text{current}}}}{z^*}
		= \frac{11\sqrt{140}}{1.96}
		\approx 66.4664
		\]

		Step 2: Compute required total sample size.
		\[
		n_{\text{required}} = \left(\frac{z^*\sigma}{E_{\text{new}}}\right)^2
		= \left(\frac{1.96 \cdot 66.4664}{8}\right)^2
		\approx 265.2344
		\]
		Round up: $n_{\text{required}} = 266$.

		Step 3: Compute additional observations.
		\[
		n_{\text{additional}} = n_{\text{required}} - n_{\text{current}} = 266 - 140 = 126
		\]
		The team needs 126 additional users. This verifies Criterion 6 because the target margin is translated into a rounded required total and then into added observations.

		\item
		\subsection*{Subscription Margin Reduction Plan}
		A streaming-subscription survey has $n_{\text{current}} = 121$ and current margin of error $E_{\text{current}} = 9$ USD at 95\% confidence. Management requests a 20\% reduction in margin of error. Find the new target margin, required total sample size, and additional observations.

		\subsection*{C6}
		Step 1: Convert the percentage reduction into the new margin.
		\[
		E_{\text{new}} = (1-r)E_{\text{current}} = (1-0.20)(9) = 7.2
		\]
		At 95\% confidence, $z^* = 1.96$, and recover $\sigma$ from the current setup:
		\[
		\sigma = \frac{E_{\text{current}}\sqrt{n_{\text{current}}}}{z^*}
		= \frac{9\sqrt{121}}{1.96}
		= \frac{99}{1.96}
		\approx 50.5102
		\]

		Step 2: Compute required total sample size.
		\[
		n_{\text{required}} = \left(\frac{1.96 \cdot 50.5102}{7.2}\right)^2
		= \left(13.75\right)^2
		= 189.0625
		\]
		Round up: $n_{\text{required}} = 190$.

		Step 3: Compute additional observations.
		\[
		n_{\text{additional}} = 190 - 121 = 69
		\]
		So 69 additional observations are required. This verifies Criterion 6 because the percentage reduction is correctly converted to a new error target and then to the minimum integer sample size.

		\item
		\subsection*{Mobile Banking Precision Stages}
		A mobile-banking usage survey currently has $n_{\text{current}} = 160$ and margin of error $E_{\text{current}} = 14$ USD at 90\% confidence. The policy team wants two staged improvements: first a 10\% reduction, then a 30\% reduction from the current margin. Determine both required total sample sizes and additional observations.

		\subsection*{C6}
		Step 1: Compute both target margins using $E_{\text{new}}=(1-r)E_{\text{current}}$.
		\[
		E_{10\%} = (1-0.10)(14) = 12.6
		\]
		\[
		E_{30\%} = (1-0.30)(14) = 9.8
		\]
		At 90\% confidence, $z^* = 1.645$. Recover $\sigma$:
		\[
		\sigma = \frac{E_{\text{current}}\sqrt{n_{\text{current}}}}{z^*}
		= \frac{14\sqrt{160}}{1.645}
		\approx 107.5830
		\]

		Step 2: Compute required sample sizes for both targets.
		\[
		n_{10\%} = \left(\frac{1.645 \cdot 107.5830}{12.6}\right)^2
		\approx 197.5309 \Rightarrow 198
		\]
		\[
		n_{30\%} = \left(\frac{1.645 \cdot 107.5830}{9.8}\right)^2
		\approx 326.5306 \Rightarrow 327
		\]

		Step 3: Compute additional observations and compare.
		\[
		n_{\text{add},10\%} = 198 - 160 = 38
		\]
		\[
		n_{\text{add},30\%} = 327 - 160 = 167
		\]
		The stronger reduction needs many more added observations. This verifies Criterion 6 because both reduction plans are solved by explicit z-based sample-size calculations.

		\item
		\subsection*{Nonessential Spending Sample Upgrade}
		A personal-finance app study has current sample size $n_{\text{current}} = 100$ users. Long-run records indicate $\sigma = 75$ USD for monthly nonessential spending. The confidence level stays at 95\%. The team wants to reduce the current margin of error by 20\%. First recover the current margin of error, then find the new required total sample size and additional observations.

		\subsection*{C6}
		Step 1: Recover the current margin and then the new target margin.
		\[
		E_{\text{current}} = z^*\frac{\sigma}{\sqrt{n_{\text{current}}}}
		= 1.96\frac{75}{\sqrt{100}}
		= 1.96\cdot 7.5
		= 14.7
		\]
		Apply a 20\% reduction:
		\[
		E_{\text{new}} = (1-0.20)E_{\text{current}} = 0.80(14.7) = 11.76
		\]

		Step 2: Compute required total sample size.
		\[
		n_{\text{required}} = \left(\frac{z^*\sigma}{E_{\text{new}}}\right)^2
		= \left(\frac{1.96\cdot 75}{11.76}\right)^2
		= \left(12.5\right)^2
		= 156.25
		\]
		Round up: $n_{\text{required}} = 157$.

		Step 3: Compute additional observations.
		\[
		n_{\text{additional}} = 157 - 100 = 57
		\]
		So 57 additional users are required. This verifies Criterion 6 because the missing current margin is recovered first and then used for full planning.

		\item
		\subsection*{Side-Income Precision Scenarios}
		A job-market survey currently has $n_{\text{current}} = 144$ respondents, and long-run salary-spread data give $\sigma = 120$ USD for weekly side-income. Confidence level is 90\%. The analysts are considering two reductions from the current margin of error: 25\% and 40\%. Recover the current margin first, then compute both required sample sizes and compare.

		\subsection*{C6}
		Step 1: Recover current margin, then compute both targets. For 90\% confidence, $z^* = 1.645$.
		\[
		E_{\text{current}} = 1.645\frac{120}{\sqrt{144}}
		= 1.645\cdot 10
		= 16.45
		\]
		\[
		E_{25\%} = (1-0.25)(16.45) = 12.3375
		\]
		\[
		E_{40\%} = (1-0.40)(16.45) = 9.87
		\]

		Step 2: Compute required sample sizes for each reduction target.
		\[
		n_{25\%} = \left(\frac{1.645\cdot 120}{12.3375}\right)^2
		= \left(16\right)^2
		= 256
		\]
		\[
		n_{40\%} = \left(\frac{1.645\cdot 120}{9.87}\right)^2
		= \left(20\right)^2
		= 400
		\]

		Step 3: Compare required sample sizes and additional observations.
		\[
		n_{\text{add},25\%} = 256 - 144 = 112
		\]
		\[
		n_{\text{add},40\%} = 400 - 144 = 256
		\]
		The 40\% reduction requires a much larger increase. This verifies Criterion 6 because both targets are solved after recovering the previously unknown current margin.

		\item
		\subsection*{Dual Study Precision Comparison}
		Two independent youth-economy studies use 95\% confidence and want the same 30\% margin-of-error reduction.
		Study A: $n_{A,\text{current}} = 81$, $\sigma_A = 90$ USD.
		Study B: $n_{B,\text{current}} = 196$, $\sigma_B = 140$ USD.
		For each study, recover the current margin first, then find the required total sample size and additional observations. Compare which study needs fewer additional observations.

		\subsection*{C6}
		Step 1: Recover current margins and compute new targets. For 95\% confidence, $z^* = 1.96$.
		\[
		E_{A,\text{current}} = 1.96\frac{90}{\sqrt{81}} = 1.96\cdot 10 = 19.6
		\]
		\[
		E_{B,\text{current}} = 1.96\frac{140}{\sqrt{196}} = 1.96\cdot 10 = 19.6
		\]
		Apply the 30\% reduction to each:
		\[
		E_{A,\text{new}} = 0.70(19.6) = 13.72
		\]
		\[
		E_{B,\text{new}} = 0.70(19.6) = 13.72
		\]

		Step 2: Compute required sample sizes.
		\[
		n_{A,\text{required}} = \left(\frac{1.96\cdot 90}{13.72}\right)^2
		= \left(12.8571\right)^2
		\approx 165.3061 \Rightarrow 166
		\]
		\[
		n_{B,\text{required}} = \left(\frac{1.96\cdot 140}{13.72}\right)^2
		= \left(20\right)^2
		= 400
		\]

		Step 3: Compute additions and compare.
		\[
		n_{A,\text{add}} = 166 - 81 = 85
		\]
		\[
		n_{B,\text{add}} = 400 - 196 = 204
		\]
		Study A needs fewer additional observations. This verifies Criterion 6 because each study recovers the current margin, then applies the same reduction and complete sample-size planning.

		\item
		\subsection*{Startup Cost Margin Targets}
		An entrepreneurship-finance study currently has $n_{\text{current}} = 225$ records, with known long-run standard deviation $\sigma = 180$ USD for monthly startup operating costs. Confidence level is 95\%. The team wants to evaluate three reductions from the current margin of error: 15\%, 30\%, and 50\%. Recover the current margin first, then compute all new target margins, required total sample sizes, and additional observations.

		\subsection*{C6}
		Step 1: Recover the current margin and compute all new targets. For 95\% confidence, $z^* = 1.96$.
		\[
		E_{\text{current}} = 1.96\frac{180}{\sqrt{225}} = 1.96\cdot 12 = 23.52
		\]
		\[
		E_{15\%} = (1-0.15)(23.52) = 19.992
		\]
		\[
		E_{30\%} = (1-0.30)(23.52) = 16.464
		\]
		\[
		E_{50\%} = (1-0.50)(23.52) = 11.76
		\]

		Step 2: Compute required total sample sizes for each target margin.
		\[
		n_{15\%} = \left(\frac{1.96\cdot 180}{19.992}\right)^2
		= \left(17.6471\right)^2
		\approx 311.4187 \Rightarrow 312
		\]
		\[
		n_{30\%} = \left(\frac{1.96\cdot 180}{16.464}\right)^2
		= \left(21.4286\right)^2
		\approx 459.1840 \Rightarrow 460
		\]
		\[
		n_{50\%} = \left(\frac{1.96\cdot 180}{11.76}\right)^2
		= \left(30\right)^2
		= 900
		\]

		Step 3: Compute additional observations and explain the inverse-square effect.
		\[
		n_{\text{add},15\%} = 312 - 225 = 87
		\]
		\[
		n_{\text{add},30\%} = 460 - 225 = 235
		\]
		\[
		n_{\text{add},50\%} = 900 - 225 = 675
		\]
		Additional observations required are 87, 235, and 675. Conceptually, since $n = \left(\frac{z^*\sigma}{E}\right)^2$, sample size is proportional to $\frac{1}{E^2}$, so halving the margin of error requires about four times as many observations. This verifies Criterion 6 because the current margin is recovered and all three advanced planning targets are fully solved.

	\end{ExamProblems}
\end{document}
