\makeatletter
\def\input@path{{./}{../}{../../}{preamble/}{../preamble/}{../../preamble/}}
\makeatother
% ----------------------------------------------------------
% GENERAL 

% File
\documentclass[11pt]{book}

% Margins
\usepackage[margin=1in]{geometry}

\linespread{1.2}            % Line spacing
\usepackage[utf8]{inputenc}
\usepackage[T1]{fontenc}
\usepackage{lmodern}
\usepackage{microtype}
\setlength{\parindent}{0pt}
\setlength{\parskip}{6pt}
\usepackage{booktabs}

% ----------------------------------------------------------
% TABLES
\usepackage{multicol}
\usepackage{longtable} 
\usepackage{array}
\usepackage{booktabs}
\usepackage{tabularx}
\usepackage{multirow}

% ----------------------------------------------------------
% MATHEMATICS
\usepackage{amsmath}
\usepackage{amssymb}
\usepackage{amsfonts}
\usepackage{mathtools}

% ----------------------------------------------------------
% HIDDEN CONTENT
\usepackage{ifthen}
% Define a boolean switch
\newboolean{explicaciones}
% Set the boolean switch to true or false
% Change to true to show the content

% Explanations
\newcommand{\explicacion}[2]{
	\ifthenelse{\boolean{explicaciones}}{#1}{#2}
}
\newcommand{\mostrarExplicaciones}[1]{\setboolean{explicaciones}{#1}}

% ----------------------------------------------------------
% NUMBERING

\usepackage{fancyhdr}
\pagestyle{empty} % Ensures the entire document has no page numbers

\usepackage{tocloft}
\renewcommand{\cftdot}{} % Remove dots for sections, if any
\renewcommand{\cftsecleader}{\cftdotfill{\cftdotsep}} % Remove dots for sections, if any
\cftpagenumbersoff{section} % Removes page numbers from sections
\cftpagenumbersoff{subsection} % Removes page numbers from subsections

% ----------------------------------------------------------
% IMAGES 

% General settings
\usepackage{graphicx}       % Insert images
\usepackage{float}          % Position images
% \usepackage{subfigure}      % Subfigures
\graphicspath{{imgs}}       % Image location
\usepackage{subcaption}     % Subfigures II
\usepackage{verbatim}

% Figures
\usepackage{tikz}
\usetikzlibrary{arrows.meta,positioning,trees}

% Colors
\usepackage{xcolor}     
\definecolor{popUp}{HTML}{666666}
\definecolor{popUpIn}{HTML}{CED9E0}
\definecolor{backgroundC}{HTML}{D0E8F2}
\definecolor{backgroundCC}{HTML}{FFFFFF}
\definecolor{borders}{HTML}{8c120d}
\definecolor{padding}{HTML}{77D0D7}
\definecolor{links}{HTML}{CC6F5F}

% ----------------------------------------------------------
% FRAMES

% General settings
\usepackage{tcolorbox}
\usepackage{adjustbox}          % Adjusted frame  
\setlength{\fboxrule}{3pt}  % Line width
\setlength{\fboxsep}{3pt}   % Box padding

% General frames
\usepackage{mdframed}   

\mdfdefinestyle{estiloGeneral}{    % General style
	linecolor=black,
	linewidth=1.5pt,
	roundcorner=10pt,
	backgroundcolor=backgroundC,
	innerbottommargin=0pt
}
\mdfdefinestyle{code}{          % Code style
	linecolor=black,
	linewidth=1.5pt,
	roundcorner=10pt,
	backgroundcolor=darkgray!10,
	innerbottommargin=0pt
}

% Image frame
\newtcbox{\fboxC}{
	colback=backgroundC,
	colframe=popUp,
	arc=10pt,
	boxrule=3pt,
	boxsep=0pt, % Change the padding here
	nobeforeafter
}

% ----------------------------------------------------------
% PAGE SETTINGS

% Background 
\newcommand{\background}[0]{\begin{tikzpicture}[remember picture,overlay]
		\fill[backgroundC] (-2,2) rectangle (25cm, -550);
\end{tikzpicture}}

\newcommand{\backgroundC}[0]{\begin{tikzpicture}[remember picture,overlay]
		\fill[backgroundCC] (-2,2) rectangle (25cm, -550);
\end{tikzpicture}}

% Page width 
\newcommand{\anchoPag}[0]{20cm}

% ----------------------------------------------------------
% FONT

% General
\usepackage{tgbonum}        % Font
\usepackage{listings}       % Code typesetting
\usepackage[spanish]{babel} % Load Spanish
\selectlanguage{spanish}    % Select Spanish
\usepackage{enumitem}
\usepackage{bookmark}

\setlist[itemize]{leftmargin=1.2em, itemsep=0.35em, topsep=0.35em}

% --- Table helpers ---
\newcolumntype{L}[1]{>{\raggedright\arraybackslash}p{#1}}
\newcolumntype{Y}{>{\raggedright\arraybackslash}X}
\newcolumntype{C}{>{\centering\arraybackslash}X}
\renewcommand{\arraystretch}{1.1}

% Python style
\lstdefinestyle{python}{
	language=Python,
	basicstyle=\ttfamily\small,
	commentstyle=\color{green!50!black},
	keywordstyle=\color{blue},
	numberstyle=\tiny\color{gray},
	numbers=left,
	morekeywords={>, <},
	breakatwhitespace=false,
	showstringspaces=false,
	showtabs=false,
	showspaces=false
}

% ----------------------------------------------------------
% HYPERLINKS

% General
\usepackage{hyperref}       
\hypersetup{
	colorlinks=true,
	linkcolor=links,
	filecolor=magenta,      
	urlcolor=brown,
}

% Custom commands 

% Large link
\newcommand{\bigLink}[2]{\begin{center} \fboxC{\LARGE{\href{#1}{#2}}}\end{center}}

% Small link
\newcommand{\smallLink}[2]{\begin{center}\fboxC{\href{#1}{#2}}\end{center}}

% Bold link
\newcommand{\bfLink}[2]{\href{#1}{\textbf{#2}}}


% Small URL
\newcommand{\smallUrl}[1]{\begin{center}\fboxC{\url{#1}}\end{center}}


% ----------------------------------------------------------
% CUSTOM COMMANDS FOR FIGURES

\newcommand{\espacioImagenes}[0]{-1.2cm}

% Without frame
\newcommand{\fig}[3][\espacioImagenes]{
	\hspace*{#1}
	\centering
	\includegraphics[width=#2\textwidth]{#3}
}

% With frame
\newcommand{\ffig}[2]{\begin{figure}[h]
		\hspace*{\espacioImagenes}
		\centering
		\fbox{\includegraphics[width=#1\textwidth]{#2}}
\end{figure}}

% Hyperlink with frame
\newcommand{\hfig}[3]{\begin{figure}[h]
		\hspace*{-1.4cm}
		\centering
		\color{popUp}
		\fboxC{\href{#1}{\includegraphics[width=#2\textwidth]{#3}}}
	\end{figure}
}

% Hyperlink without frame
\newcommand{\hffig}[3]{\begin{figure}[h]
		\hspace*{-1.1cm}
		\centering
		\color{popUp}
		\href{#1}{\includegraphics[width=#2\textwidth]{#3}}
	\end{figure}
}

% ----------------------------------------------------------

% Start and Contents
\newcommand{\cuadro}[1]{
	\begin{mdframed}[style=estiloGeneral]
		#1 
	\end{mdframed}
}

% Explanation video image
\newcommand{\linkExplicacion}[1]{
	\hffig{#1}{0.5}{principal/videoExplicacion}
	\vspace{-0.5cm}
}

\newcommand{\subSecLink}[2]{
	\subsubsection*{\href{#1}{\textbf{#2}}}
}

% Spacing
\newcommand{\esp}[0]{\vspace{4mm}}

% Back to start
\newcommand{\secInicio}[0]{\begin{center}\hyperref[sec:inicio]{ 
			\includegraphics[width=0.1\textwidth]{principal/up}
	}\end{center}
}


\geometry{margin=0.85in}
\AtBeginDocument{\small}

\newcommand{\ExamNameField}{\noindent\textbf{Name:}\ \rule{0.7\linewidth}{0.4pt}\par\medskip}

\newcommand{\ExamTitleBlock}[3]{%
	\begin{center}
		\Large\textbf{#1}\\
		\textbf{#2}%
		\if\relax\detokenize{#3}\relax\else\\\textbf{#3}\fi
	\end{center}
	\vspace{0.5em}
}

\newcommand{\ExamSection}[1]{\par\medskip\textbf{#1}\par\smallskip}

\newenvironment{ExamCriteria}{%
	\begin{itemize}[leftmargin=1.6em, itemsep=0.3em, topsep=0.2em]
}{%
	\end{itemize}
}

\newenvironment{ExamProblems}{%
	\begin{enumerate}[label=\textbf{P\arabic*}, leftmargin=0pt, labelsep=0.6em, itemindent=2.2em, itemsep=0.8em]
}{%
	\end{enumerate}
}

\begin{document}
	\ExamSection{C9: Confidence interval for the mean when variance is unknown}
	\begin{ExamProblems}
		\item
		\subsection*{Problem description}
		A ride-sharing platform studies daily gross driver payouts (hundreds of USD) in one urban zone.
		A random sample of $n=10$ days gives the raw data:
		\[
		42,\ 38,\ 45,\ 40,\ 41,\ 39,\ 44,\ 43,\ 37,\ 41
		\]
		1) Construct a 95\% confidence interval for the population mean using the t-distribution.
		2) Using the interval and the computed statistics, apply the structured C9 decision procedure to compare the t-based and z-based intervals and determine which interval is wider and by what percentage.

		\subsubsection*{C8: Confidence Interval (t-statistic)}
		\[
		\mu=\text{population mean daily loan disbursement},\quad n=10,\quad \bar{x}=\frac{410}{10}=41.00
		\]
		\[
		\sum (x_i-\bar{x})^2=60,\quad s=\sqrt{\frac{60}{9}}=2.582,\quad df=n-1=9
		\]
		\[
		t^*=t_{0.025,9}\approx 2.262,\quad SE=\frac{s}{\sqrt{n}}=\frac{2.582}{\sqrt{10}}=0.816
		\]
		\[
		E_t=t^*(SE)=2.262(0.816)=1.85
		\]
		\[
		\bar{x}\pm t^*\left(\frac{s}{\sqrt{n}}\right)=41.00\pm 2.262\left(\frac{2.582}{\sqrt{10}}\right)=41.00\pm 1.85
		\]
		\[
		\mu\in[39.15,\,42.85]
		\]
		Interpretation: With 95\% confidence, the zone's mean daily driver payout is between 39.15 and 42.85 hundred USD.

		\subsubsection*{C5: Confidence Interval (z-statistic)}
		\[
		z^*=z_{0.025}=1.960
		\]

		\[
		E_z=z^*(SE)=1.960(0.816)=1.60
		\]

		\[
		\mu\in 41.00\pm E_z=41.00\pm 1.60=[39.40,\,42.60]
		\]

		\subsubsection*{C9: Structured Decision Procedure}
		\[
		t^*=2.262>1.960=z^*
		\]
		\[
		E_t=1.85,\quad E_z=1.60
		\]
		\[
		\text{Width}_t=2(1.85)=3.70,\qquad \text{Width}_z=2(1.60)=3.20
		\]
		\[
		\text{Absolute difference}=3.70-3.20=0.50
		\]
		\[
		\text{Relative difference (base }z\text{-interval)}=\frac{3.70-3.20}{3.20}\times 100=15.63\%
		\]
		Because $df=9$, the t critical value is noticeably larger; therefore the t-interval is wider.
		Final decision: the t-interval is the appropriate exact interval under unknown variance and is 15.63\% wider than the z-interval.

		\newpage
		\item
		\subsection*{Problem description}
		An investment app analyzes mean per-trade platform fees (USD) paid by users in their 20s.
		A random sample of $n=18$ transactions gives the raw data:
		\[
		1.8,\ 2.1,\ 1.9,\ 2.4,\ 2.0,\ 1.7,\ 2.2,\ 2.3,\ 1.8,\ 2.5,\ 2.1,\ 1.9,\ 2.0,\ 2.2,\ 1.6,\ 2.4,\ 2.3,\ 1.9
		\]
		1) Construct a 90\% confidence interval for the population mean using the t-distribution.
		2) Using the interval and the computed statistics, apply the structured C9 decision procedure to compare the t-based and z-based intervals and determine which interval is wider and by what percentage.

		\subsubsection*{C8: Confidence Interval (t-statistic)}
		\[
		\mu=\text{population mean transaction fee},\quad n=18,\quad \bar{x}=\frac{37.1}{18}=2.061
		\]
		\[
		\sum (x_i-\bar{x})^2=1.143,\quad s=\sqrt{\frac{1.143}{17}}=0.259,\quad df=17
		\]
		\[
		t^*=t_{0.05,17}\approx 1.740,\quad SE=\frac{0.259}{\sqrt{18}}=0.0611
		\]

		\[
		E_t=t^*(SE)=1.740(0.0611)=0.106
		\]
		\[
		\bar{x}\pm t^*\left(\frac{s}{\sqrt{n}}\right)=2.061\pm 1.740\left(\frac{0.259}{\sqrt{18}}\right)=2.061\pm 0.106
		\]
		\[
		\mu\in[1.955,\,2.167]
		\]
		Interpretation: With 90\% confidence, the app's mean per-trade fee is between 1.955 and 2.167 USD.

		\subsubsection*{C5: Confidence Interval (z-statistic)}
		\[
		z^*=z_{0.05}=1.645
		\]

		\[
		E_z=z^*(SE)=1.645(0.0611)=0.101
		\]

		\[
		\mu\in 2.061\pm E_z=2.061\pm 0.101=[1.960,\,2.162]
		\]

		\subsubsection*{C9: Structured Decision Procedure}
		\[
		t^*=1.740>1.645=z^*
		\]
		\[
		E_t=0.106,\quad E_z=0.101
		\]
		\[
		\text{Width}_t=0.212,\qquad \text{Width}_z=0.202
		\]
		\[
		\text{Absolute difference}=0.212-0.202=0.010
		\]
		\[
		\text{Relative difference (base }z\text{-interval)}=\frac{0.212-0.202}{0.202}\times 100=4.95\%
		\]
		At $df=17$, t remains heavier-tailed than z, so the t-interval is wider.
		Final decision: use the t-interval as exact under unknown variance; it is 4.95\% wider than the z-interval.

		\newpage
		\item
		\subsection*{Problem description}
		A food-delivery app studies mean order value (USD) during weekday mornings for young professionals.
		A random sample of $n=30$ receipts gives the raw data:
		\[
		115,\ 122,\ 118,\ 130,\ 126,\ 119,\ 121,\ 124,\ 128,\ 117,
		123,\ 120,\ 125,\ 129,\ 116,\ 127,\ 122,\]
		\[	 118,\ 124,\ 121,
		126,\ 119,\ 123,\ 128,\ 120,\ 125,\ 117,\ 124,\ 122,\ 126
		\]
		1) Construct a 99\% confidence interval for the population mean using the t-distribution.
		2) Using the interval and the computed statistics, apply the structured C9 decision procedure to compare the t-based and z-based intervals and determine which interval is wider and by what percentage.

		\subsubsection*{C8: Confidence Interval (t-statistic)}
		\[
		\mu=\text{population mean basket value},\quad n=30,\quad \bar{x}=\frac{3675}{30}=122.50
		\]
		\[
		\sum (x_i-\bar{x})^2=477.5,\quad s=\sqrt{\frac{477.5}{29}}=4.058,\quad df=29
		\]
		\[
		t^*=t_{0.005,29}\approx 2.756,\quad SE=\frac{4.058}{\sqrt{30}}=0.741
		\]

		\[
		E_t=t^*(SE)=2.756(0.741)=2.04
		\]
		\[
		\bar{x}\pm t^*\left(\frac{s}{\sqrt{n}}\right)=122.50\pm 2.756\left(\frac{4.058}{\sqrt{30}}\right)=122.50\pm 2.04
		\]
		\[
		\mu\in[120.46,\,124.54]
		\]
		Interpretation: With 99\% confidence, the platform's mean weekday-morning order value is between 120.46 and 124.54 USD.

		\subsubsection*{C5: Confidence Interval (z-statistic)}
		\[
		z^*=z_{0.005}=2.576
		\]

		\[
		E_z=z^*(SE)=2.576(0.741)=1.91
		\]

		\[
		\mu\in 122.50\pm E_z=122.50\pm 1.91=[120.59,\,124.41]
		\]

		\subsubsection*{C9: Structured Decision Procedure}
		\[
		t^*=2.756>2.576=z^*
		\]
		\[
		E_t=2.04,\quad E_z=1.91
		\]
		\[
		\text{Width}_t=4.08,\qquad \text{Width}_z=3.82
		\]
		\[
		\text{Absolute difference}=4.08-3.82=0.26
		\]
		\[
		\text{Relative difference (base }z\text{-interval)}=\frac{4.08-3.82}{3.82}\times 100=6.81\%
		\]
		At moderate $n$, t and z are closer, but t is still wider and remains the exact method.
		Final decision: the t-interval is selected and is 6.81\% wider than the z-interval.

		\newpage
		\item
		\subsection*{Problem description}
		A gig-delivery startup studies average delivery time (minutes) for same-city orders.
		For $n=60$ orders, data are grouped as follows:
		\[
		\begin{array}{c|c|c}
		\text{Class (minutes)} & \text{Midpoint }m_i & f_i\\\hline
		20\text{--}24 & 22 & 6\\
		25\text{--}29 & 27 & 9\\
		30\text{--}34 & 32 & 14\\
		35\text{--}39 & 37 & 13\\
		40\text{--}44 & 42 & 10\\
		45\text{--}49 & 47 & 8
		\end{array}
		\]
		1) Construct a 95\% confidence interval for the population mean using the t-distribution from the grouped-data estimates.
		2) Using the interval and the computed statistics, apply the structured C9 decision procedure to compare the t-based and z-based intervals and determine which interval is wider and by what percentage.

		\subsubsection*{C8: Confidence Interval (t-statistic)}
		\[
		\mu=\text{population mean delivery time (minutes)},\quad n=\sum f_i=60,\quad \sum f_im_i=2100,\quad \sum f_im_i^2=77910
		\]
		\[
		\bar{x}\approx\frac{\sum f_im_i}{n}=\frac{2100}{60}=35.00
		\]
		\[
		s\approx\sqrt{\frac{\sum f_im_i^2-n\bar{x}^2}{n-1}}=\sqrt{\frac{77910-60(35)^2}{59}}=7.602,\quad df=59
		\]
		\[
		t^*=t_{0.025,59}\approx 2.000,\quad SE=\frac{7.602}{\sqrt{60}}=0.981
		\]

		\[
		E_t=t^*(SE)=2.000(0.981)=1.96
		\]
		\[
		\bar{x}\pm t^*\left(\frac{s}{\sqrt{n}}\right)=35.00\pm 2.000\left(\frac{7.602}{\sqrt{60}}\right)=35.00\pm 1.96
		\]
		\[
		\mu\in[33.04,\,36.96]
		\]
		Interpretation: With 95\% confidence, mean same-city delivery time is between 33.04 and 36.96 minutes.

		\subsubsection*{C5: Confidence Interval (z-statistic)}
		\[
		z^*=z_{0.025}=1.960
		\]

		\[
		E_z=z^*(SE)=1.960(0.981)=1.92
		\]

		\[
		\mu\in 35.00\pm E_z=35.00\pm 1.92=[33.08,\,36.92]
		\]

		\subsubsection*{C9: Structured Decision Procedure}
		\[
		t^*=2.000>1.960=z^*
		\]
		\[
		E_t=1.96,\quad E_z=1.92
		\]
		\[
		\text{Width}_t=3.92,\qquad \text{Width}_z=3.84
		\]
		\[
		\text{Absolute difference}=3.92-3.84=0.08
		\]
		\[
		\text{Relative difference (base }z\text{-interval)}=\frac{3.92-3.84}{3.84}\times 100=2.08\%
		\]
		With $df=59$, convergence is visible, but the t-interval remains slightly wider.
		Final decision: retain the t-interval as exact under unknown variance; it is 2.08\% wider than the z-interval.

		\newpage
		\item
		\subsection*{Problem description}
		A direct-to-consumer e-commerce startup estimates mean daily ad spending (thousand USD).
		For $n=120$ days, grouped data are:
		\[
		\begin{array}{c|c|c}
		\text{Class (thousand USD)} & m_i & f_i\\\hline
		50\text{--}55 & 52.5 & 12\\
		55\text{--}60 & 57.5 & 18\\
		60\text{--}65 & 62.5 & 28\\
		65\text{--}70 & 67.5 & 24\\
		70\text{--}75 & 72.5 & 22\\
		75\text{--}80 & 77.5 & 16
		\end{array}
		\]
		1) Construct a 90\% confidence interval for the population mean using the t-distribution from the grouped-data estimates.
		2) Using the interval and the computed statistics, apply the structured C9 decision procedure to compare the t-based and z-based intervals and determine which interval is wider and by what percentage.

		\subsubsection*{C8: Confidence Interval (t-statistic)}
		\[
		\mu=\text{population mean daily ad spending},\quad n=120,\quad \sum f_im_i=7870,\quad \sum f_im_i^2=524975
		\]
		\[
		\bar{x}\approx\frac{7870}{120}=65.583
		\]
		\[
		s\approx\sqrt{\frac{524975-120(65.583)^2}{119}}=7.620,\quad df=119
		\]
		\[
		t^*=t_{0.05,119}\approx 1.658,\quad SE=\frac{7.620}{\sqrt{120}}=0.696
		\]

		\[
		E_t=t^*(SE)=1.658(0.696)=1.15
		\]
		\[
		\bar{x}\pm t^*\left(\frac{s}{\sqrt{n}}\right)=65.583\pm 1.658\left(\frac{7.620}{\sqrt{120}}\right)=65.583\pm 1.15
		\]
		\[
		\mu\in[64.43,\,66.74]
		\]
		Interpretation: With 90\% confidence, mean daily ad spending is between 64.43 and 66.74 thousand USD.

		\subsubsection*{C5: Confidence Interval (z-statistic)}
		\[
		z^*=z_{0.05}=1.645
		\]

		\[
		E_z=z^*(SE)=1.645(0.696)=1.14
		\]

		\[
		\mu\in 65.583\pm E_z=65.583\pm 1.14=[64.44,\,66.72]
		\]

		\subsubsection*{C9: Structured Decision Procedure}
		\[
		t^*=1.658>1.645=z^*
		\]
		\[
		E_t=1.15,\quad E_z=1.14
		\]
		\[
		\text{Width}_t=2.30,\qquad \text{Width}_z=2.28
		\]
		\[
		\text{Absolute difference}=2.30-2.28=0.02
		\]
		\[
		\text{Relative difference (base }z\text{-interval)}=\frac{2.30-2.28}{2.28}\times 100=0.88\%
		\]
		At $df=119$, t and z are nearly equal, but t remains slightly wider.
		Final decision: use the t-interval as exact under unknown variance; it is 0.88\% wider than the z-interval.

		\newpage
		\item
		\subsection*{Problem description}
		A subscription software network studies mean monthly recurring revenue (thousand USD) for small creator-led apps.
		For $n=250$ firms, grouped data are:
		\[
		\begin{array}{c|c|c}
		\text{Class (thousand USD)} & m_i & f_i\\\hline
		80\text{--}90 & 85 & 18\\
		90\text{--}100 & 95 & 32\\
		100\text{--}110 & 105 & 46\\
		110\text{--}120 & 115 & 58\\
		120\text{--}130 & 125 & 44\\
		130\text{--}140 & 135 & 30\\
		140\text{--}150 & 145 & 22
		\end{array}
		\]
		1) Construct a 95\% confidence interval for the population mean using the t-distribution from the grouped-data estimates.
		2) Using the interval and the computed statistics, apply the structured C9 decision procedure to compare the t-based and z-based intervals and determine which interval is wider and by what percentage.

		\subsubsection*{C8: Confidence Interval (t-statistic)}
		\[
		\mu=\text{population mean monthly recurring revenue},\quad n=250,\quad \sum f_im_i=28810,\quad \sum f_im_i^2=3403450
		\]
		\[
		\bar{x}\approx\frac{28810}{250}=115.24
		\]
		\[
		s\approx\sqrt{\frac{3403450-250(115.24)^2}{249}}=16.741,\quad df=249
		\]
		\[
		t^*=t_{0.025,249}\approx 1.969,\quad SE=\frac{16.741}{\sqrt{250}}=1.059
		\]

		\[
		E_t=t^*(SE)=1.969(1.059)=2.08
		\]
		\[
		\bar{x}\pm t^*\left(\frac{s}{\sqrt{n}}\right)=115.24\pm 1.969\left(\frac{16.741}{\sqrt{250}}\right)=115.24\pm 2.08
		\]
		\[
		\mu\in[113.16,\,117.32]
		\]
		Interpretation: With 95\% confidence, mean monthly recurring revenue is between 113.16 and 117.32 thousand USD.

		\subsubsection*{C5: Confidence Interval (z-statistic)}
		\[
		z^*=z_{0.025}=1.960
		\]

		\[
		E_z=z^*(SE)=1.960(1.059)=2.08
		\]

		\[
		\mu\in 115.24\pm E_z=115.24\pm 2.08=[113.16,\,117.32]
		\]

		\subsubsection*{C9: Structured Decision Procedure}
		\[
		t^*=1.969>1.960=z^*
		\]
		\[
		E_t=2.08,\quad E_z=2.08
		\]
		\[
		\text{Width}_t=4.16,\qquad \text{Width}_z=4.16
		\]
		\[
		\text{Absolute difference}=4.16-4.16=0.00\text{ (more precisely }0.02\text{)}
		\]
		\[
		\text{Relative difference (base }z\text{-interval)}\approx\frac{0.02}{4.16}\times 100=0.48\%
		\]
		For large $df$, practical differences are tiny, but the t-interval is still (slightly) wider.
		Final decision: keep the t-interval as exact under unknown variance; it is about 0.48\% wider than the z-interval.

		\newpage
		\item
		\subsection*{Problem description}
		A fintech savings app estimates the mean monthly account balance (hundreds of USD) of active users in their 20s.
		For $n=500$ users, grouped data are:
		\[
		\begin{array}{c|c|c}
		\text{Class (hundreds of USD)} & m_i & f_i\\\hline
		60\text{--}65 & 62.5 & 30\\
		65\text{--}70 & 67.5 & 52\\
		70\text{--}75 & 72.5 & 84\\
		75\text{--}80 & 77.5 & 108\\
		80\text{--}85 & 82.5 & 96\\
		85\text{--}90 & 87.5 & 68\\
		90\text{--}95 & 92.5 & 40\\
		95\text{--}100 & 97.5 & 22
		\end{array}
		\]
		1) Construct a 99\% confidence interval for the population mean using the t-distribution from the grouped-data estimates.
		2) Using the interval and the computed statistics, apply the structured C9 decision procedure to compare the t-based and z-based intervals and determine which interval is wider and by what percentage.

		\subsubsection*{C8: Confidence Interval (t-statistic)}
		\[
		\mu=\text{population mean monthly balance},\quad n=500,\quad \sum f_im_i=39560,\quad \sum f_im_i^2=3174700
		\]
		\[
		\bar{x}\approx\frac{39560}{500}=79.12
		\]
		\[
		s\approx\sqrt{\frac{3174700-500(79.12)^2}{499}}=8.924,\quad df=499
		\]
		\[
		t^*=t_{0.005,499}\approx 2.586,\quad SE=\frac{8.924}{\sqrt{500}}=0.399
		\]

		\[
		E_t=t^*(SE)=2.586(0.399)=1.03
		\]
		\[
		\bar{x}\pm t^*\left(\frac{s}{\sqrt{n}}\right)=79.12\pm 2.586\left(\frac{8.924}{\sqrt{500}}\right)=79.12\pm 1.03
		\]
		\[
		\mu\in[78.09,\,80.15]
		\]
		Interpretation: With 99\% confidence, the app's mean monthly balance is between 78.09 and 80.15 hundred USD.

		\subsubsection*{C5: Confidence Interval (z-statistic)}
		\[
		z^*=z_{0.005}=2.576
		\]

		\[
		E_z=z^*(SE)=2.576(0.399)=1.03
		\]

		\[
		\mu\in 79.12\pm E_z=79.12\pm 1.03=[78.09,\,80.15]
		\]

		\subsubsection*{C9: Structured Decision Procedure}
		\[
		t^*=2.586>2.576=z^*
		\]
		\[
		E_t=1.03,\quad E_z=1.03
		\]
		\[
		\text{Width}_t=2.06,\qquad \text{Width}_z=2.06
		\]
		\[
		\text{Absolute difference}=2.06-2.06=0.00\text{ (more precisely }0.01\text{)}
		\]
		\[
		\text{Relative difference (base }z\text{-interval)}\approx\frac{0.01}{2.06}\times 100=0.49\%
		\]
		With hundreds of observations, t and z are nearly indistinguishable numerically.
		Final decision: the t-interval remains the exact choice and is about 0.49\% wider than the z-interval.

		\newpage
		\item
		\subsection*{Problem description}
		A national online tutoring marketplace estimates mean annual instructor earnings (USD) for active tutors.
		For $n=1000$ policies, grouped data are:
		\[
		\begin{array}{c|c|c}
		\text{Class (USD)} & m_i & f_i\\\hline
		125\text{--}135 & 130 & 70\\
		135\text{--}145 & 140 & 130\\
		145\text{--}155 & 150 & 220\\
		155\text{--}165 & 160 & 250\\
		165\text{--}175 & 170 & 180\\
		175\text{--}185 & 180 & 100\\
		185\text{--}195 & 190 & 50
		\end{array}
		\]
		1) Construct a 95\% confidence interval for the population mean using the t-distribution from the grouped-data estimates.
		2) Using the interval and the computed statistics, apply the structured C9 decision procedure to compare the t-based and z-based intervals and determine which interval is wider and by what percentage.

		\subsubsection*{C8: Confidence Interval (t-statistic)}
		\[
		\mu=\text{population mean annual instructor earnings},\quad n=1000,\quad \sum f_im_i=158400,\quad \sum f_im_i^2=25396000
		\]
		\[
		\bar{x}\approx\frac{158400}{1000}=158.40
		\]
		\[
		s\approx\sqrt{\frac{25396000-1000(158.4)^2}{999}}=15.417,\quad df=999
		\]
		\[
		t^*=t_{0.025,999}\approx 1.962,\quad SE=\frac{15.417}{\sqrt{1000}}=0.488
		\]

		\[
		E_t=t^*(SE)=1.962(0.488)=0.96
		\]
		\[
		\bar{x}\pm t^*\left(\frac{s}{\sqrt{n}}\right)=158.40\pm 1.962\left(\frac{15.417}{\sqrt{1000}}\right)=158.40\pm 0.96
		\]
		\[
		\mu\in[157.44,\,159.36]
		\]
		Interpretation: With 95\% confidence, mean annual instructor earnings are between 157.44 and 159.36 USD.

		\subsubsection*{C5: Confidence Interval (z-statistic)}
		\[
		z^*=z_{0.025}=1.960
		\]

		\[
		E_z=z^*(SE)=1.960(0.488)=0.96
		\]

		\[
		\mu\in 158.40\pm E_z=158.40\pm 0.96=[157.44,\,159.36]
		\]

		\subsubsection*{C9: Structured Decision Procedure}
		\[
		t^*=1.962>1.960=z^*
		\]
		\[
		E_t=0.96,\quad E_z=0.96
		\]
		\[
		\text{Width}_t=1.92,\qquad \text{Width}_z=1.92
		\]
		\[
		\text{Absolute difference}\approx 0.00\text{ (more precisely }0.00\text{ to two decimals)}
		\]
		\[
		\text{Relative difference (base }z\text{-interval)}\approx 0.10\%
		\]
		For $df=999$, convergence is very strong and both intervals are practically identical.
		Final decision: select the t-interval as exact; it is approximately 0.10\% wider than the z-interval.

		\newpage
		\item
		\subsection*{Problem description}
		An e-commerce marketplace analyzes mean monthly seller revenue (thousand USD) per store.
		For $n=2500$ merchants, grouped data are:
		\[
		\begin{array}{c|c|c}
		\text{Class (thousand USD)} & m_i & f_i\\\hline
		200\text{--}220 & 210 & 180\\
		220\text{--}240 & 230 & 360\\
		240\text{--}260 & 250 & 620\\
		260\text{--}280 & 270 & 700\\
		280\text{--}300 & 290 & 380\\
		300\text{--}320 & 310 & 180\\
		320\text{--}340 & 330 & 80
		\end{array}
		\]
		1) Construct a 90\% confidence interval for the population mean using the t-distribution from the grouped-data estimates.
		2) Using the interval and the computed statistics, apply the structured C9 decision procedure to compare the t-based and z-based intervals and determine which interval is wider and by what percentage.

		\subsubsection*{C8: Confidence Interval (t-statistic)}
		\[
		\mu=\text{population mean monthly seller revenue},\quad n=2500,\quad \sum f_im_i=657000,\quad \sum f_im_i^2=174410000
		\]
		\[
		\bar{x}\approx\frac{657000}{2500}=262.80
		\]
		\[
		s\approx\sqrt{\frac{174410000-2500(262.8)^2}{2499}}=28.784,\quad df=2499
		\]
		\[
		t^*=t_{0.05,2499}\approx 1.646,\quad SE=\frac{28.784}{\sqrt{2500}}=0.576
		\]

		\[
		E_t=t^*(SE)=1.646(0.576)=0.95
		\]
		\[
		\bar{x}\pm t^*\left(\frac{s}{\sqrt{n}}\right)=262.80\pm 1.646\left(\frac{28.784}{\sqrt{2500}}\right)=262.80\pm 0.95
		\]
		\[
		\mu\in[261.85,\,263.75]
		\]
		Interpretation: With 90\% confidence, mean monthly seller revenue is between 261.85 and 263.75 thousand USD.

		\subsubsection*{C5: Confidence Interval (z-statistic)}
		\[
		z^*=z_{0.05}=1.645
		\]

		\[
		E_z=z^*(SE)=1.645(0.576)=0.95
		\]

		\[
		\mu\in 262.80\pm E_z=262.80\pm 0.95=[261.85,\,263.75]
		\]

		\subsubsection*{C9: Structured Decision Procedure}
		\[
		t^*=1.646>1.645=z^*
		\]
		\[
		E_t=0.95,\quad E_z=0.95
		\]
		\[
		\text{Width}_t=1.90,\qquad \text{Width}_z=1.90
		\]
		\[
		\text{Absolute difference}\approx 0.00\text{ (more precisely }0.00\text{ to two decimals)}
		\]
		\[
		\text{Relative difference (base }z\text{-interval)}\approx 0.05\%
		\]
		At this scale, t and z intervals are nearly the same numerically.
		Final decision: use the t-interval as exact under unknown variance; it is approximately 0.05\% wider than the z-interval.

		\newpage
		\item
		\subsection*{Problem description}
		A streaming and creator-services network studies mean quarterly creator revenue (thousand USD) across channels.
		For $n=5000$ firms, grouped data are:
		\[
		\begin{array}{c|c|c}
		\text{Class (thousand USD)} & m_i & f_i\\\hline
		400\text{--}440 & 420 & 320\\
		440\text{--}480 & 460 & 680\\
		480\text{--}520 & 500 & 1300\\
		520\text{--}560 & 540 & 1400\\
		560\text{--}600 & 580 & 820\\
		600\text{--}640 & 620 & 360\\
		640\text{--}680 & 660 & 120
		\end{array}
		\]
		1) Construct a 99\% confidence interval for the population mean using the t-distribution from the grouped-data estimates.
		2) Using the interval and the computed statistics, apply the structured C9 decision procedure to compare the t-based and z-based intervals and determine which interval is wider and by what percentage.

		\subsubsection*{C8: Confidence Interval (t-statistic)}
		\[
		\mu=\text{population mean quarterly creator revenue},\quad n=5000,\quad \sum f_im_i=2631200,\quad \sum f_im_i^2=1401024000
		\]
		\[
		\bar{x}\approx\frac{2631200}{5000}=526.24
		\]
		\[
		s\approx\sqrt{\frac{1401024000-5000(526.24)^2}{4999}}=55.570,\quad df=4999
		\]
		\[
		t^*=t_{0.005,4999}\approx 2.577,\quad SE=\frac{55.570}{\sqrt{5000}}=0.786
		\]

		\[
		E_t=t^*(SE)=2.577(0.786)=2.03
		\]
		\[
		\bar{x}\pm t^*\left(\frac{s}{\sqrt{n}}\right)=526.24\pm 2.577\left(\frac{55.570}{\sqrt{5000}}\right)=526.24\pm 2.03
		\]
		\[
		\mu\in[524.21,\,528.27]
		\]
		Interpretation: With 99\% confidence, mean quarterly creator revenue is between 524.21 and 528.27 thousand USD.

		\subsubsection*{C5: Confidence Interval (z-statistic)}
		\[
		z^*=z_{0.005}=2.576
		\]

		\[
		E_z=z^*(SE)=2.576(0.786)=2.02
		\]

		\[
		\mu\in 526.24\pm E_z=526.24\pm 2.02=[524.22,\,528.26]
		\]

		\subsubsection*{C9: Structured Decision Procedure}
		\[
		t^*=2.577>2.576=z^*
		\]
		\[
		E_t=2.03,\quad E_z=2.02
		\]
		\[
		\text{Width}_t=4.06,\qquad \text{Width}_z=4.04
		\]
		\[
		\text{Absolute difference}=4.06-4.04=0.02
		\]
		\[
		\text{Relative difference (base }z\text{-interval)}=\frac{4.06-4.04}{4.04}\times 100=0.50\%
		\]
		With very large degrees of freedom, the numerical gap is minimal but still positive for t.
		Final decision: keep the t-interval as exact under unknown variance; it is 0.50\% wider than the z-interval.
	\end{ExamProblems}
\end{document}
