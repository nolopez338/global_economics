\makeatletter
\def\input@path{{./}{../}{../../}{preamble/}{../preamble/}{../../preamble/}}
\makeatother
% ----------------------------------------------------------
% GENERAL 

% File
\documentclass[11pt]{book}

% Margins
\usepackage[margin=1in]{geometry}

\linespread{1.2}            % Line spacing
\usepackage[utf8]{inputenc}
\usepackage[T1]{fontenc}
\usepackage{lmodern}
\usepackage{microtype}
\setlength{\parindent}{0pt}
\setlength{\parskip}{6pt}
\usepackage{booktabs}

% ----------------------------------------------------------
% TABLES
\usepackage{multicol}
\usepackage{longtable} 
\usepackage{array}
\usepackage{booktabs}
\usepackage{tabularx}
\usepackage{multirow}

% ----------------------------------------------------------
% MATHEMATICS
\usepackage{amsmath}
\usepackage{amssymb}
\usepackage{amsfonts}
\usepackage{mathtools}

% ----------------------------------------------------------
% HIDDEN CONTENT
\usepackage{ifthen}
% Define a boolean switch
\newboolean{explicaciones}
% Set the boolean switch to true or false
% Change to true to show the content

% Explanations
\newcommand{\explicacion}[2]{
	\ifthenelse{\boolean{explicaciones}}{#1}{#2}
}
\newcommand{\mostrarExplicaciones}[1]{\setboolean{explicaciones}{#1}}

% ----------------------------------------------------------
% NUMBERING

\usepackage{fancyhdr}
\pagestyle{empty} % Ensures the entire document has no page numbers

\usepackage{tocloft}
\renewcommand{\cftdot}{} % Remove dots for sections, if any
\renewcommand{\cftsecleader}{\cftdotfill{\cftdotsep}} % Remove dots for sections, if any
\cftpagenumbersoff{section} % Removes page numbers from sections
\cftpagenumbersoff{subsection} % Removes page numbers from subsections

% ----------------------------------------------------------
% IMAGES 

% General settings
\usepackage{graphicx}       % Insert images
\usepackage{float}          % Position images
% \usepackage{subfigure}      % Subfigures
\graphicspath{{imgs}}       % Image location
\usepackage{subcaption}     % Subfigures II
\usepackage{verbatim}

% Figures
\usepackage{tikz}
\usetikzlibrary{arrows.meta,positioning,trees}

% Colors
\usepackage{xcolor}     
\definecolor{popUp}{HTML}{666666}
\definecolor{popUpIn}{HTML}{CED9E0}
\definecolor{backgroundC}{HTML}{D0E8F2}
\definecolor{backgroundCC}{HTML}{FFFFFF}
\definecolor{borders}{HTML}{8c120d}
\definecolor{padding}{HTML}{77D0D7}
\definecolor{links}{HTML}{CC6F5F}

% ----------------------------------------------------------
% FRAMES

% General settings
\usepackage{tcolorbox}
\usepackage{adjustbox}          % Adjusted frame  
\setlength{\fboxrule}{3pt}  % Line width
\setlength{\fboxsep}{3pt}   % Box padding

% General frames
\usepackage{mdframed}   

\mdfdefinestyle{estiloGeneral}{    % General style
	linecolor=black,
	linewidth=1.5pt,
	roundcorner=10pt,
	backgroundcolor=backgroundC,
	innerbottommargin=0pt
}
\mdfdefinestyle{code}{          % Code style
	linecolor=black,
	linewidth=1.5pt,
	roundcorner=10pt,
	backgroundcolor=darkgray!10,
	innerbottommargin=0pt
}

% Image frame
\newtcbox{\fboxC}{
	colback=backgroundC,
	colframe=popUp,
	arc=10pt,
	boxrule=3pt,
	boxsep=0pt, % Change the padding here
	nobeforeafter
}

% ----------------------------------------------------------
% PAGE SETTINGS

% Background 
\newcommand{\background}[0]{\begin{tikzpicture}[remember picture,overlay]
		\fill[backgroundC] (-2,2) rectangle (25cm, -550);
\end{tikzpicture}}

\newcommand{\backgroundC}[0]{\begin{tikzpicture}[remember picture,overlay]
		\fill[backgroundCC] (-2,2) rectangle (25cm, -550);
\end{tikzpicture}}

% Page width 
\newcommand{\anchoPag}[0]{20cm}

% ----------------------------------------------------------
% FONT

% General
\usepackage{tgbonum}        % Font
\usepackage{listings}       % Code typesetting
\usepackage[spanish]{babel} % Load Spanish
\selectlanguage{spanish}    % Select Spanish
\usepackage{enumitem}
\usepackage{bookmark}

\setlist[itemize]{leftmargin=1.2em, itemsep=0.35em, topsep=0.35em}

% --- Table helpers ---
\newcolumntype{L}[1]{>{\raggedright\arraybackslash}p{#1}}
\newcolumntype{Y}{>{\raggedright\arraybackslash}X}
\newcolumntype{C}{>{\centering\arraybackslash}X}
\renewcommand{\arraystretch}{1.1}

% Python style
\lstdefinestyle{python}{
	language=Python,
	basicstyle=\ttfamily\small,
	commentstyle=\color{green!50!black},
	keywordstyle=\color{blue},
	numberstyle=\tiny\color{gray},
	numbers=left,
	morekeywords={>, <},
	breakatwhitespace=false,
	showstringspaces=false,
	showtabs=false,
	showspaces=false
}

% ----------------------------------------------------------
% HYPERLINKS

% General
\usepackage{hyperref}       
\hypersetup{
	colorlinks=true,
	linkcolor=links,
	filecolor=magenta,      
	urlcolor=brown,
}

% Custom commands 

% Large link
\newcommand{\bigLink}[2]{\begin{center} \fboxC{\LARGE{\href{#1}{#2}}}\end{center}}

% Small link
\newcommand{\smallLink}[2]{\begin{center}\fboxC{\href{#1}{#2}}\end{center}}

% Bold link
\newcommand{\bfLink}[2]{\href{#1}{\textbf{#2}}}


% Small URL
\newcommand{\smallUrl}[1]{\begin{center}\fboxC{\url{#1}}\end{center}}


% ----------------------------------------------------------
% CUSTOM COMMANDS FOR FIGURES

\newcommand{\espacioImagenes}[0]{-1.2cm}

% Without frame
\newcommand{\fig}[3][\espacioImagenes]{
	\hspace*{#1}
	\centering
	\includegraphics[width=#2\textwidth]{#3}
}

% With frame
\newcommand{\ffig}[2]{\begin{figure}[h]
		\hspace*{\espacioImagenes}
		\centering
		\fbox{\includegraphics[width=#1\textwidth]{#2}}
\end{figure}}

% Hyperlink with frame
\newcommand{\hfig}[3]{\begin{figure}[h]
		\hspace*{-1.4cm}
		\centering
		\color{popUp}
		\fboxC{\href{#1}{\includegraphics[width=#2\textwidth]{#3}}}
	\end{figure}
}

% Hyperlink without frame
\newcommand{\hffig}[3]{\begin{figure}[h]
		\hspace*{-1.1cm}
		\centering
		\color{popUp}
		\href{#1}{\includegraphics[width=#2\textwidth]{#3}}
	\end{figure}
}

% ----------------------------------------------------------

% Start and Contents
\newcommand{\cuadro}[1]{
	\begin{mdframed}[style=estiloGeneral]
		#1 
	\end{mdframed}
}

% Explanation video image
\newcommand{\linkExplicacion}[1]{
	\hffig{#1}{0.5}{principal/videoExplicacion}
	\vspace{-0.5cm}
}

\newcommand{\subSecLink}[2]{
	\subsubsection*{\href{#1}{\textbf{#2}}}
}

% Spacing
\newcommand{\esp}[0]{\vspace{4mm}}

% Back to start
\newcommand{\secInicio}[0]{\begin{center}\hyperref[sec:inicio]{ 
			\includegraphics[width=0.1\textwidth]{principal/up}
	}\end{center}
}


\geometry{margin=0.85in}
\AtBeginDocument{\small}

\newcommand{\ExamNameField}{\noindent\textbf{Name:}\ \rule{0.7\linewidth}{0.4pt}\par\medskip}

\newcommand{\ExamTitleBlock}[3]{%
	\begin{center}
		\Large\textbf{#1}\\
		\textbf{#2}%
		\if\relax\detokenize{#3}\relax\else\\\textbf{#3}\fi
	\end{center}
	\vspace{0.5em}
}

\newcommand{\ExamSection}[1]{\par\medskip\textbf{#1}\par\smallskip}

\newenvironment{ExamCriteria}{%
	\begin{itemize}[leftmargin=1.6em, itemsep=0.3em, topsep=0.2em]
}{%
	\end{itemize}
}

\newenvironment{ExamProblems}{%
	\begin{enumerate}[label=\textbf{P\arabic*}, leftmargin=0pt, labelsep=0.6em, itemindent=2.2em, itemsep=0.8em]
}{%
	\end{enumerate}
}

\begin{document}
	\ExamTitleBlock{11th grade}{Term 2 -- Lesson 3 MATERIAL: t-Confidence Intervals for Means (Worked Solutions)}{}
	
	\ExamSection{Evaluated criteria}
	\begin{ExamCriteria}
		\item C8: Calculates confidence intervals applying the t-student distribution in context situations.
		\item C9: Estimates the confidence interval for the mean when the variance is unknown in context situations.
		\item C10: Concludes about a statistical parameter in situations in finance and economy.
	\end{ExamCriteria}

	\ExamSection{C8: t-student confidence intervals in context situations}
	\begin{ExamProblems}
		\item
		\subsection*{Problem description}
		A logistics center wants to estimate the mean loading time (minutes) for trucks in a new dispatch lane.
		A random sample of $n=12$ trucks gives $\bar{x}=84.60$ minutes and sample standard deviation $s=6.80$ minutes.
		The population variance is unknown, and the sample size is small, so a t-interval is required.
		Construct and interpret a 95\% confidence interval for the population mean loading time.
		
		\subsection*{C8}
		Step 1.
		Identify the parameter and sample statistics.
		\[
		\mu=\text{population mean loading time},\quad n=12,\quad \bar{x}=84.60,\quad s=6.80
		\]
		Step 2.
		State degrees of freedom.
		\[
		df=n-1=12-1=11
		\]
		Step 3.
		Identify the t critical value for 95\% confidence.
		\[
		t^*=t_{0.025,11}\approx 2.201
		\]
		Step 4.
		Compute the standard error.
		\[
		SE=\frac{s}{\sqrt{n}}=\frac{6.80}{\sqrt{12}}\approx 1.96
		\]
		Step 5.
		Compute the margin of error.
		\[
		E=t^*\cdot SE=2.201(1.96)\approx 4.31
		\]
		Step 6.
		Construct the confidence interval.
		\[
		\mu\in\bar{x}\pm E\Rightarrow 84.60\pm 4.31=[80.29,\,88.91]
		\]
		Step 7.
		Interpretation.
		With 95\% confidence, the mean loading time in the new lane is between 80.29 and 88.91 minutes.
		
		\newpage
		\item
		\subsection*{Problem description}
		An agricultural exporter tracks the average customs processing time (hours) for a new shipping route.
		A random sample of $n=18$ shipments gives $\bar{x}=42.30$ hours and $s=5.10$ hours.
		Because the population variance is unknown and the sample is moderate, use a t-interval.
		Construct and interpret a 90\% confidence interval for the population mean processing time.
		
		\subsection*{C8}
		Step 1.
		Identify the parameter and sample statistics.
		\[
		\mu=\text{population mean processing time},\quad n=18,\quad \bar{x}=42.30,\quad s=5.10
		\]
		Step 2.
		State degrees of freedom.
		\[
		df=18-1=17
		\]
		Step 3.
		Identify the t critical value for 90\% confidence.
		\[
		t^*=t_{0.05,17}\approx 1.740
		\]
		Step 4.
		Compute the standard error.
		\[
		SE=\frac{5.10}{\sqrt{18}}\approx 1.20
		\]
		Step 5.
		Compute the margin of error.
		\[
		E=1.740(1.20)\approx 2.09
		\]
		Step 6.
		Construct the confidence interval.
		\[
		\mu\in 42.30\pm 2.09=[40.21,\,44.39]
		\]
		Step 7.
		Interpretation.
		With 90\% confidence, the true mean customs processing time for this route lies between 40.21 and 44.39 hours.
		
		\newpage
		\item
		\subsection*{Problem description}
		A school cafeteria estimates the mean daily demand (kg) for a new meal plan ingredient.
		A random sample of $n=25$ school days gives $\bar{x}=73.50$ kg and $s=8.40$ kg.
		The population variance is unknown.
		Construct and interpret a 99\% confidence interval for the population mean daily demand.
		
		\subsection*{C8}
		Step 1.
		Identify the parameter and sample statistics.
		\[
		\mu=\text{population mean daily demand},\quad n=25,\quad \bar{x}=73.50,\quad s=8.40
		\]
		Step 2.
		State degrees of freedom.
		\[
		df=25-1=24
		\]
		Step 3.
		Identify the t critical value for 99\% confidence.
		\[
		t^*=t_{0.005,24}\approx 2.797
		\]
		Step 4.
		Compute the standard error.
		\[
		SE=\frac{8.40}{\sqrt{25}}=1.68
		\]
		Step 5.
		Compute the margin of error.
		\[
		E=2.797(1.68)\approx 4.70
		\]
		Step 6.
		Construct the confidence interval.
		\[
		\mu\in 73.50\pm 4.70=[68.80,\,78.20]
		\]
		Step 7.
		Interpretation.
		With 99\% confidence, the population mean daily demand is between 68.80 kg and 78.20 kg.
	\end{ExamProblems}

	\ExamSection{C9: Confidence interval for the mean when variance is unknown}
	\begin{ExamProblems}
		\item
		\subsection*{Problem description}
		A retail bank studies the mean monthly card spending (hundreds of USD) of new digital-account clients.
		From a large random sample, the summary statistics are: $n=36$, $\bar{x}=52.40$, $s=8.10$.
		Construct both a 95\% t-confidence interval and a 95\% z-confidence interval for the population mean, then compare them.
		
		\subsection*{C9}
		Step 1: Identify parameter and statistics
		\[
		\mu=\text{population mean monthly card spending},\quad n=36,\quad \bar{x}=52.40,\quad s=8.10
		\]
		\[
		df=n-1=35,\quad n\ge 30\text{ (large sample)}
		\]
		Step 2: Construct t-interval
		\[
		SE=\frac{s}{\sqrt{n}}=\frac{8.10}{\sqrt{36}}=1.35,\qquad t^*=t_{0.025,35}\approx 2.030
		\]
		\[
		\mu\in \bar{x}\pm t^*SE=52.40\pm 2.030(1.35)=52.40\pm 2.74=[49.66,\,55.14]
		\]
		Step 3: Construct z-interval
		\[
		z^*=z_{0.025}=1.960,\qquad \mu\in \bar{x}\pm z^*SE=52.40\pm 1.960(1.35)=52.40\pm 2.65=[49.75,\,55.05]
		\]
		Step 4: Numerical comparison
		\[
		t^*=2.030>1.960=z^*,\quad E_t=2.74>2.65=E_z
		\]
		\[
		\text{Width}_t=2(2.74)=5.48,\qquad \text{Width}_z=2(2.65)=5.30
		\]
		The t-interval is slightly wider.
		Step 5: Conceptual explanation of convergence
		Because $df=35$ is large, the t distribution is very close to the standard normal distribution.
		As $df$ increases, $t(df)\to N(0,1)$, so the t and z intervals become very similar.
		The normal approximation is reasonable here, while the t-interval remains the exact method for unknown variance.
		Step 6: Interpretation in economic/financial context
		Both intervals indicate that the bank's mean monthly card spending is around 50 to 55 hundred USD.
		Using the exact t method, a 95\% confidence interval is $[49.66,55.14]$ hundred USD.
		
		\newpage
		\item
		\subsection*{Problem description}
		An import company estimates the mean daily shipping cost per container (thousand USD).
		From a large random sample, the summary statistics are: $n=64$, $\bar{x}=1.84$, $s=0.56$.
		Construct both a 90\% t-confidence interval and a 90\% z-confidence interval for the population mean, then compare them.
		
		\subsection*{C9}
		Step 1: Identify parameter and statistics
		\[
		\mu=\text{population mean daily shipping cost per container},\quad n=64,\quad \bar{x}=1.84,\quad s=0.56
		\]
		\[
		df=63,\quad n\ge 30\text{ (large sample)}
		\]
		Step 2: Construct t-interval
		\[
		SE=\frac{0.56}{\sqrt{64}}=0.07,\qquad t^*=t_{0.05,63}\approx 1.669
		\]
		\[
		\mu\in 1.84\pm 1.669(0.07)=1.84\pm 0.117=[1.723,\,1.957]
		\]
		Step 3: Construct z-interval
		\[
		z^*=z_{0.05}=1.645,\qquad \mu\in 1.84\pm 1.645(0.07)=1.84\pm 0.115=[1.725,\,1.955]
		\]
		Step 4: Numerical comparison
		\[
		t^*=1.669>1.645=z^*,\quad E_t=0.117>0.115=E_z
		\]
		\[
		\text{Width}_t=0.234,\qquad \text{Width}_z=0.230
		\]
		The t-interval is slightly wider.
		Step 5: Conceptual explanation of convergence
		With $df=63$, the t critical value is already very close to the z critical value.
		This reflects convergence: as $df$ increases, $t(df)$ converges to $N(0,1)$.
		Therefore, the normal approximation is reasonable in this large-sample setting.
		Step 6: Interpretation in economic/financial context
		The company can estimate mean shipping cost at about 1.72 to 1.96 thousand USD per container.
		Using the exact t method, the 90\% confidence interval is $[1.723,1.957]$ thousand USD.
		
		\newpage
		\item
		\subsection*{Problem description}
		An economist estimates the mean monthly fuel expense (thousand USD) for urban delivery firms.
		From a large random sample, the summary statistics are: $n=49$, $\bar{x}=27.30$, $s=4.90$.
		Construct both a 99\% t-confidence interval and a 99\% z-confidence interval for the population mean, then compare them.
		
		\subsection*{C9}
		Step 1: Identify parameter and statistics
		\[
		\mu=\text{population mean monthly fuel expense},\quad n=49,\quad \bar{x}=27.30,\quad s=4.90
		\]
		\[
		df=48,\quad n\ge 30\text{ (large sample)}
		\]
		Step 2: Construct t-interval
		\[
		SE=\frac{4.90}{\sqrt{49}}=0.70,\qquad t^*=t_{0.005,48}\approx 2.682
		\]
		\[
		\mu\in 27.30\pm 2.682(0.70)=27.30\pm 1.88=[25.42,\,29.18]
		\]
		Step 3: Construct z-interval
		\[
		z^*=z_{0.005}=2.576,\qquad \mu\in 27.30\pm 2.576(0.70)=27.30\pm 1.80=[25.50,\,29.10]
		\]
		Step 4: Numerical comparison
		\[
		t^*=2.682>2.576=z^*,\quad E_t=1.88>1.80=E_z
		\]
		\[
		\text{Width}_t=3.76,\qquad \text{Width}_z=3.60
		\]
		The t-interval is slightly wider.
		Step 5: Conceptual explanation of convergence
		Even at 99\% confidence, the two intervals are close because the sample is large.
		As degrees of freedom increase, the t distribution converges to the standard normal distribution, $t(df)\to N(0,1)$.
		So the z interval is a reasonable approximation, while the t interval remains the exact choice when $\sigma$ is unknown.
		Step 6: Interpretation in economic/financial context
		The estimated mean monthly fuel expense is around 25.4 to 29.2 thousand USD.
		Using the exact method, the 99\% t-confidence interval is $[25.42,29.18]$ thousand USD.
	\end{ExamProblems}

	\ExamSection{C10: Conclusions about a mean parameter in finance and economy}
	\begin{ExamProblems}
		\item
		\subsection*{Problem description}
		A commercial bank tracks daily net fee revenue (thousand USD) from a new SME payments package for 16 days:
		\[
		12,\ 13,\ 14,\ 13,\ 12,\ 14,\ 13,\ 13,\ 14,\ 12,\ 13,\ 14,\ 13,\ 12,\ 14,\ 13
		\]
		Use these raw data to construct a 95\% confidence interval for the population mean daily net fee revenue.
		Management claims that the true mean equals 12.90 thousand USD.
		Evaluate this claim in context.
		
		\subsection*{C10}
		Step 1 -- Identify parameter and compute statistics
		\[
		\mu=\text{population mean daily net fee revenue},\qquad n=16
		\]
		\[
		\bar{x}=\frac{\sum x_i}{n}=\frac{209}{16}=13.0625
		\]
		\[
		s^2=\frac{\sum (x_i-\bar{x})^2}{n-1}=0.5958,\qquad s=\sqrt{0.5958}=0.7719
		\]
		Step 2 -- Degrees of freedom and critical value
		\[
		df=n-1=16-1=15,\qquad t^*=t_{0.025,15}\approx 2.131
		\]
		Step 3 -- Standard error
		\[
		SE=\frac{s}{\sqrt{n}}=\frac{0.7719}{\sqrt{16}}=0.1930
		\]
		Step 4 -- Margin of error
		\[
		E=t^*\cdot SE=2.131(0.1930)=0.4113
		\]
		Step 5 -- Confidence interval
		\[
		\mu\in \bar{x}\pm E=13.0625\pm 0.4113=[12.6512,\,13.4738]
		\]
		Step 6 -- Interpretation
		With 95\% confidence, the bank's mean daily net fee revenue from this package is between 12.651 and 13.474 thousand USD.
		Step 7 -- Conclusion about the parameter
		The managerial claim $\mu=12.90$ lies inside the interval, so it is plausible given this sample.
		However, the interval also supports nearby values, so the claim is not proven as the unique true mean.
		
		\newpage
		\item
		\subsection*{Problem description}
		An export firm groups daily freight insurance cost (USD per shipment) into classes for 30 days:
		\[
		\begin{array}{c|c}
		\text{Class interval} & \text{Frequency}\\ \hline
		\text{[80,90)} & 3\\
		\text{[90,100)} & 7\\
		\text{[100,110)} & 9\\
		\text{[110,120)} & 6\\
		\text{[120,130)} & 5
		\end{array}
		\]
		Using class midpoints, construct an approximate 95\% t-confidence interval for the population mean daily insurance cost.
		The finance director uses 112 USD as the benchmark mean.
		Assess whether this benchmark is supported.
		
		\subsection*{C10}
		Step 1 -- Identify parameter and compute statistics
		\[
		\mu=\text{population mean daily freight insurance cost},\qquad n=3+7+9+6+5=30
		\]
		\[
		\text{Midpoints }m_i:\ 85,\ 95,\ 105,\ 115,\ 125
		\]
		\[
		\bar{x}\approx\frac{\sum f_im_i}{n}=\frac{3(85)+7(95)+9(105)+6(115)+5(125)}{30}=106.00
		\]
		\[
		s^2\approx\frac{\sum f_i(m_i-\bar{x})^2}{n-1}=154.14,\qquad s\approx\sqrt{154.14}=12.42
		\]
		Step 2 -- Degrees of freedom and critical value
		\[
		df=n-1=29,\qquad t^*=t_{0.025,29}\approx 2.045
		\]
		Step 3 -- Standard error
		\[
		SE=\frac{s}{\sqrt{n}}=\frac{12.42}{\sqrt{30}}\approx 2.27
		\]
		Step 4 -- Margin of error
		\[
		E=t^*\cdot SE=2.045(2.27)\approx 4.64
		\]
		Step 5 -- Confidence interval
		\[
		\mu\in 106.00\pm 4.64=[101.36,\,110.64]\text{ USD}
		\]
		Step 6 -- Interpretation
		With 95\% confidence, the firm's mean daily freight insurance cost is between 101.36 and 110.64 USD per shipment.
		Step 7 -- Conclusion about the parameter
		The benchmark value 112 USD is outside the interval, so the data do not support that benchmark for the population mean.
		A lower budgeting baseline, close to 106 USD, is more consistent with observed costs.
		
		\newpage
		\item
		\subsection*{Problem description}
		A fintech support unit studies the mean ticket-resolution time (hours).
		It has two data sources in the same month:
		\begin{itemize}
			\item Individual observations (8 tickets):
			\[
			3,\ 3,\ 4,\ 3,\ 3,\ 2,\ 4,\ 4
			\]
			\item Additional tickets summarized in grouped form:
			\[
			\begin{array}{c|c}
			\text{Class interval} & \text{Frequency}\\ \hline
			\text{[2,3)} & 3\\
			\text{[3,4)} & 5\\
			\text{[4,5)} & 4
			\end{array}
			\]
		\end{itemize}
		Use midpoint approximations for the grouped part, combine all information, and construct a 90\% t-confidence interval for the population mean resolution time.
		The operations team claims the true mean is 3.10 hours; evaluate this claim and discuss whether a target below 3.00 hours is secured.
		
		\subsection*{C10}
		Step 1 -- Identify parameter and compute statistics
		\[
		\mu=\text{population mean ticket-resolution time},\qquad n_1=8
		\]
		\[
		n_2=3+5+4=12,\quad n=n_1+n_2=20
		\]
		\[
		\sum x_{\text{raw}}=26,\qquad
		\sum x_{\text{grouped}}\approx 3(2.5)+5(3.5)+4(4.5)=43
		\]
		\[
		\bar{x}=\frac{26+43}{20}=3.45\text{ hours}
		\]
		\[
		s^2\approx\frac{\sum (x_i-\bar{x})^2}{n-1}=0.5763,\qquad s\approx\sqrt{0.5763}=0.7592\text{ hours}
		\]
		Step 2 -- Degrees of freedom and critical value
		\[
		df=n-1=19,\qquad t^*=t_{0.05,19}\approx 1.729
		\]
		Step 3 -- Standard error
		\[
		SE=\frac{s}{\sqrt{n}}=\frac{0.7592}{\sqrt{20}}\approx 0.1698
		\]
		Step 4 -- Margin of error
		\[
		E=t^*\cdot SE=1.729(0.1698)\approx 0.2936
		\]
		Step 5 -- Confidence interval
		\[
		\mu\in 3.45\pm 0.2936=[3.156,\,3.744]\text{ hours}
		\]
		Step 6 -- Interpretation
		With 90\% confidence, the mean ticket-resolution time is between 3.156 and 3.744 hours.
		Step 7 -- Conclusion about the parameter
		The claim $\mu=3.10$ hours is not supported because 3.10 is below the lower bound of the interval.
		A target below 3.00 hours is also not secured, since the entire interval is above 3.00 hours.
	\end{ExamProblems}
\end{document}
