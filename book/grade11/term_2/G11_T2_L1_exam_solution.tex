\makeatletter
\def\input@path{{./}{../}{../../}{preamble/}{../preamble/}{../../preamble/}}
\makeatother
% ----------------------------------------------------------
% GENERAL 

% File
\documentclass[11pt]{book}

% Margins
\usepackage[margin=1in]{geometry}

\linespread{1.2}            % Line spacing
\usepackage[utf8]{inputenc}
\usepackage[T1]{fontenc}
\usepackage{lmodern}
\usepackage{microtype}
\setlength{\parindent}{0pt}
\setlength{\parskip}{6pt}
\usepackage{booktabs}

% ----------------------------------------------------------
% TABLES
\usepackage{multicol}
\usepackage{longtable} 
\usepackage{array}
\usepackage{booktabs}
\usepackage{tabularx}
\usepackage{multirow}

% ----------------------------------------------------------
% MATHEMATICS
\usepackage{amsmath}
\usepackage{amssymb}
\usepackage{amsfonts}
\usepackage{mathtools}

% ----------------------------------------------------------
% HIDDEN CONTENT
\usepackage{ifthen}
% Define a boolean switch
\newboolean{explicaciones}
% Set the boolean switch to true or false
% Change to true to show the content

% Explanations
\newcommand{\explicacion}[2]{
	\ifthenelse{\boolean{explicaciones}}{#1}{#2}
}
\newcommand{\mostrarExplicaciones}[1]{\setboolean{explicaciones}{#1}}

% ----------------------------------------------------------
% NUMBERING

\usepackage{fancyhdr}
\pagestyle{empty} % Ensures the entire document has no page numbers

\usepackage{tocloft}
\renewcommand{\cftdot}{} % Remove dots for sections, if any
\renewcommand{\cftsecleader}{\cftdotfill{\cftdotsep}} % Remove dots for sections, if any
\cftpagenumbersoff{section} % Removes page numbers from sections
\cftpagenumbersoff{subsection} % Removes page numbers from subsections

% ----------------------------------------------------------
% IMAGES 

% General settings
\usepackage{graphicx}       % Insert images
\usepackage{float}          % Position images
% \usepackage{subfigure}      % Subfigures
\graphicspath{{imgs}}       % Image location
\usepackage{subcaption}     % Subfigures II
\usepackage{verbatim}

% Figures
\usepackage{tikz}
\usetikzlibrary{arrows.meta,positioning,trees}

% Colors
\usepackage{xcolor}     
\definecolor{popUp}{HTML}{666666}
\definecolor{popUpIn}{HTML}{CED9E0}
\definecolor{backgroundC}{HTML}{D0E8F2}
\definecolor{backgroundCC}{HTML}{FFFFFF}
\definecolor{borders}{HTML}{8c120d}
\definecolor{padding}{HTML}{77D0D7}
\definecolor{links}{HTML}{CC6F5F}

% ----------------------------------------------------------
% FRAMES

% General settings
\usepackage{tcolorbox}
\usepackage{adjustbox}          % Adjusted frame  
\setlength{\fboxrule}{3pt}  % Line width
\setlength{\fboxsep}{3pt}   % Box padding

% General frames
\usepackage{mdframed}   

\mdfdefinestyle{estiloGeneral}{    % General style
	linecolor=black,
	linewidth=1.5pt,
	roundcorner=10pt,
	backgroundcolor=backgroundC,
	innerbottommargin=0pt
}
\mdfdefinestyle{code}{          % Code style
	linecolor=black,
	linewidth=1.5pt,
	roundcorner=10pt,
	backgroundcolor=darkgray!10,
	innerbottommargin=0pt
}

% Image frame
\newtcbox{\fboxC}{
	colback=backgroundC,
	colframe=popUp,
	arc=10pt,
	boxrule=3pt,
	boxsep=0pt, % Change the padding here
	nobeforeafter
}

% ----------------------------------------------------------
% PAGE SETTINGS

% Background 
\newcommand{\background}[0]{\begin{tikzpicture}[remember picture,overlay]
		\fill[backgroundC] (-2,2) rectangle (25cm, -550);
\end{tikzpicture}}

\newcommand{\backgroundC}[0]{\begin{tikzpicture}[remember picture,overlay]
		\fill[backgroundCC] (-2,2) rectangle (25cm, -550);
\end{tikzpicture}}

% Page width 
\newcommand{\anchoPag}[0]{20cm}

% ----------------------------------------------------------
% FONT

% General
\usepackage{tgbonum}        % Font
\usepackage{listings}       % Code typesetting
\usepackage[spanish]{babel} % Load Spanish
\selectlanguage{spanish}    % Select Spanish
\usepackage{enumitem}
\usepackage{bookmark}

\setlist[itemize]{leftmargin=1.2em, itemsep=0.35em, topsep=0.35em}

% --- Table helpers ---
\newcolumntype{L}[1]{>{\raggedright\arraybackslash}p{#1}}
\newcolumntype{Y}{>{\raggedright\arraybackslash}X}
\newcolumntype{C}{>{\centering\arraybackslash}X}
\renewcommand{\arraystretch}{1.1}

% Python style
\lstdefinestyle{python}{
	language=Python,
	basicstyle=\ttfamily\small,
	commentstyle=\color{green!50!black},
	keywordstyle=\color{blue},
	numberstyle=\tiny\color{gray},
	numbers=left,
	morekeywords={>, <},
	breakatwhitespace=false,
	showstringspaces=false,
	showtabs=false,
	showspaces=false
}

% ----------------------------------------------------------
% HYPERLINKS

% General
\usepackage{hyperref}       
\hypersetup{
	colorlinks=true,
	linkcolor=links,
	filecolor=magenta,      
	urlcolor=brown,
}

% Custom commands 

% Large link
\newcommand{\bigLink}[2]{\begin{center} \fboxC{\LARGE{\href{#1}{#2}}}\end{center}}

% Small link
\newcommand{\smallLink}[2]{\begin{center}\fboxC{\href{#1}{#2}}\end{center}}

% Bold link
\newcommand{\bfLink}[2]{\href{#1}{\textbf{#2}}}


% Small URL
\newcommand{\smallUrl}[1]{\begin{center}\fboxC{\url{#1}}\end{center}}


% ----------------------------------------------------------
% CUSTOM COMMANDS FOR FIGURES

\newcommand{\espacioImagenes}[0]{-1.2cm}

% Without frame
\newcommand{\fig}[3][\espacioImagenes]{
	\hspace*{#1}
	\centering
	\includegraphics[width=#2\textwidth]{#3}
}

% With frame
\newcommand{\ffig}[2]{\begin{figure}[h]
		\hspace*{\espacioImagenes}
		\centering
		\fbox{\includegraphics[width=#1\textwidth]{#2}}
\end{figure}}

% Hyperlink with frame
\newcommand{\hfig}[3]{\begin{figure}[h]
		\hspace*{-1.4cm}
		\centering
		\color{popUp}
		\fboxC{\href{#1}{\includegraphics[width=#2\textwidth]{#3}}}
	\end{figure}
}

% Hyperlink without frame
\newcommand{\hffig}[3]{\begin{figure}[h]
		\hspace*{-1.1cm}
		\centering
		\color{popUp}
		\href{#1}{\includegraphics[width=#2\textwidth]{#3}}
	\end{figure}
}

% ----------------------------------------------------------

% Start and Contents
\newcommand{\cuadro}[1]{
	\begin{mdframed}[style=estiloGeneral]
		#1 
	\end{mdframed}
}

% Explanation video image
\newcommand{\linkExplicacion}[1]{
	\hffig{#1}{0.5}{principal/videoExplicacion}
	\vspace{-0.5cm}
}

\newcommand{\subSecLink}[2]{
	\subsubsection*{\href{#1}{\textbf{#2}}}
}

% Spacing
\newcommand{\esp}[0]{\vspace{4mm}}

% Back to start
\newcommand{\secInicio}[0]{\begin{center}\hyperref[sec:inicio]{ 
			\includegraphics[width=0.1\textwidth]{principal/up}
	}\end{center}
}


\geometry{margin=0.85in}
\AtBeginDocument{\small}

\newcommand{\ExamNameField}{\noindent\textbf{Name:}\ \rule{0.7\linewidth}{0.4pt}\par\medskip}

\newcommand{\ExamTitleBlock}[3]{%
	\begin{center}
		\Large\textbf{#1}\\
		\textbf{#2}%
		\if\relax\detokenize{#3}\relax\else\\\textbf{#3}\fi
	\end{center}
	\vspace{0.5em}
}

\newcommand{\ExamSection}[1]{\par\medskip\textbf{#1}\par\smallskip}

\newenvironment{ExamCriteria}{%
	\begin{itemize}[leftmargin=1.6em, itemsep=0.3em, topsep=0.2em]
}{%
	\end{itemize}
}

\newenvironment{ExamProblems}{%
	\begin{enumerate}[label=\textbf{P\arabic*}, leftmargin=0pt, labelsep=0.6em, itemindent=2.2em, itemsep=0.8em]
}{%
	\end{enumerate}
}

\begin{document}
	\ExamTitleBlock{11th grade}{Learning evidence 2.1: Confidence Interval Planning (Solutions)}{}
	
	\ExamSection{Evaluated criteria}
	\begin{ExamCriteria}
		\item C1: Compute the sample mean and sample standard deviation.
		\item C2: Distinguish between sample statistics and population parameters.
		\item C3: Explain why interval estimation is preferred over a single point estimate.
		\item C5: Construct a X\% confidence interval using known population standard deviation.
		\item C4: Interpret the meaning of a X\% confidence interval in context.
	\end{ExamCriteria}

	\ExamSection{Problems}
	\begin{ExamProblems}
		\newpage
		\item
		\subsection*{Problem description}
		A city budgeting office wants to estimate the average weekly transit subsidy per rider for two pilot zones.
		The two samples are different random samples drawn from the same underlying population of weekly subsidy amounts.
		Zone A recorded the following sample of weekly subsidy amounts (USD):

		\begin{itemize}
			\item Zone A sample (\(n_A = 6\)): 18, 22, 19, 21, 20, 17.
			\item Zone B sample (\(n_B = 3\)): 24, 26, 23.
		\end{itemize}

		An annual audit provides the population standard deviation for this population: \(\sigma = 3.5\).
		For classroom purposes, suppose the true population mean weekly subsidy for this population is \(\mu = 21\).
		Construct and interpret a 95\% confidence interval for the population mean weekly subsidy in each zone.

		\subsection*{C1}
		First, we calculate the sample means.
		\[
		\bar{x}_A = \frac{\sum x_A}{n_A} \qquad \longrightarrow \qquad \sum x_A = 18 + 22 + 19 + 21 + 20 + 17 = 117 \qquad \longrightarrow \qquad \bar{x}_A = \frac{117}{6} = 19.5
		\]
		\[
		\bar{x}_B = \frac{\sum x_B}{n_B}  \qquad \longrightarrow \qquad \sum x_B = 24 + 26 + 23 = 73  \qquad \longrightarrow \qquad \bar{x}_B = \frac{73}{3} \approx 24.33
		\]
		Second, we calculate the sample variance and then the sample standard deviation.
		\[
		s_A^2 = \frac{\sum (x_A-\bar{x}_A)^2}{n_A-1} \qquad \qquad s_B^2 = \frac{\sum (x_B-\bar{x}_B)^2}{n_B-1}
		\]
		\[
		\sum (x_A-\bar{x}_A)^2 = (18-19.5)^2 + (22-19.5)^2 + (19-19.5)^2 + (21-19.5)^2 + (20-19.5)^2 + (17-19.5)^2 = 17.5
		\]
		\[
		s_A^2 = \frac{17.5}{6-1} = 3.5 \qquad \longrightarrow \qquad s_A = \sqrt{s_A^2} = \sqrt{3.5} \approx 1.87
		\]
		\[
		\sum (x_B-\bar{x}_B)^2 = (24-24.33)^2 + (26-24.33)^2 + (23-24.33)^2 \approx 4.67
		\]
		\[
		s_B^2 = \frac{4.67}{3-1} \approx 2.33  \qquad \longrightarrow \qquad s_B = \sqrt{s_B^2} \approx 1.53
		\]
		\subsection*{C2}
		The population parameters are $\mu = 21$ and $\sigma = 3.5$, while the sample statistics are $\bar{x}_A = 19.5$, $s_A \approx 1.87$, $\bar{x}_B \approx 24.33$, and $s_B \approx 1.53$. The values $\bar{x}_A$ and $\bar{x}_B$ estimate the population mean $\mu$, and $s_A$ and $s_B$ estimate the spread even though $\sigma$ is known for this exercise. This comparison shows that statistical inference relies on sample statistics when population values are unknown or only partially known.

		\subsection*{C3}
		A single point estimate can miss the true mean because of sampling variability in the weekly subsidy amounts. A confidence interval is preferred because it accounts for that uncertainty and gives a range of plausible values for $\mu$. For budgeting in the two zones, a range of plausible means is more informative than one value when setting subsidy policies.

		\subsection*{C5}
		With known $\sigma = 3.5$ and $n_A = 6$, For 95\% confidence, $z_{0.025}=1.96$.
		\[
		SE_A = \frac{\sigma}{\sqrt{n_A}}  \qquad \longrightarrow \qquad SE_A = \frac{3.5}{\sqrt{6}} \approx 1.43  \qquad \longrightarrow \qquad  E_A = z_{0.025} \cdot SE_A = 1.96(1.43) \approx 2.80
		\]
		\[
		\mu \in \bar{x}_A \pm E_A  \qquad \longrightarrow \qquad \mu \in 19.5 \pm 2.80 = [19.5 - 2.80, 19.5 + 2.80] = [16.70, 22.30]
		\]
		With known $\sigma = 3.5$ and $n_B = 3$,
		\[
		SE_B = \frac{\sigma}{\sqrt{n_B}}  \qquad \longrightarrow \qquad SE_B = \frac{3.5}{\sqrt{3}} \approx 2.02  \qquad \longrightarrow \qquad E_B = z_{0.025} \cdot SE_B = 1.96(2.02) \approx 3.96
		\]
		\[
		\mu \in \bar{x}_B \pm E_B   \qquad \longrightarrow \qquad \mu \in 24.33 \pm 3.96 = [24.33 - 3.96, 24.33 + 3.96] = [20.37, 28.29]
		\]

		\subsection*{C4}
		We are 95\% confident the true mean weekly subsidy in the population is between about 16.70 and 22.30 dollars based on Sample A, and 95\% confident it is between about 20.37 and 28.29 dollars based on Sample B. The confidence level means that in repeated sampling about 95\% of the resulting intervals would capture $\mu$, and it does not mean there is a 95\% chance that $\mu$ lies in one specific interval. Since the classroom value $\mu = 21$ falls in both intervals, the samples support that value as plausible. Decision makers can also compare each interval with a target subsidy, such as checking whether values above 23 dollars appear plausible, to evaluate budgeting claims.
		
		\newpage
		\item
		\subsection*{Problem description}
		A regional warehouse studies the number of orders processed per hour to plan staffing.
		A random sample of 54 hours is summarized in grouped form below. The operations team also reports a known population standard deviation of \(\sigma = 6.2\) orders per hour.
		For academic purposes only, assume the true population mean order rate is \(\mu = 22\) orders per hour.
		The grouped data pairs list orders per hour first and the frequency (number of hours) second: \(14 \rightarrow 8\), \(18 \rightarrow 16\), \(22 \rightarrow 20\), and \(26 \rightarrow 10\).
		Construct and interpret a 90\% confidence interval and a 98\% confidence interval for the population mean orders per hour.

		\subsection*{C1}
		Using grouped values $x$ with frequencies $f$.
		\[
		n = \sum f   \qquad \longrightarrow \qquad n = 8 + 16 + 20 + 10 = 54
		\]
		\[
		\bar{x} = \frac{\sum f x}{n} \qquad \longrightarrow \qquad \sum f x = 8(14) + 16(18) + 20(22) + 10(26) = 1100  \qquad \longrightarrow \qquad \bar{x} = \frac{1100}{54} \approx 20.37
		\]
		Second, we calculate the sample variance and then the sample standard deviation $s$.
		\[
		s^2 = \frac{\sum f(x-\bar{x})^2}{n-1} \qquad \longrightarrow \qquad \sum f(x-\bar{x})^2 \approx 8(14-20.37)^2 + 16(18-20.37)^2 + 20(22-20.37)^2 + 10(26-20.37)^2
		\]
		\[
		\sum f(x-\bar{x})^2 \approx 324.65 + 89.90 + 53.11 + 316.93 = 784.59
		\]
		\[
		s^2 = \frac{784.59}{54-1} \approx 14.80  \qquad \longrightarrow \qquad s = \sqrt{s^2} \approx 3.85
		\]
		\subsection*{C2}
		The population parameters are $\mu$ and $\sigma = 6.2$, while the sample statistics are $\bar{x} \approx 20.37$ and $s \approx 3.85$. The sample mean $\bar{x}$ estimates the population mean $\mu$, and the sample standard deviation $s$ estimates $\sigma$ even though $\sigma$ is known here. This comparison shows that inference for the mean depends on sample statistics when the population mean is not directly observed.

		\subsection*{C3}
		The sample summarizes only 54 hours, so the true mean could differ from the point estimate because of sampling variability. An interval estimate is preferred because it captures that uncertainty and provides a range of plausible values for $\mu$. In staffing decisions, a plausible range for the order rate is more useful than a single number.

		\subsection*{C5}
		With known $\sigma = 6.2$ and $n=54$,
		\[
		SE = \frac{\sigma}{\sqrt{n}} \qquad \longrightarrow \qquad SE = \frac{6.2}{\sqrt{54}} \approx 0.84
		\]
		For 90\% confidence, $z_{0.05}=1.645$.
		\[
		E_{90} = z_{0.05} \cdot SE \qquad \longrightarrow \qquad E_{90} = 1.645(0.84) \approx 1.39
		\]
		\[
		\mu \in \bar{x} \pm E_{90} \qquad \longrightarrow \qquad \mu \in 20.37 \pm 1.39 = [20.37 - 1.39, 20.37 + 1.39] = [18.98, 21.76]
		\]
		For 98\% confidence, $z_{0.01}=2.326$.
		\[
		E_{98} = z_{0.01} \cdot SE \qquad \longrightarrow \qquad E_{98} = 2.326(0.84) \approx 1.96
		\]
		\[
		\mu \in \bar{x} \pm E_{98} \qquad \longrightarrow \qquad \mu \in 20.37 \pm 1.96 = [20.37 - 1.96, 20.37 + 1.96] = [18.41, 22.33]
		\]

		\subsection*{C4}
		We are 90\% confident the true mean orders processed per hour is between about 18.98 and 21.76 orders, and 98\% confident it is between about 18.41 and 22.33 orders. The confidence level means that in repeated sampling about 90\% or 98\% of such intervals would contain $\mu$, and it does not mean there is a 90\% or 98\% chance that $\mu$ lies in this one interval. Decision makers can compare the intervals with a staffing benchmark such as 22 orders per hour to judge whether the data support a claim about typical performance.

		\newpage
		\item
		\subsection*{Problem description}
		A logistics firm monitors the time (in minutes) required to load containers at a port.
		A sample of 55 loading times is grouped into class intervals below. The firm has a historical estimate of the population standard deviation, \(\sigma = 5.8\) minutes.
		The grouped intervals and frequencies are 25--29 minutes (18), 30--34 minutes (20), and 35--39 minutes (17).
		Construct and interpret a 95\% confidence interval for the population mean loading time.

		\subsection*{C1}
		Using grouped midpoints $x$ with frequencies $f$.
		\[
		n = \sum f \qquad \longrightarrow \qquad n = 18 + 20 + 17 = 55
		\]
		\[
		\bar{x} = \frac{\sum f x}{n} \qquad \longrightarrow \qquad \sum f x = 18(27) + 20(32) + 17(37) = 1755 \qquad \longrightarrow \qquad \bar{x} = \frac{1755}{55} \approx 31.91
		\]
		Second, we calculate the sample variance and then the sample standard deviation $s$.
		\[
		s^2 = \frac{\sum f(x-\bar{x})^2}{n-1} \qquad \longrightarrow \qquad \sum f(x-\bar{x})^2 \approx 18(27-31.91)^2 + 20(32-31.91)^2 + 17(37-31.91)^2
		\]
		\[
		\sum f(x-\bar{x})^2 \approx 874.55
		\]
		\[
		s^2 = \frac{874.55}{55-1} \approx 16.20  \qquad \longrightarrow \qquad s = \sqrt{s^2} \approx 4.02
		\]
		\subsection*{C2}
		The population parameters are $\mu$ and $\sigma = 5.8$, while the sample statistics are $\bar{x} \approx 31.91$ and $s \approx 4.02$. The sample mean $\bar{x}$ estimates the population mean $\mu$, and the sample standard deviation $s$ estimates $\sigma$ even though $\sigma$ is given. This comparison shows that we still rely on sample statistics to infer $\mu$ when the population mean is unknown.

		\subsection*{C3}
		Even with 55 observations, a point estimate does not show how much the mean could vary because of sampling variability. An interval estimate is preferred because it summarizes that uncertainty by giving a plausible range for $\mu$. For planning port operations, a range of likely loading times is more informative than a single value.

		\subsection*{C5}
		With known $\sigma = 5.8$ and $n=55$,
		\[
		SE = \frac{\sigma}{\sqrt{n}} \qquad \longrightarrow \qquad SE = \frac{5.8}{\sqrt{55}} \approx 0.78
		\]
		For 95\% confidence, $z_{0.025}=1.96$.
		\[
		E = z_{0.025} \cdot SE \qquad \longrightarrow \qquad E = 1.96(0.78) \approx 1.53
		\]
		\[
		\mu \in \bar{x} \pm E \qquad \longrightarrow \qquad \mu \in 31.91 \pm 1.53 = [31.91 - 1.53, 31.91 + 1.53] = [30.38, 33.44]
		\]

		\subsection*{C4}
		We are 95\% confident the true mean loading time is between about 30.38 and 33.44 minutes. The confidence level means that in repeated sampling about 95\% of such intervals would contain $\mu$, and it does not mean there is a 95\% chance that $\mu$ lies in this one interval. Managers can compare this interval with a performance threshold such as 32 minutes to decide whether current loading times meet goals or whether a claim about faster loading is plausible.
	\end{ExamProblems}
\end{document}
