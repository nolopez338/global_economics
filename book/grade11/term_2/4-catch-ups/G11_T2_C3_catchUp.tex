\documentclass[12pt]{article}
\usepackage[a4paper,left=1.2cm,right=1.2cm,top=1.5cm,bottom=1.5cm]{geometry}
\usepackage{graphicx}
\usepackage{xcolor}
\usepackage{array}
\usepackage{tabularx}
\usepackage{multicol}
\usepackage[T1]{fontenc}
\usepackage[utf8]{inputenc}
\setlength{\parindent}{0pt}
\setlength{\tabcolsep}{6pt}
\renewcommand{\arraystretch}{1.15}
\newcolumntype{Y}{>{\raggedright\arraybackslash}m{\dimexpr0.55\textwidth-2\tabcolsep-2\arrayrulewidth\relax}}
\newcolumntype{Z}{>{\raggedright\arraybackslash}m{\dimexpr0.45\textwidth-2\tabcolsep-2\arrayrulewidth\relax}}
\newcolumntype{C}[1]{>{\centering\arraybackslash}m{#1}}
\newcommand{\SubsectionBox}[1]{%
\noindent\colorbox{gray!30}{\parbox{\linewidth}{\textbf{#1}}}\par\vspace{0.35cm}%
}
\newcommand{\CellCenter}[1]{\parbox{\linewidth}{\centering #1}}

\begin{document}

\noindent
\begin{tabularx}{\textwidth}{|C{2.8cm}|C{\dimexpr\textwidth-6cm-4\tabcolsep-4\arrayrulewidth\relax}|C{2.8cm}|}
\hline
\centering\vspace{3mm}\includegraphics[width=2.5cm]{../../preamble/logo.png}&
\CellCenter{\vspace{-5mm}\textbf{GLOBAL ECONOMICS}\par\textbf{GRADE: 11TH}\par\textbf{CATCH-UP}\par\textbf{CRITERION C3}\par\textbf{TEACHER'S NAME: Nicolás López Cuéllar}}&
\CellCenter{\textbf{SECOND TERM}\par\textbf{2025--2026}}\\
\hline
\end{tabularx}

\vspace{0.5cm}
\noindent
\begin{tabular}{|Y|Z|}
\hline
{\footnotesize\textbf{Learning objective:} Explain point estimation versus interval estimation in confidence interval planning contexts.}&
{\footnotesize\textbf{Assessment criteria:}\par C3: Explain why interval estimation is preferred over a single point estimate.}\\
\hline
\end{tabular}

\begin{multicols}{2}
	\SubsectionBox{Criteria assessment}\vspace{-0.25cm}
	Each assessment criterion is evaluated across the problems in this catch-up exam. A criterion is considered passed when it is correctly activated in 9 of the 10 problems of this activity.

	\vspace{0.25cm}
\SubsectionBox{Problem 1}
\subsection*{Subscription Spending Point Vs Interval}
A music subscription app wants to estimate average monthly spending for all users in their 20s in one country.
		The team observed a sample of 25 users and computed an average of 48 USD.

		Identify what was observed and what is being estimated.
		Then explain why one value may not be enough for a financial decision.

\vspace{0.5cm}
\SubsectionBox{Problem 2}
\subsection*{Transport Spending Estimate Limits}
An online budgeting platform studies monthly transport spending of young adults in a large city.
		From a sample of 40 users, the average spending was 132 USD.

		State the observed sample information and the broader quantity being estimated.
		Explain why reporting only 132 USD can be limited.

\vspace{0.5cm}
\SubsectionBox{Problem 3}
\subsection*{Rider Income Decision Risk}
A food delivery company reviews weekly gig income for riders in their 20s.
		It observed 18 riders and found an average weekly income of 515 USD.
		Management plans to cut bonus support because 515 USD looks high.

		What could go wrong if the decision is based only on this single observed average?

\vspace{0.5cm}
\SubsectionBox{Problem 4}
\subsection*{Product Sales Estimate Risk}
A small online store tracks daily sales from a new product line.
		From 12 observed days, the average daily revenue was 390 USD.
		The founder wants to sign a one-year advertising contract based on this value.

		Explain the risk of making this decision using only the point estimate.

\vspace{0.5cm}
\SubsectionBox{Problem 5}
\subsection*{Rent Mean Interval Interpretation}
A rent advisory service estimates average monthly rent paid by young professionals.
		From a sample of 60 apartments, the sample average rent is 1,620 USD.
		A 95\% confidence interval for the population mean rent is [1,540, 1,700] USD.

		Compare the point and interval information.

\vspace{0.5cm}
\SubsectionBox{Problem 6}
\subsection*{Designer Income Interval Advantage}
A freelancing platform estimates monthly income of beginner designers in their 20s.
		A sample of 35 freelancers gives an average of 980 USD.
		A reported confidence interval for the population mean is [880, 1,080] USD.

		Explain how the interval addresses limitations of the single average.

\vspace{0.5cm}
\SubsectionBox{Problem 7}
\subsection*{Rent Commitment Decision Uncertainty}
A recent graduate is choosing between two rent commitments.
		They saw data from a sample of 30 tenants with an average monthly rent of 1,450 USD and interval [1,280, 1,620] USD for the population mean.

		Scenario A decides using only 1,450 USD.
		Scenario B decides using the interval estimate.
		Which approach is safer and why?

\vspace{0.5cm}
\SubsectionBox{Problem 8}
\subsection*{Salary Negotiation Under Uncertainty}
A young professional is negotiating salary after seeing market data.
		From a sample of 22 job offers, the average offer is 54,000 USD with interval [49,000, 59,000] USD for the population mean offer.

		Scenario A argues from the point estimate only.
		Scenario B uses the interval estimate in negotiation planning.
		Which approach is safer and what consequences follow from ignoring uncertainty?

\vspace{0.5cm}
\SubsectionBox{Problem 9}
\subsection*{Investment Return Sample Size Effects}
An investment app tracks monthly returns for beginners.
		Sample A observed 20 investors and produced a mean return of 85 USD with interval [55, 115] USD.
		Sample B observed 100 investors and produced a mean return of 88 USD with interval [78, 98] USD.

		Explain how sample size changes confidence in population conclusions.
		Why can relying only on point estimates hide important information?

\vspace{0.5cm}
\SubsectionBox{Problem 10}
\subsection*{Sales Study Precision Comparison}
A side-business coach compares two studies of monthly sales for creators in their 20s.
		Both studies report the same sample mean, 1,100 USD.
		Study X uses 16 observations and gives interval [900, 1,300] USD.
		Study Y uses 144 observations and gives interval [1,040, 1,160] USD.

		Analyze why the intervals differ even with the same point estimate.
		Why could relying only on the point estimate hide critical decision information?

\end{multicols}

\end{document}
