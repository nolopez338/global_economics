\documentclass[12pt]{article}
\usepackage[a4paper,left=1.2cm,right=1.2cm,top=1.5cm,bottom=1.5cm]{geometry}
\usepackage{graphicx}
\usepackage{xcolor}
\usepackage{array}
\usepackage{tabularx}
\usepackage{multicol}
\usepackage[T1]{fontenc}
\usepackage[utf8]{inputenc}
\setlength{\parindent}{0pt}
\setlength{\tabcolsep}{6pt}
\renewcommand{\arraystretch}{1.15}
\newcolumntype{Y}{>{\raggedright\arraybackslash}m{\dimexpr0.30\textwidth-2\tabcolsep-2\arrayrulewidth\relax}}
\newcolumntype{Z}{>{\raggedright\arraybackslash}m{\dimexpr0.70\textwidth-2\tabcolsep-2\arrayrulewidth\relax}}
\newcolumntype{C}[1]{>{\centering\arraybackslash}m{#1}}
\newcommand{\SubsectionBox}[1]{%
\noindent\colorbox{gray!30}{\parbox{\linewidth}{\textbf{#1}}}\par\vspace{0.35cm}%
}
\newcommand{\CellCenter}[1]{\parbox{\linewidth}{\centering #1}}

\begin{document}

\noindent
\begin{tabularx}{\textwidth}{|C{2.8cm}|C{\dimexpr\textwidth-6cm-4\tabcolsep-4\arrayrulewidth\relax}|C{2.8cm}|}
\hline
\centering\vspace{3mm}\includegraphics[width=2.5cm]{../../preamble/logo.png}&
\CellCenter{\vspace{-5mm}\textbf{GLOBAL ECONOMICS}\par\textbf{GRADE: 11TH}\par\textbf{CATCH-UP}\par\textbf{CRITERION C7}\par\textbf{TEACHER'S NAME: Nicolás López Cuéllar}}&
\CellCenter{\textbf{SECOND TERM}\par\textbf{2025--2026}}\\
\hline
\end{tabularx}

\vspace{0.5cm}
\noindent
\begin{tabular}{|Y|Z|}
\hline
{\footnotesize\textbf{Learning objective:} Construct and interpret a confidence interval for a population proportion.}&
{\footnotesize\textbf{Assessment criteria:}\par C7: Constructs the confidence interval for a proportion.}\\
\hline
\end{tabular}

\begin{multicols}{2}
	\SubsectionBox{Criteria assessment}\vspace{-0.25cm}
	Each assessment criterion is evaluated across the problems in this catch-up exam. A criterion is considered passed when it is correctly activated in 9 of the 10 problems of this activity.

	\vspace{0.25cm}
\SubsectionBox{1. Problem 1 --- Problem description}\vspace{-0.25cm}
A digital bank wants to estimate the proportion of young adult clients (ages 20--29) who use automatic monthly savings transfers. In a random sample of \(n=120\) clients, \(x=66\) report using automatic transfers. Construct and interpret a \(90\%\) confidence interval for the population proportion.

\vspace{0.25cm}
\SubsectionBox{2. Problem 2 --- Problem description}\vspace{-0.25cm}
A rideshare platform wants to estimate the proportion of drivers in their 20s who bought private accident insurance this year. In a sample of \(n=200\) drivers, \(x=92\) purchased insurance. Construct \(90\%\), \(95\%\), and \(99\%\) confidence intervals for the population proportion, and give a brief interpretation.

\vspace{0.25cm}
\SubsectionBox{3. Problem 3 --- Problem description}\vspace{-0.25cm}
A fintech app wants to estimate the proportion of users in their 20s who pay credit card balances in full each month. In a sample of \(n=260\) users, \(x=182\) report paying in full. Construct and interpret a \(95\%\) confidence interval. Then answer: is a benchmark value of \(p=0.75\) plausible based on this interval?

\vspace{0.25cm}
\SubsectionBox{4. Problem 4 --- Problem description}\vspace{-0.25cm}
A micro-investing platform is evaluating a support program. It wants the proportion of young clients who invest every month to be at least \(60\%\). After one cycle, a random sample of \(n=320\) clients finds \(x=173\) monthly investors. Construct a \(98\%\) confidence interval for the population proportion and decide whether the \(60\%\) benchmark is plausible. Justify your decision using the interval bounds.

\vspace{0.25cm}
\SubsectionBox{5. Problem 5 --- Problem description}\vspace{-0.25cm}
A budgeting startup compares two groups of users in their 20s.
\begin{itemize}
  \item Group A (users who completed a financial bootcamp): \(n=250\), \(x=170\) keep an emergency fund.
  \item Group B (users without bootcamp): \(n=250\), \(x=142\) keep an emergency fund.
\end{itemize}
Construct a \(95\%\) confidence interval for each group proportion. Compare the intervals and discuss what overlap or non-overlap suggests about relative plausibility of higher emergency-fund behavior.

\vspace{0.25cm}
\SubsectionBox{6. Problem 6 --- Problem description}\vspace{-0.25cm}
A national survey studies young workers and compares two sectors.
\begin{itemize}
  \item Sector 1 (gig logistics): \(n=300\), \(x=126\) contributed to retirement savings last month.
  \item Sector 2 (junior office jobs): \(n=280\), \(x=151\) contributed last month.
\end{itemize}
Construct \(90\%\) confidence intervals for both sector proportions. Which sector appears safer in terms of consistently higher retirement-saving participation, and which estimate is more uncertain?

\vspace{0.25cm}
\SubsectionBox{7. Problem 7 --- Problem description}\vspace{-0.25cm}
A payment app tracks budgeting-rule usage (the 50-30-20 rule) among users in their 20s in two cities.
\begin{itemize}
  \item City A: \(n=360\), \(x=198\) use the rule.
  \item City B: \(n=340\), \(x=170\) use the rule.
\end{itemize}
For each city, construct \(90\%\) and \(95\%\) confidence intervals. Then explain how the decision about whether City A is above \(55\%\) changes when the confidence level increases.

\vspace{0.25cm}
\SubsectionBox{8. Problem 8 --- Problem description}\vspace{-0.25cm}
An online broker compares two periods for the same age group.
\begin{itemize}
  \item Period 1: \(n=420\), \(x=231\) users made at least one ETF purchase.
  \item Period 2: \(n=410\), \(x=254\) users made at least one ETF purchase.
\end{itemize}
Construct a \(95\%\) confidence interval for each period. Compare uncertainty and discuss whether adoption appears to have increased.

\vspace{0.25cm}
\SubsectionBox{9. Problem 9 --- Problem description}\vspace{-0.25cm}
A policy team evaluates three training channels that teach debt-management skills to young adults.
\begin{itemize}
  \item Channel A: \(n=280\), \(x=168\) participants reduced debt for three consecutive months.
  \item Channel B: \(n=300\), \(x=171\) participants reduced debt for three consecutive months.
  \item Channel C: \(n=260\), \(x=130\) participants reduced debt for three consecutive months.
\end{itemize}
Construct \(90\%\) and \(99\%\) confidence intervals for all three channels. Then recommend which channel should be prioritized if the target is to keep the true success proportion around or above \(60\%\), considering uncertainty.

\vspace{0.25cm}
\SubsectionBox{10. Problem 10 --- Problem description}\vspace{-0.25cm}
A youth-finance nonprofit pilots three versions of a savings habit program.
\begin{itemize}
  \item Program X: \(n=350\), \(x=221\) participants stayed on a weekly savings plan for 12 weeks.
  \item Program Y: \(n=330\), \(x=191\) participants stayed on plan.
  \item Program Z: \(n=340\), \(x=180\) participants stayed on plan.
\end{itemize}
Construct \(95\%\) and \(99\%\) confidence intervals for each program. Then give a decision recommendation if funding should prioritize programs with a plausibly highest retention proportion and relatively stable uncertainty.

\end{multicols}

\end{document}
