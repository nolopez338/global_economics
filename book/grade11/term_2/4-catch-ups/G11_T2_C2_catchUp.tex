\documentclass[12pt]{article}
\usepackage[a4paper,left=1.2cm,right=1.2cm,top=1.5cm,bottom=1.5cm]{geometry}
\usepackage{graphicx}
\usepackage{xcolor}
\usepackage{array}
\usepackage{tabularx}
\usepackage{multicol}
\usepackage[T1]{fontenc}
\usepackage[utf8]{inputenc}
\setlength{\parindent}{0pt}
\setlength{\tabcolsep}{6pt}
\renewcommand{\arraystretch}{1.15}
\newcolumntype{Y}{>{\raggedright\arraybackslash}m{\dimexpr0.55\textwidth-2\tabcolsep-2\arrayrulewidth\relax}}
\newcolumntype{Z}{>{\raggedright\arraybackslash}m{\dimexpr0.45\textwidth-2\tabcolsep-2\arrayrulewidth\relax}}
\newcolumntype{C}[1]{>{\centering\arraybackslash}m{#1}}
\newcommand{\SubsectionBox}[1]{%
\noindent\colorbox{gray!30}{\parbox{\linewidth}{\textbf{#1}}}\par\vspace{0.35cm}%
}
\newcommand{\CellCenter}[1]{\parbox{\linewidth}{\centering #1}}

\begin{document}

\noindent
\begin{tabularx}{\textwidth}{|C{2.8cm}|C{\dimexpr\textwidth-6cm-4\tabcolsep-4\arrayrulewidth\relax}|C{2.8cm}|}
\hline
\centering\vspace{3mm}\includegraphics[width=2.5cm]{../../preamble/logo.png}&
\CellCenter{\vspace{-5mm}\textbf{GLOBAL ECONOMICS}\par\textbf{GRADE: 11TH}\par\textbf{CATCH-UP}\par\textbf{CRITERION C2}\par\textbf{TEACHER'S NAME: Nicolás López Cuéllar}}&
\CellCenter{\textbf{SECOND TERM}\par\textbf{2025--2026}}\\
\hline
\end{tabularx}

\vspace{0.5cm}
\noindent
\begin{tabular}{|Y|Z|}
\hline
{\footnotesize\textbf{Learning objective:} Distinguish clearly between sample statistics and population parameters in confidence interval planning contexts.}&
{\footnotesize\textbf{Assessment criteria:}\par C2: Distinguish between sample statistics and population parameters.}\\
\hline
\end{tabular}

\begin{multicols}{2}
	\SubsectionBox{Criteria assessment}\vspace{-0.25cm}
	Each assessment criterion is evaluated across the problems in this catch-up exam. A criterion is considered passed when it is correctly activated in 9 of the 10 problems of this activity.

	\vspace{0.25cm}
\SubsectionBox{Problem 1}
\subsection*{Studio Rent Record Comparison}
A rental platform has years of records on monthly studio rent for young professionals in one city area, and across that full record the typical monthly level is 1850 USD.
		The same full platform dataset reports overall variability of 10000 in (USD)$^2$.
		In a recent audit of listings from one shorter period, the observed average in the reviewed group is 1810 USD, with variability summarized by 8100 in (USD)$^2$.
		From the context above, identify which values describe the long-run behavior of the entire population and which values describe only the reviewed sample.

\vspace{0.5cm}
\SubsectionBox{Problem 2}
\subsection*{Graduate Offer Mean Assessment}
A recently reviewed batch of new offers has an observed average of 53100 USD, with spread reported as 5500 USD.
		A company keeps a long-term database of first-year salary offers for recent graduates hired into a data support role, and across all offers in that database the central level is 52000 USD.
		The same complete record shows a typical spread of 6000 USD around that value.
		Based on this scenario, determine which quantities refer to the full population over time and which refer only to the reviewed sample.

\vspace{0.5cm}
\SubsectionBox{Problem 3}
\subsection*{Gig Platform Earnings Benchmarks}
Two gig platforms are analyzed separately using their own data systems.
		For Platform A (food delivery), company-wide weekly earning records indicate an overall level of 720 USD, while a recently reviewed rider group has an observed average of 750 USD.
		Its long-run variability is 4900 in (USD)$^2$, and the reviewed group has variability of 6400 in (USD)$^2$.
		For Platform B (freelance design), full-platform records indicate 980 USD as the long-run typical level, while a recent creator group has an observed average of 940 USD.
		Its overall spread is 80 USD, and the reviewed group spread is 70 USD.
		Using the information provided, identify which values are population-level descriptors and which values summarize only the reviewed samples.

\vspace{0.5cm}
\SubsectionBox{Problem 4}
\subsection*{Metro Card Spending Review}
A bank tracks monthly credit card spending for customers in their 20s across one metro area over a long period.
		Two separate recent review groups were then pulled from that same metro customer base.
		Group 1 has an observed average of 1290 USD and a reported spread of 280 USD.
		Group 2 has an observed average of 1410 USD and a reported spread of 320 USD.
		That full customer record is summarized by a long-term typical level of 1350 USD and a typical spread of 300 USD.
		From this description, state which values characterize the entire long-run population and which values belong only to the reviewed groups.

\vspace{0.5cm}
\SubsectionBox{Problem 5}
\subsection*{Emergency Deposit Mean Review}
A savings app monitors monthly deposits into emergency funds made by users in their 20s.
		Across the app's full user history, the deposit level is summarized by a long-term typical value of 420 USD and overall variability of 1600 in (USD)$^2$.
		Three recent review groups, all drawn from that same user base, were summarized as follows.
		Group A has an observed average of 400 USD, and its variability is 900 in (USD)$^2$.
		Group B has an observed average of 435 USD, and its variability is 2500 in (USD)$^2$.
		Group C has an observed average of 415 USD, and its variability is 3600 in (USD)$^2$.
		Based on the context, indicate which quantities describe the full user population in the long run and which quantities describe only the reviewed samples.

\vspace{0.5cm}
\SubsectionBox{Problem 6}
\subsection*{Subscription Spending Interval Check}
A streaming platform studies monthly subscription spending by young adults using both long-term company records and a recent account review.
		For one reviewed group, the 95\% confidence interval for that group's average spending runs from 58 USD to 74 USD.
		Across the full platform history, variability is summarized by 400 in (USD)$^2$, while the reviewed group's spread is reported as 24 USD.
		Using the narrative above, identify which reported values represent population information and which represent sample-only information.

\vspace{0.5cm}
\SubsectionBox{Problem 7}
\subsection*{Weekend Trip Spending Assessment}
A travel-budget app tracks weekend trip spending for users in their 20s throughout the year.
		Across all users in its full records, the spending level is summarized by a long-term typical value of 460 USD with typical spread of 70 USD.
		For one recently reviewed user group, the 90\% confidence interval for the group's average runs from 420 USD to 480 USD, and the group's spread is reported as 60 USD.
		From the details given, determine which values correspond to the full population behavior over time and which correspond only to the reviewed sample.

\vspace{0.5cm}
\SubsectionBox{Problem 8}
\subsection*{Online Store Revenue Intervals}
Two small online stores are analyzed separately, each with its own full business records and its own recent review group.
		For one reviewed period at Store A (handmade accessories), the 95\% confidence interval for average daily revenue runs from 240 USD to 300 USD, and variability is reported as 1600 in (USD)$^2$.
		Store A's full records show a long-run daily revenue level of 275 USD and the overall spread of 50 USD.
		For one reviewed period at Store B (digital templates), the 90\% confidence interval for average daily revenue runs from 330 USD to 390 USD, and the spread is 55 USD.
		Store B's full records are summarized by a long-run typical value of 350 USD and variability of 3600 in (USD)$^2$.
		Based on this context, identify which values describe long-run population patterns and which values describe only the reviewed samples.

\vspace{0.5cm}
\SubsectionBox{Problem 9}
\subsection*{Rider Earnings Group Comparison}
A delivery company analyzes weekly rider earnings in one city using complete citywide records plus two recent rider review groups from that same city.
		The full city record is summarized by a long-term typical value of 740 USD and variability of 6400 in (USD)$^2$.
		For Review Group 1, the 95\% confidence interval for average weekly earnings runs from 680 USD to 760 USD.
		For Review Group 2, the 90\% confidence interval for average weekly earnings runs from 700 USD to 780 USD.
		Spread details are reported in mixed form: Review Group 1 has spread of 70 USD, and Review Group 2 has variability of 4900 in (USD)$^2$.
		Using the information above, determine which quantities belong to the full long-run population and which belong only to the reviewed samples.

\vspace{0.5cm}
\SubsectionBox{Problem 10}
\subsection*{Net Deposit Interval Ranking}
An investment app tracks monthly net deposits (deposits minus withdrawals) for users in their 20s.
		Three recent review groups were drawn from that same user base, and their confidence intervals for average net deposits are:
		Sample A (95\%): from 580 USD to 640 USD,
		Sample B (90\%): from 600 USD to 660 USD,
		Sample C (95\%): from 550 USD to 630 USD.
		Their spread summaries are 6400 in (USD)$^2$ for Sample A, 85 USD for Sample B, and 10000 in (USD)$^2$ for Sample C.
		Across the full user base over time, net deposits are summarized by a long-term typical value of 620 USD with spread of 90 USD.
		From this problem statement, identify which values describe population-level behavior over time and which values summarize only the reviewed samples.

\end{multicols}

\end{document}
