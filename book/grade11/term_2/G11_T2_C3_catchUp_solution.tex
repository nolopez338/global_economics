\makeatletter
\def\input@path{{./}{../}{../../}{preamble/}{../preamble/}{../../preamble/}}
\makeatother
% ----------------------------------------------------------
% GENERAL 

% File
\documentclass[11pt]{book}

% Margins
\usepackage[margin=1in]{geometry}

\linespread{1.2}            % Line spacing
\usepackage[utf8]{inputenc}
\usepackage[T1]{fontenc}
\usepackage{lmodern}
\usepackage{microtype}
\setlength{\parindent}{0pt}
\setlength{\parskip}{6pt}
\usepackage{booktabs}

% ----------------------------------------------------------
% TABLES
\usepackage{multicol}
\usepackage{longtable} 
\usepackage{array}
\usepackage{booktabs}
\usepackage{tabularx}
\usepackage{multirow}

% ----------------------------------------------------------
% MATHEMATICS
\usepackage{amsmath}
\usepackage{amssymb}
\usepackage{amsfonts}
\usepackage{mathtools}

% ----------------------------------------------------------
% HIDDEN CONTENT
\usepackage{ifthen}
% Define a boolean switch
\newboolean{explicaciones}
% Set the boolean switch to true or false
% Change to true to show the content

% Explanations
\newcommand{\explicacion}[2]{
	\ifthenelse{\boolean{explicaciones}}{#1}{#2}
}
\newcommand{\mostrarExplicaciones}[1]{\setboolean{explicaciones}{#1}}

% ----------------------------------------------------------
% NUMBERING

\usepackage{fancyhdr}
\pagestyle{empty} % Ensures the entire document has no page numbers

\usepackage{tocloft}
\renewcommand{\cftdot}{} % Remove dots for sections, if any
\renewcommand{\cftsecleader}{\cftdotfill{\cftdotsep}} % Remove dots for sections, if any
\cftpagenumbersoff{section} % Removes page numbers from sections
\cftpagenumbersoff{subsection} % Removes page numbers from subsections

% ----------------------------------------------------------
% IMAGES 

% General settings
\usepackage{graphicx}       % Insert images
\usepackage{float}          % Position images
% \usepackage{subfigure}      % Subfigures
\graphicspath{{imgs}}       % Image location
\usepackage{subcaption}     % Subfigures II
\usepackage{verbatim}

% Figures
\usepackage{tikz}
\usetikzlibrary{arrows.meta,positioning,trees}

% Colors
\usepackage{xcolor}     
\definecolor{popUp}{HTML}{666666}
\definecolor{popUpIn}{HTML}{CED9E0}
\definecolor{backgroundC}{HTML}{D0E8F2}
\definecolor{backgroundCC}{HTML}{FFFFFF}
\definecolor{borders}{HTML}{8c120d}
\definecolor{padding}{HTML}{77D0D7}
\definecolor{links}{HTML}{CC6F5F}

% ----------------------------------------------------------
% FRAMES

% General settings
\usepackage{tcolorbox}
\usepackage{adjustbox}          % Adjusted frame  
\setlength{\fboxrule}{3pt}  % Line width
\setlength{\fboxsep}{3pt}   % Box padding

% General frames
\usepackage{mdframed}   

\mdfdefinestyle{estiloGeneral}{    % General style
	linecolor=black,
	linewidth=1.5pt,
	roundcorner=10pt,
	backgroundcolor=backgroundC,
	innerbottommargin=0pt
}
\mdfdefinestyle{code}{          % Code style
	linecolor=black,
	linewidth=1.5pt,
	roundcorner=10pt,
	backgroundcolor=darkgray!10,
	innerbottommargin=0pt
}

% Image frame
\newtcbox{\fboxC}{
	colback=backgroundC,
	colframe=popUp,
	arc=10pt,
	boxrule=3pt,
	boxsep=0pt, % Change the padding here
	nobeforeafter
}

% ----------------------------------------------------------
% PAGE SETTINGS

% Background 
\newcommand{\background}[0]{\begin{tikzpicture}[remember picture,overlay]
		\fill[backgroundC] (-2,2) rectangle (25cm, -550);
\end{tikzpicture}}

\newcommand{\backgroundC}[0]{\begin{tikzpicture}[remember picture,overlay]
		\fill[backgroundCC] (-2,2) rectangle (25cm, -550);
\end{tikzpicture}}

% Page width 
\newcommand{\anchoPag}[0]{20cm}

% ----------------------------------------------------------
% FONT

% General
\usepackage{tgbonum}        % Font
\usepackage{listings}       % Code typesetting
\usepackage[spanish]{babel} % Load Spanish
\selectlanguage{spanish}    % Select Spanish
\usepackage{enumitem}
\usepackage{bookmark}

\setlist[itemize]{leftmargin=1.2em, itemsep=0.35em, topsep=0.35em}

% --- Table helpers ---
\newcolumntype{L}[1]{>{\raggedright\arraybackslash}p{#1}}
\newcolumntype{Y}{>{\raggedright\arraybackslash}X}
\newcolumntype{C}{>{\centering\arraybackslash}X}
\renewcommand{\arraystretch}{1.1}

% Python style
\lstdefinestyle{python}{
	language=Python,
	basicstyle=\ttfamily\small,
	commentstyle=\color{green!50!black},
	keywordstyle=\color{blue},
	numberstyle=\tiny\color{gray},
	numbers=left,
	morekeywords={>, <},
	breakatwhitespace=false,
	showstringspaces=false,
	showtabs=false,
	showspaces=false
}

% ----------------------------------------------------------
% HYPERLINKS

% General
\usepackage{hyperref}       
\hypersetup{
	colorlinks=true,
	linkcolor=links,
	filecolor=magenta,      
	urlcolor=brown,
}

% Custom commands 

% Large link
\newcommand{\bigLink}[2]{\begin{center} \fboxC{\LARGE{\href{#1}{#2}}}\end{center}}

% Small link
\newcommand{\smallLink}[2]{\begin{center}\fboxC{\href{#1}{#2}}\end{center}}

% Bold link
\newcommand{\bfLink}[2]{\href{#1}{\textbf{#2}}}


% Small URL
\newcommand{\smallUrl}[1]{\begin{center}\fboxC{\url{#1}}\end{center}}


% ----------------------------------------------------------
% CUSTOM COMMANDS FOR FIGURES

\newcommand{\espacioImagenes}[0]{-1.2cm}

% Without frame
\newcommand{\fig}[3][\espacioImagenes]{
	\hspace*{#1}
	\centering
	\includegraphics[width=#2\textwidth]{#3}
}

% With frame
\newcommand{\ffig}[2]{\begin{figure}[h]
		\hspace*{\espacioImagenes}
		\centering
		\fbox{\includegraphics[width=#1\textwidth]{#2}}
\end{figure}}

% Hyperlink with frame
\newcommand{\hfig}[3]{\begin{figure}[h]
		\hspace*{-1.4cm}
		\centering
		\color{popUp}
		\fboxC{\href{#1}{\includegraphics[width=#2\textwidth]{#3}}}
	\end{figure}
}

% Hyperlink without frame
\newcommand{\hffig}[3]{\begin{figure}[h]
		\hspace*{-1.1cm}
		\centering
		\color{popUp}
		\href{#1}{\includegraphics[width=#2\textwidth]{#3}}
	\end{figure}
}

% ----------------------------------------------------------

% Start and Contents
\newcommand{\cuadro}[1]{
	\begin{mdframed}[style=estiloGeneral]
		#1 
	\end{mdframed}
}

% Explanation video image
\newcommand{\linkExplicacion}[1]{
	\hffig{#1}{0.5}{principal/videoExplicacion}
	\vspace{-0.5cm}
}

\newcommand{\subSecLink}[2]{
	\subsubsection*{\href{#1}{\textbf{#2}}}
}

% Spacing
\newcommand{\esp}[0]{\vspace{4mm}}

% Back to start
\newcommand{\secInicio}[0]{\begin{center}\hyperref[sec:inicio]{ 
			\includegraphics[width=0.1\textwidth]{principal/up}
	}\end{center}
}


\geometry{margin=0.85in}
\AtBeginDocument{\small}

\newcommand{\ExamNameField}{\noindent\textbf{Name:}\ \rule{0.7\linewidth}{0.4pt}\par\medskip}

\newcommand{\ExamTitleBlock}[3]{%
	\begin{center}
		\Large\textbf{#1}\\
		\textbf{#2}%
		\if\relax\detokenize{#3}\relax\else\\\textbf{#3}\fi
	\end{center}
	\vspace{0.5em}
}

\newcommand{\ExamSection}[1]{\par\medskip\textbf{#1}\par\smallskip}

\newenvironment{ExamCriteria}{%
	\begin{itemize}[leftmargin=1.6em, itemsep=0.3em, topsep=0.2em]
}{%
	\end{itemize}
}

\newenvironment{ExamProblems}{%
	\begin{enumerate}[label=\textbf{P\arabic*}, leftmargin=0pt, labelsep=0.6em, itemindent=2.2em, itemsep=0.8em]
}{%
	\end{enumerate}
}

\begin{document}
	\ExamTitleBlock{11th grade}{Term 2 Catch-Up: C3 Point vs Interval Estimation (Solutions)}{}

	\ExamSection{Problems}
	\begin{ExamProblems}
		\item
		\subsection*{Problem 1}
		\subsection*{Subscription Spending Point Vs Interval}
		A music subscription app wants to estimate average monthly spending for all users in their 20s in one country.
		The team observed a sample of 25 users and computed an average of 48 USD.

		Identify what was observed and what is being estimated.
		Then explain why one value may not be enough for a financial decision.

		\subsection*{C3}
		The observed value 48 USD comes from a sample of 25 users, so it is a sample statistic.
		The unknown target is the population parameter, which is the true average monthly spending for all users in their 20s in that country.

		The point estimate is 48 USD.
		No interval estimate is provided in this problem.

		A point estimate is useful because it gives a quick summary, but it can hide uncertainty from having only 25 observations.
		If managers set pricing only from 48 USD, they may overestimate what the full population can afford.
		Decision conclusion: use 48 USD as a starting reference, not as a final pricing rule.

		\newpage
		\item
		\subsection*{Problem 2}
		\subsection*{Transport Spending Estimate Limits}
		An online budgeting platform studies monthly transport spending of young adults in a large city.
		From a sample of 40 users, the average spending was 132 USD.

		State the observed sample information and the broader quantity being estimated.
		Explain why reporting only 132 USD can be limited.

		\subsection*{C3}
		The sample statistic is the 132 USD average computed from the 40 observed users.
		The population parameter is the true mean monthly transport spending for all young adults in that city who match the target group.

		The point estimate is 132 USD.
		An interval estimate is not reported.

		The inferential leap is from 40 observed users to the full city population.
		Because different samples could give different averages, a single point estimate can create false certainty.
		Decision conclusion: a transit-finance program should avoid locking budgets using only 132 USD.

		\newpage
		\item
		\subsection*{Problem 3}
		\subsection*{Rider Income Decision Risk}
		A food delivery company reviews weekly gig income for riders in their 20s.
		It observed 18 riders and found an average weekly income of 515 USD.
		Management plans to cut bonus support because 515 USD looks high.

		What could go wrong if the decision is based only on this single observed average?

		\subsection*{C3}
		The sample statistic is 515 USD from a sample of 18 riders.
		The population parameter is the true mean weekly income of all riders in their 20s in that city.

		The point estimate is 515 USD.
		No interval estimate is given, so uncertainty is not quantified.

		With only 18 observations, the sample may not represent the whole rider population well.
		If the sample happened to include riders from unusually busy weeks, relying only on the point estimate could cause a bonus cut that harms many riders.
		Decision conclusion: do not reduce support until uncertainty around the estimate is examined.

		\newpage
		\item
		\subsection*{Problem 4}
		\subsection*{Product Sales Estimate Risk}
		A small online store tracks daily sales from a new product line.
		From 12 observed days, the average daily revenue was 390 USD.
		The founder wants to sign a one-year advertising contract based on this value.

		Explain the risk of making this decision using only the point estimate.

		\subsection*{C3}
		The sample statistic is the 390 USD average from the 12 sampled days.
		The population parameter is the true mean daily revenue for all relevant days in the full business cycle.

		The point estimate is 390 USD.
		There is no interval estimate provided.

		A sample of 12 days can be strongly affected by short-term promotions, holidays, or one viral post.
		If the founder treats the point estimate as certain, they may commit to ad costs that normal months cannot support.
		Decision conclusion: postpone the long contract until interval-based evidence gives a safer revenue range.

		\newpage
		\item
		\subsection*{Problem 5}
		\subsection*{Rent Mean Interval Interpretation}
		A rent advisory service estimates average monthly rent paid by young professionals.
		From a sample of 60 apartments, the sample average rent is 1,620 USD.
		A 95\% confidence interval for the population mean rent is [1,540, 1,700] USD.

		Compare the point and interval information.

		\subsection*{C3}
		The sample statistic is the sample mean of 1,620 USD from 60 observed apartments.
		The population parameter is the true mean monthly rent for all similar apartments in the city segment.

		The point estimate is 1,620 USD.
		The interval estimate is the confidence interval [1,540, 1,700] USD.
		\[
		\mu \in [1540,1700]
		\]

		The interval estimate adds uncertainty information that the point estimate alone cannot show.
		It shows that reasonable population values extend well below and above 1,620 USD.
		Decision conclusion: renters should plan with the interval, not with a single rent expectation.

		\newpage
		\item
		\subsection*{Problem 6}
		\subsection*{Designer Income Interval Advantage}
		A freelancing platform estimates monthly income of beginner designers in their 20s.
		A sample of 35 freelancers gives an average of 980 USD.
		A reported confidence interval for the population mean is [880, 1,080] USD.

		Explain how the interval addresses limitations of the single average.

		\subsection*{C3}
		The sample statistic is 980 USD computed from 35 observed freelancers.
		The population parameter is the true mean monthly income of all beginner designers in the target market.

		The point estimate is 980 USD.
		The interval estimate is [880, 1,080] USD.
		\[
		\mu \in [880,1080]
		\]

		The point estimate suggests one center, but the interval estimate shows plausible variation around that center.
		This reduces overconfidence when projecting income for loan approval or budgeting.
		Decision conclusion: use the interval to set conservative income expectations for new freelancers.

		\newpage
		\item
		\subsection*{Problem 7}
		\subsection*{Rent Commitment Decision Uncertainty}
		A recent graduate is choosing between two rent commitments.
		They saw data from a sample of 30 tenants with an average monthly rent of 1,450 USD and interval [1,280, 1,620] USD for the population mean.

		Scenario A decides using only 1,450 USD.
		Scenario B decides using the interval estimate.
		Which approach is safer and why?

		\subsection*{C3}
		The sample statistic is the sample average 1,450 USD from 30 tenants.
		The population parameter is the true mean rent for all comparable tenants in that city area.

		The point estimate is 1,450 USD.
		The interval estimate is [1,280, 1,620] USD.

		Scenario A may underestimate risk if actual rents are closer to the upper end.
		Scenario B is safer because it recognizes uncertainty and prepares for a wider set of realistic outcomes.
		In real life, relying only on the point estimate can cause lease commitments that exceed monthly cash flow.
		Decision conclusion: Scenario B is safer for rent decisions because it protects against optimistic underbudgeting.

		\newpage
		\item
		\subsection*{Problem 8}
		\subsection*{Salary Negotiation Under Uncertainty}
		A young professional is negotiating salary after seeing market data.
		From a sample of 22 job offers, the average offer is 54,000 USD with interval [49,000, 59,000] USD for the population mean offer.

		Scenario A argues from the point estimate only.
		Scenario B uses the interval estimate in negotiation planning.
		Which approach is safer and what consequences follow from ignoring uncertainty?

		\subsection*{C3}
		The sample statistic is the 54,000 USD average from 22 observed offers.
		The population parameter is the true mean salary offer for all relevant entry-level jobs in that market.

		The point estimate is 54,000 USD.
		The interval estimate is [49,000, 59,000] USD.

		Using only the point estimate can create a rigid target and poor negotiation strategy.
		Using the interval estimate supports flexible expectations and better preparation for realistic offer variation.
		If uncertainty is ignored, the candidate may reject acceptable offers or accept too quickly below a reasonable range.
		Decision conclusion: Scenario B is safer because interval-based planning aligns better with real salary uncertainty.

		\newpage
		\item
		\subsection*{Problem 9}
		\subsection*{Investment Return Sample Size Effects}
		An investment app tracks monthly returns for beginners.
		Sample A observed 20 investors and produced a mean return of 85 USD with interval [55, 115] USD.
		Sample B observed 100 investors and produced a mean return of 88 USD with interval [78, 98] USD.

		Explain how sample size changes confidence in population conclusions.
		Why can relying only on point estimates hide important information?

		\subsection*{C3}
		The sample statistic values are 85 USD from 20 investors and 88 USD from 100 investors.
		The population parameter in both cases is the true mean monthly return for all beginners on the app.

		The point estimate is 85 USD for Sample A and 88 USD for Sample B.
		The interval estimate is [55, 115] USD for Sample A and [78, 98] USD for Sample B.

		The larger sample has a much narrower interval estimate, which gives more stable inference about the population parameter.
		If we look only at point estimates, the two studies seem similarly informative, but interval width shows very different uncertainty levels.
		Decision conclusion: prefer Sample B for planning because its larger sample gives a more dependable range.

		\newpage
		\item
		\subsection*{Problem 10}
		\subsection*{Sales Study Precision Comparison}
		A side-business coach compares two studies of monthly sales for creators in their 20s.
		Both studies report the same sample mean, 1,100 USD.
		Study X uses 16 observations and gives interval [900, 1,300] USD.
		Study Y uses 144 observations and gives interval [1,040, 1,160] USD.

		Analyze why the intervals differ even with the same point estimate.
		Why could relying only on the point estimate hide critical decision information?

		\subsection*{C3}
		The sample statistic is 1,100 USD in both studies, computed from different sample sizes.
		The population parameter is the true mean monthly sales for the full creator population of interest.

		The point estimate is 1,100 USD for both Study X and Study Y.
		The interval estimate is [900, 1,300] USD for Study X and [1,040, 1,160] USD for Study Y.

		The smaller sample size in Study X creates a wider interval estimate because uncertainty is greater.
		The larger sample size in Study Y produces a narrower interval and more precise population inference.
		If we only compare the shared point estimate, we miss the fact that decision risk is much higher in Study X.
		Decision conclusion: for financing and inventory commitments, Study Y provides the safer evidence base.

	\end{ExamProblems}
\end{document}
