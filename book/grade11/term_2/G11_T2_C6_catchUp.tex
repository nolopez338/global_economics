\documentclass[12pt]{article}
\usepackage[a4paper,left=1.2cm,right=1.2cm,top=1.5cm,bottom=1.5cm]{geometry}
\usepackage{graphicx}
\usepackage{xcolor}
\usepackage{array}
\usepackage{tabularx}
\usepackage{multicol}
\usepackage{amsmath}
\usepackage[T1]{fontenc}
\usepackage[utf8]{inputenc}
\setlength{\parindent}{0pt}
\setlength{\tabcolsep}{6pt}
\renewcommand{\arraystretch}{1.15}
\newcolumntype{Y}{>{\raggedright\arraybackslash}m{\dimexpr0.55\textwidth-2\tabcolsep-2\arrayrulewidth\relax}}
\newcolumntype{Z}{>{\raggedright\arraybackslash}m{\dimexpr0.45\textwidth-2\tabcolsep-2\arrayrulewidth\relax}}
\newcolumntype{C}[1]{>{\centering\arraybackslash}m{#1}}
\newcommand{\SubsectionBox}[1]{%
	\noindent\colorbox{gray!30}{\parbox{\linewidth}{\textbf{#1}}}\par\vspace{0.35cm}%
}
\newcommand{\CellCenter}[1]{\parbox{\linewidth}{\centering #1}}

\begin{document}
	
	\noindent
	\begin{tabularx}{\textwidth}{|C{2.8cm}|C{\dimexpr\textwidth-6cm-4\tabcolsep-4\arrayrulewidth\relax}|C{2.8cm}|}
		\hline
		\centering\vspace{3mm}\includegraphics[width=2.5cm]{../../preamble/logo.png}&
		\CellCenter{\vspace{-5mm}\textbf{GLOBAL ECONOMICS}\par\textbf{GRADE: 11TH}\par\textbf{CATCH-UP}\par\textbf{CRITERION C6}\par\textbf{TEACHER'S NAME: Nicolás López Cuéllar}}&
		\CellCenter{\textbf{SECOND TERM}\par\textbf{2025--2026}}\\
		\hline
	\end{tabularx}
	
	\vspace{0.5cm}
	\noindent
	\begin{tabular}{|Y|Z|}
		\hline
		{\footnotesize\textbf{Learning objective:} Plan sample size to achieve a target margin of error in confidence interval planning.}&
		{\footnotesize\textbf{Assessment criteria:}\par C6: Stablishes the sample size so the error will not exceed a specified amount.}\\
		\hline
	\end{tabular}
	
	\begin{multicols}{2}
		\SubsectionBox{Criteria assessment}\vspace{-0.25cm}
		Each assessment criterion is evaluated across the problems in this catch-up exam. A criterion is considered passed when it is correctly activated in 11 of the 12 problems of this activity.
		
		\vspace{0.25cm}
		\SubsectionBox{Problem 1}
		\subsection*{Weekly Spending Sample Size}
		A digital-payment study wants to estimate mean weekly spending by first-job workers. Long-run records show $\sigma = 96$ USD. The team wants 95\% confidence with target margin of error $E = 12$ USD. Determine the minimum required sample size.
		
		\vspace{0.5cm}
		\SubsectionBox{Problem 2}
		\subsection*{Utilities Spending Sample Size}
		A youth housing survey estimates mean monthly utilities spending. The long-run standard deviation is known as $\sigma = 54$ USD. The planner requires 90\% confidence and margin of error $E = 6$ USD. Determine the minimum required sample size.
		
		\vspace{0.5cm}
		\SubsectionBox{Problem 3}
		\subsection*{Designer Earnings Sample Planning}
		A freelance-income team wants the mean weekly platform earnings for young designers. Long-run variability is $\sigma = 125$ USD. The team wants a 95\% confidence estimate with margin of error $E = 20$ USD. Extra context says there are 9{,}800 active designers and last week\'s median earning was 410 USD. Determine the minimum required sample size.
		
		\vspace{0.5cm}
		\SubsectionBox{Problem 4}
		\subsection*{Ride-Share Interval Width Planning}
		A transportation budget survey estimates mean monthly ride-share spending by university graduates. The known long-run spread is $\sigma = 72$ USD. The required confidence interval width is $W = 18$ USD at 95\% confidence. Determine the minimum required sample size.
		
		\vspace{0.5cm}
		\SubsectionBox{Problem 5}
		\subsection*{Confidence Level Margin Tradeoff}
		An e-commerce savings study tracks mean monthly discount use by young adults. Long-run standard deviation is $\sigma = 84$ USD. The team wants the same target margin $E = 10$ USD under two confidence levels: 90\% and 95\%. Compute both required sample sizes and compare.
		
		\vspace{0.5cm}
		\SubsectionBox{Problem 6}
		\subsection*{Delivery Cost Precision Upgrade}
		A food-delivery cost survey already has $n_{\text{current}} = 140$ users and currently reports margin of error $E_{\text{current}} = 11$ USD at 95\% confidence. The organization now wants margin of error $E_{\text{new}} = 8$ USD at the same confidence level. Determine the required total sample size and additional observations needed.
		
		\vspace{0.5cm}
		\SubsectionBox{Problem 7}
		\subsection*{Subscription Margin Reduction Plan}
		A streaming-subscription survey has $n_{\text{current}} = 121$ and current margin of error $E_{\text{current}} = 9$ USD at 95\% confidence. Management requests a 20\% reduction in margin of error. Find the new target margin, required total sample size, and additional observations.
		
		\vspace{0.5cm}
		\SubsectionBox{Problem 8}
		\subsection*{Mobile Banking Precision Stages}
		A mobile-banking usage survey currently has $n_{\text{current}} = 160$ and margin of error $E_{\text{current}} = 14$ USD at 90\% confidence. The policy team wants two staged improvements: first a 10\% reduction, then a 30\% reduction from the current margin. Determine both required total sample sizes and additional observations.
		
		\vspace{0.5cm}
		\SubsectionBox{Problem 9}
		\subsection*{Nonessential Spending Sample Upgrade}
		A personal-finance app study has current sample size $n_{\text{current}} = 100$ users. Long-run records indicate $\sigma = 75$ USD for monthly nonessential spending. The confidence level stays at 95\%. The team wants to reduce the current margin of error by 20\%. First recover the current margin of error, then find the new required total sample size and additional observations.
		
		\vspace{0.5cm}
		\SubsectionBox{Problem 10}
		\subsection*{Side-Income Precision Scenarios}
		A job-market survey currently has $n_{\text{current}} = 144$ respondents, and long-run salary-spread data give $\sigma = 120$ USD for weekly side-income. Confidence level is 90\%. The analysts are considering two reductions from the current margin of error: 25\% and 40\%. Recover the current margin first, then compute both required sample sizes and compare.
		
		\vspace{0.5cm}
		\SubsectionBox{Problem 11}
		\subsection*{Dual Study Precision Comparison}
		Two independent youth-economy studies use 95\% confidence and want the same 30\% margin-of-error reduction.
		Study A: $n_{A,\text{current}} = 81$, $\sigma_A = 90$ USD.
		Study B: $n_{B,\text{current}} = 196$, $\sigma_B = 140$ USD.
		For each study, recover the current margin first, then find the required total sample size and additional observations. Compare which study needs fewer additional observations.
		
		\vspace{0.5cm}
		\SubsectionBox{Problem 12}
		\subsection*{Startup Cost Margin Targets}
		An entrepreneurship-finance study currently has $n_{\text{current}} = 225$ records, with known long-run standard deviation $\sigma = 180$ USD for monthly startup operating costs. Confidence level is 95\%. The team wants to evaluate three reductions from the current margin of error: 15\%, 30\%, and 50\%. Recover the current margin first, then compute all new target margins, required total sample sizes, and additional observations.
		
	\end{multicols}
	
\end{document}