\documentclass[12pt]{article}

% Page size and tighter margins
\usepackage[a4paper,left=1.2cm,right=1.2cm,top=1.5cm,bottom=1.5cm]{geometry}

% Core packages
\usepackage{graphicx}
\usepackage{xcolor}
\usepackage{array}
\usepackage{tabularx}
\usepackage{multicol}
\usepackage{amsmath}
\usepackage[T1]{fontenc}
\usepackage[utf8]{inputenc}
\usepackage{enumitem}


\setlength{\parindent}{0pt}
\setlength{\tabcolsep}{6pt}
\renewcommand{\arraystretch}{1.15}
\setlength{\emergencystretch}{3em}

% Column types
\newcolumntype{Y}{>{\raggedright\arraybackslash}m{\dimexpr0.60\textwidth-2\tabcolsep-2\arrayrulewidth\relax}}
\newcolumntype{Z}{>{\raggedright\arraybackslash}m{\dimexpr0.40\textwidth-2\tabcolsep-2\arrayrulewidth\relax}}
\newcolumntype{C}[1]{>{\centering\arraybackslash}m{#1}}

% Gray subsection header box
\newcommand{\SubsectionBox}[1]{%
	\noindent\colorbox{gray!30}{%
		\parbox{\linewidth}{\textbf{#1}}%
	}\par\vspace{0.35cm}%
}

% Centered multi-line cell helper
\newcommand{\CellCenter}[1]{%
	\parbox{\linewidth}{\centering #1}%
}

\begin{document}
	
	% =========================
	% HEADER BOX (3 COLUMNS)
	% =========================
	\noindent
	\begin{tabularx}{\textwidth}{|C{2.8cm}|C{\dimexpr\textwidth-6cm-4\tabcolsep-4\arrayrulewidth\relax}|C{2.8cm}|}
		\hline
		\centering
		\vspace{3mm}
		\includegraphics[width=2.5cm]{../../preamble/logo.png}
		&
		\CellCenter{%
			\vspace{-5mm}
			\textbf{GLOBAL ECONOMICS}\par
			\textbf{GRADE: 11TH}\par
			\textbf{TERM 2 -- PROYECT}\par
			\textbf{AI WEBPAGE FOR CONFIDENCE INTERVALS}\par
			\textbf{TEACHER: Nicolás López Cuéllar}
		}
		&
		\CellCenter{%
			\textbf{SECOND TERM}\par
			\textbf{2025--2026}%
		}
		\\
		\hline
	\end{tabularx}
	
	\vspace{0.5cm}
	
	% =========================
	% OBJECTIVE + CRITERIA
	% =========================
	\noindent
	\begin{tabular}{|Y|Z|}
		\hline
		{\small
			\textbf{Learning objective:} Present and defend an AI-generated web page that estimates a confidence interval for the mean with unknown variance (t-student), and supports contextual conclusions in finance and economy.
		}
		&
		{\footnotesize
			\textbf{Assessment criterion:}\par
			\textbf{Global Economics}\par
			C10: Concludes about a statistical parameter in situations in finance and economy.\par
		}
		\\
		\hline
	\end{tabular}
	
	\vspace{0.4cm}
	
	\begin{multicols}{2}
		
		\vspace{0.25cm}
		\SubsectionBox{1. Proyect brief}\vspace{-0.25cm}
		
		\textbf{1.1 Purpose of the proyect}\par
		This is an additional proyect proposed for the class as an interesting real-world application.
		It is not linked to extra criteria; it is a practical enrichment application that supports the Global Economics statistical conclusion work under C10.
		Students will design and implement an AI-generated web page that receives real economic or financial data and estimates confidence intervals for the population mean when variance is unknown.
		The web page must make the procedure simple, transparent, and repeatable.
		
		\textbf{1.2 Input specification}\par
		The web page must receive:
		\begin{itemize}
			\item A dataset of numerical observations (raw data), typed or pasted as a list.
			\item Optional metadata: context label, units, date range, and scenario label.
			\item A confidence level selector (for example 90\%, 95\%, or 99\%).
		\end{itemize}
		
		\textbf{1.3 Processing requirements}\par
		The web page must:
		\begin{itemize}
			\item Compute sample size $n$, sample mean $\bar{x}$, sample standard deviation $s$, and standard error $SE=\frac{s}{\sqrt{n}}$.
			\item For unknown variance, compute and report t-based outputs: t statistic/critical value used for the interval ($t^*$ or equivalent), margin of error $ME_t=t^*\cdot SE$, and confidence interval $CI_t=(\bar{x}-ME_t,\bar{x}+ME_t)$.
			\item Compute and report z-based approximation outputs: z statistic/critical value used for the interval ($z^*$ or equivalent), margin of error $ME_z=z^*\cdot SE$, and confidence interval $CI_z=(\bar{x}-ME_z,\bar{x}+ME_z)$.
			\item Provide a clear conclusion template in context language, prioritizing the t-based interval when variance is unknown.
		\end{itemize}
		
		\textbf{1.4 Output specification}\par
		For each dataset and selected confidence level, the web page output must include:
		\begin{itemize}
			\item \textbf{t-based outputs (variance unknown):} $n$, $\bar{x}$, $s$, $SE=\frac{s}{\sqrt{n}}$, $t^*$ (or equivalent t statistic), $ME_t=t^*\cdot SE$, and $CI_t=(\bar{x}-ME_t,\bar{x}+ME_t)$.
			\item \textbf{z-based approximation outputs:} $z^*$ (or equivalent z statistic), $ME_z=z^*\cdot SE$, and $CI_z=(\bar{x}-ME_z,\bar{x}+ME_z)$.
			\item A side-by-side comparison view with interval widths and/or margin-of-error comparison between t and z methods.
		\end{itemize}
		
		\textbf{1.5 AI workflow and verification requirement}\par
		\begin{enumerate}
			\item Use an AI tool to generate the HTML/CSS/JavaScript code for the web page.
			\item Run the generated web page and test its outputs with at least two different datasets and at least two confidence levels.
			\item Verify that both methods are shown for each test: t-based ($n$, $\bar{x}$, $s$, $SE$, $t^*$, $CI_t$) and z-based ($z^*$, $CI_z$).
			\item Compare CI widths or margins of error side-by-side and interpret the results in a finance/economy context.
		\end{enumerate}
		
		\newpage
		\SubsectionBox{2. Solved Example 1}
		
		\textbf{Financial context: weekly profit from an online sales project}\par
		A university student runs a small online resale project and tracks the daily profit earned (in COP) to evaluate whether the activity is financially sustainable. To understand how stable the earnings are and to decide whether to scale the project, the student records the daily profit over one week. Raw data are entered as individual observations in the web page:
		
		\begin{center}
			\begin{tabular}{l}
				\hline
				Profit data (COP) \\
				\hline
				49.2, 51.4, 47.8, 52.1, 50.5, 48.9,\\
				53.0, 49.7, 50.9, 51.8, 48.6, 52.4 \\
				\hline
			\end{tabular}
		\end{center}
		The confidence selector is set to \textbf{95\% confidence} and scenario \textbf{Week 1 performance}.
		
		\textbf{Web page output}\par
		\begin{center}
			\begin{tabular}{l c}
				\hline
				Metric & Value \\
				\hline
				$n$ & 12 \\
				$\bar{x}$ (COP) & 50.52 \\
				$s$ (COP) & 1.68 \\
				$SE$ & 0.484 \\
				$t^*$ (95\%, $df=11$) & 2.201 \\
				$ME_t$ & 1.06 \\
				$CI_t$ lower bound & 49.46 \\
				$CI_t$ upper bound & 51.59 \\
				$z^*$ (95\% normal approx.) & 1.960 \\
				$ME_z$ & 0.95 \\
				$CI_z$ lower bound & 49.57 \\
				$CI_z$ upper bound & 51.47 \\
				$ME_t-ME_z$ & 0.11 \\
				\hline
			\end{tabular}
		\end{center}
		
		\textbf{Conclusion}\par
		With 95\% confidence, using the preferred t-based method for unknown variance, the mean daily profit for the selected week is between 49.46 and 51.59 COP. The z-approximation interval is 49.57 to 51.47 COP, which is close but slightly narrower. In this student business context, the t-based interval is used for the final conclusion because the population variance is unknown and the sample size is limited.
		
		
		
		\vspace{2cm}
		\SubsectionBox{3. Solved Example 2}
		
		\textbf{Financial context: daily revenue from a snack stand}\par
		A group of people manage a small snack stand during events and records the daily revenue earned (thousand COP) to evaluate whether the project is financially viable and to plan future inventory purchases. To estimate the average daily revenue and understand how consistent their earnings are, they enter the observed values into the web page as individual observations:
		
		\begin{center}
			\begin{tabular}{l}
				\hline
				Revenue data (thousand COP) \\
				\hline
				172, 181, 175, 188, 193, 179,\\
				185, 190, 177, 184, 196, 187,\\
				180, 192, 174, 189, 183, 195 \\
				\hline
			\end{tabular}
		\end{center}
		The confidence selector is set to \textbf{90\% confidence} and the activity scenario \textbf{School event cycle}.
		
		\textbf{Web page output}\par
		\begin{center}
			\begin{tabular}{l c}
				\hline
				Metric & Value \\
				\hline
				$n$ & 18 \\
				$\bar{x}$ (thousand COP) & 184.4 \\
				$s$ (thousand COP) & 7.4 \\
				$SE$ & 1.7 \\
				$t^*$ (90\%, $df=17$) & 1.740 \\
				$ME_t$ & 3.0 \\
				$CI_t$ lower bound & 181.4 \\
				$CI_t$ upper bound & 187.4 \\
				$z^*$ (90\% normal approx.) & 1.645 \\
				$ME_z$ & 2.8 \\
				$CI_z$ lower bound & 181.6 \\
				$CI_z$ upper bound & 187.2 \\
				$ME_t-ME_z$ & 0.2 \\
				\hline
			\end{tabular}
		\end{center}
		
		\textbf{Conclusion}\par
		With 90\% confidence, using the preferred t-based method for unknown variance, the mean daily revenue for this project is between 181.4 and 187.4 thousand COP. The z-approximation interval is 181.6 to 187.2 thousand COP, very close to the t-based result in this case. In this student business context, the t-based interval supports the final conclusion because the population variance is unknown and the estimate is based on sample data.
		
		
		
	\end{multicols}

\end{document}
