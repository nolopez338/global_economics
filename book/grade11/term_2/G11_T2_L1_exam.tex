\documentclass[12pt]{article}

% Page size and tighter margins
\usepackage[a4paper,left=1.2cm,right=1.2cm,top=1.5cm,bottom=1.5cm]{geometry}

% Core packages
\usepackage{graphicx}
\usepackage{xcolor}
\usepackage{array}
\usepackage{tabularx}
\usepackage{multicol}
\usepackage[T1]{fontenc}
\usepackage[utf8]{inputenc}

\setlength{\parindent}{0pt}
\setlength{\tabcolsep}{6pt}
\renewcommand{\arraystretch}{1.15}

% Column types
\newcolumntype{Y}{>{\raggedright\arraybackslash}m{\dimexpr0.30\textwidth-2\tabcolsep-2\arrayrulewidth\relax}}
\newcolumntype{Z}{>{\raggedright\arraybackslash}m{\dimexpr0.70\textwidth-2\tabcolsep-2\arrayrulewidth\relax}}
\newcolumntype{C}[1]{>{\centering\arraybackslash}m{#1}}

% Gray subsection header box
\newcommand{\SubsectionBox}[1]{%
	\noindent\colorbox{gray!30}{%
		\parbox{\linewidth}{\textbf{#1}}%
	}\par\vspace{0.35cm}%
}

% Centered multi-line cell helper
\newcommand{\CellCenter}[1]{%
	\parbox{\linewidth}{\centering #1}%
}

\begin{document}

	% =========================
	% HEADER BOX (3 COLUMNS)
	% =========================
	\noindent
	\begin{tabularx}{\textwidth}{|C{2.8cm}|C{\dimexpr\textwidth-6cm-4\tabcolsep-4\arrayrulewidth\relax}|C{2.8cm}|}
		\hline
		\centering
		\vspace{3mm}
		\includegraphics[width=2.5cm]{../../preamble/logo.png}
		&
		\CellCenter{%
			\vspace{-5mm}
			\textbf{GLOBAL ECONOMICS}\par
			\textbf{GRADE: 11TH}\par
			\textbf{LEARNING EVIDENCE 1}\par
			\textbf{CONFIDENCE INTERVAL PLANNING}\par
			\textbf{TEACHER'S NAME: Nicolás López Cuéllar}
		}
		&
		\CellCenter{%
			\textbf{SECOND TERM}\par
			\textbf{2025--2026}%
		}
		\\
		\hline
	\end{tabularx}

	\vspace{0.5cm}

	% =========================
	% OBJECTIVE + CRITERIA
	% =========================
	\noindent
	\begin{tabular}{|Y|Z|}
		\hline
		{\small
			\textbf{Learning objective:} Compute and interpret confidence intervals for population means using known population standard deviation.
		}
		&
		{\footnotesize
			\textbf{Assessment criteria:}\par
			C1: Compute the sample mean and sample standard deviation.\par
			C2: Distinguish between sample statistics and population parameters.\par
			C3: Explain why interval estimation is preferred over a single point estimate.\par
			C5: Construct a X\% confidence interval using known population standard deviation.\par
			C4: Interpret the meaning of a X\% confidence interval in context.\par
		}
		\\
		\hline
	\end{tabular}

	\vspace{0.4cm}

	% =========================
	% STUDENT LINE
	% =========================
	\noindent
	\textbf{Student's name:} \rule{7cm}{0.4pt}\hfill
	\textbf{Group:} \rule{2cm}{0.4pt}\hfill
	\textbf{Date:} \rule{3cm}{0.4pt}

	% =========================
	% EXAM BODY
	% =========================
	\begin{multicols}{2}
		\SubsectionBox{Criteria assessment}\vspace{-0.25cm}
		Each assessment criterion is evaluated across the three problems. A criterion is considered passed when it is correctly activated in at least two of the three problems.
		
		\vspace{0.25cm}
		\SubsectionBox{1. Transit subsidy estimates}\vspace{-0.25cm}
		A city budgeting office wants to estimate the average weekly transit subsidy per rider for two pilot zones.
		The two samples are different random samples drawn from the same underlying population of weekly subsidy amounts.
		Zone A recorded the following sample of weekly subsidy amounts (USD):

		\begin{itemize}
			\item Zone A sample (\(n_A = 6\)): 18, 22, 19, 21, 20, 17.
			\item Zone B sample (\(n_B = 3\)): 24, 26, 23.
		\end{itemize}

		An annual audit provides the population standard deviation for this population: \(\sigma = 3.5\).
		For classroom purposes, suppose the true population mean weekly subsidy for this population is \(\mu = 21\).
		Construct and interpret a 95\% confidence interval for the population mean weekly subsidy in each zone.

		\vspace{0.25cm}
		\SubsectionBox{2. Warehouse order rates}\vspace{-0.25cm}
		A regional warehouse studies the number of orders processed per hour to plan staffing.
		A random sample of 54 hours is summarized in grouped form below. The operations team also reports a known population standard deviation of \(\sigma = 6.2\) orders per hour.
		For academic purposes only, assume the true population mean order rate is \(\mu = 22\) orders per hour.
		The grouped data pairs list orders per hour first and the frequency (number of hours) second: \(14 \rightarrow 8\), \(18 \rightarrow 16\), \(22 \rightarrow 20\), and \(26 \rightarrow 10\).
		Construct and interpret a 90\% confidence interval and a 98\% confidence interval for the population mean orders per hour.

		\vspace{0.25cm}
		\SubsectionBox{3. Container loading times}\vspace{-0.25cm}
		A logistics firm monitors the time (in minutes) required to load containers at a port.
		A sample of 55 loading times is grouped into class intervals below. The firm has a historical estimate of the population standard deviation, \(\sigma = 5.8\) minutes.
		The grouped intervals and frequencies are 25--29 minutes (18), 30--34 minutes (20), and 35--39 minutes (17).
		Construct and interpret a 95\% confidence interval for the population mean loading time.

	\end{multicols}

\end{document}
