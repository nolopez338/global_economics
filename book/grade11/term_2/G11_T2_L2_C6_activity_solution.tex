\makeatletter
\def\input@path{{./}{../}{../../}{preamble/}{../preamble/}{../../preamble/}}
\makeatother
% ----------------------------------------------------------
% GENERAL 

% File
\documentclass[11pt]{book}

% Margins
\usepackage[margin=1in]{geometry}

\linespread{1.2}            % Line spacing
\usepackage[utf8]{inputenc}
\usepackage[T1]{fontenc}
\usepackage{lmodern}
\usepackage{microtype}
\setlength{\parindent}{0pt}
\setlength{\parskip}{6pt}
\usepackage{booktabs}

% ----------------------------------------------------------
% TABLES
\usepackage{multicol}
\usepackage{longtable} 
\usepackage{array}
\usepackage{booktabs}
\usepackage{tabularx}
\usepackage{multirow}

% ----------------------------------------------------------
% MATHEMATICS
\usepackage{amsmath}
\usepackage{amssymb}
\usepackage{amsfonts}
\usepackage{mathtools}

% ----------------------------------------------------------
% HIDDEN CONTENT
\usepackage{ifthen}
% Define a boolean switch
\newboolean{explicaciones}
% Set the boolean switch to true or false
% Change to true to show the content

% Explanations
\newcommand{\explicacion}[2]{
	\ifthenelse{\boolean{explicaciones}}{#1}{#2}
}
\newcommand{\mostrarExplicaciones}[1]{\setboolean{explicaciones}{#1}}

% ----------------------------------------------------------
% NUMBERING

\usepackage{fancyhdr}
\pagestyle{empty} % Ensures the entire document has no page numbers

\usepackage{tocloft}
\renewcommand{\cftdot}{} % Remove dots for sections, if any
\renewcommand{\cftsecleader}{\cftdotfill{\cftdotsep}} % Remove dots for sections, if any
\cftpagenumbersoff{section} % Removes page numbers from sections
\cftpagenumbersoff{subsection} % Removes page numbers from subsections

% ----------------------------------------------------------
% IMAGES 

% General settings
\usepackage{graphicx}       % Insert images
\usepackage{float}          % Position images
% \usepackage{subfigure}      % Subfigures
\graphicspath{{imgs}}       % Image location
\usepackage{subcaption}     % Subfigures II
\usepackage{verbatim}

% Figures
\usepackage{tikz}
\usetikzlibrary{arrows.meta,positioning,trees}

% Colors
\usepackage{xcolor}     
\definecolor{popUp}{HTML}{666666}
\definecolor{popUpIn}{HTML}{CED9E0}
\definecolor{backgroundC}{HTML}{D0E8F2}
\definecolor{backgroundCC}{HTML}{FFFFFF}
\definecolor{borders}{HTML}{8c120d}
\definecolor{padding}{HTML}{77D0D7}
\definecolor{links}{HTML}{CC6F5F}

% ----------------------------------------------------------
% FRAMES

% General settings
\usepackage{tcolorbox}
\usepackage{adjustbox}          % Adjusted frame  
\setlength{\fboxrule}{3pt}  % Line width
\setlength{\fboxsep}{3pt}   % Box padding

% General frames
\usepackage{mdframed}   

\mdfdefinestyle{estiloGeneral}{    % General style
	linecolor=black,
	linewidth=1.5pt,
	roundcorner=10pt,
	backgroundcolor=backgroundC,
	innerbottommargin=0pt
}
\mdfdefinestyle{code}{          % Code style
	linecolor=black,
	linewidth=1.5pt,
	roundcorner=10pt,
	backgroundcolor=darkgray!10,
	innerbottommargin=0pt
}

% Image frame
\newtcbox{\fboxC}{
	colback=backgroundC,
	colframe=popUp,
	arc=10pt,
	boxrule=3pt,
	boxsep=0pt, % Change the padding here
	nobeforeafter
}

% ----------------------------------------------------------
% PAGE SETTINGS

% Background 
\newcommand{\background}[0]{\begin{tikzpicture}[remember picture,overlay]
		\fill[backgroundC] (-2,2) rectangle (25cm, -550);
\end{tikzpicture}}

\newcommand{\backgroundC}[0]{\begin{tikzpicture}[remember picture,overlay]
		\fill[backgroundCC] (-2,2) rectangle (25cm, -550);
\end{tikzpicture}}

% Page width 
\newcommand{\anchoPag}[0]{20cm}

% ----------------------------------------------------------
% FONT

% General
\usepackage{tgbonum}        % Font
\usepackage{listings}       % Code typesetting
\usepackage[spanish]{babel} % Load Spanish
\selectlanguage{spanish}    % Select Spanish
\usepackage{enumitem}
\usepackage{bookmark}

\setlist[itemize]{leftmargin=1.2em, itemsep=0.35em, topsep=0.35em}

% --- Table helpers ---
\newcolumntype{L}[1]{>{\raggedright\arraybackslash}p{#1}}
\newcolumntype{Y}{>{\raggedright\arraybackslash}X}
\newcolumntype{C}{>{\centering\arraybackslash}X}
\renewcommand{\arraystretch}{1.1}

% Python style
\lstdefinestyle{python}{
	language=Python,
	basicstyle=\ttfamily\small,
	commentstyle=\color{green!50!black},
	keywordstyle=\color{blue},
	numberstyle=\tiny\color{gray},
	numbers=left,
	morekeywords={>, <},
	breakatwhitespace=false,
	showstringspaces=false,
	showtabs=false,
	showspaces=false
}

% ----------------------------------------------------------
% HYPERLINKS

% General
\usepackage{hyperref}       
\hypersetup{
	colorlinks=true,
	linkcolor=links,
	filecolor=magenta,      
	urlcolor=brown,
}

% Custom commands 

% Large link
\newcommand{\bigLink}[2]{\begin{center} \fboxC{\LARGE{\href{#1}{#2}}}\end{center}}

% Small link
\newcommand{\smallLink}[2]{\begin{center}\fboxC{\href{#1}{#2}}\end{center}}

% Bold link
\newcommand{\bfLink}[2]{\href{#1}{\textbf{#2}}}


% Small URL
\newcommand{\smallUrl}[1]{\begin{center}\fboxC{\url{#1}}\end{center}}


% ----------------------------------------------------------
% CUSTOM COMMANDS FOR FIGURES

\newcommand{\espacioImagenes}[0]{-1.2cm}

% Without frame
\newcommand{\fig}[3][\espacioImagenes]{
	\hspace*{#1}
	\centering
	\includegraphics[width=#2\textwidth]{#3}
}

% With frame
\newcommand{\ffig}[2]{\begin{figure}[h]
		\hspace*{\espacioImagenes}
		\centering
		\fbox{\includegraphics[width=#1\textwidth]{#2}}
\end{figure}}

% Hyperlink with frame
\newcommand{\hfig}[3]{\begin{figure}[h]
		\hspace*{-1.4cm}
		\centering
		\color{popUp}
		\fboxC{\href{#1}{\includegraphics[width=#2\textwidth]{#3}}}
	\end{figure}
}

% Hyperlink without frame
\newcommand{\hffig}[3]{\begin{figure}[h]
		\hspace*{-1.1cm}
		\centering
		\color{popUp}
		\href{#1}{\includegraphics[width=#2\textwidth]{#3}}
	\end{figure}
}

% ----------------------------------------------------------

% Start and Contents
\newcommand{\cuadro}[1]{
	\begin{mdframed}[style=estiloGeneral]
		#1 
	\end{mdframed}
}

% Explanation video image
\newcommand{\linkExplicacion}[1]{
	\hffig{#1}{0.5}{principal/videoExplicacion}
	\vspace{-0.5cm}
}

\newcommand{\subSecLink}[2]{
	\subsubsection*{\href{#1}{\textbf{#2}}}
}

% Spacing
\newcommand{\esp}[0]{\vspace{4mm}}

% Back to start
\newcommand{\secInicio}[0]{\begin{center}\hyperref[sec:inicio]{ 
			\includegraphics[width=0.1\textwidth]{principal/up}
	}\end{center}
}


\geometry{margin=0.85in}
\AtBeginDocument{\small}

\newcommand{\ExamNameField}{\noindent\textbf{Name:}\ \rule{0.7\linewidth}{0.4pt}\par\medskip}

\newcommand{\ExamTitleBlock}[3]{%
	\begin{center}
		\Large\textbf{#1}\\
		\textbf{#2}%
		\if\relax\detokenize{#3}\relax\else\\\textbf{#3}\fi
	\end{center}
	\vspace{0.5em}
}

\newcommand{\ExamSection}[1]{\par\medskip\textbf{#1}\par\smallskip}

\newenvironment{ExamCriteria}{%
	\begin{itemize}[leftmargin=1.6em, itemsep=0.3em, topsep=0.2em]
}{%
	\end{itemize}
}

\newenvironment{ExamProblems}{%
	\begin{enumerate}[label=\textbf{P\arabic*}, leftmargin=0pt, labelsep=0.6em, itemindent=2.2em, itemsep=0.8em]
}{%
	\end{enumerate}
}

\begin{document}
	\ExamTitleBlock{11th grade}{Term 2 Learning Evidence 2: C6 Sample Size Activity}{}
	
	\ExamSection{Evaluated criteria}
	\begin{ExamCriteria}
		\item C6: Determine the required sample size to achieve a target margin of error.
	\end{ExamCriteria}
	
	\ExamSection{Problems}
	\begin{ExamProblems}
		\item
		\subsection*{Sample Size Determination for Estimating a Mean Commute Time}
		A transportation planner wants to estimate the mean commute time (in minutes) for a metropolitan rail line. Historical records from previous years report a long-run average commute time of $\mu = 42$ minutes. Past studies show the population standard deviation is $\sigma = 8$ minutes. The planner wants a 99\% confidence estimate with a maximum error of $E = 2$ minutes. Determine the minimum sample size needed.
		
		\subsection*{C6}
		This solution sets the sample size so the margin of error does not exceed the target value by selecting $z^*$, substituting into the formula, and rounding up to the smallest integer that meets the requirement.
		
		Step 1: Identify the target error and confidence level. The maximum error is $E = 2$ minutes and the confidence level is 99\%, so $z^* = 2.576$.
		
		Step 2: Apply the sample size formula.
		\[
		n = \left(\frac{z^*\,\sigma}{E}\right)^2
		= \left(\frac{2.576 \cdot 8}{2}\right)^2
		= \left(10.304\right)^2
		= 106.18
		\]
		
		Step 3: Round up and verify Criterion 6. Round up to $n = 107$. With $n=107$, the margin of error is at most 2 minutes, satisfying Criterion 6.
		
		\item
		\subsection*{Sample Size Determination for Estimating a Mean Weekly Wage}
		A labor analyst wants to estimate the mean weekly wage (in dollars) for a specific industry. Long-term administrative records for this industry report an average weekly wage of $\mu = 820$ dollars. The population standard deviation is known to be $\sigma = 15$ dollars. The analyst needs a 95\% confidence estimate with a maximum error of $E = 3$ dollars. Determine the minimum sample size required.
		
		\subsection*{C6}
		This solution determines the sample size that guarantees the margin of error does not exceed the specified value by using the $z^*$ value and rounding up.
		
		Step 1: Identify the target error and confidence level. The maximum error is $E = 3$ dollars and the confidence level is 95\%, so $z^* = 1.96$.
		
		Step 2: Apply the sample size formula.
		\[
		n = \left(\frac{z^*\,\sigma}{E}\right)^2
		= \left(\frac{1.96 \cdot 15}{3}\right)^2
		= \left(9.80\right)^2
		= 96.04
		\]
		
		Step 3: Round up and verify Criterion 6. Round up to $n = 97$. With $n=97$, the margin of error is at most 3 dollars, satisfying Criterion 6.
		
		\item
		\subsection*{Sample Size Determination for Estimating Mean Crop Yield}
		An agriculture board wants to estimate the mean yield (in tons per hectare) of a crop. Historical production reports for the region indicate a long-run average yield of $\mu = 6.8$ tons per hectare. The population standard deviation is $\sigma = 4.5$ tons per hectare. The board wants a 90\% confidence estimate with a maximum error of $E = 1$ ton per hectare. Determine the minimum sample size required.
		
		\subsection*{C6}
		This solution identifies the target error and confidence level, applies the sample size formula, and rounds up to the smallest acceptable sample size.
		
		Step 1: Identify the target error and confidence level. The maximum error is $E = 1$ ton per hectare and the confidence level is 90\%, so $z^* = 1.645$.
		
		Step 2: Apply the sample size formula.
		\[
		n = \left(\frac{z^*\,\sigma}{E}\right)^2
		= \left(\frac{1.645 \cdot 4.5}{1}\right)^2
		= \left(7.4025\right)^2
		= 54.80
		\]
		
		Step 3: Round up and verify Criterion 6. Round up to $n = 55$. With $n=55$, the margin of error is at most 1 ton per hectare, satisfying Criterion 6.
		
		\item
		\subsection*{Additional Sample Size Needed to Reduce Margin of Error for a Mean}
		A utilities department wants to estimate the mean monthly electricity bill (in dollars). The population standard deviation is $\sigma = 5.2$ dollars, and $n_{\text{current}} = 49$ bills have been recorded. From these 49 bills, a sample mean of $\bar{x} = 120.5$ dollars and a sample standard deviation of $s = 5.0$ dollars were obtained. How many additional observations are required to reduce the margin of error to $E = 1.0$ dollar at 95\% confidence?
		
		\subsection*{C6}
		This subsection determines the required sample size so the margin of error does not exceed the target amount.
		
		Step 1: Use the target error and confidence level. The target margin of error is $E = 1.0$ dollar and for 95\% confidence, $z^* = 1.96$.
		
		Step 2: Compute the required sample size.
		\[
		n_{\text{required}} = \left(\frac{z^*\,\sigma}{E}\right)^2
		= \left(\frac{1.96 \cdot 5.2}{1.0}\right)^2
		= \left(10.192\right)^2
		\approx 103.88
		\]
		
		Step 3: Determine additional observations. Round up to $n_{\text{required}} = 104$. The additional observations needed are $104 - 49 = 55$.
		
		Conclusion. The department needs 55 more observations so the error will not exceed 1.0 dollar.
		
		\item
		\subsection*{Additional Sample Size Needed to Reduce Margin of Error for a Mean Exam Score}
		A school district is estimating the mean final exam score for a course. The population standard deviation is $\sigma = 6$ points, and $n_{\text{current}} = 25$ scores are available. From these scores, a sample mean of $\bar{x} = 78.2$ points and a sample standard deviation of $s = 5.8$ points were obtained. How many additional observations are required to reduce the margin of error to $E = 1.2$ points at 90\% confidence?
		
		\subsection*{C6}
		This subsection determines the required sample size so the margin of error does not exceed the target amount.
		
		Step 1: Use the target error and confidence level. The target margin of error is $E = 1.2$ points and for 90\% confidence, $z^* = 1.645$.
		
		Step 2: Compute the required sample size.
		\[
		n_{\text{required}} = \left(\frac{z^*\,\sigma}{E}\right)^2
		= \left(\frac{1.645 \cdot 6}{1.2}\right)^2
		= \left(8.225\right)^2
		\approx 67.65
		\]
		
		Step 3: Determine additional observations. Round up to $n_{\text{required}} = 68$. The additional observations needed are $68 - 25 = 43$.
		
		Conclusion. The district needs 43 more observations so the error will not exceed 1.2 points.
		
		\item
		\subsection*{Additional Sample Size Needed to Reduce Margin of Error for Mean Water Use}
		A water authority is estimating the mean daily household water use (in liters). The population standard deviation is $\sigma = 18$ liters, and $n_{\text{current}} = 64$ households have been surveyed. From these households, a sample mean of $\bar{x} = 210$ liters and a sample standard deviation of $s = 17.2$ liters were obtained. How many additional observations are required to reduce the margin of error to $E = 4$ liters at 99\% confidence?
		
		\subsection*{C6}
		This subsection determines the required sample size so the margin of error does not exceed the target amount.
		
		Step 1: Use the target error and confidence level. The target margin of error is $E = 4$ liters and for 99\% confidence, $z^* = 2.576$.
		
		Step 2: Compute the required sample size.
		\[
		n_{\text{required}} = \left(\frac{z^*\,\sigma}{E}\right)^2
		= \left(\frac{2.576 \cdot 18}{4}\right)^2
		= \left(11.592\right)^2
		\approx 134.36
		\]
		
		Step 3: Determine additional observations. Round up to $n_{\text{required}} = 135$. The additional observations needed are $135 - 64 = 71$.
		
		Conclusion. The authority needs 71 more observations so the error will not exceed 4 liters.
		
	\end{ExamProblems}

	\section*{Summary}
	\begin{enumerate}
		\item \textbf{Sample Size Determination for Estimating a Mean Commute Time}
		\[
		n = \left(\frac{z^*\sigma}{E}\right)^2
		= \left(\frac{2.576\cdot 8}{2}\right)^2
		= 106.18,
		\qquad \boxed{n=107}
		\]

		\item \textbf{Sample Size Determination for Estimating a Mean Weekly Wage}
		\[
		n = \left(\frac{z^*\sigma}{E}\right)^2
		= \left(\frac{1.96\cdot 15}{3}\right)^2
		= 96.04,
		\qquad \boxed{n=97}
		\]

		\item \textbf{Sample Size Determination for Estimating Mean Crop Yield}
		\[
		n = \left(\frac{z^*\sigma}{E}\right)^2
		= \left(\frac{1.645\cdot 4.5}{1}\right)^2
		= 54.80,
		\qquad \boxed{n=55}
		\]

		\item \textbf{Additional Sample Size Needed to Reduce Margin of Error for a Mean}
		\[
		n_{\text{required}} = \left(\frac{z^*\sigma}{E}\right)^2
		= \left(\frac{1.96\cdot 5.2}{1.0}\right)^2
		\approx 103.88,
		\qquad \boxed{n_{\text{required}}=104},
		\qquad \boxed{104-49=55\text{ additional}}
		\]

		\item \textbf{Additional Sample Size Needed to Reduce Margin of Error for a Mean Exam Score}
		\[
		n_{\text{required}} = \left(\frac{z^*\sigma}{E}\right)^2
		= \left(\frac{1.645\cdot 6}{1.2}\right)^2
		\approx 67.65,
		\qquad \boxed{n_{\text{required}}=68},
		\qquad \boxed{68-25=43\text{ additional}}
		\]

		\item \textbf{Additional Sample Size Needed to Reduce Margin of Error for Mean Water Use}
		\[
		n_{\text{required}} = \left(\frac{z^*\sigma}{E}\right)^2
		= \left(\frac{2.576\cdot 18}{4}\right)^2
		\approx 134.36,
		\qquad \boxed{n_{\text{required}}=135},
		\qquad \boxed{135-64=71\text{ additional}}
		\]
	\end{enumerate}
\end{document}
