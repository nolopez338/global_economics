\makeatletter
\def\input@path{{./}{../}{../../}{preamble/}{../preamble/}{../../preamble/}}
\makeatother
% ----------------------------------------------------------
% GENERAL 

% File
\documentclass[11pt]{book}

% Margins
\usepackage[margin=1in]{geometry}

\linespread{1.2}            % Line spacing
\usepackage[utf8]{inputenc}
\usepackage[T1]{fontenc}
\usepackage{lmodern}
\usepackage{microtype}
\setlength{\parindent}{0pt}
\setlength{\parskip}{6pt}
\usepackage{booktabs}

% ----------------------------------------------------------
% TABLES
\usepackage{multicol}
\usepackage{longtable} 
\usepackage{array}
\usepackage{booktabs}
\usepackage{tabularx}
\usepackage{multirow}

% ----------------------------------------------------------
% MATHEMATICS
\usepackage{amsmath}
\usepackage{amssymb}
\usepackage{amsfonts}
\usepackage{mathtools}

% ----------------------------------------------------------
% HIDDEN CONTENT
\usepackage{ifthen}
% Define a boolean switch
\newboolean{explicaciones}
% Set the boolean switch to true or false
% Change to true to show the content

% Explanations
\newcommand{\explicacion}[2]{
	\ifthenelse{\boolean{explicaciones}}{#1}{#2}
}
\newcommand{\mostrarExplicaciones}[1]{\setboolean{explicaciones}{#1}}

% ----------------------------------------------------------
% NUMBERING

\usepackage{fancyhdr}
\pagestyle{empty} % Ensures the entire document has no page numbers

\usepackage{tocloft}
\renewcommand{\cftdot}{} % Remove dots for sections, if any
\renewcommand{\cftsecleader}{\cftdotfill{\cftdotsep}} % Remove dots for sections, if any
\cftpagenumbersoff{section} % Removes page numbers from sections
\cftpagenumbersoff{subsection} % Removes page numbers from subsections

% ----------------------------------------------------------
% IMAGES 

% General settings
\usepackage{graphicx}       % Insert images
\usepackage{float}          % Position images
% \usepackage{subfigure}      % Subfigures
\graphicspath{{imgs}}       % Image location
\usepackage{subcaption}     % Subfigures II
\usepackage{verbatim}

% Figures
\usepackage{tikz}
\usetikzlibrary{arrows.meta,positioning,trees}

% Colors
\usepackage{xcolor}     
\definecolor{popUp}{HTML}{666666}
\definecolor{popUpIn}{HTML}{CED9E0}
\definecolor{backgroundC}{HTML}{D0E8F2}
\definecolor{backgroundCC}{HTML}{FFFFFF}
\definecolor{borders}{HTML}{8c120d}
\definecolor{padding}{HTML}{77D0D7}
\definecolor{links}{HTML}{CC6F5F}

% ----------------------------------------------------------
% FRAMES

% General settings
\usepackage{tcolorbox}
\usepackage{adjustbox}          % Adjusted frame  
\setlength{\fboxrule}{3pt}  % Line width
\setlength{\fboxsep}{3pt}   % Box padding

% General frames
\usepackage{mdframed}   

\mdfdefinestyle{estiloGeneral}{    % General style
	linecolor=black,
	linewidth=1.5pt,
	roundcorner=10pt,
	backgroundcolor=backgroundC,
	innerbottommargin=0pt
}
\mdfdefinestyle{code}{          % Code style
	linecolor=black,
	linewidth=1.5pt,
	roundcorner=10pt,
	backgroundcolor=darkgray!10,
	innerbottommargin=0pt
}

% Image frame
\newtcbox{\fboxC}{
	colback=backgroundC,
	colframe=popUp,
	arc=10pt,
	boxrule=3pt,
	boxsep=0pt, % Change the padding here
	nobeforeafter
}

% ----------------------------------------------------------
% PAGE SETTINGS

% Background 
\newcommand{\background}[0]{\begin{tikzpicture}[remember picture,overlay]
		\fill[backgroundC] (-2,2) rectangle (25cm, -550);
\end{tikzpicture}}

\newcommand{\backgroundC}[0]{\begin{tikzpicture}[remember picture,overlay]
		\fill[backgroundCC] (-2,2) rectangle (25cm, -550);
\end{tikzpicture}}

% Page width 
\newcommand{\anchoPag}[0]{20cm}

% ----------------------------------------------------------
% FONT

% General
\usepackage{tgbonum}        % Font
\usepackage{listings}       % Code typesetting
\usepackage[spanish]{babel} % Load Spanish
\selectlanguage{spanish}    % Select Spanish
\usepackage{enumitem}
\usepackage{bookmark}

\setlist[itemize]{leftmargin=1.2em, itemsep=0.35em, topsep=0.35em}

% --- Table helpers ---
\newcolumntype{L}[1]{>{\raggedright\arraybackslash}p{#1}}
\newcolumntype{Y}{>{\raggedright\arraybackslash}X}
\newcolumntype{C}{>{\centering\arraybackslash}X}
\renewcommand{\arraystretch}{1.1}

% Python style
\lstdefinestyle{python}{
	language=Python,
	basicstyle=\ttfamily\small,
	commentstyle=\color{green!50!black},
	keywordstyle=\color{blue},
	numberstyle=\tiny\color{gray},
	numbers=left,
	morekeywords={>, <},
	breakatwhitespace=false,
	showstringspaces=false,
	showtabs=false,
	showspaces=false
}

% ----------------------------------------------------------
% HYPERLINKS

% General
\usepackage{hyperref}       
\hypersetup{
	colorlinks=true,
	linkcolor=links,
	filecolor=magenta,      
	urlcolor=brown,
}

% Custom commands 

% Large link
\newcommand{\bigLink}[2]{\begin{center} \fboxC{\LARGE{\href{#1}{#2}}}\end{center}}

% Small link
\newcommand{\smallLink}[2]{\begin{center}\fboxC{\href{#1}{#2}}\end{center}}

% Bold link
\newcommand{\bfLink}[2]{\href{#1}{\textbf{#2}}}


% Small URL
\newcommand{\smallUrl}[1]{\begin{center}\fboxC{\url{#1}}\end{center}}


% ----------------------------------------------------------
% CUSTOM COMMANDS FOR FIGURES

\newcommand{\espacioImagenes}[0]{-1.2cm}

% Without frame
\newcommand{\fig}[3][\espacioImagenes]{
	\hspace*{#1}
	\centering
	\includegraphics[width=#2\textwidth]{#3}
}

% With frame
\newcommand{\ffig}[2]{\begin{figure}[h]
		\hspace*{\espacioImagenes}
		\centering
		\fbox{\includegraphics[width=#1\textwidth]{#2}}
\end{figure}}

% Hyperlink with frame
\newcommand{\hfig}[3]{\begin{figure}[h]
		\hspace*{-1.4cm}
		\centering
		\color{popUp}
		\fboxC{\href{#1}{\includegraphics[width=#2\textwidth]{#3}}}
	\end{figure}
}

% Hyperlink without frame
\newcommand{\hffig}[3]{\begin{figure}[h]
		\hspace*{-1.1cm}
		\centering
		\color{popUp}
		\href{#1}{\includegraphics[width=#2\textwidth]{#3}}
	\end{figure}
}

% ----------------------------------------------------------

% Start and Contents
\newcommand{\cuadro}[1]{
	\begin{mdframed}[style=estiloGeneral]
		#1 
	\end{mdframed}
}

% Explanation video image
\newcommand{\linkExplicacion}[1]{
	\hffig{#1}{0.5}{principal/videoExplicacion}
	\vspace{-0.5cm}
}

\newcommand{\subSecLink}[2]{
	\subsubsection*{\href{#1}{\textbf{#2}}}
}

% Spacing
\newcommand{\esp}[0]{\vspace{4mm}}

% Back to start
\newcommand{\secInicio}[0]{\begin{center}\hyperref[sec:inicio]{ 
			\includegraphics[width=0.1\textwidth]{principal/up}
	}\end{center}
}


\geometry{margin=0.85in}
\AtBeginDocument{\small}

\newcommand{\ExamNameField}{\noindent\textbf{Name:}\ \rule{0.7\linewidth}{0.4pt}\par\medskip}

\newcommand{\ExamTitleBlock}[3]{%
	\begin{center}
		\Large\textbf{#1}\\
		\textbf{#2}%
		\if\relax\detokenize{#3}\relax\else\\\textbf{#3}\fi
	\end{center}
	\vspace{0.5em}
}

\newcommand{\ExamSection}[1]{\par\medskip\textbf{#1}\par\smallskip}

\newenvironment{ExamCriteria}{%
	\begin{itemize}[leftmargin=1.6em, itemsep=0.3em, topsep=0.2em]
}{%
	\end{itemize}
}

\newenvironment{ExamProblems}{%
	\begin{enumerate}[label=\textbf{P\arabic*}, leftmargin=0pt, labelsep=0.6em, itemindent=2.2em, itemsep=0.8em]
}{%
	\end{enumerate}
}

\begin{document}
		\ExamTitleBlock{11th grade}{Midterm Confidence \& Interval Estimation (Solutions)}{}
	
	\begin{ExamProblems}
		\item
		\subsection*{Comparing Processing Costs Using Grouped Sample Data}
		\textbf{Context:} A logistics startup studies processing costs (USD) using two grouped random samples drawn from the same shipment population.
		
		\textbf{Sample A (30 shipments):} The processing costs fall into these classes:
		\begin{itemize}
			\item 70 USD occurred in 9 shipments.
			\item 95 USD occurred in 10 shipments.
			\item 120 USD occurred in 7 shipments.
			\item 250 USD occurred in 4 shipments.
		\end{itemize}
		
		\textbf{Sample B (25 shipments):} The processing costs fall into these classes:
		\begin{itemize}
			\item 70 USD occurred in 5 shipments.
			\item 95 USD occurred in 8 shipments.
			\item 120 USD occurred in 9 shipments.
			\item 250 USD occurred in 3 shipments.
		\end{itemize}
		
		\textbf{Known:} long-run audits show the population standard deviation is $\sigma = 60$ USD.
		
		\textbf{Goal:} construct 95\% confidence intervals for the mean processing cost in each sample and compare them for planning in a high-volatility environment.
		
		\subsection*{C1}
		\textbf{Sample A:}
		\begin{center}
			\begin{tabular}{c c}
				\toprule
				Cost (USD) & Frequency \\
				\midrule
				70 & 9 \\
				95 & 10 \\
				120 & 7 \\
				250 & 4 \\
				\bottomrule
			\end{tabular}
		\end{center}
		\[
		\sum f x = 9(70) + 10(95) + 7(120) + 4(250)
		= 630 + 950 + 840 + 1000
		= 3420
		\]
		\[
		\bar{x}_A = \frac{3420}{30} = 114.00
		\]
		\[
		\begin{aligned}
			\sum f(x-\bar{x}_A)^2 &= 9(70-114)^2 + 10(95-114)^2 + 7(120-114)^2 + 4(250-114)^2 \\
			&= 9(1936) + 10(361) + 7(36) + 4(18496) \\
			&= 17424 + 3610 + 252 + 73984 \\
			&= 95270
		\end{aligned}
		\]
		\[
		s_A^2 = \frac{95270}{30-1} = 3285.17 \text{ USD}^2, \qquad s_A = \sqrt{3285.17} \approx 57.32 \text{ USD}
		\]
		
		\textbf{Sample B:}
		\begin{center}
			\begin{tabular}{c c}
				\toprule
				Cost (USD) & Frequency \\
				\midrule
				70 & 5 \\
				95 & 8 \\
				120 & 9 \\
				250 & 3 \\
				\bottomrule
			\end{tabular}
		\end{center}
		\[
		\sum f x = 5(70) + 8(95) + 9(120) + 3(250)
		= 350 + 760 + 1080 + 750
		= 2940
		\]
		\[
		\bar{x}_B = \frac{2940}{25} = 117.60
		\]
		\[
		\begin{aligned}
			\sum f(x-\bar{x}_B)^2 &= 5(70-117.60)^2 + 8(95-117.60)^2 + 9(120-117.60)^2 + 3(250-117.60)^2 \\
			&= 5(2265.76) + 8(510.76) + 9(5.76) + 3(17529.76) \\
			&= 11328.80 + 4086.08 + 51.84 + 52589.28 \\
			&= 68056.00
		\end{aligned}
		\]
		\[
		s_B^2 = \frac{68056.00}{25-1} = 2835.67 \text{ USD}^2, \qquad s_B = \sqrt{2835.67} \approx 53.25 \text{ USD}
		\]
		
		\subsection*{C2}
		The population mean $\mu$ is the cost parameter of interest. Sample A estimates it with $\bar{x}_A = 114.00$ and Sample B with $\bar{x}_B = 117.60$.
		
		The known population spread $\sigma = 60$ replaces the sample spreads $s_A$ and $s_B$ when building confidence intervals.
		
		\subsection*{C3}
		The point estimates $\bar{x}_A$ and $\bar{x}_B$ summarize each sample, while confidence intervals provide ranges of plausible values for $\mu$ given cost volatility.
		
		Because processing costs are volatile, the interval estimates are essential for comparing likely mean costs across the two samples.
		
		\subsection*{C5}
		Use the known population standard deviation to build 95\% confidence intervals for each sample.
		\[
		SE_A = \frac{\sigma}{\sqrt{30}} \approx 10.95, \quad E_A = 1.96(10.95) \approx 21.47
		\]
		\[
		\text{Sample A: } \mu \in 114.00 \pm 21.47 = [92.53,\ 135.47] \text{ USD}
		\]
		\[
		SE_B = \frac{\sigma}{\sqrt{25}} = 12.00, \quad E_B = 1.96(12.00) = 23.52
		\]
		\[
		\text{Sample B: } \mu \in 117.60 \pm 23.52 = [94.08,\ 141.12] \text{ USD}
		\]
		
		\subsection*{C4}
		The intervals overlap (approximately 94.08 to 135.47), so the mean processing costs could be similar across the two samples. However, Sample B’s interval is shifted higher and wider, suggesting slightly higher typical costs with more uncertainty. Planners can treat Sample B as a higher-cost scenario while noting the overlap means the difference is not definitive.

		
		\item
		\subsection*{Estimating Average Delivery Time from a Small Sample}
		\textbf{Context:} A food court tracks lunch delivery times (in minutes) for a random sample of \textbf{12 orders} during the weekday rush.
		
		\textbf{Recorded times:}
		\[
		28,\ 30,\ 31,\ 29,\ 27,\ 32,\ 33,\ 30,\ 28,\ 31,\ 29,\ 34.
		\]
		
		\textbf{Known:} operational benchmarks give the population standard deviation as $\sigma = 2.1$ minutes (population variance $\sigma^2 = 4.41$ minutes$^2$).
		
		\textbf{Goal:} construct 90\%, 95\%, and 98\% confidence intervals for the population mean delivery time using the known population variance.
		
		\subsection*{C1}
		Let the delivery times be $x_1,\dots,x_{12}$. Compute the sample mean and sample standard deviation.
		\[
		\bar{x} = \frac{28 + 30 + 31 + 29 + 27 + 32 + 33 + 30 + 28 + 31 + 29 + 34}{12}
		= \frac{362}{12} = 30.17 \text{ minutes}
		\]
		\[
		\sum (x_i-\bar{x})^2 = 49.67
		\]
		\[
		s^2 = \frac{\sum (x_i-\bar{x})^2}{n-1} = \frac{49.67}{11} = 4.52 \text{ minutes}^2
		\]
		\[
		s = \sqrt{4.52} \approx 2.12 \text{ minutes}
		\]
		
		\subsection*{C2}
		The parameter of interest is the population mean delivery time $\mu$. The statistic $\bar{x} = 30.17$ estimates it from the sample.
		
		The population spread is known as $\sigma = 2.1$ minutes, while the sample standard deviation $s \approx 2.12$ summarizes the sample.
		
		\subsection*{C3}
		The point estimate of the mean delivery time is $\bar{x} = 30.17$ minutes. Interval estimation adds a margin of error to capture plausible values for $\mu$.
		
		Because $s^2 = 4.52$ minutes$^2$, the interval provides a more realistic planning range for average delivery time than the single point estimate.
		
		\subsection*{C5}
		Use a $z$-distribution because the population variance is known.
		\[
		SE = \frac{\sigma}{\sqrt{n}} = \frac{2.1}{\sqrt{12}} \approx 0.61
		\]
		\[
		\text{90\% CI: } E = 1.645(0.61) \approx 1.00, \quad \mu \in 30.17 \pm 1.00 = [29.17,\ 31.17]
		\]
		\[
		\text{95\% CI: } E = 1.96(0.61) \approx 1.19, \quad \mu \in 30.17 \pm 1.19 = [28.98,\ 31.36]
		\]
		\[
		\text{98\% CI: } E = 2.326(0.61) \approx 1.41, \quad \mu \in 30.17 \pm 1.41 = [28.76,\ 31.58]
		\]
		
		\subsection*{C4}
		At 90\% confidence, managers can be 90\% confident the true mean delivery time is between 29.17 and 31.17 minutes; the 95\% and 98\% intervals widen to reflect higher confidence. Higher confidence means a wider interval, so the 98\% interval provides the most conservative planning buffer for staffing and dispatch.
		
		\item
		\subsection*{Estimating a Population Proportion from Survey Data}
		A policy team surveys \textbf{200 households} across a region to estimate the share that approve a proposed renewable energy subsidy. In the sample, \textbf{128 households} say they approve.
		
		Construct and interpret a \textbf{90\% confidence interval} for the population proportion of households that approve the subsidy.
		
		\subsection*{C7}
		To construct the confidence interval for a proportion, identify $x$ and $n$ and compute $\hat{p}$, verify the normal approximation conditions $n\hat{p} \ge 10$ and $n(1-\hat{p}) \ge 10$, choose the appropriate $z^*$ value, compute the standard error and margin of error, construct the interval, and interpret it in context.
		
		\[
		\hat{p} = \frac{x}{n} = \frac{128}{200} = 0.64
		\]
		
		\[
		n\hat{p} = 200(0.64) = 128 \ge 10, \qquad n(1-\hat{p}) = 200(0.36) = 72 \ge 10
		\]
		
		\[
		SE = \sqrt{\frac{\hat{p}(1-\hat{p})}{n}} = \sqrt{\frac{0.64(0.36)}{200}} \approx 0.0339
		\]
		
		For a 90\% confidence level, $z^* = 1.645$.
		\[
		E = z^* \cdot SE = 1.645(0.0339) \approx 0.0558
		\]
		
		\[
		\hat{p} \pm E = 0.64 \pm 0.0558 \Rightarrow [0.584,\ 0.696]
		\]
		
		We are 90\% confident that between approximately 58.4\% and 69.6\% of all households in the region approve the renewable energy subsidy.
		
		\item
		\subsection*{Estimating the Mean Price of a Staple Food Basket}
		An inflation monitoring team wants to estimate the mean price of a staple food basket (in dollars) across markets.
		
		They know the population standard deviation is $\sigma = 0.60$ dollars and the current sample mean is $\bar{x} = 2.85$ dollars.
		
		The team has collected $n_{\text{current}} = 36$ market prices and wants a 90\% confidence interval.
		
		(a) Construct the confidence interval for the mean price.
		
		(b) How many additional observations are required to reduce the margin of error to $E = 0.12$ dollars?
		
		\subsection*{C5}
		This solution calculates the confidence interval for the mean with known variance using the current sample information.
		
		Step 1: Compute the standard error.
		\[
		SE = \frac{\sigma}{\sqrt{n_{\text{current}}}}
		= \frac{0.60}{\sqrt{36}}
		= 0.10
		\]
		Step 2: Construct the confidence interval. For 90\% confidence, $z^* = 1.645$.
		\[
		\bar{x} \pm z^* SE = 2.85 \pm 1.645(0.10)
		= 2.85 \pm 0.16
		= (2.69,\ 3.01)
		\]
		Conclusion. The 90\% confidence interval is $(2.69,\ 3.01)$ dollars.
		
	\subsection*{C6}
		This subsection establishes the required sample size so the margin of error does not exceed the target amount.
		
		Step 1: Use the target error and confidence level. The target margin of error is $E = 0.12$ dollars and for 90\% confidence, $z^* = 1.645$.
		
		Step 2: Compute the required sample size.
		\[
		n_{\text{required}} = \left(\frac{z^*\,\sigma}{E}\right)^2
		= \left(\frac{1.645 \cdot 0.60}{0.12}\right)^2
		= \left(8.225\right)^2
		\approx 67.65
		\]
		Step 3: Determine additional observations. Round up to $n_{\text{required}} = 68$. The additional observations needed are $68 - 36 = 32$.
		
		Conclusion. The team needs 32 more observations so the error will not exceed 0.12 dollars.
		
	\end{ExamProblems}
	

\end{document}