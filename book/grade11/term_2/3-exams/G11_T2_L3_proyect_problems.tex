\documentclass[12pt]{article}

% Page size and tighter margins
\usepackage[a4paper,left=1.2cm,right=1.2cm,top=1.5cm,bottom=1.5cm]{geometry}

% Core packages
\usepackage{graphicx}
\usepackage{xcolor}
\usepackage{array}
\usepackage{tabularx}
\usepackage{multicol}
\usepackage{amsmath}
\usepackage[T1]{fontenc}
\usepackage[utf8]{inputenc}

\setlength{\parindent}{0pt}
\setlength{\tabcolsep}{6pt}
\renewcommand{\arraystretch}{1.15}

% Column types
\newcolumntype{Y}{>{\raggedright\arraybackslash}m{\dimexpr0.50\textwidth-2\tabcolsep-2\arrayrulewidth\relax}}
\newcolumntype{Z}{>{\raggedright\arraybackslash}m{\dimexpr0.50\textwidth-2\tabcolsep-2\arrayrulewidth\relax}}
\newcolumntype{C}[1]{>{\centering\arraybackslash}m{#1}}

% Gray subsection header box
\newcommand{\SubsectionBox}[1]{%
\noindent\colorbox{gray!30}{%
\parbox{\dimexpr\linewidth-2\fboxsep\relax}{\textbf{#1}}%
}\par\vspace{0.35cm}%
}

% Centered multi-line cell helper
\newcommand{\CellCenter}[1]{%
\parbox{\linewidth}{\centering #1}%
}

\begin{document}
	
	% =========================
	% HEADER BOX (3 COLUMNS)
	% =========================
	\noindent
	\begin{tabularx}{\textwidth}{|C{2.8cm}|>{\centering\arraybackslash}X|C{2.8cm}|}
		\hline
		\centering
		\vspace{3mm}
		\includegraphics[width=2.5cm]{../../preamble/logo.png}
		&
		\CellCenter{%
			% \vspace{5mm}
			\textbf{GLOBAL ECONOMICS}\par
			\textbf{GRADE: 11TH}\par
			\textbf{TERM 2 -- LESSON 3 PROJECT}\par
			\textbf{PROJECT PROBLEMS}\par
			\textbf{TEACHER'S NAME: Nicolás López Cuéllar}
		}
		&
		\CellCenter{%
			\textbf{SECOND TERM}\par
			\textbf{2025--2026}%
		}
		\\
		\hline
	\end{tabularx}
	
	\vspace{0.5cm}
	
	% =========================
	% OBJECTIVE + CRITERIA
	% =========================
	\noindent
	\begin{tabular}{|Y|Z|}
		\hline
		{\small
			\textbf{Learning objective:} Calculate and interpret confidence intervals for the population mean using the t-student distribution, and justify interval-based decisions when variance is unknown in finance and economy contexts.
		}
		&
		{\footnotesize
			\textbf{Assessment criteria:}\par
			C8: Calculates confidence intervals applying the t-student distribution in context situations.\par
			C9: Estimates the confidence interval for the mean when the variance is unknown in context situations.\par
		}
		\\
		\hline
	\end{tabular}
	
	% =========================
	% EXAM BODY
	% =========================
	\begin{multicols}{2}
		\SubsectionBox{Criteria assessment}\vspace{-0.25cm}
		Each assessment criterion is evaluated across the ten problems. A criterion is considered passed when it is correctly activated in at least 8 of the 10 problems.
		
		\vspace{0.25cm}
		\SubsectionBox{1. Driver payouts in ride-sharing}\vspace{-0.25cm}
		A ride-sharing platform studies daily gross driver payouts (hundreds of USD) in one urban zone.
		A random sample of $n=10$ days gives the raw data:
		\[
		42,\ 38,\ 45,\ 40,\ 41,\ 39,\ 44,\ 43,\ 37,\ 41
		\]
		1) Construct a 95\% confidence interval for the population mean using the t-distribution.
		2) Using the interval and the computed statistics, apply the structured C9 decision procedure to compare the t-based and z-based intervals and determine which interval is wider and by what percentage.
		
		\vspace{0.25cm}
		\SubsectionBox{2. Platform fees for investment app users}\vspace{-0.25cm}
		An investment app analyzes mean per-trade platform fees (USD) paid by users in their 20s.
		A random sample of $n=18$ transactions gives the raw data:
		\[
		\begin{aligned}
			1.8,\ 2.1,\ 1.9,\ 2.4,\ 2.0,\ 1.7,\ 2.2,\ 2.3,\ 1.8,\\
			2.5,\ 2.1,\ 1.9,\ 2.0,\ 2.2,\ 1.6,\ 2.4,\ 2.3,\ 1.9
		\end{aligned}
		\]
		1) Construct a 90\% confidence interval for the population mean using the t-distribution.
		2) Using the interval and the computed statistics, apply the structured C9 decision procedure to compare the t-based and z-based intervals and determine which interval is wider and by what percentage.
		
		\vspace{0.25cm}
		\SubsectionBox{3. Weekday-morning order values}\vspace{-0.25cm}
		A food-delivery app studies mean order value (USD) during weekday mornings for young professionals.
		A random sample of $n=30$ receipts gives the raw data:
		\[
		\begin{aligned}
			115,\ 122,\ 118,\ 130,\ 126,\ 119,\ 121,\ 124,\\
			128,\ 117,\ 123,\ 120,\ 125,\ 129,\ 116,\ 127,\\
			122,\ 118,\ 124,\ 121,\ 126,\ 119,\ 123,\ 128,\\
			120,\ 125,\ 117,\ 124,\ 122,\ 126
		\end{aligned}
		\]
		1) Construct a 99\% confidence interval for the population mean using the t-distribution.
		2) Using the interval and the computed statistics, apply the structured C9 decision procedure to compare the t-based and z-based intervals and determine which interval is wider and by what percentage.
		
		\vspace{0.25cm}
		\SubsectionBox{4. Same-city delivery time classes}\vspace{-0.25cm}
		A gig-delivery startup studies average delivery time (minutes) for same-city orders.
		For $n=60$ orders, data are grouped as follows:
		\[
		\begin{array}{c|c|c}
		\text{Class (minutes)} & \text{Midpoint }m_i & f_i\\\hline
		20\text{--}24 & 22 & 6\\
		25\text{--}29 & 27 & 9\\
		30\text{--}34 & 32 & 14\\
		35\text{--}39 & 37 & 13\\
		40\text{--}44 & 42 & 10\\
		45\text{--}49 & 47 & 8
		\end{array}
		\]
		1) Construct a 95\% confidence interval for the population mean using the t-distribution from the grouped-data estimates.
		2) Using the interval and the computed statistics, apply the structured C9 decision procedure to compare the t-based and z-based intervals and determine which interval is wider and by what percentage.
		
		\vspace{0.25cm}
		\SubsectionBox{5. Daily ad spending classes}\vspace{-0.25cm}
		A direct-to-consumer e-commerce startup estimates mean daily ad spending (thousand USD).
		For $n=120$ days, grouped data are:
		\[
		\begin{array}{c|c|c}
		\text{Class (thousand USD)} & m_i & f_i\\\hline
		50\text{--}55 & 52.5 & 12\\
		55\text{--}60 & 57.5 & 18\\
		60\text{--}65 & 62.5 & 28\\
		65\text{--}70 & 67.5 & 24\\
		70\text{--}75 & 72.5 & 22\\
		75\text{--}80 & 77.5 & 16
		\end{array}
		\]
		1) Construct a 90\% confidence interval for the population mean using the t-distribution from the grouped-data estimates.
		2) Using the interval and the computed statistics, apply the structured C9 decision procedure to compare the t-based and z-based intervals and determine which interval is wider and by what percentage.
		
		\vspace{0.25cm}
		\SubsectionBox{6. Monthly recurring revenue classes}\vspace{-0.25cm}
		A subscription software network studies mean monthly recurring revenue (thousand USD) for small creator-led apps.
		For $n=250$ firms, grouped data are:
		\[
		\begin{array}{c|c|c}
		\text{Class (thousand USD)} & m_i & f_i\\\hline
		80\text{--}90 & 85 & 18\\
		90\text{--}100 & 95 & 32\\
		100\text{--}110 & 105 & 46\\
		110\text{--}120 & 115 & 58\\
		120\text{--}130 & 125 & 44\\
		130\text{--}140 & 135 & 30\\
		140\text{--}150 & 145 & 22
		\end{array}
		\]
		1) Construct a 95\% confidence interval for the population mean using the t-distribution from the grouped-data estimates.
		2) Using the interval and the computed statistics, apply the structured C9 decision procedure to compare the t-based and z-based intervals and determine which interval is wider and by what percentage.
		
		\vspace{0.25cm}
		\SubsectionBox{7. Monthly account balance classes}\vspace{-0.25cm}
		A fintech savings app estimates the mean monthly account balance (hundreds of USD) of active users in their 20s.
		For $n=500$ users, grouped data are:
		\[
		\begin{array}{c|c|c}
		\text{Class (hundreds of USD)} & m_i & f_i\\\hline
		60\text{--}65 & 62.5 & 30\\
		65\text{--}70 & 67.5 & 52\\
		70\text{--}75 & 72.5 & 84\\
		75\text{--}80 & 77.5 & 108\\
		80\text{--}85 & 82.5 & 96\\
		85\text{--}90 & 87.5 & 68\\
		90\text{--}95 & 92.5 & 40\\
		95\text{--}100 & 97.5 & 22
		\end{array}
		\]
		1) Construct a 99\% confidence interval for the population mean using the t-distribution from the grouped-data estimates.
		2) Using the interval and the computed statistics, apply the structured C9 decision procedure to compare the t-based and z-based intervals and determine which interval is wider and by what percentage.
		
		\vspace{0.25cm}
		\SubsectionBox{8. Annual instructor earnings classes}\vspace{-0.25cm}
		A national online tutoring marketplace estimates mean annual instructor earnings (USD) for active tutors.
		For $n=1000$ policies, grouped data are:
		\[
		\begin{array}{c|c|c}
		\text{Class (USD)} & m_i & f_i\\\hline
		125\text{--}135 & 130 & 70\\
		135\text{--}145 & 140 & 130\\
		145\text{--}155 & 150 & 220\\
		155\text{--}165 & 160 & 250\\
		165\text{--}175 & 170 & 180\\
		175\text{--}185 & 180 & 100\\
		185\text{--}195 & 190 & 50
		\end{array}
		\]
		1) Construct a 95\% confidence interval for the population mean using the t-distribution from the grouped-data estimates.
		2) Using the interval and the computed statistics, apply the structured C9 decision procedure to compare the t-based and z-based intervals and determine which interval is wider and by what percentage.
		
		\vspace{0.25cm}
		\SubsectionBox{9. Monthly seller revenue classes}\vspace{-0.25cm}
		An e-commerce marketplace analyzes mean monthly seller revenue (thousand USD) per store.
		For $n=2500$ merchants, grouped data are:
		\[
		\begin{array}{c|c|c}
		\text{Class (thousand USD)} & m_i & f_i\\\hline
		200\text{--}220 & 210 & 180\\
		220\text{--}240 & 230 & 360\\
		240\text{--}260 & 250 & 620\\
		260\text{--}280 & 270 & 700\\
		280\text{--}300 & 290 & 380\\
		300\text{--}320 & 310 & 180\\
		320\text{--}340 & 330 & 80
		\end{array}
		\]
		1) Construct a 90\% confidence interval for the population mean using the t-distribution from the grouped-data estimates.
		2) Using the interval and the computed statistics, apply the structured C9 decision procedure to compare the t-based and z-based intervals and determine which interval is wider and by what percentage.
		
		\vspace{0.25cm}
		\SubsectionBox{10. Quarterly creator revenue classes}\vspace{-0.25cm}
		A streaming and creator-services network studies mean quarterly creator revenue (thousand USD) across channels.
		For $n=5000$ firms, grouped data are:
		\[
		\begin{array}{c|c|c}
		\text{Class (thousand USD)} & m_i & f_i\\\hline
		400\text{--}440 & 420 & 320\\
		440\text{--}480 & 460 & 680\\
		480\text{--}520 & 500 & 1300\\
		520\text{--}560 & 540 & 1400\\
		560\text{--}600 & 580 & 820\\
		600\text{--}640 & 620 & 360\\
		640\text{--}680 & 660 & 120
		\end{array}
		\]
		1) Construct a 99\% confidence interval for the population mean using the t-distribution from the grouped-data estimates.
		2) Using the interval and the computed statistics, apply the structured C9 decision procedure to compare the t-based and z-based intervals and determine which interval is wider and by what percentage.
		
	\end{multicols}
	
\end{document}
