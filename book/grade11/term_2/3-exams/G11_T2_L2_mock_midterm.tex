\documentclass[12pt]{article}

% Page size and tighter margins
\usepackage[a4paper,left=1.2cm,right=1.2cm,top=1.5cm,bottom=1.5cm]{geometry}

% Core packages
\usepackage{graphicx}
\usepackage{xcolor}
\usepackage{array}
\usepackage{tabularx}
\usepackage{multicol}
\usepackage[T1]{fontenc}
\usepackage[utf8]{inputenc}

\setlength{\parindent}{0pt}
\setlength{\tabcolsep}{6pt}
\renewcommand{\arraystretch}{1.15}

% Column types
\newcolumntype{Y}{>{\raggedright\arraybackslash}m{\dimexpr0.25\textwidth-2\tabcolsep-2\arrayrulewidth\relax}}
\newcolumntype{Z}{>{\raggedright\arraybackslash}m{\dimexpr0.75\textwidth-2\tabcolsep-2\arrayrulewidth\relax}}
\newcolumntype{C}[1]{>{\centering\arraybackslash}m{#1}}

% Gray subsection header box
\newcommand{\SubsectionBox}[1]{%
	\noindent\colorbox{gray!30}{%
		\parbox{\linewidth}{\textbf{#1}}%
	}\par\vspace{0.35cm}%
}

% Centered multi-line cell helper
\newcommand{\CellCenter}[1]{%
	\parbox{\linewidth}{\centering #1}%
}

\begin{document}

	% =========================
	% HEADER BOX (3 COLUMNS)
	% =========================
	\noindent
	\begin{tabularx}{\textwidth}{|C{2.8cm}|C{\dimexpr\textwidth-6cm-4\tabcolsep-4\arrayrulewidth\relax}|C{2.8cm}|}
		\hline
		\centering
		\vspace{3mm}
		\includegraphics[width=2.5cm]{../../preamble/logo.png}
		&
		\CellCenter{%
			\vspace{-5mm}
			\textbf{GLOBAL ECONOMICS}\par
			\textbf{GRADE: 11TH}\par
			\textbf{SECOND TERM MIDTERM MOCK}\par
			\textbf{CONFIDENCE INTERVAL PLANNING}\par
			\textbf{TEACHER'S NAME: Nicolás López Cuéllar}
		}
		&
		\CellCenter{%
			\textbf{SECOND TERM}\par
			\textbf{2025--2026}%
		}
		\\
		\hline
	\end{tabularx}

	\vspace{0.5cm}

	% =========================
	% OBJECTIVE + CRITERIA
	% =========================
	\noindent
	\begin{tabular}{|Y|Z|}
		\hline
		{\small
			\textbf{Learning objective:} Apply confidence interval methods for means and proportions in context, including interpretation and sample-size planning.
		}
		&
		{\footnotesize
			\textbf{Assessment criteria:}\par
			C1: Compute the sample mean and sample standard deviation.\par
			C2: Distinguish between sample statistics and population parameters.\par
			C3: Explain why interval estimation is preferred over a single point estimate.\par
			C4: Interpret the meaning of a X\% confidence interval in context.\par
			C5: Construct a X\% confidence interval using known population standard deviation.\par
			C6: Determine the required sample size to achieve a target margin of error.\par
			C7: Construct and interpret a confidence interval for a population proportion.\par
		}
		\\
		\hline
	\end{tabular}


	% =========================
	% EXAM BODY
	% =========================
	\begin{multicols}{2}
		\SubsectionBox{Criteria assessment}\vspace{-0.25cm}
		To pass a criterion, Criteria C1–C6 must each be correctly applied in at least two problems. Criterion C7 must be correctly applied in at least one problem.

		\vspace{0.25cm}
		\SubsectionBox{(C1,C5,C6) 1. Metropolitan Transit Fare Study}\vspace{-0.25cm}
		A metropolitan transit authority studies the average weekday fare collected per rider for two expanded pilot corridors.
		The two samples are distinct random samples drawn from the same population of fare amounts.
		The grouped fare data (USD) are below.

		\begin{multicols}{2}
			\textbf{Sample A (20 riders):}
			\begin{itemize}
				\item 38 USD occurred in 4 riders.
				\item 45 USD occurred in 5 riders.
				\item 52 USD occurred in 4 riders.
				\item 60 USD occurred in 3 riders.
				\item 70 USD occurred in 4 riders.
			\end{itemize}

			\columnbreak

			\textbf{Sample B (24 riders):}
			\begin{itemize}
				\item 38 USD occurred in 5 riders.
				\item 45 USD occurred in 4 riders.
				\item 52 USD occurred in 6 riders.
				\item 60 USD occurred in 5 riders.
				\item 70 USD occurred in 4 riders.
			\end{itemize}
		\end{multicols}

		Long-run audits show the population standard deviation is $\sigma = 11.5$ USD.
		For classroom purposes, suppose the true population mean fare is $\mu = 53$ USD.
		Construct and interpret 95\% confidence intervals, 90\% confidence intervals, and 99\% confidence intervals for the population mean in each corridor.
		At the end, determine how many additional observations are required in each sample group to reach a margin of error target of $E = 2.8$ USD for 95\% confidence, 90\% confidence, and 99\% confidence.

		\vspace{0.25cm}
		\SubsectionBox{(C1,C5,C6) 2. University Dining Checkout Times}\vspace{-0.25cm}
		A university dining analytics team compares average lunch checkout times (minutes) at two larger kiosks during peak hours.
		The two samples are distinct random samples with sizes $n_1 = 8$ and $n_2 = 10$ drawn from the same population of checkout times.

		\textbf{Sample 1:} 17, 19, 21, 22, 20, 18, 24, 23.

		\textbf{Sample 2:} 16, 18, 20, 21, 22, 24, 19, 23, 25, 20.

		Operational logs indicate the population standard deviation is $\sigma = 4.2$ minutes.
		Construct and interpret 95\% confidence intervals, 90\% confidence intervals, and 99\% confidence intervals for the population mean checkout time at each kiosk.
		At the end, determine how many additional observations are required in each kiosk sample to reach a margin of error target of $E = 1.6$ minutes for 95\%, 90\%, and 99\% confidence.

		\vspace{0.25cm}
		\SubsectionBox{(C7) 3. Utility Reimbursement Proportions}\vspace{-0.25cm}
		A national public finance office studies the proportion of quarterly utility reimbursements that are flagged for expedited approval in two expanded district networks.
		The two samples are distinct random samples with different sizes drawn from the same population of reimbursements.

		\textbf{Sample A (120 reimbursements):}
		In this sample, $x_A = 74$ reimbursements are flagged for expedited approval.

		\textbf{Sample B (95 reimbursements):}
		In this sample, $x_B = 52$ reimbursements are flagged for expedited approval.

		Construct 95\%, 90\%, and 99\% confidence intervals for the population proportion of expedited reimbursements using each sample.
		At the end, determine how many additional observations are required in each sample group to reach margin-of-error targets of $E = 0.04$ and $E = 0.03$ for 95\% confidence, using each sample estimated proportion.

		\vspace{0.25cm}
		\SubsectionBox{(C2,C3,C4) 4. Municipal Invoice Monitoring}\vspace{-0.25cm}
		A municipal purchasing authority now monitors invoice amounts across three departments (transport, maintenance, and health procurement) using a unified benchmark system. From long-term records covering all municipal purchases, invoices are known to have a typical average value of 445 USD and a typical spread of 62 USD around that average. These values summarize how invoice amounts behave across the full purchasing process over time.

		To assess current activity, two independent audits are conducted using recent invoices from all three departments combined. In Sample A, the invoices reviewed produce a 90\% confidence interval from 418 USD to 452 USD, and the variability within this audit is summarized by a variance of 3025 (USD)$^2$. In Sample B, the invoices reviewed produce a 90\% confidence interval from 432 USD to 476 USD, and the variability within this audit is summarized by a variance of 4096 (USD)$^2$.

		In both audits, the authority requires a clear judgment about whether current behavior is centered near the benchmark and whether observed spread appears operationally aligned with long-run variation.

		\vspace{0.25cm}
		\SubsectionBox{(C2,C3,C4) 5. Cooperative Loan Size Comparison}\vspace{-0.25cm}
		An agricultural finance board compares average loan sizes issued by two cooperative systems: rural cooperatives and urban cooperatives, across an expanded national program that includes seasonal credit lines and emergency refinancing windows. From long-term financial records covering all loans issued in each system, rural cooperatives are known to have a typical loan size of 318 USD with a usual spread of 46 USD, while urban cooperatives are known to have a typical loan size of 362 USD with a usual spread of 58 USD.

		To assess current lending activity, analysts examine recent loans from each system under the expanded program. The rural sample produces a 95\% confidence interval from 300 USD to 334 USD, and the urban sample produces a 95\% confidence interval from 345 USD to 383 USD.

		The board must assess not only whether current means differ, but also whether the degree of uncertainty still supports a stable policy distinction between rural and urban systems.

		\vspace{0.25cm}
		\SubsectionBox{(C2,C3,C4) 6. Mobile Savings Platform Adoption}\vspace{-0.25cm}
		A financial regulator studies the share of households using a mobile savings platform under a new nationwide rollout that now includes urban districts, peri-urban zones, and remote rural communities. From prior nationwide records, the regulator has a benchmark indicating that about 57\% of households use the platform, with corresponding long-run variability of approximately 0.2451.

		To assess current usage under the expanded rollout, two large survey waves are combined, and analysts report a 90\% confidence interval for the share of households using the platform from 0.52 to 0.62.

		Explain which values describe long-run adoption behavior versus those obtained from survey data, distinguish between single-value and range-based estimation for proportions, and interpret the reported interval in context while accounting for the broader rollout setting.

	\end{multicols}

\end{document}
