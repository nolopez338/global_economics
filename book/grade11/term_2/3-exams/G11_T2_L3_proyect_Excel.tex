\documentclass[12pt]{article}

% Page size and tighter margins
\usepackage[a4paper,left=1.2cm,right=1.2cm,top=1.5cm,bottom=1.5cm]{geometry}

% Core packages
\usepackage{graphicx}
\usepackage{xcolor}
\usepackage{array}
\usepackage{tabularx}
\usepackage{multicol}
\usepackage{amsmath}
\usepackage[T1]{fontenc}
\usepackage[utf8]{inputenc}
\usepackage{enumitem}


\setlength{\parindent}{0pt}
\setlength{\tabcolsep}{6pt}
\renewcommand{\arraystretch}{1.15}
\setlength{\emergencystretch}{3em}

% Column types
\newcolumntype{Y}{>{\raggedright\arraybackslash}m{\dimexpr0.30\textwidth-2\tabcolsep-2\arrayrulewidth\relax}}
\newcolumntype{Z}{>{\raggedright\arraybackslash}m{\dimexpr0.70\textwidth-2\tabcolsep-2\arrayrulewidth\relax}}
\newcolumntype{C}[1]{>{\centering\arraybackslash}m{#1}}

% Gray subsection header box
\newcommand{\SubsectionBox}[1]{%
	\noindent\colorbox{gray!30}{%
		\parbox{\linewidth}{\textbf{#1}}%
	}\par\vspace{0.35cm}%
}

% Centered multi-line cell helper
\newcommand{\CellCenter}[1]{%
	\parbox{\linewidth}{\centering #1}%
}

\begin{document}
	
	% =========================
	% HEADER BOX (3 COLUMNS)
	% =========================
	\noindent
	\begin{tabularx}{\textwidth}{|C{2.8cm}|C{\dimexpr\textwidth-6cm-4\tabcolsep-4\arrayrulewidth\relax}|C{2.8cm}|}
		\hline
		\centering
		\vspace{3mm}
		\includegraphics[width=2.5cm]{../../preamble/logo.png}
		&
		\CellCenter{%
			\vspace{-5mm}
			\textbf{GLOBAL ECONOMICS}\par
			\textbf{COMPUTATIONAL THINKING}\par
			\textbf{GRADE: 11TH}\par
			\textbf{TERM 2 -- PROJECT}\par
			\textbf{EXCEL t-STUDENT CONFIDENCE INTERVAL PROJECT}\par
			\textbf{Nicolás López Cuéllar, Jaime David Navarrete}
		}
		&
		\CellCenter{%
			\textbf{SECOND TERM}\par
			\textbf{2025--2026}%
		}
		\\
		\hline
	\end{tabularx}
	
	\vspace{0.5cm}
	
	% =========================
	% OBJECTIVE + CRITERIA
	% =========================
	\noindent
	\begin{tabular}{|Y|Z|}
		\hline
		{\small
			\textbf{Learning objective:} Present and defend an Excel tool that estimates a confidence interval for the mean with unknown variance (t-student), using dashboards and simple macros to support contextual conclusions in finance and economy.
		}
		&
		{\footnotesize
			\textbf{Assessment criteria:}\par
			\textbf{Global Economics}\par
			C10: Concludes about a statistical parameter in situations in finance and economy.\par
			\medskip
			\textbf{Computational Thinking}\par
			C8: Infers conclusions based on data analysis results.\par
			C9: Justifies conclusions through interactive dashboards.\par
			C10: Formulates solutions to different scenarios described by data using dashboards and simple macros.\par
		}
		\\
		\hline
	\end{tabular}
	
	\vspace{0.4cm}
	
	\begin{multicols}{2}
		
		\vspace{0.25cm}
		\SubsectionBox{1. Project brief}\vspace{-0.25cm}
		
		\textbf{1.1 Purpose of the project}\par
		This interdisciplinary proyect requires students to design and implement an Excel workbook that receives real economic or financial data and estimates confidence intervals for the population mean when variance is unknown.
		The workbook must make the procedure simple, transparent, and repeatable through dashboards and short macros.
		The presentation and defense of this proyect are the direct evidence for C10 in both subjects.
		Presenting this proyect will also allow the student to present problems for:
		\begin{itemize}
			\item C8: Calculates confidence intervals applying t-student distribution in context situations.
			\item C9: Estimates the confidence interval for the mean when the variance is unknown in context situations.
		\end{itemize}
		
		\textbf{1.2 Computational Thinking Background}\par
		Before this proyect, students have already worked with:
		\begin{itemize}[noitemsep, topsep=0pt, parsep=0pt, partopsep=0pt]
			\item C1: Integrates external data sources into Excel.
			\item C2: Understands array formulas in spreadsheets.
			\item C3: Uses array formulas in spreadsheets.
			\item C4: Understands macros in spreadsheets.
			\item C5: Uses macros to automate repetitive tasks in spreadsheets.
			\item C6: Identifies Data Analysis tools in spreadsheets.
			\item C7: Uses Data Analysis tools in spreadsheets.
		\end{itemize}
		
		\textbf{1.3 Input specification}\par
		The Excel workbook must receive:
		\begin{itemize}
			\item A dataset of numerical observations (raw data), typed, pasted, or imported from an external source.
			\item Optional metadata: context label, units, date range, and scenario label.
			\item A confidence level selector (for example 90\%, 95\%, or 99\%).
		\end{itemize}
		
		\textbf{1.4 Processing requirements}\par
		The workbook must:
		\begin{itemize}
			\item Compute sample size $n$, sample mean $\bar{x}$, sample standard deviation $s$, and standard error $SE=\frac{s}{\sqrt{n}}$.
			\item For unknown variance, compute and report t-based outputs: t statistic/critical value used for the interval ($t^*$ or equivalent), margin of error $ME_t=t^*\cdot SE$, and confidence interval $CI_t=(\bar{x}-ME_t,\bar{x}+ME_t)$.
			\item Compute and report z-based approximation outputs: z statistic/critical value used for the interval ($z^*$ or equivalent), margin of error $ME_z=z^*\cdot SE$, and confidence interval $CI_z=(\bar{x}-ME_z,\bar{x}+ME_z)$.
			\item Provide a clear conclusion template in context language, prioritizing the t-based interval when variance is unknown.
			\item Use dashboards (dropdowns, slicers, scenario selectors, and charts) to justify conclusions.
			\item Use simple macros to:
			\begin{itemize}
				\item clear or reset inputs,
				\item recompute and snapshot outputs for different scenarios,
				\item export or log scenario results in a comparison table that stores both $CI_t$ and $CI_z$.
			\end{itemize}
		\end{itemize}
		
		\textbf{1.5 Output specification}\par
		For each dataset and selected confidence level, the workbook output must include:
		\begin{itemize}
			\item \textbf{t-based outputs (variance unknown):} $n$, $\bar{x}$, $s$, $SE=\frac{s}{\sqrt{n}}$, $t^*$ (or equivalent t statistic), $ME_t=t^*\cdot SE$, and $CI_t=(\bar{x}-ME_t,\bar{x}+ME_t)$.
			\item \textbf{z-based approximation outputs:} $z^*$ (or equivalent z statistic), $ME_z=z^*\cdot SE$, and $CI_z=(\bar{x}-ME_z,\bar{x}+ME_z)$.
			\item Dashboard visuals supporting the justification.
			\item A scenario comparison view with at least two scenarios, supported by dashboard and/or macro outputs.
		\end{itemize}
		
		\textbf{1.6 Comparison requirement}\par
		The dashboard must display $CI_t$ and $CI_z$ side-by-side and include a short interpretation comparing interval width differences, when both methods are close, and the role of sample size in reducing the t-vs-z gap.
		
		\textbf{1.7 Dashboard and workflow requirement}\par
		\begin{enumerate}
			\item Import or paste raw data and define metadata and confidence level.
			\item Execute workbook calculations to obtain both t-based and z-based outputs.
			\item Use a dashboard toggle or side-by-side view for t vs z values, including explicit visualization of both intervals.
			\item Report $ME_t-ME_z$ and/or interval-width difference, and use scenario selectors to compare contexts and justify conclusions.
		\end{enumerate}
		
		\newpage
		\SubsectionBox{2. Solved Example 1}
		
		\textbf{Financial context: daily exchange-rate spread in a broker desk}\par
		A brokerage team analyzes the daily spread (basis points) observed in a currency operation window.
		Raw data are entered as individual observations in Excel:
		
		\begin{center}
			\begin{tabular}{l}
				\hline
				Spread data (bp) \\
				\hline
				49.2, 51.4, 47.8, 52.1, 50.5, 48.9,\\
				53.0, 49.7, 50.9, 51.8, 48.6, 52.4 \\
				\hline
			\end{tabular}
		\end{center}
		The dashboard selector is set to \textbf{95\% confidence} and scenario \textbf{Base week}.
		
		\textbf{Workbook output}\par
		\begin{center}
			\begin{tabular}{l c}
				\hline
				Metric & Value \\
				\hline
				$n$ & 12 \\
				$\bar{x}$ (bp) & 50.52 \\
				$s$ (bp) & 1.68 \\
				$SE$ & 0.484 \\
				$t^*$ (95\%, $df=11$) & 2.201 \\
				$ME_t$ & 1.06 \\
				$CI_t$ lower bound & 49.46 \\
				$CI_t$ upper bound & 51.59 \\
				$z^*$ (95\% normal approx.) & 1.960 \\
				$ME_z$ & 0.95 \\
				$CI_z$ lower bound & 49.57 \\
				$CI_z$ upper bound & 51.47 \\
				$ME_t-ME_z$ & 0.11 \\
				\hline
			\end{tabular}
		\end{center}
		
		\textbf{Conclusion}\par
		With 95\% confidence, using the preferred t-based method for unknown variance, the mean daily spread for the selected week is between 49.46 and 51.59 basis points. The z-approximation interval is 49.57 to 51.47 basis points, which is close but slightly narrower.
		
		\vspace{1cm}
		\SubsectionBox{3. Solved Example 2}
		
		\textbf{Economic context: import shipping cost per container}\par
		An import company records shipping cost per container (thousand USD) for one route.
		Data are imported from CSV to Excel as individual observations:
		
		\begin{center}
			\begin{tabular}{l}
				\hline
				Cost data (thousand USD) \\
				\hline
				1.72, 1.81, 1.75, 1.88, 1.93, 1.79,\\
				1.85, 1.90, 1.77, 1.84, 1.96, 1.87,\\
				1.80, 1.92, 1.74, 1.89, 1.83, 1.95 \\
				\hline
			\end{tabular}
		\end{center}
		The dashboard selector is set to \textbf{90\% confidence} and the route scenario \textbf{Port A -- Q2}.
		
		\textbf{Workbook output}\par
		\begin{center}
			\begin{tabular}{l c}
				\hline
				Metric & Value \\
				\hline
				$n$ & 18 \\
				$\bar{x}$ (thousand USD) & 1.844 \\
				$s$ (thousand USD) & 0.074 \\
				$SE$ & 0.017 \\
				$t^*$ (90\%, $df=17$) & 1.740 \\
				$ME_t$ & 0.030 \\
				$CI_t$ lower bound & 1.814 \\
				$CI_t$ upper bound & 1.874 \\
				$z^*$ (90\% normal approx.) & 1.645 \\
				$ME_z$ & 0.028 \\
				$CI_z$ lower bound & 1.816 \\
				$CI_z$ upper bound & 1.872 \\
				$ME_t-ME_z$ & 0.002 \\
				\hline
			\end{tabular}
		\end{center}
		
		\textbf{Conclusion}\par
		With 90\% confidence, using the preferred t-based method for unknown variance, the mean shipping cost parameter for this route is between 1.814 and 1.874 thousand USD per container. The z-approximation interval is 1.816 to 1.872 thousand USD, very close to the t-based result in this case.
		
	\end{multicols}
	
\end{document}
