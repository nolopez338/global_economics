\makeatletter
\def\input@path{{./}{../}{../../}{preamble/}{../preamble/}{../../preamble/}}
\makeatother
% ----------------------------------------------------------
% GENERAL 

% File
\documentclass[11pt]{book}

% Margins
\usepackage[margin=1in]{geometry}

\linespread{1.2}            % Line spacing
\usepackage[utf8]{inputenc}
\usepackage[T1]{fontenc}
\usepackage{lmodern}
\usepackage{microtype}
\setlength{\parindent}{0pt}
\setlength{\parskip}{6pt}
\usepackage{booktabs}

% ----------------------------------------------------------
% TABLES
\usepackage{multicol}
\usepackage{longtable} 
\usepackage{array}
\usepackage{booktabs}
\usepackage{tabularx}
\usepackage{multirow}

% ----------------------------------------------------------
% MATHEMATICS
\usepackage{amsmath}
\usepackage{amssymb}
\usepackage{amsfonts}
\usepackage{mathtools}

% ----------------------------------------------------------
% HIDDEN CONTENT
\usepackage{ifthen}
% Define a boolean switch
\newboolean{explicaciones}
% Set the boolean switch to true or false
% Change to true to show the content

% Explanations
\newcommand{\explicacion}[2]{
	\ifthenelse{\boolean{explicaciones}}{#1}{#2}
}
\newcommand{\mostrarExplicaciones}[1]{\setboolean{explicaciones}{#1}}

% ----------------------------------------------------------
% NUMBERING

\usepackage{fancyhdr}
\pagestyle{empty} % Ensures the entire document has no page numbers

\usepackage{tocloft}
\renewcommand{\cftdot}{} % Remove dots for sections, if any
\renewcommand{\cftsecleader}{\cftdotfill{\cftdotsep}} % Remove dots for sections, if any
\cftpagenumbersoff{section} % Removes page numbers from sections
\cftpagenumbersoff{subsection} % Removes page numbers from subsections

% ----------------------------------------------------------
% IMAGES 

% General settings
\usepackage{graphicx}       % Insert images
\usepackage{float}          % Position images
% \usepackage{subfigure}      % Subfigures
\graphicspath{{imgs}}       % Image location
\usepackage{subcaption}     % Subfigures II
\usepackage{verbatim}

% Figures
\usepackage{tikz}
\usetikzlibrary{arrows.meta,positioning,trees}

% Colors
\usepackage{xcolor}     
\definecolor{popUp}{HTML}{666666}
\definecolor{popUpIn}{HTML}{CED9E0}
\definecolor{backgroundC}{HTML}{D0E8F2}
\definecolor{backgroundCC}{HTML}{FFFFFF}
\definecolor{borders}{HTML}{8c120d}
\definecolor{padding}{HTML}{77D0D7}
\definecolor{links}{HTML}{CC6F5F}

% ----------------------------------------------------------
% FRAMES

% General settings
\usepackage{tcolorbox}
\usepackage{adjustbox}          % Adjusted frame  
\setlength{\fboxrule}{3pt}  % Line width
\setlength{\fboxsep}{3pt}   % Box padding

% General frames
\usepackage{mdframed}   

\mdfdefinestyle{estiloGeneral}{    % General style
	linecolor=black,
	linewidth=1.5pt,
	roundcorner=10pt,
	backgroundcolor=backgroundC,
	innerbottommargin=0pt
}
\mdfdefinestyle{code}{          % Code style
	linecolor=black,
	linewidth=1.5pt,
	roundcorner=10pt,
	backgroundcolor=darkgray!10,
	innerbottommargin=0pt
}

% Image frame
\newtcbox{\fboxC}{
	colback=backgroundC,
	colframe=popUp,
	arc=10pt,
	boxrule=3pt,
	boxsep=0pt, % Change the padding here
	nobeforeafter
}

% ----------------------------------------------------------
% PAGE SETTINGS

% Background 
\newcommand{\background}[0]{\begin{tikzpicture}[remember picture,overlay]
		\fill[backgroundC] (-2,2) rectangle (25cm, -550);
\end{tikzpicture}}

\newcommand{\backgroundC}[0]{\begin{tikzpicture}[remember picture,overlay]
		\fill[backgroundCC] (-2,2) rectangle (25cm, -550);
\end{tikzpicture}}

% Page width 
\newcommand{\anchoPag}[0]{20cm}

% ----------------------------------------------------------
% FONT

% General
\usepackage{tgbonum}        % Font
\usepackage{listings}       % Code typesetting
\usepackage[spanish]{babel} % Load Spanish
\selectlanguage{spanish}    % Select Spanish
\usepackage{enumitem}
\usepackage{bookmark}

\setlist[itemize]{leftmargin=1.2em, itemsep=0.35em, topsep=0.35em}

% --- Table helpers ---
\newcolumntype{L}[1]{>{\raggedright\arraybackslash}p{#1}}
\newcolumntype{Y}{>{\raggedright\arraybackslash}X}
\newcolumntype{C}{>{\centering\arraybackslash}X}
\renewcommand{\arraystretch}{1.1}

% Python style
\lstdefinestyle{python}{
	language=Python,
	basicstyle=\ttfamily\small,
	commentstyle=\color{green!50!black},
	keywordstyle=\color{blue},
	numberstyle=\tiny\color{gray},
	numbers=left,
	morekeywords={>, <},
	breakatwhitespace=false,
	showstringspaces=false,
	showtabs=false,
	showspaces=false
}

% ----------------------------------------------------------
% HYPERLINKS

% General
\usepackage{hyperref}       
\hypersetup{
	colorlinks=true,
	linkcolor=links,
	filecolor=magenta,      
	urlcolor=brown,
}

% Custom commands 

% Large link
\newcommand{\bigLink}[2]{\begin{center} \fboxC{\LARGE{\href{#1}{#2}}}\end{center}}

% Small link
\newcommand{\smallLink}[2]{\begin{center}\fboxC{\href{#1}{#2}}\end{center}}

% Bold link
\newcommand{\bfLink}[2]{\href{#1}{\textbf{#2}}}


% Small URL
\newcommand{\smallUrl}[1]{\begin{center}\fboxC{\url{#1}}\end{center}}


% ----------------------------------------------------------
% CUSTOM COMMANDS FOR FIGURES

\newcommand{\espacioImagenes}[0]{-1.2cm}

% Without frame
\newcommand{\fig}[3][\espacioImagenes]{
	\hspace*{#1}
	\centering
	\includegraphics[width=#2\textwidth]{#3}
}

% With frame
\newcommand{\ffig}[2]{\begin{figure}[h]
		\hspace*{\espacioImagenes}
		\centering
		\fbox{\includegraphics[width=#1\textwidth]{#2}}
\end{figure}}

% Hyperlink with frame
\newcommand{\hfig}[3]{\begin{figure}[h]
		\hspace*{-1.4cm}
		\centering
		\color{popUp}
		\fboxC{\href{#1}{\includegraphics[width=#2\textwidth]{#3}}}
	\end{figure}
}

% Hyperlink without frame
\newcommand{\hffig}[3]{\begin{figure}[h]
		\hspace*{-1.1cm}
		\centering
		\color{popUp}
		\href{#1}{\includegraphics[width=#2\textwidth]{#3}}
	\end{figure}
}

% ----------------------------------------------------------

% Start and Contents
\newcommand{\cuadro}[1]{
	\begin{mdframed}[style=estiloGeneral]
		#1 
	\end{mdframed}
}

% Explanation video image
\newcommand{\linkExplicacion}[1]{
	\hffig{#1}{0.5}{principal/videoExplicacion}
	\vspace{-0.5cm}
}

\newcommand{\subSecLink}[2]{
	\subsubsection*{\href{#1}{\textbf{#2}}}
}

% Spacing
\newcommand{\esp}[0]{\vspace{4mm}}

% Back to start
\newcommand{\secInicio}[0]{\begin{center}\hyperref[sec:inicio]{ 
			\includegraphics[width=0.1\textwidth]{principal/up}
	}\end{center}
}


\geometry{margin=0.85in}
\AtBeginDocument{\small}

\newcommand{\ExamNameField}{\noindent\textbf{Name:}\ \rule{0.7\linewidth}{0.4pt}\par\medskip}

\newcommand{\ExamTitleBlock}[3]{%
	\begin{center}
		\Large\textbf{#1}\\
		\textbf{#2}%
		\if\relax\detokenize{#3}\relax\else\\\textbf{#3}\fi
	\end{center}
	\vspace{0.5em}
}

\newcommand{\ExamSection}[1]{\par\medskip\textbf{#1}\par\smallskip}

\newenvironment{ExamCriteria}{%
	\begin{itemize}[leftmargin=1.6em, itemsep=0.3em, topsep=0.2em]
}{%
	\end{itemize}
}

\newenvironment{ExamProblems}{%
	\begin{enumerate}[label=\textbf{P\arabic*}, leftmargin=0pt, labelsep=0.6em, itemindent=2.2em, itemsep=0.8em]
}{%
	\end{enumerate}
}

\begin{document}
	\ExamTitleBlock{11th grade}{Term 2 Catch-Up: C2 Population vs Sample (Solutions)}{}

	\ExamSection{Problems}
	\begin{ExamProblems}
		\item
		\subsection*{Problem 1}
		\subsection*{Studio Rent Record Comparison}
		A rental platform has years of records on monthly studio rent for young professionals in one city area, and across that full record the typical monthly level is 1850 USD.
		The same full platform dataset reports overall variability of 10000 in (USD)$^2$.
		In a recent audit of listings from one shorter period, the observed average in the reviewed group is 1810 USD, with variability summarized by 8100 in (USD)$^2$.
		From the context above, identify which values describe the long-run behavior of the entire population and which values describe only the reviewed sample.

\subsection*{C2}
		The population parameters are $\mu = 1850$ USD and $\sigma^2 = 10000$ (USD)$^2$.
		These describe the long-run center and long-run spread of all monthly studio rents in this city area.

		The sample statistic for location is $\bar{x} = 1810$ USD.
		This is the sample mean rent from the observed group, not the full population.

		For spread, the sample statistic is $s^2 = 8100$ (USD)$^2$.
		Convert both variances to standard deviations for an easier spread comparison:
		\[
		\sigma = \sqrt{10000} = 100 \text{ USD},
		\qquad
		s = \sqrt{8100} = 90 \text{ USD}.
		\]
		So the population standard deviation/spread is 100 USD, and the sample standard deviation/spread is 90 USD.

		The sample mean is 40 USD below the population mean, and the sample spread is slightly smaller than the population spread.

		\newpage
		\item
		\subsection*{Problem 2}
		\subsection*{Graduate Offer Mean Assessment}
		A recently reviewed batch of new offers has an observed average of 53100 USD, with spread reported as 5500 USD.
		A company keeps a long-term database of first-year salary offers for recent graduates hired into a data support role, and across all offers in that database the central level is 52000 USD.
		The same complete record shows a typical spread of 6000 USD around that value.
		Based on this scenario, determine which quantities refer to the full population over time and which refer only to the reviewed sample.

\subsection*{C2}
		The long-term typical value is 52,000 USD, and the long-term spread is 6,000 USD.
		The observed average in the reviewed group is 53,100 USD, and the observed variability in that group is 5,500 USD.

		For a squared-spread comparison, the long-term spread gives 36,000,000 (USD)$^2$ because 6,000 multiplied by 6,000 is 36,000,000.
		The observed variability gives 30,250,000 (USD)$^2$ because 5,500 multiplied by 5,500 is 30,250,000.

		The observed average in the reviewed group is higher than the long-term typical value, while the observed variability is somewhat smaller than the long-term spread.


		\newpage
		\item
		\subsection*{Problem 3}
		\subsection*{Gig Platform Earnings Benchmarks}
		Two gig platforms are analyzed separately using their own data systems.
		For Platform A (food delivery), company-wide weekly earning records indicate an overall level of 720 USD, while a recently reviewed rider group has an observed average of 750 USD.
		Its long-run variability is 4900 in (USD)$^2$, and the reviewed group has variability of 6400 in (USD)$^2$.
		For Platform B (freelance design), full-platform records indicate 980 USD as the long-run typical level, while a recent creator group has an observed average of 940 USD.
		Its overall spread is 80 USD, and the reviewed group spread is 70 USD.
		Using the information provided, identify which values are population-level descriptors and which values summarize only the reviewed samples.

\subsection*{C2}
		For Platform A, the long-term typical value is 720 USD and the long-term spread is given in squared form as 4,900 (USD)$^2$.
		The observed average in the reviewed group is 750 USD, and the observed variability in that group is 6,400 (USD)$^2$.
		Taking square roots, the long-term spread is 70 USD and the observed variability in that group is 80 USD.

		For Platform B, the long-term typical value is 980 USD and the long-term spread is 80 USD.
		The observed average in the reviewed group is 940 USD, and the observed variability in that group is 70 USD.
		In squared form, the long-term spread corresponds to 6,400 (USD)$^2$, and the observed variability corresponds to 4,900 (USD)$^2$.

		For Platform A, the observed average in the reviewed group is above the long-term typical value and the observed variability is larger than the long-term spread.
		For Platform B, the observed average in the reviewed group is below the long-term typical value and the observed variability is smaller than the long-term spread.


		\newpage
		\item
		\subsection*{Problem 4}
		\subsection*{Metro Card Spending Review}
		A bank tracks monthly credit card spending for customers in their 20s across one metro area over a long period.
		Two separate recent review groups were then pulled from that same metro customer base.
		Group 1 has an observed average of 1290 USD and a reported spread of 280 USD.
		Group 2 has an observed average of 1410 USD and a reported spread of 320 USD.
		That full customer record is summarized by a long-term typical level of 1350 USD and a typical spread of 300 USD.
		From this description, state which values characterize the entire long-run population and which values belong only to the reviewed groups.

\subsection*{C2}
		For this metro area, the long-term typical value is 1,350 USD and the long-term spread is 300 USD.

		For Group 1, the observed average in the reviewed group is 1,290 USD and the observed variability in that group is 280 USD.
		For Group 2, the observed average in the reviewed group is 1,410 USD and the observed variability in that group is 320 USD.

		In squared form, the long-term spread corresponds to 90,000 (USD)$^2$.
		For Group 1, the observed variability corresponds to 78,400 (USD)$^2$.
		For Group 2, the observed variability corresponds to 102,400 (USD)$^2$.

		Group 1 is below the long-term typical value with smaller variability, while Group 2 is above the long-term typical value with larger variability.


		\newpage
		\item
		\subsection*{Problem 5}
		\subsection*{Emergency Deposit Mean Review}
		A savings app monitors monthly deposits into emergency funds made by users in their 20s.
		Across the app's full user history, the deposit level is summarized by a long-term typical value of 420 USD and overall variability of 1600 in (USD)$^2$.
		Three recent review groups, all drawn from that same user base, were summarized as follows.
		Group A has an observed average of 400 USD, and its variability is 900 in (USD)$^2$.
		Group B has an observed average of 435 USD, and its variability is 2500 in (USD)$^2$.
		Group C has an observed average of 415 USD, and its variability is 3600 in (USD)$^2$.
		Based on the context, indicate which quantities describe the full user population in the long run and which quantities describe only the reviewed samples.

\subsection*{C2}
		The long-term typical value is 420 USD, and the long-term spread is provided in squared form as 1,600 (USD)$^2$.
		The square root of 1,600 is 40, so the long-term spread is 40 USD.

		For Group A, the observed average in the reviewed group is 400 USD and the observed variability in that group is 900 (USD)$^2$.
		For Group B, the observed average in the reviewed group is 435 USD and the observed variability in that group is 2,500 (USD)$^2$.
		For Group C, the observed average in the reviewed group is 415 USD and the observed variability in that group is 3,600 (USD)$^2$.

		Taking square roots gives observed variability values of 30 USD for Group A, 50 USD for Group B, and 60 USD for Group C.

		Compared with the long-term typical value and long-term spread, Group A has a lower average and smaller variability, Group B has a higher average and larger variability, and Group C has a slightly lower average with larger variability.


		\newpage
		\item
		\subsection*{Problem 6}
		\subsection*{Subscription Spending Interval Check}
		A streaming platform studies monthly subscription spending by young adults using both long-term company records and a recent account review.
		For one reviewed group, the 95\% confidence interval for that group's average spending runs from 58 USD to 74 USD.
		Across the full platform history, variability is summarized by 400 in (USD)$^2$, while the reviewed group's spread is reported as 24 USD.
		Using the narrative above, identify which reported values represent population information and which represent sample-only information.

\subsection*{C2}
		The long-term spread is provided in squared form as 400 (USD)$^2$.
		Taking the square root gives 20 USD as the long-term spread.
		No long-term typical value is provided in this problem.

		The observed average in the reviewed group comes from the midpoint of 58 and 74.
		Adding the endpoints gives 132, and dividing by two gives 66 USD.

		The observed variability in that group is reported as 24 USD.
		In squared form, that observed variability corresponds to 576 (USD)$^2$ because 24 multiplied by 24 is 576.

		So the observed average in the reviewed group is 66 USD, and the observed variability in that group is greater than the long-term spread.


		\newpage
		\item
		\subsection*{Problem 7}
		\subsection*{Weekend Trip Spending Assessment}
		A travel-budget app tracks weekend trip spending for users in their 20s throughout the year.
		Across all users in its full records, the spending level is summarized by a long-term typical value of 460 USD with typical spread of 70 USD.
		For one recently reviewed user group, the 90\% confidence interval for the group's average runs from 420 USD to 480 USD, and the group's spread is reported as 60 USD.
		From the details given, determine which values correspond to the full population behavior over time and which correspond only to the reviewed sample.

\subsection*{C2}
		The long-term typical value is 460 USD, and the long-term spread is 70 USD.

		The observed average in the reviewed group comes from the midpoint of 420 and 480.
		Adding the endpoints gives 900, and dividing by two gives 450 USD.

		The observed variability in that group is 60 USD.
		In squared form, the long-term spread corresponds to 4,900 (USD)$^2$, while the observed variability corresponds to 3,600 (USD)$^2$.

		The observed average in the reviewed group is 10 USD below the long-term typical value, and the observed variability in that group is smaller than the long-term spread.


		\newpage
		\item
		\subsection*{Problem 8}
		\subsection*{Online Store Revenue Intervals}
		Two small online stores are analyzed separately, each with its own full business records and its own recent review group.
		For one reviewed period at Store A (handmade accessories), the 95\% confidence interval for average daily revenue runs from 240 USD to 300 USD, and variability is reported as 1600 in (USD)$^2$.
		Store A's full records show a long-run daily revenue level of 275 USD and the overall spread of 50 USD.
		For one reviewed period at Store B (digital templates), the 90\% confidence interval for average daily revenue runs from 330 USD to 390 USD, and the spread is 55 USD.
		Store B's full records are summarized by a long-run typical value of 350 USD and variability of 3600 in (USD)$^2$.
		Based on this context, identify which values describe long-run population patterns and which values describe only the reviewed samples.

\subsection*{C2}
		For Store A, the long-term typical value is 275 USD and the long-term spread is 50 USD.
		The observed average in the reviewed group is the midpoint of 240 and 300.
		Adding those endpoints gives 540, and dividing by two gives 270 USD.
		The observed variability in that group is provided in squared form as 1,600 (USD)$^2$.
		Taking the square root gives 40 USD for the observed variability in that group.

		For Store B, the long-term typical value is 350 USD and the long-term spread is provided in squared form as 3,600 (USD)$^2$.
		Taking the square root gives 60 USD as the long-term spread.
		The observed average in the reviewed group is the midpoint of 330 and 390.
		Adding those endpoints gives 720, and dividing by two gives 360 USD.
		The observed variability in that group is 55 USD.
		In squared form, the observed variability corresponds to 3,025 (USD)$^2$.

		Store A has an observed average slightly below its long-term typical value with smaller variability.
		Store B has an observed average above its long-term typical value with slightly smaller variability than its long-term spread.


		\newpage
		\item
		\subsection*{Problem 9}
		\subsection*{Rider Earnings Group Comparison}
		A delivery company analyzes weekly rider earnings in one city using complete citywide records plus two recent rider review groups from that same city.
		The full city record is summarized by a long-term typical value of 740 USD and variability of 6400 in (USD)$^2$.
		For Review Group 1, the 95\% confidence interval for average weekly earnings runs from 680 USD to 760 USD.
		For Review Group 2, the 90\% confidence interval for average weekly earnings runs from 700 USD to 780 USD.
		Spread details are reported in mixed form: Review Group 1 has spread of 70 USD, and Review Group 2 has variability of 4900 in (USD)$^2$.
		Using the information above, determine which quantities belong to the full long-run population and which belong only to the reviewed samples.

\subsection*{C2}
		The long-term typical value is 740 USD, and the long-term spread is provided in squared form as 6,400 (USD)$^2$.
		Taking the square root gives 80 USD as the long-term spread.

		For Review Group 1, the observed average in the reviewed group is the midpoint of 680 and 760.
		Adding the endpoints gives 1,440, and dividing by two gives 720 USD.

		For Review Group 2, the observed average in the reviewed group is the midpoint of 700 and 780.
		Adding the endpoints gives 1,480, and dividing by two gives 740 USD.

		For observed variability, Review Group 1 is reported as 70 USD, which corresponds to 4,900 (USD)$^2$ in squared form.
		Review Group 2 is reported as 4,900 (USD)$^2$ in squared form, which corresponds to 70 USD after taking the square root.

		Review Group 1 is below the long-term typical value, Review Group 2 matches the long-term typical value, and both groups have smaller variability than the long-term spread.


		\newpage
		\item
		\subsection*{Problem 10}
		\subsection*{Net Deposit Interval Ranking}
		An investment app tracks monthly net deposits (deposits minus withdrawals) for users in their 20s.
		Three recent review groups were drawn from that same user base, and their confidence intervals for average net deposits are:
		Sample A (95\%): from 580 USD to 640 USD,
		Sample B (90\%): from 600 USD to 660 USD,
		Sample C (95\%): from 550 USD to 630 USD.
		Their spread summaries are 6400 in (USD)$^2$ for Sample A, 85 USD for Sample B, and 10000 in (USD)$^2$ for Sample C.
		Across the full user base over time, net deposits are summarized by a long-term typical value of 620 USD with spread of 90 USD.
		From this problem statement, identify which values describe population-level behavior over time and which values summarize only the reviewed samples.

\subsection*{C2}
		The long-term typical value is 620 USD and the long-term spread is 90 USD.
		In squared form, the long-term spread corresponds to 8,100 (USD)$^2$.

		For Sample A, the observed average in the reviewed group is the midpoint of 580 and 640.
		Adding the endpoints gives 1,220, and dividing by two gives 610 USD.

		For Sample B, the observed average in the reviewed group is the midpoint of 600 and 660.
		Adding the endpoints gives 1,260, and dividing by two gives 630 USD.

		For Sample C, the observed average in the reviewed group is the midpoint of 550 and 630.
		Adding the endpoints gives 1,180, and dividing by two gives 590 USD.

		For observed variability, Sample A is provided in squared form as 6,400 (USD)$^2$, which corresponds to 80 USD after taking the square root.
		Sample B is provided as 85 USD, which corresponds to 7,225 (USD)$^2$ in squared form.
		Sample C is provided in squared form as 10,000 (USD)$^2$, which corresponds to 100 USD after taking the square root.

		Sample B is above the long-term typical value, while Samples A and C are below it.
		Compared with the long-term spread, variability is smaller for Samples A and B but larger for Sample C.


		\end{ExamProblems}
\end{document}
