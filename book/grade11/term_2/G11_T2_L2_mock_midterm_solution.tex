\makeatletter
\def\input@path{{./}{../}{../../}{preamble/}{../preamble/}{../../preamble/}}
\makeatother
% ----------------------------------------------------------
% GENERAL 

% File
\documentclass[11pt]{book}

% Margins
\usepackage[margin=1in]{geometry}

\linespread{1.2}            % Line spacing
\usepackage[utf8]{inputenc}
\usepackage[T1]{fontenc}
\usepackage{lmodern}
\usepackage{microtype}
\setlength{\parindent}{0pt}
\setlength{\parskip}{6pt}
\usepackage{booktabs}

% ----------------------------------------------------------
% TABLES
\usepackage{multicol}
\usepackage{longtable} 
\usepackage{array}
\usepackage{booktabs}
\usepackage{tabularx}
\usepackage{multirow}

% ----------------------------------------------------------
% MATHEMATICS
\usepackage{amsmath}
\usepackage{amssymb}
\usepackage{amsfonts}
\usepackage{mathtools}

% ----------------------------------------------------------
% HIDDEN CONTENT
\usepackage{ifthen}
% Define a boolean switch
\newboolean{explicaciones}
% Set the boolean switch to true or false
% Change to true to show the content

% Explanations
\newcommand{\explicacion}[2]{
	\ifthenelse{\boolean{explicaciones}}{#1}{#2}
}
\newcommand{\mostrarExplicaciones}[1]{\setboolean{explicaciones}{#1}}

% ----------------------------------------------------------
% NUMBERING

\usepackage{fancyhdr}
\pagestyle{empty} % Ensures the entire document has no page numbers

\usepackage{tocloft}
\renewcommand{\cftdot}{} % Remove dots for sections, if any
\renewcommand{\cftsecleader}{\cftdotfill{\cftdotsep}} % Remove dots for sections, if any
\cftpagenumbersoff{section} % Removes page numbers from sections
\cftpagenumbersoff{subsection} % Removes page numbers from subsections

% ----------------------------------------------------------
% IMAGES 

% General settings
\usepackage{graphicx}       % Insert images
\usepackage{float}          % Position images
% \usepackage{subfigure}      % Subfigures
\graphicspath{{imgs}}       % Image location
\usepackage{subcaption}     % Subfigures II
\usepackage{verbatim}

% Figures
\usepackage{tikz}
\usetikzlibrary{arrows.meta,positioning,trees}

% Colors
\usepackage{xcolor}     
\definecolor{popUp}{HTML}{666666}
\definecolor{popUpIn}{HTML}{CED9E0}
\definecolor{backgroundC}{HTML}{D0E8F2}
\definecolor{backgroundCC}{HTML}{FFFFFF}
\definecolor{borders}{HTML}{8c120d}
\definecolor{padding}{HTML}{77D0D7}
\definecolor{links}{HTML}{CC6F5F}

% ----------------------------------------------------------
% FRAMES

% General settings
\usepackage{tcolorbox}
\usepackage{adjustbox}          % Adjusted frame  
\setlength{\fboxrule}{3pt}  % Line width
\setlength{\fboxsep}{3pt}   % Box padding

% General frames
\usepackage{mdframed}   

\mdfdefinestyle{estiloGeneral}{    % General style
	linecolor=black,
	linewidth=1.5pt,
	roundcorner=10pt,
	backgroundcolor=backgroundC,
	innerbottommargin=0pt
}
\mdfdefinestyle{code}{          % Code style
	linecolor=black,
	linewidth=1.5pt,
	roundcorner=10pt,
	backgroundcolor=darkgray!10,
	innerbottommargin=0pt
}

% Image frame
\newtcbox{\fboxC}{
	colback=backgroundC,
	colframe=popUp,
	arc=10pt,
	boxrule=3pt,
	boxsep=0pt, % Change the padding here
	nobeforeafter
}

% ----------------------------------------------------------
% PAGE SETTINGS

% Background 
\newcommand{\background}[0]{\begin{tikzpicture}[remember picture,overlay]
		\fill[backgroundC] (-2,2) rectangle (25cm, -550);
\end{tikzpicture}}

\newcommand{\backgroundC}[0]{\begin{tikzpicture}[remember picture,overlay]
		\fill[backgroundCC] (-2,2) rectangle (25cm, -550);
\end{tikzpicture}}

% Page width 
\newcommand{\anchoPag}[0]{20cm}

% ----------------------------------------------------------
% FONT

% General
\usepackage{tgbonum}        % Font
\usepackage{listings}       % Code typesetting
\usepackage[spanish]{babel} % Load Spanish
\selectlanguage{spanish}    % Select Spanish
\usepackage{enumitem}
\usepackage{bookmark}

\setlist[itemize]{leftmargin=1.2em, itemsep=0.35em, topsep=0.35em}

% --- Table helpers ---
\newcolumntype{L}[1]{>{\raggedright\arraybackslash}p{#1}}
\newcolumntype{Y}{>{\raggedright\arraybackslash}X}
\newcolumntype{C}{>{\centering\arraybackslash}X}
\renewcommand{\arraystretch}{1.1}

% Python style
\lstdefinestyle{python}{
	language=Python,
	basicstyle=\ttfamily\small,
	commentstyle=\color{green!50!black},
	keywordstyle=\color{blue},
	numberstyle=\tiny\color{gray},
	numbers=left,
	morekeywords={>, <},
	breakatwhitespace=false,
	showstringspaces=false,
	showtabs=false,
	showspaces=false
}

% ----------------------------------------------------------
% HYPERLINKS

% General
\usepackage{hyperref}       
\hypersetup{
	colorlinks=true,
	linkcolor=links,
	filecolor=magenta,      
	urlcolor=brown,
}

% Custom commands 

% Large link
\newcommand{\bigLink}[2]{\begin{center} \fboxC{\LARGE{\href{#1}{#2}}}\end{center}}

% Small link
\newcommand{\smallLink}[2]{\begin{center}\fboxC{\href{#1}{#2}}\end{center}}

% Bold link
\newcommand{\bfLink}[2]{\href{#1}{\textbf{#2}}}


% Small URL
\newcommand{\smallUrl}[1]{\begin{center}\fboxC{\url{#1}}\end{center}}


% ----------------------------------------------------------
% CUSTOM COMMANDS FOR FIGURES

\newcommand{\espacioImagenes}[0]{-1.2cm}

% Without frame
\newcommand{\fig}[3][\espacioImagenes]{
	\hspace*{#1}
	\centering
	\includegraphics[width=#2\textwidth]{#3}
}

% With frame
\newcommand{\ffig}[2]{\begin{figure}[h]
		\hspace*{\espacioImagenes}
		\centering
		\fbox{\includegraphics[width=#1\textwidth]{#2}}
\end{figure}}

% Hyperlink with frame
\newcommand{\hfig}[3]{\begin{figure}[h]
		\hspace*{-1.4cm}
		\centering
		\color{popUp}
		\fboxC{\href{#1}{\includegraphics[width=#2\textwidth]{#3}}}
	\end{figure}
}

% Hyperlink without frame
\newcommand{\hffig}[3]{\begin{figure}[h]
		\hspace*{-1.1cm}
		\centering
		\color{popUp}
		\href{#1}{\includegraphics[width=#2\textwidth]{#3}}
	\end{figure}
}

% ----------------------------------------------------------

% Start and Contents
\newcommand{\cuadro}[1]{
	\begin{mdframed}[style=estiloGeneral]
		#1 
	\end{mdframed}
}

% Explanation video image
\newcommand{\linkExplicacion}[1]{
	\hffig{#1}{0.5}{principal/videoExplicacion}
	\vspace{-0.5cm}
}

\newcommand{\subSecLink}[2]{
	\subsubsection*{\href{#1}{\textbf{#2}}}
}

% Spacing
\newcommand{\esp}[0]{\vspace{4mm}}

% Back to start
\newcommand{\secInicio}[0]{\begin{center}\hyperref[sec:inicio]{ 
			\includegraphics[width=0.1\textwidth]{principal/up}
	}\end{center}
}


\geometry{margin=0.85in}
\AtBeginDocument{\small}

\newcommand{\ExamNameField}{\noindent\textbf{Name:}\ \rule{0.7\linewidth}{0.4pt}\par\medskip}

\newcommand{\ExamTitleBlock}[3]{%
	\begin{center}
		\Large\textbf{#1}\\
		\textbf{#2}%
		\if\relax\detokenize{#3}\relax\else\\\textbf{#3}\fi
	\end{center}
	\vspace{0.5em}
}

\newcommand{\ExamSection}[1]{\par\medskip\textbf{#1}\par\smallskip}

\newenvironment{ExamCriteria}{%
	\begin{itemize}[leftmargin=1.6em, itemsep=0.3em, topsep=0.2em]
}{%
	\end{itemize}
}

\newenvironment{ExamProblems}{%
	\begin{enumerate}[label=\textbf{P\arabic*}, leftmargin=0pt, labelsep=0.6em, itemindent=2.2em, itemsep=0.8em]
}{%
	\end{enumerate}
}

\begin{document}
	\ExamTitleBlock{11th grade}{Term 2 Midterm Mock: Confidence Interval Planning (Solutions)}{}
	
	\ExamSection{Evaluated criteria}
	\begin{ExamCriteria}
		\item C1: Compute the sample mean and sample standard deviation.
		\item C2: Distinguish between sample statistics and population parameters.
		\item C3: Explain why interval estimation is preferred over a single point estimate.
		\item C4: Interpret the meaning of a X\% confidence interval in context.
		\item C5: Construct a X\% confidence interval using known population standard deviation.
		\item C6: Determine the required sample size to achieve a target margin of error.
		\item C7: Construct and interpret a confidence interval for a population proportion.
	\end{ExamCriteria}

	\ExamSection{Problems}
	\begin{ExamProblems}
		\newpage
		\item
		\subsection*{Problem description}
		A metropolitan transit authority studies the average weekday fare collected per rider for two expanded pilot corridors.
		The two samples are distinct random samples drawn from the same population of fare amounts.
		The grouped fare data (USD) are below.
		
		\begin{multicols}{2}
			\textbf{Sample A (20 riders):}
			\begin{itemize}
				\item 38 USD occurred in 4 riders.
				\item 45 USD occurred in 5 riders.
				\item 52 USD occurred in 4 riders.
				\item 60 USD occurred in 3 riders.
				\item 70 USD occurred in 4 riders.
			\end{itemize}
			
			\columnbreak
			
			\textbf{Sample B (24 riders):}
			\begin{itemize}
				\item 38 USD occurred in 5 riders.
				\item 45 USD occurred in 4 riders.
				\item 52 USD occurred in 6 riders.
				\item 60 USD occurred in 5 riders.
				\item 70 USD occurred in 4 riders.
			\end{itemize}
		\end{multicols}
		
		Long-run audits show the population standard deviation is $\sigma = 11.5$ USD.
		For classroom purposes, suppose the true population mean fare is $\mu = 53$ USD.
		Construct and interpret 95\% confidence intervals, 90\% confidence intervals, and 99\% confidence intervals for the population mean in each corridor.
		At the end, determine how many additional observations are required in each sample group to reach a margin of error target of $E = 2.8$ USD for 95\% confidence, 90\% confidence, and 99\% confidence.
		
		\subsection*{C1}
		\textbf{Sample A:}
		\[
		\sum f x = 4(38) + 5(45) + 4(52) + 3(60) + 4(70) = 152 + 225 + 208 + 180 + 280 = 1045
		\]
		\[
		\bar{x}_A = \frac{1045}{20} = 52.25
		\]
		\[
		\sum f(x-\bar{x}_A)^2 = 4(38-52.25)^2 + 5(45-52.25)^2 + 4(52-52.25)^2 + 3(60-52.25)^2 + 4(70-52.25)^2
		\]
		\[
		\sum f(x-\bar{x}_A)^2 = 812.25 + 262.81 + 0.25 + 180.19 + 1260.25 = 2515.75
		\]
		\[
		s_A^2 = \frac{2515.75}{20-1} \approx 132.41, \qquad s_A = \sqrt{132.41} \approx 11.51
		\]
		
		\textbf{Sample B:}
		\[
		\sum f x = 5(38) + 4(45) + 6(52) + 5(60) + 4(70) = 190 + 180 + 312 + 300 + 280 = 1262
		\]
		\[
		\bar{x}_B = \frac{1262}{24} \approx 52.58
		\]
		\[
		\sum f(x-\bar{x}_B)^2 = 5(38-52.58)^2 + 4(45-52.58)^2 + 6(52-52.58)^2 + 5(60-52.58)^2 + 4(70-52.58)^2
		\]
		\[
		\sum f(x-\bar{x}_B)^2 \approx 1063.37 + 230.03 + 2.04 + 275.03 + 1213.36 = 2783.83
		\]
		\[
		s_B^2 = \frac{2783.83}{24-1} \approx 121.04, \qquad s_B = \sqrt{121.04} \approx 11.00
		\]
		
		\subsection*{C5}
		Use the known population standard deviation to build confidence intervals for each sample at three different confidence levels.
		
		First confidence interval uses 95\% confidence with $z^*_{95} = 1.96$.
		\[
		SE_A = \frac{\sigma}{\sqrt{20}} \approx 2.57, \qquad E_{A,95} = 1.96(2.57) \approx 5.04
		\]
		\[
		\mu \in \bar{x}_A \pm E_{A,95} \Rightarrow 52.25 \pm 5.04 = [47.21,\ 57.29]
		\]
		\[
		SE_B = \frac{\sigma}{\sqrt{24}} \approx 2.35, \qquad E_{B,95} = 1.96(2.35) \approx 4.60
		\]
		\[
		\mu \in \bar{x}_B \pm E_{B,95} \Rightarrow 52.58 \pm 4.60 = [47.98,\ 57.18]
		\]
		
		Second confidence interval uses 90\% confidence with $z^*_{90} = 1.645$.
		\[
		E_{A,90} = 1.645(2.57) \approx 4.23
		\]
		\[
		\mu \in \bar{x}_A \pm E_{A,90} \Rightarrow 52.25 \pm 4.23 = [48.02,\ 56.48]
		\]
		\[
		E_{B,90} = 1.645(2.35) \approx 3.86
		\]
		\[
		\mu \in \bar{x}_B \pm E_{B,90} \Rightarrow 52.58 \pm 3.86 = [48.72,\ 56.44]
		\]
		
		Third confidence interval uses 99\% confidence with $z^*_{99} = 2.576$.
		\[
		E_{A,99} = 2.576(2.57) \approx 6.62
		\]
		\[
		\mu \in \bar{x}_A \pm E_{A,99} \Rightarrow 52.25 \pm 6.62 = [45.63,\ 58.87]
		\]
		\[
		E_{B,99} = 2.576(2.35) \approx 6.05
		\]
		\[
		\mu \in \bar{x}_B \pm E_{B,99} \Rightarrow 52.58 \pm 6.05 = [46.53,\ 58.63]
		\]
		
		Interpretation.
		For each corridor, the 90\% confidence interval is narrowest, the 95\% interval is wider, and the 99\% interval is widest.
		All intervals overlap and include the classroom value $\mu = 53$ USD, so the expanded evidence remains consistent with similar mean fares across the two corridors.
		
		\subsection*{C6}
		This subsection establishes the required sample size so the margin of error does not exceed the target amount.
		
		For 95\% confidence, the target margin of error is $E = 2.8$ USD and $z^*_{95} = 1.96$.
		\[
		n_{\text{required,95}} = \left(\frac{z^*_{95}\,\sigma}{E}\right)^2
		= \left(\frac{1.96 \cdot 11.5}{2.8}\right)^2
		= \left(8.05\right)^2
		\approx 64.78
		\]
		Round up to $n_{\text{required,95}} = 65$.
		Additional observations needed.
		Sample A needs $65 - 20 = 45$ more riders, and Sample B needs $65 - 24 = 41$ more riders.
		
		For 90\% confidence, the target margin of error is $E = 2.8$ USD and $z^*_{90} = 1.645$.
		\[
		n_{\text{required,90}} = \left(\frac{z^*_{90}\,\sigma}{E}\right)^2
		= \left(\frac{1.645 \cdot 11.5}{2.8}\right)^2
		\approx 45.66
		\]
		Round up to $n_{\text{required,90}} = 46$.
		Additional observations needed.
		Sample A needs $46 - 20 = 26$ more riders, and Sample B needs $46 - 24 = 22$ more riders.
		
		For 99\% confidence, the target margin of error is $E = 2.8$ USD and $z^*_{99} = 2.576$.
		\[
		n_{\text{required,99}} = \left(\frac{z^*_{99}\,\sigma}{E}\right)^2
		= \left(\frac{2.576 \cdot 11.5}{2.8}\right)^2
		\approx 111.91
		\]
		Round up to $n_{\text{required,99}} = 112$.
		Additional observations needed.
		Sample A needs $112 - 20 = 92$ more riders, and Sample B needs $112 - 24 = 88$ more riders.
		
		Conclusion.
		Higher confidence requires larger total sample size for the same margin-of-error target.
		For this fare study, a 99\% target is the most demanding design.

		\newpage
		\item
		\subsection*{Problem description}
		A university dining analytics team compares average lunch checkout times (minutes) at two larger kiosks during peak hours.
		The two samples are distinct random samples with sizes $n_1 = 8$ and $n_2 = 10$ drawn from the same population of checkout times.
		
		\textbf{Sample 1:} 17, 19, 21, 22, 20, 18, 24, 23.
		
		\textbf{Sample 2:} 16, 18, 20, 21, 22, 24, 19, 23, 25, 20.
		
		Operational logs indicate the population standard deviation is $\sigma = 4.2$ minutes.
		Construct and interpret 95\% confidence intervals, 90\% confidence intervals, and 99\% confidence intervals for the population mean checkout time at each kiosk.
		At the end, determine how many additional observations are required in each kiosk sample to reach a margin of error target of $E = 1.6$ minutes for 95\%, 90\%, and 99\% confidence.
		
		\subsection*{C1}
		First, compute the sample means.
		\[
		\bar{x}_1 = \frac{17+19+21+22+20+18+24+23}{8} = \frac{164}{8} = 20.50
		\]
		\[
		\bar{x}_2 = \frac{16+18+20+21+22+24+19+23+25+20}{10} = \frac{208}{10} = 20.80
		\]
		Second, compute the sample variances and sample standard deviations.
		\[
		s_1^2 = \frac{(17-20.5)^2+(19-20.5)^2+(21-20.5)^2+(22-20.5)^2+(20-20.5)^2+(18-20.5)^2+(24-20.5)^2+(23-20.5)^2}{8-1}
		\]
		\[
		s_1^2 = \frac{12.25+2.25+0.25+2.25+0.25+6.25+12.25+6.25}{7} = \frac{42}{7} = 6.00,
		\qquad s_1 = \sqrt{6.00} \approx 2.45
		\]
		\[
		s_2^2 = \frac{(16-20.8)^2+(18-20.8)^2+(20-20.8)^2+(21-20.8)^2+(22-20.8)^2+(24-20.8)^2+(19-20.8)^2+(23-20.8)^2+(25-20.8)^2+(20-20.8)^2}{10-1}
		\]
		\[
		s_2^2 = \frac{23.04+7.84+0.64+0.04+1.44+10.24+3.24+4.84+17.64+0.64}{9}
		= \frac{69.60}{9} \approx 7.73,
		\qquad s_2 = \sqrt{7.73} \approx 2.78
		\]
		
		\subsection*{C5}
		Use the known population standard deviation to build confidence intervals at three different confidence levels.
		
		First confidence interval uses 95\% confidence.
		With known $\sigma = 4.2$ and 95\% confidence, $z_{0.025}=1.96$.
		\[
		SE_1 = \frac{4.2}{\sqrt{8}} \approx 1.48, \qquad E_{1,95} = 1.96(1.48) \approx 2.91
		\]
		\[
		\mu \in \bar{x}_1 \pm E_{1,95} \Rightarrow 20.50 \pm 2.91 = [17.59,\ 23.41]
		\]
		\[
		SE_2 = \frac{4.2}{\sqrt{10}} \approx 1.33, \qquad E_{2,95} = 1.96(1.33) \approx 2.60
		\]
		\[
		\mu \in \bar{x}_2 \pm E_{2,95} \Rightarrow 20.80 \pm 2.60 = [18.20,\ 23.40]
		\]
		
		Second confidence interval uses 90\% confidence.
		With known $\sigma = 4.2$ and 90\% confidence, $z_{0.05}=1.645$.
		\[
		E_{1,90} = 1.645(1.48) \approx 2.44
		\]
		\[
		\mu \in \bar{x}_1 \pm E_{1,90} \Rightarrow 20.50 \pm 2.44 = [18.06,\ 22.94]
		\]
		\[
		E_{2,90} = 1.645(1.33) \approx 2.18
		\]
		\[
		\mu \in \bar{x}_2 \pm E_{2,90} \Rightarrow 20.80 \pm 2.18 = [18.62,\ 22.98]
		\]
		
		Third confidence interval uses 99\% confidence.
		With known $\sigma = 4.2$ and 99\% confidence, $z_{0.005}=2.576$.
		\[
		E_{1,99} = 2.576(1.48) \approx 3.82
		\]
		\[
		\mu \in \bar{x}_1 \pm E_{1,99} \Rightarrow 20.50 \pm 3.82 = [16.68,\ 24.32]
		\]
		\[
		E_{2,99} = 2.576(1.33) \approx 3.42
		\]
		\[
		\mu \in \bar{x}_2 \pm E_{2,99} \Rightarrow 20.80 \pm 3.42 = [17.38,\ 24.22]
		\]
		
		Interpretation.
		The confidence intervals for both kiosks overlap substantially at all three confidence levels.
		As expected, increasing confidence from 90\% to 95\% to 99\% increases the margin of error and widens each interval.
		
		\subsection*{C6}
		This subsection establishes the required sample size so the margin of error does not exceed the target amount.
		
		For 95\% confidence, the target margin of error is $E = 1.6$ minutes and $z^*_{95} = 1.96$.
		\[
		n_{\text{required,95}} = \left(\frac{z^*_{95}\,\sigma}{E}\right)^2
		= \left(\frac{1.96 \cdot 4.2}{1.6}\right)^2
		= \left(5.15\right)^2
		\approx 26.47
		\]
		Round up to $n_{\text{required,95}} = 27$.
		Additional observations needed.
		Kiosk 1 needs $27 - 8 = 19$ more observations, and Kiosk 2 needs $27 - 10 = 17$ more observations.
		
		For 90\% confidence, the target margin of error is $E = 1.6$ minutes and $z^*_{90} = 1.645$.
		\[
		n_{\text{required,90}} = \left(\frac{z^*_{90}\,\sigma}{E}\right)^2
		= \left(\frac{1.645 \cdot 4.2}{1.6}\right)^2
		\approx 18.66
		\]
		Round up to $n_{\text{required,90}} = 19$.
		Additional observations needed.
		Kiosk 1 needs $19 - 8 = 11$ more observations, and Kiosk 2 needs $19 - 10 = 9$ more observations.
		
		For 99\% confidence, the target margin of error is $E = 1.6$ minutes and $z^*_{99} = 2.576$.
		\[
		n_{\text{required,99}} = \left(\frac{z^*_{99}\,\sigma}{E}\right)^2
		= \left(\frac{2.576 \cdot 4.2}{1.6}\right)^2
		\approx 45.73
		\]
		Round up to $n_{\text{required,99}} = 46$.
		Additional observations needed.
		Kiosk 1 needs $46 - 8 = 38$ more observations, and Kiosk 2 needs $46 - 10 = 36$ more observations.
		
		Conclusion.
		For a fixed precision target, the required sample size increases as the confidence level increases.

		\newpage
		\item
		\subsection*{Problem description}
		A national public finance office studies the proportion of quarterly utility reimbursements that are flagged for expedited approval in two expanded district networks.
		The two samples are distinct random samples with different sizes drawn from the same population of reimbursements.
		
		\textbf{Sample A (120 reimbursements):}
		In this sample, $x_A = 74$ reimbursements are flagged for expedited approval.
		
		\textbf{Sample B (95 reimbursements):}
		In this sample, $x_B = 52$ reimbursements are flagged for expedited approval.
		
		Construct 95\%, 90\%, and 99\% confidence intervals for the population proportion of expedited reimbursements using each sample.
		At the end, determine how many additional observations are required in each sample group to reach margin-of-error targets of $E = 0.04$ and $E = 0.03$ for 95\% confidence, using each sample estimated proportion.
		
		\subsection*{C1}
		The estimator used in this problem is the sample proportion.
		In mean-based notation, $\bar{x}$ corresponds to the proportion estimator $\hat{p}$.
		The variability of $\hat{p}$ is based on the Bernoulli variance.
		
		For a population proportion $p$ and sample size $n$,
		\[
		\mathrm{Var}(\hat{p}) = \frac{p(1-p)}{n}
		\qquad
		\widehat{\mathrm{Var}}(\hat{p}) = \frac{\hat{p}(1-\hat{p})}{n}
		\]
		and the standard error of the estimator is
		\[
		\mathrm{SE}(\hat{p}) = \sqrt{\frac{\hat{p}(1-\hat{p})}{n}}.
		\]
		
		\textbf{Sample A:}
		Step 1.
		Compute the sample proportion.
		\[
		\hat{p}_A = \frac{x_A}{n_A} = \frac{74}{120} \approx 0.6167
		\]
		Step 2.
		Compute the estimated variance of $\hat{p}_A$.
		\[
		\widehat{\mathrm{Var}}(\hat{p}_A) = \frac{\hat{p}_A(1-\hat{p}_A)}{n_A}
		= \frac{0.6167(1-0.6167)}{120}
		= \frac{0.6167(0.3833)}{120}
		\approx \frac{0.2364}{120}
		\approx 0.00197
		\]
		Step 3.
		Compute the standard error of $\hat{p}_A$.
		\[
		\mathrm{SE}(\hat{p}_A) = \sqrt{\widehat{\mathrm{Var}}(\hat{p}_A)}
		= \sqrt{0.00197}
		\approx 0.0444
		\]
		
		\textbf{Sample B:}
		Step 1.
		Compute the sample proportion.
		\[
		\hat{p}_B = \frac{x_B}{n_B} = \frac{52}{95} \approx 0.5474
		\]
		Step 2.
		Compute the estimated variance of $\hat{p}_B$.
		\[
		\widehat{\mathrm{Var}}(\hat{p}_B) = \frac{\hat{p}_B(1-\hat{p}_B)}{n_B}
		= \frac{0.5474(1-0.5474)}{95}
		= \frac{0.5474(0.4526)}{95}
		\approx \frac{0.2478}{95}
		\approx 0.00261
		\]
		Step 3.
		Compute the standard error of $\hat{p}_B$.
		\[
		\mathrm{SE}(\hat{p}_B) = \sqrt{\widehat{\mathrm{Var}}(\hat{p}_B)}
		= \sqrt{0.00261}
		\approx 0.0511
		\]
		
		\subsection*{C5 and C7}
		The parameter of interest is a population proportion.
		The estimator value $\hat{p}$ and its standard error are identified before constructing confidence intervals.
		
		\textbf{Sample A:}
		Step 1.
		Check the large-sample conditions.
		\[
		n_A\hat{p}_A = 120(0.6167) = 74.00 \ge 10
		\qquad
		n_A(1-\hat{p}_A) = 120(0.3833) = 46.00 \ge 10
		\]
		Step 2.
		Compute margins of error for each confidence level.
		\[
		E_{A,95} = 1.96\,\mathrm{SE}(\hat{p}_A) = 1.96(0.0444) \approx 0.0870
		\]
		\[
		E_{A,90} = 1.645\,\mathrm{SE}(\hat{p}_A) = 1.645(0.0444) \approx 0.0730
		\]
		\[
		E_{A,99} = 2.576\,\mathrm{SE}(\hat{p}_A) = 2.576(0.0444) \approx 0.1144
		\]
		Step 3.
		Construct the confidence intervals.
		\[
		p \in \hat{p}_A \pm E_{A,95}
		\Rightarrow 0.6167 \pm 0.0870
		\Rightarrow [0.5297,\ 0.7037]
		\]
		\[
		p \in \hat{p}_A \pm E_{A,90}
		\Rightarrow 0.6167 \pm 0.0730
		\Rightarrow [0.5437,\ 0.6897]
		\]
		\[
		p \in \hat{p}_A \pm E_{A,99}
		\Rightarrow 0.6167 \pm 0.1144
		\Rightarrow [0.5023,\ 0.7311]
		\]
		
		\textbf{Sample B:}
		Step 1.
		Check the large-sample conditions.
		\[
		n_B\hat{p}_B = 95(0.5474) = 52.00 \ge 10
		\qquad
		n_B(1-\hat{p}_B) = 95(0.4526) = 43.00 \ge 10
		\]
		Step 2.
		Compute margins of error for each confidence level.
		\[
		E_{B,95} = 1.96\,\mathrm{SE}(\hat{p}_B) = 1.96(0.0511) \approx 0.1002
		\]
		\[
		E_{B,90} = 1.645\,\mathrm{SE}(\hat{p}_B) = 1.645(0.0511) \approx 0.0841
		\]
		\[
		E_{B,99} = 2.576\,\mathrm{SE}(\hat{p}_B) = 2.576(0.0511) \approx 0.1316
		\]
		Step 3.
		Construct the confidence intervals.
		\[
		p \in \hat{p}_B \pm E_{B,95}
		\Rightarrow 0.5474 \pm 0.1002
		\Rightarrow [0.4472,\ 0.6476]
		\]
		\[
		p \in \hat{p}_B \pm E_{B,90}
		\Rightarrow 0.5474 \pm 0.0841
		\Rightarrow [0.4633,\ 0.6315]
		\]
		\[
		p \in \hat{p}_B \pm E_{B,99}
		\Rightarrow 0.5474 \pm 0.1316
		\Rightarrow [0.4158,\ 0.6790]
		\]
		
		Interpretation.
		For both district networks, higher confidence produces wider intervals.
		The intervals overlap across networks, and all intervals support expedited-approval proportions near one-half to two-thirds.
		
		\subsection*{C6}
		This subsection establishes the required sample size so the margin of error for the proportion does not exceed the target amount.
		
		Step 1.
		Use the confidence level and each target error value.
		For 95\% confidence, $z^* = 1.96$.
		
		Step 2.
		Compute the required sample size using each sample estimated proportion.
		
		\textbf{Target $E = 0.04$ using Sample A estimated proportion:}
		\[
		n_{\text{required},A,0.04} = \frac{(z^*)^2\,\hat{p}_A(1-\hat{p}_A)}{E^2}
		= \frac{(1.96)^2(0.6167)(0.3833)}{(0.04)^2}
		\approx \frac{3.8416\cdot 0.2364}{0.0016}
		\approx \frac{0.9085}{0.0016}
		\approx 567.81
		\]
		Round up to $n_{\text{required},A,0.04} = 568$.
		Additional observations needed: $568 - 120 = 448$.
		
		\textbf{Target $E = 0.04$ using Sample B estimated proportion:}
		\[
		n_{\text{required},B,0.04} = \frac{(z^*)^2\,\hat{p}_B(1-\hat{p}_B)}{E^2}
		= \frac{(1.96)^2(0.5474)(0.4526)}{(0.04)^2}
		\approx \frac{3.8416\cdot 0.2478}{0.0016}
		\approx \frac{0.9519}{0.0016}
		\approx 594.94
		\]
		Round up to $n_{\text{required},B,0.04} = 595$.
		Additional observations needed: $595 - 95 = 500$.
		
		\textbf{Target $E = 0.03$ using Sample A estimated proportion:}
		\[
		n_{\text{required},A,0.03} = \frac{(z^*)^2\,\hat{p}_A(1-\hat{p}_A)}{E^2}
		= \frac{3.8416\cdot 0.2364}{0.0009}
		\approx \frac{0.9085}{0.0009}
		\approx 1009.44
		\]
		Round up to $n_{\text{required},A,0.03} = 1010$.
		Additional observations needed: $1010 - 120 = 890$.
		
		\textbf{Target $E = 0.03$ using Sample B estimated proportion:}
		\[
		n_{\text{required},B,0.03} = \frac{(z^*)^2\,\hat{p}_B(1-\hat{p}_B)}{E^2}
		= \frac{3.8416\cdot 0.2478}{0.0009}
		\approx \frac{0.9519}{0.0009}
		\approx 1057.68
		\]
		Round up to $n_{\text{required},B,0.03} = 1058$.
		Additional observations needed: $1058 - 95 = 963$.
		
		Conclusion.
		A smaller target margin of error requires a much larger sample size.
		Moving from $E = 0.04$ to $E = 0.03$ substantially increases data collection requirements for both district networks.
		
		\newpage
		\item
		\subsection*{Problem description}
		A municipal purchasing authority now monitors invoice amounts across three departments (transport, maintenance, and health procurement) using a unified benchmark system. From long-term records covering all municipal purchases, invoices are known to have a typical average value of 445 USD and a typical spread of 62 USD around that average. These values summarize how invoice amounts behave across the full purchasing process over time.
		
		To assess current activity, two independent audits are conducted using recent invoices from all three departments combined. In Sample A, the invoices reviewed produce a 90\% confidence interval from 418 USD to 452 USD, and the variability within this audit is summarized by a variance of 3025 (USD)$^2$. In Sample B, the invoices reviewed produce a 90\% confidence interval from 432 USD to 476 USD, and the variability within this audit is summarized by a variance of 4096 (USD)$^2$.
		
		In both audits, the authority requires a clear judgment about whether current behavior is centered near the benchmark and whether observed spread appears operationally aligned with long-run variation.
		
		\subsection*{C2}
		The value 445 USD represents the typical invoice amount across all municipal purchases, and 62 USD represents the typical spread of invoice amounts over time.
		
		For Sample A, the observed average invoice amount is obtained as the midpoint of the confidence interval:
		\[
		\frac{418 + 452}{2} = 435 \text{ USD}.
		\]
		For Sample B, the observed average invoice amount is obtained as the midpoint of its confidence interval:
		\[
		\frac{432 + 476}{2} = 454 \text{ USD}.
		\]
		These observed averages are compared directly to the benchmark average of 445 USD.
		
		The variability within each audit is summarized by variances of 3025 (USD)$^2$ for Sample A and 4096 (USD)$^2$ for Sample B. Taking square roots gives observed spreads of
		\[
		\sqrt{3025} = 55 \text{ USD} \quad \text{and} \quad \sqrt{4096} = 64 \text{ USD}.
		\]
		These observed spreads are then compared with the long-run spread of 62 USD.
		
		\subsection*{C3}
		Point estimation approximates a characteristic of invoice behavior with a single value computed from audit data. In this problem, the point estimates for the typical invoice amount are 435 USD for Sample A and 454 USD for Sample B. Interval estimation instead provides ranges, namely [418, 452] and [432, 476], which explicitly include sampling uncertainty. In the expanded multi-department context, interval estimates are more informative because they show whether operational differences between departments can still produce benchmark-compatible ranges instead of relying only on one reported average.
		
		\subsection*{C4}
		For Sample A, the 90\% confidence interval from 418 USD to 452 USD means invoice averages outside this range are not consistent with the observed audit data at the chosen confidence level. For Sample B, the interval from 432 USD to 476 USD gives the corresponding plausible range.
		
		Because the benchmark value 445 USD lies inside both intervals, both audits remain compatible with long-run invoice behavior, although Sample B is shifted upward relative to Sample A. Also, the observed spread in Sample A (55 USD) is lower than the benchmark spread, while Sample B (64 USD) is slightly higher. This richer interpretation suggests the purchasing authority should continue monitoring department mix and contract timing, since central tendency remains benchmark-consistent but variability patterns differ across the two audits.
		
		\newpage
		\item
		\subsection*{Problem description}
		An agricultural finance board compares average loan sizes issued by two cooperative systems: rural cooperatives and urban cooperatives, across an expanded national program that includes seasonal credit lines and emergency refinancing windows. From long-term financial records covering all loans issued in each system, rural cooperatives are known to have a typical loan size of 318 USD with a usual spread of 46 USD, while urban cooperatives are known to have a typical loan size of 362 USD with a usual spread of 58 USD.
		
		To assess current lending activity, analysts examine recent loans from each system under the expanded program. The rural sample produces a 95\% confidence interval from 300 USD to 334 USD, and the urban sample produces a 95\% confidence interval from 345 USD to 383 USD.
		
		The board must assess not only whether current means differ, but also whether the degree of uncertainty still supports a stable policy distinction between rural and urban systems.
		
		\subsection*{C2}
		The values 318 USD and 46 USD describe long-run behavior of loan sizes in rural cooperatives, while 362 USD and 58 USD describe long-run behavior in urban cooperatives.
		
		For the rural sample, the observed average loan size is the midpoint of its confidence interval:
		\[
		\frac{300 + 334}{2} = 317 \text{ USD}.
		\]
		For the urban sample, the observed average loan size is the midpoint of its confidence interval:
		\[
		\frac{345 + 383}{2} = 364 \text{ USD}.
		\]
		These observed averages are directly compared with long-run values 318 USD and 362 USD.
		
		\subsection*{C3}
		Point estimation summarizes each system with one number, here 317 USD for rural and 364 USD for urban cooperatives. Interval estimation uses [300, 334] and [345, 383] to represent plausible ranges after accounting for sampling variation.
		
		In the expanded program context, interval estimation is preferred because policy decisions now combine normal lending with seasonal and emergency loans. A single point estimate could hide this additional uncertainty, while intervals show precision and allow clearer risk-aware comparison between cooperative systems.
		
		\subsection*{C4}
		For rural cooperatives, values below 300 USD or above 334 USD are not consistent with the observed rural data at 95\% confidence. For urban cooperatives, values below 345 USD or above 383 USD are not consistent with observed urban data.
		
		The intervals do not overlap, so the current evidence still supports a higher typical loan size in urban cooperatives even under the broader national program. In addition, each system's long-run benchmark (318 USD for rural and 362 USD for urban) lies within its corresponding interval, which suggests the expanded operating conditions have not fundamentally shifted the underlying center of each lending distribution.
		
		\newpage
		\item
		\subsection*{Problem description}
		A financial regulator studies the share of households using a mobile savings platform under a new nationwide rollout that now includes urban districts, peri-urban zones, and remote rural communities. From prior nationwide records, the regulator has a benchmark indicating that about 57\% of households use the platform, with corresponding long-run variability of approximately 0.2451.
		
		To assess current usage under the expanded rollout, two large survey waves are combined, and analysts report a 90\% confidence interval for the share of households using the platform from 0.52 to 0.62.
		
		Explain which values describe long-run adoption behavior versus those obtained from survey data, distinguish between single-value and range-based estimation for proportions, and interpret the reported interval in context while accounting for the broader rollout setting.
		
		\subsection*{C2}
		The value 0.57 represents the long-run benchmark share of households using the platform across the full population, and 0.2451 represents long-run variability in usage.
		
		The observed share in the current survey evidence is obtained as the midpoint of the confidence interval:
		\[
		\frac{0.52 + 0.62}{2} = 0.57.
		\]
		This observed value can be directly compared to the benchmark 0.57.
		
		The variability implied by the current survey share is
		\[
		0.57(1-0.57) = 0.57(0.43) = 0.2451.
		\]
		Taking square roots gives an observed spread of approximately
		\[
		\sqrt{0.2451} \approx 0.50,
		\]
		which is consistent with the benchmark spread.
		
		\subsection*{C3}
		Point estimation gives a single value for the adoption share, here 0.57. Interval estimation gives a range, here [0.52, 0.62], to account for uncertainty from surveying only part of the population.
		
		Under the expanded nationwide rollout, interval estimation is especially useful because adoption conditions vary across urban, peri-urban, and remote areas. The interval communicates precision and acknowledges that observed adoption can fluctuate across subregions even when the national point estimate equals the benchmark.
		
		\subsection*{C4}
		In this context, the confidence interval from 0.52 to 0.62 means adoption rates below 52\% or above 62\% are not consistent with the observed combined survey data at the chosen confidence level.
		
		Because the benchmark value 0.57 lies inside the interval, the current evidence is compatible with prior national behavior. At the same time, the width of the interval indicates moderate uncertainty that is reasonable for a broadened rollout involving diverse regions. Thus, the regulator can treat the rollout as broadly on target while continuing regional monitoring to detect localized deviations.
		
	\end{ExamProblems}
\end{document}
