\documentclass[12pt]{article}

% Page size and tighter margins
\usepackage[a4paper,left=1.2cm,right=1.2cm,top=1.5cm,bottom=1.5cm]{geometry}

% Core packages
\usepackage{graphicx}
\usepackage{xcolor}
\usepackage{array}
\usepackage{tabularx}
\usepackage{multicol}
\usepackage[T1]{fontenc}
\usepackage[utf8]{inputenc}

\setlength{\parindent}{0pt}
\setlength{\tabcolsep}{6pt}
\renewcommand{\arraystretch}{1.15}

% Column types
\newcolumntype{Y}{>{\raggedright\arraybackslash}m{\dimexpr0.30\textwidth-2\tabcolsep-2\arrayrulewidth\relax}}
\newcolumntype{Z}{>{\raggedright\arraybackslash}m{\dimexpr0.70\textwidth-2\tabcolsep-2\arrayrulewidth\relax}}
\newcolumntype{C}[1]{>{\centering\arraybackslash}m{#1}}

% Gray subsection header box
\newcommand{\SubsectionBox}[1]{%
	\noindent\colorbox{gray!30}{%
		\parbox{\linewidth}{\textbf{#1}}%
	}\par\vspace{0.35cm}%
}

% Centered multi-line cell helper
\newcommand{\CellCenter}[1]{%
	\parbox{\linewidth}{\centering #1}%
}

\begin{document}

	% =========================
	% HEADER BOX (3 COLUMNS)
	% =========================
	\noindent
	\begin{tabularx}{\textwidth}{|C{2.8cm}|C{\dimexpr\textwidth-6cm-4\tabcolsep-4\arrayrulewidth\relax}|C{2.8cm}|}
		\hline
		\centering
		\vspace{3mm}
		\includegraphics[width=2.5cm]{../../preamble/logo.png}
		&
		\CellCenter{%
			\vspace{-5mm}
			\textbf{GLOBAL ECONOMICS}\par
			\textbf{GRADE: 11TH}\par
			\textbf{LEARNING EVIDENCE 1}\par
			\textbf{CONFIDENCE INTERVAL PLANNING}\par
			\textbf{TEACHER'S NAME: Nicolás López Cuéllar}
		}
		&
		\CellCenter{%
			\textbf{SECOND TERM}\par
			\textbf{2025--2026}%
		}
		\\
		\hline
	\end{tabularx}

	\vspace{0.5cm}

	% =========================
	% OBJECTIVE + CRITERIA
	% =========================
	\noindent
	\begin{tabular}{|Y|Z|}
		\hline
		{\small
			\textbf{Learning objective:} Compute and interpret confidence intervals for population means using known population standard deviation.
		}
		&
		{\footnotesize
			\textbf{Assessment criteria:}\par
			C1: Compute the sample mean and sample standard deviation.\par
			C2: Distinguish between sample statistics and population parameters.\par
			C3: Explain why interval estimation is preferred over a single point estimate.\par
			C5: Construct a X\% confidence interval using known population standard deviation.\par
			C4: Interpret the meaning of a X\% confidence interval in context.\par
		}
		\\
		\hline
	\end{tabular}

	\vspace{0.4cm}

	% =========================
	% STUDENT LINE
	% =========================
	\noindent

	% =========================
	% EXAM BODY
	% =========================
	\begin{multicols}{2}
		\SubsectionBox{Criteria assessment}\vspace{-0.25cm}
		Each assessment criterion is evaluated across the seven problems. A criterion is considered passed when it is correctly activated in at least six of the seven problems.

		\vspace{0.25cm}
		\SubsectionBox{1. Fruit servings sold per day}\vspace{-0.25cm}
		A school cafeteria wants to estimate the average number of fruit servings sold per day.
		A random sample of 8 days is recorded as raw data (servings per day): 12, 15, 14, 16, 13, 17, 15, 14.
		The cafeteria has a historical estimate of the population standard deviation, \(\sigma = 2.4\) servings.
		For classroom purposes, suppose the true population mean is \(\mu = 15\) servings.
		Construct and interpret a 90\% confidence interval, a 95\% confidence interval, and a 99\% confidence interval for the population mean number of servings sold per day.

		\vspace{0.25cm}
		\SubsectionBox{2. Study rooms booked per day}\vspace{-0.25cm}
		A city library wants to estimate the average number of study rooms booked per day.
		Two different random samples are drawn from the same population of daily bookings.
		The first sample (Sample A) records 18, 20, 19, 21, 17, 22 bookings and the second sample (Sample B) records 16, 18, 17, 19, 20 bookings.
		The population standard deviation is known from long-term records to be \(\sigma = 3.2\) bookings.
		For classroom purposes, suppose the true population mean is \(\mu = 19\) bookings.
		Construct and interpret a 95\% confidence interval for the population mean based on each sample.

		\vspace{0.25cm}
		\SubsectionBox{3. Completed modules per week}\vspace{-0.25cm}
		A training center tracks the number of completed modules per week.
		A random sample of 30 weeks is summarized in grouped form. The population standard deviation is known to be \(\sigma = 4.5\) modules per week.
		For academic purposes, assume the true population mean is \(\mu = 14\) modules per week.
		The grouped data pairs list modules per week first and the frequency (number of weeks) second: \(9 \rightarrow 6\), \(12 \rightarrow 10\), \(15 \rightarrow 8\), and \(18 \rightarrow 6\).
		Construct and interpret a 90\% confidence interval, a 95\% confidence interval, and a 99\% confidence interval for the population mean modules per week.

		\vspace{0.25cm}
		\SubsectionBox{4. Packages processed per shift}\vspace{-0.25cm}
		A delivery company wants to estimate the average number of packages processed per shift.
		Two different random samples are drawn from the same population of processing counts.
		Sample A is grouped as \(14 \rightarrow 7\), \(18 \rightarrow 9\), \(22 \rightarrow 4\), and Sample B is grouped as \(14 \rightarrow 5\), \(18 \rightarrow 8\), \(22 \rightarrow 7\).
		The population standard deviation is known to be \(\sigma = 3.6\) packages.
		For classroom purposes, suppose the true population mean is \(\mu = 18\) packages.
		Construct and interpret a 95\% confidence interval for the population mean based on each sample.

		\vspace{0.25cm}
		\SubsectionBox{5. Class session attendance}\vspace{-0.25cm}
		A fitness center records the number of active members who attend a class session.
		A random sample of 50 sessions is grouped into class intervals. The population standard deviation is known to be \(\sigma = 5.2\) attendees.
		For academic purposes, assume the true population mean is \(\mu = 49\) attendees.
		The grouped intervals and frequencies are 40--44 (12), 45--49 (18), 50--54 (14), and 55--59 (6).
		Construct and interpret a 90\% confidence interval, a 95\% confidence interval, and a 99\% confidence interval for the population mean number of attendees.

		\vspace{0.25cm}
		\SubsectionBox{6. Call length in minutes}\vspace{-0.25cm}
		A customer service center wants to estimate the average call length in minutes.
		Two different random samples are drawn from the same population of call lengths and summarized in class intervals.
		Sample A has intervals 28--32 (10), 33--37 (15), and 38--42 (9).
		Sample B has intervals 28--32 (8), 33--37 (14), and 38--42 (12).
		The population standard deviation is known to be \(\sigma = 4.1\) minutes.
		For classroom purposes, suppose the true population mean is \(\mu = 35\) minutes.
		Construct and interpret a 95\% confidence interval for the population mean based on each sample.

		\vspace{0.25cm}
		\SubsectionBox{7. Package weight monitoring}\vspace{-0.25cm}
		A manufacturing plant monitors the weight (in grams) of packaged items.
		Most of the data are summarized in grouped values, but a few extreme packages were recorded separately as outliers.
		The grouped data are 50 $\rightarrow$ 8, 55 $\rightarrow$ 12, and 60 $\rightarrow$ 10.
		The outlier raw observations are 72, 74, and 45 grams.
		The population standard deviation is known to be \(\sigma = 6.0\) grams.
		For classroom purposes, suppose the true population mean is \(\mu = 56\) grams.
		Construct and interpret a 95\% confidence interval for the population mean package weight.

	\end{multicols}

\end{document}
