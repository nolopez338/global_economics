\documentclass[12pt]{article}
\usepackage[a4paper,left=1.2cm,right=1.2cm,top=1.5cm,bottom=1.5cm]{geometry}
\usepackage{graphicx}
\usepackage{xcolor}
\usepackage{array}
\usepackage{tabularx}
\usepackage{multicol}
\usepackage[T1]{fontenc}
\usepackage[utf8]{inputenc}
\setlength{\parindent}{0pt}
\setlength{\tabcolsep}{6pt}
\renewcommand{\arraystretch}{1.15}
\newcolumntype{Y}{>{\raggedright\arraybackslash}m{\dimexpr0.55\textwidth-2\tabcolsep-2\arrayrulewidth\relax}}
\newcolumntype{Z}{>{\raggedright\arraybackslash}m{\dimexpr0.45\textwidth-2\tabcolsep-2\arrayrulewidth\relax}}
\newcolumntype{C}[1]{>{\centering\arraybackslash}m{#1}}
\newcommand{\SubsectionBox}[1]{%
\noindent\colorbox{gray!30}{\parbox{\linewidth}{\textbf{#1}}}\par\vspace{0.35cm}%
}
\newcommand{\CellCenter}[1]{\parbox{\linewidth}{\centering #1}}

\begin{document}

\noindent
\begin{tabularx}{\textwidth}{|C{2.8cm}|C{\dimexpr\textwidth-6cm-4\tabcolsep-4\arrayrulewidth\relax}|C{2.8cm}|}
\hline
\centering\vspace{3mm}\includegraphics[width=2.5cm]{../../preamble/logo.png}&
\CellCenter{\vspace{-5mm}\textbf{GLOBAL ECONOMICS}\par\textbf{GRADE: 11TH}\par\textbf{CATCH-UP}\par\textbf{CRITERION C5}\par\textbf{TEACHER'S NAME: Nicolás López Cuéllar}}&
\CellCenter{\textbf{SECOND TERM}\par\textbf{2025--2026}}\\
\hline
\end{tabularx}

\vspace{0.5cm}
\noindent
\begin{tabular}{|Y|Z|}
\hline
{\footnotesize\textbf{Learning objective:} Construct confidence intervals for a population mean when long-run spread is known.}&
{\footnotesize\textbf{Assessment criteria:}\par C5: Construct a X\% confidence interval using known population standard deviation.}\\
\hline
\end{tabular}

\begin{multicols}{2}
	\SubsectionBox{Criteria assessment}\vspace{-0.25cm}
	Each assessment criterion is evaluated across the problems in this catch-up exam. A criterion is considered passed when it is correctly activated in 9 of the 10 problems of this activity.

	\vspace{0.25cm}
\SubsectionBox{Problem 1}
\subsection*{One-Bedroom Rent Confidence Interval}
A rental platform reviewed 36 recent one-bedroom listings aimed at young professionals.
		The average listed monthly rent in that review was 1240 USD.
		From several years of records, the long-run spread is known as $\sigma=180$ USD.
		Construct a 95\% confidence interval for the average monthly rent.

\vspace{0.5cm}
\SubsectionBox{Problem 2}
\subsection*{Rider Earnings Confidence Interval}
A delivery app checked 64 weekly earning records from active riders.
		The average from those 64 records was 540 USD per week.
		Across many years on the platform, the typical spread around that level is 96 USD.
		Build estimated intervals for the overall average weekly earning at 90\%, 95\%, and 99\%.

\vspace{0.5cm}
\SubsectionBox{Problem 3}
\subsection*{Store Daily Sales Interval}
An online marketplace examined 49 daily sales totals from small student-run stores.
		The average daily total in this group was 860 USD.
		From long historical tracking, the usual spread in daily totals is 140 USD.
		Find intervals for the overall average daily total using confidence probabilities of 92\%, 97\%, and 99\%.

\vspace{0.5cm}
\SubsectionBox{Problem 4}
\subsection*{Freelance Project Payment Interval}
A freelance payment platform inspected 25 completed projects from the last month.
		The average payment in that group was 680 USD.
		Across many years, the long-run spread for project payments is 110 USD.
		The platform also reports a long-term overall average of 700 USD and says the recent group had its own spread of 95 USD.
		Construct 90\% and 95\% confidence intervals for the overall average payment.

\vspace{0.5cm}
\SubsectionBox{Problem 5}
\subsection*{Subscription Spending Precision Check}
A subscription management app reviewed 100 monthly spending records from users in their 20s.
		The average monthly amount was 84 USD.
		The known long-run spread from historical audits is 30 USD.
		Additional details showed a long-term average of 82 USD and a recent-group spread of 27 USD.
		Construct 95\% and 99\% confidence intervals for the overall average monthly subscription spending.

\vspace{0.5cm}
\SubsectionBox{Problem 6}
\subsection*{Driver Fuel Reimbursement Interval}
A ride-share company sampled 64 weekly fuel reimbursements paid to drivers.
		The average reimbursement was 126 USD.
		Long-run variability from company records is given as 256 USD$^2$.
		The median in this sample was 124 USD.
		Construct a 95\% confidence interval for the overall average weekly reimbursement.

\vspace{0.5cm}
\SubsectionBox{Problem 7}
\subsection*{Beginner Portfolio Result Interval}
An investment learning club recorded 36 monthly portfolio results in USD for beginner members.
		The average result was 410 USD.
		Long-run variability from prior cohorts is 441 USD$^2$.
		In the same month, the highest value was 520 USD and the lowest was 300 USD.
		Construct a 90\% confidence interval for the overall average monthly result.

\vspace{0.5cm}
\SubsectionBox{Problem 8}
\subsection*{Online Order Value Interval}
A small online store reviewed 49 daily order values from one product line.
		The average order value in the review was 72 USD.
		Across previous years, the known long-run spread is 14 USD.
		Other dashboard details were: total revenue that period was 3528 USD, minimum order was 39 USD, and maximum order was 118 USD.
		Construct a 95\% confidence interval for the overall average order value.

\vspace{0.5cm}
\SubsectionBox{Problem 9}
\subsection*{Weekend Trip Cost Interval}
A travel budgeting service examined 25 recent weekend trip costs for young adults.
		The observed average cost was 465 USD.
		From long-run data, the known spread in similar trip costs is 85 USD.
		The report also included a previous year's average of 452 USD, a median of 458 USD, and noted there are 52 weeks in a year.
		Construct a 99\% confidence interval for the overall average weekend trip cost.

\vspace{0.5cm}
\SubsectionBox{Problem 10}
\subsection*{Desk Rental Payment Interval}
A co-working platform reviewed 100 monthly desk-rental payments from freelancers.
		The average payment in that review was 312 USD.
		From long historical records, the known spread is 40 USD.
		Additional details included: total payments were 31{,}200 USD, lowest payment was 220 USD, highest payment was 395 USD, and last year's average was 305 USD.
		Construct a 95\% confidence interval for the overall average monthly desk-rental payment.

\end{multicols}

\end{document}
