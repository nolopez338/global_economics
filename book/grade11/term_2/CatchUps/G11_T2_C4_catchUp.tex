\documentclass[12pt]{article}

\usepackage[a4paper,left=1.2cm,right=1.2cm,top=1.5cm,bottom=1.5cm]{geometry}

\usepackage{graphicx}
\usepackage{xcolor}
\usepackage{array}
\usepackage{tabularx}
\usepackage{multicol}
\usepackage{amsmath}
\usepackage[T1]{fontenc}
\usepackage[utf8]{inputenc}

\setlength{\parindent}{0pt}
\setlength{\tabcolsep}{6pt}
\renewcommand{\arraystretch}{1.15}

\newcolumntype{Y}{>{\raggedright\arraybackslash}m{\dimexpr0.55\textwidth-2\tabcolsep-2\arrayrulewidth\relax}}
\newcolumntype{Z}{>{\raggedright\arraybackslash}m{\dimexpr0.45\textwidth-2\tabcolsep-2\arrayrulewidth\relax}}
\newcolumntype{C}[1]{>{\centering\arraybackslash}m{#1}}

\newcommand{\SubsectionBox}[1]{%
	\noindent\colorbox{gray!30}{\parbox{\linewidth}{\textbf{#1}}}\par\vspace{0.35cm}%
}
\newcommand{\CellCenter}[1]{\parbox{\linewidth}{\centering #1}}

\begin{document}
	
	\noindent
	\begin{tabularx}{\textwidth}{|C{2.8cm}|C{\dimexpr\textwidth-6cm-4\tabcolsep-4\arrayrulewidth\relax}|C{2.8cm}|}
		\hline
		\centering\vspace{3mm}\includegraphics[width=2.5cm]{../../preamble/logo.png}&
		\CellCenter{\vspace{-5mm}\textbf{GLOBAL ECONOMICS}\par\textbf{GRADE: 11TH}\par\textbf{CATCH-UP}\par\textbf{CRITERION C4}\par\textbf{TEACHER'S NAME: Nicolás López Cuéllar}}&
		\CellCenter{\textbf{SECOND TERM}\par\textbf{2025--2026}}\\
		\hline
	\end{tabularx}
	
	\vspace{0.5cm}
	\noindent
	\begin{tabular}{|Y|Z|}
		\hline
		{\footnotesize\textbf{Learning objective:} Interpret confidence intervals in context for confidence interval planning decisions.}&
		{\footnotesize\textbf{Assessment criteria:}\par C4: Interpret the meaning of a X\% confidence interval in context.}\\
		\hline
	\end{tabular}
	
	\begin{multicols}{2}
		\SubsectionBox{Criteria assessment}\vspace{-0.25cm}
		Each assessment criterion is evaluated across the problems in this catch-up exam. A criterion is considered passed when it is correctly activated in 9 of the 10 problems of this activity.
		
		\vspace{0.25cm}
		\SubsectionBox{Problem 1}
		\subsection*{Downtown Listing Mean Estimate}
		A housing app samples new one-bedroom listings for early-career workers in a downtown area.
		A 95\% confidence interval for the population mean monthly rent is:
		\[
		\mu \in [1180, 1320]\ \text{USD}.
		\]
		A relocation agency must decide whether to market this area as ``typically below 1200 USD per month.''
		Is that claim supported by the interval?
		
		\vspace{0.5cm}
		\SubsectionBox{Problem 2}
		\subsection*{Designer Income Mean Estimate}
		A freelance platform studies monthly income for beginner graphic designers aged 20--29.
		A 95\% confidence interval for the population mean monthly freelance income is:
		\[
		\mu \in [1720, 2140]\ \text{USD}.
		\]
		A training program advertises that most beginners can expect around 2000 USD per month.
		Is 2000 USD a plausible benchmark for the population mean based on this interval?
		
		\vspace{0.5cm}
		\SubsectionBox{Problem 3}
		\subsection*{Courier Earnings Mean Comparison}
		Two ride-delivery apps estimate the population mean weekly earnings of active couriers.
		App A reports a 95\% confidence interval:
		\[
		\mu_A \in [460, 540]\ \text{USD}.
		\]
		App B reports a 95\% confidence interval:
		\[
		\mu_B \in [520, 610]\ \text{USD}.
		\]
		A student chooses one app and asks: does the evidence suggest App B has a higher population mean weekly earning than App A?
		
		\vspace{0.5cm}
		\SubsectionBox{Problem 4}
		\subsection*{Credit Card Full-Payment Rate}
		A personal finance survey estimates the population proportion of young card users who pay their full credit card balance each month.
		A 95\% confidence interval is:
		\[
		p \in [0.41, 0.53].
		\]
		A policy analyst states: ``There is a 95\% chance that exactly 47\% of users pay in full.''
		Explain what is incorrect in that statement.
		
		\vspace{0.5cm}
		\SubsectionBox{Problem 5}
		\subsection*{Positive Return Proportion Estimate}
		An investing app tracks first-year users and estimates the population proportion who finish a month with positive return.
		A 95\% confidence interval for that population proportion is:
		\[
		p \in [0.57, 0.69].
		\]
		The company considers the statement: ``At least 70\% of users are profitable in a typical month.''
		Is this statement supported by the confidence interval?
		
		\vspace{0.5cm}
		\SubsectionBox{Problem 6}
		\subsection*{Student Loan Repayment Comparison}
		A bank compares average monthly student loan repayment among two groups of recent graduates.
		Group A (public sector jobs) has a 95\% confidence interval:
		\[
		\mu_A \in [240, 290]\ \text{USD}.
		\]
		Group B (private sector jobs) has a 95\% confidence interval:
		\[
		\mu_B \in [305, 360]\ \text{USD}.
		\]
		Should the bank design separate repayment guidance for the two groups?
		
		\vspace{0.5cm}
		\SubsectionBox{Problem 7}
		\subsection*{Employee Cost of Living}
		A startup surveys first-year employees about total monthly cost of living in a major city.
		For the same dataset, analysts report:
		\[
		\mu \in [2320, 2480]\ \text{USD} \quad \text{at 90\% confidence,}
		\]
		\[
		\mu \in [2280, 2520]\ \text{USD} \quad \text{at 95\% confidence.}
		\]
		Management asks which interval should be used for budgeting decisions and why.
		
		\vspace{0.5cm}
		\SubsectionBox{Problem 8}
		\subsection*{Micro-Shop Revenue Mean}
		An online seller cooperative estimates the population mean weekly revenue for new micro-shops.
		A 95\% confidence interval is:
		\[
		\mu \in [680, 790]\ \text{USD}.
		\]
		A mentor says a benchmark of 800 USD per week is a realistic average target for beginners.
		Is 800 USD plausible as the population mean according to this interval?
		
		\vspace{0.5cm}
		\SubsectionBox{Problem 9}
		\subsection*{Subscription Spending Mean Estimate}
		A fintech app estimates the population mean monthly subscription spending by users aged 21--29.
		A 95\% confidence interval is:
		\[
		\mu \in [64, 88]\ \text{USD}.
		\]
		A budgeting coach proposes a ``safe cap'' of 90 USD as the typical monthly mean for this age group.
		Based on the interval, should 90 USD be treated as a plausible population mean?
		
		\vspace{0.5cm}
		\SubsectionBox{Problem 10}
		\subsection*{Branch Commission Mean Comparison}
		A sales startup evaluates the population mean monthly commission for first-year representatives in two branches.
		Branch East reports a 95\% confidence interval:
		\[
		\mu_E \in [910, 1080]\ \text{USD}.
		\]
		Branch West reports a 95\% confidence interval:
		\[
		\mu_W \in [860, 980]\ \text{USD}.
		\]
		Leadership wants to know whether East can be considered clearly stronger in average commission, and whether training incentives should differ.
		
	\end{multicols}
	
\end{document}