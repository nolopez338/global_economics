\makeatletter
\def\input@path{{./}{../}{../../}{preamble/}{../preamble/}{../../preamble/}}
\makeatother
% ----------------------------------------------------------
% GENERAL 

% File
\documentclass[11pt]{book}

% Margins
\usepackage[margin=1in]{geometry}

\linespread{1.2}            % Line spacing
\usepackage[utf8]{inputenc}
\usepackage[T1]{fontenc}
\usepackage{lmodern}
\usepackage{microtype}
\setlength{\parindent}{0pt}
\setlength{\parskip}{6pt}
\usepackage{booktabs}

% ----------------------------------------------------------
% TABLES
\usepackage{multicol}
\usepackage{longtable} 
\usepackage{array}
\usepackage{booktabs}
\usepackage{tabularx}
\usepackage{multirow}

% ----------------------------------------------------------
% MATHEMATICS
\usepackage{amsmath}
\usepackage{amssymb}
\usepackage{amsfonts}
\usepackage{mathtools}

% ----------------------------------------------------------
% HIDDEN CONTENT
\usepackage{ifthen}
% Define a boolean switch
\newboolean{explicaciones}
% Set the boolean switch to true or false
% Change to true to show the content

% Explanations
\newcommand{\explicacion}[2]{
	\ifthenelse{\boolean{explicaciones}}{#1}{#2}
}
\newcommand{\mostrarExplicaciones}[1]{\setboolean{explicaciones}{#1}}

% ----------------------------------------------------------
% NUMBERING

\usepackage{fancyhdr}
\pagestyle{empty} % Ensures the entire document has no page numbers

\usepackage{tocloft}
\renewcommand{\cftdot}{} % Remove dots for sections, if any
\renewcommand{\cftsecleader}{\cftdotfill{\cftdotsep}} % Remove dots for sections, if any
\cftpagenumbersoff{section} % Removes page numbers from sections
\cftpagenumbersoff{subsection} % Removes page numbers from subsections

% ----------------------------------------------------------
% IMAGES 

% General settings
\usepackage{graphicx}       % Insert images
\usepackage{float}          % Position images
% \usepackage{subfigure}      % Subfigures
\graphicspath{{imgs}}       % Image location
\usepackage{subcaption}     % Subfigures II
\usepackage{verbatim}

% Figures
\usepackage{tikz}
\usetikzlibrary{arrows.meta,positioning,trees}

% Colors
\usepackage{xcolor}     
\definecolor{popUp}{HTML}{666666}
\definecolor{popUpIn}{HTML}{CED9E0}
\definecolor{backgroundC}{HTML}{D0E8F2}
\definecolor{backgroundCC}{HTML}{FFFFFF}
\definecolor{borders}{HTML}{8c120d}
\definecolor{padding}{HTML}{77D0D7}
\definecolor{links}{HTML}{CC6F5F}

% ----------------------------------------------------------
% FRAMES

% General settings
\usepackage{tcolorbox}
\usepackage{adjustbox}          % Adjusted frame  
\setlength{\fboxrule}{3pt}  % Line width
\setlength{\fboxsep}{3pt}   % Box padding

% General frames
\usepackage{mdframed}   

\mdfdefinestyle{estiloGeneral}{    % General style
	linecolor=black,
	linewidth=1.5pt,
	roundcorner=10pt,
	backgroundcolor=backgroundC,
	innerbottommargin=0pt
}
\mdfdefinestyle{code}{          % Code style
	linecolor=black,
	linewidth=1.5pt,
	roundcorner=10pt,
	backgroundcolor=darkgray!10,
	innerbottommargin=0pt
}

% Image frame
\newtcbox{\fboxC}{
	colback=backgroundC,
	colframe=popUp,
	arc=10pt,
	boxrule=3pt,
	boxsep=0pt, % Change the padding here
	nobeforeafter
}

% ----------------------------------------------------------
% PAGE SETTINGS

% Background 
\newcommand{\background}[0]{\begin{tikzpicture}[remember picture,overlay]
		\fill[backgroundC] (-2,2) rectangle (25cm, -550);
\end{tikzpicture}}

\newcommand{\backgroundC}[0]{\begin{tikzpicture}[remember picture,overlay]
		\fill[backgroundCC] (-2,2) rectangle (25cm, -550);
\end{tikzpicture}}

% Page width 
\newcommand{\anchoPag}[0]{20cm}

% ----------------------------------------------------------
% FONT

% General
\usepackage{tgbonum}        % Font
\usepackage{listings}       % Code typesetting
\usepackage[spanish]{babel} % Load Spanish
\selectlanguage{spanish}    % Select Spanish
\usepackage{enumitem}
\usepackage{bookmark}

\setlist[itemize]{leftmargin=1.2em, itemsep=0.35em, topsep=0.35em}

% --- Table helpers ---
\newcolumntype{L}[1]{>{\raggedright\arraybackslash}p{#1}}
\newcolumntype{Y}{>{\raggedright\arraybackslash}X}
\newcolumntype{C}{>{\centering\arraybackslash}X}
\renewcommand{\arraystretch}{1.1}

% Python style
\lstdefinestyle{python}{
	language=Python,
	basicstyle=\ttfamily\small,
	commentstyle=\color{green!50!black},
	keywordstyle=\color{blue},
	numberstyle=\tiny\color{gray},
	numbers=left,
	morekeywords={>, <},
	breakatwhitespace=false,
	showstringspaces=false,
	showtabs=false,
	showspaces=false
}

% ----------------------------------------------------------
% HYPERLINKS

% General
\usepackage{hyperref}       
\hypersetup{
	colorlinks=true,
	linkcolor=links,
	filecolor=magenta,      
	urlcolor=brown,
}

% Custom commands 

% Large link
\newcommand{\bigLink}[2]{\begin{center} \fboxC{\LARGE{\href{#1}{#2}}}\end{center}}

% Small link
\newcommand{\smallLink}[2]{\begin{center}\fboxC{\href{#1}{#2}}\end{center}}

% Bold link
\newcommand{\bfLink}[2]{\href{#1}{\textbf{#2}}}


% Small URL
\newcommand{\smallUrl}[1]{\begin{center}\fboxC{\url{#1}}\end{center}}


% ----------------------------------------------------------
% CUSTOM COMMANDS FOR FIGURES

\newcommand{\espacioImagenes}[0]{-1.2cm}

% Without frame
\newcommand{\fig}[3][\espacioImagenes]{
	\hspace*{#1}
	\centering
	\includegraphics[width=#2\textwidth]{#3}
}

% With frame
\newcommand{\ffig}[2]{\begin{figure}[h]
		\hspace*{\espacioImagenes}
		\centering
		\fbox{\includegraphics[width=#1\textwidth]{#2}}
\end{figure}}

% Hyperlink with frame
\newcommand{\hfig}[3]{\begin{figure}[h]
		\hspace*{-1.4cm}
		\centering
		\color{popUp}
		\fboxC{\href{#1}{\includegraphics[width=#2\textwidth]{#3}}}
	\end{figure}
}

% Hyperlink without frame
\newcommand{\hffig}[3]{\begin{figure}[h]
		\hspace*{-1.1cm}
		\centering
		\color{popUp}
		\href{#1}{\includegraphics[width=#2\textwidth]{#3}}
	\end{figure}
}

% ----------------------------------------------------------

% Start and Contents
\newcommand{\cuadro}[1]{
	\begin{mdframed}[style=estiloGeneral]
		#1 
	\end{mdframed}
}

% Explanation video image
\newcommand{\linkExplicacion}[1]{
	\hffig{#1}{0.5}{principal/videoExplicacion}
	\vspace{-0.5cm}
}

\newcommand{\subSecLink}[2]{
	\subsubsection*{\href{#1}{\textbf{#2}}}
}

% Spacing
\newcommand{\esp}[0]{\vspace{4mm}}

% Back to start
\newcommand{\secInicio}[0]{\begin{center}\hyperref[sec:inicio]{ 
			\includegraphics[width=0.1\textwidth]{principal/up}
	}\end{center}
}


\geometry{margin=0.85in}
\AtBeginDocument{\small}

\newcommand{\ExamNameField}{\noindent\textbf{Name:}\ \rule{0.7\linewidth}{0.4pt}\par\medskip}

\newcommand{\ExamTitleBlock}[3]{%
	\begin{center}
		\Large\textbf{#1}\\
		\textbf{#2}%
		\if\relax\detokenize{#3}\relax\else\\\textbf{#3}\fi
	\end{center}
	\vspace{0.5em}
}

\newcommand{\ExamSection}[1]{\par\medskip\textbf{#1}\par\smallskip}

\newenvironment{ExamCriteria}{%
	\begin{itemize}[leftmargin=1.6em, itemsep=0.3em, topsep=0.2em]
}{%
	\end{itemize}
}

\newenvironment{ExamProblems}{%
	\begin{enumerate}[label=\textbf{P\arabic*}, leftmargin=0pt, labelsep=0.6em, itemindent=2.2em, itemsep=0.8em]
}{%
	\end{enumerate}
}

\begin{document}
	\ExamTitleBlock{11th grade}{Learning evidence 2.1: Confidence Interval Planning (Supplementary Solutions)}{}
	
	\ExamSection{Evaluated criteria}
	\begin{ExamCriteria}
		\item C1: Compute the sample mean and sample standard deviation.
		\item C2: Distinguish between sample statistics and population parameters.
		\item C3: Explain why interval estimation is preferred over a single point estimate.
		\item C5: Construct a X\% confidence interval using known population standard deviation.
		\item C4: Interpret the meaning of a X\% confidence interval in context.
	\end{ExamCriteria}

	\ExamSection{Problems}
	\begin{ExamProblems}
		\newpage
		\item
		\subsection*{Problem description}
		A school cafeteria wants to estimate the average number of fruit servings sold per day.
		A random sample of 8 days is recorded as raw data (servings per day): 12, 15, 14, 16, 13, 17, 15, 14.
		The cafeteria has a historical estimate of the population standard deviation, \(\sigma = 2.4\) servings.
		For classroom purposes, suppose the true population mean is \(\mu = 15\) servings.
		Construct and interpret a 90\% confidence interval, a 95\% confidence interval, and a 99\% confidence interval for the population mean number of servings sold per day.

		\subsection*{C1}
		First, we calculate the sample mean.
		\[
		\bar{x} = \frac{\sum x}{n} \qquad \longrightarrow \qquad \sum x = 12 + 15 + 14 + 16 + 13 + 17 + 15 + 14 = 116 \qquad \longrightarrow \qquad \bar{x} = \frac{116}{8} = 14.5
		\]
		Second, we calculate the sample variance and then the sample standard deviation.
		\[
		s^2 = \frac{\sum (x-\bar{x})^2}{n-1}
		\]
		\[
		\sum (x-\bar{x})^2 = (12-14.5)^2 + (15-14.5)^2 + (14-14.5)^2 + (16-14.5)^2 + (13-14.5)^2 + (17-14.5)^2 + (15-14.5)^2 + (14-14.5)^2 = 18
		\]
		\[
		s^2 = \frac{18}{8-1} \approx 2.57 \qquad \longrightarrow \qquad s = \sqrt{s^2} \approx 1.60
		\]

		\subsection*{C2}
		The population parameters are $\mu$ and $\sigma = 2.4$, while the sample statistics are $\bar{x} = 14.5$ and $s \approx 1.60$. The sample mean $\bar{x}$ estimates the population mean $\mu$, and the sample standard deviation $s$ estimates $\sigma$ even though $\sigma$ is given here. This comparison shows that inference about the mean depends on sample statistics when the population mean is unknown.

		\subsection*{C3}
		The sample includes only 8 days, so a single point estimate could miss the true mean because of sampling variability. An interval estimate is preferred because it accounts for that uncertainty and gives a range of plausible values for $\mu$. For planning cafeteria inventory, a range of likely daily servings is more informative than a single number.

		\subsection*{C5}
		With known $\sigma = 2.4$ and $n=8$,
		\[
		SE = \frac{\sigma}{\sqrt{n}} \qquad \longrightarrow \qquad SE = \frac{2.4}{\sqrt{8}} \approx 0.85
		\]
		For 90\% confidence, $z_{0.05}=1.645$.
		\[
		E_{90} = z_{0.05} \cdot SE \qquad \longrightarrow \qquad E_{90} = 1.645(0.85) \approx 1.40
		\]
		\[
		\mu \in \bar{x} \pm E_{90} \qquad \longrightarrow \qquad \mu \in 14.5 \pm 1.40 = [14.5 - 1.40, 14.5 + 1.40] = [13.10, 15.90]
		\]
		For 95\% confidence, $z_{0.025}=1.96$.
		\[
		E_{95} = z_{0.025} \cdot SE \qquad \longrightarrow \qquad E_{95} = 1.96(0.85) \approx 1.66
		\]
		\[
		\mu \in \bar{x} \pm E_{95} \qquad \longrightarrow \qquad \mu \in 14.5 \pm 1.66 = [14.5 - 1.66, 14.5 + 1.66] = [12.84, 16.16]
		\]
		For 99\% confidence, $z_{0.005}=2.576$.
		\[
		E_{99} = z_{0.005} \cdot SE \qquad \longrightarrow \qquad E_{99} = 2.576(0.85) \approx 2.19
		\]
		\[
		\mu \in \bar{x} \pm E_{99} \qquad \longrightarrow \qquad \mu \in 14.5 \pm 2.19 = [14.5 - 2.19, 14.5 + 2.19] = [12.31, 16.69]
		\]

		\subsection*{C4}
		We are 90\% confident the true mean daily servings is between about 13.10 and 15.90 servings, 95\% confident it is between about 12.84 and 16.16 servings, and 99\% confident it is between about 12.31 and 16.69 servings. The confidence level means that in repeated sampling about 90\%, 95\%, or 99\% of the resulting intervals would capture $\mu$, and it does not mean there is a 90\%, 95\%, or 99\% chance that $\mu$ lies in one specific interval. Since the classroom value $\mu = 15$ is inside all three intervals, the sample supports that value as plausible for the cafeteria.

		\newpage
		\item
		\subsection*{Problem description}
		A city library wants to estimate the average number of study rooms booked per day.
		Two different random samples are drawn from the same population of daily bookings.
		The first sample (Sample A) records 18, 20, 19, 21, 17, 22 bookings and the second sample (Sample B) records 16, 18, 17, 19, 20 bookings.
		The population standard deviation is known from long-term records to be \(\sigma = 3.2\) bookings.
		For classroom purposes, suppose the true population mean is \(\mu = 19\) bookings.
		Construct and interpret a 95\% confidence interval for the population mean based on each sample.

		\subsection*{C1}
		First, we calculate the sample means.
		\[
		\bar{x}_A = \frac{\sum x_A}{n_A} \qquad \longrightarrow \qquad \sum x_A = 18 + 20 + 19 + 21 + 17 + 22 = 117 \qquad \longrightarrow \qquad \bar{x}_A = \frac{117}{6} = 19.5
		\]
		\[
		\bar{x}_B = \frac{\sum x_B}{n_B}  \qquad \longrightarrow \qquad \sum x_B = 16 + 18 + 17 + 19 + 20 = 90  \qquad \longrightarrow \qquad \bar{x}_B = \frac{90}{5} = 18
		\]
		Second, we calculate the sample variance and then the sample standard deviation.
		\[
		s_A^2 = \frac{\sum (x_A-\bar{x}_A)^2}{n_A-1} \qquad \qquad s_B^2 = \frac{\sum (x_B-\bar{x}_B)^2}{n_B-1}
		\]
		\[
		\sum (x_A-\bar{x}_A)^2 = (18-19.5)^2 + (20-19.5)^2 + (19-19.5)^2 + (21-19.5)^2 + (17-19.5)^2 + (22-19.5)^2 = 17.5
		\]
		\[
		s_A^2 = \frac{17.5}{6-1} = 3.5 \qquad \longrightarrow \qquad s_A = \sqrt{s_A^2} \approx 1.87
		\]
		\[
		\sum (x_B-\bar{x}_B)^2 = (16-18)^2 + (18-18)^2 + (17-18)^2 + (19-18)^2 + (20-18)^2 = 10
		\]
		\[
		s_B^2 = \frac{10}{5-1} = 2.5  \qquad \longrightarrow \qquad s_B = \sqrt{s_B^2} \approx 1.58
		\]

		\subsection*{C2}
		The population parameters are $\mu$ and $\sigma = 3.2$, while the sample statistics are $\bar{x}_A = 19.5$, $s_A \approx 1.87$, $\bar{x}_B = 18$, and $s_B \approx 1.58$. The values $\bar{x}_A$ and $\bar{x}_B$ estimate the population mean $\mu$, and $s_A$ and $s_B$ estimate the spread even though $\sigma$ is known for this exercise. This comparison shows that statistical inference relies on sample statistics when the population mean is not directly observed.

		\subsection*{C3}
		A single point estimate can miss the true mean because of sampling variability in the booking counts. A confidence interval is preferred because it accounts for that uncertainty and gives a range of plausible values for $\mu$. For planning daily staffing, a range is more informative than one value.

		\subsection*{C5}
		With known $\sigma = 3.2$ and $n_A = 6$, for 95\% confidence, $z_{0.025}=1.96$.
		\[
		SE_A = \frac{\sigma}{\sqrt{n_A}}  \qquad \longrightarrow \qquad SE_A = \frac{3.2}{\sqrt{6}} \approx 1.31  \qquad \longrightarrow \qquad  E_A = z_{0.025} \cdot SE_A = 1.96(1.31) \approx 2.56
		\]
		\[
		\mu \in \bar{x}_A \pm E_A  \qquad \longrightarrow \qquad \mu \in 19.5 \pm 2.56 = [19.5 - 2.56, 19.5 + 2.56] = [16.94, 22.06]
		\]
		With known $\sigma = 3.2$ and $n_B = 5$,
		\[
		SE_B = \frac{\sigma}{\sqrt{n_B}}  \qquad \longrightarrow \qquad SE_B = \frac{3.2}{\sqrt{5}} \approx 1.43  \qquad \longrightarrow \qquad  E_B = z_{0.025} \cdot SE_B = 1.96(1.43) \approx 2.80
		\]
		\[
		\mu \in \bar{x}_B \pm E_B   \qquad \longrightarrow \qquad \mu \in 18 \pm 2.80 = [18 - 2.80, 18 + 2.80] = [15.20, 20.80]
		\]

		\subsection*{C4}
		We are 95\% confident the true mean daily bookings in the population is between about 16.94 and 22.06 bookings based on Sample A, and 95\% confident it is between about 15.20 and 20.80 bookings based on Sample B. The confidence level means that in repeated sampling about 95\% of the resulting intervals would capture $\mu$, and it does not mean there is a 95\% chance that $\mu$ lies in one specific interval. Since both samples come from the same population, the two intervals are estimating the same parameter $\mu$, and both intervals include the classroom value $\mu = 19$ as plausible.

		\newpage
		\item
		\subsection*{Problem description}
		A training center tracks the number of completed modules per week.
		A random sample of 30 weeks is summarized in grouped form. The population standard deviation is known to be \(\sigma = 4.5\) modules per week.
		For academic purposes, assume the true population mean is \(\mu = 14\) modules per week.
		The grouped data pairs list modules per week first and the frequency (number of weeks) second: \(9 \rightarrow 6\), \(12 \rightarrow 10\), \(15 \rightarrow 8\), and \(18 \rightarrow 6\).
		Construct and interpret a 90\% confidence interval, a 95\% confidence interval, and a 99\% confidence interval for the population mean modules per week.

		\subsection*{C1}
		Using grouped values $x$ with frequencies $f$.
		\[
		n = \sum f   \qquad \longrightarrow \qquad n = 6 + 10 + 8 + 6 = 30
		\]
		\[
		\bar{x} = \frac{\sum f x}{n} \qquad \longrightarrow \qquad \sum f x = 6(9) + 10(12) + 8(15) + 6(18) = 402  \qquad \longrightarrow \qquad \bar{x} = \frac{402}{30} = 13.4
		\]
		Second, we calculate the sample variance and then the sample standard deviation $s$.
		\[
		s^2 = \frac{\sum f(x-\bar{x})^2}{n-1} \qquad \longrightarrow \qquad \sum f(x-\bar{x})^2 = 6(9-13.4)^2 + 10(12-13.4)^2 + 8(15-13.4)^2 + 6(18-13.4)^2
		\]
		\[
		\sum f(x-\bar{x})^2 = 116.16 + 19.6 + 20.48 + 126.96 = 283.2
		\]
		\[
		s^2 = \frac{283.2}{30-1} \approx 9.77  \qquad \longrightarrow \qquad s = \sqrt{s^2} \approx 3.12
		\]

		\subsection*{C2}
		The population parameters are $\mu$ and $\sigma = 4.5$, while the sample statistics are $\bar{x} = 13.4$ and $s \approx 3.12$. The sample mean $\bar{x}$ estimates the population mean $\mu$, and the sample standard deviation $s$ estimates $\sigma$ even though $\sigma$ is given. This comparison shows that inference for the mean depends on sample statistics when the population mean is unknown.

		\subsection*{C3}
		The sample summarizes only 30 weeks, so the true mean could differ from the point estimate because of sampling variability. An interval estimate is preferred because it captures that uncertainty and provides a range of plausible values for $\mu$. For planning training schedules, a plausible range for the completion rate is more useful than a single number.

		\subsection*{C5}
		With known $\sigma = 4.5$ and $n=30$,
		\[
		SE = \frac{\sigma}{\sqrt{n}} \qquad \longrightarrow \qquad SE = \frac{4.5}{\sqrt{30}} \approx 0.82
		\]
		For 90\% confidence, $z_{0.05}=1.645$.
		\[
		E_{90} = z_{0.05} \cdot SE \qquad \longrightarrow \qquad E_{90} = 1.645(0.82) \approx 1.35
		\]
		\[
		\mu \in \bar{x} \pm E_{90} \qquad \longrightarrow \qquad \mu \in 13.4 \pm 1.35 = [13.4 - 1.35, 13.4 + 1.35] = [12.05, 14.75]
		\]
		For 95\% confidence, $z_{0.025}=1.96$.
		\[
		E_{95} = z_{0.025} \cdot SE \qquad \longrightarrow \qquad E_{95} = 1.96(0.82) \approx 1.61
		\]
		\[
		\mu \in \bar{x} \pm E_{95} \qquad \longrightarrow \qquad \mu \in 13.4 \pm 1.61 = [13.4 - 1.61, 13.4 + 1.61] = [11.79, 15.01]
		\]
		For 99\% confidence, $z_{0.005}=2.576$.
		\[
		E_{99} = z_{0.005} \cdot SE \qquad \longrightarrow \qquad E_{99} = 2.576(0.82) \approx 2.12
		\]
		\[
		\mu \in \bar{x} \pm E_{99} \qquad \longrightarrow \qquad \mu \in 13.4 \pm 2.12 = [13.4 - 2.12, 13.4 + 2.12] = [11.28, 15.52]
		\]

		\subsection*{C4}
		We are 90\% confident the true mean completion rate is between about 12.05 and 14.75 modules, 95\% confident it is between about 11.79 and 15.01 modules, and 99\% confident it is between about 11.28 and 15.52 modules. The confidence level means that in repeated sampling about 90\%, 95\%, or 99\% of such intervals would contain $\mu$, and it does not mean there is a 90\%, 95\%, or 99\% chance that $\mu$ lies in this one interval. Since the classroom value $\mu = 14$ lies in all three intervals, the sample supports that value as plausible for the training center.

		\newpage
		\item
		\subsection*{Problem description}
		A delivery company wants to estimate the average number of packages processed per shift.
		Two different random samples are drawn from the same population of processing counts.
		Sample A is grouped as \(14 \rightarrow 7\), \(18 \rightarrow 9\), \(22 \rightarrow 4\), and Sample B is grouped as \(14 \rightarrow 5\), \(18 \rightarrow 8\), \(22 \rightarrow 7\).
		The population standard deviation is known to be \(\sigma = 3.6\) packages.
		For classroom purposes, suppose the true population mean is \(\mu = 18\) packages.
		Construct and interpret a 95\% confidence interval for the population mean based on each sample.

		\subsection*{C1}
		Using grouped values $x$ with frequencies $f$.
		\[
		n_A = \sum f_A \qquad \longrightarrow \qquad n_A = 7 + 9 + 4 = 20
		\]
		\[
		\bar{x}_A = \frac{\sum f_A x}{n_A} \qquad \longrightarrow \qquad \sum f_A x = 7(14) + 9(18) + 4(22) = 348  \qquad \longrightarrow \qquad \bar{x}_A = \frac{348}{20} = 17.4
		\]
		\[
		\sum f_A(x-\bar{x}_A)^2 = 7(14-17.4)^2 + 9(18-17.4)^2 + 4(22-17.4)^2
		\]
		\[
		\sum f_A(x-\bar{x}_A)^2 = 80.92 + 3.24 + 84.64 = 168.8
		\]
		\[
		s_A^2 = \frac{168.8}{20-1} \approx 8.88 \qquad \longrightarrow \qquad s_A = \sqrt{s_A^2} \approx 2.98
		\]
		\[
		n_B = \sum f_B \qquad \longrightarrow \qquad n_B = 5 + 8 + 7 = 20
		\]
		\[
		\bar{x}_B = \frac{\sum f_B x}{n_B} \qquad \longrightarrow \qquad \sum f_B x = 5(14) + 8(18) + 7(22) = 368  \qquad \longrightarrow \qquad \bar{x}_B = \frac{368}{20} = 18.4
		\]
		\[
		\sum f_B(x-\bar{x}_B)^2 = 5(14-18.4)^2 + 8(18-18.4)^2 + 7(22-18.4)^2
		\]
		\[
		\sum f_B(x-\bar{x}_B)^2 = 96.8 + 1.28 + 90.72 = 188.8
		\]
		\[
		s_B^2 = \frac{188.8}{20-1} \approx 9.94 \qquad \longrightarrow \qquad s_B = \sqrt{s_B^2} \approx 3.15
		\]

		\subsection*{C2}
		The population parameters are $\mu$ and $\sigma = 3.6$, while the sample statistics are $\bar{x}_A = 17.4$, $s_A \approx 2.98$, $\bar{x}_B = 18.4$, and $s_B \approx 3.15$. The sample means estimate the population mean $\mu$, and the sample standard deviations estimate $\sigma$ even though $\sigma$ is known. This comparison shows that inference for the mean relies on grouped sample statistics when the population mean is unknown.

		\subsection*{C3}
		Each sample represents only 20 shifts, so a point estimate may miss the true mean because of sampling variability. An interval estimate is preferred because it captures that uncertainty and gives a range of plausible values for $\mu$. For staffing decisions, a plausible range for the processing rate is more useful than a single number.

		\subsection*{C5}
		With known $\sigma = 3.6$ and $n_A = 20$, for 95\% confidence, $z_{0.025}=1.96$.
		\[
		SE_A = \frac{\sigma}{\sqrt{n_A}}  \qquad \longrightarrow \qquad SE_A = \frac{3.6}{\sqrt{20}} \approx 0.80  \qquad \longrightarrow \qquad  E_A = z_{0.025} \cdot SE_A = 1.96(0.80) \approx 1.58
		\]
		\[
		\mu \in \bar{x}_A \pm E_A  \qquad \longrightarrow \qquad \mu \in 17.4 \pm 1.58 = [17.4 - 1.58, 17.4 + 1.58] = [15.82, 18.98]
		\]
		With known $\sigma = 3.6$ and $n_B = 20$,
		\[
		SE_B = \frac{\sigma}{\sqrt{n_B}}  \qquad \longrightarrow \qquad SE_B = \frac{3.6}{\sqrt{20}} \approx 0.80  \qquad \longrightarrow \qquad  E_B = z_{0.025} \cdot SE_B = 1.96(0.80) \approx 1.58
		\]
		\[
		\mu \in \bar{x}_B \pm E_B   \qquad \longrightarrow \qquad \mu \in 18.4 \pm 1.58 = [18.4 - 1.58, 18.4 + 1.58] = [16.82, 19.98]
		\]

		\subsection*{C4}
		We are 95\% confident the true mean packages per shift is between about 15.82 and 18.98 packages based on Sample A, and 95\% confident it is between about 16.82 and 19.98 packages based on Sample B. The confidence level means that in repeated sampling about 95\% of the resulting intervals would capture $\mu$, and it does not mean there is a 95\% chance that $\mu$ lies in one specific interval. Since both samples come from the same population, the two intervals are estimating the same parameter $\mu$, and both intervals include the classroom value $\mu = 18$ as plausible.

		\newpage
		\item
		\subsection*{Problem description}
		A fitness center records the number of active members who attend a class session.
		A random sample of 50 sessions is grouped into class intervals. The population standard deviation is known to be \(\sigma = 5.2\) attendees.
		For academic purposes, assume the true population mean is \(\mu = 49\) attendees.
		The grouped intervals and frequencies are 40--44 (12), 45--49 (18), 50--54 (14), and 55--59 (6).
		Construct and interpret a 90\% confidence interval, a 95\% confidence interval, and a 99\% confidence interval for the population mean number of attendees.

		\subsection*{C1}
		Using grouped midpoints $x$ with frequencies $f$.
		\[
		n = \sum f \qquad \longrightarrow \qquad n = 12 + 18 + 14 + 6 = 50
		\]
		\[
		\bar{x} = \frac{\sum f x}{n} \qquad \longrightarrow \qquad \sum f x = 12(42) + 18(47) + 14(52) + 6(57) = 2420 \qquad \longrightarrow \qquad \bar{x} = \frac{2420}{50} = 48.4
		\]
		Second, we calculate the sample variance and then the sample standard deviation $s$.
		\[
		s^2 = \frac{\sum f(x-\bar{x})^2}{n-1} \qquad \longrightarrow \qquad \sum f(x-\bar{x})^2 = 12(42-48.4)^2 + 18(47-48.4)^2 + 14(52-48.4)^2 + 6(57-48.4)^2
		\]
		\[
		\sum f(x-\bar{x})^2 = 491.52 + 35.28 + 181.44 + 443.76 = 1152
		\]
		\[
		s^2 = \frac{1152}{50-1} \approx 23.51  \qquad \longrightarrow \qquad s = \sqrt{s^2} \approx 4.85
		\]

		\subsection*{C2}
		The population parameters are $\mu$ and $\sigma = 5.2$, while the sample statistics are $\bar{x} = 48.4$ and $s \approx 4.85$. The sample mean $\bar{x}$ estimates the population mean $\mu$, and the sample standard deviation $s$ estimates $\sigma$ even though $\sigma$ is given. This comparison shows that inference for the mean depends on sample statistics when the population mean is unknown.

		\subsection*{C3}
		Even with 50 observations, a point estimate does not show how much the mean could vary because of sampling variability. An interval estimate is preferred because it summarizes that uncertainty by giving a plausible range for $\mu$. For planning class capacity, a range of likely attendance is more informative than a single value.

		\subsection*{C5}
		With known $\sigma = 5.2$ and $n=50$,
		\[
		SE = \frac{\sigma}{\sqrt{n}} \qquad \longrightarrow \qquad SE = \frac{5.2}{\sqrt{50}} \approx 0.74
		\]
		For 90\% confidence, $z_{0.05}=1.645$.
		\[
		E_{90} = z_{0.05} \cdot SE \qquad \longrightarrow \qquad E_{90} = 1.645(0.74) \approx 1.21
		\]
		\[
		\mu \in \bar{x} \pm E_{90} \qquad \longrightarrow \qquad \mu \in 48.4 \pm 1.21 = [48.4 - 1.21, 48.4 + 1.21] = [47.19, 49.61]
		\]
		For 95\% confidence, $z_{0.025}=1.96$.
		\[
		E_{95} = z_{0.025} \cdot SE \qquad \longrightarrow \qquad E_{95} = 1.96(0.74) \approx 1.44
		\]
		\[
		\mu \in \bar{x} \pm E_{95} \qquad \longrightarrow \qquad \mu \in 48.4 \pm 1.44 = [48.4 - 1.44, 48.4 + 1.44] = [46.96, 49.84]
		\]
		For 99\% confidence, $z_{0.005}=2.576$.
		\[
		E_{99} = z_{0.005} \cdot SE \qquad \longrightarrow \qquad E_{99} = 2.576(0.74) \approx 1.89
		\]
		\[
		\mu \in \bar{x} \pm E_{99} \qquad \longrightarrow \qquad \mu \in 48.4 \pm 1.89 = [48.4 - 1.89, 48.4 + 1.89] = [46.51, 50.29]
		\]

		\subsection*{C4}
		We are 90\% confident the true mean attendance is between about 47.19 and 49.61 attendees, 95\% confident it is between about 46.96 and 49.84 attendees, and 99\% confident it is between about 46.51 and 50.29 attendees. The confidence level means that in repeated sampling about 90\%, 95\%, or 99\% of such intervals would contain $\mu$, and it does not mean there is a 90\%, 95\%, or 99\% chance that $\mu$ lies in this one interval. Since the classroom value $\mu = 49$ is in all three intervals, the sample supports that value as plausible for the fitness center.

		\newpage
		\item
		\subsection*{Problem description}
		A customer service center wants to estimate the average call length in minutes.
		Two different random samples are drawn from the same population of call lengths and summarized in class intervals.
		Sample A has intervals 28--32 (10), 33--37 (15), and 38--42 (9).
		Sample B has intervals 28--32 (8), 33--37 (14), and 38--42 (12).
		The population standard deviation is known to be \(\sigma = 4.1\) minutes.
		For classroom purposes, suppose the true population mean is \(\mu = 35\) minutes.
		Construct and interpret a 95\% confidence interval for the population mean based on each sample.

		\subsection*{C1}
		Using grouped midpoints $x$ with frequencies $f$.
		\[
		n_A = \sum f_A \qquad \longrightarrow \qquad n_A = 10 + 15 + 9 = 34
		\]
		\[
		\bar{x}_A = \frac{\sum f_A x}{n_A} \qquad \longrightarrow \qquad \sum f_A x = 10(30) + 15(35) + 9(40) = 1185 \qquad \longrightarrow \qquad \bar{x}_A = \frac{1185}{34} \approx 34.85
		\]
		\[
		\sum f_A(x-\bar{x}_A)^2 = 10(30-34.85)^2 + 15(35-34.85)^2 + 9(40-34.85)^2
		\]
		\[
		\sum f_A(x-\bar{x}_A)^2 \approx 235.51 + 0.32 + 238.43 = 474.26
		\]
		\[
		s_A^2 = \frac{474.26}{34-1} \approx 14.37 \qquad \longrightarrow \qquad s_A = \sqrt{s_A^2} \approx 3.79
		\]
		\[
		n_B = \sum f_B \qquad \longrightarrow \qquad n_B = 8 + 14 + 12 = 34
		\]
		\[
		\bar{x}_B = \frac{\sum f_B x}{n_B} \qquad \longrightarrow \qquad \sum f_B x = 8(30) + 14(35) + 12(40) = 1210 \qquad \longrightarrow \qquad \bar{x}_B = \frac{1210}{34} \approx 35.59
		\]
		\[
		\sum f_B(x-\bar{x}_B)^2 = 8(30-35.59)^2 + 14(35-35.59)^2 + 12(40-35.59)^2
		\]
		\[
		\sum f_B(x-\bar{x}_B)^2 \approx 249.83 + 4.84 + 233.56 = 488.24
		\]
		\[
		s_B^2 = \frac{488.24}{34-1} \approx 14.80 \qquad \longrightarrow \qquad s_B = \sqrt{s_B^2} \approx 3.85
		\]

		\subsection*{C2}
		The population parameters are $\mu$ and $\sigma = 4.1$, while the sample statistics are $\bar{x}_A \approx 34.85$, $s_A \approx 3.79$, $\bar{x}_B \approx 35.59$, and $s_B \approx 3.85$. The sample means estimate the population mean $\mu$, and the sample standard deviations estimate $\sigma$ even though $\sigma$ is known. This comparison shows that inference for the mean relies on grouped sample statistics when the population mean is unknown.

		\subsection*{C3}
		Each sample represents only 34 calls, so a point estimate may miss the true mean because of sampling variability. An interval estimate is preferred because it captures that uncertainty and gives a range of plausible values for $\mu$. For staffing the call center, a plausible range for average call length is more useful than a single number.

		\subsection*{C5}
		With known $\sigma = 4.1$ and $n_A = 34$, for 95\% confidence, $z_{0.025}=1.96$.
		\[
		SE_A = \frac{\sigma}{\sqrt{n_A}}  \qquad \longrightarrow \qquad SE_A = \frac{4.1}{\sqrt{34}} \approx 0.70  \qquad \longrightarrow \qquad  E_A = z_{0.025} \cdot SE_A = 1.96(0.70) \approx 1.38
		\]
		\[
		\mu \in \bar{x}_A \pm E_A  \qquad \longrightarrow \qquad \mu \in 34.85 \pm 1.38 = [34.85 - 1.38, 34.85 + 1.38] = [33.47, 36.23]
		\]
		With known $\sigma = 4.1$ and $n_B = 34$,
		\[
		SE_B = \frac{\sigma}{\sqrt{n_B}}  \qquad \longrightarrow \qquad SE_B = \frac{4.1}{\sqrt{34}} \approx 0.70  \qquad \longrightarrow \qquad  E_B = z_{0.025} \cdot SE_B = 1.96(0.70) \approx 1.38
		\]
		\[
		\mu \in \bar{x}_B \pm E_B   \qquad \longrightarrow \qquad \mu \in 35.59 \pm 1.38 = [35.59 - 1.38, 35.59 + 1.38] = [34.21, 36.97]
		\]

		\subsection*{C4}
		We are 95\% confident the true mean call length is between about 33.47 and 36.23 minutes based on Sample A, and 95\% confident it is between about 34.21 and 36.97 minutes based on Sample B. The confidence level means that in repeated sampling about 95\% of the resulting intervals would capture $\mu$, and it does not mean there is a 95\% chance that $\mu$ lies in one specific interval. Since both samples come from the same population, the two intervals are estimating the same parameter $\mu$, and both intervals include the classroom value $\mu = 35$ as plausible.

		\newpage
		\item
		\subsection*{Problem description}
		A manufacturing plant monitors the weight (in grams) of packaged items.
		Most of the data are summarized in grouped values, but a few extreme packages were recorded separately as outliers.
		The grouped data are 50 $\rightarrow$ 8, 55 $\rightarrow$ 12, and 60 $\rightarrow$ 10.
		The outlier raw observations are 72, 74, and 45 grams.
		The population standard deviation is known to be \(\sigma = 6.0\) grams.
		For classroom purposes, suppose the true population mean is \(\mu = 56\) grams.
		Construct and interpret a 95\% confidence interval for the population mean package weight.

		\subsection*{C1}
		Using grouped values $x$ with frequencies $f$, plus raw outliers.
		\[
		n = \sum f + \text{(number of outliers)} \qquad \longrightarrow \qquad n = (8 + 12 + 10) + 3 = 33
		\]
		\[
		\bar{x} = \frac{\sum f x + \sum x_{\text{out}}}{n} \qquad \longrightarrow \qquad \sum f x = 8(50) + 12(55) + 10(60) = 1660
		\]
		\[
		\sum x_{\text{out}} = 72 + 74 + 45 = 191 \qquad \longrightarrow \qquad \bar{x} = \frac{1660 + 191}{33} \approx 56.09
		\]
		Second, we calculate the sample variance and then the sample standard deviation.
		\[
		s^2 = \frac{\sum f(x-\bar{x})^2 + \sum (x_{\text{out}}-\bar{x})^2}{n-1}
		\]
		\[
		\sum f(x-\bar{x})^2 = 8(50-56.09)^2 + 12(55-56.09)^2 + 10(60-56.09)^2
		\]
		\[
		\sum f(x-\bar{x})^2 \approx 296.79 + 14.28 + 152.81 = 463.88
		\]
		\[
		\sum (x_{\text{out}}-\bar{x})^2 = (72-56.09)^2 + (74-56.09)^2 + (45-56.09)^2 \approx 253.10 + 320.74 + 123.01 = 696.85
		\]
		\[
		s^2 = \frac{463.88 + 696.85}{33-1} \approx 36.27 \qquad \longrightarrow \qquad s = \sqrt{s^2} \approx 6.02
		\]

		\subsection*{C2}
		The population parameters are $\mu$ and $\sigma = 6.0$, while the sample statistics are $\bar{x} \approx 56.09$ and $s \approx 6.02$. The sample mean $\bar{x}$ estimates the population mean $\mu$, and the sample standard deviation $s$ estimates $\sigma$ even though $\sigma$ is given. This comparison shows that inference for the mean relies on combining grouped data and raw outliers to compute consistent sample statistics.

		\subsection*{C3}
		The combined sample includes only 33 packages, and the outliers show that some packages are far from the typical grouped values. A confidence interval is preferred because it reflects this variability and gives a range of plausible values for $\mu$. For quality control, a range of likely mean weights is more informative than a single value.

		\subsection*{C5}
		With known $\sigma = 6.0$ and $n=33$, for 95\% confidence, $z_{0.025}=1.96$.
		\[
		SE = \frac{\sigma}{\sqrt{n}} \qquad \longrightarrow \qquad SE = \frac{6.0}{\sqrt{33}} \approx 1.04
		\]
		\[
		E = z_{0.025} \cdot SE \qquad \longrightarrow \qquad E = 1.96(1.04) \approx 2.05
		\]
		\[
		\mu \in \bar{x} \pm E \qquad \longrightarrow \qquad \mu \in 56.09 \pm 2.05 = [56.09 - 2.05, 56.09 + 2.05] = [54.04, 58.14]
		\]

		\subsection*{C4}
		We are 95\% confident the true mean package weight is between about 54.04 and 58.14 grams. The confidence level means that in repeated sampling about 95\% of the resulting intervals would capture $\mu$, and it does not mean there is a 95\% chance that $\mu$ lies in this one interval. The outliers pull the sample mean upward compared with the grouped values alone and increase the spread, which widens the confidence interval; nevertheless, the interval still provides a plausible range for the overall population mean.
	\end{ExamProblems}
\end{document}
