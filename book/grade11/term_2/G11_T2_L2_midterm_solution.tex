\makeatletter
\def\input@path{{./}{../}{../../}{preamble/}{../preamble/}{../../preamble/}}
\makeatother
% ----------------------------------------------------------
% GENERAL 

% File
\documentclass[11pt]{book}

% Margins
\usepackage[margin=1in]{geometry}

\linespread{1.2}            % Line spacing
\usepackage[utf8]{inputenc}
\usepackage[T1]{fontenc}
\usepackage{lmodern}
\usepackage{microtype}
\setlength{\parindent}{0pt}
\setlength{\parskip}{6pt}
\usepackage{booktabs}

% ----------------------------------------------------------
% TABLES
\usepackage{multicol}
\usepackage{longtable} 
\usepackage{array}
\usepackage{booktabs}
\usepackage{tabularx}
\usepackage{multirow}

% ----------------------------------------------------------
% MATHEMATICS
\usepackage{amsmath}
\usepackage{amssymb}
\usepackage{amsfonts}
\usepackage{mathtools}

% ----------------------------------------------------------
% HIDDEN CONTENT
\usepackage{ifthen}
% Define a boolean switch
\newboolean{explicaciones}
% Set the boolean switch to true or false
% Change to true to show the content

% Explanations
\newcommand{\explicacion}[2]{
	\ifthenelse{\boolean{explicaciones}}{#1}{#2}
}
\newcommand{\mostrarExplicaciones}[1]{\setboolean{explicaciones}{#1}}

% ----------------------------------------------------------
% NUMBERING

\usepackage{fancyhdr}
\pagestyle{empty} % Ensures the entire document has no page numbers

\usepackage{tocloft}
\renewcommand{\cftdot}{} % Remove dots for sections, if any
\renewcommand{\cftsecleader}{\cftdotfill{\cftdotsep}} % Remove dots for sections, if any
\cftpagenumbersoff{section} % Removes page numbers from sections
\cftpagenumbersoff{subsection} % Removes page numbers from subsections

% ----------------------------------------------------------
% IMAGES 

% General settings
\usepackage{graphicx}       % Insert images
\usepackage{float}          % Position images
% \usepackage{subfigure}      % Subfigures
\graphicspath{{imgs}}       % Image location
\usepackage{subcaption}     % Subfigures II
\usepackage{verbatim}

% Figures
\usepackage{tikz}
\usetikzlibrary{arrows.meta,positioning,trees}

% Colors
\usepackage{xcolor}     
\definecolor{popUp}{HTML}{666666}
\definecolor{popUpIn}{HTML}{CED9E0}
\definecolor{backgroundC}{HTML}{D0E8F2}
\definecolor{backgroundCC}{HTML}{FFFFFF}
\definecolor{borders}{HTML}{8c120d}
\definecolor{padding}{HTML}{77D0D7}
\definecolor{links}{HTML}{CC6F5F}

% ----------------------------------------------------------
% FRAMES

% General settings
\usepackage{tcolorbox}
\usepackage{adjustbox}          % Adjusted frame  
\setlength{\fboxrule}{3pt}  % Line width
\setlength{\fboxsep}{3pt}   % Box padding

% General frames
\usepackage{mdframed}   

\mdfdefinestyle{estiloGeneral}{    % General style
	linecolor=black,
	linewidth=1.5pt,
	roundcorner=10pt,
	backgroundcolor=backgroundC,
	innerbottommargin=0pt
}
\mdfdefinestyle{code}{          % Code style
	linecolor=black,
	linewidth=1.5pt,
	roundcorner=10pt,
	backgroundcolor=darkgray!10,
	innerbottommargin=0pt
}

% Image frame
\newtcbox{\fboxC}{
	colback=backgroundC,
	colframe=popUp,
	arc=10pt,
	boxrule=3pt,
	boxsep=0pt, % Change the padding here
	nobeforeafter
}

% ----------------------------------------------------------
% PAGE SETTINGS

% Background 
\newcommand{\background}[0]{\begin{tikzpicture}[remember picture,overlay]
		\fill[backgroundC] (-2,2) rectangle (25cm, -550);
\end{tikzpicture}}

\newcommand{\backgroundC}[0]{\begin{tikzpicture}[remember picture,overlay]
		\fill[backgroundCC] (-2,2) rectangle (25cm, -550);
\end{tikzpicture}}

% Page width 
\newcommand{\anchoPag}[0]{20cm}

% ----------------------------------------------------------
% FONT

% General
\usepackage{tgbonum}        % Font
\usepackage{listings}       % Code typesetting
\usepackage[spanish]{babel} % Load Spanish
\selectlanguage{spanish}    % Select Spanish
\usepackage{enumitem}
\usepackage{bookmark}

\setlist[itemize]{leftmargin=1.2em, itemsep=0.35em, topsep=0.35em}

% --- Table helpers ---
\newcolumntype{L}[1]{>{\raggedright\arraybackslash}p{#1}}
\newcolumntype{Y}{>{\raggedright\arraybackslash}X}
\newcolumntype{C}{>{\centering\arraybackslash}X}
\renewcommand{\arraystretch}{1.1}

% Python style
\lstdefinestyle{python}{
	language=Python,
	basicstyle=\ttfamily\small,
	commentstyle=\color{green!50!black},
	keywordstyle=\color{blue},
	numberstyle=\tiny\color{gray},
	numbers=left,
	morekeywords={>, <},
	breakatwhitespace=false,
	showstringspaces=false,
	showtabs=false,
	showspaces=false
}

% ----------------------------------------------------------
% HYPERLINKS

% General
\usepackage{hyperref}       
\hypersetup{
	colorlinks=true,
	linkcolor=links,
	filecolor=magenta,      
	urlcolor=brown,
}

% Custom commands 

% Large link
\newcommand{\bigLink}[2]{\begin{center} \fboxC{\LARGE{\href{#1}{#2}}}\end{center}}

% Small link
\newcommand{\smallLink}[2]{\begin{center}\fboxC{\href{#1}{#2}}\end{center}}

% Bold link
\newcommand{\bfLink}[2]{\href{#1}{\textbf{#2}}}


% Small URL
\newcommand{\smallUrl}[1]{\begin{center}\fboxC{\url{#1}}\end{center}}


% ----------------------------------------------------------
% CUSTOM COMMANDS FOR FIGURES

\newcommand{\espacioImagenes}[0]{-1.2cm}

% Without frame
\newcommand{\fig}[3][\espacioImagenes]{
	\hspace*{#1}
	\centering
	\includegraphics[width=#2\textwidth]{#3}
}

% With frame
\newcommand{\ffig}[2]{\begin{figure}[h]
		\hspace*{\espacioImagenes}
		\centering
		\fbox{\includegraphics[width=#1\textwidth]{#2}}
\end{figure}}

% Hyperlink with frame
\newcommand{\hfig}[3]{\begin{figure}[h]
		\hspace*{-1.4cm}
		\centering
		\color{popUp}
		\fboxC{\href{#1}{\includegraphics[width=#2\textwidth]{#3}}}
	\end{figure}
}

% Hyperlink without frame
\newcommand{\hffig}[3]{\begin{figure}[h]
		\hspace*{-1.1cm}
		\centering
		\color{popUp}
		\href{#1}{\includegraphics[width=#2\textwidth]{#3}}
	\end{figure}
}

% ----------------------------------------------------------

% Start and Contents
\newcommand{\cuadro}[1]{
	\begin{mdframed}[style=estiloGeneral]
		#1 
	\end{mdframed}
}

% Explanation video image
\newcommand{\linkExplicacion}[1]{
	\hffig{#1}{0.5}{principal/videoExplicacion}
	\vspace{-0.5cm}
}

\newcommand{\subSecLink}[2]{
	\subsubsection*{\href{#1}{\textbf{#2}}}
}

% Spacing
\newcommand{\esp}[0]{\vspace{4mm}}

% Back to start
\newcommand{\secInicio}[0]{\begin{center}\hyperref[sec:inicio]{ 
			\includegraphics[width=0.1\textwidth]{principal/up}
	}\end{center}
}


\geometry{margin=0.85in}
\AtBeginDocument{\small}

\newcommand{\ExamNameField}{\noindent\textbf{Name:}\ \rule{0.7\linewidth}{0.4pt}\par\medskip}

\newcommand{\ExamTitleBlock}[3]{%
	\begin{center}
		\Large\textbf{#1}\\
		\textbf{#2}%
		\if\relax\detokenize{#3}\relax\else\\\textbf{#3}\fi
	\end{center}
	\vspace{0.5em}
}

\newcommand{\ExamSection}[1]{\par\medskip\textbf{#1}\par\smallskip}

\newenvironment{ExamCriteria}{%
	\begin{itemize}[leftmargin=1.6em, itemsep=0.3em, topsep=0.2em]
}{%
	\end{itemize}
}

\newenvironment{ExamProblems}{%
	\begin{enumerate}[label=\textbf{P\arabic*}, leftmargin=0pt, labelsep=0.6em, itemindent=2.2em, itemsep=0.8em]
}{%
	\end{enumerate}
}

\begin{document}
	\ExamTitleBlock{11th grade}{Term 2 Midterm: Confidence Interval Planning (Solutions)}{}
	
	\ExamSection{Evaluated criteria}
	\begin{ExamCriteria}
		\item C1: Compute the sample mean and sample standard deviation.
		\item C2: Distinguish between sample statistics and population parameters.
		\item C3: Explain why interval estimation is preferred over a single point estimate.
		\item C4: Interpret the meaning of a X\% confidence interval in context.
		\item C5: Construct a X\% confidence interval using known population standard deviation.
		\item C6: Determine the required sample size to achieve a target margin of error.
		\item C7: Construct and interpret a confidence interval for a population proportion.
	\end{ExamCriteria}

	\ExamSection{Problems}
	\begin{ExamProblems}
		\newpage
		\item
		\subsection*{Problem description}
		A regional transit office studies the average weekday fare collected per rider for two pilot corridors.
		The two samples are distinct random samples drawn from the same population of fare amounts.
		The grouped fare data (USD) are below.
		
		\begin{multicols}{2}
			\textbf{Sample A (12 riders):}
			\begin{itemize}
				\item 40 USD occurred in 4 riders.
				\item 50 USD occurred in 3 riders.
				\item 65 USD occurred in 5 riders.
			\end{itemize}
			
			\columnbreak
			
			\textbf{Sample B (13 riders):}
			\begin{itemize}
				\item 40 USD occurred in 5 riders.
				\item 50 USD occurred in 2 riders.
				\item 65 USD occurred in 6 riders.
			\end{itemize}
		\end{multicols}
		
		Long-run audits show the population standard deviation is $\sigma = 12$ USD.
		For classroom purposes, suppose the true population mean fare is $\mu = 53$ USD.
		Construct and interpret 95\% confidence intervals and 90\% confidence intervals for the population mean in each corridor.
		At the end, determine how many additional observations are required in each sample group to reach a margin of error target of $E_{95} = 3.5$ USD for 95\% confidence and $E_{90} = 3.5$ USD for 90\% confidence.
		
		\subsection*{C1}
		\textbf{Sample A:}
		\[
		\sum f x = 4(40) + 3(50) + 5(65) = 160 + 150 + 325 = 635
		\]
		\[
		\bar{x}_A = \frac{635}{12} \approx 52.92
		\]
		\[
		\sum f(x-\bar{x}_A)^2 = 4(40-52.92)^2 + 3(50-52.92)^2 + 5(65-52.92)^2 \approx 1422.92
		\]
		\[
		s_A^2 = \frac{1422.92}{12-1} \approx 129.36, \qquad s_A = \sqrt{129.36} \approx 11.37
		\]
		
		\textbf{Sample B:}
		\[
		\sum f x = 5(40) + 2(50) + 6(65) = 200 + 100 + 390 = 690
		\]
		\[
		\bar{x}_B = \frac{690}{13} \approx 53.08
		\]
		\[
		\sum f(x-\bar{x}_B)^2 = 5(40-53.08)^2 + 2(50-53.08)^2 + 6(65-53.08)^2 \approx 1726.92
		\]
		\[
		s_B^2 = \frac{1726.92}{13-1} \approx 143.91, \qquad s_B = \sqrt{143.91} \approx 12.00
		\]
		
		\subsection*{C5}
		Use the known population standard deviation to build confidence intervals for each sample at two different confidence levels.
		
		First confidence interval uses 95\% confidence with $z^*_{95} = 1.96$.
		\[
		SE_A = \frac{\sigma}{\sqrt{12}} \approx 3.46, \qquad E_{A,95} = 1.96(3.46) \approx 6.79
		\]
		\[
		\mu \in \bar{x}_A \pm E_{A,95} \Rightarrow 52.92 \pm 6.79 = [46.13,\ 59.71]
		\]
		\[
		SE_B = \frac{\sigma}{\sqrt{13}} \approx 3.33, \qquad E_{B,95} = 1.96(3.33) \approx 6.52
		\]
		\[
		\mu \in \bar{x}_B \pm E_{B,95} \Rightarrow 53.08 \pm 6.52 = [46.56,\ 59.60]
		\]
		
		Second confidence interval uses 90\% confidence with $z^*_{90} = 1.645$.
		\[
		E_{A,90} = 1.645(3.46) \approx 5.69
		\]
		\[
		\mu \in \bar{x}_A \pm E_{A,90} \Rightarrow 52.92 \pm 5.69 = [47.23,\ 58.61]
		\]
		\[
		E_{B,90} = 1.645(3.33) \approx 5.48
		\]
		\[
		\mu \in \bar{x}_B \pm E_{B,90} \Rightarrow 53.08 \pm 5.48 = [47.60,\ 58.56]
		\]
		
		Interpretation.
		The 90\% confidence intervals are narrower than the 95\% confidence intervals.
		All intervals overlap and contain the classroom value $\mu = 53$ USD, so both confidence levels support similar mean fares for the two pilot corridors.
		
		\subsection*{C6}
		This subsection establishes the required sample size so the margin of error does not exceed the target amount.
		
		For 95\% confidence, the target margin of error is $E_{95} = 3.5$ USD and $z^*_{95} = 1.96$.
		\[
		n_{\text{required,95}} = \left(\frac{z^*_{95}\,\sigma}{E_{95}}\right)^2
		= \left(\frac{1.96 \cdot 12}{3.5}\right)^2
		= \left(6.72\right)^2
		\approx 45.16
		\]
		Round up to $n_{\text{required,95}} = 46$.
		Additional observations needed.
		Sample A needs $46 - 12 = 34$ more riders, and Sample B needs $46 - 13 = 33$ more riders.
		
		For 90\% confidence, the target margin of error is $E_{90} = 3.5$ USD and $z^*_{90} = 1.645$.
		\[
		n_{\text{required,90}} = \left(\frac{z^*_{90}\,\sigma}{E_{90}}\right)^2
		= \left(\frac{1.645 \cdot 12}{3.5}\right)^2
		\approx 31.81
		\]
		Round up to $n_{\text{required,90}} = 32$.
		Additional observations needed.
		Sample A needs $32 - 12 = 20$ more riders, and Sample B needs $32 - 13 = 19$ more riders.
		
		Conclusion.
		For 95\% confidence, each corridor should collect enough data to reach a total of 46 riders to ensure the margin of error is at most $E_{95} = 3.5$ USD.
		For 90\% confidence, each corridor should collect enough data to reach a total of 32 riders to ensure the margin of error is at most $E_{90} = 3.5$ USD.
		
		\newpage
		\item
		\subsection*{Problem description}
		A campus dining service compares average lunch checkout times (minutes) at two kiosks.
		The two samples are distinct random samples with sizes $n_1 = 3$ and $n_2 = 4$ drawn from the same population of checkout times.
		
		\textbf{Sample 1:} 18, 22, 20.
		
		\textbf{Sample 2:} 19, 21, 23, 20.
		
		Operational logs indicate the population standard deviation is $\sigma = 3.6$ minutes.
		Construct and interpret 95\% confidence intervals and 90\% confidence intervals for the population mean checkout time at each kiosk.
		At the end, determine how many additional observations are required in each kiosk sample to reach a margin of error target of $E_{95} = 2.0$ minutes for 95\% confidence and $E_{90} = 2.0$ minutes for 90\% confidence.
		
		\subsection*{C1}
		First, compute the sample means.
		\[
		\bar{x}_1 = \frac{18 + 22 + 20}{3} = 20.00
		\qquad
		\bar{x}_2 = \frac{19 + 21 + 23 + 20}{4} = 20.75
		\]
		Second, compute the sample variances and sample standard deviations.
		\[
		s_1^2 = \frac{(18-20)^2 + (22-20)^2 + (20-20)^2}{3-1} = \frac{8}{2} = 4.00,
		\qquad s_1 = 2.00
		\]
		\[
		s_2^2 = \frac{(19-20.75)^2 + (21-20.75)^2 + (23-20.75)^2 + (20-20.75)^2}{4-1} \approx 2.92,
		\qquad s_2 \approx 1.71
		\]
		
		\subsection*{C5}
		Use the known population standard deviation to build confidence intervals at two different confidence levels.
		
		First confidence interval uses 95\% confidence.
		With known $\sigma = 3.6$ and 95\% confidence, $z_{0.025}=1.96$.
		\[
		SE_1 = \frac{3.6}{\sqrt{3}} \approx 2.08, \qquad E_{1,95} = 1.96(2.08) \approx 4.07
		\]
		\[
		\mu \in \bar{x}_1 \pm E_{1,95} \Rightarrow 20.00 \pm 4.07 = [15.93,\ 24.07]
		\]
		\[
		SE_2 = \frac{3.6}{\sqrt{4}} = 1.80, \qquad E_{2,95} = 1.96(1.80) \approx 3.53
		\]
		\[
		\mu \in \bar{x}_2 \pm E_{2,95} \Rightarrow 20.75 \pm 3.53 = [17.22,\ 24.28]
		\]
		
		Second confidence interval uses 90\% confidence.
		With known $\sigma = 3.6$ and 90\% confidence, $z_{0.05}=1.645$.
		\[
		E_{1,90} = 1.645(2.08) \approx 3.42
		\]
		\[
		\mu \in \bar{x}_1 \pm E_{1,90} \Rightarrow 20.00 \pm 3.42 = [16.58,\ 23.42]
		\]
		\[
		E_{2,90} = 1.645(1.80) \approx 2.96
		\]
		\[
		\mu \in \bar{x}_2 \pm E_{2,90} \Rightarrow 20.75 \pm 2.96 = [17.79,\ 23.71]
		\]
		
		Interpretation.
		The 90\% confidence intervals are narrower than the 95\% confidence intervals.
		Both confidence levels provide plausible ranges for the true mean checkout time at each kiosk.
		
		\subsection*{C6}
		This subsection establishes the required sample size so the margin of error does not exceed the target amount.
		
		For 95\% confidence, the target margin of error is $E_{95} = 2.0$ minutes and $z^*_{95} = 1.96$.
		\[
		n_{\text{required,95}} = \left(\frac{z^*_{95}\,\sigma}{E_{95}}\right)^2
		= \left(\frac{1.96 \cdot 3.6}{2.0}\right)^2
		= \left(3.53\right)^2
		\approx 12.44
		\]
		Round up to $n_{\text{required,95}} = 13$.
		Additional observations needed.
		Kiosk 1 needs $13 - 3 = 10$ more observations, and Kiosk 2 needs $13 - 4 = 9$ more observations.
		
		For 90\% confidence, the target margin of error is $E_{90} = 2.0$ minutes and $z^*_{90} = 1.645$.
		\[
		n_{\text{required,90}} = \left(\frac{z^*_{90}\,\sigma}{E_{90}}\right)^2
		= \left(\frac{1.645 \cdot 3.6}{2.0}\right)^2
		\approx 8.77
		\]
		Round up to $n_{\text{required,90}} = 9$.
		Additional observations needed.
		Kiosk 1 needs $9 - 3 = 6$ more observations, and Kiosk 2 needs $9 - 4 = 5$ more observations.
		
		Conclusion.
		For 95\% confidence, each kiosk should collect enough data to reach a total of 13 observations to ensure the margin of error is at most $E_{95} = 2.0$ minutes.
		For 90\% confidence, each kiosk should collect enough data to reach a total of 9 observations to ensure the margin of error is at most $E_{90} = 2.0$ minutes.
		
		\newpage
		\item
		\subsection*{Problem description}
		A public finance office studies the proportion of monthly utility reimbursements that are flagged for expedited approval for two district teams.
		The two samples are distinct random samples with different sizes drawn from the same population of reimbursements.
		
		\textbf{Sample A (40 reimbursements):}
		In this sample, $x_A = 24$ reimbursements are flagged for expedited approval.
		
		\textbf{Sample B (28 reimbursements):}
		In this sample, $x_B = 18$ reimbursements are flagged for expedited approval.
		
		Construct 95\% confidence intervals for the population proportion of expedited reimbursements using each sample.
		At the end, determine how many additional observations are required in each sample group to reach a margin of error target of $E = 0.05$ for the proportion estimate using each sample estimated proportion.
		
		\subsection*{C1}
		The estimator used in this problem is the sample proportion.
		In mean-based notation, $\bar{x}$ corresponds to the proportion estimator $\hat{p}$.
		The variability of $\hat{p}$ is based on the Bernoulli variance.
		
		For a population proportion $p$ and sample size $n$,
		\[
		\mathrm{Var}(\hat{p}) = \frac{p(1-p)}{n}
		\qquad
		\widehat{\mathrm{Var}}(\hat{p}) = \frac{\hat{p}(1-\hat{p})}{n}
		\]
		and the standard error of the estimator is
		\[
		\mathrm{SE}(\hat{p}) = \sqrt{\frac{\hat{p}(1-\hat{p})}{n}}.
		\]
		
		\textbf{Sample A:}
		Step 1.
		Compute the sample proportion.
		\[
		\hat{p}_A = \frac{x_A}{n_A} = \frac{24}{40} = 0.60
		\]
		Step 2.
		Compute the estimated variance of $\hat{p}_A$.
		\[
		\widehat{\mathrm{Var}}(\hat{p}_A) = \frac{\hat{p}_A(1-\hat{p}_A)}{n_A}
		= \frac{0.60(1-0.60)}{40}
		= \frac{0.60(0.40)}{40}
		= \frac{0.24}{40}
		= 0.006
		\]
		Step 3.
		Compute the standard error of $\hat{p}_A$.
		\[
		\mathrm{SE}(\hat{p}_A) = \sqrt{\widehat{\mathrm{Var}}(\hat{p}_A)}
		= \sqrt{0.006}
		\approx 0.0775
		\]
		
		\textbf{Sample B:}
		Step 1.
		Compute the sample proportion.
		\[
		\hat{p}_B = \frac{x_B}{n_B} = \frac{18}{28} \approx 0.64
		\]
		Step 2.
		Compute the estimated variance of $\hat{p}_B$.
		\[
		\widehat{\mathrm{Var}}(\hat{p}_B) = \frac{\hat{p}_B(1-\hat{p}_B)}{n_B}
		= \frac{0.64(1-0.64)}{28}
		= \frac{0.64(0.36)}{28}
		= \frac{0.2304}{28}
		\approx 0.00823
		\]
		Step 3.
		Compute the standard error of $\hat{p}_B$.
		\[
		\mathrm{SE}(\hat{p}_B) = \sqrt{\widehat{\mathrm{Var}}(\hat{p}_B)}
		= \sqrt{0.00823}
		\approx 0.0907
		\]
		
		\subsection*{C5 and C7}
		The parameter of interest is a population proportion.
		The estimator value $\hat{p}$ and its standard error are identified before constructing the confidence interval.
		
		\textbf{C5 and C7:} Construct 95\% confidence intervals for the proportion of expedited reimbursements.
		
		\textbf{Sample A:}
		Step 1.
		Check the large-sample conditions.
		\[
		n_A\hat{p}_A = 40(0.60) = 24 \ge 10
		\qquad
		n_A(1-\hat{p}_A) = 40(0.40) = 16 \ge 10
		\]
		Step 2.
		Compute the margin of error.
		\[
		E_{p_A} = 1.96\,\mathrm{SE}(\hat{p}_A)
		= 1.96(0.0775)
		\approx 0.1518
		\]
		Step 3.
		Construct the confidence interval.
		\[
		p \in \hat{p}_A \pm E_{p_A}
		= 0.60 \pm 0.1518
		\Rightarrow [0.60-0.1518,\ 0.60+0.1518]
		\Rightarrow [0.4482,\ 0.7518]
		\Rightarrow [0.448,\ 0.752]
		\]
		
		\textbf{Sample B:}
		Step 1.
		Check the large-sample conditions.
		\[
		n_B\hat{p}_B = 28(0.64) = 17.92 \ge 10
		\qquad
		n_B(1-\hat{p}_B) = 28(0.36) = 10.08 \ge 10
		\]
		Step 2.
		Compute the margin of error.
		\[
		E_{p_B} = 1.96\,\mathrm{SE}(\hat{p}_B)
		= 1.96(0.0907)
		\approx 0.1778
		\]
		Step 3.
		Construct the confidence interval.
		\[
		p \in \hat{p}_B \pm E_{p_B}
		\approx 0.64 \pm 0.1778
		\Rightarrow [0.64-0.1778,\ 0.64+0.1778]
		\Rightarrow [0.4622,\ 0.8178]
		\Rightarrow [0.462,\ 0.818]
		\]
		
		Interpretation.
		The confidence intervals suggest that the expedited approval proportion is plausibly between about 0.45 and 0.75 in Sample A and between about 0.46 and 0.82 in Sample B.
		
		\subsection*{C6}
		This subsection establishes the required sample size so the margin of error for the proportion does not exceed the target amount.
		
		Step 1.
		Use the target error and confidence level.
		The target margin of error is $E = 0.05$ and for 95\% confidence, $z^* = 1.96$.
		
		Step 2.
		Compute the required sample size using each sample estimated proportion.
		
		\textbf{Using Sample A estimated proportion:}
		Step 2.1.
		Compute $\hat{p}_A(1-\hat{p}_A)$.
		\[
		\hat{p}_A(1-\hat{p}_A) = 0.60(1-0.60) = 0.60(0.40) = 0.24
		\]
		Step 2.2.
		Compute $(z^*)^2$.
		\[
		(z^*)^2 = (1.96)^2 = 3.8416
		\]
		Step 2.3.
		Compute $E^2$.
		\[
		E^2 = (0.05)^2 = 0.0025
		\]
		Step 2.4.
		Compute the required sample size.
		\[
		n_{\text{required},A} = \frac{(z^*)^2\,\hat{p}_A(1-\hat{p}_A)}{E^2}
		= \frac{3.8416 \cdot 0.24}{0.0025}
		= \frac{0.921984}{0.0025}
		= 368.7936
		\]
		Round up to $n_{\text{required},A} = 369$.
		Additional observations needed.
		Sample A needs $369 - 40 = 329$ more observations.
		
		\textbf{Using Sample B estimated proportion:}
		Step 2.1.
		Compute $\hat{p}_B(1-\hat{p}_B)$.
		\[
		\hat{p}_B(1-\hat{p}_B) = 0.64(1-0.64) = 0.64(0.36) = 0.2304
		\]
		Step 2.2.
		Compute $(z^*)^2$.
		\[
		(z^*)^2 = (1.96)^2 = 3.8416
		\]
		Step 2.3.
		Compute $E^2$.
		\[
		E^2 = (0.05)^2 = 0.0025
		\]
		Step 2.4.
		Compute the required sample size.
		\[
		n_{\text{required},B} = \frac{(z^*)^2\,\hat{p}_B(1-\hat{p}_B)}{E^2}
		= \frac{3.8416 \cdot 0.2304}{0.0025}
		= \frac{0.88489984}{0.0025}
		= 353.959936
		\]
		Round up to $n_{\text{required},B} = 354$.
		Additional observations needed.
		Sample B needs $354 - 28 = 326$ more observations.
		
		Conclusion.
		Using the Sample A estimate, the team should collect enough data to reach a total of 369 reimbursements to ensure the margin of error is at most 0.05.
		Using the Sample B estimate, the team should collect enough data to reach a total of 354 reimbursements to ensure the margin of error is at most 0.05.
		
		\newpage
		\item
		\subsection*{Problem description}
		A municipal purchasing office monitors invoice amounts. From long-term records covering all municipal purchases, invoices are known to have a typical average value of 430 USD and a typical spread of 55 USD around that average. These values summarize how invoice amounts behave across the full purchasing process over time.
		
		To assess current activity, two independent audits are conducted using recent invoices. In Sample A, the invoices reviewed produce a 90\% confidence interval from 402 USD to 438 USD, and the variability within this audit is summarized by a variance of 2500 (USD)$^2$. In Sample B, the invoices reviewed produce a 90\% confidence interval from 420 USD to 460 USD, and the variability within this audit is summarized by a variance of 3600 (USD)$^2$.
		
		\subsection*{C2}
		The value 430 USD represents the typical invoice amount across all municipal purchases, and 55 USD represents the typical spread of invoice amounts over time.
		
		For Sample A, the observed average invoice amount is obtained as the midpoint of the confidence interval:
		\[
		\frac{402 + 438}{2} = 420 \text{ USD}.
		\]
		For Sample B, the observed average invoice amount is obtained as the midpoint of its confidence interval:
		\[
		\frac{420 + 460}{2} = 440 \text{ USD}.
		\]
		These observed averages (420 USD for Sample A and 440 USD for Sample B) can be directly compared to the long-run typical invoice amount of 430 USD.
		
		The variability within each audit is summarized by variances of 2500 (USD)$^2$ for Sample A and 3600 (USD)$^2$ for Sample B. Taking square roots gives observed spreads of
		\[
		\sqrt{2500} = 50 \text{ USD} \quad \text{and} \quad \sqrt{3600} = 60 \text{ USD}.
		\]
		These observed spreads (50 USD and 60 USD) are compared to the long-run spread of 55 USD.
		
		\subsection*{C3}
		Point estimation approximates a characteristic of the overall invoice behavior using a single numerical value computed from audit data. In this problem, the point estimates of the typical invoice amount are 420 USD for Sample A and 440 USD for Sample B, obtained as the midpoints of their respective confidence intervals. Interval estimation, in contrast, uses a range of values to estimate that same characteristic, explicitly accounting for uncertainty due to limited data. The intervals [402, 438] and [420, 460] show how far the point estimates may reasonably vary when different invoices are audited, providing information about precision that a single number alone cannot convey.
		
		\subsection*{C4}
		In this problem, the confidence interval from 402 USD to 438 USD for Sample A means that, based on the invoices actually reviewed in that audit, values of the typical invoice amount below 402 USD or above 438 USD are not consistent with the observed data at the chosen confidence level. Similarly, the interval from 420 USD to 460 USD for Sample B means that the invoices reviewed in that audit rule out typical invoice amounts outside this range. In practical terms, these intervals indicate which values of the typical invoice amount remain plausible after accounting for the observed audit data and which values are unlikely, allowing the purchasing office to judge whether current invoice behavior is in line with the benchmark value of 430 USD.
		
		\newpage
		\item
		\subsection*{Problem description}
		An agricultural finance board compares average loan sizes issued by two different cooperative systems: rural cooperatives and urban cooperatives. From long-term financial records covering all loans issued in each system, rural cooperatives are known to have a typical loan size of 310 USD with a usual spread of 40 USD, while urban cooperatives are known to have a typical loan size of 350 USD with a usual spread of 55 USD. These values describe how loan sizes behave over time within each cooperative system.
		
		To assess current lending activity, analysts examine recent loans from each system. Using the loans reviewed in each case, they construct confidence intervals for the typical loan size. The rural sample produces a 95\% confidence interval from 295 USD to 325 USD, and the urban sample produces a 95\% confidence interval from 330 USD to 370 USD.
		
		\subsection*{C2}
		The values 310 USD and 40 USD describe the long-run behavior of loan sizes in rural cooperatives, while 350 USD and 55 USD describe the long-run behavior of loan sizes in urban cooperatives.
		
		For the rural sample, the observed average loan size is obtained as the midpoint of the confidence interval:
		\[
		\frac{295 + 325}{2} = 310 \text{ USD}.
		\]
		For the urban sample, the observed average loan size is obtained as the midpoint of its confidence interval:
		\[
		\frac{330 + 370}{2} = 350 \text{ USD}.
		\]
		These observed averages are directly compared to the long-run typical loan sizes of 310 USD for rural cooperatives and 350 USD for urban cooperatives.
		
		\subsection*{C3}
		Point estimation uses a single numerical value from the data to approximate the typical loan size within each cooperative system. In this problem, the point estimate for rural cooperatives is 310 USD and the point estimate for urban cooperatives is 350 USD, obtained as the midpoints of their respective confidence intervals. Interval estimation uses ranges of values, namely [295, 325] for rural cooperatives and [330, 370] for urban cooperatives, to account for uncertainty due to limited data and to show how precise each point estimate is.
		
		\subsection*{C4}
		In this context, the confidence interval from 295 USD to 325 USD means that rural loan sizes below 295 USD or above 325 USD are not consistent with the observed rural loan data at the chosen confidence level. Similarly, the interval from 330 USD to 370 USD means that urban loan sizes outside this range are not supported by the observed urban loan data. Because the two intervals do not overlap, the audit results indicate that typical loan sizes in urban cooperatives are higher than those in rural cooperatives when accounting for uncertainty in the data.
		
		\newpage
		\item
		\subsection*{Problem description}
		A financial regulator studies the share of households using a new mobile savings platform. From prior nationwide records, the regulator has an established benchmark indicating that about 55\% of households use the platform, with a corresponding long-run variability of approximately 0.2475. These values summarize how platform adoption behaves across the full population over time.
		
		To assess current usage, recent household surveys are conducted. Based on the data collected, analysts report a 90\% confidence interval for the share of households using the platform that ranges from 0.49 to 0.61.
		
		Explain which values describe long-run adoption behavior versus those obtained from the survey data, distinguish between single-value and range-based estimation for proportions, and interpret the reported interval in context.
		
		\subsection*{C2}
		The value 0.55 represents the typical share of households using the platform across the full population, and 0.2475 represents the long-run variability in platform usage.
		
		The observed share of households using the platform in the survey is obtained as the midpoint of the confidence interval:
		\[
		\frac{0.49 + 0.61}{2} = 0.55.
		\]
		This observed value can be directly compared to the benchmark share of 0.55.
		
		The variability implied by the survey data is obtained from the observed share, giving
		\[
		0.55(1 - 0.55) = 0.2475.
		\]
		Taking square roots yields an observed spread of approximately
		\[
		\sqrt{0.2475} \approx 0.50,
		\]
		which matches the long-run spread implied by the benchmark. This shows that both the central value and the variability observed in the survey align with prior population behavior.
		
		\subsection*{C3}
		Point estimation uses a single numerical value from the survey to approximate the typical share of households using the platform. In this case, the point estimate is 0.55, obtained as the midpoint of the confidence interval. Interval estimation uses a range of values, here [0.49, 0.61], to account for uncertainty due to limited survey data and to show how precise the point estimate is.
		
		\subsection*{C4}
		In this context, the confidence interval from 0.49 to 0.61 means that, based on the households actually surveyed, platform usage levels below 49\% or above 61\% are not consistent with the observed data at the chosen confidence level. The interval identifies which adoption rates remain plausible given the survey evidence and shows that the data are compatible with the benchmark value of about 55\%, while allowing for moderate variation due to sampling.
		
	\end{ExamProblems}
\end{document}
