\makeatletter
\def\input@path{{./}{../}{../../}{preamble/}{../preamble/}{../../preamble/}}
\makeatother
% ----------------------------------------------------------
% GENERAL 

% File
\documentclass[11pt]{book}

% Margins
\usepackage[margin=1in]{geometry}

\linespread{1.2}            % Line spacing
\usepackage[utf8]{inputenc}
\usepackage[T1]{fontenc}
\usepackage{lmodern}
\usepackage{microtype}
\setlength{\parindent}{0pt}
\setlength{\parskip}{6pt}
\usepackage{booktabs}

% ----------------------------------------------------------
% TABLES
\usepackage{multicol}
\usepackage{longtable} 
\usepackage{array}
\usepackage{booktabs}
\usepackage{tabularx}
\usepackage{multirow}

% ----------------------------------------------------------
% MATHEMATICS
\usepackage{amsmath}
\usepackage{amssymb}
\usepackage{amsfonts}
\usepackage{mathtools}

% ----------------------------------------------------------
% HIDDEN CONTENT
\usepackage{ifthen}
% Define a boolean switch
\newboolean{explicaciones}
% Set the boolean switch to true or false
% Change to true to show the content

% Explanations
\newcommand{\explicacion}[2]{
	\ifthenelse{\boolean{explicaciones}}{#1}{#2}
}
\newcommand{\mostrarExplicaciones}[1]{\setboolean{explicaciones}{#1}}

% ----------------------------------------------------------
% NUMBERING

\usepackage{fancyhdr}
\pagestyle{empty} % Ensures the entire document has no page numbers

\usepackage{tocloft}
\renewcommand{\cftdot}{} % Remove dots for sections, if any
\renewcommand{\cftsecleader}{\cftdotfill{\cftdotsep}} % Remove dots for sections, if any
\cftpagenumbersoff{section} % Removes page numbers from sections
\cftpagenumbersoff{subsection} % Removes page numbers from subsections

% ----------------------------------------------------------
% IMAGES 

% General settings
\usepackage{graphicx}       % Insert images
\usepackage{float}          % Position images
% \usepackage{subfigure}      % Subfigures
\graphicspath{{imgs}}       % Image location
\usepackage{subcaption}     % Subfigures II
\usepackage{verbatim}

% Figures
\usepackage{tikz}
\usetikzlibrary{arrows.meta,positioning,trees}

% Colors
\usepackage{xcolor}     
\definecolor{popUp}{HTML}{666666}
\definecolor{popUpIn}{HTML}{CED9E0}
\definecolor{backgroundC}{HTML}{D0E8F2}
\definecolor{backgroundCC}{HTML}{FFFFFF}
\definecolor{borders}{HTML}{8c120d}
\definecolor{padding}{HTML}{77D0D7}
\definecolor{links}{HTML}{CC6F5F}

% ----------------------------------------------------------
% FRAMES

% General settings
\usepackage{tcolorbox}
\usepackage{adjustbox}          % Adjusted frame  
\setlength{\fboxrule}{3pt}  % Line width
\setlength{\fboxsep}{3pt}   % Box padding

% General frames
\usepackage{mdframed}   

\mdfdefinestyle{estiloGeneral}{    % General style
	linecolor=black,
	linewidth=1.5pt,
	roundcorner=10pt,
	backgroundcolor=backgroundC,
	innerbottommargin=0pt
}
\mdfdefinestyle{code}{          % Code style
	linecolor=black,
	linewidth=1.5pt,
	roundcorner=10pt,
	backgroundcolor=darkgray!10,
	innerbottommargin=0pt
}

% Image frame
\newtcbox{\fboxC}{
	colback=backgroundC,
	colframe=popUp,
	arc=10pt,
	boxrule=3pt,
	boxsep=0pt, % Change the padding here
	nobeforeafter
}

% ----------------------------------------------------------
% PAGE SETTINGS

% Background 
\newcommand{\background}[0]{\begin{tikzpicture}[remember picture,overlay]
		\fill[backgroundC] (-2,2) rectangle (25cm, -550);
\end{tikzpicture}}

\newcommand{\backgroundC}[0]{\begin{tikzpicture}[remember picture,overlay]
		\fill[backgroundCC] (-2,2) rectangle (25cm, -550);
\end{tikzpicture}}

% Page width 
\newcommand{\anchoPag}[0]{20cm}

% ----------------------------------------------------------
% FONT

% General
\usepackage{tgbonum}        % Font
\usepackage{listings}       % Code typesetting
\usepackage[spanish]{babel} % Load Spanish
\selectlanguage{spanish}    % Select Spanish
\usepackage{enumitem}
\usepackage{bookmark}

\setlist[itemize]{leftmargin=1.2em, itemsep=0.35em, topsep=0.35em}

% --- Table helpers ---
\newcolumntype{L}[1]{>{\raggedright\arraybackslash}p{#1}}
\newcolumntype{Y}{>{\raggedright\arraybackslash}X}
\newcolumntype{C}{>{\centering\arraybackslash}X}
\renewcommand{\arraystretch}{1.1}

% Python style
\lstdefinestyle{python}{
	language=Python,
	basicstyle=\ttfamily\small,
	commentstyle=\color{green!50!black},
	keywordstyle=\color{blue},
	numberstyle=\tiny\color{gray},
	numbers=left,
	morekeywords={>, <},
	breakatwhitespace=false,
	showstringspaces=false,
	showtabs=false,
	showspaces=false
}

% ----------------------------------------------------------
% HYPERLINKS

% General
\usepackage{hyperref}       
\hypersetup{
	colorlinks=true,
	linkcolor=links,
	filecolor=magenta,      
	urlcolor=brown,
}

% Custom commands 

% Large link
\newcommand{\bigLink}[2]{\begin{center} \fboxC{\LARGE{\href{#1}{#2}}}\end{center}}

% Small link
\newcommand{\smallLink}[2]{\begin{center}\fboxC{\href{#1}{#2}}\end{center}}

% Bold link
\newcommand{\bfLink}[2]{\href{#1}{\textbf{#2}}}


% Small URL
\newcommand{\smallUrl}[1]{\begin{center}\fboxC{\url{#1}}\end{center}}


% ----------------------------------------------------------
% CUSTOM COMMANDS FOR FIGURES

\newcommand{\espacioImagenes}[0]{-1.2cm}

% Without frame
\newcommand{\fig}[3][\espacioImagenes]{
	\hspace*{#1}
	\centering
	\includegraphics[width=#2\textwidth]{#3}
}

% With frame
\newcommand{\ffig}[2]{\begin{figure}[h]
		\hspace*{\espacioImagenes}
		\centering
		\fbox{\includegraphics[width=#1\textwidth]{#2}}
\end{figure}}

% Hyperlink with frame
\newcommand{\hfig}[3]{\begin{figure}[h]
		\hspace*{-1.4cm}
		\centering
		\color{popUp}
		\fboxC{\href{#1}{\includegraphics[width=#2\textwidth]{#3}}}
	\end{figure}
}

% Hyperlink without frame
\newcommand{\hffig}[3]{\begin{figure}[h]
		\hspace*{-1.1cm}
		\centering
		\color{popUp}
		\href{#1}{\includegraphics[width=#2\textwidth]{#3}}
	\end{figure}
}

% ----------------------------------------------------------

% Start and Contents
\newcommand{\cuadro}[1]{
	\begin{mdframed}[style=estiloGeneral]
		#1 
	\end{mdframed}
}

% Explanation video image
\newcommand{\linkExplicacion}[1]{
	\hffig{#1}{0.5}{principal/videoExplicacion}
	\vspace{-0.5cm}
}

\newcommand{\subSecLink}[2]{
	\subsubsection*{\href{#1}{\textbf{#2}}}
}

% Spacing
\newcommand{\esp}[0]{\vspace{4mm}}

% Back to start
\newcommand{\secInicio}[0]{\begin{center}\hyperref[sec:inicio]{ 
			\includegraphics[width=0.1\textwidth]{principal/up}
	}\end{center}
}


\geometry{margin=0.85in}
\AtBeginDocument{\small}

\newcommand{\ExamNameField}{\noindent\textbf{Name:}\ \rule{0.7\linewidth}{0.4pt}\par\medskip}

\newcommand{\ExamTitleBlock}[3]{%
	\begin{center}
		\Large\textbf{#1}\\
		\textbf{#2}%
		\if\relax\detokenize{#3}\relax\else\\\textbf{#3}\fi
	\end{center}
	\vspace{0.5em}
}

\newcommand{\ExamSection}[1]{\par\medskip\textbf{#1}\par\smallskip}

\newenvironment{ExamCriteria}{%
	\begin{itemize}[leftmargin=1.6em, itemsep=0.3em, topsep=0.2em]
}{%
	\end{itemize}
}

\newenvironment{ExamProblems}{%
	\begin{enumerate}[label=\textbf{P\arabic*}, leftmargin=0pt, labelsep=0.6em, itemindent=2.2em, itemsep=0.8em]
}{%
	\end{enumerate}
}

\begin{document}
	\ExamTitleBlock{11th grade}{Term 2 Learning Evidence 2: C7 Activity Solution}{}
	
\section*{1. Local Transport Satisfaction}

A city transport office wants to estimate the proportion of all city commuters who are satisfied with the reliability of the bus system so it can evaluate service quality. The office randomly surveys \textbf{100 commuters}, and \textbf{58 commuters} in the sample say they are satisfied.

Using this sample information, construct and interpret a \textbf{90\% confidence interval} for the population proportion of all city commuters who are satisfied with the bus system.
	
	\subsubsection*{(C7) Constructs the confidence interval for a proportion}
	To construct the confidence interval for a proportion, we identify the population proportion, compute \(\hat{p}\), verify large-sample conditions, select the correct \(z^*\), compute \(SE\) and the margin of error, construct the interval, and interpret the result in context.
	
	\textbf{Step 1: Identify the population parameter and the sample proportion.}
	Let \(p\) be the population proportion of commuters who are satisfied. The sample proportion is
	\[
	\hat{p} = \frac{x}{n} = \frac{58}{100} = 0.58
	\]
	Large-sample conditions:
	\[
	n\hat{p} = 58 \ge 10, \qquad n(1-\hat{p}) = 42 \ge 10
	\]
	
	\textbf{Step 2: Compute the standard error of the proportion.}
	\[
	SE = \sqrt{\frac{0.58(0.42)}{100}} \approx 0.0494
	\]
	
	\textbf{Step 3: Compute the margin of error (90\% confidence, \(z^* = 1.645\)).}
	\[
	E = 1.645(0.0494) \approx 0.0812
	\]
	
	\textbf{Step 4: Construct the confidence interval.}
	\[
	0.58 \pm 0.0812 \Rightarrow [0.499,\ 0.661]
	\]
	
	\textbf{Step 5: Interpret the confidence interval in context.}
	We are 90\% confident that between about 49.9\% and 66.1\% of all commuters in the city are satisfied with the bus system.
	
\section*{2. Digital Payment Adoption}

A business association is studying the proportion of all small retailers in the region that accept digital payments to understand technology adoption in retail. It surveys \textbf{180 small retailers}, and \textbf{81} of those retailers report that they accept digital payments.

Based on the survey, construct and interpret a \textbf{95\% confidence interval} for the population proportion of small retailers who accept digital payments.
	
	\subsubsection*{(C7) Constructs the confidence interval for a proportion}
	
	\textbf{Step 1: Identify the population parameter and the sample proportion.}
	Let \(p\) be the population proportion of small retailers who accept digital payments.
	\[
	\hat{p} = \frac{81}{180} = 0.45
	\]
	Large-sample conditions:
	\[
	81 \ge 10, \qquad 99 \ge 10
	\]
	
	\textbf{Step 2: Compute the standard error of the proportion.}
	\[
	SE = \sqrt{\frac{0.45(0.55)}{180}} \approx 0.0371
	\]
	
	\textbf{Step 3: Compute the margin of error (95\% confidence, \(z^* = 1.96\)).}
	\[
	E = 1.96(0.0371) \approx 0.0727
	\]
	
	\textbf{Step 4: Construct the confidence interval.}
	\[
	[0.377,\ 0.523]
	\]
	
	\textbf{Step 5: Interpret the confidence interval in context.}
	We are 95\% confident that between about 37.7\% and 52.3\% of all small retailers accept digital payments.
	
\section*{3. Water-Saving Appliance Use}

An environmental agency wants to estimate the proportion of all households in the province that use water-saving appliances in order to plan conservation programs. It surveys \textbf{250 households}, and \textbf{190} of them report using water-saving appliances.

Use the survey results to construct and interpret a \textbf{95\% confidence interval} for the population proportion of households that use water-saving appliances.
	
	\subsubsection*{(C7) Constructs the confidence interval for a proportion}
	
	\textbf{Step 1: Identify the population parameter and the sample proportion.}
	\[
	\hat{p} = \frac{190}{250} = 0.76
	\]
	
	\textbf{Step 2: Compute the standard error.}
	\[
	SE = \sqrt{\frac{0.76(0.24)}{250}} \approx 0.0270
	\]
	
	\textbf{Step 3: Margin of error (95\%, \(z^* = 1.96\)).}
	\[
	E \approx 0.0529
	\]
	
	\textbf{Step 4: Confidence interval.}
	\[
	[0.707,\ 0.813]
	\]
	
	\textbf{Step 5: Interpretation.}
	We are 95\% confident that between about 70.7\% and 81.3\% of households use water-saving appliances.
	
\section*{4. Export Quality Compliance}

A trade ministry monitors export quality and wants to estimate the proportion of all export shipments that fail to meet a required quality standard. In a recent inspection of \textbf{320 export shipments}, \textbf{68} shipments fail the standard. The ministry has publicly stated that \textbf{no more than 20\%} of shipments fail.

Construct a \textbf{98\% confidence interval} for the population proportion of shipments that fail the standard, and use the interval to evaluate whether the ministry’s claim is plausible at the 98\% confidence level.
	
	\subsubsection*{(C7) Constructs the confidence interval for a proportion}
	
	\textbf{Step 1: Sample proportion.}
	\[
	\hat{p} = \frac{68}{320} = 0.2125
	\]
	
	\textbf{Step 2: Standard error.}
	\[
	SE = \sqrt{\frac{0.2125(0.7875)}{320}} \approx 0.0229
	\]
	
	\textbf{Step 3: Margin of error (98\%, \(z^* = 2.326\)).}
	\[
	E \approx 0.0532
	\]
	
	\textbf{Step 4: Confidence interval.}
	\[
	[0.159,\ 0.266]
	\]
	
	\textbf{Step 5: Interpretation and decision.}
	Since 0.20 is contained within the confidence interval, the ministry’s claim that no more than 20\% of shipments fail the standard is plausible at the 98\% confidence level.
	
\section*{5. Financial Literacy Completion}

A national education board wants to estimate the proportion of all students who would pass a financial literacy assessment to judge the effectiveness of its curriculum. The board tests \textbf{500 students}, and \textbf{275} of them pass. The board’s policy goal is that \textbf{at least 60\%} of students should pass.

Construct and interpret a \textbf{99\% confidence interval} for the population proportion of students who pass, and use it to assess whether the policy statement is supported at the 99\% confidence level.
	
	\subsubsection*{(C7) Constructs the confidence interval for a proportion}
	
	\textbf{Step 1: Sample proportion.}
	\[
	\hat{p} = \frac{275}{500} = 0.55
	\]
	
	\textbf{Step 2: Standard error.}
	\[
	SE = \sqrt{\frac{0.55(0.45)}{500}} \approx 0.0222
	\]
	
	\textbf{Step 3: Margin of error (99\%, \(z^* = 2.576\)).}
	\[
	E \approx 0.0573
	\]
	
	\textbf{Step 4: Confidence interval.}
	\[
	[0.493,\ 0.607]
	\]
	
	\textbf{Step 5: Interpretation and decision.}
	Because 0.60 lies within the confidence interval, the data do not contradict the board’s statement at the 99\% confidence level.
	
	\section*{6. Regional Health Insurance Coverage}
	
	A health ministry needs an estimate of the proportion of all adults in the region who have health insurance to guide funding and policy decisions. It surveys \textbf{600 adults}, and \textbf{312} report having health insurance.
	
	Construct and interpret \textbf{90\%, 95\%, and 98\% confidence intervals} for the population proportion of insured adults, and explain how the confidence level affects the precision of the estimate in this policy context.
	
	\subsubsection*{(C7) Constructs the confidence interval for a proportion}
	
	\textbf{Step 1: Sample proportion.}
	\[
	\hat{p} = \frac{312}{600} = 0.52
	\]
	Large-sample conditions:
	\[
	n\hat{p} = 312 \ge 10, \qquad n(1-\hat{p}) = 288 \ge 10
	\]
	
	\textbf{Step 2: Standard error.}
	\[
	SE = \sqrt{\frac{0.52(0.48)}{600}} \approx 0.0204
	\]
	
	\textbf{Step 3: Margins of error.}
	\[
	E_{90} = 1.645(0.0204) \approx 0.0336
	\]
	\[
	E_{95} = 1.96(0.0204) \approx 0.0400
	\]
	\[
	E_{98} = 2.326(0.0204) \approx 0.0474
	\]
	
	\textbf{Step 4: Confidence intervals.}
	\[
	90\%:\ [0.52 \pm 0.0336] \Rightarrow [0.486,\ 0.554]
	\]
	\[
	95\%:\ [0.52 \pm 0.0400] \Rightarrow [0.480,\ 0.560]
	\]
	\[
	98\%:\ [0.52 \pm 0.0474] \Rightarrow [0.473,\ 0.567]
	\]
	
	\textbf{Step 5: Interpretation and justification.}
	As the confidence level increases from 90\% to 98\%, the confidence interval becomes wider. This reflects greater certainty that the interval contains the true population proportion, which is appropriate in a public policy context where underestimating or overestimating health insurance coverage can have significant consequences.

	\section*{7. Food Safety Inspection Compliance}
	
	A national agency inspects \textbf{850 facilities} and finds \textbf{102} noncompliant. A previous inspection one year earlier found \textbf{120 noncompliant facilities out of 900}.
	
	Construct a \textbf{99\% confidence interval} for the current inspection and compare it with the previous one.
	
	\subsubsection*{(C7) Constructs the confidence interval for a proportion}
	
	\textbf{Step 1: Current sample proportion.}
	\[
	\hat{p}_1 = \frac{102}{850} = 0.12
	\]
	
	\textbf{Step 2: Standard error.}
	\[
	SE_1 = \sqrt{\frac{0.12(0.88)}{850}} \approx 0.0111
	\]
	
	\textbf{Step 3: Margin of error (99\%, \(z^* = 2.576\)).}
	\[
	E_1 \approx 0.0287
	\]
	
	\textbf{Step 4: Current confidence interval.}
	\[
	[0.091,\ 0.149]
	\]
	
	\textbf{Step 5: Comparison and interpretation.}
	The previous inspection proportion was \(\hat{p}_0 = 120/900 = 0.133\). Since this value lies within the current confidence interval, there is no clear statistical evidence of a change in noncompliance at the 99\% confidence level.
	
\end{document}
