\documentclass[12pt]{article}

% Page size and tighter margins
\usepackage[a4paper,left=1.2cm,right=1.2cm,top=1.5cm,bottom=1.5cm]{geometry}

% Core packages
\usepackage{graphicx}
\usepackage{xcolor}
\usepackage{array}
\usepackage{tabularx}
\usepackage{multicol}
\usepackage[T1]{fontenc}
\usepackage[utf8]{inputenc}

\setlength{\parindent}{0pt}
\setlength{\tabcolsep}{6pt}
\renewcommand{\arraystretch}{1.15}

% Column types
\newcolumntype{Y}{>{\raggedright\arraybackslash}m{\dimexpr0.30\textwidth-2\tabcolsep-2\arrayrulewidth\relax}}
\newcolumntype{Z}{>{\raggedright\arraybackslash}m{\dimexpr0.70\textwidth-2\tabcolsep-2\arrayrulewidth\relax}}
\newcolumntype{C}[1]{>{\centering\arraybackslash}m{#1}}

% Gray subsection header box
\newcommand{\SubsectionBox}[1]{%
	\noindent\colorbox{gray!30}{%
		\parbox{\linewidth}{\textbf{#1}}%
	}\par\vspace{0.35cm}%
}

% Centered multi-line cell helper
\newcommand{\CellCenter}[1]{%
	\parbox{\linewidth}{\centering #1}%
}

\begin{document}

	% =========================
	% HEADER BOX (3 COLUMNS)
	% =========================
	\noindent
	\begin{tabularx}{\textwidth}{|C{2.8cm}|C{\dimexpr\textwidth-6cm-4\tabcolsep-4\arrayrulewidth\relax}|C{2.8cm}|}
		\hline
		\centering
		\vspace{3mm}
		\includegraphics[width=2.5cm]{../../preamble/logo.png}
		&
		\CellCenter{%
			\vspace{-5mm}
			\textbf{GLOBAL ECONOMICS}\par
			\textbf{GRADE: 11TH}\par
			\textbf{LEARNING EVIDENCE 1}\par
			\textbf{CONFIDENCE INTERVAL PLANNING}\par
			\textbf{TEACHER'S NAME: Nicolás López Cuéllar}
		}
		&
		\CellCenter{%
			\textbf{SECOND TERM}\par
			\textbf{2025--2026}%
		}
		\\
		\hline
	\end{tabularx}

	\vspace{0.5cm}

	% =========================
	% OBJECTIVE + CRITERIA
	% =========================
	\noindent
	\begin{tabular}{|Y|Z|}
		\hline
		{\footnotesize
			\textbf{Learning objective:} Apply statistical inference to estimate population parameters by constructing, interpreting, and justifying confidence intervals and required sample sizes based on sample data.
		}
		&
		{\footnotesize
			\textbf{Assessment criteria:}\par
			C1: Compute the sample mean and sample standard deviation.\par
			C2: Distinguish between sample statistics and population parameters.\par
			C3: Explain why interval estimation is preferred over a single point estimate.\par
			C4: Interpret the meaning of a X\% confidence interval in context.\par
			C5: Construct a X\% confidence interval using known population standard deviation.\par
			C6: Stablishes the sample size so the error will not exceed a specified amount.\par
			C7: Constructs the confidence interval for a proportion.
		}
		\\
		\hline
	\end{tabular}

	\vspace{0.4cm}

	% =========================
	% STUDENT LINE
	% =========================
	\noindent
	\textbf{Student's name:} \rule{7cm}{0.4pt}\hfill
	\textbf{Group:} \rule{2cm}{0.4pt}\hfill
	\textbf{Date:} \rule{3cm}{0.4pt}

	% =========================
	% EXAM BODY
	% =========================
	\begin{multicols}{2}
		\SubsectionBox{Criteria assessment}\vspace{-0.25cm}
		Each assessment criterion 1-6 is evaluated across three problems. A criterion is considered passed when it is correctly activated in at least two of the three problems. Criteria 7 is considered passed if it is correctly activated in problem 3

		\vspace{0.25cm}
		\SubsectionBox{1. (C1,C5,C6) Comparing Mean Fare per Rider}\vspace{-0.25cm}
		A regional transit office studies the average weekday fare collected per rider for two pilot corridors.
		The two samples are distinct random samples drawn from the same population of fare amounts.
		The grouped fare data (USD) are below.

		\begin{multicols}{2}
			\textbf{Sample A (12 riders):}
			\begin{itemize}
				\item 40 USD occurred in 4 riders.
				\item 50 USD occurred in 3 riders.
				\item 65 USD occurred in 5 riders.
			\end{itemize}

			\columnbreak

			\textbf{Sample B (13 riders):}
			\begin{itemize}
				\item 40 USD occurred in 5 riders.
				\item 50 USD occurred in 2 riders.
				\item 65 USD occurred in 6 riders.
			\end{itemize}
		\end{multicols}

		Long-run audits show the population standard deviation is $\sigma = 12$ USD.
		For classroom purposes, suppose the true population mean fare is $\mu = 53$ USD.
		Construct and interpret 95\% confidence intervals and 90\% confidence intervals for the population mean in each corridor.
		At the end, determine how many additional observations are required in each sample group to reach a margin of error target of $E_{95} = 3.5$ USD for 95\% confidence and $E_{90} = 3.5$ USD for 90\% confidence.

		\vspace{0.25cm}
		\SubsectionBox{2. (C1,C5,C6) Comparing Average Checkout Times}\vspace{-0.25cm}
		A campus dining service compares average lunch checkout times (minutes) at two kiosks.
		The two samples are distinct random samples with sizes $n_1 = 3$ and $n_2 = 4$ drawn from the same population of checkout times.

		\textbf{Sample 1:} 18, 22, 20.

		\textbf{Sample 2:} 19, 21, 23, 20.

		Operational logs indicate the population standard deviation is $\sigma = 3.6$ minutes.
		Construct and interpret 95\% confidence intervals and 90\% confidence intervals for the population mean checkout time at each kiosk.
		At the end, determine how many additional observations are required in each kiosk sample to reach a margin of error target of $E_{95} = 2.0$ minutes for 95\% confidence and $E_{90} = 2.0$ minutes for 90\% confidence.

		\vspace{0.25cm}
		\SubsectionBox{3. (C1,C5,C6,C7) Comparing Proportions of Expedited Reimbursements}\vspace{-0.25cm}
		A public finance office studies the proportion of monthly utility reimbursements that are flagged for expedited approval for two district teams.
		The two samples are distinct random samples with different sizes drawn from the same population of reimbursements.

		\textbf{Sample A (40 reimbursements):}
		In this sample, $x_A = 24$ reimbursements are flagged for expedited approval.

		\textbf{Sample B (28 reimbursements):}
		In this sample, $x_B = 18$ reimbursements are flagged for expedited approval.

		Construct 95\% confidence intervals for the population proportion of expedited reimbursements using each sample.
		At the end, determine how many additional observations are required in each sample group to reach a margin of error target of $E = 0.05$ for the proportion estimate using each sample estimated proportion.

		\vspace{0.25cm}
		\SubsectionBox{4. (C2,C3,C4) Estimating Mean Invoice Amount}\vspace{-0.25cm}
		A municipal purchasing office monitors invoice amounts. From long-term records covering all municipal purchases, invoices are known to have a typical average value of 430 USD and a typical spread of 55 USD around that average. These values summarize how invoice amounts behave across the full purchasing process over time.

		To assess current activity, two independent audits are conducted using recent invoices. In Sample A, the invoices reviewed produce a 90\% confidence interval from 402 USD to 438 USD, and the variability within this audit is summarized by a variance of 2500 (USD)$^2$. In Sample B, the invoices reviewed produce a 90\% confidence interval from 420 USD to 460 USD, and the variability within this audit is summarized by a variance of 3600 (USD)$^2$.

		\vspace{0.25cm}
		\SubsectionBox{5. (C2,C3,C4) Comparing Mean Loan Sizes Across Cooperative Systems}\vspace{-0.25cm}
		An agricultural finance board compares average loan sizes issued by two different cooperative systems: rural cooperatives and urban cooperatives. From long-term financial records covering all loans issued in each system, rural cooperatives are known to have a typical loan size of 310 USD with a usual spread of 40 USD, while urban cooperatives are known to have a typical loan size of 350 USD with a usual spread of 55 USD. These values describe how loan sizes behave over time within each cooperative system.

		To assess current lending activity, analysts examine recent loans from each system. Using the loans reviewed in each case, they construct confidence intervals for the typical loan size. The rural sample produces a 95\% confidence interval from 295 USD to 325 USD, and the urban sample produces a 95\% confidence interval from 330 USD to 370 USD.

		\vspace{0.25cm}
		\SubsectionBox{6. (C2,C3,C4) Estimating Population Proportion for Mobile Savings Adoption}\vspace{-0.25cm}
		A financial regulator studies the share of households using a new mobile savings platform. From prior nationwide records, the regulator has an established benchmark indicating that about 55\% of households use the platform, with a corresponding long-run variability of approximately 0.2475. These values summarize how platform adoption behaves across the full population over time.

		To assess current usage, recent household surveys are conducted. Based on the data collected, analysts report a 90\% confidence interval for the share of households using the platform that ranges from 0.49 to 0.61.

		Explain which values describe long-run adoption behavior versus those obtained from the survey data, distinguish between single-value and range-based estimation for proportions, and interpret the reported interval in context.
	\end{multicols}

\end{document}
