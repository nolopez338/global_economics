\makeatletter
\def\input@path{{./}{../}{../../}{preamble/}{../preamble/}{../../preamble/}}
\makeatother
% ----------------------------------------------------------
% GENERAL 

% File
\documentclass[11pt]{book}

% Margins
\usepackage[margin=1in]{geometry}

\linespread{1.2}            % Line spacing
\usepackage[utf8]{inputenc}
\usepackage[T1]{fontenc}
\usepackage{lmodern}
\usepackage{microtype}
\setlength{\parindent}{0pt}
\setlength{\parskip}{6pt}
\usepackage{booktabs}

% ----------------------------------------------------------
% TABLES
\usepackage{multicol}
\usepackage{longtable} 
\usepackage{array}
\usepackage{booktabs}
\usepackage{tabularx}
\usepackage{multirow}

% ----------------------------------------------------------
% MATHEMATICS
\usepackage{amsmath}
\usepackage{amssymb}
\usepackage{amsfonts}
\usepackage{mathtools}

% ----------------------------------------------------------
% HIDDEN CONTENT
\usepackage{ifthen}
% Define a boolean switch
\newboolean{explicaciones}
% Set the boolean switch to true or false
% Change to true to show the content

% Explanations
\newcommand{\explicacion}[2]{
	\ifthenelse{\boolean{explicaciones}}{#1}{#2}
}
\newcommand{\mostrarExplicaciones}[1]{\setboolean{explicaciones}{#1}}

% ----------------------------------------------------------
% NUMBERING

\usepackage{fancyhdr}
\pagestyle{empty} % Ensures the entire document has no page numbers

\usepackage{tocloft}
\renewcommand{\cftdot}{} % Remove dots for sections, if any
\renewcommand{\cftsecleader}{\cftdotfill{\cftdotsep}} % Remove dots for sections, if any
\cftpagenumbersoff{section} % Removes page numbers from sections
\cftpagenumbersoff{subsection} % Removes page numbers from subsections

% ----------------------------------------------------------
% IMAGES 

% General settings
\usepackage{graphicx}       % Insert images
\usepackage{float}          % Position images
% \usepackage{subfigure}      % Subfigures
\graphicspath{{imgs}}       % Image location
\usepackage{subcaption}     % Subfigures II
\usepackage{verbatim}

% Figures
\usepackage{tikz}
\usetikzlibrary{arrows.meta,positioning,trees}

% Colors
\usepackage{xcolor}     
\definecolor{popUp}{HTML}{666666}
\definecolor{popUpIn}{HTML}{CED9E0}
\definecolor{backgroundC}{HTML}{D0E8F2}
\definecolor{backgroundCC}{HTML}{FFFFFF}
\definecolor{borders}{HTML}{8c120d}
\definecolor{padding}{HTML}{77D0D7}
\definecolor{links}{HTML}{CC6F5F}

% ----------------------------------------------------------
% FRAMES

% General settings
\usepackage{tcolorbox}
\usepackage{adjustbox}          % Adjusted frame  
\setlength{\fboxrule}{3pt}  % Line width
\setlength{\fboxsep}{3pt}   % Box padding

% General frames
\usepackage{mdframed}   

\mdfdefinestyle{estiloGeneral}{    % General style
	linecolor=black,
	linewidth=1.5pt,
	roundcorner=10pt,
	backgroundcolor=backgroundC,
	innerbottommargin=0pt
}
\mdfdefinestyle{code}{          % Code style
	linecolor=black,
	linewidth=1.5pt,
	roundcorner=10pt,
	backgroundcolor=darkgray!10,
	innerbottommargin=0pt
}

% Image frame
\newtcbox{\fboxC}{
	colback=backgroundC,
	colframe=popUp,
	arc=10pt,
	boxrule=3pt,
	boxsep=0pt, % Change the padding here
	nobeforeafter
}

% ----------------------------------------------------------
% PAGE SETTINGS

% Background 
\newcommand{\background}[0]{\begin{tikzpicture}[remember picture,overlay]
		\fill[backgroundC] (-2,2) rectangle (25cm, -550);
\end{tikzpicture}}

\newcommand{\backgroundC}[0]{\begin{tikzpicture}[remember picture,overlay]
		\fill[backgroundCC] (-2,2) rectangle (25cm, -550);
\end{tikzpicture}}

% Page width 
\newcommand{\anchoPag}[0]{20cm}

% ----------------------------------------------------------
% FONT

% General
\usepackage{tgbonum}        % Font
\usepackage{listings}       % Code typesetting
\usepackage[spanish]{babel} % Load Spanish
\selectlanguage{spanish}    % Select Spanish
\usepackage{enumitem}
\usepackage{bookmark}

\setlist[itemize]{leftmargin=1.2em, itemsep=0.35em, topsep=0.35em}

% --- Table helpers ---
\newcolumntype{L}[1]{>{\raggedright\arraybackslash}p{#1}}
\newcolumntype{Y}{>{\raggedright\arraybackslash}X}
\newcolumntype{C}{>{\centering\arraybackslash}X}
\renewcommand{\arraystretch}{1.1}

% Python style
\lstdefinestyle{python}{
	language=Python,
	basicstyle=\ttfamily\small,
	commentstyle=\color{green!50!black},
	keywordstyle=\color{blue},
	numberstyle=\tiny\color{gray},
	numbers=left,
	morekeywords={>, <},
	breakatwhitespace=false,
	showstringspaces=false,
	showtabs=false,
	showspaces=false
}

% ----------------------------------------------------------
% HYPERLINKS

% General
\usepackage{hyperref}       
\hypersetup{
	colorlinks=true,
	linkcolor=links,
	filecolor=magenta,      
	urlcolor=brown,
}

% Custom commands 

% Large link
\newcommand{\bigLink}[2]{\begin{center} \fboxC{\LARGE{\href{#1}{#2}}}\end{center}}

% Small link
\newcommand{\smallLink}[2]{\begin{center}\fboxC{\href{#1}{#2}}\end{center}}

% Bold link
\newcommand{\bfLink}[2]{\href{#1}{\textbf{#2}}}


% Small URL
\newcommand{\smallUrl}[1]{\begin{center}\fboxC{\url{#1}}\end{center}}


% ----------------------------------------------------------
% CUSTOM COMMANDS FOR FIGURES

\newcommand{\espacioImagenes}[0]{-1.2cm}

% Without frame
\newcommand{\fig}[3][\espacioImagenes]{
	\hspace*{#1}
	\centering
	\includegraphics[width=#2\textwidth]{#3}
}

% With frame
\newcommand{\ffig}[2]{\begin{figure}[h]
		\hspace*{\espacioImagenes}
		\centering
		\fbox{\includegraphics[width=#1\textwidth]{#2}}
\end{figure}}

% Hyperlink with frame
\newcommand{\hfig}[3]{\begin{figure}[h]
		\hspace*{-1.4cm}
		\centering
		\color{popUp}
		\fboxC{\href{#1}{\includegraphics[width=#2\textwidth]{#3}}}
	\end{figure}
}

% Hyperlink without frame
\newcommand{\hffig}[3]{\begin{figure}[h]
		\hspace*{-1.1cm}
		\centering
		\color{popUp}
		\href{#1}{\includegraphics[width=#2\textwidth]{#3}}
	\end{figure}
}

% ----------------------------------------------------------

% Start and Contents
\newcommand{\cuadro}[1]{
	\begin{mdframed}[style=estiloGeneral]
		#1 
	\end{mdframed}
}

% Explanation video image
\newcommand{\linkExplicacion}[1]{
	\hffig{#1}{0.5}{principal/videoExplicacion}
	\vspace{-0.5cm}
}

\newcommand{\subSecLink}[2]{
	\subsubsection*{\href{#1}{\textbf{#2}}}
}

% Spacing
\newcommand{\esp}[0]{\vspace{4mm}}

% Back to start
\newcommand{\secInicio}[0]{\begin{center}\hyperref[sec:inicio]{ 
			\includegraphics[width=0.1\textwidth]{principal/up}
	}\end{center}
}


\geometry{margin=0.85in}
\AtBeginDocument{\small}

\newcommand{\ExamNameField}{\noindent\textbf{Name:}\ \rule{0.7\linewidth}{0.4pt}\par\medskip}

\newcommand{\ExamTitleBlock}[3]{%
	\begin{center}
		\Large\textbf{#1}\\
		\textbf{#2}%
		\if\relax\detokenize{#3}\relax\else\\\textbf{#3}\fi
	\end{center}
	\vspace{0.5em}
}

\newcommand{\ExamSection}[1]{\par\medskip\textbf{#1}\par\smallskip}

\newenvironment{ExamCriteria}{%
	\begin{itemize}[leftmargin=1.6em, itemsep=0.3em, topsep=0.2em]
}{%
	\end{itemize}
}

\newenvironment{ExamProblems}{%
	\begin{enumerate}[label=\textbf{P\arabic*}, leftmargin=0pt, labelsep=0.6em, itemindent=2.2em, itemsep=0.8em]
}{%
	\end{enumerate}
}

\begin{document}
	\ExamTitleBlock{11th grade}{Practice Activity 2.1: C6 Practice}{}
	
	\ExamSection{Evaluated criteria}
	\begin{ExamCriteria}
		\item C6: Determine the required sample size to achieve a target margin of error.
	\end{ExamCriteria}
	
	\ExamSection{Problems}
	\begin{ExamProblems}
		\item
		\subsection*{Problem description}
		A regional logistics agency wants to estimate the mean time (in minutes) for cargo to clear port inspections. Historical monitoring shows the population standard deviation is $\sigma = 12$ minutes. The agency wants a 95\% confidence estimate with a maximum error of $E = 3$ minutes. Determine the minimum sample size needed.
		
		\subsection*{C6}
		This solution establishes the sample size so the error will not exceed a specified amount by identifying $E$, selecting $z^*$, substituting into the formula, and rounding up to the minimum integer sample size.
		
		Step 1: Identify the target error and confidence level. The maximum error is $E = 3$ minutes and the confidence level is 95\%, so $z^* = 1.96$.
		
		Step 2: Apply the sample size formula.
		\[
		n = \left(\frac{z^*\,\sigma}{E}\right)^2
		= \left(\frac{1.96 \cdot 12}{3}\right)^2
		= \left(7.84\right)^2
		= 61.47
		\]
		
		Step 3: Round up and verify Criterion 6. Because the sample size must ensure the error does not exceed $E$, we round up: $n = 62$. With $n = 62$, the margin of error is at most 3 minutes, satisfying Criterion 6.

		\item
		\subsection*{Problem description}
		A development agency wants to estimate the mean monthly food spending for households in a region. A pilot survey suggests the population standard deviation is about $\sigma \approx 45$ currency units. The agency wants a 90\% confidence estimate with a target interval width of $W = 16$ currency units. The width is defined as $W = \text{Upper} - \text{Lower} = 2E$. Determine the minimum sample size needed.
		
		\subsection*{C6}
		This solution establishes the sample size so the interval width will not exceed a specified amount by converting $W$ to $E$, selecting the correct $z^*$, substituting into the formula, and rounding up to the minimum safe integer.
		
		Step 1: Identify the target width and confidence level. The target width is $W = 16$, so $E = W/2 = 8$. The confidence level is 90\%, so $z^* = 1.645$.
		
		Step 2: Apply the sample size formula.
		\[
		n = \left(\frac{z^*\,\sigma}{E}\right)^2
		= \left(\frac{1.645 \cdot 45}{8}\right)^2
		= \left(9.253\right)^2
		\approx 85.62
		\]
		
		Step 3: Round up and verify Criterion 6. We must round up to keep the width within $W$, so $n = 86$. This minimum sample size guarantees the interval width is no more than 16 currency units.

		\item
		\subsection*{Problem description}
		A shipping cooperative wants to estimate the mean fuel cost per container on a trade route. Company records indicate the population standard deviation is $\sigma = 18$ dollars. The cooperative wants the maximum margin of error to be no more than $E = 5$ dollars. Determine the minimum sample size needed for 90\%, 95\%, and 98\% confidence.
		
		\subsection*{C6}
		This solution calculates the minimum sample size needed to keep the margin of error at or below $E = 5$ dollars for three confidence levels.
		
		Step 1: Identify the confidence levels and $z^*$ values. Use $z^* = 1.645$ for 90\%, $z^* = 1.96$ for 95\%, and $z^* = 2.326$ for 98\% confidence.
		
		Step 2: Apply the sample size formula for each confidence level.
		\[
		\text{90\%: } n = \left(\frac{z^*\,\sigma}{E}\right)^2
		= \left(\frac{1.645 \cdot 18}{5}\right)^2
		= \left(5.922\right)^2
		\approx 35.07
		\]
		\[
		\text{95\%: } n = \left(\frac{z^*\,\sigma}{E}\right)^2
		= \left(\frac{1.96 \cdot 18}{5}\right)^2
		= \left(7.056\right)^2
		\approx 49.79
		\]
		\[
		\text{98\%: } n = \left(\frac{z^*\,\sigma}{E}\right)^2
		= \left(\frac{2.326 \cdot 18}{5}\right)^2
		= \left(8.374\right)^2
		\approx 70.12
		\]
		
		Step 3: Round up to ensure the error does not exceed 5 dollars. The required sample sizes are $n = 36$ for 90\%, $n = 50$ for 95\%, and $n = 71$ for 98\% confidence.

		\item
		\subsection*{Problem description}
		A trade corridor study tracks the mean daily transit time (in minutes) for cross-border trucks. The population standard deviation is known to be $\sigma = 12$ minutes. The planners will use a 95\% confidence level. Determine the minimum sample size needed for each target margin of error: $E = 3$ minutes, $E = 4$ minutes, and $E = 5$ minutes.
		
		\subsection*{C6}
		This solution determines the minimum sample size for a fixed 95\% confidence level with three different target margins of error.
		
		Step 1: Identify the confidence level and $z^*$. For 95\% confidence, $z^* = 1.96$.
		
		Step 2: Apply the sample size formula for each margin of error.
		\[
		\text{For } E = 3: \quad n = \left(\frac{z^*\,\sigma}{E}\right)^2
		= \left(\frac{1.96 \cdot 12}{3}\right)^2
		= \left(7.84\right)^2
		= 61.47
		\]
		\[
		\text{For } E = 4: \quad n = \left(\frac{z^*\,\sigma}{E}\right)^2
		= \left(\frac{1.96 \cdot 12}{4}\right)^2
		= \left(5.88\right)^2
		= 34.57
		\]
		\[
		\text{For } E = 5: \quad n = \left(\frac{z^*\,\sigma}{E}\right)^2
		= \left(\frac{1.96 \cdot 12}{5}\right)^2
		= \left(4.704\right)^2
		= 22.13
		\]
		
		Step 3: Round up to ensure each error target is met. The minimum sample sizes are $n = 62$ for $E = 3$, $n = 35$ for $E = 4$, and $n = 23$ for $E = 5$.

		\item
		\subsection*{Problem description}
		A regional logistics agency is updating its estimate of the mean time (in minutes) for cargo to clear port inspections. Historical monitoring shows the population standard deviation is $\sigma = 10$ minutes. The current study already includes $n_{\text{current}} = 40$ observations, with sample summaries $\bar{x} = 47.8$ minutes and $s = 10.9$ minutes (for context only). Because $\sigma$ is known, planning must use $\sigma$. The agency wants a 95\% confidence estimate with target margin of error $E = 2.5$ minutes. How many additional observations are required?
		
		\subsection*{C6}
		Step 1: Identify $E$ and $z^*$ from the confidence level. The target margin is $E = 2.5$ minutes. For 95\% confidence, $z^* = 1.96$.
		
		Step 2: Compute $n_{\text{required}}$ using the formula.
		\[
		n_{\text{required}} = \left(\frac{z^*\,\sigma}{E}\right)^2
		= \left(\frac{1.96 \cdot 10}{2.5}\right)^2
		= \left(7.84\right)^2
		= 61.47
		\]
		
		Step 3: Round up $n_{\text{required}}$ to the next integer. $n_{\text{required}} = 62$.
		
		Step 4: Compute additional observations. $n_{\text{required}} - n_{\text{current}} = 62 - 40 = 22$.
		
		Conclusion: The agency needs $22$ additional observations.

		\item
		\subsection*{Problem description}
		An economic survey team is refining an estimate of mean monthly household food spending (in currency units). Prior studies show the population standard deviation is $\sigma = 40$ units. The current dataset has $n_{\text{current}} = 95$ households, with sample summaries $\bar{x} = 512$ and $s = 42$ (reported for context only). Because $\sigma$ is known, the planning calculation uses $\sigma$. For 90\% confidence, the team wants the confidence interval width to be at most $W_{\text{target}} = 12$ units. How many additional observations are required?
		
		\subsection*{C6}
		Step 1: Identify the target width and $z^*$ from the confidence level. The target width is $W_{\text{target}} = 12$ units. For 90\% confidence, $z^* = 1.645$.
		
		Step 2: Convert width to the target margin of error.
		\[
		E_{\text{target}} = \frac{W_{\text{target}}}{2}
		= \frac{12}{2}
		= 6
		\]
		
		Step 3: Compute $n_{\text{required}}$ using the sample size formula.
		\[
		n_{\text{required}} = \left(\frac{z^*\,\sigma}{E_{\text{target}}}\right)^2
		= \left(\frac{1.645 \cdot 40}{6}\right)^2
		= \left(10.967\right)^2
		= 120.27
		\]
		
		Step 4: Round up $n_{\text{required}}$ to the next integer. $n_{\text{required}} = 121$.
		
		Step 5: Compute additional observations. $n_{\text{required}} - n_{\text{current}} = 121 - 95 = 26$.
		
		Conclusion: The survey needs $26$ additional household observations to keep the interval width at or below 12 units.

		\item
		\subsection*{Problem description}
		A shipping cooperative is extending a study of mean fuel cost per container on a trade route. Accounting records indicate the population standard deviation is $\sigma = 22$ dollars, and the team has already collected $n_{\text{current}} = 100$ observations. The cooperative wants the maximum margin of error to be no more than $E = 4$ dollars. For 90\%, 95\%, and 98\% confidence, determine how many additional observations are required.
		
		\subsection*{C6}
		Step 1: Identify the fixed target margin, current sample size, and the $z^*$ values. Use $E = 4$ dollars, $n_{\text{current}} = 100$, with $z^* = 1.645$ for 90\%, $z^* = 1.96$ for 95\%, and $z^* = 2.326$ for 98\% confidence.
		
		Step 2: Use the sample size formula to find $n_{\text{required}}$ for each confidence-level case.
		\[
		n_{\text{required}} = \left(\frac{z^*\,\sigma}{E}\right)^2
		\]
		\[
		\text{90\%: } n_{\text{required}} = \left(\frac{1.645 \cdot 22}{4}\right)^2
		= \left(9.048\right)^2
		\approx 81.86 \Rightarrow 82
		\]
		\[
		\text{95\%: } n_{\text{required}} = \left(\frac{1.96 \cdot 22}{4}\right)^2
		= \left(10.78\right)^2
		\approx 116.21 \Rightarrow 117
		\]
		\[
		\text{98\%: } n_{\text{required}} = \left(\frac{2.326 \cdot 22}{4}\right)^2
		= \left(12.793\right)^2
		\approx 163.67 \Rightarrow 164
		\]
		
		Step 3: Compute additional observations for each case.
		\[
		n_{\text{additional}} = n_{\text{required}} - n_{\text{current}}
		\]
		\[
		\text{90\%: } n_{\text{additional}} = 82 - 100 = -18 \Rightarrow 0
		\]
		At 90\% confidence, no additional observations are required.
		\[
		\text{95\%: } n_{\text{additional}} = 117 - 100 = 17
		\]
		\[
		\text{98\%: } n_{\text{additional}} = 164 - 100 = 64
		\]
		Additional observations needed: $0$ (90\%), $17$ (95\%), and $64$ (98\%).

		\item
		\subsection*{Problem description}
		A trade corridor team is updating its estimate of mean daily cross-border transit time (in minutes) for trucks. Historical analysis gives a known population standard deviation of $\sigma = 15$ minutes, and the team has already collected $n_{\text{current}} = 80$ observations. At 95\% confidence, determine how many additional observations are required for each target margin of error: $E_1 = 2.5$ minutes, $E_2 = 3.5$ minutes, and $E_3 = 4.5$ minutes.
		
		\subsection*{C6}
		Step 1: Identify the confidence level, current sample size, and $z^*$. For 95\% confidence, $z^* = 1.96$, with $n_{\text{current}} = 80$.
		
		Step 2: Use the sample size formula to find $n_{\text{required}}$ for each margin-of-error case.
		\[
		n_{\text{required}} = \left(\frac{z^*\,\sigma}{E}\right)^2
		\]
		\[
		E_1 = 2.5: \quad n_{\text{required}} = \left(\frac{1.96 \cdot 15}{2.5}\right)^2
		= \left(11.76\right)^2
		\approx 138.30 \Rightarrow 139
		\]
		\[
		E_2 = 3.5: \quad n_{\text{required}} = \left(\frac{1.96 \cdot 15}{3.5}\right)^2
		= \left(8.4\right)^2
		= 70.56 \Rightarrow 71
		\]
		\[
		E_3 = 4.5: \quad n_{\text{required}} = \left(\frac{1.96 \cdot 15}{4.5}\right)^2
		= \left(6.533\right)^2
		\approx 42.68 \Rightarrow 43
		\]
		
		Step 3: Compute additional observations for each margin target.
		\[
		n_{\text{additional}} = n_{\text{required}} - n_{\text{current}}
		\]
		\[
		E_1: \quad n_{\text{additional}} = 139 - 80 = 59
		\]
		\[
		E_2: \quad n_{\text{additional}} = 71 - 80 = -9 \Rightarrow 0
		\]
		For $E_2$, no additional observations are required.
		\[
		E_3: \quad n_{\text{additional}} = 43 - 80 = -37 \Rightarrow 0
		\]
		For $E_3$, no additional observations are required.
		Additional observations needed: $59$ for $E_1 = 2.5$, $0$ for $E_2 = 3.5$, and $0$ for $E_3 = 4.5$.
		
	\end{ExamProblems}
\end{document}
