\makeatletter
\def\input@path{{./}{../}{../../}{preamble/}{../preamble/}{../../preamble/}}
\makeatother
% ----------------------------------------------------------
% GENERAL 

% File
\documentclass[11pt]{book}

% Margins
\usepackage[margin=1in]{geometry}

\linespread{1.2}            % Line spacing
\usepackage[utf8]{inputenc}
\usepackage[T1]{fontenc}
\usepackage{lmodern}
\usepackage{microtype}
\setlength{\parindent}{0pt}
\setlength{\parskip}{6pt}
\usepackage{booktabs}

% ----------------------------------------------------------
% TABLES
\usepackage{multicol}
\usepackage{longtable} 
\usepackage{array}
\usepackage{booktabs}
\usepackage{tabularx}
\usepackage{multirow}

% ----------------------------------------------------------
% MATHEMATICS
\usepackage{amsmath}
\usepackage{amssymb}
\usepackage{amsfonts}
\usepackage{mathtools}

% ----------------------------------------------------------
% HIDDEN CONTENT
\usepackage{ifthen}
% Define a boolean switch
\newboolean{explicaciones}
% Set the boolean switch to true or false
% Change to true to show the content

% Explanations
\newcommand{\explicacion}[2]{
	\ifthenelse{\boolean{explicaciones}}{#1}{#2}
}
\newcommand{\mostrarExplicaciones}[1]{\setboolean{explicaciones}{#1}}

% ----------------------------------------------------------
% NUMBERING

\usepackage{fancyhdr}
\pagestyle{empty} % Ensures the entire document has no page numbers

\usepackage{tocloft}
\renewcommand{\cftdot}{} % Remove dots for sections, if any
\renewcommand{\cftsecleader}{\cftdotfill{\cftdotsep}} % Remove dots for sections, if any
\cftpagenumbersoff{section} % Removes page numbers from sections
\cftpagenumbersoff{subsection} % Removes page numbers from subsections

% ----------------------------------------------------------
% IMAGES 

% General settings
\usepackage{graphicx}       % Insert images
\usepackage{float}          % Position images
% \usepackage{subfigure}      % Subfigures
\graphicspath{{imgs}}       % Image location
\usepackage{subcaption}     % Subfigures II
\usepackage{verbatim}

% Figures
\usepackage{tikz}
\usetikzlibrary{arrows.meta,positioning,trees}

% Colors
\usepackage{xcolor}     
\definecolor{popUp}{HTML}{666666}
\definecolor{popUpIn}{HTML}{CED9E0}
\definecolor{backgroundC}{HTML}{D0E8F2}
\definecolor{backgroundCC}{HTML}{FFFFFF}
\definecolor{borders}{HTML}{8c120d}
\definecolor{padding}{HTML}{77D0D7}
\definecolor{links}{HTML}{CC6F5F}

% ----------------------------------------------------------
% FRAMES

% General settings
\usepackage{tcolorbox}
\usepackage{adjustbox}          % Adjusted frame  
\setlength{\fboxrule}{3pt}  % Line width
\setlength{\fboxsep}{3pt}   % Box padding

% General frames
\usepackage{mdframed}   

\mdfdefinestyle{estiloGeneral}{    % General style
	linecolor=black,
	linewidth=1.5pt,
	roundcorner=10pt,
	backgroundcolor=backgroundC,
	innerbottommargin=0pt
}
\mdfdefinestyle{code}{          % Code style
	linecolor=black,
	linewidth=1.5pt,
	roundcorner=10pt,
	backgroundcolor=darkgray!10,
	innerbottommargin=0pt
}

% Image frame
\newtcbox{\fboxC}{
	colback=backgroundC,
	colframe=popUp,
	arc=10pt,
	boxrule=3pt,
	boxsep=0pt, % Change the padding here
	nobeforeafter
}

% ----------------------------------------------------------
% PAGE SETTINGS

% Background 
\newcommand{\background}[0]{\begin{tikzpicture}[remember picture,overlay]
		\fill[backgroundC] (-2,2) rectangle (25cm, -550);
\end{tikzpicture}}

\newcommand{\backgroundC}[0]{\begin{tikzpicture}[remember picture,overlay]
		\fill[backgroundCC] (-2,2) rectangle (25cm, -550);
\end{tikzpicture}}

% Page width 
\newcommand{\anchoPag}[0]{20cm}

% ----------------------------------------------------------
% FONT

% General
\usepackage{tgbonum}        % Font
\usepackage{listings}       % Code typesetting
\usepackage[spanish]{babel} % Load Spanish
\selectlanguage{spanish}    % Select Spanish
\usepackage{enumitem}
\usepackage{bookmark}

\setlist[itemize]{leftmargin=1.2em, itemsep=0.35em, topsep=0.35em}

% --- Table helpers ---
\newcolumntype{L}[1]{>{\raggedright\arraybackslash}p{#1}}
\newcolumntype{Y}{>{\raggedright\arraybackslash}X}
\newcolumntype{C}{>{\centering\arraybackslash}X}
\renewcommand{\arraystretch}{1.1}

% Python style
\lstdefinestyle{python}{
	language=Python,
	basicstyle=\ttfamily\small,
	commentstyle=\color{green!50!black},
	keywordstyle=\color{blue},
	numberstyle=\tiny\color{gray},
	numbers=left,
	morekeywords={>, <},
	breakatwhitespace=false,
	showstringspaces=false,
	showtabs=false,
	showspaces=false
}

% ----------------------------------------------------------
% HYPERLINKS

% General
\usepackage{hyperref}       
\hypersetup{
	colorlinks=true,
	linkcolor=links,
	filecolor=magenta,      
	urlcolor=brown,
}

% Custom commands 

% Large link
\newcommand{\bigLink}[2]{\begin{center} \fboxC{\LARGE{\href{#1}{#2}}}\end{center}}

% Small link
\newcommand{\smallLink}[2]{\begin{center}\fboxC{\href{#1}{#2}}\end{center}}

% Bold link
\newcommand{\bfLink}[2]{\href{#1}{\textbf{#2}}}


% Small URL
\newcommand{\smallUrl}[1]{\begin{center}\fboxC{\url{#1}}\end{center}}


% ----------------------------------------------------------
% CUSTOM COMMANDS FOR FIGURES

\newcommand{\espacioImagenes}[0]{-1.2cm}

% Without frame
\newcommand{\fig}[3][\espacioImagenes]{
	\hspace*{#1}
	\centering
	\includegraphics[width=#2\textwidth]{#3}
}

% With frame
\newcommand{\ffig}[2]{\begin{figure}[h]
		\hspace*{\espacioImagenes}
		\centering
		\fbox{\includegraphics[width=#1\textwidth]{#2}}
\end{figure}}

% Hyperlink with frame
\newcommand{\hfig}[3]{\begin{figure}[h]
		\hspace*{-1.4cm}
		\centering
		\color{popUp}
		\fboxC{\href{#1}{\includegraphics[width=#2\textwidth]{#3}}}
	\end{figure}
}

% Hyperlink without frame
\newcommand{\hffig}[3]{\begin{figure}[h]
		\hspace*{-1.1cm}
		\centering
		\color{popUp}
		\href{#1}{\includegraphics[width=#2\textwidth]{#3}}
	\end{figure}
}

% ----------------------------------------------------------

% Start and Contents
\newcommand{\cuadro}[1]{
	\begin{mdframed}[style=estiloGeneral]
		#1 
	\end{mdframed}
}

% Explanation video image
\newcommand{\linkExplicacion}[1]{
	\hffig{#1}{0.5}{principal/videoExplicacion}
	\vspace{-0.5cm}
}

\newcommand{\subSecLink}[2]{
	\subsubsection*{\href{#1}{\textbf{#2}}}
}

% Spacing
\newcommand{\esp}[0]{\vspace{4mm}}

% Back to start
\newcommand{\secInicio}[0]{\begin{center}\hyperref[sec:inicio]{ 
			\includegraphics[width=0.1\textwidth]{principal/up}
	}\end{center}
}


\geometry{margin=0.85in}
\AtBeginDocument{\small}

\newcommand{\ExamNameField}{\noindent\textbf{Name:}\ \rule{0.7\linewidth}{0.4pt}\par\medskip}

\newcommand{\ExamTitleBlock}[3]{%
	\begin{center}
		\Large\textbf{#1}\\
		\textbf{#2}%
		\if\relax\detokenize{#3}\relax\else\\\textbf{#3}\fi
	\end{center}
	\vspace{0.5em}
}

\newcommand{\ExamSection}[1]{\par\medskip\textbf{#1}\par\smallskip}

\newenvironment{ExamCriteria}{%
	\begin{itemize}[leftmargin=1.6em, itemsep=0.3em, topsep=0.2em]
}{%
	\end{itemize}
}

\newenvironment{ExamProblems}{%
	\begin{enumerate}[label=\textbf{P\arabic*}, leftmargin=0pt, labelsep=0.6em, itemindent=2.2em, itemsep=0.8em]
}{%
	\end{enumerate}
}

\begin{document}
	\ExamTitleBlock{11th grade}{Term 2 Catch-Up: C4 Confidence Interval Interpretation (Solutions)}{}

	\ExamSection{Problems}
	\begin{ExamProblems}
		\item
		\subsection*{Problem 1}
		\subsection*{Downtown Listing Mean Estimate}
		A housing app samples new one-bedroom listings for early-career workers in a downtown area.
		A 95\% confidence interval for the population mean monthly rent is:
		\[
		\mu \in [1180, 1320]\ \text{USD}.
		\]
		A relocation agency must decide whether to market this area as ``typically below 1200 USD per month.''
		Is that claim supported by the interval?

		\subsection*{C4}
		The estimated parameter is the population mean monthly rent for all similar one-bedroom listings in that downtown area.

		A 95\% confidence level means that if the same sampling method were repeated many times, about 95\% of those confidence intervals would capture the true population mean rent.

		The interval bounds suggest a plausible long-run average rent between 1180 USD and 1320 USD.
		In economic terms, the center of the market is not pinned to one value, but the likely average lies in that 140 USD band.

		The claim ``typically below 1200 USD'' is weakly supported at best, because much of the interval is above 1200 USD.
		Since values like 1250 USD or 1300 USD are also plausible population means, the evidence does not justify promoting the area as generally below 1200 USD.

		Decision conclusion: the agency should avoid that marketing claim and present a broader expected rent range instead.

		\newpage
		\item
		\subsection*{Problem 2}
		\subsection*{Designer Income Mean Estimate}
		A freelance platform studies monthly income for beginner graphic designers aged 20--29.
		A 95\% confidence interval for the population mean monthly freelance income is:
		\[
		\mu \in [1720, 2140]\ \text{USD}.
		\]
		A training program advertises that most beginners can expect around 2000 USD per month.
		Is 2000 USD a plausible benchmark for the population mean based on this interval?

		\subsection*{C4}
		The parameter being estimated is the population mean monthly freelance income for beginner graphic designers on this platform.

		With a 95\% confidence level, we interpret the method as reliable in the long run: around 95\% of similarly constructed intervals would include the true population mean.

		The interval from 1720 USD to 2140 USD indicates a realistic range for the market average.
		This means the typical monthly mean income could be in the high 1700s, near 2000, or just above 2100.

		Because 2000 USD is inside the confidence interval, it is a plausible value for the population mean.
		So the advertisement is not contradicted by the data, although the interval also shows uncertainty around that benchmark.

		Decision conclusion: treating 2000 USD as a reasonable planning estimate is acceptable, but advisors should still communicate the full interval.

		\newpage
		\item
		\subsection*{Problem 3}
		\subsection*{Courier Earnings Mean Comparison}
		Two ride-delivery apps estimate the population mean weekly earnings of active couriers.
		App A reports a 95\% confidence interval:
		\[
		\mu_A \in [460, 540]\ \text{USD}.
		\]
		App B reports a 95\% confidence interval:
		\[
		\mu_B \in [520, 610]\ \text{USD}.
		\]
		A student chooses one app and asks: does the evidence suggest App B has a higher population mean weekly earning than App A?

		\subsection*{C4}
		The parameters are two population means: the mean weekly earning of all active couriers in App A and in App B.

		Each interval uses a 95\% confidence level, so each method has long-run coverage near 95\% for its own population mean.

		App A's plausible mean range is 460 to 540 USD, while App B's is 520 to 610 USD.
		The overlap region is 520 to 540 USD, so some common values remain possible for both means.

		Because of overlap, we cannot claim a guaranteed difference from interval evidence alone.
		However, App B's interval is shifted upward overall and has a higher upper and lower bound, which suggests higher typical earnings are more consistent with App B.

		Decision conclusion: a risk-aware choice would lean toward App B for potentially higher earnings, while acknowledging remaining uncertainty.

		\newpage
		\item
		\subsection*{Problem 4}
		\subsection*{Credit Card Full-Payment Rate}
		A personal finance survey estimates the population proportion of young card users who pay their full credit card balance each month.
		A 95\% confidence interval is:
		\[
		p \in [0.41, 0.53].
		\]
		A policy analyst states: ``There is a 95\% chance that exactly 47\% of users pay in full.''
		Explain what is incorrect in that statement.

		\subsection*{C4}
		The estimated parameter is the population proportion of all young card users who pay their balance in full each month.

		The 95\% confidence level refers to the interval-building process across repeated samples, not to the probability of one fixed value like 47\%.
		After data are collected, the true population proportion is fixed; the interval either contains it or does not.

		The bounds 0.41 and 0.53 mean plausible long-run values for the population proportion are between 41\% and 53\%.
		In practical terms, the full-payment habit appears to be around one-half of users, but not precisely known.

		So the sentence ``95\% chance that exactly 47\%'' is a misinterpretation.
		A correct interpretation is that 47\% is one plausible value inside the confidence interval, along with other values from 41\% to 53\%.

		Decision conclusion: financial education planning should target a range near 41\%--53\% full-payment behavior, not a single exact percentage.

		\newpage
		\item
		\subsection*{Problem 5}
		\subsection*{Positive Return Proportion Estimate}
		An investing app tracks first-year users and estimates the population proportion who finish a month with positive return.
		A 95\% confidence interval for that population proportion is:
		\[
		p \in [0.57, 0.69].
		\]
		The company considers the statement: ``At least 70\% of users are profitable in a typical month.''
		Is this statement supported by the confidence interval?

		\subsection*{C4}
		The parameter is the population proportion of first-year users who have a profitable month.

		At a 95\% confidence level, the method is expected to capture the true population proportion in about 95\% of repeated random samples.

		The interval indicates plausible values between 57\% and 69\%.
		Economically, profitability seems to be above one-half but still below a very high threshold.

		The claim ``at least 70\%'' is not supported because 0.70 is outside the interval and above its upper bound of 0.69.
		So the current evidence suggests the true population proportion is likely lower than 70\%.

		Decision conclusion: the company should not advertise a 70\% profitability rate based on this sample evidence.

		\newpage
		\item
		\subsection*{Problem 6}
		\subsection*{Student Loan Repayment Comparison}
		A bank compares average monthly student loan repayment among two groups of recent graduates.
		Group A (public sector jobs) has a 95\% confidence interval:
		\[
		\mu_A \in [240, 290]\ \text{USD}.
		\]
		Group B (private sector jobs) has a 95\% confidence interval:
		\[
		\mu_B \in [305, 360]\ \text{USD}.
		\]
		Should the bank design separate repayment guidance for the two groups?

		\subsection*{C4}
		The estimated parameters are two population means: monthly repayment for all comparable public-sector graduates and for all comparable private-sector graduates.

		Each interval is at the 95\% confidence level, so each has the same long-run interpretation of interval reliability.

		Group A has plausible mean repayment from 240 to 290 USD, while Group B has plausible mean repayment from 305 to 360 USD.
		These intervals do not overlap, showing clear separation between plausible averages.

		Because there is no overlap, the data strongly suggest different population mean repayment levels.
		A single guidance plan could miss the lower-payment profile of Group A and the higher-payment pressure of Group B.

		Decision conclusion: the bank should provide separate repayment guidance tracks for the two groups.

		\newpage
		\item
		\subsection*{Problem 7}
		\subsection*{Employee Cost of Living}
		A startup surveys first-year employees about total monthly cost of living in a major city.
		For the same dataset, analysts report:
		\[
		\mu \in [2320, 2480]\ \text{USD} \quad \text{at 90\% confidence,}
		\]
		\[
		\mu \in [2280, 2520]\ \text{USD} \quad \text{at 95\% confidence.}
		\]
		Management asks which interval should be used for budgeting decisions and why.

		\subsection*{C4}
		The parameter is the population mean monthly cost of living for all first-year employees in that city.

		The 90\% confidence level gives less confidence but a narrower interval, while the 95\% confidence level gives more confidence and a wider interval.
		Both interpretations refer to long-run coverage of repeated interval construction.

		The 90\% interval (2320 to 2480 USD) is tighter and useful for precise planning.
		The 95\% interval (2280 to 2520 USD) is broader and more cautious, capturing a wider set of plausible averages.

		For budgeting, if management wants lower risk of underestimating average living costs, the 95\% interval is better.
		If management prioritizes tighter short-term forecasts and accepts more risk, the 90\% interval may be preferred.

		Decision conclusion: for conservative compensation policy, management should base decisions on the 95\% interval.

		\newpage
		\item
		\subsection*{Problem 8}
		\subsection*{Micro-Shop Revenue Mean}
		An online seller cooperative estimates the population mean weekly revenue for new micro-shops.
		A 95\% confidence interval is:
		\[
		\mu \in [680, 790]\ \text{USD}.
		\]
		A mentor says a benchmark of 800 USD per week is a realistic average target for beginners.
		Is 800 USD plausible as the population mean according to this interval?

		\subsection*{C4}
		The estimated parameter is the population mean weekly revenue across all comparable new micro-shops in the cooperative.

		At a 95\% confidence level, the interval method has about 95\% long-run success in containing the true population mean.

		The lower and upper bounds imply likely average weekly revenue between 680 USD and 790 USD.
		This range reflects realistic market uncertainty but places the center below 800 USD.

		Since 800 USD lies above the upper bound, it is not plausible as the population mean under this confidence interval.
		The benchmark may still be a stretch goal for individuals, but not a supported estimate of the overall average.

		Decision conclusion: planning should use a mean benchmark within the interval, not 800 USD.

		\newpage
		\item
		\subsection*{Problem 9}
		\subsection*{Subscription Spending Mean Estimate}
		A fintech app estimates the population mean monthly subscription spending by users aged 21--29.
		A 95\% confidence interval is:
		\[
		\mu \in [64, 88]\ \text{USD}.
		\]
		A budgeting coach proposes a ``safe cap'' of 90 USD as the typical monthly mean for this age group.
		Based on the interval, should 90 USD be treated as a plausible population mean?

		\subsection*{C4}
		The parameter is the population mean monthly subscription spending for all users in that age segment.

		The confidence level is 95\%, meaning the interval procedure would include the true population mean in about 95\% of repeated samples.

		The interval from 64 to 88 USD describes plausible values for the average spending level in the market.
		It suggests the typical mean is likely in the upper 60s to 80s, not at or above 90.

		Because 90 USD is outside the interval, it should not be treated as a plausible population mean from this evidence.
		Using 90 USD as ``typical'' could overstate expected spending and bias budget advice.

		Decision conclusion: coaches should set typical planning caps within the 64--88 USD interval.

		\newpage
		\item
		\subsection*{Problem 10}
		\subsection*{Branch Commission Mean Comparison}
		A sales startup evaluates the population mean monthly commission for first-year representatives in two branches.
		Branch East reports a 95\% confidence interval:
		\[
		\mu_E \in [910, 1080]\ \text{USD}.
		\]
		Branch West reports a 95\% confidence interval:
		\[
		\mu_W \in [860, 980]\ \text{USD}.
		\]
		Leadership wants to know whether East can be considered clearly stronger in average commission, and whether training incentives should differ.

		\subsection*{C4}
		The parameters are two population means: the mean monthly commission for all first-year representatives in East and in West.

		Both intervals are at 95\% confidence, so each carries the same long-run confidence level interpretation for its branch mean.

		East has plausible mean commission from 910 to 1080 USD, while West has 860 to 980 USD.
		There is overlap from 910 to 980 USD, so a range of common mean values is still plausible.

		Because of overlap, East is not proven to be clearly stronger from interval evidence alone.
		Still, East's interval is shifted upward and includes higher top-end outcomes, suggesting a possible advantage that merits monitoring rather than a final claim.

		Decision conclusion: leadership should use cautious, partially differentiated incentives and collect more data before making strong branch-level performance claims.

	\end{ExamProblems}
\end{document}
