\makeatletter
\def\input@path{{./}{../}{../../}{preamble/}{../preamble/}{../../preamble/}}
\makeatother
% ----------------------------------------------------------
% GENERAL 

% File
\documentclass[11pt]{book}

% Margins
\usepackage[margin=1in]{geometry}

\linespread{1.2}            % Line spacing
\usepackage[utf8]{inputenc}
\usepackage[T1]{fontenc}
\usepackage{lmodern}
\usepackage{microtype}
\setlength{\parindent}{0pt}
\setlength{\parskip}{6pt}
\usepackage{booktabs}

% ----------------------------------------------------------
% TABLES
\usepackage{multicol}
\usepackage{longtable} 
\usepackage{array}
\usepackage{booktabs}
\usepackage{tabularx}
\usepackage{multirow}

% ----------------------------------------------------------
% MATHEMATICS
\usepackage{amsmath}
\usepackage{amssymb}
\usepackage{amsfonts}
\usepackage{mathtools}

% ----------------------------------------------------------
% HIDDEN CONTENT
\usepackage{ifthen}
% Define a boolean switch
\newboolean{explicaciones}
% Set the boolean switch to true or false
% Change to true to show the content

% Explanations
\newcommand{\explicacion}[2]{
	\ifthenelse{\boolean{explicaciones}}{#1}{#2}
}
\newcommand{\mostrarExplicaciones}[1]{\setboolean{explicaciones}{#1}}

% ----------------------------------------------------------
% NUMBERING

\usepackage{fancyhdr}
\pagestyle{empty} % Ensures the entire document has no page numbers

\usepackage{tocloft}
\renewcommand{\cftdot}{} % Remove dots for sections, if any
\renewcommand{\cftsecleader}{\cftdotfill{\cftdotsep}} % Remove dots for sections, if any
\cftpagenumbersoff{section} % Removes page numbers from sections
\cftpagenumbersoff{subsection} % Removes page numbers from subsections

% ----------------------------------------------------------
% IMAGES 

% General settings
\usepackage{graphicx}       % Insert images
\usepackage{float}          % Position images
% \usepackage{subfigure}      % Subfigures
\graphicspath{{imgs}}       % Image location
\usepackage{subcaption}     % Subfigures II
\usepackage{verbatim}

% Figures
\usepackage{tikz}
\usetikzlibrary{arrows.meta,positioning,trees}

% Colors
\usepackage{xcolor}     
\definecolor{popUp}{HTML}{666666}
\definecolor{popUpIn}{HTML}{CED9E0}
\definecolor{backgroundC}{HTML}{D0E8F2}
\definecolor{backgroundCC}{HTML}{FFFFFF}
\definecolor{borders}{HTML}{8c120d}
\definecolor{padding}{HTML}{77D0D7}
\definecolor{links}{HTML}{CC6F5F}

% ----------------------------------------------------------
% FRAMES

% General settings
\usepackage{tcolorbox}
\usepackage{adjustbox}          % Adjusted frame  
\setlength{\fboxrule}{3pt}  % Line width
\setlength{\fboxsep}{3pt}   % Box padding

% General frames
\usepackage{mdframed}   

\mdfdefinestyle{estiloGeneral}{    % General style
	linecolor=black,
	linewidth=1.5pt,
	roundcorner=10pt,
	backgroundcolor=backgroundC,
	innerbottommargin=0pt
}
\mdfdefinestyle{code}{          % Code style
	linecolor=black,
	linewidth=1.5pt,
	roundcorner=10pt,
	backgroundcolor=darkgray!10,
	innerbottommargin=0pt
}

% Image frame
\newtcbox{\fboxC}{
	colback=backgroundC,
	colframe=popUp,
	arc=10pt,
	boxrule=3pt,
	boxsep=0pt, % Change the padding here
	nobeforeafter
}

% ----------------------------------------------------------
% PAGE SETTINGS

% Background 
\newcommand{\background}[0]{\begin{tikzpicture}[remember picture,overlay]
		\fill[backgroundC] (-2,2) rectangle (25cm, -550);
\end{tikzpicture}}

\newcommand{\backgroundC}[0]{\begin{tikzpicture}[remember picture,overlay]
		\fill[backgroundCC] (-2,2) rectangle (25cm, -550);
\end{tikzpicture}}

% Page width 
\newcommand{\anchoPag}[0]{20cm}

% ----------------------------------------------------------
% FONT

% General
\usepackage{tgbonum}        % Font
\usepackage{listings}       % Code typesetting
\usepackage[spanish]{babel} % Load Spanish
\selectlanguage{spanish}    % Select Spanish
\usepackage{enumitem}
\usepackage{bookmark}

\setlist[itemize]{leftmargin=1.2em, itemsep=0.35em, topsep=0.35em}

% --- Table helpers ---
\newcolumntype{L}[1]{>{\raggedright\arraybackslash}p{#1}}
\newcolumntype{Y}{>{\raggedright\arraybackslash}X}
\newcolumntype{C}{>{\centering\arraybackslash}X}
\renewcommand{\arraystretch}{1.1}

% Python style
\lstdefinestyle{python}{
	language=Python,
	basicstyle=\ttfamily\small,
	commentstyle=\color{green!50!black},
	keywordstyle=\color{blue},
	numberstyle=\tiny\color{gray},
	numbers=left,
	morekeywords={>, <},
	breakatwhitespace=false,
	showstringspaces=false,
	showtabs=false,
	showspaces=false
}

% ----------------------------------------------------------
% HYPERLINKS

% General
\usepackage{hyperref}       
\hypersetup{
	colorlinks=true,
	linkcolor=links,
	filecolor=magenta,      
	urlcolor=brown,
}

% Custom commands 

% Large link
\newcommand{\bigLink}[2]{\begin{center} \fboxC{\LARGE{\href{#1}{#2}}}\end{center}}

% Small link
\newcommand{\smallLink}[2]{\begin{center}\fboxC{\href{#1}{#2}}\end{center}}

% Bold link
\newcommand{\bfLink}[2]{\href{#1}{\textbf{#2}}}


% Small URL
\newcommand{\smallUrl}[1]{\begin{center}\fboxC{\url{#1}}\end{center}}


% ----------------------------------------------------------
% CUSTOM COMMANDS FOR FIGURES

\newcommand{\espacioImagenes}[0]{-1.2cm}

% Without frame
\newcommand{\fig}[3][\espacioImagenes]{
	\hspace*{#1}
	\centering
	\includegraphics[width=#2\textwidth]{#3}
}

% With frame
\newcommand{\ffig}[2]{\begin{figure}[h]
		\hspace*{\espacioImagenes}
		\centering
		\fbox{\includegraphics[width=#1\textwidth]{#2}}
\end{figure}}

% Hyperlink with frame
\newcommand{\hfig}[3]{\begin{figure}[h]
		\hspace*{-1.4cm}
		\centering
		\color{popUp}
		\fboxC{\href{#1}{\includegraphics[width=#2\textwidth]{#3}}}
	\end{figure}
}

% Hyperlink without frame
\newcommand{\hffig}[3]{\begin{figure}[h]
		\hspace*{-1.1cm}
		\centering
		\color{popUp}
		\href{#1}{\includegraphics[width=#2\textwidth]{#3}}
	\end{figure}
}

% ----------------------------------------------------------

% Start and Contents
\newcommand{\cuadro}[1]{
	\begin{mdframed}[style=estiloGeneral]
		#1 
	\end{mdframed}
}

% Explanation video image
\newcommand{\linkExplicacion}[1]{
	\hffig{#1}{0.5}{principal/videoExplicacion}
	\vspace{-0.5cm}
}

\newcommand{\subSecLink}[2]{
	\subsubsection*{\href{#1}{\textbf{#2}}}
}

% Spacing
\newcommand{\esp}[0]{\vspace{4mm}}

% Back to start
\newcommand{\secInicio}[0]{\begin{center}\hyperref[sec:inicio]{ 
			\includegraphics[width=0.1\textwidth]{principal/up}
	}\end{center}
}


\geometry{margin=0.85in}
\AtBeginDocument{\small}

\newcommand{\ExamNameField}{\noindent\textbf{Name:}\ \rule{0.7\linewidth}{0.4pt}\par\medskip}

\newcommand{\ExamTitleBlock}[3]{%
	\begin{center}
		\Large\textbf{#1}\\
		\textbf{#2}%
		\if\relax\detokenize{#3}\relax\else\\\textbf{#3}\fi
	\end{center}
	\vspace{0.5em}
}

\newcommand{\ExamSection}[1]{\par\medskip\textbf{#1}\par\smallskip}

\newenvironment{ExamCriteria}{%
	\begin{itemize}[leftmargin=1.6em, itemsep=0.3em, topsep=0.2em]
}{%
	\end{itemize}
}

\newenvironment{ExamProblems}{%
	\begin{enumerate}[label=\textbf{P\arabic*}, leftmargin=0pt, labelsep=0.6em, itemindent=2.2em, itemsep=0.8em]
}{%
	\end{enumerate}
}

\begin{document}
	\ExamTitleBlock{11th grade}{Term 2 Learning Evidence 2: C7 Confidence Intervals for Proportions Practice}{}
	
	\section*{z reference values}
	Use the following critical values for two-sided confidence intervals:
	\begin{itemize}
		\item 90\% \(z^* = 1.645\)
		\item 95\% \(z^* = 1.96\)
		\item 98\% \(z^* = 2.326\)
		\item 99\% \(z^* = 2.576\)
	\end{itemize}
	
	\section*{1. Renewable Energy Subsidy Approval}
	
	A policy team surveys \textbf{200 households} across a region to estimate the share that approve a proposed renewable energy subsidy. In the sample, \textbf{128 households} say they approve.
	
	Construct and interpret a \textbf{90\% confidence interval} for the population proportion of households that approve the subsidy.
	
	\subsubsection*{(C7) Constructs the confidence interval for a proportion}
	To construct the confidence interval for a proportion, we identify \(x\) and \(n\) and compute \(\hat{p}\), check normal approximation conditions \(n\hat{p} \ge 10\) and \(n(1-\hat{p}) \ge 10\), choose the correct \(z^*\) for the confidence level, compute \(SE\) and the margin of error, construct and report the confidence interval, and interpret the interval in context.
	
	\textbf{Step 1: Sample proportion.}
	\[
	\hat{p} = \frac{x}{n} = \frac{128}{200} = 0.64
	\]
	
	\textbf{Step 2: Large-sample conditions.}
	\[
	n\hat{p} = 200(0.64) = 128 \ge 10, \qquad n(1-\hat{p}) = 200(0.36) = 72 \ge 10
	\]
	
	\textbf{Step 3: Standard error.}
	\[
	SE = \sqrt{\frac{\hat{p}(1-\hat{p})}{n}} = \sqrt{\frac{0.64(0.36)}{200}} \approx 0.0339
	\]
	
	\textbf{Step 4: Margin of error (90\% confidence, \(z^* = 1.645\)).}
	\[
	E = z^* \cdot SE = 1.645(0.0339) \approx 0.0558
	\]
	
	\textbf{Step 5: Confidence interval.}
	\[
	\hat{p} \pm E = 0.64 \pm 0.0558 \Rightarrow [0.584,\ 0.696]
	\]
	
	\textbf{Interpretation.}
	We are 90\% confident that between about 58.4\% and 69.6\% of all households in the region approve the renewable energy subsidy.
	
	\section*{2. Export Shipment Delays}
	
	An export council reviews \textbf{150 shipments} and finds that \textbf{39} were delayed at least one day. The council wants a proportion estimate for the delay rate in the current trade season.
	
	Construct and interpret a \textbf{95\% confidence interval} for the population proportion of delayed shipments.
	
	\subsubsection*{(C7) Constructs the confidence interval for a proportion}
	To construct the confidence interval for a proportion, we identify \(x\) and \(n\) and compute \(\hat{p}\), check normal approximation conditions \(n\hat{p} \ge 10\) and \(n(1-\hat{p}) \ge 10\), choose the correct \(z^*\) for the confidence level, compute \(SE\) and the margin of error, construct and report the confidence interval, and interpret the interval in context.
	
	\textbf{Step 1: Sample proportion.}
	\[
	\hat{p} = \frac{x}{n} = \frac{39}{150} = 0.26
	\]
	
	\textbf{Step 2: Large-sample conditions.}
	\[
	n\hat{p} = 150(0.26) = 39 \ge 10, \qquad n(1-\hat{p}) = 150(0.74) = 111 \ge 10
	\]
	
	\textbf{Step 3: Standard error.}
	\[
	SE = \sqrt{\frac{\hat{p}(1-\hat{p})}{n}} = \sqrt{\frac{0.26(0.74)}{150}} \approx 0.0358
	\]
	
	\textbf{Step 4: Margin of error (95\% confidence, \(z^* = 1.96\)).}
	\[
	E = z^* \cdot SE = 1.96(0.0358) \approx 0.0702
	\]
	
	\textbf{Step 5: Confidence interval.}
	\[
	\hat{p} \pm E = 0.26 \pm 0.0702 \Rightarrow [0.190,\ 0.330]
	\]
	
	\textbf{Interpretation.}
	We are 95\% confident that between about 19.0\% and 33.0\% of all export shipments are delayed at least one day this season.
	
	\section*{3. Microloan Repayment Defaults}
	
	A microfinance program tracks \textbf{400 loans} issued to small businesses. Within the first year, \textbf{50 loans} defaulted.
	
	Construct and interpret a \textbf{98\% confidence interval} for the population proportion of loans that default within the first year.
	
	Based on your interval, does the data support the claim that the first-year default rate is at least 20\%? Explain.
	
	\subsubsection*{(C7) Constructs the confidence interval for a proportion}
	To construct the confidence interval for a proportion, we identify \(x\) and \(n\) and compute \(\hat{p}\), check normal approximation conditions \(n\hat{p} \ge 10\) and \(n(1-\hat{p}) \ge 10\), choose the correct \(z^*\) for the confidence level, compute \(SE\) and the margin of error, construct and report the confidence interval, and interpret the interval in context.
	
	\textbf{Step 1:} Sample proportion.
	\[
	\hat{p} = \frac{x}{n} = \frac{50}{400} = 0.125
	\]
	
	\textbf{Step 2:} Large-sample conditions.
	\[
	n\hat{p} = 400(0.125) = 50 \ge 10, \qquad n(1-\hat{p}) = 400(0.875) = 350 \ge 10
	\]
	
	\textbf{Step 3:} Standard error.
	\[
	SE = \sqrt{\frac{\hat{p}(1-\hat{p})}{n}} = \sqrt{\frac{0.125(0.875)}{400}} \approx 0.0165
	\]
	
	\textbf{Step 4:} Margin of error (98\% confidence, \(z^* = 2.326\)).
	\[
	E = z^* \cdot SE = 2.326(0.0165) \approx 0.0385
	\]
	
	\textbf{Step 5:} Confidence interval.
	\[
	\hat{p} \pm E = 0.125 \pm 0.0385 \Rightarrow [0.087,\ 0.164]
	\]
	
	\textbf{Step 6:} Interpretation in context.
	We are 98\% confident that between about 8.7\% and 16.4\% of loans in the full portfolio default within the first year.
	
	\textbf{Step 7:}
	The interval \([0.087, 0.164]\) is entirely below \(0.20\), so the data do \textbf{not} support the claim that the first-year default rate is at least 20\%.
	
	\section*{4. Inflation Expectations Survey}
	
	A national statistics office surveys \textbf{120 households} about whether they expect inflation to rise over the next year. \textbf{78 households} report expecting higher inflation.
	
	Construct and interpret a \textbf{99\% confidence interval} for the population proportion of households expecting higher inflation.
	
	If the confidence level were decreased from 99\% to 95\%, what would happen to the margin of error and why?
	
	\subsubsection*{(C7) Constructs the confidence interval for a proportion}
	To construct the confidence interval for a proportion, we identify \(x\) and \(n\) and compute \(\hat{p}\), check normal approximation conditions \(n\hat{p} \ge 10\) and \(n(1-\hat{p}) \ge 10\), choose the correct \(z^*\) for the confidence level, compute \(SE\) and the margin of error, construct and report the confidence interval, and interpret the interval in context.
	
	\textbf{Step 1:} Sample proportion.
	\[
	\hat{p} = \frac{x}{n} = \frac{78}{120} = 0.65
	\]
	
	\textbf{Step 2:} Large-sample conditions.
	\[
	n\hat{p} = 120(0.65) = 78 \ge 10, \qquad n(1-\hat{p}) = 120(0.35) = 42 \ge 10
	\]
	
	\textbf{Step 3:} Standard error.
	\[
	SE = \sqrt{\frac{\hat{p}(1-\hat{p})}{n}} = \sqrt{\frac{0.65(0.35)}{120}} \approx 0.0435
	\]
	
	\textbf{Step 4:} Margin of error (99\% confidence, \(z^* = 2.576\)).
	\[
	E = z^* \cdot SE = 2.576(0.0435) \approx 0.1121
	\]
	
	\textbf{Step 5:} Confidence interval.
	\[
	\hat{p} \pm E = 0.65 \pm 0.1121 \Rightarrow [0.538,\ 0.762]
	\]
	
	\textbf{Step 6:} Interpretation in context.
	We are 99\% confident that between about 53.8\% and 76.2\% of households expect inflation to rise over the next year.
	
	\textbf{Step 7:}
	Decreasing the confidence level from 99\% to 95\% lowers the critical value (from \(z^* = 2.576\) to \(z^* = 1.96\)), so \(E = z^* \cdot SE\) becomes smaller and the interval gets narrower.
	
	\section*{5. Public Transport Mobile Ticket Usage}
	
	A metropolitan transit authority wants to estimate the proportion of regular commuters who use a mobile ticketing app. It randomly surveys \textbf{260 commuters}, and \textbf{143 commuters} report using the app.
	
	Construct and interpret a \textbf{95\% confidence interval} for the population proportion of commuters who use mobile ticketing.
	
	\begin{itemize}
		\item How would the interval change if the authority reported a 99\% confidence level instead of 95\%, and why?
		\item Suppose the authority expected at least 65\% adoption. Does this interval support that expectation? Explain.
	\end{itemize}
	
	\subsubsection*{(C7) Constructs the confidence interval for a proportion}
	\textbf{Step 1:} Compute the sample proportion.
	\[
	\hat{p} = \frac{x}{n} = \frac{143}{260} = 0.55
	\]
	
	\textbf{Step 2:} Compute the standard error.
	\[
	SE = \sqrt{\frac{\hat{p}(1-\hat{p})}{n}} = \sqrt{\frac{0.55(0.45)}{260}} \approx 0.0308
	\]
	
	\textbf{Step 3:} Compute the margin of error (95\% confidence, \(z^* = 1.96\)).
	\[
	E = z^* \cdot SE = 1.96(0.0308) \approx 0.0604
	\]
	
	\textbf{Step 4:} Construct the confidence interval.
	\[
	\hat{p} \pm E = 0.55 \pm 0.0604 \Rightarrow [0.490,\ 0.610]
	\]
	
	\textbf{Step 5:} Interpretation in context.
	We are 95\% confident that between about 49.0\% and 61.0\% of all regular commuters use the mobile ticketing app.
	
	\textbf{Step 6:}
	With 99\% confidence, the critical value increases, so \(E = z^* \cdot SE\) becomes larger and the interval would be wider than \([0.490, 0.610]\).
	
	\textbf{Step 7:}
	The interval's upper endpoint is \(0.610\), which is below \(0.650\), so this interval does \textbf{not} support the expectation of at least 65\% adoption.
	
	\subsubsection*{Analysis questions}
	\begin{enumerate}
		\item Interpret the confidence interval in policy context: what does the interval suggest about citywide mobile ticket adoption?
		\item If planners need a margin of error no larger than 0.04, discuss whether this survey is precise enough and what could be changed.
	\end{enumerate}
	
	\section*{6. Small Business Cyber Training Completion}
	
	An industry chamber wants to estimate the proportion of registered small firms that completed a cybersecurity training module this year. It samples \textbf{180 firms}, and \textbf{66 firms} completed the module.
	
	Construct and interpret a \textbf{90\% confidence interval} for the population proportion of firms that completed the training.
	
	\begin{itemize}
		\item How would using a 95\% confidence level affect the interval width and the policy interpretation?
		\item The chamber claims that at least half of firms completed the module. Does this interval support that claim? Why or why not?
	\end{itemize}
	
	\subsubsection*{(C7) Constructs the confidence interval for a proportion}
	\textbf{Step 1:} Compute the sample proportion.
	\[
	\hat{p} = \frac{x}{n} = \frac{66}{180} \approx 0.3667
	\]
	
	\textbf{Step 2:} Compute the standard error.
	\[
	SE = \sqrt{\frac{\hat{p}(1-\hat{p})}{n}} = \sqrt{\frac{0.3667(0.6333)}{180}} \approx 0.0359
	\]
	
	\textbf{Step 3:} Compute the margin of error (90\% confidence, \(z^* = 1.645\)).
	\[
	E = z^* \cdot SE = 1.645(0.0359) \approx 0.0591
	\]
	
	\textbf{Step 4:} Construct the confidence interval.
	\[
	\hat{p} \pm E = 0.3667 \pm 0.0591 \Rightarrow [0.308,\ 0.426]
	\]
	
	\textbf{Step 5:} Interpretation in context.
	We are 90\% confident that between about 30.8\% and 42.6\% of all registered small firms completed the cybersecurity module.
	
	\textbf{Step 6:}
	At 95\% confidence, the critical value is larger than 1.645, so \(E = z^* \cdot SE\) increases and the interval would be wider, giving a more cautious but less precise estimate.
	
	\textbf{Step 7:}
	Because the entire interval \([0.308, 0.426]\) is below \(0.50\), the interval does \textbf{not} support the claim that at least half of firms completed the module.
	
	\subsubsection*{Analysis questions}
	\begin{enumerate}
		\item Explain in context what it means to be 90\% confident in this interval estimate.
		\item Discuss one way to reduce the margin of error without lowering the confidence level.
	\end{enumerate}
	
	\section*{7. Agricultural Insurance Enrollment}
	
	A rural development office studies enrollment in a crop-insurance program. In a random sample of \textbf{420 farmers}, \textbf{170 farmers} are currently enrolled. The office has stated that enrollment is \textbf{no more than 35\%}.
	
	Construct and interpret a \textbf{98\% confidence interval} for the population proportion of farmers enrolled, and use the interval to evaluate the office's statement.
	
	\begin{itemize}
		\item Why does a 98\% confidence level produce a wider interval than a 90\% interval from the same sample?
		\item Does the confidence interval give clear evidence that enrollment exceeds 35\%? Justify your conclusion using interval logic.
	\end{itemize}
	
	\subsubsection*{(C7) Constructs the confidence interval for a proportion}
	\textbf{Step 1:} Compute the sample proportion.
	\[
	\hat{p} = \frac{x}{n} = \frac{170}{420} \approx 0.4048
	\]
	
	\textbf{Step 2:} Compute the standard error.
	\[
	SE = \sqrt{\frac{\hat{p}(1-\hat{p})}{n}} = \sqrt{\frac{0.4048(0.5952)}{420}} \approx 0.0240
	\]
	
	\textbf{Step 3:} Compute the margin of error (98\% confidence, \(z^* = 2.326\)).
	\[
	E = z^* \cdot SE = 2.326(0.0240) \approx 0.0558
	\]
	
	\textbf{Step 4:} Construct the confidence interval.
	\[
	\hat{p} \pm E = 0.4048 \pm 0.0558 \Rightarrow [0.349,\ 0.461]
	\]
	
	\textbf{Step 5:} Interpretation in context.
	We are 98\% confident that between about 34.9\% and 46.1\% of all farmers in the region are enrolled in crop insurance.
	
	\textbf{Step 6:}
	A 98\% interval is wider than a 90\% interval from the same sample because higher confidence uses a larger critical value, which increases \(E = z^* \cdot SE\).
	
	\textbf{Step 7:}
	This interval does not give clear evidence that enrollment exceeds 35\%, because 0.35 lies inside \([0.349, 0.461]\); values at or below 35\% are still plausible.
	
	\subsubsection*{Analysis questions}
	\begin{enumerate}
		\item Interpret what the interval says about likely enrollment rates across the full farmer population.
		\item If policymakers need stronger evidence against the 35\% claim, what sample-design changes could improve precision?
	\end{enumerate}
	
	\section*{8. Port Safety Incident Reporting Compliance}
	
	A maritime regulator audits \textbf{500 port operations records} and finds \textbf{92 records} with full on-time incident reporting. Last year, a separate audit found \textbf{80 compliant records out of 480}.
	
	Construct and interpret a \textbf{99\% confidence interval} for this year's population proportion of on-time reporting compliance, and compare the interval with last year's observed proportion.
	
	\begin{itemize}
		\item How does selecting 99\% confidence affect certainty and margin of error compared with 95\% confidence?
		\item Use the interval to compare this year with last year's observed compliance proportion and state a justified conclusion.
	\end{itemize}
	
	\subsubsection*{(C7) Constructs the confidence interval for a proportion}
	\textbf{Step 1:} Compute the sample proportion.
	\[
	\hat{p} = \frac{x}{n} = \frac{92}{500} = 0.184
	\]
	
	\textbf{Step 2:} Compute the standard error.
	\[
	SE = \sqrt{\frac{\hat{p}(1-\hat{p})}{n}} = \sqrt{\frac{0.184(0.816)}{500}} \approx 0.0173
	\]
	
	\textbf{Step 3:} Compute the margin of error (99\% confidence, \(z^* = 2.576\)).
	\[
	E = z^* \cdot SE = 2.576(0.0173) \approx 0.0446
	\]
	
	\textbf{Step 4:} Construct the confidence interval.
	\[
	\hat{p} \pm E = 0.184 \pm 0.0446 \Rightarrow [0.139,\ 0.229]
	\]
	
	\textbf{Step 5:} Interpretation in context.
	We are 99\% confident that between about 13.9\% and 22.9\% of all port operations records are fully compliant with on-time incident reporting this year.
	
	\textbf{Step 6:}
	Choosing 99\% confidence increases certainty but uses a larger critical value than 95\%, so the margin of error is larger and the interval is wider.
	
	\textbf{Step 7:}
	Last year's observed rate was \(\hat{p}_{\text{last}} = \frac{80}{480} = 0.167\), and 0.167 lies inside \([0.139, 0.229]\), so this interval does not provide clear evidence of a change in compliance.
	
	\subsubsection*{Analysis questions}
	\begin{enumerate}
		\item Interpret this 99\% confidence interval in terms of likely compliance across all operations.
		\item If the regulator wants a narrower interval next year while keeping 99\% confidence, what should it change?
	\end{enumerate}
	
\end{document}
