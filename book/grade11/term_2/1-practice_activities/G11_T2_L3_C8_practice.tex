\makeatletter
\def\input@path{{./}{../}{../../}{preamble/}{../preamble/}{../../preamble/}}
\makeatother
% ----------------------------------------------------------
% GENERAL 

% File
\documentclass[11pt]{book}

% Margins
\usepackage[margin=1in]{geometry}

\linespread{1.2}            % Line spacing
\usepackage[utf8]{inputenc}
\usepackage[T1]{fontenc}
\usepackage{lmodern}
\usepackage{microtype}
\setlength{\parindent}{0pt}
\setlength{\parskip}{6pt}
\usepackage{booktabs}

% ----------------------------------------------------------
% TABLES
\usepackage{multicol}
\usepackage{longtable} 
\usepackage{array}
\usepackage{booktabs}
\usepackage{tabularx}
\usepackage{multirow}

% ----------------------------------------------------------
% MATHEMATICS
\usepackage{amsmath}
\usepackage{amssymb}
\usepackage{amsfonts}
\usepackage{mathtools}

% ----------------------------------------------------------
% HIDDEN CONTENT
\usepackage{ifthen}
% Define a boolean switch
\newboolean{explicaciones}
% Set the boolean switch to true or false
% Change to true to show the content

% Explanations
\newcommand{\explicacion}[2]{
	\ifthenelse{\boolean{explicaciones}}{#1}{#2}
}
\newcommand{\mostrarExplicaciones}[1]{\setboolean{explicaciones}{#1}}

% ----------------------------------------------------------
% NUMBERING

\usepackage{fancyhdr}
\pagestyle{empty} % Ensures the entire document has no page numbers

\usepackage{tocloft}
\renewcommand{\cftdot}{} % Remove dots for sections, if any
\renewcommand{\cftsecleader}{\cftdotfill{\cftdotsep}} % Remove dots for sections, if any
\cftpagenumbersoff{section} % Removes page numbers from sections
\cftpagenumbersoff{subsection} % Removes page numbers from subsections

% ----------------------------------------------------------
% IMAGES 

% General settings
\usepackage{graphicx}       % Insert images
\usepackage{float}          % Position images
% \usepackage{subfigure}      % Subfigures
\graphicspath{{imgs}}       % Image location
\usepackage{subcaption}     % Subfigures II
\usepackage{verbatim}

% Figures
\usepackage{tikz}
\usetikzlibrary{arrows.meta,positioning,trees}

% Colors
\usepackage{xcolor}     
\definecolor{popUp}{HTML}{666666}
\definecolor{popUpIn}{HTML}{CED9E0}
\definecolor{backgroundC}{HTML}{D0E8F2}
\definecolor{backgroundCC}{HTML}{FFFFFF}
\definecolor{borders}{HTML}{8c120d}
\definecolor{padding}{HTML}{77D0D7}
\definecolor{links}{HTML}{CC6F5F}

% ----------------------------------------------------------
% FRAMES

% General settings
\usepackage{tcolorbox}
\usepackage{adjustbox}          % Adjusted frame  
\setlength{\fboxrule}{3pt}  % Line width
\setlength{\fboxsep}{3pt}   % Box padding

% General frames
\usepackage{mdframed}   

\mdfdefinestyle{estiloGeneral}{    % General style
	linecolor=black,
	linewidth=1.5pt,
	roundcorner=10pt,
	backgroundcolor=backgroundC,
	innerbottommargin=0pt
}
\mdfdefinestyle{code}{          % Code style
	linecolor=black,
	linewidth=1.5pt,
	roundcorner=10pt,
	backgroundcolor=darkgray!10,
	innerbottommargin=0pt
}

% Image frame
\newtcbox{\fboxC}{
	colback=backgroundC,
	colframe=popUp,
	arc=10pt,
	boxrule=3pt,
	boxsep=0pt, % Change the padding here
	nobeforeafter
}

% ----------------------------------------------------------
% PAGE SETTINGS

% Background 
\newcommand{\background}[0]{\begin{tikzpicture}[remember picture,overlay]
		\fill[backgroundC] (-2,2) rectangle (25cm, -550);
\end{tikzpicture}}

\newcommand{\backgroundC}[0]{\begin{tikzpicture}[remember picture,overlay]
		\fill[backgroundCC] (-2,2) rectangle (25cm, -550);
\end{tikzpicture}}

% Page width 
\newcommand{\anchoPag}[0]{20cm}

% ----------------------------------------------------------
% FONT

% General
\usepackage{tgbonum}        % Font
\usepackage{listings}       % Code typesetting
\usepackage[spanish]{babel} % Load Spanish
\selectlanguage{spanish}    % Select Spanish
\usepackage{enumitem}
\usepackage{bookmark}

\setlist[itemize]{leftmargin=1.2em, itemsep=0.35em, topsep=0.35em}

% --- Table helpers ---
\newcolumntype{L}[1]{>{\raggedright\arraybackslash}p{#1}}
\newcolumntype{Y}{>{\raggedright\arraybackslash}X}
\newcolumntype{C}{>{\centering\arraybackslash}X}
\renewcommand{\arraystretch}{1.1}

% Python style
\lstdefinestyle{python}{
	language=Python,
	basicstyle=\ttfamily\small,
	commentstyle=\color{green!50!black},
	keywordstyle=\color{blue},
	numberstyle=\tiny\color{gray},
	numbers=left,
	morekeywords={>, <},
	breakatwhitespace=false,
	showstringspaces=false,
	showtabs=false,
	showspaces=false
}

% ----------------------------------------------------------
% HYPERLINKS

% General
\usepackage{hyperref}       
\hypersetup{
	colorlinks=true,
	linkcolor=links,
	filecolor=magenta,      
	urlcolor=brown,
}

% Custom commands 

% Large link
\newcommand{\bigLink}[2]{\begin{center} \fboxC{\LARGE{\href{#1}{#2}}}\end{center}}

% Small link
\newcommand{\smallLink}[2]{\begin{center}\fboxC{\href{#1}{#2}}\end{center}}

% Bold link
\newcommand{\bfLink}[2]{\href{#1}{\textbf{#2}}}


% Small URL
\newcommand{\smallUrl}[1]{\begin{center}\fboxC{\url{#1}}\end{center}}


% ----------------------------------------------------------
% CUSTOM COMMANDS FOR FIGURES

\newcommand{\espacioImagenes}[0]{-1.2cm}

% Without frame
\newcommand{\fig}[3][\espacioImagenes]{
	\hspace*{#1}
	\centering
	\includegraphics[width=#2\textwidth]{#3}
}

% With frame
\newcommand{\ffig}[2]{\begin{figure}[h]
		\hspace*{\espacioImagenes}
		\centering
		\fbox{\includegraphics[width=#1\textwidth]{#2}}
\end{figure}}

% Hyperlink with frame
\newcommand{\hfig}[3]{\begin{figure}[h]
		\hspace*{-1.4cm}
		\centering
		\color{popUp}
		\fboxC{\href{#1}{\includegraphics[width=#2\textwidth]{#3}}}
	\end{figure}
}

% Hyperlink without frame
\newcommand{\hffig}[3]{\begin{figure}[h]
		\hspace*{-1.1cm}
		\centering
		\color{popUp}
		\href{#1}{\includegraphics[width=#2\textwidth]{#3}}
	\end{figure}
}

% ----------------------------------------------------------

% Start and Contents
\newcommand{\cuadro}[1]{
	\begin{mdframed}[style=estiloGeneral]
		#1 
	\end{mdframed}
}

% Explanation video image
\newcommand{\linkExplicacion}[1]{
	\hffig{#1}{0.5}{principal/videoExplicacion}
	\vspace{-0.5cm}
}

\newcommand{\subSecLink}[2]{
	\subsubsection*{\href{#1}{\textbf{#2}}}
}

% Spacing
\newcommand{\esp}[0]{\vspace{4mm}}

% Back to start
\newcommand{\secInicio}[0]{\begin{center}\hyperref[sec:inicio]{ 
			\includegraphics[width=0.1\textwidth]{principal/up}
	}\end{center}
}


\geometry{margin=0.85in}
\AtBeginDocument{\small}

\newcommand{\ExamNameField}{\noindent\textbf{Name:}\ \rule{0.7\linewidth}{0.4pt}\par\medskip}

\newcommand{\ExamTitleBlock}[3]{%
	\begin{center}
		\Large\textbf{#1}\\
		\textbf{#2}%
		\if\relax\detokenize{#3}\relax\else\\\textbf{#3}\fi
	\end{center}
	\vspace{0.5em}
}

\newcommand{\ExamSection}[1]{\par\medskip\textbf{#1}\par\smallskip}

\newenvironment{ExamCriteria}{%
	\begin{itemize}[leftmargin=1.6em, itemsep=0.3em, topsep=0.2em]
}{%
	\end{itemize}
}

\newenvironment{ExamProblems}{%
	\begin{enumerate}[label=\textbf{P\arabic*}, leftmargin=0pt, labelsep=0.6em, itemindent=2.2em, itemsep=0.8em]
}{%
	\end{enumerate}
}

\begin{document}
	\ExamSection{C8: t-student confidence intervals in context situations}
	\begin{ExamProblems}
		\item
		\subsection*{Problem description}
		A logistics center wants to estimate the mean loading time (minutes) for trucks in a new dispatch lane.
		A random sample of $n=12$ trucks gives $\bar{x}=84.60$ minutes and sample standard deviation $s=6.80$ minutes.
		The population variance is unknown, and the sample size is small, so a t-interval is required.
		Construct and interpret a 95\% confidence interval for the population mean loading time.
		
		\subsection*{C8}
		Step 1.
		Identify the parameter and sample statistics.
		\[
		\mu=\text{population mean loading time},\quad n=12,\quad \bar{x}=84.60,\quad s=6.80
		\]
		Step 2.
		State degrees of freedom.
		\[
		df=n-1=12-1=11
		\]
		Step 3.
		Identify the t critical value for 95\% confidence.
		\[
		t^*=t_{0.025,11}\approx 2.201
		\]
		Step 4.
		Compute the standard error.
		\[
		SE=\frac{s}{\sqrt{n}}=\frac{6.80}{\sqrt{12}}\approx 1.96
		\]
		Step 5.
		Compute the margin of error.
		\[
		E=t^*\cdot SE=2.201(1.96)\approx 4.31
		\]
		Step 6.
		Construct the confidence interval.
		\[
		\mu\in\bar{x}\pm E\Rightarrow 84.60\pm 4.31=[80.29,\,88.91]
		\]
		Step 7.
		Interpretation.
		With 95\% confidence, the mean loading time in the new lane is between 80.29 and 88.91 minutes.
		
		\newpage
		\item
		\subsection*{Problem description}
		An agricultural exporter tracks the average customs processing time (hours) for a new shipping route.
		A random sample of $n=18$ shipments gives $\bar{x}=42.30$ hours and $s=5.10$ hours.
		Because the population variance is unknown and the sample is moderate, use a t-interval.
		Construct and interpret a 90\% confidence interval for the population mean processing time.
		
		\subsection*{C8}
		Step 1.
		Identify the parameter and sample statistics.
		\[
		\mu=\text{population mean processing time},\quad n=18,\quad \bar{x}=42.30,\quad s=5.10
		\]
		Step 2.
		State degrees of freedom.
		\[
		df=18-1=17
		\]
		Step 3.
		Identify the t critical value for 90\% confidence.
		\[
		t^*=t_{0.05,17}\approx 1.740
		\]
		Step 4.
		Compute the standard error.
		\[
		SE=\frac{5.10}{\sqrt{18}}\approx 1.20
		\]
		Step 5.
		Compute the margin of error.
		\[
		E=1.740(1.20)\approx 2.09
		\]
		Step 6.
		Construct the confidence interval.
		\[
		\mu\in 42.30\pm 2.09=[40.21,\,44.39]
		\]
		Step 7.
		Interpretation.
		With 90\% confidence, the true mean customs processing time for this route lies between 40.21 and 44.39 hours.
		
		\newpage
		\item
		\subsection*{Problem description}
		A school cafeteria estimates the mean daily demand (kg) for a new meal plan ingredient.
		A random sample of $n=25$ school days gives $\bar{x}=73.50$ kg and $s=8.40$ kg.
		The population variance is unknown.
		Construct and interpret a 99\% confidence interval for the population mean daily demand.
		
		\subsection*{C8}
		Step 1.
		Identify the parameter and sample statistics.
		\[
		\mu=\text{population mean daily demand},\quad n=25,\quad \bar{x}=73.50,\quad s=8.40
		\]
		Step 2.
		State degrees of freedom.
		\[
		df=25-1=24
		\]
		Step 3.
		Identify the t critical value for 99\% confidence.
		\[
		t^*=t_{0.005,24}\approx 2.797
		\]
		Step 4.
		Compute the standard error.
		\[
		SE=\frac{8.40}{\sqrt{25}}=1.68
		\]
		Step 5.
		Compute the margin of error.
		\[
		E=2.797(1.68)\approx 4.70
		\]
		Step 6.
		Construct the confidence interval.
		\[
		\mu\in 73.50\pm 4.70=[68.80,\,78.20]
		\]
		Step 7.
		Interpretation.
		With 99\% confidence, the population mean daily demand is between 68.80 kg and 78.20 kg.
	\end{ExamProblems}
\end{document}
