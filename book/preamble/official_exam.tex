\documentclass[12pt]{article}

% Page size and tighter margins
\usepackage[a4paper,left=1.2cm,right=1.2cm,top=1.5cm,bottom=1.5cm]{geometry}

% Core packages
\usepackage{graphicx}
\usepackage{xcolor}
\usepackage{array}
\usepackage{tabularx}
\usepackage{multicol}
\usepackage[T1]{fontenc}
\usepackage[utf8]{inputenc}

\setlength{\parindent}{0pt}
\setlength{\tabcolsep}{6pt}
\renewcommand{\arraystretch}{1.15}

% Column types
\newcolumntype{Y}{>{\raggedright\arraybackslash}m{\dimexpr0.30\textwidth-2\tabcolsep-2\arrayrulewidth\relax}}
\newcolumntype{Z}{>{\raggedright\arraybackslash}m{\dimexpr0.70\textwidth-2\tabcolsep-2\arrayrulewidth\relax}}
\newcolumntype{C}[1]{>{\centering\arraybackslash}m{#1}}

% Gray subsection header box
\newcommand{\SubsectionBox}[1]{%
	\noindent\colorbox{gray!30}{%
		\parbox{\linewidth}{\textbf{#1}}%
	}\par\vspace{0.35cm}%
}

% Centered multi-line cell helper
\newcommand{\CellCenter}[1]{%
	\parbox{\linewidth}{\centering #1}%
}

\begin{document}
	
	% =========================
	% HEADER BOX (3 COLUMNS)
	% =========================
	\noindent
	\begin{tabularx}{\textwidth}{|C{2.8cm}|C{\dimexpr\textwidth-6cm-4\tabcolsep-4\arrayrulewidth\relax}|C{2.8cm}|}
		\hline
		\vspace{3mm}
		\centering \includegraphics[width=2.5cm]{logo.png}
		&
		\CellCenter{%
			\vspace{-5mm}
			\textbf{REASONING THINKING}\par
			\textbf{GRADE: 10TH}\par
			\textbf{PROGRESSION TEST 1ST TERM}\par
			\textbf{TEACHER'S NAME: Eliana Rodríguez -- David Amaya}%
		}
		&
		\CellCenter{%
			\textbf{FIRST TERM}\par
			\textbf{2025--2026}%
		}
		\\
		\hline
	\end{tabularx}
	
	\vspace{0.5cm}
	
	% =========================
	% OBJECTIVE + CRITERIA
	% =========================
	\noindent
	\begin{tabular}{|Y|Z|}
		\hline
		\textbf{Learning objective:} Model 3D structures analyzing their 2D triangular components through trigonometric ratios, sine and cosine rules, and vector theory.
		&
		{\small
			\textbf{Assessment criteria:}\par
			C1: Expresses degrees into radians and vice-versa.\par
			C2: Interprets three-figured bearings.\par
			C3: Uses basic operations and translations with vectors.\par
			C4: Calculates reflections, rotations, translations, and enlargements of functions.\par
			C5: Identifies trigonometric ratios in right triangles and in the unit circle.\par
			C6: Differentiates the conditions for using sine and cosine rules.\par
			C7: Establishes the conditions for using vectors, Pythagorean theorem or trigonometric ratios, and vectors in the solution of context situations.
		}
		\\
		\hline
	\end{tabular}
	
	\vspace{0.6cm}
	
	% =========================
	% STUDENT LINE
	% =========================
	\noindent
	\textbf{Student’s name:} \rule{7cm}{0.4pt}\hfill
	\textbf{Group:} \rule{2cm}{0.4pt}\hfill
	\textbf{Date:} \rule{3cm}{0.4pt}
	
	\vspace{0.8cm}
	
	% =========================
	% EXAM BODY
	% =========================
	\begin{multicols}{2}
		
		\SubsectionBox{C3: Vectors}
		
		\textbf{1. Movement of a minute hand.}
		A 10th grade student is studying how angles are measured in radians using an analog clock as a visual reference. She/He notices that when the minute hand moves from the 12 to the 3, it covers a quarter of the clock face. How many radians does the minute hand move when it goes from 12 to 3?
		
		\vspace{0.6cm}
		
		\textbf{3. Surveillance drone in farmland.}
		A surveillance drone takes off from base station E, located at the edge of a large farmland. On its first flight path, it travels 15 km North and 8 km East, represented by the vector
		\[
		\vec{v}_1=\begin{pmatrix}8\\15\end{pmatrix}.
		\]
		Then, the drone adjusts its course and flies 5 km West and 10 km North, represented by
		\[
		\vec{v}_2=\begin{pmatrix}-5\\10\end{pmatrix}.
		\]
		
	\end{multicols}
	
\end{document}
