% ----------------------------------------------------------
% GENERAL 

% File
\documentclass[11pt]{book}

% Margins
\usepackage[margin=1in]{geometry}

\linespread{1.2}            % Line spacing
\usepackage[utf8]{inputenc}
\usepackage[T1]{fontenc}
\usepackage{lmodern}
\usepackage{microtype}
\setlength{\parindent}{0pt}
\setlength{\parskip}{6pt}
\usepackage{booktabs}

% ----------------------------------------------------------
% TABLES
\usepackage{multicol}
\usepackage{longtable} 
\usepackage{array}
\usepackage{booktabs}
\usepackage{tabularx}
\usepackage{multirow}

% ----------------------------------------------------------
% MATHEMATICS
\usepackage{amsmath}
\usepackage{amssymb}
\usepackage{amsfonts}
\usepackage{mathtools}

% ----------------------------------------------------------
% HIDDEN CONTENT
\usepackage{ifthen}
% Define a boolean switch
\newboolean{explicaciones}
% Set the boolean switch to true or false
% Change to true to show the content

% Explanations
\newcommand{\explicacion}[2]{
	\ifthenelse{\boolean{explicaciones}}{#1}{#2}
}
\newcommand{\mostrarExplicaciones}[1]{\setboolean{explicaciones}{#1}}

% ----------------------------------------------------------
% NUMBERING

\usepackage{fancyhdr}
\pagestyle{empty} % Ensures the entire document has no page numbers

\usepackage{tocloft}
\renewcommand{\cftdot}{} % Remove dots for sections, if any
\renewcommand{\cftsecleader}{\cftdotfill{\cftdotsep}} % Remove dots for sections, if any
\cftpagenumbersoff{section} % Removes page numbers from sections
\cftpagenumbersoff{subsection} % Removes page numbers from subsections

% ----------------------------------------------------------
% IMAGES 

% General settings
\usepackage{graphicx}       % Insert images
\usepackage{float}          % Position images
% \usepackage{subfigure}      % Subfigures
\graphicspath{{imgs}}       % Image location
\usepackage{subcaption}     % Subfigures II
\usepackage{verbatim}

% Figures
\usepackage{tikz}
\usetikzlibrary{arrows.meta,positioning,trees}

% Colors
\usepackage{xcolor}     
\definecolor{popUp}{HTML}{666666}
\definecolor{popUpIn}{HTML}{CED9E0}
\definecolor{backgroundC}{HTML}{D0E8F2}
\definecolor{backgroundCC}{HTML}{FFFFFF}
\definecolor{borders}{HTML}{8c120d}
\definecolor{padding}{HTML}{77D0D7}
\definecolor{links}{HTML}{CC6F5F}

% ----------------------------------------------------------
% FRAMES

% General settings
\usepackage{tcolorbox}
\usepackage{adjustbox}          % Adjusted frame  
\setlength{\fboxrule}{3pt}  % Line width
\setlength{\fboxsep}{3pt}   % Box padding

% General frames
\usepackage{mdframed}   

\mdfdefinestyle{estiloGeneral}{    % General style
	linecolor=black,
	linewidth=1.5pt,
	roundcorner=10pt,
	backgroundcolor=backgroundC,
	innerbottommargin=0pt
}
\mdfdefinestyle{code}{          % Code style
	linecolor=black,
	linewidth=1.5pt,
	roundcorner=10pt,
	backgroundcolor=darkgray!10,
	innerbottommargin=0pt
}

% Image frame
\newtcbox{\fboxC}{
	colback=backgroundC,
	colframe=popUp,
	arc=10pt,
	boxrule=3pt,
	boxsep=0pt, % Change the padding here
	nobeforeafter
}

% ----------------------------------------------------------
% PAGE SETTINGS

% Background 
\newcommand{\background}[0]{\begin{tikzpicture}[remember picture,overlay]
		\fill[backgroundC] (-2,2) rectangle (25cm, -550);
\end{tikzpicture}}

\newcommand{\backgroundC}[0]{\begin{tikzpicture}[remember picture,overlay]
		\fill[backgroundCC] (-2,2) rectangle (25cm, -550);
\end{tikzpicture}}

% Page width 
\newcommand{\anchoPag}[0]{20cm}

% ----------------------------------------------------------
% FONT

% General
\usepackage{tgbonum}        % Font
\usepackage{listings}       % Code typesetting
\usepackage[spanish]{babel} % Load Spanish
\selectlanguage{spanish}    % Select Spanish
\usepackage{enumitem}
\usepackage{bookmark}

\setlist[itemize]{leftmargin=1.2em, itemsep=0.35em, topsep=0.35em}

% --- Table helpers ---
\newcolumntype{L}[1]{>{\raggedright\arraybackslash}p{#1}}
\newcolumntype{Y}{>{\raggedright\arraybackslash}X}
\newcolumntype{C}{>{\centering\arraybackslash}X}
\renewcommand{\arraystretch}{1.1}

% Python style
\lstdefinestyle{python}{
	language=Python,
	basicstyle=\ttfamily\small,
	commentstyle=\color{green!50!black},
	keywordstyle=\color{blue},
	numberstyle=\tiny\color{gray},
	numbers=left,
	morekeywords={>, <},
	breakatwhitespace=false,
	showstringspaces=false,
	showtabs=false,
	showspaces=false
}

% ----------------------------------------------------------
% HYPERLINKS

% General
\usepackage{hyperref}       
\hypersetup{
	colorlinks=true,
	linkcolor=links,
	filecolor=magenta,      
	urlcolor=brown,
}

% Custom commands 

% Large link
\newcommand{\bigLink}[2]{\begin{center} \fboxC{\LARGE{\href{#1}{#2}}}\end{center}}

% Small link
\newcommand{\smallLink}[2]{\begin{center}\fboxC{\href{#1}{#2}}\end{center}}

% Bold link
\newcommand{\bfLink}[2]{\href{#1}{\textbf{#2}}}


% Small URL
\newcommand{\smallUrl}[1]{\begin{center}\fboxC{\url{#1}}\end{center}}


% ----------------------------------------------------------
% CUSTOM COMMANDS FOR FIGURES

\newcommand{\espacioImagenes}[0]{-1.2cm}

% Without frame
\newcommand{\fig}[3][\espacioImagenes]{
	\hspace*{#1}
	\centering
	\includegraphics[width=#2\textwidth]{#3}
}

% With frame
\newcommand{\ffig}[2]{\begin{figure}[h]
		\hspace*{\espacioImagenes}
		\centering
		\fbox{\includegraphics[width=#1\textwidth]{#2}}
\end{figure}}

% Hyperlink with frame
\newcommand{\hfig}[3]{\begin{figure}[h]
		\hspace*{-1.4cm}
		\centering
		\color{popUp}
		\fboxC{\href{#1}{\includegraphics[width=#2\textwidth]{#3}}}
	\end{figure}
}

% Hyperlink without frame
\newcommand{\hffig}[3]{\begin{figure}[h]
		\hspace*{-1.1cm}
		\centering
		\color{popUp}
		\href{#1}{\includegraphics[width=#2\textwidth]{#3}}
	\end{figure}
}

% ----------------------------------------------------------

% Start and Contents
\newcommand{\cuadro}[1]{
	\begin{mdframed}[style=estiloGeneral]
		#1 
	\end{mdframed}
}

% Explanation video image
\newcommand{\linkExplicacion}[1]{
	\hffig{#1}{0.5}{principal/videoExplicacion}
	\vspace{-0.5cm}
}

\newcommand{\subSecLink}[2]{
	\subsubsection*{\href{#1}{\textbf{#2}}}
}

% Spacing
\newcommand{\esp}[0]{\vspace{4mm}}

% Back to start
\newcommand{\secInicio}[0]{\begin{center}\hyperref[sec:inicio]{ 
			\includegraphics[width=0.1\textwidth]{principal/up}
	}\end{center}
}


% ===========================================
\title{\LARGE Preamble Guide}
\date{}
% ===========================================

\begin{document}
	
	\begin{minipage}{\anchoPag}
		\thispagestyle{empty}
		\background
		
		% =================
		\maketitle
		\vspace{-2cm}
		% =================
		
		\begin{mdframed}[style=estiloGeneral]
			This document explains the structure of the LaTeX preamble used as the base template. Each section below corresponds to a block in the preamble and includes what it does, why it is useful, and the commands that depend on it.
		\end{mdframed}
		
		% =============================================================
		\subsection{Description}
		
		This file documents the preamble that is used as a shared base for multiple LaTeX documents. Other files include it with \texttt{\textbackslash input\{preambulo\}}, so changing the preamble modifies every document that imports it. The preamble includes packages, page settings, colors, frames, hyperlinks, and custom commands used across guides.
		
		% ----------------------------------------------------------
		\subsection{General}\label{sec:general}
		
		\begin{enumerate}
			\item Document type: \texttt{standalone} is used with \texttt{varwidth} to produce compact pages and allow custom margins.
			\item Margins: \texttt{geometry} sets custom margins for the page layout.
			\item Line spacing: \texttt{\textbackslash linespread\{1.2\}} increases readability.
			\item Input encoding: \texttt{inputenc} with \texttt{utf8} supports accented characters.
		\end{enumerate}
		
		\begin{mdframed}[style=code]
			\begin{verbatim}
				\documentclass[varwidth, margin=2cm, 12pt]{standalone}
				\usepackage[left=2cm, right=-8cm, top=1cm, bottom=2.5cm]{geometry}
				\linespread{1.2}
				\usepackage[utf8]{inputenc}
			\end{verbatim}
		\end{mdframed}

		
		% ----------------------------------------------------------
		\subsection{Tables}\label{sec:tables}
		
		\begin{enumerate}
			\item \texttt{multicol}: allows text or content to be split into columns.
			\item \texttt{longtable}: supports tables that span multiple pages.
			\item \texttt{array}: provides extra control over column formatting.
			\item \texttt{booktabs}: improves table rules and spacing for cleaner design.
		\end{enumerate}
		
		\begin{mdframed}[style=code]
			\begin{verbatim}
				\usepackage{multicol}
				\usepackage{longtable}
				\usepackage{array}
				\usepackage{booktabs}
			\end{verbatim}
		\end{mdframed}
		
		% ----------------------------------------------------------
		\subsection{Mathematics}\label{sec:math}
		
		\begin{enumerate}
			\item \texttt{amsmath}: improves equation environments and alignment.
			\item \texttt{amssymb} and \texttt{amsfonts}: provide additional math symbols and fonts.
			\item \texttt{mathtools}: extends \texttt{amsmath} with extra tools and fixes.
		\end{enumerate}
		
		\begin{mdframed}[style=code]
			\begin{verbatim}
				\usepackage{amsmath}
				\usepackage{amssymb}
				\usepackage{amsfonts}
				\usepackage{mathtools}
			\end{verbatim}
		\end{mdframed}
		
		% ----------------------------------------------------------
		\subsection{Hidden Content Switch}\label{sec:hidden}
		
		This section defines a boolean switch to show or hide explanatory content without deleting it.
		
		\begin{enumerate}
			\item \texttt{\textbackslash newboolean\{explicaciones\}} creates the switch.
			\item \texttt{\textbackslash mostrarExplicaciones\{true/false\}} turns explanations on or off.
			\item \texttt{\textbackslash explicacion\{A\}\{B\}} prints \texttt{A} when the switch is true, and \texttt{B} when it is false.
		\end{enumerate}
		
		\begin{mdframed}[style=code]
			\begin{verbatim}
				\newboolean{explicaciones}
				\newcommand{\explicacion}[2]{\ifthenelse{\boolean{explicaciones}}{#1}{#2}}
				\newcommand{\mostrarExplicaciones}[1]{\setboolean{explicaciones}{#1}}
			\end{verbatim}
		\end{mdframed}
		
		\begin{mdframed}[style=estiloGeneral]
			Example usage:
			
			\mostrarExplicaciones{true}
			
			\explicacion{This text will appear.}{This text will be hidden.}
		\end{mdframed}
		
		% ----------------------------------------------------------
		\subsection{Numbering and Table of Contents}\label{sec:numbering}
		
		\begin{enumerate}
			\item \texttt{fancyhdr} is loaded, but page numbering is disabled with \texttt{\textbackslash pagestyle\{empty\}}.
			\item \texttt{tocloft} customizes the table of contents.
			\item Dots and page numbers are removed for sections and subsections.
		\end{enumerate}
		
		\begin{mdframed}[style=code]
			\begin{verbatim}
				\usepackage{fancyhdr}
				\pagestyle{empty}
				
				\usepackage{tocloft}
				\renewcommand{\cftdot}{}
				\renewcommand{\cftsecleader}{\cftdotfill{\cftdotsep}}
				\cftpagenumbersoff{section}
				\cftpagenumbersoff{subsection}
			\end{verbatim}
		\end{mdframed}
		
		% ----------------------------------------------------------
		\subsection{Images, Figures, and Colors}\label{sec:images}
		
		\subsection{Images and positioning}
		
		\begin{enumerate}
			\item \texttt{graphicx} enables \texttt{\textbackslash includegraphics}.
			\item \texttt{float} allows better control of figure placement.
			\item \texttt{subcaption} enables subfigures.
			\item \texttt{\textbackslash graphicspath\{\{imgs\}\}} sets the default image folder.
		\end{enumerate}
		
		\begin{mdframed}[style=code]
			\begin{verbatim}
				\usepackage{graphicx}
				\usepackage{float}
				\graphicspath{{imgs}}
				\usepackage{subcaption}
			\end{verbatim}
		\end{mdframed}
		
		\subsection{TikZ figures}
		
		\begin{enumerate}
			\item \texttt{tikz} supports vector graphics drawn in LaTeX.
			\item \texttt{arrows.meta} adds arrow styles and tools.
		\end{enumerate}
		
		\begin{mdframed}[style=code]
			\begin{verbatim}
				\usepackage{tikz}
				\usetikzlibrary{arrows.meta}
			\end{verbatim}
		\end{mdframed}
		
		\subsection{Color palette}
		
		This palette defines named colors used across frames, background, and hyperlink styling.
		
		\begin{mdframed}[style=code]
			\begin{verbatim}
				\usepackage{xcolor}
				\definecolor{popUp}{HTML}{666666}
				\definecolor{backgroundC}{HTML}{D0E8F2}
				\definecolor{backgroundCC}{HTML}{FFFFFF}
				\definecolor{links}{HTML}{CC6F5F}
			\end{verbatim}
		\end{mdframed}
		
		% ----------------------------------------------------------
		\subsection{Frames}\label{sec:frames}
		
		Frames are used to highlight content blocks and code blocks in a consistent style.
		
		\begin{enumerate}
			\item \texttt{tcolorbox} provides advanced box tools and is used for \texttt{\textbackslash fboxC}.
			\item \texttt{adjustbox} and the \texttt{\textbackslash fboxrule} / \texttt{\textbackslash fboxsep} lengths tune borders and padding.
			\item \texttt{mdframed} defines two main styles:
			\begin{itemize}
				\item \texttt{estiloGeneral}: main content frame with rounded corners.
				\item \texttt{code}: code frame with a light gray background.
			\end{itemize}
		\end{enumerate}
		
		\begin{mdframed}[style=code]
			\begin{verbatim}
				\mdfdefinestyle{estiloGeneral}{...}
				\mdfdefinestyle{code}{...}
				\newtcbox{\fboxC}{...}
			\end{verbatim}
		\end{mdframed}
		
		\begin{mdframed}[style=estiloGeneral]
			This is a sample frame using \texttt{estiloGeneral}.
		\end{mdframed}
		
		\begin{mdframed}[style=code]
			This is a sample frame using \texttt{code}.
		\end{mdframed}
		
		% ----------------------------------------------------------
		\subsection{Page Settings}\label{sec:page}
		
		\subsection{Background commands}
		
		\begin{enumerate}
			\item \texttt{\textbackslash background} paints the page background using TikZ and \texttt{backgroundC}.
			\item \texttt{\textbackslash backgroundC} paints an alternative background using \texttt{backgroundCC}.
		\end{enumerate}
		
		\begin{mdframed}[style=code]
			\begin{verbatim}
				\newcommand{\background}{...}
				\newcommand{\backgroundC}{...}
			\end{verbatim}
		\end{mdframed}
		
		\subsection{Page width macro}
		
		\begin{enumerate}
			\item \texttt{\textbackslash anchoPag} defines the minipage width used in documents.
		\end{enumerate}
		
		\begin{mdframed}[style=code]
			\begin{verbatim}
				\newcommand{\anchoPag}[0]{20cm}
			\end{verbatim}
		\end{mdframed}
		
		% ----------------------------------------------------------
		\subsection{Font and Code Listings}\label{sec:font}
		
		\begin{enumerate}
			\item \texttt{tgbonum} sets the main text font.
			\item \texttt{listings} is used for code formatting, including a custom Python style.
			\item \texttt{babel} is configured for Spanish in this preamble, affecting hyphenation and language rules.
		\end{enumerate}
		
		\begin{mdframed}[style=code]
			\begin{verbatim}
				\usepackage{tgbonum}
				\usepackage{listings}
				\usepackage[spanish]{babel}
				\selectlanguage{spanish}
				\lstdefinestyle{python}{...}
			\end{verbatim}
		\end{mdframed}
		
		% ----------------------------------------------------------
		\subsection{Hyperlinks and Custom Link Commands}\label{sec:links}
		
		\subsection{Hyperref configuration}
		
		\begin{enumerate}
			\item \texttt{hyperref} enables clickable links.
			\item \texttt{\textbackslash hypersetup} defines link colors using the color palette.
		\end{enumerate}
		
		\begin{mdframed}[style=code]
			\begin{verbatim}
				\usepackage{hyperref}
				\hypersetup{
					colorlinks=true,
					linkcolor=links,
					filecolor=magenta,
					urlcolor=brown,
				}
			\end{verbatim}
		\end{mdframed}
		
		\subsection{Custom link commands}
		
		\begin{enumerate}
			\item \texttt{\textbackslash bigLink\{url\}\{text\}} creates a centered framed large hyperlink.
			\item \texttt{\textbackslash smallLink\{url\}\{text\}} creates a centered framed normal hyperlink.
			\item \texttt{\textbackslash bfLink\{url\}\{text\}} creates a bold clickable text link.
			\item \texttt{\textbackslash smallUrl\{url\}} prints a framed URL centered on the page.
		\end{enumerate}
		
		\begin{mdframed}[style=estiloGeneral]
			Examples:
			
			\bigLink{https://www.latex-project.org/}{LaTeX Project}
			
			\smallLink{https://ctan.org/}{CTAN}
			
			\bfLink{https://www.overleaf.com/}{Overleaf}
			
			\smallUrl{https://www.monterrosaleshomeschool.edu.co/}
		\end{mdframed}
		
		% ----------------------------------------------------------
		\subsection{Custom Figure Commands}\label{sec:figcmd}
		
		These commands wrap \texttt{\textbackslash includegraphics} with consistent spacing and optional framing, and can optionally create clickable images.
		
		\begin{enumerate}
			\item \texttt{\textbackslash espacioImagenes} controls horizontal offset applied to figures.
			\item \texttt{\textbackslash fig} inserts an image without a frame.
			\item \texttt{\textbackslash ffig} inserts an image in a \texttt{figure} environment with a visible frame.
			\item \texttt{\textbackslash hfig} inserts a framed image that links to a URL.
			\item \texttt{\textbackslash hffig} inserts an unframed image that links to a URL.
		\end{enumerate}
		
		\begin{mdframed}[style=code]
			\begin{verbatim}
				\newcommand{\fig}[3][\espacioImagenes]{...}
				\newcommand{\ffig}[2]{...}
				\newcommand{\hfig}[3]{...}
				\newcommand{\hffig}[3]{...}
			\end{verbatim}
		\end{mdframed}
		
		% ----------------------------------------------------------
		\subsection{Other Utility Commands}\label{sec:other}
		
		\begin{enumerate}
			\item \texttt{\textbackslash cuadro\{...\}} wraps content in the general styled frame.
			\item \texttt{\textbackslash linkExplicacion\{url\}} places a clickable video explanation image.
			\item \texttt{\textbackslash subSecLink\{url\}\{text\}} creates an unnumbered subsubsection that is a hyperlink.
			\item \texttt{\textbackslash esp} adds vertical spacing.
			\item \texttt{\textbackslash secInicio} creates a clickable button that jumps back to the start section labeled \texttt{sec:inicio}.
		\end{enumerate}
		
		\begin{mdframed}[style=estiloGeneral]
			Example:
			
			\cuadro{This content is inside a \texttt{\textbackslash cuadro} frame.}
			
			\esp
			
			\subSecLink{https://ctan.org/pkg/mdframed}{mdframed package}
		\end{mdframed}
		
		% ----------------------------------------------------------
		\subsection{Navigation and Visual Examples}\label{sec:navigation}
		
		\begin{enumerate}
			\item The command \texttt{\textbackslash secInicio} depends on a label called \texttt{sec:inicio}.
			\item If your document includes a table of contents section labeled \texttt{sec:inicio}, the button will jump there.
		\end{enumerate}
		
		\begin{mdframed}[style=estiloGeneral]
			Below is a visual example of a clickable image link:
			
			\hfig{https://www.monterrosaleshomeschool.edu.co/}{0.4}{principal/logo}
		\end{mdframed}
		
		\secInicio
		
	\end{minipage}
\end{document}
